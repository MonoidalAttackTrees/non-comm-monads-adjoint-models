In a LAM, the SMCC $\cat{C}$ models the commutative linear logic and the Lambeck category
$\cat{L}$ models the non-commutative variant. In Section~\ref{subsec:elle}, we will present the
term assignment for sequent calculus of both sides and prove the cut elimination theorem. In
Section~\ref{subsec:elle-nd}, we present the term assignment for natural deduction of both sides
and prove the logic is strongly normalizing.

A sequent in the commutative side is of the form $\Psi  \NDsym{,}  \Phi  \vdash_\mathcal{C}  \NDnt{t}  \NDsym{:}  \NDnt{X}$. The types must be $X$,
$Y$, $Z$, etc., which are objects in the SMCC $\cat{C}$. Tye typing contexts are multisets.
Suppose $\Psi$ is the set $x_1:X_1, x_2:X_2, ..., x_m:X_m$ and $\Phi$ is the set
$y_1:Y_1, y_2:Y_2,...,y_n:Y_n$, then the categorical interpretation of the sequent is the
morphism $(X_1\otimes X_2\otimes...\otimes X_m)\otimes(Y_1\otimes Y_2\otimes...\otimes Y_n)\rightarrow X$.

A sequent in the non-commutative side is of the form $\Gamma  \NDsym{,}  \Delta  \vdash_\mathcal{L}  \NDnt{s}  \NDsym{:}  \NDnt{A}$. The types must be $A$,
$B$, $C$, etc., which are objects in the Lambek category $\cat{L}$. Tye typing contexts are
lists instead of multisets. The typing contexts are mixed in the sense that they could include
contexts from the commutative side. When a commutative context $\Psi=\{x_1:X_1,...,x_m:X_m\}$
is included, it is interpreted as the object $F(X_1\otimes...\otimes X_m)$. Therefore, the
interpretation of the sequent $\Psi  \NDsym{,}  \Gamma  \vdash_\mathcal{L}  \NDnt{s}  \NDsym{:}  \NDnt{A}$, where $\Psi$ is defined as above and $\Gamma$
is the list $y_1:A_1,...,y_n:A_n$, is the morphism
$F(X_1\otimes...\otimes X_m)\tri(A_1\tri...\tri A_n)\rightarrow A$.

For the commutative side, since the contexts are multisets, the following exchange rule
is implicit in both sequent calculus and natural deduction:

\begin{center}
  \scriptsize
  $\ElledruleTXXbeta{}$
\end{center}



%%%%%%%%%%%%%%%%%%%%%%%%%%%%%%%%%%%%%%%%%%%%%%%%%%
\subsection{Sequent Calculus}
\label{subsec:elle}

The term assignment for sequent calculus of the commutative part of the model, i.e. the SMCC of
the adjunction, is defined in Figure~\ref{fig:elle-smcc}. And the term assignme for the
non-commutative part, i.e. the Lambek category of the adjunction, is defined in
Figure~\ref{fig:elle-lambek}. We do not have the structural rules except for exchange because
the calculus is for linear logic. 

\begin{figure}[!h]
 \scriptsize
  \begin{mdframed}
    \begin{mathpar}
      \ElledruleTXXax{} \qquad\qquad \ElledruleTXXunitL{} \qquad\qquad \ElledruleTXXunitR{} \\
      \ElledruleTXXtenL{} \qquad\qquad \ElledruleTXXtenR{} \\
      \ElledruleTXXimpL{} \qquad\qquad \ElledruleTXXimpR{} \\
      \ElledruleTXXGr{} \qquad\qquad \ElledruleTXXcut{}
    \end{mathpar}
  \end{mdframed}
\caption{Sequent Calculus: Commutative Part}
\label{fig:elle-smcc}
\end{figure}

\begin{figure}[!h]
 \scriptsize
  \begin{mdframed}
    \begin{mathpar}
      \ElledruleSXXax{} \qquad\qquad \ElledruleSXXunitR{} \qquad\qquad \ElledruleSXXunitLOne{} \\
      \ElledruleSXXunitLTwo{} \qquad\qquad \ElledruleSXXbeta{} \\
      \ElledruleSXXtenLOne{} \qquad\qquad \ElledruleSXXtenLTwo{} \\
      \ElledruleSXXtenR{} \qquad\qquad \ElledruleSXXimpL{} \\
      \ElledruleSXXimprL{} \qquad\qquad \ElledruleSXXimplL{} \\
      \ElledruleSXXimprR{} \qquad\qquad \ElledruleSXXimplR{} \qquad\qquad \ElledruleSXXFr{} \\
      \ElledruleSXXFl{} \qquad\qquad \ElledruleSXXGl{} \\
      \ElledruleSXXcutOne{} \qquad\qquad \ElledruleSXXcutTwo{} \\
    \end{mathpar}
  \end{mdframed}
\caption{Sequent Calculus: Non-Commutative Part}
\label{fig:elle-lambek}
\end{figure}

Next, we prove cut elimination for the sequent calculus. We define the \textit{degree $|X|$
(or $|A|$) of a commutative (or non-commutative) proposition} to be the number of logical
connectives in the proposition. For instance, $|\NDnt{X}  \otimes  \NDnt{Y}| = |\NDnt{X}| + |\NDnt{Y}| + 1$. The
\textit{cut rank} $c(\Pi)$ of a proof $\Pi$ is one more than the maximum of the ranks of all
the cut formulae in $\Pi$, and $0$ if $\Pi$ is cut-free. Then \textit{depth} $d(\Pi)$ of a
proof $\Pi$ is the length of the longest path in the proof tree (so the depth of an axiom is
$0$). The key to the proof of cut elimination is the following lemma, which shows how to
transform a single cut, either by removing it or by replacing it with one or more simpler cuts.

%\textbf{degree of a cut rule} is the degree of the cut formula. The following key cases
%demonstrate how we can replace a cut with at most two cuts with lower degree. The
%\textbf{degree $|\Pi|$ of a proof} $\Pi$ is the maximum of the degrees of all cut fules in the
%proof and $|\Pi|=0$ if $\Pi$ is cut-free. Finally, the \textbf{height $h(\Pi)$ of a proof
%$\Pi$} is the length of the longest path in the proof tree and the height of an axiom is $0$.

\begin{lemma}[Cut Reduction]
  \label{lem:cut-reduction}
  \begin{enumerate}
  \item If $\Pi_1$ is a proof of $\Phi  \vdash_\mathcal{C}  \Ellent{X}$ and $\Pi_2$ is a proof of $\Psi_{{\mathrm{1}}}  \Ellesym{,}  \Ellent{X}  \Ellesym{,}  \Psi_{{\mathrm{2}}}  \vdash_\mathcal{C}  \Ellent{Y}$
        with $c(\Pi_1)$, $c(\Pi_2)\leq |X|$, then there exists a proof $\Pi$ of
        $\Psi_{{\mathrm{1}}}  \Ellesym{,}  \Phi  \Ellesym{,}  \Psi_{{\mathrm{2}}}  \vdash_\mathcal{C}  \Ellent{Y}$ with $c(\Pi)\leq |X|$.
  \item If $\Pi_1$ is a proof of $\Phi  \vdash_\mathcal{C}  \Ellent{X}$ and $\Pi_2$ is a proof of $\Gamma_{{\mathrm{1}}}  \Ellesym{,}  \Ellent{X}  \Ellesym{,}  \Gamma_{{\mathrm{2}}}  \vdash_\mathcal{L}  \Ellent{A}$
        with $c(\Pi_1)$, $c(\Pi_2)\leq |X|$, then there exists a proof $\Pi$ of
        $\Gamma_{{\mathrm{1}}}  \Ellesym{,}  \Phi  \Ellesym{,}  \Gamma_{{\mathrm{2}}}  \vdash_\mathcal{L}  \Ellent{A}$ with $c(\Pi)\leq |X|$.
  \item If $\Pi_1$ is a proof of $\Gamma  \vdash_\mathcal{L}  \Ellent{A}$ and $\Pi_2$ is a proof of $\Delta_{{\mathrm{1}}}  \Ellesym{,}  \Ellent{A}  \Ellesym{,}  \Delta_{{\mathrm{2}}}  \vdash_\mathcal{L}  \Ellent{B}$
        with $c(\Pi_1)$, $c(\Pi_2)\leq |A|$, then there exists a proof $\Pi$ of
        $\Delta_{{\mathrm{1}}}  \Ellesym{,}  \Gamma  \Ellesym{,}  \Delta_{{\mathrm{2}}}  \vdash_\mathcal{L}  \Ellent{B}$ with $c(\Pi)\leq |A|$.
  \end{enumerate}
\end{lemma}

\begin{proof}
  We consider cases based on Mellies \cite{}.
  \begin{enumerate}
  \item Commuting conversion cut vs. cut:
    \begin{itemize}
    \item $\ElledruleTXXcutName/\ElledruleTXXcutName$ Case 1:
      \begin{center}
        \scriptsize
        \begin{math}
          \begin{array}{c}
            \Pi_1 \\
            {\Phi  \vdash_\mathcal{C}  \Ellent{X}}
          \end{array}
        \end{math}
        \qquad\qquad
        $\Pi_2:$
        \begin{math}
          $$\mprset{flushleft}
          \inferrule* [right={\tiny cut}] {
            {
              \begin{array}{cc}
                \pi_1 & \pi_2 \\
                {\Psi_{{\mathrm{2}}}  \Ellesym{,}  \Ellent{X}  \Ellesym{,}  \Psi_{{\mathrm{3}}}  \vdash_\mathcal{C}  \Ellent{Y}} & {\Psi_{{\mathrm{1}}}  \Ellesym{,}  \Ellent{Y}  \Ellesym{,}  \Psi_{{\mathrm{4}}}  \vdash_\mathcal{C}  \Ellent{Z}}
              \end{array}
            }
          }{\Psi_{{\mathrm{1}}}  \Ellesym{,}  \Psi_{{\mathrm{2}}}  \Ellesym{,}  \Ellent{X}  \Ellesym{,}  \Psi_{{\mathrm{3}}}  \Ellesym{,}  \Psi_{{\mathrm{4}}}  \vdash_\mathcal{C}  \Ellent{Z}}
        \end{math}
      \end{center}
      By assumption, $c(\Pi_1),c(\Pi_2)\leq |X|$. Therefore, $c(\pi_1),c(\pi_2)\leq |X|$.
      Since $Y$ is the cut formula on $\pi_1$ and $\pi_2$, we have $|Y|+1\leq|X|$. By
      induction on $\Pi_1$ and $\pi_1$, there exists a proof $\Pi'$ for sequent
      $\Psi_{{\mathrm{2}}}  \Ellesym{,}  \Phi  \Ellesym{,}  \Psi_{{\mathrm{3}}}  \vdash_\mathcal{C}  \Ellent{Y}$ s.t. $c(\Pi')\leq|X|$. So $\Pi$ can be
      constructed as follows, with $c(\Pi)\leq max\{c(\Pi'),c(\pi_2),|Y|+1\}\leq |X|$.
      \begin{center}
        \scriptsize
        \begin{math}
          $$\mprset{flushleft}
          \inferrule* [right={\tiny cut}] {
            {
              \begin{array}{cc}
                \Pi' & \pi_2 \\
                {\Psi_{{\mathrm{2}}}  \Ellesym{,}  \Phi  \Ellesym{,}  \Psi_{{\mathrm{3}}}  \vdash_\mathcal{C}  \Ellent{Y}} & {\Psi_{{\mathrm{1}}}  \Ellesym{,}  \Ellent{Y}  \Ellesym{,}  \Psi_{{\mathrm{4}}}  \vdash_\mathcal{C}  \Ellent{Z}}
              \end{array}
            }
          }{\Psi_{{\mathrm{1}}}  \Ellesym{,}  \Psi_{{\mathrm{2}}}  \Ellesym{,}  \Phi  \Ellesym{,}  \Psi_{{\mathrm{3}}}  \Ellesym{,}  \Psi_{{\mathrm{4}}}  \vdash_\mathcal{C}  \Ellent{Z}}
        \end{math}
      \end{center}

    \item $\ElledruleTXXcutName/\ElledruleTXXcutName$ Case 2:
      \begin{center}
        \scriptsize
        $\Pi_1$:
        \begin{math}
          $$\mprset{flushleft}
          \inferrule* [right={\tiny cut}] {
            {
              \begin{array}{cc}
                \pi_1 & \pi_2 \\
                {\Phi  \vdash_\mathcal{C}  \Ellent{X}} & {\Psi_{{\mathrm{2}}}  \Ellesym{,}  \Ellent{X}  \Ellesym{,}  \Psi_{{\mathrm{3}}}  \vdash_\mathcal{C}  \Ellent{Y}}
              \end{array}
            }
          }{\Psi_{{\mathrm{2}}}  \Ellesym{,}  \Phi  \Ellesym{,}  \Psi_{{\mathrm{3}}}  \vdash_\mathcal{C}  \Ellent{Y}}
        \end{math}
        \qquad\qquad
        \begin{math}
          \begin{array}{c}
            \Pi_2 \\
            {\Psi_{{\mathrm{1}}}  \Ellesym{,}  \Ellent{Y}  \Ellesym{,}  \Psi_{{\mathrm{4}}}  \vdash_\mathcal{C}  \Ellent{Z}}
          \end{array}
        \end{math}
      \end{center}
      By assumption, $c(\Pi_1),c(\Pi_2)\leq |Y|$. Since the cut rank of the last cut in
      $\Pi_1$ is $|X|+1$, then $|X|+1\leq |Y|$. By induction on $\Pi_1$ and $\Pi_2$, there is
      a proof $\Pi'$ for sequent $\Psi_{{\mathrm{1}}}  \Ellesym{,}  \Psi_{{\mathrm{2}}}  \Ellesym{,}  \Ellent{X}  \Ellesym{,}  \Psi_{{\mathrm{3}}}  \Ellesym{,}  \Psi_{{\mathrm{4}}}  \vdash_\mathcal{C}  \Ellent{Z}$ s.t. $c(\Pi')\leq|Y|$.
      Therefore, the proof $\Pi$ can be constructed as follows, and
      $c(\Pi)\leq max\{c(\pi_1),c(\Pi'),|X|+1\}\leq |Y|$.
      \begin{center}
        \scriptsize
        \begin{math}
          $$\mprset{flushleft}
          \inferrule* [right={\tiny cut}] {
            {
              \begin{array}{cc}
                \pi_1 & \Pi' \\
                {\Phi  \vdash_\mathcal{C}  \Ellent{X}} & {\Psi_{{\mathrm{1}}}  \Ellesym{,}  \Psi_{{\mathrm{2}}}  \Ellesym{,}  \Ellent{X}  \Ellesym{,}  \Psi_{{\mathrm{3}}}  \Ellesym{,}  \Psi_{{\mathrm{4}}}  \vdash_\mathcal{C}  \Ellent{Z}}
              \end{array}
            }
          }{\Psi_{{\mathrm{1}}}  \Ellesym{,}  \Psi_{{\mathrm{2}}}  \Ellesym{,}  \Phi  \Ellesym{,}  \Psi_{{\mathrm{3}}}  \Ellesym{,}  \Psi_{{\mathrm{4}}}  \vdash_\mathcal{C}  \Ellent{Z}}
        \end{math}
      \end{center}

    \item $\ElledruleTXXcutName/\ElledruleSXXcutOneName$ Case 1:
      \begin{center}
        \scriptsize
        \begin{math}
          \begin{array}{c}
            \Pi_1 \\
            {\Phi  \vdash_\mathcal{C}  \Ellent{X}}
          \end{array}
        \end{math}
        \qquad\qquad
        $\Pi_2:$
        \begin{math}
          $$\mprset{flushleft}
          \inferrule* [right={\tiny cut1}] {
            {
              \begin{array}{cc}
                \pi_2 & \pi_3 \\
                {\Psi_{{\mathrm{1}}}  \Ellesym{,}  \Ellent{X}  \Ellesym{,}  \Psi_{{\mathrm{2}}}  \vdash_\mathcal{C}  \Ellent{Y}} & {\Gamma_{{\mathrm{1}}}  \Ellesym{,}  \Ellent{Y}  \Ellesym{,}  \Gamma_{{\mathrm{2}}}  \vdash_\mathcal{L}  \Ellent{A}}
              \end{array}
            }
          }{\Gamma_{{\mathrm{1}}}  \Ellesym{,}  \Psi_{{\mathrm{1}}}  \Ellesym{,}  \Ellent{X}  \Ellesym{,}  \Psi_{{\mathrm{2}}}  \Ellesym{,}  \Gamma_{{\mathrm{2}}}  \vdash_\mathcal{L}  \Ellent{A}}
        \end{math}
      \end{center}
      By assumption, $c(\Pi_1),c(\Pi_2)\leq |X|$. Therefore, $c(\pi_1),c(\pi_2)\leq |X|$.
      Since $Y$ is the cut formula on $\pi_1$ and $\pi_2$, we have $|Y|+1\leq|X|$. By
      induction on $\Pi_1$ and $\pi_1$, there exists a proof $\Pi'$ for sequent
      $\Psi_{{\mathrm{1}}}  \Ellesym{,}  \Phi  \Ellesym{,}  \Psi_{{\mathrm{2}}}  \vdash_\mathcal{C}  \Ellent{Y}$ s.t. $c(\Pi')\leq|X|$. So $\Pi$ can be constructed as follows,
      with $c(\Pi)\leq max\{c(\Pi'),c(\pi_2),|Y|+1\}\leq |X|$.
      \begin{center}
        \scriptsize
        \begin{math}
          $$\mprset{flushleft}
          \inferrule* [right={\tiny cut1}] {
            {
              \begin{array}{cc}
                \Pi' & \pi_2 \\
                {\Psi_{{\mathrm{1}}}  \Ellesym{,}  \Phi  \Ellesym{,}  \Psi_{{\mathrm{2}}}  \vdash_\mathcal{C}  \Ellent{Y}} & {\Gamma_{{\mathrm{1}}}  \Ellesym{,}  \Ellent{Y}  \Ellesym{,}  \Gamma_{{\mathrm{2}}}  \vdash_\mathcal{L}  \Ellent{A}}
              \end{array}
            }
          }{\Gamma_{{\mathrm{1}}}  \Ellesym{,}  \Psi_{{\mathrm{1}}}  \Ellesym{,}  \Phi  \Ellesym{,}  \Psi_{{\mathrm{2}}}  \Ellesym{,}  \Gamma_{{\mathrm{2}}}  \vdash_\mathcal{L}  \Ellent{A}}
        \end{math}
      \end{center}

    \item $\ElledruleTXXcutName/\ElledruleSXXcutOneName$ Case 2:
      \begin{center}
        \scriptsize
        $\Pi_1$:
        \begin{math}
          $$\mprset{flushleft}
          \inferrule* [right={\tiny cut}] {
            {
              \begin{array}{cc}
                \pi_1 & \pi_2 \\
                {\Phi  \vdash_\mathcal{C}  \Ellent{X}} & {\Psi_{{\mathrm{1}}}  \Ellesym{,}  \Ellent{X}  \Ellesym{,}  \Psi_{{\mathrm{2}}}  \vdash_\mathcal{C}  \Ellent{Y}}
              \end{array}
            }
          }{\Psi_{{\mathrm{1}}}  \Ellesym{,}  \Phi  \Ellesym{,}  \Psi_{{\mathrm{2}}}  \vdash_\mathcal{C}  \Ellent{Y}}
        \end{math}
        \qquad\qquad
        \begin{math}
          \begin{array}{c}
            \Pi_2 \\
            {\Gamma_{{\mathrm{1}}}  \Ellesym{,}  \Ellent{Y}  \Ellesym{,}  \Gamma_{{\mathrm{2}}}  \vdash_\mathcal{L}  \Ellent{A}}
          \end{array}
        \end{math}
      \end{center}
      By assumption, $c(\Pi_1),c(\Pi_2)\leq |Y|$. Similar as above, $|X|+1\leq |Y|$ and there
      is a proof $\Pi'$ constructed from $\pi_2$ and $\Pi_2$ for sequent
      $\Gamma_{{\mathrm{1}}}  \Ellesym{,}  \Psi_{{\mathrm{1}}}  \Ellesym{,}  \Ellent{X}  \Ellesym{,}  \Psi_{{\mathrm{2}}}  \Ellesym{,}  \Gamma_{{\mathrm{2}}}  \vdash_\mathcal{L}  \Ellent{A}$ s.t. $c(\Pi')\leq|Y|$. Therefore, the proof $\Pi$ can be
      constructed as follows, and $c(\Pi)\leq max\{c(\pi_1),c(\Pi'),|X|+1\}\leq |Y|$.
      \begin{center}
        \scriptsize
        \begin{math}
          $$\mprset{flushleft}
          \inferrule* [right={\tiny cut}] {
            {
              \begin{array}{cc}
                \pi_1 & \Pi'\\
                {\Phi  \vdash_\mathcal{C}  \Ellent{X}} & {\Gamma_{{\mathrm{1}}}  \Ellesym{,}  \Psi_{{\mathrm{1}}}  \Ellesym{,}  \Ellent{X}  \Ellesym{,}  \Psi_{{\mathrm{2}}}  \Ellesym{,}  \Gamma_{{\mathrm{2}}}  \vdash_\mathcal{L}  \Ellent{A}}
              \end{array}
            }
          }{\Gamma_{{\mathrm{1}}}  \Ellesym{,}  \Psi_{{\mathrm{1}}}  \Ellesym{,}  \Phi  \Ellesym{,}  \Psi_{{\mathrm{2}}}  \Ellesym{,}  \Gamma_{{\mathrm{2}}}  \vdash_\mathcal{L}  \Ellent{A}}
        \end{math}
      \end{center}

    \item $\ElledruleSXXcutOneName/\ElledruleSXXcutTwoName$ Case 1:
      \begin{center}
        \scriptsize
        \begin{math}
          \begin{array}{c}
            \Pi_1 \\
            {\Phi  \vdash_\mathcal{C}  \Ellent{X}}
          \end{array}
        \end{math}
        \qquad\qquad
        $\Pi_2:$
        \begin{math}
          $$\mprset{flushleft}
          \inferrule* [right={\tiny cut2}] {
            {
              \begin{array}{cc}
                \pi_1 & \pi_2 \\
                {\Gamma_{{\mathrm{2}}}  \Ellesym{,}  \Ellent{X}  \Ellesym{,}  \Gamma_{{\mathrm{3}}}  \vdash_\mathcal{L}  \Ellent{A}} & {\Gamma_{{\mathrm{1}}}  \Ellesym{,}  \Ellent{A}  \Ellesym{,}  \Gamma_{{\mathrm{4}}}  \vdash_\mathcal{L}  \Ellent{B}}
              \end{array}
            }
          }{\Gamma_{{\mathrm{1}}}  \Ellesym{,}  \Gamma_{{\mathrm{2}}}  \Ellesym{,}  \Ellent{X}  \Ellesym{,}  \Gamma_{{\mathrm{3}}}  \Ellesym{,}  \Gamma_{{\mathrm{4}}}  \vdash_\mathcal{L}  \Ellent{B}}
        \end{math}
      \end{center}
      By assumption, $c(\Pi_1),c(\Pi_2)\leq |X|$. Therefore, $c(\pi_1),c(\pi_2)\leq |X|$.
      Since $A$ is the cut formula on $\pi_1$ and $\pi_2$, we have $|A|+1\leq|X|$. By
      induction on $\Pi_1$ and $\pi_1$, there exists a proof $\Pi'$ for sequent
      $\Gamma_{{\mathrm{2}}}  \Ellesym{,}  \Phi  \Ellesym{,}  \Gamma_{{\mathrm{3}}}  \vdash_\mathcal{L}  \Ellent{A}$ s.t. $c(\Pi')\leq|X|$. So $\Pi$ can be constructed as follows,
      with $c(\Pi)\leq max\{c(\Pi'),c(\pi_2),|A|+1\}\leq |X|$.
      \begin{center}
        \scriptsize
        \begin{math}
          $$\mprset{flushleft}
          \inferrule* [right={\tiny cut2}] {
            {
              \begin{array}{cc}
                \Pi' & \pi_2 \\
                {\Gamma_{{\mathrm{2}}}  \Ellesym{,}  \Phi  \Ellesym{,}  \Gamma_{{\mathrm{3}}}  \vdash_\mathcal{L}  \Ellent{A}} & {\Gamma_{{\mathrm{1}}}  \Ellesym{,}  \Ellent{A}  \Ellesym{,}  \Gamma_{{\mathrm{4}}}  \vdash_\mathcal{L}  \Ellent{B}}
              \end{array}
            }
          }{\Gamma_{{\mathrm{1}}}  \Ellesym{,}  \Gamma_{{\mathrm{2}}}  \Ellesym{,}  \Phi  \Ellesym{,}  \Gamma_{{\mathrm{3}}}  \Ellesym{,}  \Gamma_{{\mathrm{4}}}  \vdash_\mathcal{L}  \Ellent{B}}
        \end{math}
      \end{center}

    \item $\ElledruleSXXcutOneName/\ElledruleSXXcutTwoName$ Case 2:
      \begin{center}
        \scriptsize
        $\Pi_1$:
        \begin{math}
          $$\mprset{flushleft}
          \inferrule* [right={\tiny cut}] {
            {
              \begin{array}{cc}
                \pi_1 & \pi_2 \\
                {\Phi  \vdash_\mathcal{C}  \Ellent{X}} & {\Gamma_{{\mathrm{2}}}  \Ellesym{,}  \Ellent{X}  \Ellesym{,}  \Gamma_{{\mathrm{3}}}  \vdash_\mathcal{L}  \Ellent{A}}
              \end{array}
            }
          }{\Gamma_{{\mathrm{2}}}  \Ellesym{,}  \Phi  \Ellesym{,}  \Gamma_{{\mathrm{3}}}  \vdash_\mathcal{L}  \Ellent{A}}
        \end{math}
        \qquad\qquad
        \begin{math}
          \begin{array}{c}
            \Pi_2 \\
            {\Gamma_{{\mathrm{1}}}  \Ellesym{,}  \Ellent{A}  \Ellesym{,}  \Gamma_{{\mathrm{4}}}  \vdash_\mathcal{L}  \Ellent{B}}
          \end{array}
        \end{math}
      \end{center}
      By assumption, $c(\Pi_1),c(\Pi_2)\leq |A|$. Similar as above, $|X|+1\leq |A|$ and there
      is a proof $\Pi'$ constructed from $\pi_2$ and $\Pi_2$ for sequent
      $\Gamma_{{\mathrm{1}}}  \Ellesym{,}  \Gamma_{{\mathrm{2}}}  \Ellesym{,}  \Ellent{X}  \Ellesym{,}  \Gamma_{{\mathrm{3}}}  \Ellesym{,}  \Gamma_{{\mathrm{4}}}  \vdash_\mathcal{L}  \Ellent{B}$ s.t. $c(\Pi')\leq|A|$. Therefore, the proof $\Pi$ can be
      constructed as follows, and $c(\Pi)\leq max\{c(\pi_1),c(\Pi'),|X|+1\}\leq |A|$.
      \begin{center}
        \scriptsize
        \begin{math}
          $$\mprset{flushleft}
          \inferrule* [right={\tiny cut}] {
            {
              \begin{array}{cc}
                \pi_1  & \Pi' \\
                {\Phi  \vdash_\mathcal{C}  \Ellent{X}} & {\Gamma_{{\mathrm{1}}}  \Ellesym{,}  \Gamma_{{\mathrm{2}}}  \Ellesym{,}  \Ellent{X}  \Ellesym{,}  \Gamma_{{\mathrm{3}}}  \Ellesym{,}  \Gamma_{{\mathrm{4}}}  \vdash_\mathcal{L}  \Ellent{B}}
              \end{array}
            }
          }{\Gamma_{{\mathrm{1}}}  \Ellesym{,}  \Gamma_{{\mathrm{2}}}  \Ellesym{,}  \Phi  \Ellesym{,}  \Gamma_{{\mathrm{3}}}  \Ellesym{,}  \Gamma_{{\mathrm{4}}}  \vdash_\mathcal{L}  \Ellent{B}}
        \end{math}
      \end{center}

    \item $\ElledruleSXXcutTwoName/\ElledruleSXXcutTwoName$ Case 1:
      \begin{center}
        \scriptsize
        \begin{math}
          \begin{array}{c}
            \Pi_1 \\
            {\Gamma  \vdash_\mathcal{L}  \Ellent{A}}
          \end{array}
        \end{math}
        \qquad\qquad
        $\Pi_2:$
        \begin{math}
          $$\mprset{flushleft}
          \inferrule* [right={\tiny cut2}] {
            {
              \begin{array}{cc}
                \pi_1 & \pi_2 \\
                {\Delta_{{\mathrm{2}}}  \Ellesym{,}  \Ellent{A}  \Ellesym{,}  \Delta_{{\mathrm{3}}}  \vdash_\mathcal{L}  \Ellent{B}} & {\Delta_{{\mathrm{1}}}  \Ellesym{,}  \Ellent{B}  \Ellesym{,}  \Delta_{{\mathrm{4}}}  \vdash_\mathcal{L}  \Ellent{C}}
              \end{array}
            }
          }{\Delta_{{\mathrm{1}}}  \Ellesym{,}  \Delta_{{\mathrm{2}}}  \Ellesym{,}  \Ellent{A}  \Ellesym{,}  \Delta_{{\mathrm{3}}}  \Ellesym{,}  \Delta_{{\mathrm{4}}}  \vdash_\mathcal{L}  \Ellent{C}}
        \end{math}
      \end{center}
      By assumption, $c(\Pi_1),c(\Pi_2)\leq |A|$. Therefore, $c(\pi_1),c(\pi_2)\leq |A|$.
      Since $B$ is the cut formula on $\pi_1$ and $\pi_3$, we have $|B|+1\leq|A|$. By
      induction on $\Pi_1$ and $\pi_1$, there exists a proof $\Pi'$ for sequent
      $\Delta_{{\mathrm{2}}}  \Ellesym{,}  \Gamma  \Ellesym{,}  \Delta_{{\mathrm{3}}}  \vdash_\mathcal{L}  \Ellent{B}$ s.t. $c(\Pi')\leq|A|$. So $\Pi$ can be constructed as follows,
      with $c(\Pi)\leq max\{c(\Pi'),c(\pi_2),|B|+1\}\leq |A|$.
      \begin{center}
        \scriptsize
        \begin{math}
          $$\mprset{flushleft}
          \inferrule* [right={\tiny cut}] {
            {
              \begin{array}{cc}
                \Pi' & \pi_2 \\
                {\Delta_{{\mathrm{2}}}  \Ellesym{,}  \Gamma  \Ellesym{,}  \Delta_{{\mathrm{3}}}  \vdash_\mathcal{L}  \Ellent{B}} & {\Delta_{{\mathrm{1}}}  \Ellesym{,}  \Ellent{B}  \Ellesym{,}  \Delta_{{\mathrm{4}}}  \vdash_\mathcal{L}  \Ellent{C}}
              \end{array}
            }
          }{\Delta_{{\mathrm{1}}}  \Ellesym{,}  \Delta_{{\mathrm{2}}}  \Ellesym{,}  \Gamma  \Ellesym{,}  \Delta_{{\mathrm{3}}}  \Ellesym{,}  \Delta_{{\mathrm{4}}}  \vdash_\mathcal{L}  \Ellent{C}}
        \end{math}
      \end{center}

    \item $\ElledruleSXXcutTwoName/\ElledruleSXXcutTwoName$ Case 2:
      \begin{center}
        \scriptsize
        $\Pi_1$:
        \begin{math}
          $$\mprset{flushleft}
          \inferrule* [right={\tiny cut}] {
            {
              \begin{array}{cc}
                \pi_1 & \pi_2 \\
                {\Delta  \vdash_\mathcal{L}  \Ellent{A}} & {\Delta_{{\mathrm{2}}}  \Ellesym{,}  \Ellent{A}  \Ellesym{,}  \Delta_{{\mathrm{3}}}  \vdash_\mathcal{L}  \Ellent{B}}
              \end{array}
            }
          }{\Delta_{{\mathrm{2}}}  \Ellesym{,}  \Delta  \Ellesym{,}  \Delta_{{\mathrm{3}}}  \vdash_\mathcal{L}  \Ellent{A}}
        \end{math}
        \qquad\qquad
        \begin{math}
          \begin{array}{c}
            \Pi_2 \\
            {\Delta_{{\mathrm{1}}}  \Ellesym{,}  \Ellent{B}  \Ellesym{,}  \Delta_{{\mathrm{4}}}  \vdash_\mathcal{L}  \Ellent{C}}
          \end{array}
        \end{math}
      \end{center}
      By assumption, $c(\Pi_1),c(\Pi_2)\leq |B|$. Similar as above, $|A|+1\leq |B|$ and there
      is a proof $\Pi'$ constructed from $\pi_2$ and $\Pi_2$ for sequent
      $\Delta_{{\mathrm{1}}}  \Ellesym{,}  \Delta_{{\mathrm{2}}}  \Ellesym{,}  \Ellent{A}  \Ellesym{,}  \Delta_{{\mathrm{3}}}  \Ellesym{,}  \Delta_{{\mathrm{4}}}  \vdash_\mathcal{L}  \Ellent{C}$ s.t. $c(\Pi')\leq|A|$. Therefore, the proof $\Pi$ can be
      constructed as follows, and $c(\Pi)\leq max\{c(\pi_1),c(\Pi'),|A|+1\}\leq |B|$.
      \begin{center}
        \scriptsize
        \begin{math}
          $$\mprset{flushleft}
          \inferrule* [right={\tiny cut}] {
            {
              \begin{array}{cc}
                \pi_1 & \Pi' \\
                {\Gamma  \vdash_\mathcal{L}  \Ellent{A}} & {\Delta_{{\mathrm{1}}}  \Ellesym{,}  \Delta_{{\mathrm{2}}}  \Ellesym{,}  \Ellent{A}  \Ellesym{,}  \Delta_{{\mathrm{3}}}  \Ellesym{,}  \Delta_{{\mathrm{4}}}  \vdash_\mathcal{L}  \Ellent{C}}
              \end{array}
            }
          }{\Delta_{{\mathrm{1}}}  \Ellesym{,}  \Delta_{{\mathrm{2}}}  \Ellesym{,}  \Gamma  \Ellesym{,}  \Delta_{{\mathrm{3}}}  \Ellesym{,}  \Delta_{{\mathrm{4}}}  \vdash_\mathcal{L}  \Ellent{C}}
        \end{math}
      \end{center}
    
    \end{itemize}

  \item The $\eta$-expansion steps (?)
  \item The axiom steps:
    \begin{itemize}
    \item $\ElledruleTXXaxName$ Case 1:
      \begin{center}
        \scriptsize
        $\Pi_1$:
        \begin{math}
          $$\mprset{flushleft}
          \inferrule* [right={\tiny ax}] {
            \,
          }{\Ellent{X}  \vdash_\mathcal{C}  \Ellent{X}}
        \end{math}
        \qquad\qquad
        \begin{math}
          \begin{array}{c}
            \Pi_2 \\
            {\Phi_{{\mathrm{1}}}  \Ellesym{,}  \Ellent{X}  \Ellesym{,}  \Phi_{{\mathrm{2}}}  \vdash_\mathcal{C}  \Ellent{Y}}
          \end{array}
        \end{math}
      \end{center}
      By assumption, $c(\Pi_1),c(\Pi_2)\leq |X|$. The proof $\Pi$ is the same as $\Pi_2$.

    \item $\ElledruleTXXaxName$ Case 2:
      \begin{center}
        \scriptsize
        $\Pi_1$:
        \begin{math}
          $$\mprset{flushleft}
          \inferrule* [right={\tiny ax}] {
            \,
          }{\Ellent{X}  \vdash_\mathcal{C}  \Ellent{X}}
        \end{math}
        \qquad\qquad
        \begin{math}
          \begin{array}{c}
            \Pi_2 \\
            {\Gamma_{{\mathrm{1}}}  \Ellesym{,}  \Ellent{X}  \Ellesym{,}  \Gamma_{{\mathrm{2}}}  \vdash_\mathcal{L}  \Ellent{A}}
          \end{array}
        \end{math}
      \end{center}
      By assumption, $c(\Pi_1),c(\Pi_2)\leq |X|$. The proof $\Pi$ is the same as $\Pi_2$.

    \item $\ElledruleSXXaxName$:
      \begin{center}
        \scriptsize
        $\Pi_1$:
        \begin{math}
          $$\mprset{flushleft}
          \inferrule* [right={\tiny ax}] {
            \,
          }{\Ellent{A}  \vdash_\mathcal{L}  \Ellent{A}}
        \end{math}
        \qquad\qquad
        \begin{math}
          \begin{array}{c}
            \Pi_2 \\
            {\Gamma_{{\mathrm{1}}}  \Ellesym{,}  \Ellent{A}  \Ellesym{,}  \Gamma_{{\mathrm{2}}}  \vdash_\mathcal{L}  \Ellent{B}}
          \end{array}
        \end{math}
      \end{center}
      By assumption, $c(\Pi_1),c(\Pi_2)\leq |A|$. The proof $\Pi$ is the same as $\Pi_2$.

    
    \end{itemize}
  
  \item The exchange steps:
    \begin{itemize}
    \item $\ElledruleTXXbetaName / \ElledruleTXXcutName$ Case 1:
      \begin{center}
        \scriptsize
        $\Pi_1$:
        \begin{math}
          $$\mprset{flushleft}
          \inferrule* [right={\tiny beta}] {
            {
              \begin{array}{c}
                \pi \\
                {\Phi_{{\mathrm{1}}}  \Ellesym{,}  \Ellent{X_{{\mathrm{1}}}}  \Ellesym{,}  \Ellent{X_{{\mathrm{2}}}}  \Ellesym{,}  \Phi_{{\mathrm{2}}}  \vdash_\mathcal{C}  \Ellent{Y}}
              \end{array}
            }
          }{\Phi_{{\mathrm{1}}}  \Ellesym{,}  \Ellent{X_{{\mathrm{2}}}}  \Ellesym{,}  \Ellent{X_{{\mathrm{1}}}}  \Ellesym{,}  \Phi_{{\mathrm{2}}}  \vdash_\mathcal{C}  \Ellent{Y}}
        \end{math}
        \qquad\qquad
        \begin{math}
          \begin{array}{c}
            \Pi_2 \\
            {\Psi_{{\mathrm{1}}}  \Ellesym{,}  \Ellent{Y}  \Ellesym{,}  \Psi_{{\mathrm{2}}}  \vdash_\mathcal{C}  \Ellent{Z}}
          \end{array}
        \end{math}
      \end{center}
      By assumption, $c(\Pi_1),c(\Pi_2)\leq |Y|$. By induction on $\pi$ and $\Pi_2$, there is
      a proof $\Pi'$ for sequent $\Psi_{{\mathrm{1}}}  \Ellesym{,}  \Phi_{{\mathrm{1}}}  \Ellesym{,}  \Ellent{X_{{\mathrm{1}}}}  \Ellesym{,}  \Ellent{X_{{\mathrm{2}}}}  \Ellesym{,}  \Phi_{{\mathrm{2}}}  \Ellesym{,}  \Psi_{{\mathrm{2}}}  \vdash_\mathcal{C}  \Ellent{Z}$ s.t. $c(\Pi')\leq|Y|$.
      Therefore, the proof $\Pi$ can be constructed as follows, and $c(\Pi)=c(\Pi')\leq|Y|$.
      \begin{center}
        \scriptsize
        \begin{math}
          $$\mprset{flushleft}
          \inferrule* [right={\tiny beta}] {
            {
              \begin{array}{c}
                \Pi' \\
                {\Psi_{{\mathrm{1}}}  \Ellesym{,}  \Phi_{{\mathrm{1}}}  \Ellesym{,}  \Ellent{X_{{\mathrm{1}}}}  \Ellesym{,}  \Ellent{X_{{\mathrm{2}}}}  \Ellesym{,}  \Phi_{{\mathrm{2}}}  \Ellesym{,}  \Psi_{{\mathrm{2}}}  \vdash_\mathcal{C}  \Ellent{Z}}
              \end{array}
            }
          }{\Psi_{{\mathrm{1}}}  \Ellesym{,}  \Phi_{{\mathrm{1}}}  \Ellesym{,}  \Ellent{X_{{\mathrm{2}}}}  \Ellesym{,}  \Ellent{X_{{\mathrm{1}}}}  \Ellesym{,}  \Phi_{{\mathrm{2}}}  \Ellesym{,}  \Psi_{{\mathrm{2}}}  \vdash_\mathcal{C}  \Ellent{Z}}
        \end{math}
      \end{center}

    \item $\ElledruleTXXbetaName / \ElledruleTXXcutName$ Case 1:
      \begin{center}
        \scriptsize
        \begin{math}
          \begin{array}{c}
            \Pi_1 \\
            {\Psi  \vdash_\mathcal{C}  \Ellent{X_{{\mathrm{1}}}}}
          \end{array}
        \end{math}
        \qquad\qquad
        $\Pi_2$:
        \begin{math}
          $$\mprset{flushleft}
          \inferrule* [right={\tiny beta}] {
            {
              \begin{array}{c}
                \pi \\
                {\Phi_{{\mathrm{1}}}  \Ellesym{,}  \Ellent{X_{{\mathrm{1}}}}  \Ellesym{,}  \Ellent{X_{{\mathrm{2}}}}  \Ellesym{,}  \Phi_{{\mathrm{2}}}  \vdash_\mathcal{C}  \Ellent{Y}}
              \end{array}
            }
          }{\Phi_{{\mathrm{1}}}  \Ellesym{,}  \Ellent{X_{{\mathrm{2}}}}  \Ellesym{,}  \Ellent{X_{{\mathrm{1}}}}  \Ellesym{,}  \Phi_{{\mathrm{2}}}  \vdash_\mathcal{C}  \Ellent{Y}}
        \end{math}
      \end{center}
      By assumption, $c(\Pi_1),c(\Pi_2)\leq |X_1|$. By induction on $\pi$ and $\Pi_2$, there is
      a proof $\Pi'$ for sequent $\Phi_{{\mathrm{1}}}  \Ellesym{,}  \Psi  \Ellesym{,}  \Ellent{X_{{\mathrm{2}}}}  \Ellesym{,}  \Phi_{{\mathrm{2}}}  \vdash_\mathcal{C}  \Ellent{Y}$ s.t. $c(\Pi')\leq |X_1|$. Therefore,
      the proof $\Pi$ can be constructed as follows, and $c(\Pi)=c(\Pi')\leq |X_1|$.
      \begin{center}
        \scriptsize
        \begin{math}
          $$\mprset{flushleft}
          \inferrule* [right={\tiny beta}] {
            {
              \begin{array}{c}
                \Pi' \\
                {\Phi_{{\mathrm{1}}}  \Ellesym{,}  \Psi  \Ellesym{,}  \Ellent{X_{{\mathrm{2}}}}  \Ellesym{,}  \Phi_{{\mathrm{2}}}  \vdash_\mathcal{C}  \Ellent{Y}}
              \end{array}
            }
          }{\Phi_{{\mathrm{1}}}  \Ellesym{,}  \Ellent{X_{{\mathrm{2}}}}  \Ellesym{,}  \Psi  \Ellesym{,}  \Phi_{{\mathrm{2}}}  \vdash_\mathcal{C}  \Ellent{Y}}
        \end{math}
      \end{center}

    \item $\ElledruleTXXbetaName / \ElledruleSXXcutOneName$:
      \begin{center}
        \scriptsize
        $\Pi_1$:
        \begin{math}
          $$\mprset{flushleft}
          \inferrule* [right={\tiny beta}] {
            {
              \begin{array}{c}
                \pi \\
                {\Phi_{{\mathrm{1}}}  \Ellesym{,}  \Ellent{X}  \Ellesym{,}  \Ellent{Y}  \Ellesym{,}  \Phi_{{\mathrm{2}}}  \vdash_\mathcal{C}  \Ellent{Z}}
              \end{array}
            }
          }{\Phi_{{\mathrm{1}}}  \Ellesym{,}  \Ellent{Y}  \Ellesym{,}  \Ellent{X}  \Ellesym{,}  \Phi_{{\mathrm{2}}}  \vdash_\mathcal{C}  \Ellent{Z}}
        \end{math}
        \qquad\qquad
        \begin{math}
          \begin{array}{c}
            \Pi_2 \\
            {\Gamma_{{\mathrm{1}}}  \Ellesym{,}  \Ellent{Z}  \Ellesym{,}  \Gamma_{{\mathrm{2}}}  \vdash_\mathcal{L}  \Ellent{A}}
          \end{array}
        \end{math}
      \end{center}
      By assumption, $c(\Pi_1),c(\Pi_2)\leq |Z|$. Similar as above, there is a proof $\Pi'$
      constructed from $\pi$ and $\Pi_2$ for sequent $\Gamma_{{\mathrm{1}}}  \Ellesym{,}  \Phi_{{\mathrm{1}}}  \Ellesym{,}  \Ellent{X}  \Ellesym{,}  \Ellent{Y}  \Ellesym{,}  \Phi_{{\mathrm{2}}}  \Ellesym{,}  \Gamma_{{\mathrm{2}}}  \vdash_\mathcal{L}  \Ellent{A}$ s.t.
      $c(\Pi')\leq|Z|$. Therefore, the proof $\Pi$ can be constructed as follows, and
      $c(\Pi)=c(\Pi')\leq|Z|$.
      \begin{center}
        \scriptsize
        \begin{math}
          $$\mprset{flushleft}
          \inferrule* [right={\tiny beta}] {
            {
              \begin{array}{c}
                \Pi' \\
                {\Gamma_{{\mathrm{1}}}  \Ellesym{,}  \Phi_{{\mathrm{1}}}  \Ellesym{,}  \Ellent{X}  \Ellesym{,}  \Ellent{Y}  \Ellesym{,}  \Phi_{{\mathrm{2}}}  \Ellesym{,}  \Gamma_{{\mathrm{2}}}  \vdash_\mathcal{L}  \Ellent{A}}
              \end{array}
            }
          }{\Gamma_{{\mathrm{1}}}  \Ellesym{,}  \Phi_{{\mathrm{1}}}  \Ellesym{,}  \Ellent{Y}  \Ellesym{,}  \Ellent{X}  \Ellesym{,}  \Phi_{{\mathrm{2}}}  \Ellesym{,}  \Gamma_{{\mathrm{2}}}  \vdash_\mathcal{L}  \Ellent{A}}
        \end{math}
      \end{center}

    \item $\ElledruleSXXbetaName / \ElledruleSXXcutTwoName$ Case 1:
      \begin{center}
        \scriptsize
        $\Pi_1$:
        \begin{math}
          $$\mprset{flushleft}
          \inferrule* [right={\tiny beta}] {
            {
              \begin{array}{c}
                \pi \\
                {\Gamma_{{\mathrm{1}}}  \Ellesym{,}  \Ellent{X}  \Ellesym{,}  \Ellent{Y}  \Ellesym{,}  \Gamma_{{\mathrm{2}}}  \vdash_\mathcal{L}  \Ellent{A}}
              \end{array}
            }
          }{\Gamma_{{\mathrm{1}}}  \Ellesym{,}  \Ellent{Y}  \Ellesym{,}  \Ellent{X}  \Ellesym{,}  \Gamma_{{\mathrm{2}}}  \vdash_\mathcal{L}  \Ellent{A}}
        \end{math}
        \qquad\qquad
        \begin{math}
          \begin{array}{c}
            \Pi_2 \\
            {\Delta_{{\mathrm{1}}}  \Ellesym{,}  \Ellent{A}  \Ellesym{,}  \Delta_{{\mathrm{2}}}  \vdash_\mathcal{L}  \Ellent{B}}
          \end{array}
        \end{math}
      \end{center}
      By assumption, $c(\Pi_1),c(\Pi_2)\leq |A|$. Similar as above, there is a proof $\Pi'$
      constructed from $\pi$ and $\Pi_2$ for sequent $\Delta_{{\mathrm{1}}}  \Ellesym{,}  \Gamma_{{\mathrm{1}}}  \Ellesym{,}  \Ellent{X}  \Ellesym{,}  \Ellent{Y}  \Ellesym{,}  \Gamma_{{\mathrm{2}}}  \Ellesym{,}  \Delta_{{\mathrm{2}}}  \vdash_\mathcal{L}  \Ellent{B}$ s.t.
      $c(\Pi')\leq|A|$. Therefore, the proof $\Pi$ can be constructed as follows, and
      $c(\Pi)=c(\Pi')\leq|A|$.
      \begin{center}
        \scriptsize
        \begin{math}
          $$\mprset{flushleft}
          \inferrule* [right={\tiny beta}] {
            {
              \begin{array}{cc}
                \Pi' \\
                {\Delta_{{\mathrm{1}}}  \Ellesym{,}  \Gamma_{{\mathrm{1}}}  \Ellesym{,}  \Ellent{X}  \Ellesym{,}  \Ellent{Y}  \Ellesym{,}  \Gamma_{{\mathrm{2}}}  \Ellesym{,}  \Delta_{{\mathrm{2}}}  \vdash_\mathcal{L}  \Ellent{B}}
              \end{array}
            }
          }{\Delta_{{\mathrm{1}}}  \Ellesym{,}  \Gamma_{{\mathrm{1}}}  \Ellesym{,}  \Ellent{Y}  \Ellesym{,}  \Ellent{X}  \Ellesym{,}  \Gamma_{{\mathrm{2}}}  \Ellesym{,}  \Delta_{{\mathrm{2}}}  \vdash_\mathcal{L}  \Ellent{B}}
        \end{math}
      \end{center}

    \item $\ElledruleSXXbetaName / \ElledruleSXXcutTwoName$ Case 2:
      \begin{center}
        \scriptsize
        \begin{math}
          \begin{array}{c}
            \Pi_1 \\
            {\Phi  \vdash_\mathcal{C}  \Ellent{X}}
          \end{array}
        \end{math}
        \qquad\qquad
        $\Pi_1$:
        \begin{math}
          $$\mprset{flushleft}
          \inferrule* [right={\tiny beta}] {
            {
              \begin{array}{c}
                \pi \\
                {\Gamma_{{\mathrm{1}}}  \Ellesym{,}  \Ellent{X}  \Ellesym{,}  \Ellent{Y}  \Ellesym{,}  \Gamma_{{\mathrm{2}}}  \vdash_\mathcal{L}  \Ellent{A}}
              \end{array}
            }
          }{\Gamma_{{\mathrm{1}}}  \Ellesym{,}  \Ellent{Y}  \Ellesym{,}  \Ellent{X}  \Ellesym{,}  \Gamma_{{\mathrm{2}}}  \vdash_\mathcal{L}  \Ellent{A}}
        \end{math}
      \end{center}
      By assumption, $c(\Pi_1),c(\Pi_2)\leq |X|$. By induction on $\Pi_1$ and $\pi$, there is
      a proof $\Pi'$ for sequent $\Gamma_{{\mathrm{1}}}  \Ellesym{,}  \Phi  \Ellesym{,}  \Ellent{Y}  \Ellesym{,}  \Gamma_{{\mathrm{2}}}  \vdash_\mathcal{L}  \Ellent{A}$ s.t. $c(\Pi')\leq|X|$. Therefore,
      the proof $\Pi$ can be constructed as follows, and $c(\Pi)=c(\Pi')\leq|X|$.
      \begin{center}
        \scriptsize
        \begin{math}
          $$\mprset{flushleft}
          \inferrule* [right={\tiny beta}] {
            {
              \begin{array}{cc}
                \Pi' \\
                {\Gamma_{{\mathrm{1}}}  \Ellesym{,}  \Phi  \Ellesym{,}  \Ellent{Y}  \Ellesym{,}  \Gamma_{{\mathrm{2}}}  \vdash_\mathcal{L}  \Ellent{A}}
              \end{array}
            }
          }{\Gamma_{{\mathrm{1}}}  \Ellesym{,}  \Ellent{Y}  \Ellesym{,}  \Phi  \Ellesym{,}  \Gamma_{{\mathrm{2}}}  \vdash_\mathcal{L}  \Ellent{A}}
        \end{math}
      \end{center}

    \end{itemize}

  \item Right rule vs. left rule:
    \begin{itemize}

    \item $ \mathsf{UnitT} $:
      \begin{center}
        \scriptsize
        $\Pi_1:$
        \begin{math}
          $$\mprset{flushleft}
          \inferrule* [right={\tiny unitR}] {
            \,
          }{ \cdot   \vdash_\mathcal{C}   \mathsf{UnitT} }
        \end{math}
        \qquad\qquad
        $\Pi_2:$
        \begin{math}
          $$\mprset{flushleft}
          \inferrule* [right={\tiny unitL}] {
            {
              \begin{array}{c}
                \pi \\
                {\Phi  \vdash_\mathcal{C}  \Ellent{X}}
              \end{array}
            }
          }{ \mathsf{UnitT}   \Ellesym{,}  \Phi  \vdash_\mathcal{C}  \Ellent{X}}
        \end{math}
      \end{center}
      By assumption, $c(\Pi_1),c(\Pi_2)\leq | \mathsf{UnitT} |$. The proof $\Pi$ is the subproof $\pi$
      in $\Pi_2$ for sequent $\Phi  \vdash_\mathcal{C}  \Ellent{X}$. So $c(\Pi)=c(\Pi_2)\leq | \mathsf{UnitT} |$.

      \begin{center}
        \scriptsize
        $\Pi_1:$
        \begin{math}
          $$\mprset{flushleft}
          \inferrule* [right={\tiny unitR}] {
            \,
          }{ \cdot   \vdash_\mathcal{C}   \mathsf{UnitT} }
        \end{math}
        \qquad\qquad
        $\Pi_2:$
        \begin{math}
          $$\mprset{flushleft}
          \inferrule* [right={\tiny unitL1}] {
            {
              \begin{array}{c}
                \pi \\
                {\Gamma  \vdash_\mathcal{L}  \Ellent{A}}
              \end{array}
            }
          }{ \mathsf{UnitT}   \Ellesym{,}  \Gamma  \vdash_\mathcal{L}  \Ellent{A}}
        \end{math}
      \end{center}
      Similar as above, $\Pi$ is $\pi$.

    \item $\otimes$:
      \begin{center}
        \scriptsize
        $\Pi_1:$
        \begin{math}
          $$\mprset{flushleft}
          \inferrule* [right={\tiny tenR}] {
            {
              \begin{array}{cc}
                \pi_1 & \pi_2 \\
                {\Phi_{{\mathrm{1}}}  \vdash_\mathcal{C}  \Ellent{X}} & {\Phi_{{\mathrm{2}}}  \vdash_\mathcal{C}  \Ellent{Y}}
              \end{array}
            }
          }{\Phi_{{\mathrm{1}}}  \Ellesym{,}  \Phi_{{\mathrm{2}}}  \vdash_\mathcal{C}  \Ellent{X}  \otimes  \Ellent{Y}}
        \end{math}
        \qquad\qquad
        $\Pi_2:$
        \begin{math}
          $$\mprset{flushleft}
          \inferrule* [right={\tiny tenL}] {
            {
              \begin{array}{c}
                \pi_3 \\
                {\Psi_{{\mathrm{1}}}  \Ellesym{,}  \Ellent{X}  \Ellesym{,}  \Ellent{Y}  \Ellesym{,}  \Psi_{{\mathrm{2}}}  \vdash_\mathcal{C}  \Ellent{Z}}
              \end{array}
            }
          }{\Psi_{{\mathrm{1}}}  \Ellesym{,}  \Ellent{X}  \otimes  \Ellent{Y}  \Ellesym{,}  \Psi_{{\mathrm{2}}}  \vdash_\mathcal{C}  \Ellent{Z}}
        \end{math}
      \end{center}
      By assumption, $c(\Pi_1),c(\Pi_2)\leq |\Ellent{X}  \otimes  \Ellent{Y}| = |X|+|Y|+1$. The proof $\Pi$ can be
      constructed as follows, and
      $c(\Pi)\leq max\{c(\pi_1),c(\pi_2),c(\pi_3),|X|+1,|Y|+1\}\leq |X|+|Y|+1 = |\Ellent{X}  \otimes  \Ellent{Y}|$.
      \begin{center}
        \scriptsize
        \begin{math}
          $$\mprset{flushleft}
          \inferrule* [right={\tiny cut}] {
            {
              \begin{array}{c}
                \pi_1 \\
                {\Phi_{{\mathrm{1}}}  \vdash_\mathcal{C}  \Ellent{X}}
              \end{array}
            }
            $$\mprset{flushleft}
            \inferrule* [right={\tiny cut}] {
            {
              \begin{array}{cc}
                \pi_2 & \pi_3 \\
                {\Phi_{{\mathrm{2}}}  \vdash_\mathcal{C}  \Ellent{Y}} & {\Psi_{{\mathrm{1}}}  \Ellesym{,}  \Ellent{X}  \Ellesym{,}  \Ellent{Y}  \Ellesym{,}  \Psi_{{\mathrm{2}}}  \vdash_\mathcal{C}  \Ellent{Z}}
              \end{array}
            }
            }{\Psi_{{\mathrm{1}}}  \Ellesym{,}  \Ellent{X}  \Ellesym{,}  \Phi_{{\mathrm{2}}}  \Ellesym{,}  \Psi_{{\mathrm{2}}}  \vdash_\mathcal{C}  \Ellent{Z}}
          }{\Psi_{{\mathrm{1}}}  \Ellesym{,}  \Phi_{{\mathrm{1}}}  \Ellesym{,}  \Phi_{{\mathrm{2}}}  \Ellesym{,}  \Psi_{{\mathrm{2}}}  \vdash_\mathcal{C}  \Ellent{Z}}
        \end{math}
      \end{center}

    \item $\multimap$:
      \begin{center}
        \scriptsize
        $\Pi_1:$
        \begin{math}
          $$\mprset{flushleft}
          \inferrule* [right={\tiny tenR}] {
            {
              \begin{array}{c}
                \pi_1 \\
                {\Phi_{{\mathrm{1}}}  \Ellesym{,}  \Ellent{X}  \vdash_\mathcal{C}  \Ellent{Y}}
              \end{array}
            }
          }{\Phi_{{\mathrm{1}}}  \vdash_\mathcal{C}  \Ellent{X}  \multimap  \Ellent{Y}}
        \end{math}
        \qquad\qquad
        $\Pi_2:$
        \begin{math}
          $$\mprset{flushleft}
          \inferrule* [right={\tiny tenL}] {
            {
              \begin{array}{cc}
                \pi_2 & \pi_3 \\
                {\Phi_{{\mathrm{2}}}  \vdash_\mathcal{C}  \Ellent{X}} & {\Psi_{{\mathrm{1}}}  \Ellesym{,}  \Ellent{Y}  \Ellesym{,}  \Psi_{{\mathrm{2}}}  \vdash_\mathcal{C}  \Ellent{Z}}
              \end{array}
            }
          }{\Psi_{{\mathrm{1}}}  \Ellesym{,}  \Ellent{X}  \multimap  \Ellent{Y}  \Ellesym{,}  \Phi  \Ellesym{,}  \Psi_{{\mathrm{2}}}  \vdash_\mathcal{C}  \Ellent{Z}}
        \end{math}
      \end{center}
      By assumption, $c(\Pi_1),c(\Pi_2)\leq |\Ellent{X}  \multimap  \Ellent{Y}| = |X|+|Y|+1$. The proof $\Pi$ is
      constructed as follows
      $c(\Pi)\leq max\{c(\pi_1),c(\pi_2),c(\pi_3),|X|+1,|Y|+1\}\leq |X|+|Y|+1 = |\Ellent{X}  \multimap  \Ellent{Y}|$.
      \begin{center}
        \scriptsize
        \begin{math}
          $$\mprset{flushleft}
          \inferrule* [right={\tiny tenR}] {
            $$\mprset{flushleft}
            \inferrule* [right={\tiny tenR}] {
              {
                \begin{array}{cc}
                  \pi_1 & \pi_2 \\
                  {\Phi_{{\mathrm{1}}}  \Ellesym{,}  \Ellent{X}  \vdash_\mathcal{C}  \Ellent{Y}} & {\Phi_{{\mathrm{2}}}  \vdash_\mathcal{C}  \Ellent{X}}
                \end{array}
              }
            }{\Phi_{{\mathrm{1}}}  \Ellesym{,}  \Phi_{{\mathrm{2}}}  \vdash_\mathcal{C}  \Ellent{Y}} \\
             {
               \begin{array}{c}
                 \pi_3 \\
                 {\Psi_{{\mathrm{1}}}  \Ellesym{,}  \Ellent{Y}  \Ellesym{,}  \Psi_{{\mathrm{2}}}  \vdash_\mathcal{C}  \Ellent{Z}}
               \end{array}
             }
          }{\Psi_{{\mathrm{1}}}  \Ellesym{,}  \Phi_{{\mathrm{1}}}  \Ellesym{,}  \Phi_{{\mathrm{2}}}  \Ellesym{,}  \Psi_{{\mathrm{2}}}  \vdash_\mathcal{C}  \Ellent{Z}}
        \end{math}
      \end{center}
    \item $ \mathsf{UnitS} $:
      \begin{center}
        \scriptsize
        $\Pi_1:$
        \begin{math}
          $$\mprset{flushleft}
          \inferrule* [right={\tiny unitR}] {
            \,
          }{ \cdot   \vdash_\mathcal{L}   \mathsf{UnitS} }
        \end{math}
        \qquad\qquad
        $\Pi_2:$
        \begin{math}
          $$\mprset{flushleft}
          \inferrule* [right={\tiny unitL2}] {
            {
              \begin{array}{c}
                \pi \\
                {\Delta  \vdash_\mathcal{L}  \Ellent{A}}
              \end{array}
            }
          }{ \mathsf{UnitS}   \Ellesym{,}  \Delta  \vdash_\mathcal{L}  \Ellent{A}}
        \end{math}
      \end{center}
      By assumption, $c(\Pi_1),c(\Pi_2)\leq | \mathsf{UnitS} |$. The proof $\Pi$ is the subproof $\pi$
      in $\Pi_2$ for sequent $\Delta  \vdash_\mathcal{L}  \Ellent{A}$. So $c(\Pi)=c(\Pi_2)\leq | \mathsf{UnitS} |$.

    \item $\tri$:
      \begin{center}
        \scriptsize
        $\Pi_1:$
        \begin{math}
          $$\mprset{flushleft}
          \inferrule* [right={\tiny tenR}] {
            {
              \begin{array}{cc}
                \pi_1 & \pi_2 \\
                {\Gamma_{{\mathrm{1}}}  \vdash_\mathcal{L}  \Ellent{A}} & {\Gamma_{{\mathrm{2}}}  \vdash_\mathcal{L}  \Ellent{B}}
              \end{array}
            }
          }{\Gamma_{{\mathrm{1}}}  \Ellesym{,}  \Gamma_{{\mathrm{2}}}  \vdash_\mathcal{L}  \Ellent{A}  \triangleright  \Ellent{B}}
        \end{math}
        \qquad\qquad
        $\Pi_2:$
        \begin{math}
          $$\mprset{flushleft}
          \inferrule* [right={\tiny tenL1}] {
            {
              \begin{array}{c}
                \pi_3 \\
                {\Delta_{{\mathrm{1}}}  \Ellesym{,}  \Ellent{A}  \Ellesym{,}  \Ellent{B}  \Ellesym{,}  \Delta_{{\mathrm{2}}}  \vdash_\mathcal{L}  \Ellent{C}}
              \end{array}
            }
          }{\Delta_{{\mathrm{1}}}  \Ellesym{,}  \Ellent{A}  \triangleright  \Ellent{B}  \Ellesym{,}  \Delta_{{\mathrm{2}}}  \vdash_\mathcal{L}  \Ellent{C}}
        \end{math}
      \end{center}
      By assumption, $c(\Pi_1),c(\Pi_2)\leq |\Ellent{A}  \triangleright  \Ellent{B}| = |X|+|Y|+1$. The proof $\Pi$ can be
      constructed as follows, and
      $c(\Pi)\leq max\{c(\pi_1),c(\pi_2),c(\pi_3),|A|+1,|B|+1\}\leq |A|+|B|+1 = |\Ellent{A}  \triangleright  \Ellent{B}|$.
      \begin{center}
        \scriptsize
        \begin{math}
          $$\mprset{flushleft}
          \inferrule* [right={\tiny cut2}] {
            {
              \begin{array}{c}
                \pi_1 \\
                {\Gamma_{{\mathrm{1}}}  \vdash_\mathcal{L}  \Ellent{A}}
              \end{array}
            }
            $$\mprset{flushleft}
            \inferrule* [right={\tiny cut2}] {
            {
              \begin{array}{cc}
                \pi_2 & \pi_3 \\
                {\Gamma_{{\mathrm{2}}}  \vdash_\mathcal{L}  \Ellent{B}} & {\Delta_{{\mathrm{1}}}  \Ellesym{,}  \Ellent{A}  \Ellesym{,}  \Ellent{B}  \Ellesym{,}  \Delta_{{\mathrm{2}}}  \vdash_\mathcal{L}  \Ellent{C}}
              \end{array}
            }
            }{\Delta_{{\mathrm{1}}}  \Ellesym{,}  \Ellent{A}  \Ellesym{,}  \Gamma_{{\mathrm{2}}}  \Ellesym{,}  \Delta_{{\mathrm{2}}}  \vdash_\mathcal{L}  \Ellent{C}}
          }{\Delta_{{\mathrm{1}}}  \Ellesym{,}  \Gamma_{{\mathrm{1}}}  \Ellesym{,}  \Gamma_{{\mathrm{2}}}  \Ellesym{,}  \Psi_{{\mathrm{2}}}  \vdash_\mathcal{L}  \Ellent{C}}
        \end{math}
      \end{center}
    \item $\lto$:
      \begin{center}
        \scriptsize
        $\Pi_1:$
        \begin{math}
          $$\mprset{flushleft}
          \inferrule* [right={\tiny imprR}] {
            {
              \begin{array}{c}
                \pi_1 \\
                {\Gamma  \Ellesym{,}  \Ellent{A}  \vdash_\mathcal{L}  \Ellent{B}}
              \end{array}
            }
          }{\Gamma  \vdash_\mathcal{L}  \Ellent{A}  \rightharpoonup  \Ellent{B}}
        \end{math}
        \qquad\qquad
        $\Pi_2:$
        \begin{math}
          $$\mprset{flushleft}
          \inferrule* [right={\tiny imprL}] {
            {
              \begin{array}{cc}
                \pi_2 & \pi_3 \\
                {\Delta_{{\mathrm{1}}}  \vdash_\mathcal{L}  \Ellent{A}} & {\Delta_{{\mathrm{2}}}  \Ellesym{,}  \Ellent{B}  \vdash_\mathcal{L}  \Ellent{C}}
              \end{array}
            }
          }{\Delta_{{\mathrm{2}}}  \Ellesym{,}  \Ellent{A}  \rightharpoonup  \Ellent{B}  \Ellesym{,}  \Delta_{{\mathrm{1}}}  \vdash_\mathcal{L}  \Ellent{C}}
        \end{math}
      \end{center}
      By assumption, $c(\Pi_1),c(\Pi_2)\leq |\Ellent{A}  \rightharpoonup  \Ellent{B}| = |A|+|B|+1$. The proof $\Pi$ is
      constructed as follows, and
      $c(\Pi)\leq max\{c(\pi_1),c(\pi_2),c(\pi_3),|A|+1,|B|+1\}\leq |A|+|B|+1 = |\Ellent{A}  \rightharpoonup  \Ellent{B}|$.
      \begin{center}
        \scriptsize
        \begin{math}
          $$\mprset{flushleft}
          \inferrule* [right={\tiny cut2}] {
            $$\mprset{flushleft}
            \inferrule* [right={\tiny cut2}] {
              {
                \begin{array}{cc}
                  \pi_1 & \pi_2 \\
                  {\Gamma  \Ellesym{,}  \Ellent{A}  \vdash_\mathcal{L}  \Ellent{B}} & {\Delta_{{\mathrm{1}}}  \vdash_\mathcal{L}  \Ellent{A}}
                \end{array}
              }
            }{\Gamma  \Ellesym{,}  \Delta_{{\mathrm{1}}}  \vdash_\mathcal{L}  \Ellent{B}}
             {
               \begin{array}{c}
                 \pi_3 \\
                 {\Delta_{{\mathrm{2}}}  \Ellesym{,}  \Ellent{B}  \vdash_\mathcal{L}  \Ellent{C}}
               \end{array}
             }
          }{\Delta_{{\mathrm{2}}}  \Ellesym{,}  \Gamma  \Ellesym{,}  \Delta_{{\mathrm{1}}}  \vdash_\mathcal{L}  \Ellent{C}}
        \end{math}
      \end{center}

    \item $\rto$:
      \begin{center}
        \scriptsize
        $\Pi_1:$
        \begin{math}
          $$\mprset{flushleft}
          \inferrule* [right={\tiny implR}] {
            {
              \begin{array}{c}
                \pi_1 \\
                {\Ellent{A}  \Ellesym{,}  \Gamma  \vdash_\mathcal{L}  \Ellent{B}}
              \end{array}
            }
          }{\Gamma  \vdash_\mathcal{L}  \Ellent{B}  \leftharpoonup  \Ellent{A}}
        \end{math}
        \qquad\qquad
        $\Pi_2:$
        \begin{math}
          $$\mprset{flushleft}
          \inferrule* [right={\tiny implL}] {
            {
              \begin{array}{cc}
                \pi_2 & \pi_3 \\
                {\Delta_{{\mathrm{1}}}  \vdash_\mathcal{L}  \Ellent{A}} & {\Ellent{B}  \Ellesym{,}  \Delta_{{\mathrm{2}}}  \vdash_\mathcal{L}  \Ellent{C}}
              \end{array}
            }
          }{\Delta_{{\mathrm{1}}}  \Ellesym{,}  \Ellent{B}  \leftharpoonup  \Ellent{A}  \Ellesym{,}  \Delta_{{\mathrm{2}}}  \vdash_\mathcal{L}  \Ellent{C}}
        \end{math}
      \end{center}
      By assumption, $c(\Pi_1),c(\Pi_2)\leq |\Ellent{B}  \leftharpoonup  \Ellent{A}| = |A|+|B|+1$. The proof $\Pi$ is
      constructed as follows, and
      $c(\Pi)\leq max\{c(\pi_1),c(\pi_2),c(\pi_3),|A|+1,|B|+1\}\leq |A|+|B|+1 = |\Ellent{B}  \leftharpoonup  \Ellent{A}|$.
      \begin{center}
        \scriptsize
        \begin{math}
          $$\mprset{flushleft}
          \inferrule* [right={\tiny cut1}] {
            $$\mprset{flushleft}
            \inferrule* [right={\tiny cut2}] {
              {
                \begin{array}{cc}
                  \pi_1 & \pi_2 \\
                  {\Ellent{A}  \Ellesym{,}  \Gamma  \vdash_\mathcal{L}  \Ellent{B}} & {\Delta_{{\mathrm{1}}}  \vdash_\mathcal{L}  \Ellent{A}}
                \end{array}
              }
            }{\Delta_{{\mathrm{1}}}  \Ellesym{,}  \Gamma  \vdash_\mathcal{L}  \Ellent{B}}
             {
               \begin{array}{c}
                 \pi_3 \\
                 {\Ellent{B}  \Ellesym{,}  \Delta_{{\mathrm{2}}}  \vdash_\mathcal{L}  \Ellent{C}}
               \end{array}
             }
          }{\Delta_{{\mathrm{1}}}  \Ellesym{,}  \Gamma  \Ellesym{,}  \Delta_{{\mathrm{2}}}  \vdash_\mathcal{L}  \Ellent{C}}
        \end{math}
      \end{center}

    \item $F$:
      \begin{center}
        \scriptsize
        $\Pi_1:$
        \begin{math}
          $$\mprset{flushleft}
          \inferrule* [right={\tiny FR}] {
            {
              \begin{array}{c}
                \pi_1 \\
                {\Phi  \vdash_\mathcal{C}  \Ellent{X}}
              \end{array}
            }
          }{\Phi  \vdash_\mathcal{L}   \mathsf{F} \Ellent{X} }
        \end{math}
        \qquad\qquad
        $\Pi_2:$
        \begin{math}
          $$\mprset{flushleft}
          \inferrule* [right={\tiny FL}] {
            {
              \begin{array}{c}
                \pi_2 \\
                {\Gamma  \Ellesym{,}  \Ellent{X}  \Ellesym{,}  \Delta  \vdash_\mathcal{L}  \Ellent{A}}
              \end{array}
            }
          }{\Gamma  \Ellesym{,}   \mathsf{F} \Ellent{X}   \Ellesym{,}  \Delta  \vdash_\mathcal{L}  \Ellent{A}}
        \end{math}
      \end{center}
      By assumption, $c(\Pi_1),c(\Pi_2)\leq | \mathsf{F} \Ellent{X} | = |X|+1$. The proof $\Pi$ is
      constructed as follows, and $c(\Pi)\leq max\{c(\pi_1),c(\pi_2),|X|+1\}\leq | \mathsf{F} \Ellent{X} |$.
      \begin{center}
        \scriptsize
        \begin{math}
          $$\mprset{flushleft}
          \inferrule* [right={\tiny cut2}] {
            {
              \begin{array}{cc}
                \pi_1 & \pi_2 \\
                {\Phi  \vdash_\mathcal{C}  \Ellent{X}} & {\Gamma  \Ellesym{,}  \Ellent{X}  \Ellesym{,}  \Delta  \vdash_\mathcal{L}  \Ellent{A}}
              \end{array}
            }
          }{\Gamma  \Ellesym{,}  \Phi  \Ellesym{,}  \Delta  \vdash_\mathcal{L}  \Ellent{A}}
        \end{math}
      \end{center}

    \item $G$:
      \begin{center}
        \scriptsize
        $\Pi_1:$
        \begin{math}
          $$\mprset{flushleft}
          \inferrule* [right={\tiny GR}] {
            {
              \begin{array}{c}
                \pi_1 \\
                {\Phi  \vdash_\mathcal{L}  \Ellent{A}}
              \end{array}
            }
          }{\Phi  \vdash_\mathcal{C}   \mathsf{G} \Ellent{A} }
        \end{math}
        \qquad\qquad
        $\Pi_2:$
        \begin{math}
          $$\mprset{flushleft}
          \inferrule* [right={\tiny GL}] {
            {
              \begin{array}{c}
                \pi_2 \\
                {\Gamma  \Ellesym{,}  \Ellent{A}  \Ellesym{,}  \Delta  \vdash_\mathcal{L}  \Ellent{B}}
              \end{array}
            }
          }{\Gamma  \Ellesym{,}   \mathsf{G} \Ellent{A}   \Ellesym{,}  \Delta  \vdash_\mathcal{L}  \Ellent{B}}
        \end{math}
      \end{center}
      By assumption, $c(\Pi_1),c(\Pi_2)\leq | \mathsf{G} \Ellent{A} | = |A|+1$. The proof $\Pi$ is
      constructed as follows, and $c(\Pi)\leq max\{c(\pi_1),c(\pi_2),|A|+1\}\leq | \mathsf{G} \Ellent{A} |$.
      \begin{center}
        \scriptsize
        \begin{math}
          $$\mprset{flushleft}
          \inferrule* [right={\tiny GL}] {
            {
              \begin{array}{cc}
                \pi_1 & \pi_2 \\
                {\Phi  \vdash_\mathcal{L}  \Ellent{A}} & {\Gamma  \Ellesym{,}  \Ellent{A}  \Ellesym{,}  \Delta  \vdash_\mathcal{L}  \Ellent{B}}
              \end{array}
            }
          }{\Gamma  \Ellesym{,}  \Phi  \Ellesym{,}  \Delta  \vdash_\mathcal{L}  \Ellent{B}}
        \end{math}
      \end{center}

    \end{itemize}

  \item $\Pi_1$ ends with a left rule (with low priority):
    \begin{itemize}

    \item \ElledruleTXXunitLName / $\cat{C}$-sequent:
      \begin{center}
        \scriptsize
        $\Pi_1$:
        \begin{math}
          $$\mprset{flushleft}
          \inferrule* [right={\tiny unitL}] {
            {
              \begin{array}{c}
                \pi \\
                {\Phi  \vdash_\mathcal{C}  \Ellent{X}}
              \end{array}
            }
          }{ \mathsf{UnitT}   \Ellesym{,}  \Phi  \vdash_\mathcal{C}  \Ellent{X}}
        \end{math}
        \qquad\qquad
        \begin{math}
          \begin{array}{c}
            \Pi_2 \\
            {\Psi_{{\mathrm{1}}}  \Ellesym{,}  \Ellent{X}  \Ellesym{,}  \Psi_{{\mathrm{2}}}  \vdash_\mathcal{C}  \Ellent{Y}}
          \end{array}
        \end{math}
      \end{center}
      By assumption, $c(\Pi_1),c(\Pi_2)\leq |X|$. By induction, there is a proof $\Pi'$ from
      $\pi$ and $\Pi_2$ for sequent $\Psi_{{\mathrm{1}}}  \Ellesym{,}  \Phi  \Ellesym{,}  \Psi_{{\mathrm{2}}}  \vdash_\mathcal{C}  \Ellent{Y}$ s.t. $c(\Pi')\leq |X|$. Therefore,
      the proof $\Pi$ can be constructed as follows, and $c(\Pi)=c(\Pi')\leq |X|$.
      \begin{center}
        \scriptsize
        \begin{math}
          $$\mprset{flushleft}
          \inferrule* [right={\tiny beta}] {
            $$\mprset{flushleft}
            \inferrule* [right={\tiny unitL}] {
              {
                \begin{array}{c}
                  \Pi' \\
                  {\Psi_{{\mathrm{1}}}  \Ellesym{,}  \Phi  \Ellesym{,}  \Psi_{{\mathrm{2}}}  \vdash_\mathcal{C}  \Ellent{Y}}
                \end{array}
              }
            }{ \mathsf{UnitT}   \Ellesym{,}  \Psi_{{\mathrm{1}}}  \Ellesym{,}  \Phi  \Ellesym{,}  \Psi_{{\mathrm{2}}}  \vdash_\mathcal{C}  \Ellent{Y}}
          }{\Psi_{{\mathrm{1}}}  \Ellesym{,}   \mathsf{UnitT}   \Ellesym{,}  \Phi  \Ellesym{,}  \Psi_{{\mathrm{2}}}  \vdash_\mathcal{C}  \Ellent{Y}}
        \end{math}
      \end{center}

    \item \ElledruleTXXunitLName / $\cat{L}$-sequent: (DOES NOT WORK)
      \begin{center}
        \scriptsize
        $\Pi_1$:
        \begin{math}
          $$\mprset{flushleft}
          \inferrule* [right={\tiny unitL}] {
            {
              \begin{array}{c}
                \pi \\
                {\Phi  \vdash_\mathcal{C}  \Ellent{X}}
              \end{array}
            }
          }{ \mathsf{UnitT}   \Ellesym{,}  \Ellent{X}  \vdash_\mathcal{C}  \Ellent{X}}
        \end{math}
        \qquad\qquad
        \begin{math}
          \begin{array}{c}
            \Pi_2 \\
            {\Gamma_{{\mathrm{1}}}  \Ellesym{,}  \Ellent{X}  \Ellesym{,}  \Gamma_{{\mathrm{2}}}  \vdash_\mathcal{L}  \Ellent{A}}
          \end{array}
        \end{math}
      \end{center}
      By assumption, $c(\Pi_1),c(\Pi_2)\leq |X|$. Similar as above, the proof $\Pi$ can be
      constructed as follows with $c(\Pi)\leq |X|$.
      \begin{center}
        \scriptsize
        \begin{math}
          $$\mprset{flushleft}
          \inferrule* [right={\tiny unitL1}] {
            {
              \begin{array}{c}
                \Pi' \\
                {\Gamma_{{\mathrm{1}}}  \Ellesym{,}  \Phi  \Ellesym{,}  \Gamma_{{\mathrm{2}}}  \vdash_\mathcal{L}  \Ellent{A}}
              \end{array}
            }
          }{ \mathsf{UnitT}   \Ellesym{,}  \Gamma_{{\mathrm{1}}}  \Ellesym{,}  \Phi  \Ellesym{,}  \Gamma_{{\mathrm{2}}}  \vdash_\mathcal{L}  \Ellent{A}}
        \end{math}
      \end{center}

    \item \ElledruleSXXunitLOneName / $\cat{L}$-sequent: (DOES NOT WORK)
      \begin{center}
        \scriptsize
        $\Pi_1$:
        \begin{math}
          $$\mprset{flushleft}
          \inferrule* [right={\tiny unitL1}] {
            {
              \begin{array}{c}
                \pi \\
                {\Delta  \vdash_\mathcal{L}  \Ellent{A}}
              \end{array}
            }
          }{ \mathsf{UnitT}   \Ellesym{,}  \Delta  \vdash_\mathcal{L}  \Ellent{A}}
        \end{math}
        \qquad\qquad
        \begin{math}
          \begin{array}{c}
            \Pi_2 \\
            {\Gamma_{{\mathrm{1}}}  \Ellesym{,}  \Ellent{A}  \Ellesym{,}  \Gamma_{{\mathrm{2}}}  \vdash_\mathcal{L}  \Ellent{B}}
          \end{array}
        \end{math}
      \end{center}
      By assumption, $c(\Pi_1),c(\Pi_2)\leq |A|$. By induction, there is a proof $\Pi'$ from
      $\pi$ and $\Pi_2$ for sequent $\Gamma_{{\mathrm{1}}}  \Ellesym{,}  \Delta  \Ellesym{,}  \Gamma_{{\mathrm{2}}}  \vdash_\mathcal{L}  \Ellent{B}$ s.t. $c(\Pi')\leq |A|$. Therefore,
      the proof $\Pi$ can be constructed as follows, and $c(\Pi)=c(\Pi')\leq |A|$.
      \begin{center}
        \scriptsize
        \begin{math}
          $$\mprset{flushleft}
          \inferrule* [right={\tiny unitL1}] {
            {
              \begin{array}{c}
                \Pi' \\
                {\Gamma_{{\mathrm{1}}}  \Ellesym{,}  \Delta  \Ellesym{,}  \Gamma_{{\mathrm{2}}}  \vdash_\mathcal{L}  \Ellent{B}}
              \end{array}
            }
          }{ \mathsf{UnitT}   \Ellesym{,}  \Gamma_{{\mathrm{1}}}  \Ellesym{,}  \Delta  \Ellesym{,}  \Gamma_{{\mathrm{2}}}  \vdash_\mathcal{L}  \Ellent{B}}
        \end{math}
      \end{center}

    \item \ElledruleSXXunitLTwoName / $\cat{L}$-sequent (DOES NOT WORK)
      \begin{center}
        \scriptsize
        $\Pi_1$:
        \begin{math}
          $$\mprset{flushleft}
          \inferrule* [right={\tiny unitL2}] {
            {
              \begin{array}{c}
                \pi \\
                {\Delta  \vdash_\mathcal{L}  \Ellent{A}}
              \end{array}
            }
          }{ \mathsf{UnitS}   \Ellesym{,}  \Delta  \vdash_\mathcal{L}  \Ellent{A}}
        \end{math}
        \qquad\qquad
        \begin{math}
          \begin{array}{c}
            \Pi_2 \\
            {\Gamma_{{\mathrm{1}}}  \Ellesym{,}  \Ellent{A}  \Ellesym{,}  \Gamma_{{\mathrm{2}}}  \vdash_\mathcal{L}  \Ellent{B}}
          \end{array}
        \end{math}
      \end{center}
      By assumption, $c(\Pi_1),c(\Pi_2)\leq |A|$. By induction, there is a proof $\Pi'$ from
      $\pi$ and $\Pi_2$ for sequent $\Gamma_{{\mathrm{1}}}  \Ellesym{,}  \Delta  \Ellesym{,}  \Gamma_{{\mathrm{2}}}  \vdash_\mathcal{L}  \Ellent{B}$ s.t. $c(\Pi')\leq |A|$. Therefore,
      the proof $\Pi$ can be constructed as follows, and $c(\Pi)=c(\Pi')\leq |A|$.
      \begin{center}
        \scriptsize
        \begin{math}
          $$\mprset{flushleft}
          \inferrule* [right={\tiny unitL2}] {
            $$\mprset{flushleft}
            \inferrule* [right={\tiny cut2}] {
              {
                \begin{array}{cc}
                  \pi & \Pi_2 \\
                  {\Delta  \vdash_\mathcal{L}  \Ellent{A}} & {\Gamma_{{\mathrm{1}}}  \Ellesym{,}  \Ellent{A}  \Ellesym{,}  \Gamma_{{\mathrm{2}}}  \vdash_\mathcal{L}  \Ellent{B}}
                \end{array}
              }
            }{\Gamma_{{\mathrm{1}}}  \Ellesym{,}  \Delta  \Ellesym{,}  \Gamma_{{\mathrm{2}}}  \vdash_\mathcal{L}  \Ellent{B}}
          }{ \mathsf{UnitS}   \Ellesym{,}  \Gamma_{{\mathrm{1}}}  \Ellesym{,}  \Delta  \Ellesym{,}  \Gamma_{{\mathrm{2}}}  \vdash_\mathcal{L}  \Ellent{B}}
        \end{math}
      \end{center}

    \item \ElledruleTXXtenLName / $\cat{C}$-sequent:
      \begin{center}
        \scriptsize
        $\Pi_1$:
        \begin{math}
          $$\mprset{flushleft}
          \inferrule* [right={\tiny tenL}] {
            {
              \begin{array}{c}
                \pi \\
                {\Phi_{{\mathrm{1}}}  \Ellesym{,}  \Ellent{X_{{\mathrm{1}}}}  \Ellesym{,}  \Ellent{X_{{\mathrm{2}}}}  \Ellesym{,}  \Phi_{{\mathrm{2}}}  \vdash_\mathcal{C}  \Ellent{Y}}
              \end{array}
            }
          }{\Phi_{{\mathrm{1}}}  \Ellesym{,}  \Ellent{X_{{\mathrm{1}}}}  \otimes  \Ellent{X_{{\mathrm{2}}}}  \Ellesym{,}  \Phi_{{\mathrm{2}}}  \vdash_\mathcal{C}  \Ellent{Y}}
        \end{math}
        \qquad\qquad
        \begin{math}
          \begin{array}{c}
            \Pi_2 \\
            {\Psi_{{\mathrm{1}}}  \Ellesym{,}  \Ellent{Y}  \Ellesym{,}  \Psi_{{\mathrm{2}}}  \vdash_\mathcal{C}  \Ellent{Z}}
          \end{array}
        \end{math}
      \end{center}
      By assumption, $c(\Pi_1),c(\Pi_2)\leq |Y|$. By induction, there is a proof $\Pi'$ from
      $\pi$ and $\Pi_2$ for $\Psi_{{\mathrm{1}}}  \Ellesym{,}  \Phi_{{\mathrm{1}}}  \Ellesym{,}  \Ellent{X_{{\mathrm{1}}}}  \Ellesym{,}  \Ellent{X_{{\mathrm{2}}}}  \Ellesym{,}  \Phi_{{\mathrm{2}}}  \Ellesym{,}  \Psi_{{\mathrm{2}}}  \vdash_\mathcal{C}  \Ellent{Z}$ s.t. $c(\Pi')\leq |Y|$.
      Therefore, the proof $\Pi$ can be constructed as follows with $c(\Pi)\leq |Y|$.
      \begin{center}
        \scriptsize
        \begin{math}
          $$\mprset{flushleft}
          \inferrule* [right={\tiny tenL}] {
            $$\mprset{flushleft}
            \inferrule* [right={\tiny cut}] {
              {
                \begin{array}{cc}
                  \pi & \Pi_2 \\
                  {\Phi_{{\mathrm{1}}}  \Ellesym{,}  \Ellent{X_{{\mathrm{1}}}}  \Ellesym{,}  \Ellent{X_{{\mathrm{2}}}}  \Ellesym{,}  \Phi_{{\mathrm{2}}}  \vdash_\mathcal{C}  \Ellent{Y}} & {\Psi_{{\mathrm{1}}}  \Ellesym{,}  \Ellent{Y}  \Ellesym{,}  \Psi_{{\mathrm{2}}}  \vdash_\mathcal{C}  \Ellent{Z}}
                \end{array}
              }
            }{\Psi_{{\mathrm{1}}}  \Ellesym{,}  \Phi_{{\mathrm{1}}}  \Ellesym{,}  \Ellent{X_{{\mathrm{1}}}}  \Ellesym{,}  \Ellent{X_{{\mathrm{2}}}}  \Ellesym{,}  \Phi_{{\mathrm{2}}}  \Ellesym{,}  \Psi_{{\mathrm{2}}}  \vdash_\mathcal{C}  \Ellent{Z}}
          }{\Psi_{{\mathrm{1}}}  \Ellesym{,}  \Phi_{{\mathrm{1}}}  \Ellesym{,}  \Ellent{X_{{\mathrm{1}}}}  \otimes  \Ellent{X_{{\mathrm{2}}}}  \Ellesym{,}  \Phi_{{\mathrm{2}}}  \Ellesym{,}  \Psi_{{\mathrm{2}}}  \vdash_\mathcal{C}  \Ellent{Z}}
        \end{math}
      \end{center}

    \item \ElledruleTXXtenLName / $\cat{L}$-sequent:
      \begin{center}
        \scriptsize
        $\Pi_1$:
        \begin{math}
          $$\mprset{flushleft}
          \inferrule* [right={\tiny tenL}] {
            {
              \begin{array}{c}
                \pi \\
                {\Phi_{{\mathrm{1}}}  \Ellesym{,}  \Ellent{X_{{\mathrm{1}}}}  \Ellesym{,}  \Ellent{X_{{\mathrm{2}}}}  \Ellesym{,}  \Phi_{{\mathrm{2}}}  \vdash_\mathcal{C}  \Ellent{Y}}
              \end{array}
            }
          }{\Phi_{{\mathrm{1}}}  \Ellesym{,}  \Ellent{X_{{\mathrm{1}}}}  \otimes  \Ellent{X_{{\mathrm{2}}}}  \Ellesym{,}  \Phi_{{\mathrm{2}}}  \vdash_\mathcal{C}  \Ellent{Y}}
        \end{math}
        \qquad\qquad
        \begin{math}
          \begin{array}{c}
            \Pi_2 \\
            {\Gamma_{{\mathrm{1}}}  \Ellesym{,}  \Ellent{Y}  \Ellesym{,}  \Gamma_{{\mathrm{2}}}  \vdash_\mathcal{L}  \Ellent{A}}
          \end{array}
        \end{math}
      \end{center}
      By assumption, $c(\Pi_1),c(\Pi_2)\leq |Y|$. By induction, there is a proof $\Pi'$ from
      $\pi$ and $\Pi_2$ for $\Gamma_{{\mathrm{1}}}  \Ellesym{,}  \Phi_{{\mathrm{1}}}  \Ellesym{,}  \Ellent{X_{{\mathrm{1}}}}  \Ellesym{,}  \Ellent{X_{{\mathrm{2}}}}  \Ellesym{,}  \Phi_{{\mathrm{2}}}  \Ellesym{,}  \Gamma_{{\mathrm{2}}}  \vdash_\mathcal{L}  \Ellent{A}$ s.t. $c(\Pi')\leq |Y|$.
      Therefore, the proof $\Pi$ can be constructed as follows with $c(\Pi)\leq |Y|$.
      \begin{center}
        \scriptsize
        \begin{math}
          $$\mprset{flushleft}
          \inferrule* [right={\tiny tenL1}] {
            $$\mprset{flushleft}
            \inferrule* [right={\tiny cut1}] {
              {
                \begin{array}{cc}
                  \pi & \Pi_2 \\
                  {\Phi_{{\mathrm{1}}}  \Ellesym{,}  \Ellent{X_{{\mathrm{1}}}}  \Ellesym{,}  \Ellent{X_{{\mathrm{2}}}}  \Ellesym{,}  \Phi_{{\mathrm{2}}}  \vdash_\mathcal{C}  \Ellent{Y}} & {\Gamma_{{\mathrm{1}}}  \Ellesym{,}  \Ellent{Y}  \Ellesym{,}  \Gamma_{{\mathrm{2}}}  \vdash_\mathcal{L}  \Ellent{A}}
                \end{array}
              }
            }{\Gamma_{{\mathrm{1}}}  \Ellesym{,}  \Phi_{{\mathrm{1}}}  \Ellesym{,}  \Ellent{X_{{\mathrm{1}}}}  \Ellesym{,}  \Ellent{X_{{\mathrm{2}}}}  \Ellesym{,}  \Phi_{{\mathrm{2}}}  \Ellesym{,}  \Gamma_{{\mathrm{2}}}  \vdash_\mathcal{L}  \Ellent{A}}
          }{\Gamma_{{\mathrm{1}}}  \Ellesym{,}  \Phi_{{\mathrm{1}}}  \Ellesym{,}  \Ellent{X_{{\mathrm{1}}}}  \otimes  \Ellent{X_{{\mathrm{2}}}}  \Ellesym{,}  \Phi_{{\mathrm{2}}}  \Ellesym{,}  \Gamma_{{\mathrm{2}}}  \vdash_\mathcal{L}  \Ellent{A}}
        \end{math}
      \end{center}

    \item \ElledruleSXXtenLOneName / $\cat{L}$-sequent:
      \begin{center}
        \scriptsize
        $\Pi_1$:
        \begin{math}
          $$\mprset{flushleft}
          \inferrule* [right={\tiny tenL}] {
            {
              \begin{array}{c}
                \pi \\
                {\Gamma_{{\mathrm{1}}}  \Ellesym{,}  \Ellent{X}  \Ellesym{,}  \Ellent{Y}  \Ellesym{,}  \Gamma_{{\mathrm{2}}}  \vdash_\mathcal{L}  \Ellent{A}}
              \end{array}
            }
          }{\Gamma_{{\mathrm{1}}}  \Ellesym{,}  \Ellent{X}  \otimes  \Ellent{Y}  \Ellesym{,}  \Gamma_{{\mathrm{2}}}  \vdash_\mathcal{L}  \Ellent{A}}
        \end{math}
        \qquad\qquad
        \begin{math}
          \begin{array}{c}
            \Pi_2 \\
            {\Delta_{{\mathrm{1}}}  \Ellesym{,}  \Ellent{A}  \Ellesym{,}  \Delta_{{\mathrm{2}}}  \vdash_\mathcal{L}  \Ellent{B}}
          \end{array}
        \end{math}
      \end{center}
      By assumption, $c(\Pi_1),c(\Pi_2)\leq |A|$. By induction, there is a proof $\Pi'$ from
      $\pi$ and $\Pi_2$ for $\Delta_{{\mathrm{1}}}  \Ellesym{,}  \Ellent{X}  \Ellesym{,}  \Ellent{Y}  \Ellesym{,}  \Gamma_{{\mathrm{2}}}  \Ellesym{,}  \Delta_{{\mathrm{2}}}  \vdash_\mathcal{L}  \Ellent{B}$ s.t. $c(\Pi')\leq |A|$.
      Therefore, the proof $\Pi$ can be constructed as follows with $c(\Pi)\leq |A|$.
      \begin{center}
        \scriptsize
        \begin{math}
          $$\mprset{flushleft}
          \inferrule* [right={\tiny tenL1}] {
            $$\mprset{flushleft}
            \inferrule* [right={\tiny cut2}] {
              {
                \begin{array}{cc}
                  \pi & \Pi_2 \\
                  {\Gamma_{{\mathrm{1}}}  \Ellesym{,}  \Ellent{X}  \Ellesym{,}  \Ellent{Y}  \Ellesym{,}  \Gamma_{{\mathrm{2}}}  \vdash_\mathcal{L}  \Ellent{A}} & {\Delta_{{\mathrm{1}}}  \Ellesym{,}  \Ellent{A}  \Ellesym{,}  \Delta_{{\mathrm{2}}}  \vdash_\mathcal{L}  \Ellent{B}}
                \end{array}
              }
            }{\Delta_{{\mathrm{1}}}  \Ellesym{,}  \Gamma_{{\mathrm{1}}}  \Ellesym{,}  \Ellent{X}  \Ellesym{,}  \Ellent{Y}  \Ellesym{,}  \Gamma_{{\mathrm{2}}}  \Ellesym{,}  \Delta_{{\mathrm{2}}}  \vdash_\mathcal{L}  \Ellent{B}}
          }{\Delta_{{\mathrm{1}}}  \Ellesym{,}  \Gamma_{{\mathrm{1}}}  \Ellesym{,}  \Ellent{X}  \otimes  \Ellent{Y}  \Ellesym{,}  \Gamma_{{\mathrm{2}}}  \Ellesym{,}  \Delta_{{\mathrm{2}}}  \vdash_\mathcal{L}  \Ellent{B}}
        \end{math}
      \end{center}

    \item \ElledruleSXXtenLTwoName / $\cat{L}$-sequent:
      \begin{center}
        \scriptsize
        $\Pi_1$:
        \begin{math}
          $$\mprset{flushleft}
          \inferrule* [right={\tiny tenL2}] {
            {
              \begin{array}{c}
                \pi \\
                {\Gamma_{{\mathrm{1}}}  \Ellesym{,}  \Ellent{A_{{\mathrm{1}}}}  \Ellesym{,}  \Ellent{A_{{\mathrm{2}}}}  \Ellesym{,}  \Gamma_{{\mathrm{2}}}  \vdash_\mathcal{L}  \Ellent{B}}
              \end{array}
            }
          }{\Gamma_{{\mathrm{1}}}  \Ellesym{,}  \Ellent{A_{{\mathrm{1}}}}  \triangleright  \Ellent{A_{{\mathrm{2}}}}  \Ellesym{,}  \Gamma_{{\mathrm{2}}}  \vdash_\mathcal{L}  \Ellent{B}}
        \end{math}
        \qquad\qquad
        \begin{math}
          \begin{array}{c}
            \Pi_2 \\
            {\Delta_{{\mathrm{1}}}  \Ellesym{,}  \Ellent{B}  \Ellesym{,}  \Delta_{{\mathrm{2}}}  \vdash_\mathcal{L}  \Ellent{C}}
          \end{array}
        \end{math}
      \end{center}
      By assumption, $c(\Pi_1),c(\Pi_2)\leq |B|$. By induction, there is a proof $\Pi'$ from
      $\pi$ and $\Pi_2$ for $\Delta_{{\mathrm{1}}}  \Ellesym{,}  \Gamma_{{\mathrm{1}}}  \Ellesym{,}  \Ellent{A_{{\mathrm{1}}}}  \Ellesym{,}  \Ellent{A_{{\mathrm{2}}}}  \Ellesym{,}  \Gamma_{{\mathrm{2}}}  \Ellesym{,}  \Delta_{{\mathrm{2}}}  \vdash_\mathcal{L}  \Ellent{C}$ s.t. $c(\Pi')\leq |B|$.
      Therefore, the proof $\Pi$ can be constructed as follows with $c(\Pi)\leq |B|$.
      \begin{center}
        \scriptsize
        \begin{math}
          $$\mprset{flushleft}
          \inferrule* [right={\tiny tenL2}] {
            $$\mprset{flushleft}
            \inferrule* [right={\tiny cut2}] {
              {
                \begin{array}{cc}
                  \pi & \Pi_2 \\
                  {\Gamma_{{\mathrm{1}}}  \Ellesym{,}  \Ellent{A_{{\mathrm{1}}}}  \Ellesym{,}  \Ellent{A_{{\mathrm{2}}}}  \Ellesym{,}  \Gamma_{{\mathrm{2}}}  \vdash_\mathcal{L}  \Ellent{B}} & {\Delta_{{\mathrm{1}}}  \Ellesym{,}  \Ellent{B}  \Ellesym{,}  \Delta_{{\mathrm{2}}}  \vdash_\mathcal{L}  \Ellent{C}}
                \end{array}
              }
            }{\Delta_{{\mathrm{1}}}  \Ellesym{,}  \Gamma_{{\mathrm{1}}}  \Ellesym{,}  \Ellent{A_{{\mathrm{1}}}}  \Ellesym{,}  \Ellent{A_{{\mathrm{2}}}}  \Ellesym{,}  \Gamma_{{\mathrm{2}}}  \Ellesym{,}  \Delta_{{\mathrm{2}}}  \vdash_\mathcal{L}  \Ellent{C}}
          }{\Delta_{{\mathrm{1}}}  \Ellesym{,}  \Gamma_{{\mathrm{1}}}  \Ellesym{,}  \Ellent{A_{{\mathrm{1}}}}  \triangleright  \Ellent{A_{{\mathrm{2}}}}  \Ellesym{,}  \Gamma_{{\mathrm{2}}}  \Ellesym{,}  \Delta_{{\mathrm{2}}}  \vdash_\mathcal{L}  \Ellent{C}}
        \end{math}
      \end{center}

    \item \ElledruleTXXimpLName / $\cat{C}$-sequent:
      \begin{center}
        \scriptsize
        $\Pi_1$:
        \begin{math}
          $$\mprset{flushleft}
          \inferrule* [right={\tiny impL}] {
            {
              \begin{array}{cc}
                \pi_1 & \pi_2 \\
                {\Phi_{{\mathrm{1}}}  \vdash_\mathcal{C}  \Ellent{X_{{\mathrm{1}}}}} & {\Phi_{{\mathrm{2}}}  \Ellesym{,}  \Ellent{X_{{\mathrm{2}}}}  \Ellesym{,}  \Phi_{{\mathrm{3}}}  \vdash_\mathcal{C}  \Ellent{Y}}
              \end{array}
            }
          }{\Phi_{{\mathrm{2}}}  \Ellesym{,}  \Ellent{X_{{\mathrm{1}}}}  \multimap  \Ellent{X_{{\mathrm{2}}}}  \Ellesym{,}  \Phi_{{\mathrm{1}}}  \Ellesym{,}  \Phi_{{\mathrm{3}}}  \vdash_\mathcal{C}  \Ellent{Y}}
        \end{math}
        \qquad\qquad
        \begin{math}
          \begin{array}{c}
            \Pi_2 \\
            {\Psi_{{\mathrm{1}}}  \Ellesym{,}  \Ellent{Y}  \Ellesym{,}  \Psi_{{\mathrm{2}}}  \vdash_\mathcal{C}  \Ellent{Z}}
          \end{array}
        \end{math}
      \end{center}
      By assumption, $c(\Pi_1),c(\Pi_2)\leq |Y|$. By induction, there is a proof $\Pi'$ from
      $\pi_2$ and $\Pi_2$ for $\Psi_{{\mathrm{1}}}  \Ellesym{,}  \Phi_{{\mathrm{2}}}  \Ellesym{,}  \Ellent{X_{{\mathrm{2}}}}  \Ellesym{,}  \Phi_{{\mathrm{3}}}  \Ellesym{,}  \Psi_{{\mathrm{2}}}  \vdash_\mathcal{C}  \Ellent{Z}$ s.t. $c(\Pi')\leq |Y|$.
      Therefore, the proof $\Pi$ can be constructed as follows with $c(\Pi)\leq |Y|$.
      \begin{center}
        \scriptsize
        \begin{math}
          $$\mprset{flushleft}
          \inferrule* [right={\tiny impL}] {
            {
              \begin{array}{c}
                \pi_1 \\
                {\Phi_{{\mathrm{1}}}  \vdash_\mathcal{C}  \Ellent{X_{{\mathrm{1}}}}}
              \end{array}
            }
            $$\mprset{flushleft}
            \inferrule* [right={\tiny cut}] {
              {
                \begin{array}{cc}
                  \pi_2 & \Pi_2 \\
                  {\Phi_{{\mathrm{2}}}  \Ellesym{,}  \Ellent{X_{{\mathrm{2}}}}  \Ellesym{,}  \Phi_{{\mathrm{3}}}  \vdash_\mathcal{C}  \Ellent{Y}} & {\Psi_{{\mathrm{1}}}  \Ellesym{,}  \Ellent{Y}  \Ellesym{,}  \Psi_{{\mathrm{2}}}  \vdash_\mathcal{C}  \Ellent{Z}}
                \end{array}
              }
            }{\Psi_{{\mathrm{1}}}  \Ellesym{,}  \Phi_{{\mathrm{2}}}  \Ellesym{,}  \Ellent{X_{{\mathrm{2}}}}  \Ellesym{,}  \Phi_{{\mathrm{3}}}  \Ellesym{,}  \Psi_{{\mathrm{2}}}  \vdash_\mathcal{C}  \Ellent{Z}}
          }{\Psi_{{\mathrm{1}}}  \Ellesym{,}  \Phi_{{\mathrm{2}}}  \Ellesym{,}  \Ellent{X_{{\mathrm{1}}}}  \multimap  \Ellent{X_{{\mathrm{2}}}}  \Ellesym{,}  \Phi_{{\mathrm{1}}}  \Ellesym{,}  \Phi_{{\mathrm{3}}}  \Ellesym{,}  \Psi_{{\mathrm{2}}}  \vdash_\mathcal{C}  \Ellent{Z}}
        \end{math}
      \end{center}

    \item \ElledruleTXXimpLName / $\cat{L}$-sequent:
      \begin{center}
        \scriptsize
        $\Pi_1$:
        \begin{math}
          $$\mprset{flushleft}
          \inferrule* [right={\tiny impL}] {
            {
              \begin{array}{cc}
                \pi_1 & \pi_2 \\
                {\Phi_{{\mathrm{1}}}  \vdash_\mathcal{C}  \Ellent{X_{{\mathrm{1}}}}} & {\Phi_{{\mathrm{2}}}  \Ellesym{,}  \Ellent{X_{{\mathrm{2}}}}  \Ellesym{,}  \Phi_{{\mathrm{3}}}  \vdash_\mathcal{C}  \Ellent{Y}}
              \end{array}
            }
          }{\Phi_{{\mathrm{2}}}  \Ellesym{,}  \Ellent{X_{{\mathrm{1}}}}  \multimap  \Ellent{X_{{\mathrm{2}}}}  \Ellesym{,}  \Phi_{{\mathrm{1}}}  \Ellesym{,}  \Phi_{{\mathrm{3}}}  \vdash_\mathcal{C}  \Ellent{Y}}
        \end{math}
        \qquad\qquad
        \begin{math}
          \begin{array}{c}
            \Pi_2 \\
            {\Gamma_{{\mathrm{1}}}  \Ellesym{,}  \Ellent{Y}  \Ellesym{,}  \Gamma_{{\mathrm{2}}}  \vdash_\mathcal{L}  \Ellent{A}}
          \end{array}
        \end{math}
      \end{center}
      By assumption, $c(\Pi_1),c(\Pi_2)\leq |Y|$. By induction, there is a proof $\Pi'$ from
      $\pi_2$ and $\Pi_2$ for $\Gamma_{{\mathrm{1}}}  \Ellesym{,}  \Phi_{{\mathrm{2}}}  \Ellesym{,}  \Ellent{X_{{\mathrm{2}}}}  \Ellesym{,}  \Phi_{{\mathrm{3}}}  \Ellesym{,}  \Gamma_{{\mathrm{2}}}  \vdash_\mathcal{L}  \Ellent{A}$ s.t. $c(\Pi')\leq |Y|$.
      Therefore, the proof $\Pi$ can be constructed as follows with $c(\Pi)\leq |Y|$.
      \begin{center}
        \scriptsize
        \begin{math}
          $$\mprset{flushleft}
          \inferrule* [right={\tiny impL}] {
            {
              \begin{array}{c}
                \pi_1 \\
                {\Phi_{{\mathrm{1}}}  \vdash_\mathcal{C}  \Ellent{X_{{\mathrm{1}}}}}
              \end{array}
            }
            $$\mprset{flushleft}
            \inferrule* [right={\tiny cut}] {
              {
                \begin{array}{cc}
                  \pi_2 & \Pi_2 \\
                  {\Phi_{{\mathrm{2}}}  \Ellesym{,}  \Ellent{X_{{\mathrm{2}}}}  \Ellesym{,}  \Phi_{{\mathrm{3}}}  \vdash_\mathcal{C}  \Ellent{Y}} & {\Gamma_{{\mathrm{1}}}  \Ellesym{,}  \Ellent{Y}  \Ellesym{,}  \Gamma_{{\mathrm{2}}}  \vdash_\mathcal{L}  \Ellent{A}}
                \end{array}
              }
            }{\Gamma_{{\mathrm{1}}}  \Ellesym{,}  \Phi_{{\mathrm{2}}}  \Ellesym{,}  \Ellent{X_{{\mathrm{2}}}}  \Ellesym{,}  \Phi_{{\mathrm{3}}}  \Ellesym{,}  \Gamma_{{\mathrm{2}}}  \vdash_\mathcal{L}  \Ellent{A}}
          }{\Gamma_{{\mathrm{1}}}  \Ellesym{,}  \Phi_{{\mathrm{2}}}  \Ellesym{,}  \Ellent{X_{{\mathrm{1}}}}  \multimap  \Ellent{X_{{\mathrm{2}}}}  \Ellesym{,}  \Phi_{{\mathrm{1}}}  \Ellesym{,}  \Phi_{{\mathrm{3}}}  \Ellesym{,}  \Gamma_{{\mathrm{2}}}  \vdash_\mathcal{L}  \Ellent{A}}
        \end{math}
      \end{center}

    \item \ElledruleSXXimprLName / $\cat{L}$-sequent:
      \begin{center}
        \scriptsize
        $\Pi_1$:
        \begin{math}
          $$\mprset{flushleft}
          \inferrule* [right={\tiny impL}] {
            {
              \begin{array}{cc}
                \pi_1 & \pi_2 \\
                {\Gamma_{{\mathrm{1}}}  \vdash_\mathcal{L}  \Ellent{A_{{\mathrm{1}}}}} & {\Gamma_{{\mathrm{2}}}  \Ellesym{,}  \Ellent{A_{{\mathrm{2}}}}  \vdash_\mathcal{L}  \Ellent{B}}
              \end{array}
            }
          }{\Gamma_{{\mathrm{2}}}  \Ellesym{,}  \Ellent{A_{{\mathrm{1}}}}  \rightharpoonup  \Ellent{B_{{\mathrm{2}}}}  \Ellesym{,}  \Gamma_{{\mathrm{1}}}  \vdash_\mathcal{L}  \Ellent{B}}
        \end{math}
        \qquad\qquad
        \begin{math}
          \begin{array}{c}
            \Pi_2 \\
            {\Delta_{{\mathrm{1}}}  \Ellesym{,}  \Ellent{B}  \Ellesym{,}  \Delta_{{\mathrm{2}}}  \vdash_\mathcal{L}  \Ellent{C}}
          \end{array}
        \end{math}
      \end{center}
      By assumption, $c(\Pi_1),c(\Pi_2)\leq |B|$. By induction, there is a proof $\Pi'$ from
      $\pi_2$ and $\Pi_2$ for $\Delta_{{\mathrm{1}}}  \Ellesym{,}  \Gamma_{{\mathrm{2}}}  \Ellesym{,}  \Ellent{A_{{\mathrm{2}}}}  \Ellesym{,}  \Delta_{{\mathrm{2}}}  \vdash_\mathcal{L}  \Ellent{C}$ s.t. $c(\Pi')\leq |B|$.
      Therefore, the proof $\Pi$ can be constructed as follows with $c(\Pi)\leq |B|$.
      \begin{center}
        \scriptsize
        \begin{math}
          $$\mprset{flushleft}
          \inferrule* [right={\tiny impL}] {
            {
              \begin{array}{c}
                \pi_1 \\
                {\Gamma_{{\mathrm{1}}}  \vdash_\mathcal{L}  \Ellent{A_{{\mathrm{1}}}}}
              \end{array}
            }
            $$\mprset{flushleft}
            \inferrule* [right={\tiny cut}] {
              {
                \begin{array}{cc}
                  \pi_2 & \Pi_2 \\
                  {\Gamma_{{\mathrm{2}}}  \Ellesym{,}  \Ellent{A_{{\mathrm{2}}}}  \vdash_\mathcal{L}  \Ellent{B}} & {\Delta_{{\mathrm{1}}}  \Ellesym{,}  \Ellent{B}  \Ellesym{,}  \Delta_{{\mathrm{2}}}  \vdash_\mathcal{L}  \Ellent{C}}
                \end{array}
              }
            }{\Delta_{{\mathrm{1}}}  \Ellesym{,}  \Gamma_{{\mathrm{2}}}  \Ellesym{,}  \Ellent{A_{{\mathrm{2}}}}  \Ellesym{,}  \Delta_{{\mathrm{2}}}  \vdash_\mathcal{L}  \Ellent{C}}
          }{?}
        \end{math}
      \end{center}

    \item \ElledruleSXXFlName / $\cat{L}$-sequent:
      \begin{center}
        \scriptsize
        $\Pi_1$:
        \begin{math}
          $$\mprset{flushleft}
          \inferrule* [right={\tiny FL}] {
            {
              \begin{array}{c}
                \pi_1 \\
                {\Gamma_{{\mathrm{1}}}  \Ellesym{,}  \Ellent{X}  \Ellesym{,}  \Gamma_{{\mathrm{2}}}  \vdash_\mathcal{L}  \Ellent{A}}
              \end{array}
            }
          }{\Gamma_{{\mathrm{1}}}  \Ellesym{,}   \mathsf{F} \Ellent{X}   \Ellesym{,}  \Gamma_{{\mathrm{2}}}  \vdash_\mathcal{L}  \Ellent{A}}
        \end{math}
        \qquad\qquad
        \begin{math}
          \begin{array}{c}
            \Pi_2 \\
            {\Delta_{{\mathrm{1}}}  \Ellesym{,}  \Ellent{A}  \Ellesym{,}  \Delta_{{\mathrm{2}}}  \vdash_\mathcal{L}  \Ellent{B}}
          \end{array}
        \end{math}
      \end{center}
      By assumption, $c(\Pi_1),c(\Pi_2)\leq |A|$. By induction, there is a proof $\Pi'$ from
      $\pi_2$ and $\Pi_2$ for $\Delta_{{\mathrm{1}}}  \Ellesym{,}  \Gamma_{{\mathrm{1}}}  \Ellesym{,}  \Ellent{X}  \Ellesym{,}  \Gamma_{{\mathrm{2}}}  \Ellesym{,}  \Delta_{{\mathrm{2}}}  \vdash_\mathcal{L}  \Ellent{B}$ s.t. $c(\Pi')\leq |A|$.
      Therefore, the proof $\Pi$ can be constructed as follows with $c(\Pi)\leq |A|$.
      \begin{center}
        \scriptsize
        \begin{math}
          $$\mprset{flushleft}
          \inferrule* [right={\tiny FL}] {
            $$\mprset{flushleft}
            \inferrule* [right={\tiny cut2}] {
              {
                \begin{array}{cc}
                  \pi_2 & \Pi_2 \\
                  {\Gamma_{{\mathrm{1}}}  \Ellesym{,}  \Ellent{X}  \Ellesym{,}  \Gamma_{{\mathrm{2}}}  \vdash_\mathcal{L}  \Ellent{A}} & {\Delta_{{\mathrm{1}}}  \Ellesym{,}  \Ellent{A}  \Ellesym{,}  \Delta_{{\mathrm{2}}}  \vdash_\mathcal{L}  \Ellent{B}}
                \end{array}
              }
            }{\Delta_{{\mathrm{1}}}  \Ellesym{,}  \Gamma_{{\mathrm{1}}}  \Ellesym{,}  \Ellent{X}  \Ellesym{,}  \Gamma_{{\mathrm{2}}}  \Ellesym{,}  \Delta_{{\mathrm{2}}}  \vdash_\mathcal{L}  \Ellent{B}}
          }{\Delta_{{\mathrm{1}}}  \Ellesym{,}  \Gamma_{{\mathrm{1}}}  \Ellesym{,}   \mathsf{F} \Ellent{X}   \Ellesym{,}  \Gamma_{{\mathrm{2}}}  \Ellesym{,}  \Delta_{{\mathrm{2}}}  \vdash_\mathcal{L}  \Ellent{B}}
        \end{math}
      \end{center}

    \item \ElledruleSXXGlName / $\cat{L}$-sequent:
      \begin{center}
        \scriptsize
        $\Pi_1$:
        \begin{math}
          $$\mprset{flushleft}
          \inferrule* [right={\tiny GL}] {
            {
              \begin{array}{c}
                \pi_1 \\
                {\Gamma_{{\mathrm{1}}}  \Ellesym{,}  \Ellent{A}  \Ellesym{,}  \Gamma_{{\mathrm{2}}}  \vdash_\mathcal{L}  \Ellent{B}}
              \end{array}
            }
          }{\Gamma_{{\mathrm{1}}}  \Ellesym{,}   \mathsf{G} \Ellent{A}   \Ellesym{,}  \Gamma_{{\mathrm{2}}}  \vdash_\mathcal{L}  \Ellent{B}}
        \end{math}
        \qquad\qquad
        \begin{math}
          \begin{array}{c}
            \Pi_2 \\
            {\Delta_{{\mathrm{1}}}  \Ellesym{,}  \Ellent{B}  \Ellesym{,}  \Delta_{{\mathrm{2}}}  \vdash_\mathcal{L}  \Ellent{C}}
          \end{array}
        \end{math}
      \end{center}
      By assumption, $c(\Pi_1),c(\Pi_2)\leq |B|$. By induction, there is a proof $\Pi'$ from
      $\pi_2$ and $\Pi_2$ for $\Delta_{{\mathrm{1}}}  \Ellesym{,}  \Gamma_{{\mathrm{1}}}  \Ellesym{,}  \Ellent{A}  \Ellesym{,}  \Gamma_{{\mathrm{2}}}  \Ellesym{,}  \Delta_{{\mathrm{2}}}  \vdash_\mathcal{L}  \Ellent{C}$ s.t. $c(\Pi')\leq |B|$.
      Therefore, the proof $\Pi$ can be constructed as follows with $c(\Pi)\leq |B|$.
      \begin{center}
        \scriptsize
        \begin{math}
          $$\mprset{flushleft}
          \inferrule* [right={\tiny GL}] {
            $$\mprset{flushleft}
            \inferrule* [right={\tiny cut2}] {
              {
                \begin{array}{cc}
                  \pi_2 & \Pi_2 \\
                  {\Gamma_{{\mathrm{1}}}  \Ellesym{,}  \Ellent{A}  \Ellesym{,}  \Gamma_{{\mathrm{2}}}  \vdash_\mathcal{L}  \Ellent{B}} & {\Delta_{{\mathrm{1}}}  \Ellesym{,}  \Ellent{B}  \Ellesym{,}  \Delta_{{\mathrm{2}}}  \vdash_\mathcal{L}  \Ellent{C}}
                \end{array}
              }
            }{\Delta_{{\mathrm{1}}}  \Ellesym{,}  \Gamma_{{\mathrm{1}}}  \Ellesym{,}  \Ellent{A}  \Ellesym{,}  \Gamma_{{\mathrm{2}}}  \Ellesym{,}  \Delta_{{\mathrm{2}}}  \vdash_\mathcal{L}  \Ellent{C}}
          }{\Delta_{{\mathrm{1}}}  \Ellesym{,}  \Gamma_{{\mathrm{1}}}  \Ellesym{,}   \mathsf{G} \Ellent{A}   \Ellesym{,}  \Gamma_{{\mathrm{2}}}  \Ellesym{,}  \Delta_{{\mathrm{2}}}  \vdash_\mathcal{L}  \Ellent{C}}
        \end{math}
      \end{center}

    \end{itemize}

  \item The conclusion of $\Pi_1$ is secondary in $\Pi_2$, we discuss the following cases
        depending on the last rule of $\Pi_2$:

  \begin{itemize}
  \item $\ElledruleTXXunitLName$ (with low priority):
    \begin{center}
      \scriptsize
      \begin{math}
        \begin{array}{c}
          \Pi_1 \\
          {\Psi  \vdash_\mathcal{C}  \Ellent{X}}
        \end{array}
      \end{math}
      \qquad\qquad
      $\Pi_2$:
      \begin{math}
        $$\mprset{flushleft}
        \inferrule* [right={\tiny unitL}] {
          {
            \begin{array}{c}
              \pi \\
              {\Phi_{{\mathrm{1}}}  \Ellesym{,}  \Ellent{X}  \Ellesym{,}  \Phi_{{\mathrm{2}}}  \vdash_\mathcal{C}  \Ellent{Y}}
            \end{array}
          }
        }{ \mathsf{UnitT}   \Ellesym{,}  \Phi_{{\mathrm{1}}}  \Ellesym{,}  \Ellent{X}  \Ellesym{,}  \Phi_{{\mathrm{2}}}  \vdash_\mathcal{C}  \Ellent{Y}}
      \end{math}
    \end{center}
    By assumption, $c(\Pi_1),c(\Pi_2)\leq |X|$. By induction on $\Pi_1$ and $\pi$, there is a
    proof $\Pi'$ for sequent $\Phi_{{\mathrm{1}}}  \Ellesym{,}  \Psi  \Ellesym{,}  \Phi_{{\mathrm{2}}}  \vdash_\mathcal{C}  \Ellent{Y}$ s.t. $c(\Pi') \leq |X|$. Therefore, the
    proof $\Pi$ can be constructed as follows with $c(\Pi) = c(\Pi') \leq |X|$.
    \begin{center}
      \scriptsize
      \begin{math}
        $$\mprset{flushleft}
        \inferrule* [right={\tiny unitL}] {
          {
            \begin{array}{c}
              \Pi' \\
              {\Phi_{{\mathrm{1}}}  \Ellesym{,}  \Psi  \Ellesym{,}  \Phi_{{\mathrm{2}}}  \vdash_\mathcal{C}  \Ellent{Y}}
            \end{array}
          }
        }{ \mathsf{UnitT}   \Ellesym{,}  \Phi_{{\mathrm{1}}}  \Ellesym{,}  \Psi  \Ellesym{,}  \Phi_{{\mathrm{2}}}  \vdash_\mathcal{C}  \Ellent{Y}}
      \end{math}
    \end{center}

  \item $\ElledruleTXXtenRName$ Case 1:
      \begin{center}
        \scriptsize
        \begin{math}
          \begin{array}{c}
            \Pi_1 \\
            {\Phi_{{\mathrm{2}}}  \vdash_\mathcal{C}  \Ellent{X}}
          \end{array}
        \end{math}
        \qquad\qquad
        $\Pi_2$:
        \begin{math}
          $$\mprset{flushleft}
          \inferrule* [right={\tiny tenR}] {
            {
              \begin{array}{cc}
                \pi_1 & \pi_2 \\
                {\Psi_{{\mathrm{1}}}  \Ellesym{,}  \Ellent{X}  \Ellesym{,}  \Psi_{{\mathrm{2}}}  \vdash_\mathcal{C}  \Ellent{Y_{{\mathrm{1}}}}} & {\Phi_{{\mathrm{1}}}  \vdash_\mathcal{C}  \Ellent{Y_{{\mathrm{2}}}}}
              \end{array}
            }
          }{\Psi_{{\mathrm{1}}}  \Ellesym{,}  \Ellent{X}  \Ellesym{,}  \Psi_{{\mathrm{2}}}  \Ellesym{,}  \Phi_{{\mathrm{1}}}  \vdash_\mathcal{C}  \Ellent{Y_{{\mathrm{1}}}}  \otimes  \Ellent{Y_{{\mathrm{2}}}}}
        \end{math}
      \end{center}
      By assumption, $c(\Pi_1),c(\Pi_2)\leq |X|$. By induction on $\Pi_1$ and $\pi_1$, there
      is a proof $\Pi'$ for sequent $\Psi_{{\mathrm{1}}}  \Ellesym{,}  \Phi_{{\mathrm{2}}}  \Ellesym{,}  \Psi_{{\mathrm{2}}}  \vdash_\mathcal{C}  \Ellent{Y_{{\mathrm{1}}}}$ s.t. $c(\Pi') \leq |X|$.
      Therefore, the proof $\Pi$ can be constructed as follows with
      $c(\Pi) = c(\Pi') \leq |X|$.
      \begin{center}
        \scriptsize
        \begin{math}
          $$\mprset{flushleft}
          \inferrule* [right={\tiny tenR}] {
            {
              \begin{array}{cc}
                \Pi' & \pi_1 \\
                {\Psi_{{\mathrm{1}}}  \Ellesym{,}  \Phi_{{\mathrm{2}}}  \Ellesym{,}  \Psi_{{\mathrm{2}}}  \vdash_\mathcal{C}  \Ellent{Y_{{\mathrm{1}}}}} & {\Phi_{{\mathrm{1}}}  \vdash_\mathcal{C}  \Ellent{Y_{{\mathrm{2}}}}}
              \end{array}
            }
          }{\Psi_{{\mathrm{1}}}  \Ellesym{,}  \Phi_{{\mathrm{2}}}  \Ellesym{,}  \Psi_{{\mathrm{2}}}  \Ellesym{,}  \Phi_{{\mathrm{1}}}  \vdash_\mathcal{C}  \Ellent{Y_{{\mathrm{1}}}}  \otimes  \Ellent{Y_{{\mathrm{2}}}}}
        \end{math}
      \end{center}

  \item $\ElledruleTXXtenRName$ Case 2:
      \begin{center}
        \scriptsize
        \begin{math}
          \begin{array}{c}
            \Pi_1 \\
            {\Phi_{{\mathrm{2}}}  \vdash_\mathcal{C}  \Ellent{X}}
          \end{array}
        \end{math}
        \qquad\qquad
        $\Pi_2$:
        \begin{math}
          $$\mprset{flushleft}
          \inferrule* [right={\tiny tenR}] {
            {
              \begin{array}{cc}
                \pi_1 & \pi_2 \\
                {\Phi_{{\mathrm{1}}}  \vdash_\mathcal{C}  \Ellent{Y_{{\mathrm{1}}}}} & {\Psi_{{\mathrm{1}}}  \Ellesym{,}  \Ellent{X}  \Ellesym{,}  \Psi_{{\mathrm{2}}}  \vdash_\mathcal{C}  \Ellent{Y_{{\mathrm{2}}}}}
              \end{array}
            }
          }{\Phi_{{\mathrm{1}}}  \Ellesym{,}  \Psi_{{\mathrm{1}}}  \Ellesym{,}  \Ellent{X}  \Ellesym{,}  \Psi_{{\mathrm{2}}}  \vdash_\mathcal{C}  \Ellent{Y_{{\mathrm{1}}}}  \otimes  \Ellent{Y_{{\mathrm{2}}}}}
        \end{math}
      \end{center}
      By assumption, $c(\Pi_1),c(\Pi_2)\leq |X|$. By induction on $\Pi_1$ and $\pi_2$, there
      is a proof $\Pi'$ for sequent $\Psi_{{\mathrm{1}}}  \Ellesym{,}  \Phi_{{\mathrm{2}}}  \Ellesym{,}  \Psi_{{\mathrm{2}}}  \vdash_\mathcal{C}  \Ellent{Y_{{\mathrm{2}}}}$ s.t. $c(\Pi') \leq |X|$.
      Therefore, the proof $\Pi$ can be constructed as follows with
      $c(\Pi) = c(\Pi') \leq |X|$.
      \begin{center}
        \scriptsize
        \begin{math}
          $$\mprset{flushleft}
          \inferrule* [right={\tiny tenR}] {
            {
              \begin{array}{cc}
                \pi_1 & \Pi' \\
                {\Phi_{{\mathrm{1}}}  \vdash_\mathcal{C}  \Ellent{Y_{{\mathrm{1}}}}} & {\Psi_{{\mathrm{1}}}  \Ellesym{,}  \Phi_{{\mathrm{2}}}  \Ellesym{,}  \Psi_{{\mathrm{2}}}  \vdash_\mathcal{C}  \Ellent{Y_{{\mathrm{2}}}}}
              \end{array}
            }
          }{\Phi_{{\mathrm{1}}}  \Ellesym{,}  \Psi_{{\mathrm{1}}}  \Ellesym{,}  \Phi_{{\mathrm{2}}}  \Ellesym{,}  \Psi_{{\mathrm{2}}}  \vdash_\mathcal{C}  \Ellent{Y_{{\mathrm{1}}}}  \otimes  \Ellent{Y_{{\mathrm{2}}}}}
        \end{math}
      \end{center}

  \item $\ElledruleTXXtenLName$ (with low priority) Case 1:
      \begin{center}
        \scriptsize
        \begin{math}
          \begin{array}{c}
            \Pi_1 \\
            {\Phi  \vdash_\mathcal{C}  \Ellent{X}}
          \end{array}
        \end{math}
        \qquad\qquad
        $\Pi_2$:
        \begin{math}
          $$\mprset{flushleft}
          \inferrule* [right={\tiny tenL}] {
            {
              \begin{array}{c}
                \pi \\
                {\Psi_{{\mathrm{1}}}  \Ellesym{,}  \Ellent{X}  \Ellesym{,}  \Psi_{{\mathrm{2}}}  \Ellesym{,}  \Ellent{Y_{{\mathrm{1}}}}  \Ellesym{,}  \Ellent{Y_{{\mathrm{2}}}}  \Ellesym{,}  \Psi_{{\mathrm{3}}}  \vdash_\mathcal{C}  \Ellent{Z}}
              \end{array}
            }
          }{\Psi_{{\mathrm{1}}}  \Ellesym{,}  \Ellent{X}  \Ellesym{,}  \Psi_{{\mathrm{2}}}  \Ellesym{,}  \Ellent{Y_{{\mathrm{1}}}}  \otimes  \Ellent{Y_{{\mathrm{2}}}}  \Ellesym{,}  \Psi_{{\mathrm{3}}}  \vdash_\mathcal{C}  \Ellent{Z}}
        \end{math}
      \end{center}
      By assumption, $c(\Pi_1),c(\Pi_2)\leq |X|$. By induction on $\Pi_1$ and $\pi$, there is
      a proof $\Pi'$ for sequent $\Psi_{{\mathrm{1}}}  \Ellesym{,}  \Phi  \Ellesym{,}  \Psi_{{\mathrm{2}}}  \Ellesym{,}  \Ellent{Y_{{\mathrm{1}}}}  \Ellesym{,}  \Ellent{Y_{{\mathrm{2}}}}  \Ellesym{,}  \Psi_{{\mathrm{3}}}  \vdash_\mathcal{C}  \Ellent{Z}$ s.t. $c(\Pi') \leq |X|$.
      Therefore, the proof $\Pi$ can be constructed as follows with
      $c(\Pi) = c(\Pi') \leq |X|$.
      \begin{center}
        \scriptsize
        \begin{math}
          $$\mprset{flushleft}
          \inferrule* [right={\tiny tenL}] {
            {
              \begin{array}{c}
                \Pi' \\
                {\Psi_{{\mathrm{1}}}  \Ellesym{,}  \Phi  \Ellesym{,}  \Psi_{{\mathrm{2}}}  \Ellesym{,}  \Ellent{Y_{{\mathrm{1}}}}  \Ellesym{,}  \Ellent{Y_{{\mathrm{2}}}}  \Ellesym{,}  \Psi_{{\mathrm{3}}}  \vdash_\mathcal{C}  \Ellent{Z}}
              \end{array}
            }
          }{\Psi_{{\mathrm{1}}}  \Ellesym{,}  \Phi  \Ellesym{,}  \Psi_{{\mathrm{2}}}  \Ellesym{,}  \Ellent{Y_{{\mathrm{1}}}}  \otimes  \Ellent{Y_{{\mathrm{2}}}}  \Ellesym{,}  \Psi_{{\mathrm{3}}}  \vdash_\mathcal{C}  \Ellent{Z}}
        \end{math}
      \end{center}

  \item $\ElledruleTXXtenLName$ (with low priority) Case 2:
      \begin{center}
        \scriptsize
        \begin{math}
          \begin{array}{c}
            \Pi_1 \\
            {\Phi  \vdash_\mathcal{C}  \Ellent{X}}
          \end{array}
        \end{math}
        \qquad\qquad
        $\Pi_2$:
        \begin{math}
          $$\mprset{flushleft}
          \inferrule* [right={\tiny tenL}] {
            {
              \begin{array}{c}
                \pi \\
                {\Psi_{{\mathrm{1}}}  \Ellesym{,}  \Ellent{Y_{{\mathrm{1}}}}  \Ellesym{,}  \Ellent{Y_{{\mathrm{2}}}}  \Ellesym{,}  \Psi_{{\mathrm{2}}}  \Ellesym{,}  \Ellent{X}  \Ellesym{,}  \Psi_{{\mathrm{3}}}  \vdash_\mathcal{C}  \Ellent{Z}}
              \end{array}
            }
          }{\Psi_{{\mathrm{1}}}  \Ellesym{,}  \Ellent{Y_{{\mathrm{1}}}}  \otimes  \Ellent{Y_{{\mathrm{2}}}}  \Ellesym{,}  \Psi_{{\mathrm{2}}}  \Ellesym{,}  \Ellent{X}  \Ellesym{,}  \Psi_{{\mathrm{3}}}  \vdash_\mathcal{C}  \Ellent{Z}}
        \end{math}
      \end{center}
      By assumption, $c(\Pi_1),c(\Pi_2)\leq |X|$. By induction on $\Pi_1$ and $\pi$, there is
      a proof $\Pi'$ for sequent $\Psi_{{\mathrm{1}}}  \Ellesym{,}  \Ellent{Y_{{\mathrm{1}}}}  \Ellesym{,}  \Ellent{Y_{{\mathrm{2}}}}  \Ellesym{,}  \Psi_{{\mathrm{2}}}  \Ellesym{,}  \Phi  \Ellesym{,}  \Psi_{{\mathrm{3}}}  \vdash_\mathcal{C}  \Ellent{Z}$ s.t. $c(\Pi') \leq |X|$.
      Therefore, the proof $\Pi$ can be constructed as follows with
      $c(\Pi) = c(\Pi') \leq |X|$.
      \begin{center}
        \scriptsize
        \begin{math}
          $$\mprset{flushleft}
          \inferrule* [right={\tiny tenL}] {
            {
              \begin{array}{c}
                \Pi' \\
                {\Psi_{{\mathrm{1}}}  \Ellesym{,}  \Ellent{Y_{{\mathrm{1}}}}  \Ellesym{,}  \Ellent{Y_{{\mathrm{2}}}}  \Ellesym{,}  \Psi_{{\mathrm{2}}}  \Ellesym{,}  \Phi  \Ellesym{,}  \Psi_{{\mathrm{3}}}  \vdash_\mathcal{C}  \Ellent{Z}}
              \end{array}
            }
          }{\Psi_{{\mathrm{1}}}  \Ellesym{,}  \Ellent{Y_{{\mathrm{1}}}}  \otimes  \Ellent{Y_{{\mathrm{2}}}}  \Ellesym{,}  \Psi_{{\mathrm{2}}}  \Ellesym{,}  \Phi  \Ellesym{,}  \Psi_{{\mathrm{3}}}  \vdash_\mathcal{C}  \Ellent{Z}}
        \end{math}
      \end{center}

  \item $\ElledruleTXXimpRName$ (with low priority): 
      \begin{center}
        \scriptsize
        \begin{math}
          \begin{array}{c}
            \Pi_1 \\
            {\Phi  \vdash_\mathcal{C}  \Ellent{X}}
          \end{array}
        \end{math}
        \qquad\qquad
        $\Pi_2$:
        \begin{math}
          $$\mprset{flushleft}
          \inferrule* [right={\tiny impR}] {
            {
              \begin{array}{c}
                \pi \\
                {\Psi_{{\mathrm{1}}}  \Ellesym{,}  \Ellent{X}  \Ellesym{,}  \Psi_{{\mathrm{2}}}  \Ellesym{,}  \Ellent{Y_{{\mathrm{1}}}}  \vdash_\mathcal{C}  \Ellent{Y_{{\mathrm{2}}}}}
              \end{array}
            }
          }{\Psi_{{\mathrm{1}}}  \Ellesym{,}  \Ellent{X}  \Ellesym{,}  \Psi_{{\mathrm{2}}}  \vdash_\mathcal{C}  \Ellent{Y_{{\mathrm{1}}}}  \multimap  \Ellent{Y_{{\mathrm{2}}}}}
        \end{math}
      \end{center}
      By assumption, $c(\Pi_1),c(\Pi_2)\leq |X|$. By induction on $\Pi_1$ and $\pi$, there
      is a proof $\Pi'$ for sequent $\Psi_{{\mathrm{1}}}  \Ellesym{,}  \Phi  \Ellesym{,}  \Psi_{{\mathrm{2}}}  \Ellesym{,}  \Ellent{Y_{{\mathrm{1}}}}  \vdash_\mathcal{C}  \Ellent{Y_{{\mathrm{2}}}}$ s.t. $c(\Pi') \leq |X|$.
      Therefore, the proof $\Pi$ can be constructed as follows with
      $c(\Pi) = c(\Pi') \leq |X|$.
      \begin{center}
        \scriptsize
        \begin{math}
          $$\mprset{flushleft}
          \inferrule* [right={\tiny impR}] {
            {
              \begin{array}{c}
                \Pi' \\
                {\Psi_{{\mathrm{1}}}  \Ellesym{,}  \Phi  \Ellesym{,}  \Psi_{{\mathrm{2}}}  \Ellesym{,}  \Ellent{Y_{{\mathrm{1}}}}  \vdash_\mathcal{C}  \Ellent{Y_{{\mathrm{2}}}}}
              \end{array}
            }
          }{\Psi_{{\mathrm{1}}}  \Ellesym{,}  \Phi  \Ellesym{,}  \Psi_{{\mathrm{2}}}  \vdash_\mathcal{C}  \Ellent{Y_{{\mathrm{1}}}}  \multimap  \Ellent{Y_{{\mathrm{2}}}}}
        \end{math}
      \end{center}

  \item $\ElledruleTXXimpLName$ Case 1:
      \begin{center}
        \scriptsize
        \begin{math}
          \begin{array}{c}
            \Pi_1 \\
            {\Phi  \vdash_\mathcal{C}  \Ellent{X}}
          \end{array}
        \end{math}
        \qquad\qquad
        $\Pi_2$:
        \begin{math}
          $$\mprset{flushleft}
          \inferrule* [right={\tiny impL}] {
            {
              \begin{array}{cc}
                \pi_1 & \pi_2 \\
                {\Psi_{{\mathrm{2}}}  \Ellesym{,}  \Ellent{X}  \Ellesym{,}  \Psi_{{\mathrm{3}}}  \vdash_\mathcal{C}  \Ellent{Y_{{\mathrm{1}}}}} & {\Psi_{{\mathrm{1}}}  \Ellesym{,}  \Ellent{Y_{{\mathrm{2}}}}  \Ellesym{,}  \Psi_{{\mathrm{4}}}  \vdash_\mathcal{C}  \Ellent{Z}}
              \end{array}
            }
          }{\Psi_{{\mathrm{1}}}  \Ellesym{,}  \Ellent{Y_{{\mathrm{1}}}}  \multimap  \Ellent{Y_{{\mathrm{2}}}}  \Ellesym{,}  \Psi_{{\mathrm{2}}}  \Ellesym{,}  \Ellent{X}  \Ellesym{,}  \Psi_{{\mathrm{3}}}  \Ellesym{,}  \Psi_{{\mathrm{4}}}  \vdash_\mathcal{C}  \Ellent{Z}}
        \end{math}
      \end{center}
      By assumption, $c(\Pi_1),c(\Pi_2)\leq |X|$. By induction on $\Pi_1$ and $\pi_1$, there is
      a proof $\Pi'$ for sequent $\Psi_{{\mathrm{2}}}  \Ellesym{,}  \Phi  \Ellesym{,}  \Psi_{{\mathrm{3}}}  \vdash_\mathcal{C}  \Ellent{Y_{{\mathrm{1}}}}$ s.t. $c(\Pi') \leq |X|$. Therefore, the
      proof $\Pi$ can be constructed as follows with $c(\Pi) = c(\Pi') \leq |X|$.
      \begin{center}
        \scriptsize
        \begin{math}
          $$\mprset{flushleft}
          \inferrule* [right={\tiny impL}] {
            {
              \begin{array}{cc}
                \Pi' & \pi_2 \\
                {\Psi_{{\mathrm{2}}}  \Ellesym{,}  \Phi  \Ellesym{,}  \Psi_{{\mathrm{3}}}  \vdash_\mathcal{C}  \Ellent{Y_{{\mathrm{1}}}}} & {\Psi_{{\mathrm{1}}}  \Ellesym{,}  \Ellent{Y_{{\mathrm{2}}}}  \Ellesym{,}  \Psi_{{\mathrm{4}}}  \vdash_\mathcal{C}  \Ellent{Z}}
              \end{array}
            }
          }{\Psi_{{\mathrm{1}}}  \Ellesym{,}  \Ellent{Y_{{\mathrm{1}}}}  \multimap  \Ellent{Y_{{\mathrm{2}}}}  \Ellesym{,}  \Psi_{{\mathrm{2}}}  \Ellesym{,}  \Phi  \Ellesym{,}  \Psi_{{\mathrm{3}}}  \Ellesym{,}  \Psi_{{\mathrm{4}}}  \vdash_\mathcal{C}  \Ellent{Z}}
        \end{math}
      \end{center}

  \item $\ElledruleTXXimpLName$ Case 2:
      \begin{center}
        \scriptsize
        \begin{math}
          \begin{array}{c}
            \Pi_1 \\
            {\Phi  \vdash_\mathcal{C}  \Ellent{X}}
          \end{array}
        \end{math}
        \qquad\qquad
        $\Pi_2$:
        \begin{math}
          $$\mprset{flushleft}
          \inferrule* [right={\tiny impL}] {
            {
              \begin{array}{cc}
                \pi_1 & \pi_2 \\
                {\Psi_{{\mathrm{3}}}  \vdash_\mathcal{C}  \Ellent{Y_{{\mathrm{1}}}}} & {\Psi_{{\mathrm{1}}}  \Ellesym{,}  \Ellent{X}  \Ellesym{,}  \Psi_{{\mathrm{2}}}  \Ellesym{,}  \Ellent{Y_{{\mathrm{2}}}}  \Ellesym{,}  \Psi_{{\mathrm{4}}}  \vdash_\mathcal{C}  \Ellent{Z}}
              \end{array}
            }
          }{\Psi_{{\mathrm{1}}}  \Ellesym{,}  \Ellent{X}  \Ellesym{,}  \Psi_{{\mathrm{2}}}  \Ellesym{,}  \Ellent{Y_{{\mathrm{1}}}}  \multimap  \Ellent{Y_{{\mathrm{2}}}}  \Ellesym{,}  \Psi_{{\mathrm{3}}}  \Ellesym{,}  \Psi_{{\mathrm{4}}}  \vdash_\mathcal{C}  \Ellent{Z}}
        \end{math}
      \end{center}
      By assumption, $c(\Pi_1),c(\Pi_2)\leq |X|$. By induction on $\Pi_1$ and $\pi_2$, there is
      a proof $\Pi'$ for sequent $\Psi_{{\mathrm{1}}}  \Ellesym{,}  \Phi  \Ellesym{,}  \Psi_{{\mathrm{2}}}  \Ellesym{,}  \Ellent{Y_{{\mathrm{2}}}}  \Ellesym{,}  \Psi_{{\mathrm{4}}}  \vdash_\mathcal{C}  \Ellent{Z}$ s.t. $c(\Pi') \leq |X|$.
      Therefore, the proof $\Pi$ can be constructed as follows with
      $c(\Pi) = c(\Pi') \leq |X|$.
      \begin{center}
        \scriptsize
        \begin{math}
          $$\mprset{flushleft}
          \inferrule* [right={\tiny impL}] {
            {
              \begin{array}{cc}
                \pi_1 & \Pi' \\
                {\Psi_{{\mathrm{3}}}  \vdash_\mathcal{C}  \Ellent{Y_{{\mathrm{1}}}}} & {\Psi_{{\mathrm{1}}}  \Ellesym{,}  \Phi  \Ellesym{,}  \Psi_{{\mathrm{2}}}  \Ellesym{,}  \Ellent{Y_{{\mathrm{2}}}}  \Ellesym{,}  \Psi_{{\mathrm{4}}}  \vdash_\mathcal{C}  \Ellent{Z}}
              \end{array}
            }
          }{\Psi_{{\mathrm{1}}}  \Ellesym{,}  \Phi_{{\mathrm{1}}}  \Ellesym{,}  \Psi_{{\mathrm{2}}}  \Ellesym{,}  \Ellent{Y_{{\mathrm{1}}}}  \multimap  \Ellent{Y_{{\mathrm{2}}}}  \Ellesym{,}  \Psi_{{\mathrm{3}}}  \Ellesym{,}  \Psi_{{\mathrm{4}}}  \vdash_\mathcal{C}  \Ellent{Z}}
        \end{math}
      \end{center}

  \item $\ElledruleTXXimpLName$ Case 3:
      \begin{center}
        \scriptsize
        \begin{math}
          \begin{array}{c}
            \Pi_1 \\
            {\Phi  \vdash_\mathcal{C}  \Ellent{X}}
          \end{array}
        \end{math}
        \qquad\qquad
        $\Pi_2$:
        \begin{math}
          $$\mprset{flushleft}
          \inferrule* [right={\tiny impL}] {
            {
              \begin{array}{cc}
                \pi_1 & \pi_2 \\
                {\Psi_{{\mathrm{2}}}  \vdash_\mathcal{C}  \Ellent{Y_{{\mathrm{1}}}}} & {\Psi_{{\mathrm{1}}}  \Ellesym{,}  \Ellent{Y_{{\mathrm{2}}}}  \Ellesym{,}  \Psi_{{\mathrm{3}}}  \Ellesym{,}  \Ellent{X}  \Ellesym{,}  \Psi_{{\mathrm{4}}}  \vdash_\mathcal{C}  \Ellent{Z}}
              \end{array}
            }
          }{\Psi_{{\mathrm{1}}}  \Ellesym{,}  \Ellent{Y_{{\mathrm{1}}}}  \multimap  \Ellent{Y_{{\mathrm{2}}}}  \Ellesym{,}  \Psi_{{\mathrm{2}}}  \Ellesym{,}  \Psi_{{\mathrm{3}}}  \Ellesym{,}  \Ellent{X}  \Ellesym{,}  \Psi_{{\mathrm{4}}}  \vdash_\mathcal{C}  \Ellent{Z}}
        \end{math}
      \end{center}
      By assumption, $c(\Pi_1),c(\Pi_2)\leq |X|$. By induction on $\Pi_1$ and $\pi_2$, there is
      a proof $\Pi'$ for sequent $\Psi_{{\mathrm{1}}}  \Ellesym{,}  \Phi  \Ellesym{,}  \Psi_{{\mathrm{2}}}  \Ellesym{,}  \Ellent{Y_{{\mathrm{2}}}}  \Ellesym{,}  \Psi_{{\mathrm{4}}}  \vdash_\mathcal{C}  \Ellent{Z}$ s.t. $c(\Pi') \leq |X|$.
      Therefore, the proof $\Pi$ can be constructed as follows with
      $c(\Pi) = c(\Pi') \leq |X|$.
      \begin{center}
        \scriptsize
        \begin{math}
          $$\mprset{flushleft}
          \inferrule* [right={\tiny impL}] {
            {
              \begin{array}{cc}
                \pi_1 & \Pi' \\
                {\Psi_{{\mathrm{2}}}  \vdash_\mathcal{C}  \Ellent{Y_{{\mathrm{1}}}}} & {\Psi_{{\mathrm{1}}}  \Ellesym{,}  \Ellent{Y_{{\mathrm{2}}}}  \Ellesym{,}  \Psi_{{\mathrm{3}}}  \Ellesym{,}  \Phi  \Ellesym{,}  \Psi_{{\mathrm{4}}}  \vdash_\mathcal{C}  \Ellent{Z}}
              \end{array}
            }
          }{\Psi_{{\mathrm{1}}}  \Ellesym{,}  \Ellent{Y_{{\mathrm{1}}}}  \multimap  \Ellent{Y_{{\mathrm{2}}}}  \Ellesym{,}  \Psi_{{\mathrm{2}}}  \Ellesym{,}  \Psi_{{\mathrm{3}}}  \Ellesym{,}  \Phi  \Ellesym{,}  \Psi_{{\mathrm{4}}}  \vdash_\mathcal{C}  \Ellent{Z}}
        \end{math}
      \end{center}

    \item $\ElledruleTXXcutName$ Case 1:
      \begin{center}
        \scriptsize
        \begin{math}
          \begin{array}{c}
            \Pi_1 \\
            {\Phi_{{\mathrm{1}}}  \vdash_\mathcal{C}  \Ellent{X}}
          \end{array}
        \end{math}
        \qquad\qquad
        $\Pi_2$:
        \begin{math}
          $$\mprset{flushleft}
          \inferrule* [right={\tiny cut}] {
            {
              \begin{array}{cc}
                \pi_1 & \pi_2 \\
                {\Phi_{{\mathrm{2}}}  \vdash_\mathcal{C}  \Ellent{Y}} & {\Psi_{{\mathrm{1}}}  \Ellesym{,}  \Ellent{X}  \Ellesym{,}  \Psi_{{\mathrm{2}}}  \Ellesym{,}  \Ellent{Y}  \Ellesym{,}  \Psi_{{\mathrm{3}}}  \vdash_\mathcal{C}  \Ellent{Z}}
              \end{array}
            }
          }{\Psi_{{\mathrm{1}}}  \Ellesym{,}  \Ellent{X}  \Ellesym{,}  \Psi_{{\mathrm{2}}}  \Ellesym{,}  \Phi_{{\mathrm{2}}}  \Ellesym{,}  \Psi_{{\mathrm{3}}}  \vdash_\mathcal{C}  \Ellent{Z}}
        \end{math}
      \end{center}
      By assumption, $c(\Pi_1),c(\Pi_2)\leq |X|$. So $|Y|+1 \leq |X|$. By induction on $\Pi_1$
      and $\pi_2$, there is a proof $\Pi'$ for sequent $\Psi_{{\mathrm{1}}}  \Ellesym{,}  \Phi_{{\mathrm{1}}}  \Ellesym{,}  \Psi_{{\mathrm{2}}}  \Ellesym{,}  \Ellent{Y}  \Ellesym{,}  \Psi_{{\mathrm{3}}}  \vdash_\mathcal{C}  \Ellent{Z}$ s.t.
      $c(\Pi') \leq |X|$. Therefore, the proof $\Pi$ can be constructed as follows with
      $c(\Pi) = max\{\Pi', |Y|+1\} \leq |X|$.
      \begin{center}
        \scriptsize
        \begin{math}
          $$\mprset{flushleft}
          \inferrule* [right={\tiny cut}] {
            {
              \begin{array}{cc}
                \pi_1 & \Pi' \\
                {\Phi_{{\mathrm{2}}}  \vdash_\mathcal{C}  \Ellent{Y}} & {\Psi_{{\mathrm{1}}}  \Ellesym{,}  \Phi_{{\mathrm{1}}}  \Ellesym{,}  \Psi_{{\mathrm{2}}}  \Ellesym{,}  \Ellent{Y}  \Ellesym{,}  \Psi_{{\mathrm{3}}}  \vdash_\mathcal{C}  \Ellent{Z}}
              \end{array}
            }
          }{\Psi_{{\mathrm{1}}}  \Ellesym{,}  \Phi_{{\mathrm{1}}}  \Ellesym{,}  \Psi_{{\mathrm{2}}}  \Ellesym{,}  \Phi_{{\mathrm{2}}}  \Ellesym{,}  \Psi_{{\mathrm{3}}}  \vdash_\mathcal{C}  \Ellent{Z}}
        \end{math}
      \end{center}

    \item $\ElledruleTXXcutName$ Case 2:
      \begin{center}
        \scriptsize
        \begin{math}
          \begin{array}{c}
            \Pi_1 \\
            {\Phi_{{\mathrm{1}}}  \vdash_\mathcal{C}  \Ellent{X}}
          \end{array}
        \end{math}
        \qquad\qquad
        $\Pi_2$:
        \begin{math}
          $$\mprset{flushleft}
          \inferrule* [right={\tiny cut}] {
            {
              \begin{array}{cc}
                \pi_1 & \pi_2 \\
                {\Phi_{{\mathrm{2}}}  \vdash_\mathcal{C}  \Ellent{Y}} & {\Psi_{{\mathrm{1}}}  \Ellesym{,}  \Ellent{Y}  \Ellesym{,}  \Psi_{{\mathrm{2}}}  \Ellesym{,}  \Ellent{X}  \Ellesym{,}  \Psi_{{\mathrm{3}}}  \vdash_\mathcal{C}  \Ellent{Z}}
              \end{array}
            }
          }{\Psi_{{\mathrm{1}}}  \Ellesym{,}  \Ellent{X}  \Ellesym{,}  \Psi_{{\mathrm{2}}}  \Ellesym{,}  \Phi_{{\mathrm{2}}}  \Ellesym{,}  \Psi_{{\mathrm{3}}}  \vdash_\mathcal{C}  \Ellent{Z}}
        \end{math}
      \end{center}
      By assumption, $c(\Pi_1),c(\Pi_2)\leq |X|$. So $|Y|+1 \leq |X|$. By induction on $\Pi_1$
      and $\pi_2$, there is a proof $\Pi'$ for sequent $\Psi_{{\mathrm{1}}}  \Ellesym{,}  \Ellent{Y}  \Ellesym{,}  \Psi_{{\mathrm{2}}}  \Ellesym{,}  \Phi_{{\mathrm{1}}}  \Ellesym{,}  \Psi_{{\mathrm{3}}}  \vdash_\mathcal{C}  \Ellent{Z}$ s.t.
      $c(\Pi') \leq |X|$. Therefore, the proof $\Pi$ can be constructed as follows with
      $c(\Pi) = max\{\Pi', |Y|+1\} \leq |X|$.
      \begin{center}
        \scriptsize
        \begin{math}
          $$\mprset{flushleft}
          \inferrule* [right={\tiny cut}] {
            {
              \begin{array}{cc}
                \pi_1 & \Pi' \\
                {\Phi_{{\mathrm{2}}}  \vdash_\mathcal{C}  \Ellent{Y}} & {\Psi_{{\mathrm{1}}}  \Ellesym{,}  \Ellent{Y}  \Ellesym{,}  \Psi_{{\mathrm{2}}}  \Ellesym{,}  \Phi_{{\mathrm{1}}}  \Ellesym{,}  \Psi_{{\mathrm{3}}}  \vdash_\mathcal{C}  \Ellent{Z}}
              \end{array}
            }
          }{\Psi_{{\mathrm{1}}}  \Ellesym{,}  \Phi_{{\mathrm{2}}}  \Ellesym{,}  \Psi_{{\mathrm{2}}}  \Ellesym{,}  \Phi_{{\mathrm{1}}}  \Ellesym{,}  \Psi_{{\mathrm{3}}}  \vdash_\mathcal{C}  \Ellent{Z}}
        \end{math}
      \end{center}

    \item $\ElledruleTXXbetaName$ Case 1:
      \begin{center}
        \scriptsize
        \begin{math}
          \begin{array}{c}
            \Pi_1 \\
            {\Phi  \vdash_\mathcal{C}  \Ellent{X}}
          \end{array}
        \end{math}
        \qquad\qquad
        $\Pi_2$:
        \begin{math}
          $$\mprset{flushleft}
          \inferrule* [right={\tiny beta}] {
            {
              \begin{array}{c}
                \pi \\
                {\Psi_{{\mathrm{1}}}  \Ellesym{,}  \Ellent{X}  \Ellesym{,}  \Psi_{{\mathrm{2}}}  \Ellesym{,}  \Ellent{Y_{{\mathrm{1}}}}  \Ellesym{,}  \Ellent{Y_{{\mathrm{2}}}}  \Ellesym{,}  \Psi_{{\mathrm{3}}}  \vdash_\mathcal{C}  \Ellent{Z}}
              \end{array}
            }
          }{\Psi_{{\mathrm{1}}}  \Ellesym{,}  \Ellent{X}  \Ellesym{,}  \Psi_{{\mathrm{2}}}  \Ellesym{,}  \Ellent{Y_{{\mathrm{2}}}}  \Ellesym{,}  \Ellent{Y_{{\mathrm{1}}}}  \Ellesym{,}  \Psi_{{\mathrm{3}}}  \vdash_\mathcal{C}  \Ellent{Z}}
        \end{math}
      \end{center}
      By assumption, $c(\Pi_1),c(\Pi_2)\leq |X|$. By induction on $\Pi_1$ and $\pi$, there is
      a proof $\Pi'$ for sequent $\Psi_{{\mathrm{1}}}  \Ellesym{,}  \Phi  \Ellesym{,}  \Psi_{{\mathrm{2}}}  \Ellesym{,}  \Ellent{Y_{{\mathrm{1}}}}  \Ellesym{,}  \Ellent{Y_{{\mathrm{2}}}}  \Ellesym{,}  \Psi_{{\mathrm{3}}}  \vdash_\mathcal{C}  \Ellent{Z}$ s.t. $c(\Pi') \leq |X|$.
      Therefore, the proof $\Pi$ can be constructed as follows with
      $c(\Pi) = c(\Pi') \leq |X|$.
      \begin{center}
        \scriptsize
        \begin{math}
          $$\mprset{flushleft}
          \inferrule* [right={\tiny cut}] {
            {
              \begin{array}{cc}
                \Pi' \\
                {\Psi_{{\mathrm{1}}}  \Ellesym{,}  \Phi  \Ellesym{,}  \Psi_{{\mathrm{2}}}  \Ellesym{,}  \Ellent{Y_{{\mathrm{1}}}}  \Ellesym{,}  \Ellent{Y_{{\mathrm{2}}}}  \Ellesym{,}  \Psi_{{\mathrm{3}}}  \vdash_\mathcal{C}  \Ellent{Z}}
              \end{array}
            }
          }{\Psi_{{\mathrm{1}}}  \Ellesym{,}  \Phi  \Ellesym{,}  \Psi_{{\mathrm{2}}}  \Ellesym{,}  \Ellent{Y_{{\mathrm{2}}}}  \Ellesym{,}  \Ellent{Y_{{\mathrm{1}}}}  \Ellesym{,}  \Psi_{{\mathrm{3}}}  \vdash_\mathcal{C}  \Ellent{Z}}
        \end{math}
      \end{center}

    \item $\ElledruleTXXbetaName$ Case 2:
      \begin{center}
        \scriptsize
        \begin{math}
          \begin{array}{c}
            \Pi_1 \\
            {\Phi  \vdash_\mathcal{C}  \Ellent{X}}
          \end{array}
        \end{math}
        \qquad\qquad
        $\Pi_2$:
        \begin{math}
          $$\mprset{flushleft}
          \inferrule* [right={\tiny beta}] {
            {
              \begin{array}{c}
                \pi \\
                {\Psi_{{\mathrm{1}}}  \Ellesym{,}  \Ellent{Y_{{\mathrm{1}}}}  \Ellesym{,}  \Ellent{Y_{{\mathrm{2}}}}  \Ellesym{,}  \Psi_{{\mathrm{2}}}  \Ellesym{,}  \Ellent{X}  \Ellesym{,}  \Psi_{{\mathrm{3}}}  \vdash_\mathcal{C}  \Ellent{Z}}
              \end{array}
            }
          }{\Psi_{{\mathrm{1}}}  \Ellesym{,}  \Ellent{X}  \Ellesym{,}  \Psi_{{\mathrm{2}}}  \Ellesym{,}  \Ellent{Y_{{\mathrm{2}}}}  \Ellesym{,}  \Ellent{Y_{{\mathrm{1}}}}  \Ellesym{,}  \Psi_{{\mathrm{3}}}  \vdash_\mathcal{C}  \Ellent{Z}}
        \end{math}
      \end{center}
      By assumption, $c(\Pi_1),c(\Pi_2)\leq |X|$. By induction on $\Pi_1$ and $\pi$, there is
      a proof $\Pi'$ for sequent $\Psi_{{\mathrm{1}}}  \Ellesym{,}  \Ellent{Y_{{\mathrm{1}}}}  \Ellesym{,}  \Ellent{Y_{{\mathrm{2}}}}  \Ellesym{,}  \Psi_{{\mathrm{2}}}  \Ellesym{,}  \Phi  \Ellesym{,}  \Psi_{{\mathrm{3}}}  \vdash_\mathcal{C}  \Ellent{Z}$ s.t. $c(\Pi') \leq |X|$.
      Therefore, the proof $\Pi$ can be constructed as follows with
      $c(\Pi) = c(\Pi') \leq |X|$.
      \begin{center}
        \scriptsize
        \begin{math}
          $$\mprset{flushleft}
          \inferrule* [right={\tiny cut}] {
            {
              \begin{array}{cc}
                \Pi' \\
                {\Psi_{{\mathrm{1}}}  \Ellesym{,}  \Ellent{Y_{{\mathrm{1}}}}  \Ellesym{,}  \Ellent{Y_{{\mathrm{2}}}}  \Ellesym{,}  \Psi_{{\mathrm{2}}}  \Ellesym{,}  \Phi  \Ellesym{,}  \Psi_{{\mathrm{3}}}  \vdash_\mathcal{C}  \Ellent{Z}}
              \end{array}
            }
          }{\Psi_{{\mathrm{1}}}  \Ellesym{,}  \Ellent{Y_{{\mathrm{2}}}}  \Ellesym{,}  \Ellent{Y_{{\mathrm{1}}}}  \Ellesym{,}  \Psi_{{\mathrm{2}}}  \Ellesym{,}  \Phi  \Ellesym{,}  \Psi_{{\mathrm{3}}}  \vdash_\mathcal{C}  \Ellent{Z}}
        \end{math}
      \end{center}

  \item $\ElledruleSXXunitLOneName$ (with low priority) Case 1:
    \begin{center}
      \scriptsize
      \begin{math}
        \begin{array}{c}
          \Pi_1 \\
          {\Phi  \vdash_\mathcal{C}  \Ellent{X}}
        \end{array}
      \end{math}
      \qquad\qquad
      $\Pi_2$:
      \begin{math}
        $$\mprset{flushleft}
        \inferrule* [right={\tiny unitL1}] {
          {
            \begin{array}{c}
              \pi \\
              {\Gamma_{{\mathrm{1}}}  \Ellesym{,}  \Ellent{X}  \Ellesym{,}  \Gamma_{{\mathrm{2}}}  \vdash_\mathcal{L}  \Ellent{A}}
            \end{array}
          }
        }{ \mathsf{UnitT}   \Ellesym{,}  \Gamma_{{\mathrm{1}}}  \Ellesym{,}  \Ellent{X}  \Ellesym{,}  \Gamma_{{\mathrm{2}}}  \vdash_\mathcal{L}  \Ellent{A}}
      \end{math}
    \end{center}
    By assumption, $c(\Pi_1),c(\Pi_2)\leq |X|$. By induction on $\Pi_1$ and $\pi$, there is a
    proof $\Pi'$ for sequent $\Gamma_{{\mathrm{1}}}  \Ellesym{,}  \Phi  \Ellesym{,}  \Gamma_{{\mathrm{2}}}  \vdash_\mathcal{L}  \Ellent{A}$ s.t. $c(\Pi') \leq |X|$. Therefore, the
    proof $\Pi$ can be constructed as follows with $c(\Pi) = c(\Pi') \leq |X|$.
    \begin{center}
      \scriptsize
      \begin{math}
        $$\mprset{flushleft}
        \inferrule* [right={\tiny unitL1}] {
          {
            \begin{array}{c}
              \Pi' \\
              {\Gamma_{{\mathrm{1}}}  \Ellesym{,}  \Phi  \Ellesym{,}  \Gamma_{{\mathrm{2}}}  \vdash_\mathcal{L}  \Ellent{A}}
            \end{array}
          }
        }{ \mathsf{UnitT}   \Ellesym{,}  \Gamma_{{\mathrm{1}}}  \Ellesym{,}  \Phi  \Ellesym{,}  \Gamma_{{\mathrm{2}}}  \vdash_\mathcal{L}  \Ellent{A}}
      \end{math}
    \end{center}

  \item $\ElledruleSXXunitLOneName$ (with low priority) Case 2:
    \begin{center}
      \scriptsize
      \begin{math}
        \begin{array}{c}
          \Pi_1 \\
          {\Delta  \vdash_\mathcal{L}  \Ellent{B}}
        \end{array}
      \end{math}
      \qquad\qquad
      $\Pi_2$:
      \begin{math}
        $$\mprset{flushleft}
        \inferrule* [right={\tiny unitL1}] {
          {
            \begin{array}{c}
              \pi \\
              {\Gamma_{{\mathrm{1}}}  \Ellesym{,}  \Ellent{B}  \Ellesym{,}  \Gamma_{{\mathrm{2}}}  \vdash_\mathcal{L}  \Ellent{A}}
            \end{array}
          }
        }{ \mathsf{UnitT}   \Ellesym{,}  \Gamma_{{\mathrm{1}}}  \Ellesym{,}  \Ellent{B}  \Ellesym{,}  \Gamma_{{\mathrm{2}}}  \vdash_\mathcal{L}  \Ellent{A}}
      \end{math}
    \end{center}
    By assumption, $c(\Pi_1),c(\Pi_2)\leq |B|$. By induction on $\Pi_1$ and $\pi$, there is a
    proof $\Pi'$ for sequent $\Gamma_{{\mathrm{1}}}  \Ellesym{,}  \Delta  \Ellesym{,}  \Gamma_{{\mathrm{2}}}  \vdash_\mathcal{L}  \Ellent{A}$ s.t. $c(\Pi') \leq |B|$. Therefore, the
    proof $\Pi$ can be constructed as follows with $c(\Pi) = c(\Pi') \leq |B|$.
    \begin{center}
      \scriptsize
      \begin{math}
        $$\mprset{flushleft}
        \inferrule* [right={\tiny unitL1}] {
          {
            \begin{array}{c}
              \Pi' \\
              {\Gamma_{{\mathrm{1}}}  \Ellesym{,}  \Delta  \Ellesym{,}  \Gamma_{{\mathrm{2}}}  \vdash_\mathcal{L}  \Ellent{A}}
            \end{array}
          }
        }{ \mathsf{UnitT}   \Ellesym{,}  \Gamma_{{\mathrm{1}}}  \Ellesym{,}  \Delta  \Ellesym{,}  \Gamma_{{\mathrm{2}}}  \vdash_\mathcal{L}  \Ellent{A}}
      \end{math}
    \end{center}

  \item $\ElledruleSXXunitLTwoName$ (with low priority) Case 1:
    \begin{center}
      \scriptsize
      \begin{math}
        \begin{array}{c}
          \Pi_1 \\
          {\Phi  \vdash_\mathcal{C}  \Ellent{X}}
        \end{array}
      \end{math}
      \qquad\qquad
      $\Pi_2$:
      \begin{math}
        $$\mprset{flushleft}
        \inferrule* [right={\tiny unitL2}] {
          {
            \begin{array}{c}
              \pi \\
              {\Gamma_{{\mathrm{1}}}  \Ellesym{,}  \Ellent{X}  \Ellesym{,}  \Gamma_{{\mathrm{2}}}  \vdash_\mathcal{L}  \Ellent{A}}
            \end{array}
          }
        }{ \mathsf{UnitS}   \Ellesym{,}  \Gamma_{{\mathrm{1}}}  \Ellesym{,}  \Ellent{X}  \Ellesym{,}  \Gamma_{{\mathrm{2}}}  \vdash_\mathcal{L}  \Ellent{A}}
      \end{math}
    \end{center}
    By assumption, $c(\Pi_1),c(\Pi_2)\leq |X|$. By induction on $\Pi_1$ and $\pi$, there is a
    proof $\Pi'$ for sequent $\Gamma_{{\mathrm{1}}}  \Ellesym{,}  \Phi  \Ellesym{,}  \Gamma_{{\mathrm{2}}}  \vdash_\mathcal{L}  \Ellent{A}$ s.t. $c(\Pi') \leq |X|$. Therefore, the
    proof $\Pi$ can be constructed as follows with $c(\Pi) = c(\Pi') \leq |X|$.
    \begin{center}
      \scriptsize
      \begin{math}
        $$\mprset{flushleft}
        \inferrule* [right={\tiny unitL2}] {
          {
            \begin{array}{c}
              \Pi' \\
              {\Gamma_{{\mathrm{1}}}  \Ellesym{,}  \Phi  \Ellesym{,}  \Gamma_{{\mathrm{2}}}  \vdash_\mathcal{L}  \Ellent{A}}
            \end{array}
          }
        }{ \mathsf{UnitS}   \Ellesym{,}  \Gamma_{{\mathrm{1}}}  \Ellesym{,}  \Phi  \Ellesym{,}  \Gamma_{{\mathrm{2}}}  \vdash_\mathcal{L}  \Ellent{A}}
      \end{math}
    \end{center}

  \item $\ElledruleSXXunitLTwoName$ (with low priority) Case 2:
    \begin{center}
      \scriptsize
      \begin{math}
        \begin{array}{c}
          \Pi_1 \\
          {\Delta  \vdash_\mathcal{L}  \Ellent{B}}
        \end{array}
      \end{math}
      \qquad\qquad
      $\Pi_2$:
      \begin{math}
        $$\mprset{flushleft}
        \inferrule* [right={\tiny unitL2}] {
          {
            \begin{array}{c}
              \pi \\
              {\Gamma_{{\mathrm{1}}}  \Ellesym{,}  \Ellent{B}  \Ellesym{,}  \Gamma_{{\mathrm{2}}}  \vdash_\mathcal{L}  \Ellent{A}}
            \end{array}
          }
        }{ \mathsf{UnitS}   \Ellesym{,}  \Gamma_{{\mathrm{1}}}  \Ellesym{,}  \Ellent{B}  \Ellesym{,}  \Gamma_{{\mathrm{2}}}  \vdash_\mathcal{L}  \Ellent{A}}
      \end{math}
    \end{center}
    By assumption, $c(\Pi_1),c(\Pi_2)\leq |B|$. By induction on $\Pi_1$ and $\pi$, there is a
    proof $\Pi'$ for sequent $\Gamma_{{\mathrm{1}}}  \Ellesym{,}  \Delta  \Ellesym{,}  \Gamma_{{\mathrm{2}}}  \vdash_\mathcal{L}  \Ellent{A}$ s.t. $c(\Pi') \leq |B|$. Therefore, the
    proof $\Pi$ can be constructed as follows with $c(\Pi) = c(\Pi') \leq |B|$.
    \begin{center}
      \scriptsize
      \begin{math}
        $$\mprset{flushleft}
        \inferrule* [right={\tiny unitL2}] {
          {
            \begin{array}{c}
              \Pi' \\
              {\Gamma_{{\mathrm{1}}}  \Ellesym{,}  \Delta  \Ellesym{,}  \Gamma_{{\mathrm{2}}}  \vdash_\mathcal{L}  \Ellent{A}}
            \end{array}
          }
        }{ \mathsf{UnitS}   \Ellesym{,}  \Gamma_{{\mathrm{1}}}  \Ellesym{,}  \Delta  \Ellesym{,}  \Gamma_{{\mathrm{2}}}  \vdash_\mathcal{L}  \Ellent{A}}
      \end{math}
    \end{center}

  \item $\ElledruleSXXtenRName$ Case 1:
      \begin{center}
        \scriptsize
        \begin{math}
          \begin{array}{c}
            \Pi_1 \\
            {\Phi  \vdash_\mathcal{C}  \Ellent{X}}
          \end{array}
        \end{math}
        \qquad\qquad
        $\Pi_2$:
        \begin{math}
          $$\mprset{flushleft}
          \inferrule* [right={\tiny tenR}] {
            {
              \begin{array}{cc}
                \pi_1 & \pi_2 \\
                {\Gamma_{{\mathrm{1}}}  \Ellesym{,}  \Ellent{X}  \Ellesym{,}  \Gamma_{{\mathrm{2}}}  \vdash_\mathcal{L}  \Ellent{A}} & {\Gamma_{{\mathrm{3}}}  \vdash_\mathcal{L}  \Ellent{B}}
              \end{array}
            }
          }{\Gamma_{{\mathrm{1}}}  \Ellesym{,}  \Ellent{X}  \Ellesym{,}  \Gamma_{{\mathrm{2}}}  \Ellesym{,}  \Gamma_{{\mathrm{3}}}  \vdash_\mathcal{L}  \Ellent{A}  \triangleright  \Ellent{B}}
        \end{math}
      \end{center}
      By assumption, $c(\Pi_1),c(\Pi_2)\leq |X|$. By induction on $\Pi_1$ and $\pi_1$, there
      is a proof $\Pi'$ for sequent $\Gamma_{{\mathrm{1}}}  \Ellesym{,}  \Phi  \Ellesym{,}  \Gamma_{{\mathrm{2}}}  \vdash_\mathcal{L}  \Ellent{A}$ s.t. $c(\Pi') \leq |X|$. Therefore,
      the proof $\Pi$ can be constructed as follows with $c(\Pi) = c(\Pi') \leq |X|$.
      \begin{center}
        \scriptsize
        \begin{math}
          $$\mprset{flushleft}
          \inferrule* [right={\tiny tenR}] {
            {
              \begin{array}{cc}
                \Pi' & \pi_1 \\
                {\Gamma_{{\mathrm{1}}}  \Ellesym{,}  \Phi  \Ellesym{,}  \Gamma_{{\mathrm{2}}}  \vdash_\mathcal{L}  \Ellent{A}} & {\Gamma_{{\mathrm{3}}}  \vdash_\mathcal{L}  \Ellent{B}}
              \end{array}
            }
          }{\Gamma_{{\mathrm{1}}}  \Ellesym{,}  \Phi  \Ellesym{,}  \Gamma_{{\mathrm{2}}}  \Ellesym{,}  \Gamma_{{\mathrm{3}}}  \vdash_\mathcal{L}  \Ellent{A}  \triangleright  \Ellent{B}}
        \end{math}
      \end{center}

  \item $\ElledruleSXXtenRName$ Case 2:
      \begin{center}
        \scriptsize
        \begin{math}
          \begin{array}{c}
            \Pi_1 \\
            {\Delta  \vdash_\mathcal{L}  \Ellent{C}}
          \end{array}
        \end{math}
        \qquad\qquad
        $\Pi_2$:
        \begin{math}
          $$\mprset{flushleft}
          \inferrule* [right={\tiny tenR}] {
            {
              \begin{array}{cc}
                \pi_1 & \pi_2 \\
                {\Gamma_{{\mathrm{1}}}  \Ellesym{,}  \Ellent{C}  \Ellesym{,}  \Gamma_{{\mathrm{2}}}  \vdash_\mathcal{L}  \Ellent{A}} & {\Gamma_{{\mathrm{3}}}  \vdash_\mathcal{L}  \Ellent{B}}
              \end{array}
            }
          }{\Gamma_{{\mathrm{1}}}  \Ellesym{,}  \Ellent{C}  \Ellesym{,}  \Gamma_{{\mathrm{2}}}  \Ellesym{,}  \Gamma_{{\mathrm{3}}}  \vdash_\mathcal{L}  \Ellent{A}  \triangleright  \Ellent{B}}
        \end{math}
      \end{center}
      By assumption, $c(\Pi_1),c(\Pi_2)\leq |C|$. By induction on $\Pi_1$ and $\pi_1$, there
      is a proof $\Pi'$ for sequent $\Gamma_{{\mathrm{1}}}  \Ellesym{,}  \Delta  \Ellesym{,}  \Gamma_{{\mathrm{2}}}  \vdash_\mathcal{L}  \Ellent{A}$ s.t. $c(\Pi') \leq |C|$. Therefore,
      the proof $\Pi$ can be constructed as follows with $c(\Pi) = c(\Pi') \leq |C|$.
      \begin{center}
        \scriptsize
        \begin{math}
          $$\mprset{flushleft}
          \inferrule* [right={\tiny tenR}] {
            {
              \begin{array}{cc}
                \Pi' & \pi_1 \\
                {\Gamma_{{\mathrm{1}}}  \Ellesym{,}  \Delta  \Ellesym{,}  \Gamma_{{\mathrm{2}}}  \vdash_\mathcal{L}  \Ellent{A}} & {\Gamma_{{\mathrm{3}}}  \vdash_\mathcal{L}  \Ellent{B}}
              \end{array}
            }
          }{\Gamma_{{\mathrm{1}}}  \Ellesym{,}  \Delta  \Ellesym{,}  \Gamma_{{\mathrm{2}}}  \Ellesym{,}  \Gamma_{{\mathrm{3}}}  \vdash_\mathcal{L}  \Ellent{A}  \triangleright  \Ellent{B}}
        \end{math}
      \end{center}

  \item $\ElledruleSXXtenRName$ Case 3:
      \begin{center}
        \scriptsize
        \begin{math}
          \begin{array}{c}
            \Pi_1 \\
            {\Phi  \vdash_\mathcal{C}  \Ellent{X}}
          \end{array}
        \end{math}
        \qquad\qquad
        $\Pi_2$:
        \begin{math}
          $$\mprset{flushleft}
          \inferrule* [right={\tiny tenR}] {
            {
              \begin{array}{cc}
                \pi_1 & \pi_2 \\
                {\Gamma_{{\mathrm{1}}}  \vdash_\mathcal{L}  \Ellent{A}} & {\Gamma_{{\mathrm{2}}}  \Ellesym{,}  \Ellent{X}  \Ellesym{,}  \Gamma_{{\mathrm{3}}}  \vdash_\mathcal{L}  \Ellent{B}}
              \end{array}
            }
          }{\Gamma_{{\mathrm{1}}}  \Ellesym{,}  \Gamma_{{\mathrm{2}}}  \Ellesym{,}  \Ellent{X}  \Ellesym{,}  \Gamma_{{\mathrm{3}}}  \vdash_\mathcal{L}  \Ellent{A}  \triangleright  \Ellent{B}}
        \end{math}
      \end{center}
      By assumption, $c(\Pi_1),c(\Pi_2)\leq |X|$. By induction on $\Pi_1$ and $\pi_2$, there
      is a proof $\Pi'$ for sequent $\Gamma_{{\mathrm{2}}}  \Ellesym{,}  \Phi  \Ellesym{,}  \Gamma_{{\mathrm{3}}}  \vdash_\mathcal{L}  \Ellent{B}$ s.t. $c(\Pi') \leq |X|$. Therefore,
      the proof $\Pi$ can be constructed as follows with $c(\Pi) = c(\Pi') \leq |X|$.
      \begin{center}
        \scriptsize
        \begin{math}
          $$\mprset{flushleft}
          \inferrule* [right={\tiny tenR}] {
            {
              \begin{array}{cc}
                \pi_1 & \Pi' \\
                {\Gamma_{{\mathrm{1}}}  \vdash_\mathcal{L}  \Ellent{A}} & {\Gamma_{{\mathrm{2}}}  \Ellesym{,}  \Phi  \Ellesym{,}  \Gamma_{{\mathrm{3}}}  \vdash_\mathcal{L}  \Ellent{B}}
              \end{array}
            }
          }{\Gamma_{{\mathrm{1}}}  \Ellesym{,}  \Gamma_{{\mathrm{2}}}  \Ellesym{,}  \Phi  \Ellesym{,}  \Gamma_{{\mathrm{3}}}  \vdash_\mathcal{L}  \Ellent{A}  \triangleright  \Ellent{B}}
        \end{math}
      \end{center}

  \item $\ElledruleSXXtenRName$ Case 4:
      \begin{center}
        \scriptsize
        \begin{math}
          \begin{array}{c}
            \Pi_1 \\
            {\Delta  \vdash_\mathcal{L}  \Ellent{C}}
          \end{array}
        \end{math}
        \qquad\qquad
        $\Pi_2$:
        \begin{math}
          $$\mprset{flushleft}
          \inferrule* [right={\tiny tenR}] {
            {
              \begin{array}{cc}
                \pi_1 & \pi_2 \\
                {\Gamma_{{\mathrm{1}}}  \vdash_\mathcal{L}  \Ellent{A}} & {\Gamma_{{\mathrm{2}}}  \Ellesym{,}  \Ellent{C}  \Ellesym{,}  \Gamma_{{\mathrm{3}}}  \vdash_\mathcal{L}  \Ellent{B}}
              \end{array}
            }
          }{\Gamma_{{\mathrm{1}}}  \Ellesym{,}  \Gamma_{{\mathrm{2}}}  \Ellesym{,}  \Ellent{C}  \Ellesym{,}  \Gamma_{{\mathrm{3}}}  \vdash_\mathcal{L}  \Ellent{A}  \triangleright  \Ellent{B}}
        \end{math}
      \end{center}
      By assumption, $c(\Pi_1),c(\Pi_2)\leq |C|$. By induction on $\Pi_1$ and $\pi_2$, there
      is a proof $\Pi'$ for sequent $\Gamma_{{\mathrm{2}}}  \Ellesym{,}  \Delta  \Ellesym{,}  \Gamma_{{\mathrm{3}}}  \vdash_\mathcal{L}  \Ellent{B}$ s.t. $c(\Pi') \leq |C|$. Therefore,
      the proof $\Pi$ can be constructed as follows with $c(\Pi) = c(\Pi') \leq |C|$.
      \begin{center}
        \scriptsize
        \begin{math}
          $$\mprset{flushleft}
          \inferrule* [right={\tiny tenR}] {
            {
              \begin{array}{cc}
                \pi_1 & \Pi' \\
                {\Gamma_{{\mathrm{1}}}  \vdash_\mathcal{L}  \Ellent{A}} & {\Gamma_{{\mathrm{2}}}  \Ellesym{,}  \Delta  \Ellesym{,}  \Gamma_{{\mathrm{3}}}  \vdash_\mathcal{L}  \Ellent{B}}
              \end{array}
            }
          }{\Gamma_{{\mathrm{1}}}  \Ellesym{,}  \Gamma_{{\mathrm{2}}}  \Ellesym{,}  \Delta  \Ellesym{,}  \Gamma_{{\mathrm{3}}}  \vdash_\mathcal{L}  \Ellent{A}  \triangleright  \Ellent{B}}
        \end{math}
      \end{center}

  \item $\ElledruleSXXtenLOneName$ (with low priority) Case 1:
      \begin{center}
        \scriptsize
        \begin{math}
          \begin{array}{c}
            \Pi_1 \\
            {\Phi  \vdash_\mathcal{C}  \Ellent{X}}
          \end{array}
        \end{math}
        \qquad\qquad
        $\Pi_2$:
        \begin{math}
          $$\mprset{flushleft}
          \inferrule* [right={\tiny tenL}] {
            {
              \begin{array}{c}
                \pi \\
                {\Gamma_{{\mathrm{1}}}  \Ellesym{,}  \Ellent{X}  \Ellesym{,}  \Gamma_{{\mathrm{2}}}  \Ellesym{,}  \Ellent{Y_{{\mathrm{1}}}}  \Ellesym{,}  \Ellent{Y_{{\mathrm{2}}}}  \Ellesym{,}  \Gamma_{{\mathrm{3}}}  \vdash_\mathcal{L}  \Ellent{A}}
              \end{array}
            }
          }{\Gamma_{{\mathrm{1}}}  \Ellesym{,}  \Ellent{X}  \Ellesym{,}  \Gamma_{{\mathrm{2}}}  \Ellesym{,}  \Ellent{Y_{{\mathrm{1}}}}  \otimes  \Ellent{Y_{{\mathrm{2}}}}  \Ellesym{,}  \Gamma_{{\mathrm{3}}}  \vdash_\mathcal{L}  \Ellent{A}}
        \end{math}
      \end{center}
      By assumption, $c(\Pi_1),c(\Pi_2)\leq |X|$. By induction on $\Pi_1$ and $\pi$, there is
      a proof $\Pi'$ for sequent $\Gamma_{{\mathrm{1}}}  \Ellesym{,}  \Phi  \Ellesym{,}  \Gamma_{{\mathrm{2}}}  \Ellesym{,}  \Ellent{Y_{{\mathrm{1}}}}  \Ellesym{,}  \Ellent{Y_{{\mathrm{2}}}}  \Ellesym{,}  \Gamma_{{\mathrm{3}}}  \vdash_\mathcal{L}  \Ellent{A}$ s.t. $c(\Pi') \leq |X|$.
      Therefore, the proof $\Pi$ can be constructed as follows with
      $c(\Pi) = c(\Pi') \leq |X|$.
      \begin{center}
        \scriptsize
        \begin{math}
          $$\mprset{flushleft}
          \inferrule* [right={\tiny tenL}] {
            {
              \begin{array}{c}
                \Pi' \\
                {\Gamma_{{\mathrm{1}}}  \Ellesym{,}  \Phi  \Ellesym{,}  \Gamma_{{\mathrm{2}}}  \Ellesym{,}  \Ellent{Y_{{\mathrm{1}}}}  \Ellesym{,}  \Ellent{Y_{{\mathrm{2}}}}  \Ellesym{,}  \Gamma_{{\mathrm{3}}}  \vdash_\mathcal{L}  \Ellent{A}}
              \end{array}
            }
          }{\Gamma_{{\mathrm{1}}}  \Ellesym{,}  \Phi  \Ellesym{,}  \Gamma_{{\mathrm{2}}}  \Ellesym{,}  \Ellent{Y_{{\mathrm{1}}}}  \otimes  \Ellent{Y_{{\mathrm{2}}}}  \Ellesym{,}  \Gamma_{{\mathrm{3}}}  \vdash_\mathcal{L}  \Ellent{A}}
        \end{math}
      \end{center}

  \item $\ElledruleSXXtenLOneName$ (with low priority) Case 2:
      \begin{center}
        \scriptsize
        \begin{math}
          \begin{array}{c}
            \Pi_1 \\
            {\Delta  \vdash_\mathcal{L}  \Ellent{B}}
          \end{array}
        \end{math}
        \qquad\qquad
        $\Pi_2$:
        \begin{math}
          $$\mprset{flushleft}
          \inferrule* [right={\tiny tenL}] {
            {
              \begin{array}{c}
                \pi \\
                {\Gamma_{{\mathrm{1}}}  \Ellesym{,}  \Ellent{B}  \Ellesym{,}  \Gamma_{{\mathrm{2}}}  \Ellesym{,}  \Ellent{Y_{{\mathrm{1}}}}  \Ellesym{,}  \Ellent{Y_{{\mathrm{2}}}}  \Ellesym{,}  \Gamma_{{\mathrm{3}}}  \vdash_\mathcal{L}  \Ellent{A}}
              \end{array}
            }
          }{\Gamma_{{\mathrm{1}}}  \Ellesym{,}  \Ellent{B}  \Ellesym{,}  \Gamma_{{\mathrm{2}}}  \Ellesym{,}  \Ellent{Y_{{\mathrm{1}}}}  \otimes  \Ellent{Y_{{\mathrm{2}}}}  \Ellesym{,}  \Gamma_{{\mathrm{3}}}  \vdash_\mathcal{L}  \Ellent{A}}
        \end{math}
      \end{center}
      By assumption, $c(\Pi_1),c(\Pi_2)\leq |B|$. By induction on $\Pi_1$ and $\pi$, there is
      a proof $\Pi'$ for sequent $\Gamma_{{\mathrm{1}}}  \Ellesym{,}  \Ellent{B}  \Ellesym{,}  \Gamma_{{\mathrm{2}}}  \Ellesym{,}  \Ellent{Y_{{\mathrm{1}}}}  \Ellesym{,}  \Ellent{Y_{{\mathrm{2}}}}  \Ellesym{,}  \Gamma_{{\mathrm{3}}}  \vdash_\mathcal{L}  \Ellent{A}$ s.t. $c(\Pi') \leq |B|$.
      Therefore, the proof $\Pi$ can be constructed as follows with
      $c(\Pi) = c(\Pi') \leq |B|$.
      \begin{center}
        \scriptsize
        \begin{math}
          $$\mprset{flushleft}
          \inferrule* [right={\tiny tenL}] {
            {
              \begin{array}{c}
                \Pi' \\
                {\Gamma_{{\mathrm{1}}}  \Ellesym{,}  \Delta  \Ellesym{,}  \Gamma_{{\mathrm{2}}}  \Ellesym{,}  \Ellent{Y_{{\mathrm{1}}}}  \Ellesym{,}  \Ellent{Y_{{\mathrm{2}}}}  \Ellesym{,}  \Gamma_{{\mathrm{3}}}  \vdash_\mathcal{L}  \Ellent{A}}
              \end{array}
            }
          }{\Gamma_{{\mathrm{1}}}  \Ellesym{,}  \Delta  \Ellesym{,}  \Gamma_{{\mathrm{2}}}  \Ellesym{,}  \Ellent{Y_{{\mathrm{1}}}}  \otimes  \Ellent{Y_{{\mathrm{2}}}}  \Ellesym{,}  \Gamma_{{\mathrm{3}}}  \vdash_\mathcal{L}  \Ellent{A}}
        \end{math}
      \end{center}

  \item $\ElledruleSXXtenLOneName$ (with low priority) Case 3:
      \begin{center}
        \scriptsize
        \begin{math}
          \begin{array}{c}
            \Pi_1 \\
            {\Phi  \vdash_\mathcal{C}  \Ellent{X}}
          \end{array}
        \end{math}
        \qquad\qquad
        $\Pi_2$:
        \begin{math}
          $$\mprset{flushleft}
          \inferrule* [right={\tiny tenL}] {
            {
              \begin{array}{c}
                \pi \\
                {\Gamma_{{\mathrm{1}}}  \Ellesym{,}  \Ellent{Y_{{\mathrm{1}}}}  \Ellesym{,}  \Ellent{Y_{{\mathrm{2}}}}  \Ellesym{,}  \Gamma_{{\mathrm{2}}}  \Ellesym{,}  \Ellent{X}  \Ellesym{,}  \Gamma_{{\mathrm{3}}}  \vdash_\mathcal{L}  \Ellent{A}}
              \end{array}
            }
          }{\Gamma_{{\mathrm{1}}}  \Ellesym{,}  \Ellent{Y_{{\mathrm{1}}}}  \otimes  \Ellent{Y_{{\mathrm{2}}}}  \Ellesym{,}  \Gamma_{{\mathrm{2}}}  \Ellesym{,}  \Ellent{X}  \Ellesym{,}  \Gamma_{{\mathrm{3}}}  \vdash_\mathcal{L}  \Ellent{A}}
        \end{math}
      \end{center}
      By assumption, $c(\Pi_1),c(\Pi_2)\leq |X|$. By induction on $\Pi_1$ and $\pi$, there is
      a proof $\Pi'$ for sequent $\Gamma_{{\mathrm{1}}}  \Ellesym{,}  \Ellent{Y_{{\mathrm{1}}}}  \Ellesym{,}  \Ellent{Y_{{\mathrm{2}}}}  \Ellesym{,}  \Gamma_{{\mathrm{2}}}  \Ellesym{,}  \Phi  \Ellesym{,}  \Gamma_{{\mathrm{3}}}  \vdash_\mathcal{L}  \Ellent{A}$ s.t. $c(\Pi') \leq |X|$.
      Therefore, the proof $\Pi$ can be constructed as follows with
      $c(\Pi) = c(\Pi') \leq |X|$.
      \begin{center}
        \scriptsize
        \begin{math}
          $$\mprset{flushleft}
          \inferrule* [right={\tiny tenL}] {
            {
              \begin{array}{c}
                \Pi' \\
                {\Gamma_{{\mathrm{1}}}  \Ellesym{,}  \Ellent{Y_{{\mathrm{1}}}}  \Ellesym{,}  \Ellent{Y_{{\mathrm{2}}}}  \Ellesym{,}  \Gamma_{{\mathrm{2}}}  \Ellesym{,}  \Phi  \Ellesym{,}  \Gamma_{{\mathrm{3}}}  \vdash_\mathcal{L}  \Ellent{A}}
              \end{array}
            }
          }{\Gamma_{{\mathrm{1}}}  \Ellesym{,}  \Ellent{Y_{{\mathrm{1}}}}  \otimes  \Ellent{Y_{{\mathrm{2}}}}  \Ellesym{,}  \Gamma_{{\mathrm{2}}}  \Ellesym{,}  \Phi  \Ellesym{,}  \Gamma_{{\mathrm{3}}}  \vdash_\mathcal{L}  \Ellent{A}}
        \end{math}
      \end{center}

  \item $\ElledruleSXXtenLOneName$ (with low priority) Case 4:
      \begin{center}
        \scriptsize
        \begin{math}
          \begin{array}{c}
            \Pi_1 \\
            {\Delta  \vdash_\mathcal{L}  \Ellent{B}}
          \end{array}
        \end{math}
        \qquad\qquad
        $\Pi_2$:
        \begin{math}
          $$\mprset{flushleft}
          \inferrule* [right={\tiny tenL}] {
            {
              \begin{array}{c}
                \pi \\
                {\Gamma_{{\mathrm{1}}}  \Ellesym{,}  \Ellent{Y_{{\mathrm{1}}}}  \Ellesym{,}  \Ellent{Y_{{\mathrm{2}}}}  \Ellesym{,}  \Gamma_{{\mathrm{2}}}  \Ellesym{,}  \Ellent{B}  \Ellesym{,}  \Gamma_{{\mathrm{3}}}  \vdash_\mathcal{L}  \Ellent{A}}
              \end{array}
            }
          }{\Gamma_{{\mathrm{1}}}  \Ellesym{,}  \Ellent{Y_{{\mathrm{1}}}}  \otimes  \Ellent{Y_{{\mathrm{2}}}}  \Ellesym{,}  \Gamma_{{\mathrm{2}}}  \Ellesym{,}  \Ellent{B}  \Ellesym{,}  \Gamma_{{\mathrm{3}}}  \vdash_\mathcal{L}  \Ellent{A}}
        \end{math}
      \end{center}
      By assumption, $c(\Pi_1),c(\Pi_2)\leq |B|$. By induction on $\Pi_1$ and $\pi$, there is
      a proof $\Pi'$ for sequent $\Gamma_{{\mathrm{1}}}  \Ellesym{,}  \Ellent{Y_{{\mathrm{1}}}}  \Ellesym{,}  \Ellent{Y_{{\mathrm{2}}}}  \Ellesym{,}  \Gamma_{{\mathrm{2}}}  \Ellesym{,}  \Delta  \Ellesym{,}  \Gamma_{{\mathrm{3}}}  \vdash_\mathcal{L}  \Ellent{A}$ s.t. $c(\Pi') \leq |B|$.
      Therefore, the proof $\Pi$ can be constructed as follows with
      $c(\Pi) = c(\Pi') \leq |B|$.
      \begin{center}
        \scriptsize
        \begin{math}
          $$\mprset{flushleft}
          \inferrule* [right={\tiny tenL}] {
            {
              \begin{array}{c}
                \Pi' \\
                {\Gamma_{{\mathrm{1}}}  \Ellesym{,}  \Ellent{Y_{{\mathrm{1}}}}  \Ellesym{,}  \Ellent{Y_{{\mathrm{2}}}}  \Ellesym{,}  \Gamma_{{\mathrm{2}}}  \Ellesym{,}  \Delta  \Ellesym{,}  \Gamma_{{\mathrm{3}}}  \vdash_\mathcal{L}  \Ellent{A}}
              \end{array}
            }
          }{\Gamma_{{\mathrm{1}}}  \Ellesym{,}  \Ellent{Y_{{\mathrm{1}}}}  \otimes  \Ellent{Y_{{\mathrm{2}}}}  \Ellesym{,}  \Gamma_{{\mathrm{2}}}  \Ellesym{,}  \Delta  \Ellesym{,}  \Gamma_{{\mathrm{3}}}  \vdash_\mathcal{L}  \Ellent{A}}
        \end{math}
      \end{center}

  \item $\ElledruleSXXtenLTwoName$ (with low priority) Case 1:
      \begin{center}
        \scriptsize
        \begin{math}
          \begin{array}{c}
            \Pi_1 \\
            {\Phi  \vdash_\mathcal{C}  \Ellent{X}}
          \end{array}
        \end{math}
        \qquad\qquad
        $\Pi_2$:
        \begin{math}
          $$\mprset{flushleft}
          \inferrule* [right={\tiny tenL}] {
            {
              \begin{array}{c}
                \pi \\
                {\Gamma_{{\mathrm{1}}}  \Ellesym{,}  \Ellent{X}  \Ellesym{,}  \Gamma_{{\mathrm{2}}}  \Ellesym{,}  \Ellent{A_{{\mathrm{1}}}}  \Ellesym{,}  \Ellent{A_{{\mathrm{2}}}}  \Ellesym{,}  \Gamma_{{\mathrm{3}}}  \vdash_\mathcal{L}  \Ellent{B}}
              \end{array}
            }
          }{\Gamma_{{\mathrm{1}}}  \Ellesym{,}  \Ellent{X}  \Ellesym{,}  \Gamma_{{\mathrm{2}}}  \Ellesym{,}  \Ellent{A_{{\mathrm{1}}}}  \triangleright  \Ellent{A_{{\mathrm{2}}}}  \Ellesym{,}  \Gamma_{{\mathrm{3}}}  \vdash_\mathcal{L}  \Ellent{B}}
        \end{math}
      \end{center}
      By assumption, $c(\Pi_1),c(\Pi_2)\leq |X|$. By induction on $\Pi_1$ and $\pi$, there is
      a proof $\Pi'$ for sequent $\Gamma_{{\mathrm{1}}}  \Ellesym{,}  \Phi  \Ellesym{,}  \Gamma_{{\mathrm{2}}}  \Ellesym{,}  \Ellent{A_{{\mathrm{1}}}}  \Ellesym{,}  \Ellent{A_{{\mathrm{2}}}}  \Ellesym{,}  \Gamma_{{\mathrm{3}}}  \vdash_\mathcal{L}  \Ellent{B}$ s.t. $c(\Pi') \leq |X|$.
      Therefore, the proof $\Pi$ can be constructed as follows with
      $c(\Pi) = c(\Pi') \leq |X|$.
      \begin{center}
        \scriptsize
        \begin{math}
          $$\mprset{flushleft}
          \inferrule* [right={\tiny tenL}] {
            {
              \begin{array}{c}
                \Pi' \\
                {\Gamma_{{\mathrm{1}}}  \Ellesym{,}  \Phi  \Ellesym{,}  \Gamma_{{\mathrm{2}}}  \Ellesym{,}  \Ellent{A_{{\mathrm{1}}}}  \Ellesym{,}  \Ellent{A_{{\mathrm{2}}}}  \Ellesym{,}  \Gamma_{{\mathrm{3}}}  \vdash_\mathcal{L}  \Ellent{B}}
              \end{array}
            }
          }{\Gamma_{{\mathrm{1}}}  \Ellesym{,}  \Phi  \Ellesym{,}  \Gamma_{{\mathrm{2}}}  \Ellesym{,}  \Ellent{A_{{\mathrm{1}}}}  \triangleright  \Ellent{A_{{\mathrm{2}}}}  \Ellesym{,}  \Gamma_{{\mathrm{3}}}  \vdash_\mathcal{L}  \Ellent{B}}
        \end{math}
      \end{center}

  \item $\ElledruleSXXtenLTwoName$ (with low priority) Case 2:
      \begin{center}
        \scriptsize
        \begin{math}
          \begin{array}{c}
            \Pi_1 \\
            {\Delta  \vdash_\mathcal{L}  \Ellent{B}}
          \end{array}
        \end{math}
        \qquad\qquad
        $\Pi_2$:
        \begin{math}
          $$\mprset{flushleft}
          \inferrule* [right={\tiny tenL}] {
            {
              \begin{array}{c}
                \pi \\
                {\Gamma_{{\mathrm{1}}}  \Ellesym{,}  \Ellent{B}  \Ellesym{,}  \Gamma_{{\mathrm{2}}}  \Ellesym{,}  \Ellent{A_{{\mathrm{1}}}}  \Ellesym{,}  \Ellent{A_{{\mathrm{2}}}}  \Ellesym{,}  \Gamma_{{\mathrm{3}}}  \vdash_\mathcal{L}  \Ellent{C}}
              \end{array}
            }
          }{\Gamma_{{\mathrm{1}}}  \Ellesym{,}  \Ellent{B}  \Ellesym{,}  \Gamma_{{\mathrm{2}}}  \Ellesym{,}  \Ellent{A_{{\mathrm{1}}}}  \triangleright  \Ellent{A_{{\mathrm{2}}}}  \Ellesym{,}  \Gamma_{{\mathrm{3}}}  \vdash_\mathcal{L}  \Ellent{C}}
        \end{math}
      \end{center}
      By assumption, $c(\Pi_1),c(\Pi_2)\leq |B|$. By induction on $\Pi_1$ and $\pi$, there is
      a proof $\Pi'$ for sequent $\Gamma_{{\mathrm{1}}}  \Ellesym{,}  \Delta  \Ellesym{,}  \Gamma_{{\mathrm{2}}}  \Ellesym{,}  \Ellent{A_{{\mathrm{1}}}}  \Ellesym{,}  \Ellent{A_{{\mathrm{2}}}}  \Ellesym{,}  \Gamma_{{\mathrm{3}}}  \vdash_\mathcal{L}  \Ellent{C}$ s.t. $c(\Pi') \leq |B|$.
      Therefore, the proof $\Pi$ can be constructed as follows with
      $c(\Pi) = c(\Pi') \leq |B|$.
      \begin{center}
        \scriptsize
        \begin{math}
          $$\mprset{flushleft}
          \inferrule* [right={\tiny tenL}] {
            {
              \begin{array}{c}
                \Pi' \\
                {\Gamma_{{\mathrm{1}}}  \Ellesym{,}  \Delta  \Ellesym{,}  \Gamma_{{\mathrm{2}}}  \Ellesym{,}  \Ellent{A_{{\mathrm{1}}}}  \Ellesym{,}  \Ellent{A_{{\mathrm{2}}}}  \Ellesym{,}  \Gamma_{{\mathrm{3}}}  \vdash_\mathcal{L}  \Ellent{C}}
              \end{array}
            }
          }{\Gamma_{{\mathrm{1}}}  \Ellesym{,}  \Delta  \Ellesym{,}  \Gamma_{{\mathrm{2}}}  \Ellesym{,}  \Ellent{A_{{\mathrm{1}}}}  \triangleright  \Ellent{A_{{\mathrm{2}}}}  \Ellesym{,}  \Gamma_{{\mathrm{3}}}  \vdash_\mathcal{L}  \Ellent{C}}
        \end{math}
      \end{center}

  \item $\ElledruleSXXtenLTwoName$ (with low priority) Case 3:
      \begin{center}
        \scriptsize
        \begin{math}
          \begin{array}{c}
            \Pi_1 \\
            {\Phi  \vdash_\mathcal{C}  \Ellent{X}}
          \end{array}
        \end{math}
        \qquad\qquad
        $\Pi_2$:
        \begin{math}
          $$\mprset{flushleft}
          \inferrule* [right={\tiny tenL}] {
            {
              \begin{array}{c}
                \pi \\
                {\Gamma_{{\mathrm{1}}}  \Ellesym{,}  \Ellent{A_{{\mathrm{1}}}}  \Ellesym{,}  \Ellent{A_{{\mathrm{2}}}}  \Ellesym{,}  \Gamma_{{\mathrm{2}}}  \Ellesym{,}  \Ellent{X}  \Ellesym{,}  \Gamma_{{\mathrm{3}}}  \vdash_\mathcal{L}  \Ellent{B}}
              \end{array}
            }
          }{\Gamma_{{\mathrm{1}}}  \Ellesym{,}  \Ellent{A_{{\mathrm{1}}}}  \triangleright  \Ellent{A_{{\mathrm{2}}}}  \Ellesym{,}  \Gamma_{{\mathrm{2}}}  \Ellesym{,}  \Ellent{X}  \Ellesym{,}  \Gamma_{{\mathrm{3}}}  \vdash_\mathcal{L}  \Ellent{B}}
        \end{math}
      \end{center}
      By assumption, $c(\Pi_1),c(\Pi_2)\leq |X|$. By induction on $\Pi_1$ and $\pi$, there is
      a proof $\Pi'$ for sequent $\Gamma_{{\mathrm{1}}}  \Ellesym{,}  \Ellent{A_{{\mathrm{1}}}}  \Ellesym{,}  \Ellent{A_{{\mathrm{2}}}}  \Ellesym{,}  \Gamma_{{\mathrm{2}}}  \Ellesym{,}  \Phi  \Ellesym{,}  \Gamma_{{\mathrm{3}}}  \vdash_\mathcal{L}  \Ellent{A}$ s.t. $c(\Pi') \leq |X|$.
      Therefore, the proof $\Pi$ can be constructed as follows with
      $c(\Pi) = c(\Pi') \leq |X|$.
      \begin{center}
        \scriptsize
        \begin{math}
          $$\mprset{flushleft}
          \inferrule* [right={\tiny tenL}] {
            {
              \begin{array}{c}
                \Pi' \\
                {\Gamma_{{\mathrm{1}}}  \Ellesym{,}  \Ellent{A_{{\mathrm{1}}}}  \Ellesym{,}  \Ellent{A_{{\mathrm{2}}}}  \Ellesym{,}  \Gamma_{{\mathrm{2}}}  \Ellesym{,}  \Phi  \Ellesym{,}  \Gamma_{{\mathrm{3}}}  \vdash_\mathcal{L}  \Ellent{B}}
              \end{array}
            }
          }{\Gamma_{{\mathrm{1}}}  \Ellesym{,}  \Ellent{A_{{\mathrm{1}}}}  \triangleright  \Ellent{A_{{\mathrm{2}}}}  \Ellesym{,}  \Gamma_{{\mathrm{2}}}  \Ellesym{,}  \Phi  \Ellesym{,}  \Gamma_{{\mathrm{3}}}  \vdash_\mathcal{L}  \Ellent{B}}
        \end{math}
      \end{center}

  \item $\ElledruleSXXtenLTwoName$ (with low priority) Case 4:
      \begin{center}
        \scriptsize
        \begin{math}
          \begin{array}{c}
            \Pi_1 \\
            {\Delta  \vdash_\mathcal{L}  \Ellent{B}}
          \end{array}
        \end{math}
        \qquad\qquad
        $\Pi_2$:
        \begin{math}
          $$\mprset{flushleft}
          \inferrule* [right={\tiny tenL}] {
            {
              \begin{array}{c}
                \pi \\
                {\Gamma_{{\mathrm{1}}}  \Ellesym{,}  \Ellent{A_{{\mathrm{1}}}}  \Ellesym{,}  \Ellent{A_{{\mathrm{2}}}}  \Ellesym{,}  \Gamma_{{\mathrm{2}}}  \Ellesym{,}  \Ellent{B}  \Ellesym{,}  \Gamma_{{\mathrm{3}}}  \vdash_\mathcal{L}  \Ellent{C}}
              \end{array}
            }
          }{\Gamma_{{\mathrm{1}}}  \Ellesym{,}  \Ellent{A_{{\mathrm{1}}}}  \triangleright  \Ellent{A_{{\mathrm{2}}}}  \Ellesym{,}  \Gamma_{{\mathrm{2}}}  \Ellesym{,}  \Ellent{B}  \Ellesym{,}  \Gamma_{{\mathrm{3}}}  \vdash_\mathcal{L}  \Ellent{C}}
        \end{math}
      \end{center}
      By assumption, $c(\Pi_1),c(\Pi_2)\leq |B|$. By induction on $\Pi_1$ and $\pi$, there is
      a proof $\Pi'$ for sequent $\Gamma_{{\mathrm{1}}}  \Ellesym{,}  \Ellent{A_{{\mathrm{1}}}}  \Ellesym{,}  \Ellent{A_{{\mathrm{2}}}}  \Ellesym{,}  \Gamma_{{\mathrm{2}}}  \Ellesym{,}  \Delta  \Ellesym{,}  \Gamma_{{\mathrm{3}}}  \vdash_\mathcal{L}  \Ellent{C}$ s.t. $c(\Pi') \leq |B|$.
      Therefore, the proof $\Pi$ can be constructed as follows with
      $c(\Pi) = c(\Pi') \leq |B|$.
      \begin{center}
        \scriptsize
        \begin{math}
          $$\mprset{flushleft}
          \inferrule* [right={\tiny tenL}] {
            {
              \begin{array}{c}
                \Pi' \\
                {\Gamma_{{\mathrm{1}}}  \Ellesym{,}  \Ellent{A_{{\mathrm{1}}}}  \Ellesym{,}  \Ellent{A_{{\mathrm{2}}}}  \Ellesym{,}  \Gamma_{{\mathrm{2}}}  \Ellesym{,}  \Delta  \Ellesym{,}  \Gamma_{{\mathrm{3}}}  \vdash_\mathcal{L}  \Ellent{C}}
              \end{array}
            }
          }{\Gamma_{{\mathrm{1}}}  \Ellesym{,}  \Ellent{A_{{\mathrm{1}}}}  \triangleright  \Ellent{A_{{\mathrm{2}}}}  \Ellesym{,}  \Gamma_{{\mathrm{2}}}  \Ellesym{,}  \Delta  \Ellesym{,}  \Gamma_{{\mathrm{3}}}  \vdash_\mathcal{L}  \Ellent{C}}
        \end{math}
      \end{center}

  \item $\ElledruleSXXimprRName$ (with low priority) Case 1:
      \begin{center}
        \scriptsize
        \begin{math}
          \begin{array}{c}
            \Pi_1 \\
            {\Phi  \vdash_\mathcal{C}  \Ellent{X}}
          \end{array}
        \end{math}
        \qquad\qquad
        $\Pi_2$:
        \begin{math}
          $$\mprset{flushleft}
          \inferrule* [right={\tiny impR}] {
            {
              \begin{array}{c}
                \pi \\
                {\Gamma_{{\mathrm{1}}}  \Ellesym{,}  \Ellent{X}  \Ellesym{,}  \Gamma_{{\mathrm{2}}}  \Ellesym{,}  \Ellent{A}  \vdash_\mathcal{L}  \Ellent{B}}
              \end{array}
            }
          }{\Gamma_{{\mathrm{1}}}  \Ellesym{,}  \Ellent{X}  \Ellesym{,}  \Gamma_{{\mathrm{2}}}  \vdash_\mathcal{L}  \Ellent{A}  \rightharpoonup  \Ellent{B}}
        \end{math}
      \end{center}
      By assumption, $c(\Pi_1),c(\Pi_2)\leq |X|$. By induction on $\Pi_1$ and $\pi$, there
      is a proof $\Pi'$ for sequent $\Gamma_{{\mathrm{1}}}  \Ellesym{,}  \Phi  \Ellesym{,}  \Gamma_{{\mathrm{2}}}  \Ellesym{,}  \Ellent{A}  \vdash_\mathcal{L}  \Ellent{B}$ s.t. $c(\Pi') \leq |X|$.
      Therefore, the proof $\Pi$ can be constructed as follows with
      $c(\Pi) = c(\Pi') \leq |X|$.
      \begin{center}
        \scriptsize
        \begin{math}
          $$\mprset{flushleft}
          \inferrule* [right={\tiny impR}] {
            {
              \begin{array}{c}
                \Pi' \\
                {\Gamma_{{\mathrm{1}}}  \Ellesym{,}  \Phi  \Ellesym{,}  \Gamma_{{\mathrm{2}}}  \Ellesym{,}  \Ellent{A}  \vdash_\mathcal{L}  \Ellent{B}}
              \end{array}
            }
          }{\Gamma_{{\mathrm{1}}}  \Ellesym{,}  \Phi  \Ellesym{,}  \Gamma_{{\mathrm{2}}}  \vdash_\mathcal{L}  \Ellent{A}  \rightharpoonup  \Ellent{B}}
        \end{math}
      \end{center}

  \item $\ElledruleSXXimprRName$ (with low priority) Case 2:
      \begin{center}
        \scriptsize
        \begin{math}
          \begin{array}{c}
            \Pi_1 \\
            {\Delta  \vdash_\mathcal{L}  \Ellent{C}}
          \end{array}
        \end{math}
        \qquad\qquad
        $\Pi_2$:
        \begin{math}
          $$\mprset{flushleft}
          \inferrule* [right={\tiny impR}] {
            {
              \begin{array}{c}
                \pi \\
                {\Gamma_{{\mathrm{1}}}  \Ellesym{,}  \Ellent{C}  \Ellesym{,}  \Gamma_{{\mathrm{2}}}  \Ellesym{,}  \Ellent{A}  \vdash_\mathcal{L}  \Ellent{B}}
              \end{array}
            }
          }{\Gamma_{{\mathrm{1}}}  \Ellesym{,}  \Ellent{C}  \Ellesym{,}  \Gamma_{{\mathrm{2}}}  \vdash_\mathcal{L}  \Ellent{A}  \rightharpoonup  \Ellent{B}}
        \end{math}
      \end{center}
      By assumption, $c(\Pi_1),c(\Pi_2)\leq |C|$. By induction on $\Pi_1$ and $\pi$, there
      is a proof $\Pi'$ for sequent $\Gamma_{{\mathrm{1}}}  \Ellesym{,}  \Delta  \Ellesym{,}  \Gamma_{{\mathrm{2}}}  \Ellesym{,}  \Ellent{A}  \vdash_\mathcal{L}  \Ellent{B}$ s.t. $c(\Pi') \leq |C|$.
      Therefore, the proof $\Pi$ can be constructed as follows with
      $c(\Pi) = c(\Pi') \leq |C|$.
      \begin{center}
        \scriptsize
        \begin{math}
          $$\mprset{flushleft}
          \inferrule* [right={\tiny impR}] {
            {
              \begin{array}{c}
                \Pi' \\
                {\Gamma_{{\mathrm{1}}}  \Ellesym{,}  \Delta  \Ellesym{,}  \Gamma_{{\mathrm{2}}}  \Ellesym{,}  \Ellent{A}  \vdash_\mathcal{L}  \Ellent{B}}
              \end{array}
            }
          }{\Gamma_{{\mathrm{1}}}  \Ellesym{,}  \Delta  \Ellesym{,}  \Gamma_{{\mathrm{2}}}  \vdash_\mathcal{L}  \Ellent{A}  \rightharpoonup  \Ellent{B}}
        \end{math}
      \end{center}

  \item $\ElledruleSXXimplRName$ (with low priority) Case 1:
      \begin{center}
        \scriptsize
        \begin{math}
          \begin{array}{c}
            \Pi_1 \\
            {\Phi  \vdash_\mathcal{C}  \Ellent{X}}
          \end{array}
        \end{math}
        \qquad\qquad
        $\Pi_2$:
        \begin{math}
          $$\mprset{flushleft}
          \inferrule* [right={\tiny impL}] {
            {
              \begin{array}{c}
                \pi \\
                {\Ellent{A}  \Ellesym{,}  \Gamma_{{\mathrm{1}}}  \Ellesym{,}  \Ellent{X}  \Ellesym{,}  \Gamma_{{\mathrm{2}}}  \vdash_\mathcal{L}  \Ellent{B}}
              \end{array}
            }
          }{\Gamma_{{\mathrm{1}}}  \Ellesym{,}  \Ellent{X}  \Ellesym{,}  \Gamma_{{\mathrm{2}}}  \vdash_\mathcal{L}  \Ellent{B}  \leftharpoonup  \Ellent{A}}
        \end{math}
      \end{center}
      By assumption, $c(\Pi_1),c(\Pi_2)\leq |X|$. By induction on $\Pi_1$ and $\pi$, there
      is a proof $\Pi'$ for sequent $\Ellent{A}  \Ellesym{,}  \Gamma_{{\mathrm{1}}}  \Ellesym{,}  \Phi  \Ellesym{,}  \Gamma_{{\mathrm{2}}}  \vdash_\mathcal{L}  \Ellent{B}$ s.t. $c(\Pi') \leq |X|$.
      Therefore, the proof $\Pi$ can be constructed as follows with
      $c(\Pi) = c(\Pi') \leq |X|$.
      \begin{center}
        \scriptsize
        \begin{math}
          $$\mprset{flushleft}
          \inferrule* [right={\tiny impR}] {
            {
              \begin{array}{c}
                \Pi' \\
                {\Ellent{A}  \Ellesym{,}  \Gamma_{{\mathrm{1}}}  \Ellesym{,}  \Phi  \Ellesym{,}  \Gamma_{{\mathrm{2}}}  \vdash_\mathcal{L}  \Ellent{B}}
              \end{array}
            }
          }{\Gamma_{{\mathrm{1}}}  \Ellesym{,}  \Phi  \Ellesym{,}  \Gamma_{{\mathrm{2}}}  \vdash_\mathcal{L}  \Ellent{B}  \leftharpoonup  \Ellent{A}}
        \end{math}
      \end{center}

  \item $\ElledruleSXXimplRName$ (with low priority) Case 2:
      \begin{center}
        \scriptsize
        \begin{math}
          \begin{array}{c}
            \Pi_1 \\
            {\Delta  \vdash_\mathcal{L}  \Ellent{C}}
          \end{array}
        \end{math}
        \qquad\qquad
        $\Pi_2$:
        \begin{math}
          $$\mprset{flushleft}
          \inferrule* [right={\tiny impR}] {
            {
              \begin{array}{c}
                \pi \\
                {\Ellent{A}  \Ellesym{,}  \Gamma_{{\mathrm{1}}}  \Ellesym{,}  \Ellent{C}  \Ellesym{,}  \Gamma_{{\mathrm{2}}}  \vdash_\mathcal{L}  \Ellent{B}}
              \end{array}
            }
          }{\Gamma_{{\mathrm{1}}}  \Ellesym{,}  \Ellent{C}  \Ellesym{,}  \Gamma_{{\mathrm{2}}}  \vdash_\mathcal{L}  \Ellent{B}  \leftharpoonup  \Ellent{A}}
        \end{math}
      \end{center}
      By assumption, $c(\Pi_1),c(\Pi_2)\leq |C|$. By induction on $\Pi_1$ and $\pi$, there
      is a proof $\Pi'$ for sequent $\Gamma_{{\mathrm{1}}}  \Ellesym{,}  \Delta  \Ellesym{,}  \Gamma_{{\mathrm{2}}}  \Ellesym{,}  \Ellent{A}  \vdash_\mathcal{L}  \Ellent{B}$ s.t. $c(\Pi') \leq |C|$.
      Therefore, the proof $\Pi$ can be constructed as follows with
      $c(\Pi) = c(\Pi') \leq |C|$.
      \begin{center}
        \scriptsize
        \begin{math}
          $$\mprset{flushleft}
          \inferrule* [right={\tiny impR}] {
            {
              \begin{array}{c}
                \Pi' \\
                {\Ellent{A}  \Ellesym{,}  \Gamma_{{\mathrm{1}}}  \Ellesym{,}  \Delta  \Ellesym{,}  \Gamma_{{\mathrm{2}}}  \vdash_\mathcal{L}  \Ellent{B}}
              \end{array}
            }
          }{\Gamma_{{\mathrm{1}}}  \Ellesym{,}  \Delta  \Ellesym{,}  \Gamma_{{\mathrm{2}}}  \vdash_\mathcal{L}  \Ellent{B}  \leftharpoonup  \Ellent{A}}
        \end{math}
      \end{center}

  \item $\ElledruleSXXimpLName$ Case 1:
      \begin{center}
        \scriptsize
        \begin{math}
          \begin{array}{c}
            \Pi_1 \\
            {\Phi  \vdash_\mathcal{C}  \Ellent{X}}
          \end{array}
        \end{math}
        \qquad\qquad
        $\Pi_2$:
        \begin{math}
          $$\mprset{flushleft}
          \inferrule* [right={\tiny impL}] {
            {
              \begin{array}{cc}
                \pi_1 & \pi_2 \\
                {\Psi_{{\mathrm{1}}}  \Ellesym{,}  \Ellent{X}  \Ellesym{,}  \Psi_{{\mathrm{2}}}  \vdash_\mathcal{C}  \Ellent{Y_{{\mathrm{1}}}}} & {\Gamma_{{\mathrm{1}}}  \Ellesym{,}  \Ellent{Y_{{\mathrm{2}}}}  \Ellesym{,}  \Gamma_{{\mathrm{2}}}  \vdash_\mathcal{L}  \Ellent{A}}
              \end{array}
            }
          }{\Gamma_{{\mathrm{1}}}  \Ellesym{,}  \Ellent{Y_{{\mathrm{1}}}}  \multimap  \Ellent{Y_{{\mathrm{2}}}}  \Ellesym{,}  \Psi_{{\mathrm{1}}}  \Ellesym{,}  \Ellent{X}  \Ellesym{,}  \Psi_{{\mathrm{2}}}  \Ellesym{,}  \Gamma_{{\mathrm{2}}}  \vdash_\mathcal{L}  \Ellent{A}}
        \end{math}
      \end{center}
      By assumption, $c(\Pi_1),c(\Pi_2)\leq |X|$. By induction on $\Pi_1$ and $\pi_1$, there is
      a proof $\Pi'$ for sequent $\Psi_{{\mathrm{1}}}  \Ellesym{,}  \Phi  \Ellesym{,}  \Psi_{{\mathrm{2}}}  \vdash_\mathcal{C}  \Ellent{Y_{{\mathrm{1}}}}$ s.t. $c(\Pi') \leq |X|$. Therefore, the
      proof $\Pi$ can be constructed as follows with $c(\Pi) = c(\Pi') \leq |X|$.
      \begin{center}
        \scriptsize
        \begin{math}
          $$\mprset{flushleft}
          \inferrule* [right={\tiny impL}] {
            {
              \begin{array}{cc}
                \Pi' & \pi_2 \\
                {\Psi_{{\mathrm{1}}}  \Ellesym{,}  \Phi  \Ellesym{,}  \Psi_{{\mathrm{2}}}  \vdash_\mathcal{C}  \Ellent{Y_{{\mathrm{1}}}}} & {\Gamma_{{\mathrm{1}}}  \Ellesym{,}  \Ellent{Y_{{\mathrm{2}}}}  \Ellesym{,}  \Gamma_{{\mathrm{2}}}  \vdash_\mathcal{L}  \Ellent{A}}
              \end{array}
            }
          }{\Gamma_{{\mathrm{1}}}  \Ellesym{,}  \Ellent{Y_{{\mathrm{1}}}}  \multimap  \Ellent{Y_{{\mathrm{2}}}}  \Ellesym{,}  \Psi_{{\mathrm{1}}}  \Ellesym{,}  \Phi  \Ellesym{,}  \Psi_{{\mathrm{2}}}  \Ellesym{,}  \Gamma_{{\mathrm{2}}}  \vdash_\mathcal{L}  \Ellent{A}}
        \end{math}
      \end{center}

  \item $\ElledruleSXXimpLName$ Case 2:
      \begin{center}
        \scriptsize
        \begin{math}
          \begin{array}{c}
            \Pi_1 \\
            {\Phi  \vdash_\mathcal{C}  \Ellent{X}}
          \end{array}
        \end{math}
        \qquad\qquad
        $\Pi_2$:
        \begin{math}
          $$\mprset{flushleft}
          \inferrule* [right={\tiny impL}] {
            {
              \begin{array}{cc}
                \pi_1 & \pi_2 \\
                {\Psi  \vdash_\mathcal{C}  \Ellent{Y_{{\mathrm{1}}}}} & {\Gamma_{{\mathrm{1}}}  \Ellesym{,}  \Ellent{X}  \Ellesym{,}  \Gamma_{{\mathrm{2}}}  \Ellesym{,}  \Ellent{Y_{{\mathrm{2}}}}  \Ellesym{,}  \Gamma_{{\mathrm{3}}}  \vdash_\mathcal{L}  \Ellent{A}}
              \end{array}
            }
          }{\Gamma_{{\mathrm{1}}}  \Ellesym{,}  \Ellent{X}  \Ellesym{,}  \Gamma_{{\mathrm{2}}}  \Ellesym{,}  \Ellent{Y_{{\mathrm{1}}}}  \multimap  \Ellent{Y_{{\mathrm{2}}}}  \Ellesym{,}  \Psi  \Ellesym{,}  \Gamma_{{\mathrm{3}}}  \vdash_\mathcal{L}  \Ellent{A}}
        \end{math}
      \end{center}
      By assumption, $c(\Pi_1),c(\Pi_2)\leq |X|$. By induction on $\Pi_1$ and $\pi_2$, there is
      a proof $\Pi'$ for sequent $\Gamma_{{\mathrm{1}}}  \Ellesym{,}  \Phi  \Ellesym{,}  \Gamma_{{\mathrm{2}}}  \Ellesym{,}  \Ellent{Y_{{\mathrm{2}}}}  \Ellesym{,}  \Gamma_{{\mathrm{3}}}  \vdash_\mathcal{L}  \Ellent{A}$ s.t. $c(\Pi') \leq |X|$.
      Therefore, the proof $\Pi$ can be constructed as follows with
      $c(\Pi) = c(\Pi') \leq |X|$.
      \begin{center}
        \scriptsize
        \begin{math}
          $$\mprset{flushleft}
          \inferrule* [right={\tiny impL}] {
            {
              \begin{array}{cc}
                \pi_1 & \Pi' \\
                {\Psi  \vdash_\mathcal{C}  \Ellent{Y_{{\mathrm{1}}}}} & {\Gamma_{{\mathrm{1}}}  \Ellesym{,}  \Phi  \Ellesym{,}  \Gamma_{{\mathrm{2}}}  \Ellesym{,}  \Ellent{Y_{{\mathrm{2}}}}  \Ellesym{,}  \Gamma_{{\mathrm{3}}}  \vdash_\mathcal{L}  \Ellent{A}}
              \end{array}
            }
          }{\Gamma_{{\mathrm{1}}}  \Ellesym{,}  \Phi  \Ellesym{,}  \Gamma_{{\mathrm{2}}}  \Ellesym{,}  \Ellent{Y_{{\mathrm{1}}}}  \multimap  \Ellent{Y_{{\mathrm{2}}}}  \Ellesym{,}  \Psi  \Ellesym{,}  \Gamma_{{\mathrm{3}}}  \vdash_\mathcal{L}  \Ellent{A}}
        \end{math}
      \end{center}

  \item $\ElledruleSXXimpLName$ Case 3:
      \begin{center}
        \scriptsize
        \begin{math}
          \begin{array}{c}
            \Pi_1 \\
            {\Delta  \vdash_\mathcal{L}  \Ellent{B}}
          \end{array}
        \end{math}
        \qquad\qquad
        $\Pi_2$:
        \begin{math}
          $$\mprset{flushleft}
          \inferrule* [right={\tiny impL}] {
            {
              \begin{array}{cc}
                \pi_1 & \pi_2 \\
                {\Psi  \vdash_\mathcal{C}  \Ellent{Y_{{\mathrm{1}}}}} & {\Gamma_{{\mathrm{1}}}  \Ellesym{,}  \Ellent{B}  \Ellesym{,}  \Gamma_{{\mathrm{2}}}  \Ellesym{,}  \Ellent{Y_{{\mathrm{2}}}}  \Ellesym{,}  \Gamma_{{\mathrm{3}}}  \vdash_\mathcal{L}  \Ellent{A}}
              \end{array}
            }
          }{\Gamma_{{\mathrm{1}}}  \Ellesym{,}  \Ellent{B}  \Ellesym{,}  \Gamma_{{\mathrm{2}}}  \Ellesym{,}  \Ellent{Y_{{\mathrm{1}}}}  \multimap  \Ellent{Y_{{\mathrm{2}}}}  \Ellesym{,}  \Psi  \Ellesym{,}  \Gamma_{{\mathrm{3}}}  \vdash_\mathcal{L}  \Ellent{A}}
        \end{math}
      \end{center}
      By assumption, $c(\Pi_1),c(\Pi_2)\leq |B|$. By induction on $\Pi_1$ and $\pi_2$, there is
      a proof $\Pi'$ for sequent $\Gamma_{{\mathrm{1}}}  \Ellesym{,}  \Delta  \Ellesym{,}  \Gamma_{{\mathrm{2}}}  \Ellesym{,}  \Ellent{Y_{{\mathrm{2}}}}  \Ellesym{,}  \Gamma_{{\mathrm{3}}}  \vdash_\mathcal{L}  \Ellent{A}$ s.t. $c(\Pi') \leq |B|$.
      Therefore, the proof $\Pi$ can be constructed as follows with
      $c(\Pi) = c(\Pi') \leq |B|$.
      \begin{center}
        \scriptsize
        \begin{math}
          $$\mprset{flushleft}
          \inferrule* [right={\tiny impL}] {
            {
              \begin{array}{cc}
                \pi_1 & \Pi' \\
                {\Psi  \vdash_\mathcal{C}  \Ellent{Y_{{\mathrm{1}}}}} & {\Gamma_{{\mathrm{1}}}  \Ellesym{,}  \Delta  \Ellesym{,}  \Gamma_{{\mathrm{2}}}  \Ellesym{,}  \Ellent{Y_{{\mathrm{2}}}}  \Ellesym{,}  \Gamma_{{\mathrm{3}}}  \vdash_\mathcal{L}  \Ellent{A}}
              \end{array}
            }
          }{\Gamma_{{\mathrm{1}}}  \Ellesym{,}  \Delta  \Ellesym{,}  \Gamma_{{\mathrm{2}}}  \Ellesym{,}  \Ellent{Y_{{\mathrm{1}}}}  \multimap  \Ellent{Y_{{\mathrm{2}}}}  \Ellesym{,}  \Psi  \Ellesym{,}  \Gamma_{{\mathrm{3}}}  \vdash_\mathcal{L}  \Ellent{A}}
        \end{math}
      \end{center}

  \item $\ElledruleSXXimpLName$ Case 4:
      \begin{center}
        \scriptsize
        \begin{math}
          \begin{array}{c}
            \Pi_1 \\
            {\Phi  \vdash_\mathcal{C}  \Ellent{X}}
          \end{array}
        \end{math}
        \qquad\qquad
        $\Pi_2$:
        \begin{math}
          $$\mprset{flushleft}
          \inferrule* [right={\tiny impL}] {
            {
              \begin{array}{cc}
                \pi_1 & \pi_2 \\
                {\Psi  \vdash_\mathcal{C}  \Ellent{Y_{{\mathrm{1}}}}} & {\Gamma_{{\mathrm{1}}}  \Ellesym{,}  \Ellent{Y_{{\mathrm{2}}}}  \Ellesym{,}  \Gamma_{{\mathrm{2}}}  \Ellesym{,}  \Ellent{X}  \Ellesym{,}  \Gamma_{{\mathrm{3}}}  \vdash_\mathcal{L}  \Ellent{A}}
              \end{array}
            }
          }{\Gamma_{{\mathrm{1}}}  \Ellesym{,}  \Ellent{Y_{{\mathrm{1}}}}  \multimap  \Ellent{Y_{{\mathrm{2}}}}  \Ellesym{,}  \Psi  \Ellesym{,}  \Gamma_{{\mathrm{2}}}  \Ellesym{,}  \Ellent{X}  \Ellesym{,}  \Gamma_{{\mathrm{3}}}  \vdash_\mathcal{L}  \Ellent{A}}
        \end{math}
      \end{center}
      By assumption, $c(\Pi_1),c(\Pi_2)\leq |X|$. By induction on $\Pi_1$ and $\pi_2$, there is
      a proof $\Pi'$ for sequent $\Gamma_{{\mathrm{1}}}  \Ellesym{,}  \Ellent{Y_{{\mathrm{2}}}}  \Ellesym{,}  \Gamma_{{\mathrm{2}}}  \Ellesym{,}  \Phi  \Ellesym{,}  \Gamma_{{\mathrm{3}}}  \vdash_\mathcal{L}  \Ellent{A}$ s.t. $c(\Pi') \leq |X|$.
      Therefore, the proof $\Pi$ can be constructed as follows with
      $c(\Pi) = c(\Pi') \leq |X|$.
      \begin{center}
        \scriptsize
        \begin{math}
          $$\mprset{flushleft}
          \inferrule* [right={\tiny impL}] {
            {
              \begin{array}{cc}
                \pi_1 & \Pi' \\
                {\Psi  \vdash_\mathcal{C}  \Ellent{Y_{{\mathrm{1}}}}} & {\Gamma_{{\mathrm{1}}}  \Ellesym{,}  \Ellent{Y_{{\mathrm{2}}}}  \Ellesym{,}  \Gamma_{{\mathrm{2}}}  \Ellesym{,}  \Phi  \Ellesym{,}  \Gamma_{{\mathrm{3}}}  \vdash_\mathcal{L}  \Ellent{A}}
              \end{array}
            }
          }{\Gamma_{{\mathrm{1}}}  \Ellesym{,}  \Ellent{Y_{{\mathrm{1}}}}  \multimap  \Ellent{Y_{{\mathrm{2}}}}  \Ellesym{,}  \Psi  \Ellesym{,}  \Gamma_{{\mathrm{2}}}  \Ellesym{,}  \Phi  \Ellesym{,}  \Gamma_{{\mathrm{3}}}  \vdash_\mathcal{L}  \Ellent{A}}
        \end{math}
      \end{center}

  \item $\ElledruleSXXimpLName$ Case 5:
      \begin{center}
        \scriptsize
        \begin{math}
          \begin{array}{c}
            \Pi_1 \\
            {\Delta  \vdash_\mathcal{L}  \Ellent{B}}
          \end{array}
        \end{math}
        \qquad\qquad
        $\Pi_2$:
        \begin{math}
          $$\mprset{flushleft}
          \inferrule* [right={\tiny impL}] {
            {
              \begin{array}{cc}
                \pi_1 & \pi_2 \\
                {\Psi  \vdash_\mathcal{C}  \Ellent{Y_{{\mathrm{1}}}}} & {\Gamma_{{\mathrm{1}}}  \Ellesym{,}  \Ellent{Y_{{\mathrm{2}}}}  \Ellesym{,}  \Gamma_{{\mathrm{2}}}  \Ellesym{,}  \Ellent{B}  \Ellesym{,}  \Gamma_{{\mathrm{3}}}  \vdash_\mathcal{L}  \Ellent{A}}
              \end{array}
            }
          }{\Gamma_{{\mathrm{1}}}  \Ellesym{,}  \Ellent{Y_{{\mathrm{1}}}}  \multimap  \Ellent{Y_{{\mathrm{2}}}}  \Ellesym{,}  \Psi  \Ellesym{,}  \Gamma_{{\mathrm{2}}}  \Ellesym{,}  \Ellent{B}  \Ellesym{,}  \Gamma_{{\mathrm{3}}}  \vdash_\mathcal{L}  \Ellent{A}}
        \end{math}
      \end{center}
      By assumption, $c(\Pi_1),c(\Pi_2)\leq |B|$. By induction on $\Pi_1$ and $\pi_2$, there is
      a proof $\Pi'$ for sequent $\Gamma_{{\mathrm{1}}}  \Ellesym{,}  \Ellent{Y_{{\mathrm{2}}}}  \Ellesym{,}  \Gamma_{{\mathrm{2}}}  \Ellesym{,}  \Delta  \Ellesym{,}  \Gamma_{{\mathrm{3}}}  \vdash_\mathcal{L}  \Ellent{A}$ s.t. $c(\Pi') \leq |B|$.
      Therefore, the proof $\Pi$ can be constructed as follows with
      $c(\Pi) = c(\Pi') \leq |B|$.
      \begin{center}
        \scriptsize
        \begin{math}
          $$\mprset{flushleft}
          \inferrule* [right={\tiny impL}] {
            {
              \begin{array}{cc}
                \pi_1 & \Pi' \\
                {\Psi  \vdash_\mathcal{C}  \Ellent{Y_{{\mathrm{1}}}}} & {\Gamma_{{\mathrm{1}}}  \Ellesym{,}  \Ellent{Y_{{\mathrm{2}}}}  \Ellesym{,}  \Gamma_{{\mathrm{2}}}  \Ellesym{,}  \Delta  \Ellesym{,}  \Gamma_{{\mathrm{3}}}  \vdash_\mathcal{L}  \Ellent{A}}
              \end{array}
            }
          }{\Gamma_{{\mathrm{1}}}  \Ellesym{,}  \Ellent{Y_{{\mathrm{1}}}}  \multimap  \Ellent{Y_{{\mathrm{2}}}}  \Ellesym{,}  \Psi  \Ellesym{,}  \Gamma_{{\mathrm{2}}}  \Ellesym{,}  \Delta  \Ellesym{,}  \Gamma_{{\mathrm{3}}}  \vdash_\mathcal{L}  \Ellent{A}}
        \end{math}
      \end{center}

  \item $\ElledruleSXXimprLName$ Case 1:
      \begin{center}
        \scriptsize
        \begin{math}
          \begin{array}{c}
            \Pi_1 \\
            {\Phi  \vdash_\mathcal{C}  \Ellent{X}}
          \end{array}
        \end{math}
        \qquad\qquad
        $\Pi_2$:
        \begin{math}
          $$\mprset{flushleft}
          \inferrule* [right={\tiny imprL}] {
            {
              \begin{array}{cc}
                \pi_1 & \pi_2 \\
                {\Delta_{{\mathrm{1}}}  \Ellesym{,}  \Ellent{X}  \Ellesym{,}  \Delta_{{\mathrm{2}}}  \vdash_\mathcal{L}  \Ellent{A_{{\mathrm{1}}}}} & {\Gamma_{{\mathrm{1}}}  \Ellesym{,}  \Ellent{A_{{\mathrm{2}}}}  \Ellesym{,}  \Gamma_{{\mathrm{2}}}  \vdash_\mathcal{L}  \Ellent{B}}
              \end{array}
            }
          }{\Gamma_{{\mathrm{1}}}  \Ellesym{,}  \Ellent{A_{{\mathrm{1}}}}  \rightharpoonup  \Ellent{A_{{\mathrm{2}}}}  \Ellesym{,}  \Delta_{{\mathrm{1}}}  \Ellesym{,}  \Ellent{X}  \Ellesym{,}  \Delta_{{\mathrm{2}}}  \Ellesym{,}  \Gamma_{{\mathrm{2}}}  \vdash_\mathcal{L}  \Ellent{B}}
        \end{math}
      \end{center}
      By assumption, $c(\Pi_1),c(\Pi_2)\leq |X|$. By induction on $\Pi_1$ and $\pi_1$, there is
      a proof $\Pi'$ for sequent $\Delta_{{\mathrm{1}}}  \Ellesym{,}  \Phi  \Ellesym{,}  \Delta_{{\mathrm{2}}}  \vdash_\mathcal{L}  \Ellent{A_{{\mathrm{1}}}}$ s.t. $c(\Pi') \leq |X|$. Therefore, the
      proof $\Pi$ can be constructed as follows with $c(\Pi) = c(\Pi') \leq |X|$.
      \begin{center}
        \scriptsize
        \begin{math}
          $$\mprset{flushleft}
          \inferrule* [right={\tiny impL}] {
            {
              \begin{array}{cc}
                \Pi' & \pi_2 \\
                {\Delta_{{\mathrm{1}}}  \Ellesym{,}  \Phi  \Ellesym{,}  \Delta_{{\mathrm{2}}}  \vdash_\mathcal{L}  \Ellent{A_{{\mathrm{1}}}}} & {\Gamma_{{\mathrm{1}}}  \Ellesym{,}  \Ellent{A_{{\mathrm{2}}}}  \Ellesym{,}  \Gamma_{{\mathrm{2}}}  \vdash_\mathcal{L}  \Ellent{B}}
              \end{array}
            }
          }{\Gamma_{{\mathrm{1}}}  \Ellesym{,}  \Ellent{A_{{\mathrm{1}}}}  \rightharpoonup  \Ellent{A_{{\mathrm{2}}}}  \Ellesym{,}  \Delta_{{\mathrm{1}}}  \Ellesym{,}  \Phi  \Ellesym{,}  \Delta_{{\mathrm{2}}}  \Ellesym{,}  \Gamma_{{\mathrm{2}}}  \vdash_\mathcal{L}  \Ellent{B}}
        \end{math}
      \end{center}

  \item $\ElledruleSXXimprLName$ Case 2:
      \begin{center}
        \scriptsize
        \begin{math}
          \begin{array}{c}
            \Pi_1 \\
            {\Gamma  \vdash_\mathcal{L}  \Ellent{C}}
          \end{array}
        \end{math}
        \qquad\qquad
        $\Pi_2$:
        \begin{math}
          $$\mprset{flushleft}
          \inferrule* [right={\tiny imprL}] {
            {
              \begin{array}{cc}
                \pi_1 & \pi_2 \\
                {\Delta_{{\mathrm{1}}}  \Ellesym{,}  \Ellent{C}  \Ellesym{,}  \Delta_{{\mathrm{2}}}  \vdash_\mathcal{L}  \Ellent{A_{{\mathrm{1}}}}} & {\Gamma_{{\mathrm{1}}}  \Ellesym{,}  \Ellent{A_{{\mathrm{2}}}}  \Ellesym{,}  \Gamma_{{\mathrm{2}}}  \vdash_\mathcal{L}  \Ellent{B}}
              \end{array}
            }
          }{\Gamma_{{\mathrm{1}}}  \Ellesym{,}  \Ellent{A_{{\mathrm{1}}}}  \rightharpoonup  \Ellent{A_{{\mathrm{2}}}}  \Ellesym{,}  \Delta_{{\mathrm{1}}}  \Ellesym{,}  \Ellent{C}  \Ellesym{,}  \Delta_{{\mathrm{2}}}  \Ellesym{,}  \Gamma_{{\mathrm{2}}}  \vdash_\mathcal{L}  \Ellent{B}}
        \end{math}
      \end{center}
      By assumption, $c(\Pi_1),c(\Pi_2)\leq |C|$. By induction on $\Pi_1$ and $\pi_1$, there is
      a proof $\Pi'$ for sequent $\Delta_{{\mathrm{1}}}  \Ellesym{,}  \Gamma  \Ellesym{,}  \Delta_{{\mathrm{2}}}  \vdash_\mathcal{L}  \Ellent{A_{{\mathrm{1}}}}$ s.t. $c(\Pi') \leq |C|$. Therefore, the
      proof $\Pi$ can be constructed as follows with $c(\Pi) = c(\Pi') \leq |C|$.
      \begin{center}
        \scriptsize
        \begin{math}
          $$\mprset{flushleft}
          \inferrule* [right={\tiny imprL}] {
            {
              \begin{array}{cc}
                \Pi' & \pi_2 \\
                {\Delta_{{\mathrm{1}}}  \Ellesym{,}  \Gamma  \Ellesym{,}  \Delta_{{\mathrm{2}}}  \vdash_\mathcal{L}  \Ellent{A_{{\mathrm{1}}}}} & {\Gamma_{{\mathrm{1}}}  \Ellesym{,}  \Ellent{A_{{\mathrm{2}}}}  \Ellesym{,}  \Gamma_{{\mathrm{2}}}  \vdash_\mathcal{L}  \Ellent{B}}
              \end{array}
            }
          }{\Gamma_{{\mathrm{1}}}  \Ellesym{,}  \Ellent{A_{{\mathrm{1}}}}  \rightharpoonup  \Ellent{A_{{\mathrm{2}}}}  \Ellesym{,}  \Delta_{{\mathrm{1}}}  \Ellesym{,}  \Gamma  \Ellesym{,}  \Delta_{{\mathrm{2}}}  \Ellesym{,}  \Gamma_{{\mathrm{2}}}  \vdash_\mathcal{L}  \Ellent{B}}
        \end{math}
      \end{center}

  \item $\ElledruleSXXimprLName$ Case 3:
      \begin{center}
        \scriptsize
        \begin{math}
          \begin{array}{c}
            \Pi_1 \\
            {\Phi  \vdash_\mathcal{C}  \Ellent{X}}
          \end{array}
        \end{math}
        \qquad\qquad
        $\Pi_2$:
        \begin{math}
          $$\mprset{flushleft}
          \inferrule* [right={\tiny imprL}] {
            {
              \begin{array}{cc}
                \pi_1 & \pi_2 \\
                {\Delta  \vdash_\mathcal{L}  \Ellent{A_{{\mathrm{1}}}}} & {\Gamma_{{\mathrm{1}}}  \Ellesym{,}  \Ellent{X}  \Ellesym{,}  \Gamma_{{\mathrm{2}}}  \Ellesym{,}  \Ellent{A_{{\mathrm{2}}}}  \Ellesym{,}  \Gamma_{{\mathrm{3}}}  \vdash_\mathcal{L}  \Ellent{B}}
              \end{array}
            }
          }{\Gamma_{{\mathrm{1}}}  \Ellesym{,}  \Ellent{X}  \Ellesym{,}  \Gamma_{{\mathrm{2}}}  \Ellesym{,}  \Ellent{A_{{\mathrm{1}}}}  \rightharpoonup  \Ellent{A_{{\mathrm{2}}}}  \Ellesym{,}  \Delta  \Ellesym{,}  \Gamma_{{\mathrm{3}}}  \vdash_\mathcal{L}  \Ellent{B}}
        \end{math}
      \end{center}
      By assumption, $c(\Pi_1),c(\Pi_2)\leq |X|$. By induction on $\Pi_1$ and $\pi_2$, there is
      a proof $\Pi'$ for sequent $\Gamma_{{\mathrm{1}}}  \Ellesym{,}  \Phi  \Ellesym{,}  \Gamma_{{\mathrm{2}}}  \Ellesym{,}  \Ellent{A_{{\mathrm{2}}}}  \Ellesym{,}  \Gamma_{{\mathrm{3}}}  \vdash_\mathcal{L}  \Ellent{B}$ s.t. $c(\Pi') \leq |X|$.
      Therefore, the proof $\Pi$ can be constructed as follows with
      $c(\Pi) = c(\Pi') \leq |X|$.
      \begin{center}
        \scriptsize
        \begin{math}
          $$\mprset{flushleft}
          \inferrule* [right={\tiny imprL}] {
            {
              \begin{array}{cc}
                \pi_1 & \Pi' \\
                {\Delta  \vdash_\mathcal{L}  \Ellent{A_{{\mathrm{1}}}}} & {\Gamma_{{\mathrm{1}}}  \Ellesym{,}  \Phi  \Ellesym{,}  \Gamma_{{\mathrm{2}}}  \Ellesym{,}  \Ellent{A_{{\mathrm{2}}}}  \Ellesym{,}  \Gamma_{{\mathrm{3}}}  \vdash_\mathcal{L}  \Ellent{B}}
              \end{array}
            }
          }{\Gamma_{{\mathrm{1}}}  \Ellesym{,}  \Phi  \Ellesym{,}  \Gamma_{{\mathrm{2}}}  \Ellesym{,}  \Ellent{A_{{\mathrm{1}}}}  \rightharpoonup  \Ellent{A_{{\mathrm{2}}}}  \Ellesym{,}  \Delta  \Ellesym{,}  \Gamma_{{\mathrm{3}}}  \vdash_\mathcal{L}  \Ellent{B}}
        \end{math}
      \end{center}

  \item $\ElledruleSXXimprLName$ Case 4:
      \begin{center}
        \scriptsize
        \begin{math}
          \begin{array}{c}
            \Pi_1 \\
            {\Delta_{{\mathrm{1}}}  \vdash_\mathcal{L}  \Ellent{B}}
          \end{array}
        \end{math}
        \qquad\qquad
        $\Pi_2$:
        \begin{math}
          $$\mprset{flushleft}
          \inferrule* [right={\tiny imprL}] {
            {
              \begin{array}{cc}
                \pi_1 & \pi_2 \\
                {\Delta_{{\mathrm{2}}}  \vdash_\mathcal{L}  \Ellent{A_{{\mathrm{1}}}}} & {\Gamma_{{\mathrm{1}}}  \Ellesym{,}  \Ellent{B}  \Ellesym{,}  \Gamma_{{\mathrm{2}}}  \Ellesym{,}  \Ellent{A_{{\mathrm{2}}}}  \Ellesym{,}  \Gamma_{{\mathrm{3}}}  \vdash_\mathcal{L}  \Ellent{C}}
              \end{array}
            }
          }{\Gamma_{{\mathrm{1}}}  \Ellesym{,}  \Ellent{B}  \Ellesym{,}  \Gamma_{{\mathrm{2}}}  \Ellesym{,}  \Ellent{A_{{\mathrm{1}}}}  \rightharpoonup  \Ellent{A_{{\mathrm{2}}}}  \Ellesym{,}  \Delta_{{\mathrm{2}}}  \Ellesym{,}  \Gamma_{{\mathrm{3}}}  \vdash_\mathcal{L}  \Ellent{C}}
        \end{math}
      \end{center}
      By assumption, $c(\Pi_1),c(\Pi_2)\leq |B|$. By induction on $\Pi_1$ and $\pi_2$, there is
      a proof $\Pi'$ for sequent $\Gamma_{{\mathrm{1}}}  \Ellesym{,}  \Delta_{{\mathrm{1}}}  \Ellesym{,}  \Gamma_{{\mathrm{2}}}  \Ellesym{,}  \Ellent{A_{{\mathrm{2}}}}  \Ellesym{,}  \Gamma_{{\mathrm{3}}}  \vdash_\mathcal{L}  \Ellent{C}$ s.t. $c(\Pi') \leq |B|$.
      Therefore, the proof $\Pi$ can be constructed as follows with
      $c(\Pi) = c(\Pi') \leq |B|$.
      \begin{center}
        \scriptsize
        \begin{math}
          $$\mprset{flushleft}
          \inferrule* [right={\tiny imprL}] {
            {
              \begin{array}{cc}
                \pi_1 & \Pi' \\
                {\Delta_{{\mathrm{2}}}  \vdash_\mathcal{L}  \Ellent{A_{{\mathrm{1}}}}} & {\Gamma_{{\mathrm{1}}}  \Ellesym{,}  \Delta_{{\mathrm{1}}}  \Ellesym{,}  \Gamma_{{\mathrm{2}}}  \Ellesym{,}  \Ellent{A_{{\mathrm{2}}}}  \Ellesym{,}  \Gamma_{{\mathrm{3}}}  \vdash_\mathcal{L}  \Ellent{C}}
              \end{array}
            }
          }{\Gamma_{{\mathrm{1}}}  \Ellesym{,}  \Delta_{{\mathrm{1}}}  \Ellesym{,}  \Gamma_{{\mathrm{2}}}  \Ellesym{,}  \Ellent{A_{{\mathrm{1}}}}  \rightharpoonup  \Ellent{A_{{\mathrm{2}}}}  \Ellesym{,}  \Delta_{{\mathrm{2}}}  \Ellesym{,}  \Gamma_{{\mathrm{3}}}  \vdash_\mathcal{L}  \Ellent{C}}
        \end{math}
      \end{center}

  \item $\ElledruleSXXimprLName$ Case 5:
      \begin{center}
        \scriptsize
        \begin{math}
          \begin{array}{c}
            \Pi_1 \\
            {\Phi  \vdash_\mathcal{C}  \Ellent{X}}
          \end{array}
        \end{math}
        \qquad\qquad
        $\Pi_2$:
        \begin{math}
          $$\mprset{flushleft}
          \inferrule* [right={\tiny imprL}] {
            {
              \begin{array}{cc}
                \pi_1 & \pi_2 \\
                {\Delta  \vdash_\mathcal{L}  \Ellent{A_{{\mathrm{1}}}}} & {\Gamma_{{\mathrm{1}}}  \Ellesym{,}  \Ellent{A_{{\mathrm{2}}}}  \Ellesym{,}  \Gamma_{{\mathrm{2}}}  \Ellesym{,}  \Ellent{X}  \Ellesym{,}  \Gamma_{{\mathrm{3}}}  \vdash_\mathcal{L}  \Ellent{B}}
              \end{array}
            }
          }{\Gamma_{{\mathrm{1}}}  \Ellesym{,}  \Ellent{A_{{\mathrm{1}}}}  \rightharpoonup  \Ellent{A_{{\mathrm{2}}}}  \Ellesym{,}  \Delta  \Ellesym{,}  \Gamma_{{\mathrm{2}}}  \Ellesym{,}  \Ellent{X}  \Ellesym{,}  \Gamma_{{\mathrm{3}}}  \vdash_\mathcal{L}  \Ellent{B}}
        \end{math}
      \end{center}
      By assumption, $c(\Pi_1),c(\Pi_2)\leq |X|$. By induction on $\Pi_1$ and $\pi_2$, there is
      a proof $\Pi'$ for sequent $\Gamma_{{\mathrm{1}}}  \Ellesym{,}  \Ellent{A_{{\mathrm{2}}}}  \Ellesym{,}  \Gamma_{{\mathrm{2}}}  \Ellesym{,}  \Phi  \Ellesym{,}  \Gamma_{{\mathrm{3}}}  \vdash_\mathcal{L}  \Ellent{B}$ s.t. $c(\Pi') \leq |X|$.
      Therefore, the proof $\Pi$ can be constructed as follows with
      $c(\Pi) = c(\Pi') \leq |X|$.
      \begin{center}
        \scriptsize
        \begin{math}
          $$\mprset{flushleft}
          \inferrule* [right={\tiny imprL}] {
            {
              \begin{array}{cc}
                \pi_1 & \Pi' \\
                {\Delta  \vdash_\mathcal{L}  \Ellent{A_{{\mathrm{1}}}}} & {\Gamma_{{\mathrm{1}}}  \Ellesym{,}  \Ellent{A_{{\mathrm{2}}}}  \Ellesym{,}  \Gamma_{{\mathrm{2}}}  \Ellesym{,}  \Phi  \Ellesym{,}  \Gamma_{{\mathrm{3}}}  \vdash_\mathcal{L}  \Ellent{B}}
              \end{array}
            }
          }{\Gamma_{{\mathrm{1}}}  \Ellesym{,}  \Ellent{A_{{\mathrm{1}}}}  \rightharpoonup  \Ellent{A_{{\mathrm{2}}}}  \Ellesym{,}  \Delta  \Ellesym{,}  \Gamma_{{\mathrm{2}}}  \Ellesym{,}  \Phi  \Ellesym{,}  \Gamma_{{\mathrm{3}}}  \vdash_\mathcal{L}  \Ellent{B}}
        \end{math}
      \end{center}

  \item $\ElledruleSXXimprLName$ Case 6:
      \begin{center}
        \scriptsize
        \begin{math}
          \begin{array}{c}
            \Pi_1 \\
            {\Delta_{{\mathrm{1}}}  \vdash_\mathcal{L}  \Ellent{B}}
          \end{array}
        \end{math}
        \qquad\qquad
        $\Pi_2$:
        \begin{math}
          $$\mprset{flushleft}
          \inferrule* [right={\tiny imprL}] {
            {
              \begin{array}{cc}
                \pi_1 & \pi_2 \\
                {\Delta_{{\mathrm{2}}}  \vdash_\mathcal{L}  \Ellent{A_{{\mathrm{1}}}}} & {\Gamma_{{\mathrm{1}}}  \Ellesym{,}  \Ellent{A_{{\mathrm{2}}}}  \Ellesym{,}  \Gamma_{{\mathrm{2}}}  \Ellesym{,}  \Ellent{B}  \Ellesym{,}  \Gamma_{{\mathrm{3}}}  \vdash_\mathcal{L}  \Ellent{C}}
              \end{array}
            }
          }{\Gamma_{{\mathrm{1}}}  \Ellesym{,}  \Ellent{A_{{\mathrm{1}}}}  \rightharpoonup  \Ellent{A_{{\mathrm{2}}}}  \Ellesym{,}  \Delta_{{\mathrm{2}}}  \Ellesym{,}  \Gamma_{{\mathrm{2}}}  \Ellesym{,}  \Ellent{B}  \Ellesym{,}  \Gamma_{{\mathrm{3}}}  \vdash_\mathcal{L}  \Ellent{C}}
        \end{math}
      \end{center}
      By assumption, $c(\Pi_1),c(\Pi_2)\leq |B|$. By induction on $\Pi_1$ and $\pi_2$, there is
      a proof $\Pi'$ for sequent $\Gamma_{{\mathrm{1}}}  \Ellesym{,}  \Ellent{A_{{\mathrm{2}}}}  \Ellesym{,}  \Gamma_{{\mathrm{2}}}  \Ellesym{,}  \Delta_{{\mathrm{1}}}  \Ellesym{,}  \Gamma_{{\mathrm{3}}}  \vdash_\mathcal{L}  \Ellent{C}$ s.t. $c(\Pi') \leq |B|$.
      Therefore, the proof $\Pi$ can be constructed as follows with
      $c(\Pi) = c(\Pi') \leq |B|$.
      \begin{center}
        \scriptsize
        \begin{math}
          $$\mprset{flushleft}
          \inferrule* [right={\tiny imprL}] {
            {
              \begin{array}{cc}
                \pi_1 & \Pi' \\
                {\Delta_{{\mathrm{2}}}  \vdash_\mathcal{L}  \Ellent{A_{{\mathrm{1}}}}} & {\Gamma_{{\mathrm{1}}}  \Ellesym{,}  \Ellent{A_{{\mathrm{2}}}}  \Ellesym{,}  \Gamma_{{\mathrm{2}}}  \Ellesym{,}  \Delta_{{\mathrm{1}}}  \Ellesym{,}  \Gamma_{{\mathrm{3}}}  \vdash_\mathcal{L}  \Ellent{C}}
              \end{array}
            }
          }{\Gamma_{{\mathrm{1}}}  \Ellesym{,}  \Ellent{A_{{\mathrm{1}}}}  \rightharpoonup  \Ellent{A_{{\mathrm{2}}}}  \Ellesym{,}  \Delta_{{\mathrm{2}}}  \Ellesym{,}  \Gamma_{{\mathrm{2}}}  \Ellesym{,}  \Delta_{{\mathrm{1}}}  \Ellesym{,}  \Gamma_{{\mathrm{3}}}  \vdash_\mathcal{L}  \Ellent{C}}
        \end{math}
      \end{center}

  \item $\ElledruleSXXimplLName$ Case 1:
      \begin{center}
        \scriptsize
        \begin{math}
          \begin{array}{c}
            \Pi_1 \\
            {\Phi  \vdash_\mathcal{C}  \Ellent{X}}
          \end{array}
        \end{math}
        \qquad\qquad
        $\Pi_2$:
        \begin{math}
          $$\mprset{flushleft}
          \inferrule* [right={\tiny implL}] {
            {
              \begin{array}{cc}
                \pi_1 & \pi_2 \\
                {\Delta_{{\mathrm{1}}}  \Ellesym{,}  \Ellent{X}  \Ellesym{,}  \Delta_{{\mathrm{2}}}  \vdash_\mathcal{L}  \Ellent{A_{{\mathrm{1}}}}} & {\Gamma_{{\mathrm{1}}}  \Ellesym{,}  \Ellent{A_{{\mathrm{2}}}}  \Ellesym{,}  \Gamma_{{\mathrm{2}}}  \vdash_\mathcal{L}  \Ellent{B}}
              \end{array}
            }
          }{\Gamma_{{\mathrm{1}}}  \Ellesym{,}  \Delta_{{\mathrm{1}}}  \Ellesym{,}  \Ellent{A_{{\mathrm{2}}}}  \leftharpoonup  \Ellent{A_{{\mathrm{1}}}}  \Ellesym{,}  \Ellent{X}  \Ellesym{,}  \Delta_{{\mathrm{2}}}  \Ellesym{,}  \Gamma_{{\mathrm{2}}}  \vdash_\mathcal{L}  \Ellent{B}}
        \end{math}
      \end{center}
      By assumption, $c(\Pi_1),c(\Pi_2)\leq |X|$. By induction on $\Pi_1$ and $\pi_1$, there is
      a proof $\Pi'$ for sequent $\Delta_{{\mathrm{1}}}  \Ellesym{,}  \Phi  \Ellesym{,}  \Delta_{{\mathrm{2}}}  \vdash_\mathcal{L}  \Ellent{A_{{\mathrm{1}}}}$ s.t. $c(\Pi') \leq |X|$. Therefore, the
      proof $\Pi$ can be constructed as follows with $c(\Pi) = c(\Pi') \leq |X|$.
      \begin{center}
        \scriptsize
        \begin{math}
          $$\mprset{flushleft}
          \inferrule* [right={\tiny implL}] {
            {
              \begin{array}{cc}
                \Pi' & \pi_2 \\
                {\Delta_{{\mathrm{1}}}  \Ellesym{,}  \Phi  \Ellesym{,}  \Delta_{{\mathrm{2}}}  \vdash_\mathcal{L}  \Ellent{A_{{\mathrm{1}}}}} & {\Gamma_{{\mathrm{1}}}  \Ellesym{,}  \Ellent{A_{{\mathrm{2}}}}  \Ellesym{,}  \Gamma_{{\mathrm{2}}}  \vdash_\mathcal{L}  \Ellent{B}}
              \end{array}
            }
          }{\Gamma_{{\mathrm{1}}}  \Ellesym{,}  \Delta_{{\mathrm{1}}}  \Ellesym{,}  \Ellent{A_{{\mathrm{2}}}}  \leftharpoonup  \Ellent{A_{{\mathrm{1}}}}  \Ellesym{,}  \Phi  \Ellesym{,}  \Delta_{{\mathrm{2}}}  \Ellesym{,}  \Gamma_{{\mathrm{2}}}  \vdash_\mathcal{L}  \Ellent{B}}
        \end{math}
      \end{center}

  \item $\ElledruleSXXimplLName$ Case 2:
      \begin{center}
        \scriptsize
        \begin{math}
          \begin{array}{c}
            \Pi_1 \\
            {\Gamma  \vdash_\mathcal{L}  \Ellent{C}}
          \end{array}
        \end{math}
        \qquad\qquad
        $\Pi_2$:
        \begin{math}
          $$\mprset{flushleft}
          \inferrule* [right={\tiny implL}] {
            {
              \begin{array}{cc}
                \pi_1 & \pi_2 \\
                {\Delta_{{\mathrm{1}}}  \Ellesym{,}  \Ellent{C}  \Ellesym{,}  \Delta_{{\mathrm{2}}}  \vdash_\mathcal{L}  \Ellent{A_{{\mathrm{1}}}}} & {\Gamma_{{\mathrm{1}}}  \Ellesym{,}  \Ellent{A_{{\mathrm{2}}}}  \Ellesym{,}  \Gamma_{{\mathrm{2}}}  \vdash_\mathcal{L}  \Ellent{B}}
              \end{array}
            }
          }{\Gamma_{{\mathrm{1}}}  \Ellesym{,}  \Delta_{{\mathrm{1}}}  \Ellesym{,}  \Ellent{C}  \Ellesym{,}  \Delta_{{\mathrm{2}}}  \Ellesym{,}  \Ellent{A_{{\mathrm{2}}}}  \leftharpoonup  \Ellent{A_{{\mathrm{1}}}}  \Ellesym{,}  \Gamma_{{\mathrm{2}}}  \vdash_\mathcal{L}  \Ellent{B}}
        \end{math}
      \end{center}
      By assumption, $c(\Pi_1),c(\Pi_2)\leq |C|$. By induction on $\Pi_1$ and $\pi_1$, there is
      a proof $\Pi'$ for sequent $\Delta_{{\mathrm{1}}}  \Ellesym{,}  \Gamma  \Ellesym{,}  \Delta_{{\mathrm{2}}}  \vdash_\mathcal{L}  \Ellent{A_{{\mathrm{1}}}}$ s.t. $c(\Pi') \leq |C|$. Therefore, the
      proof $\Pi$ can be constructed as follows with $c(\Pi) = c(\Pi') \leq |C|$.
      \begin{center}
        \scriptsize
        \begin{math}
          $$\mprset{flushleft}
          \inferrule* [right={\tiny implL}] {
            {
              \begin{array}{cc}
                \Pi' & \pi_2 \\
                {\Delta_{{\mathrm{1}}}  \Ellesym{,}  \Gamma  \Ellesym{,}  \Delta_{{\mathrm{2}}}  \vdash_\mathcal{L}  \Ellent{A_{{\mathrm{1}}}}} & {\Gamma_{{\mathrm{1}}}  \Ellesym{,}  \Ellent{A_{{\mathrm{2}}}}  \Ellesym{,}  \Gamma_{{\mathrm{2}}}  \vdash_\mathcal{L}  \Ellent{B}}
              \end{array}
            }
          }{\Gamma_{{\mathrm{1}}}  \Ellesym{,}  \Delta_{{\mathrm{1}}}  \Ellesym{,}  \Gamma  \Ellesym{,}  \Delta_{{\mathrm{2}}}  \Ellesym{,}  \Ellent{A_{{\mathrm{2}}}}  \leftharpoonup  \Ellent{A_{{\mathrm{1}}}}  \Ellesym{,}  \Gamma_{{\mathrm{2}}}  \vdash_\mathcal{L}  \Ellent{B}}
        \end{math}
      \end{center}

  \item $\ElledruleSXXimplLName$ Case 3:
      \begin{center}
        \scriptsize
        \begin{math}
          \begin{array}{c}
            \Pi_1 \\
            {\Phi  \vdash_\mathcal{C}  \Ellent{X}}
          \end{array}
        \end{math}
        \qquad\qquad
        $\Pi_2$:
        \begin{math}
          $$\mprset{flushleft}
          \inferrule* [right={\tiny implL}] {
            {
              \begin{array}{cc}
                \pi_1 & \pi_2 \\
                {\Delta  \vdash_\mathcal{L}  \Ellent{A_{{\mathrm{1}}}}} & {\Gamma_{{\mathrm{1}}}  \Ellesym{,}  \Ellent{X}  \Ellesym{,}  \Gamma_{{\mathrm{2}}}  \Ellesym{,}  \Ellent{A_{{\mathrm{2}}}}  \Ellesym{,}  \Gamma_{{\mathrm{3}}}  \vdash_\mathcal{L}  \Ellent{B}}
              \end{array}
            }
          }{\Gamma_{{\mathrm{1}}}  \Ellesym{,}  \Ellent{X}  \Ellesym{,}  \Gamma_{{\mathrm{2}}}  \Ellesym{,}  \Delta  \Ellesym{,}  \Ellent{A_{{\mathrm{2}}}}  \leftharpoonup  \Ellent{A_{{\mathrm{1}}}}  \Ellesym{,}  \Gamma_{{\mathrm{3}}}  \vdash_\mathcal{L}  \Ellent{B}}
        \end{math}
      \end{center}
      By assumption, $c(\Pi_1),c(\Pi_2)\leq |X|$. By induction on $\Pi_1$ and $\pi_2$, there is
      a proof $\Pi'$ for sequent $\Gamma_{{\mathrm{1}}}  \Ellesym{,}  \Phi  \Ellesym{,}  \Gamma_{{\mathrm{2}}}  \Ellesym{,}  \Ellent{A_{{\mathrm{2}}}}  \Ellesym{,}  \Gamma_{{\mathrm{3}}}  \vdash_\mathcal{L}  \Ellent{B}$ s.t. $c(\Pi') \leq |X|$.
      Therefore, the proof $\Pi$ can be constructed as follows with
      $c(\Pi) = c(\Pi') \leq |X|$.
      \begin{center}
        \scriptsize
        \begin{math}
          $$\mprset{flushleft}
          \inferrule* [right={\tiny implL}] {
            {
              \begin{array}{cc}
                \pi_1 & \Pi' \\
                {\Delta  \vdash_\mathcal{L}  \Ellent{A_{{\mathrm{1}}}}} & {\Gamma_{{\mathrm{1}}}  \Ellesym{,}  \Phi  \Ellesym{,}  \Gamma_{{\mathrm{2}}}  \Ellesym{,}  \Ellent{A_{{\mathrm{2}}}}  \Ellesym{,}  \Gamma_{{\mathrm{3}}}  \vdash_\mathcal{L}  \Ellent{B}}
              \end{array}
            }
          }{\Gamma_{{\mathrm{1}}}  \Ellesym{,}  \Phi  \Ellesym{,}  \Gamma_{{\mathrm{2}}}  \Ellesym{,}  \Delta  \Ellesym{,}  \Ellent{A_{{\mathrm{2}}}}  \leftharpoonup  \Ellent{A_{{\mathrm{1}}}}  \Ellesym{,}  \Gamma_{{\mathrm{3}}}  \vdash_\mathcal{L}  \Ellent{B}}
        \end{math}
      \end{center}

  \item $\ElledruleSXXimplLName$ Case 4:
      \begin{center}
        \scriptsize
        \begin{math}
          \begin{array}{c}
            \Pi_1 \\
            {\Delta_{{\mathrm{1}}}  \vdash_\mathcal{L}  \Ellent{B}}
          \end{array}
        \end{math}
        \qquad\qquad
        $\Pi_2$:
        \begin{math}
          $$\mprset{flushleft}
          \inferrule* [right={\tiny implL}] {
            {
              \begin{array}{cc}
                \pi_1 & \pi_2 \\
                {\Delta_{{\mathrm{2}}}  \vdash_\mathcal{L}  \Ellent{A_{{\mathrm{1}}}}} & {\Gamma_{{\mathrm{1}}}  \Ellesym{,}  \Ellent{B}  \Ellesym{,}  \Gamma_{{\mathrm{2}}}  \Ellesym{,}  \Ellent{A_{{\mathrm{2}}}}  \Ellesym{,}  \Gamma_{{\mathrm{3}}}  \vdash_\mathcal{L}  \Ellent{C}}
              \end{array}
            }
          }{\Gamma_{{\mathrm{1}}}  \Ellesym{,}  \Ellent{B}  \Ellesym{,}  \Gamma_{{\mathrm{2}}}  \Ellesym{,}  \Delta_{{\mathrm{2}}}  \Ellesym{,}  \Ellent{A_{{\mathrm{2}}}}  \leftharpoonup  \Ellent{A_{{\mathrm{1}}}}  \Ellesym{,}  \Gamma_{{\mathrm{3}}}  \vdash_\mathcal{L}  \Ellent{C}}
        \end{math}
      \end{center}
      By assumption, $c(\Pi_1),c(\Pi_2)\leq |B|$. By induction on $\Pi_1$ and $\pi_2$, there is
      a proof $\Pi'$ for sequent $\Gamma_{{\mathrm{1}}}  \Ellesym{,}  \Delta_{{\mathrm{1}}}  \Ellesym{,}  \Gamma_{{\mathrm{2}}}  \Ellesym{,}  \Ellent{A_{{\mathrm{2}}}}  \Ellesym{,}  \Gamma_{{\mathrm{3}}}  \vdash_\mathcal{L}  \Ellent{C}$ s.t. $c(\Pi') \leq |B|$.
      Therefore, the proof $\Pi$ can be constructed as follows with
      $c(\Pi) = c(\Pi') \leq |B|$.
      \begin{center}
        \scriptsize
        \begin{math}
          $$\mprset{flushleft}
          \inferrule* [right={\tiny implL}] {
            {
              \begin{array}{cc}
                \pi_1 & \Pi' \\
                {\Delta_{{\mathrm{2}}}  \vdash_\mathcal{L}  \Ellent{A_{{\mathrm{1}}}}} & {\Gamma_{{\mathrm{1}}}  \Ellesym{,}  \Delta_{{\mathrm{1}}}  \Ellesym{,}  \Gamma_{{\mathrm{2}}}  \Ellesym{,}  \Ellent{A_{{\mathrm{2}}}}  \Ellesym{,}  \Gamma_{{\mathrm{3}}}  \vdash_\mathcal{L}  \Ellent{C}}
              \end{array}
            }
          }{\Gamma_{{\mathrm{1}}}  \Ellesym{,}  \Delta_{{\mathrm{1}}}  \Ellesym{,}  \Gamma_{{\mathrm{2}}}  \Ellesym{,}  \Delta_{{\mathrm{2}}}  \Ellesym{,}  \Ellent{A_{{\mathrm{2}}}}  \leftharpoonup  \Ellent{A_{{\mathrm{1}}}}  \Ellesym{,}  \Gamma_{{\mathrm{3}}}  \vdash_\mathcal{L}  \Ellent{C}}
        \end{math}
      \end{center}

  \item $\ElledruleSXXimplLName$ Case 5:
      \begin{center}
        \scriptsize
        \begin{math}
          \begin{array}{c}
            \Pi_1 \\
            {\Phi  \vdash_\mathcal{C}  \Ellent{X}}
          \end{array}
        \end{math}
        \qquad\qquad
        $\Pi_2$:
        \begin{math}
          $$\mprset{flushleft}
          \inferrule* [right={\tiny implL}] {
            {
              \begin{array}{cc}
                \pi_1 & \pi_2 \\
                {\Delta  \vdash_\mathcal{L}  \Ellent{A_{{\mathrm{1}}}}} & {\Gamma_{{\mathrm{1}}}  \Ellesym{,}  \Ellent{A_{{\mathrm{2}}}}  \Ellesym{,}  \Gamma_{{\mathrm{2}}}  \Ellesym{,}  \Ellent{X}  \Ellesym{,}  \Gamma_{{\mathrm{3}}}  \vdash_\mathcal{L}  \Ellent{B}}
              \end{array}
            }
          }{\Gamma_{{\mathrm{1}}}  \Ellesym{,}  \Delta  \Ellesym{,}  \Ellent{A_{{\mathrm{2}}}}  \leftharpoonup  \Ellent{A_{{\mathrm{1}}}}  \Ellesym{,}  \Delta  \Ellesym{,}  \Gamma_{{\mathrm{2}}}  \Ellesym{,}  \Ellent{X}  \Ellesym{,}  \Gamma_{{\mathrm{3}}}  \vdash_\mathcal{L}  \Ellent{B}}
        \end{math}
      \end{center}
      By assumption, $c(\Pi_1),c(\Pi_2)\leq |X|$. By induction on $\Pi_1$ and $\pi_2$, there is
      a proof $\Pi'$ for sequent $\Gamma_{{\mathrm{1}}}  \Ellesym{,}  \Ellent{A_{{\mathrm{2}}}}  \Ellesym{,}  \Gamma_{{\mathrm{2}}}  \Ellesym{,}  \Phi  \Ellesym{,}  \Gamma_{{\mathrm{3}}}  \vdash_\mathcal{L}  \Ellent{B}$ s.t. $c(\Pi') \leq |X|$.
      Therefore, the proof $\Pi$ can be constructed as follows with
      $c(\Pi) = c(\Pi') \leq |X|$.
      \begin{center}
        \scriptsize
        \begin{math}
          $$\mprset{flushleft}
          \inferrule* [right={\tiny implL}] {
            {
              \begin{array}{cc}
                \pi_1 & \Pi' \\
                {\Delta  \vdash_\mathcal{L}  \Ellent{A_{{\mathrm{1}}}}} & {\Gamma_{{\mathrm{1}}}  \Ellesym{,}  \Ellent{A_{{\mathrm{2}}}}  \Ellesym{,}  \Gamma_{{\mathrm{2}}}  \Ellesym{,}  \Phi  \Ellesym{,}  \Gamma_{{\mathrm{3}}}  \vdash_\mathcal{L}  \Ellent{B}}
              \end{array}
            }
          }{\Gamma_{{\mathrm{1}}}  \Ellesym{,}  \Delta  \Ellesym{,}  \Ellent{A_{{\mathrm{2}}}}  \leftharpoonup  \Ellent{A_{{\mathrm{1}}}}  \Ellesym{,}  \Gamma_{{\mathrm{2}}}  \Ellesym{,}  \Phi  \Ellesym{,}  \Gamma_{{\mathrm{3}}}  \vdash_\mathcal{L}  \Ellent{B}}
        \end{math}
      \end{center}

  \item $\ElledruleSXXimplLName$ Case 6:
    \begin{center}
      \scriptsize
      \begin{math}
        \begin{array}{c}
          \Pi_1 \\
          {\Delta_{{\mathrm{1}}}  \vdash_\mathcal{L}  \Ellent{B}}
        \end{array}
      \end{math}
      \qquad\qquad
      $\Pi_2$:
      \begin{math}
        $$\mprset{flushleft}
        \inferrule* [right={\tiny implL}] {
          {
            \begin{array}{cc}
              \pi_1 & \pi_2 \\
              {\Delta_{{\mathrm{2}}}  \vdash_\mathcal{L}  \Ellent{A_{{\mathrm{1}}}}} & {\Gamma_{{\mathrm{1}}}  \Ellesym{,}  \Ellent{A_{{\mathrm{2}}}}  \Ellesym{,}  \Gamma_{{\mathrm{2}}}  \Ellesym{,}  \Ellent{B}  \Ellesym{,}  \Gamma_{{\mathrm{3}}}  \vdash_\mathcal{L}  \Ellent{C}}
            \end{array}
          }
        }{\Gamma_{{\mathrm{1}}}  \Ellesym{,}  \Delta_{{\mathrm{2}}}  \Ellesym{,}  \Ellent{A_{{\mathrm{2}}}}  \leftharpoonup  \Ellent{A_{{\mathrm{1}}}}  \Ellesym{,}  \Gamma_{{\mathrm{2}}}  \Ellesym{,}  \Ellent{B}  \Ellesym{,}  \Gamma_{{\mathrm{3}}}  \vdash_\mathcal{L}  \Ellent{C}}
      \end{math}
    \end{center}
    By assumption, $c(\Pi_1),c(\Pi_2)\leq |B|$. By induction on $\Pi_1$ and $\pi_2$, there is a
    proof $\Pi'$ for sequent $\Gamma_{{\mathrm{1}}}  \Ellesym{,}  \Ellent{A_{{\mathrm{2}}}}  \Ellesym{,}  \Gamma_{{\mathrm{2}}}  \Ellesym{,}  \Delta_{{\mathrm{1}}}  \Ellesym{,}  \Gamma_{{\mathrm{3}}}  \vdash_\mathcal{L}  \Ellent{C}$ s.t. $c(\Pi') \leq |B|$. Therefore,
    the proof $\Pi$ can be constructed as follows with $c(\Pi) = c(\Pi') \leq |B|$.
    \begin{center}
      \scriptsize
      \begin{math}
        $$\mprset{flushleft}
        \inferrule* [right={\tiny implL}] {
          {
            \begin{array}{cc}
              \pi_1 & \Pi' \\
              {\Delta_{{\mathrm{2}}}  \vdash_\mathcal{L}  \Ellent{A_{{\mathrm{1}}}}} & {\Gamma_{{\mathrm{1}}}  \Ellesym{,}  \Ellent{A_{{\mathrm{2}}}}  \Ellesym{,}  \Gamma_{{\mathrm{2}}}  \Ellesym{,}  \Delta_{{\mathrm{1}}}  \Ellesym{,}  \Gamma_{{\mathrm{3}}}  \vdash_\mathcal{L}  \Ellent{C}}
            \end{array}
          }
        }{\Gamma_{{\mathrm{1}}}  \Ellesym{,}  \Delta_{{\mathrm{2}}}  \Ellesym{,}  \Ellent{A_{{\mathrm{2}}}}  \leftharpoonup  \Ellent{A_{{\mathrm{1}}}}  \Ellesym{,}  \Gamma_{{\mathrm{2}}}  \Ellesym{,}  \Delta_{{\mathrm{1}}}  \Ellesym{,}  \Gamma_{{\mathrm{3}}}  \vdash_\mathcal{L}  \Ellent{C}}
      \end{math}
    \end{center}

  \item $\ElledruleSXXFrName$:
    \begin{center}
      \scriptsize
      \begin{math}
        \begin{array}{c}
          \Pi_1 \\
          {\Phi  \vdash_\mathcal{C}  \Ellent{X}}
        \end{array}
      \end{math}
      \qquad\qquad
      $\Pi_2$:
      \begin{math}
        $$\mprset{flushleft}
        \inferrule* [right={\tiny Fr}] {
          {
            \begin{array}{c}
              \pi \\
              {\Psi_{{\mathrm{1}}}  \Ellesym{,}  \Ellent{X}  \Ellesym{,}  \Psi_{{\mathrm{2}}}  \vdash_\mathcal{C}  \Ellent{Y}}
            \end{array}
          }
        }{\Psi_{{\mathrm{1}}}  \Ellesym{,}  \Ellent{X}  \Ellesym{,}  \Psi_{{\mathrm{2}}}  \vdash_\mathcal{L}   \mathsf{F} \Ellent{Y} }
      \end{math}
    \end{center}
    By assumption, $c(\Pi_1),c(\Pi_2)\leq |X|$. By induction on $\Pi_1$ and $\pi$, there is a
    proof $\Pi'$ for sequent $\Psi_{{\mathrm{1}}}  \Ellesym{,}  \Phi  \Ellesym{,}  \Psi_{{\mathrm{2}}}  \vdash_\mathcal{C}  \Ellent{Y}$ s.t. $c(\Pi') \leq |X|$. Therefore, the
    proof $\Pi$ can be constructed as follows with $c(\Pi) = c(\Pi') \leq |X|$.
    \begin{center}
      \scriptsize
      \begin{math}
        $$\mprset{flushleft}
        \inferrule* [right={\tiny Fr}] {
          {
            \begin{array}{c}
              \Pi' \\
              {\Psi_{{\mathrm{1}}}  \Ellesym{,}  \Phi  \Ellesym{,}  \Psi_{{\mathrm{2}}}  \vdash_\mathcal{C}  \Ellent{Y}}
            \end{array}
          }
        }{\Psi_{{\mathrm{1}}}  \Ellesym{,}  \Phi  \Ellesym{,}  \Psi_{{\mathrm{2}}}  \vdash_\mathcal{L}   \mathsf{F} \Ellent{Y} }
      \end{math}
    \end{center}

  \item $\ElledruleSXXFlName$ Case 1:
    \begin{center}
      \scriptsize
      \begin{math}
        \begin{array}{c}
          \Pi_1 \\
          {\Phi  \vdash_\mathcal{C}  \Ellent{X}}
        \end{array}
      \end{math}
      \qquad\qquad
      $\Pi_2$:
      \begin{math}
        $$\mprset{flushleft}
        \inferrule* [right={\tiny Fl}] {
          {
            \begin{array}{c}
              \pi \\
              {\Gamma_{{\mathrm{1}}}  \Ellesym{,}  \Ellent{X}  \Ellesym{,}  \Gamma_{{\mathrm{2}}}  \Ellesym{,}  \Ellent{Y}  \Ellesym{,}  \Gamma_{{\mathrm{3}}}  \vdash_\mathcal{L}  \Ellent{A}}
            \end{array}
          }
        }{\Gamma_{{\mathrm{1}}}  \Ellesym{,}  \Ellent{X}  \Ellesym{,}  \Gamma_{{\mathrm{2}}}  \Ellesym{,}   \mathsf{F} \Ellent{Y}   \Ellesym{,}  \Gamma_{{\mathrm{3}}}  \vdash_\mathcal{L}  \Ellent{A}}
      \end{math}
    \end{center}
    By assumption, $c(\Pi_1),c(\Pi_2)\leq |X|$. By induction on $\Pi_1$ and $\pi$, there is a
    proof $\Pi'$ for sequent $\Gamma_{{\mathrm{1}}}  \Ellesym{,}  \Phi  \Ellesym{,}  \Gamma_{{\mathrm{2}}}  \Ellesym{,}  \Ellent{Y}  \Ellesym{,}  \Gamma_{{\mathrm{3}}}  \vdash_\mathcal{L}  \Ellent{A}$ s.t. $c(\Pi') \leq |X|$. Therefore,
    the proof $\Pi$ can be constructed as follows with $c(\Pi) = c(\Pi') \leq |X|$.
    \begin{center}
      \scriptsize
      \begin{math}
        $$\mprset{flushleft}
        \inferrule* [right={\tiny Fl}] {
          {
            \begin{array}{c}
              \Pi' \\
              {\Gamma_{{\mathrm{1}}}  \Ellesym{,}  \Phi  \Ellesym{,}  \Gamma_{{\mathrm{2}}}  \Ellesym{,}  \Ellent{Y}  \Ellesym{,}  \Gamma_{{\mathrm{3}}}  \vdash_\mathcal{L}  \Ellent{A}}
            \end{array}
          }
        }{\Gamma_{{\mathrm{1}}}  \Ellesym{,}  \Phi  \Ellesym{,}  \Gamma_{{\mathrm{2}}}  \Ellesym{,}   \mathsf{F} \Ellent{Y}   \Ellesym{,}  \Gamma_{{\mathrm{3}}}  \vdash_\mathcal{L}  \Ellent{A}}
      \end{math}
    \end{center}

  \item $\ElledruleSXXFlName$ Case 2:
    \begin{center}
      \scriptsize
      \begin{math}
        \begin{array}{c}
          \Pi_1 \\
          {\Delta  \vdash_\mathcal{L}  \Ellent{B}}
        \end{array}
      \end{math}
      \qquad\qquad
      $\Pi_2$:
      \begin{math}
        $$\mprset{flushleft}
        \inferrule* [right={\tiny Fl}] {
          {
            \begin{array}{c}
              \pi \\
              {\Gamma_{{\mathrm{1}}}  \Ellesym{,}  \Ellent{B}  \Ellesym{,}  \Gamma_{{\mathrm{2}}}  \Ellesym{,}  \Ellent{Y}  \Ellesym{,}  \Gamma_{{\mathrm{3}}}  \vdash_\mathcal{L}  \Ellent{A}}
            \end{array}
          }
        }{\Gamma_{{\mathrm{1}}}  \Ellesym{,}  \Ellent{B}  \Ellesym{,}  \Gamma_{{\mathrm{2}}}  \Ellesym{,}   \mathsf{F} \Ellent{Y}   \Ellesym{,}  \Gamma_{{\mathrm{3}}}  \vdash_\mathcal{L}  \Ellent{A}}
      \end{math}
    \end{center}
    By assumption, $c(\Pi_1),c(\Pi_2)\leq |B|$. By induction on $\Pi_1$ and $\pi$, there is a
    proof $\Pi'$ for sequent $\Gamma_{{\mathrm{1}}}  \Ellesym{,}  \Delta  \Ellesym{,}  \Gamma_{{\mathrm{2}}}  \Ellesym{,}  \Ellent{Y}  \Ellesym{,}  \Gamma_{{\mathrm{3}}}  \vdash_\mathcal{L}  \Ellent{A}$ s.t. $c(\Pi') \leq |B|$. Therefore,
    the proof $\Pi$ can be constructed as follows with $c(\Pi) = c(\Pi') \leq |B|$.
    \begin{center}
      \scriptsize
      \begin{math}
        $$\mprset{flushleft}
        \inferrule* [right={\tiny Fl}] {
          {
            \begin{array}{c}
              \Pi' \\
              {\Gamma_{{\mathrm{1}}}  \Ellesym{,}  \Delta  \Ellesym{,}  \Gamma_{{\mathrm{2}}}  \Ellesym{,}  \Ellent{Y}  \Ellesym{,}  \Gamma_{{\mathrm{3}}}  \vdash_\mathcal{L}  \Ellent{A}}
            \end{array}
          }
        }{\Gamma_{{\mathrm{1}}}  \Ellesym{,}  \Delta  \Ellesym{,}  \Gamma_{{\mathrm{2}}}  \Ellesym{,}   \mathsf{F} \Ellent{Y}   \Ellesym{,}  \Gamma_{{\mathrm{3}}}  \vdash_\mathcal{L}  \Ellent{A}}
      \end{math}
    \end{center}

  \item $\ElledruleSXXFlName$ Case 3:
    \begin{center}
      \scriptsize
      \begin{math}
        \begin{array}{c}
          \Pi_1 \\
          {\Phi  \vdash_\mathcal{C}  \Ellent{X}}
        \end{array}
      \end{math}
      \qquad\qquad
      $\Pi_2$:
      \begin{math}
        $$\mprset{flushleft}
        \inferrule* [right={\tiny Fl}] {
          {
            \begin{array}{c}
              \pi \\
              {\Gamma_{{\mathrm{1}}}  \Ellesym{,}  \Ellent{Y}  \Ellesym{,}  \Gamma_{{\mathrm{2}}}  \Ellesym{,}  \Ellent{X}  \Ellesym{,}  \Gamma_{{\mathrm{3}}}  \vdash_\mathcal{L}  \Ellent{A}}
            \end{array}
          }
        }{\Gamma_{{\mathrm{1}}}  \Ellesym{,}   \mathsf{F} \Ellent{Y}   \Ellesym{,}  \Gamma_{{\mathrm{2}}}  \Ellesym{,}  \Ellent{X}  \Ellesym{,}  \Gamma_{{\mathrm{3}}}  \vdash_\mathcal{L}  \Ellent{A}}
      \end{math}
    \end{center}
    By assumption, $c(\Pi_1),c(\Pi_2)\leq |X|$. By induction on $\Pi_1$ and $\pi$, there is a
    proof $\Pi'$ for sequent $\Gamma_{{\mathrm{1}}}  \Ellesym{,}  \Ellent{Y}  \Ellesym{,}  \Gamma_{{\mathrm{2}}}  \Ellesym{,}  \Phi  \Ellesym{,}  \Gamma_{{\mathrm{3}}}  \vdash_\mathcal{L}  \Ellent{A}$ s.t. $c(\Pi') \leq |X|$. Therefore,
    the proof $\Pi$ can be constructed as follows with $c(\Pi) = c(\Pi') \leq |X|$.
    \begin{center}
      \scriptsize
      \begin{math}
        $$\mprset{flushleft}
        \inferrule* [right={\tiny Fl}] {
          {
            \begin{array}{c}
              \Pi' \\
              {\Gamma_{{\mathrm{1}}}  \Ellesym{,}  \Ellent{Y}  \Ellesym{,}  \Gamma_{{\mathrm{2}}}  \Ellesym{,}  \Phi  \Ellesym{,}  \Gamma_{{\mathrm{3}}}  \vdash_\mathcal{L}  \Ellent{A}}
            \end{array}
          }
        }{\Gamma_{{\mathrm{1}}}  \Ellesym{,}   \mathsf{F} \Ellent{Y}   \Ellesym{,}  \Gamma_{{\mathrm{2}}}  \Ellesym{,}  \Phi  \Ellesym{,}  \Gamma_{{\mathrm{3}}}  \vdash_\mathcal{L}  \Ellent{A}}
      \end{math}
    \end{center}

  \item $\ElledruleSXXFlName$ Case 4:
    \begin{center}
      \scriptsize
      \begin{math}
        \begin{array}{c}
          \Pi_1 \\
          {\Delta  \vdash_\mathcal{L}  \Ellent{B}}
        \end{array}
      \end{math}
      \qquad\qquad
      $\Pi_2$:
      \begin{math}
        $$\mprset{flushleft}
        \inferrule* [right={\tiny Fl}] {
          {
            \begin{array}{c}
              \pi \\
              {\Gamma_{{\mathrm{1}}}  \Ellesym{,}  \Ellent{Y}  \Ellesym{,}  \Gamma_{{\mathrm{2}}}  \Ellesym{,}  \Ellent{B}  \Ellesym{,}  \Gamma_{{\mathrm{3}}}  \vdash_\mathcal{L}  \Ellent{A}}
            \end{array}
          }
        }{\Gamma_{{\mathrm{1}}}  \Ellesym{,}   \mathsf{F} \Ellent{Y}   \Ellesym{,}  \Gamma_{{\mathrm{2}}}  \Ellesym{,}  \Delta  \Ellesym{,}  \Gamma_{{\mathrm{3}}}  \vdash_\mathcal{L}  \Ellent{A}}
      \end{math}
    \end{center}
    By assumption, $c(\Pi_1),c(\Pi_2)\leq |B|$. By induction on $\Pi_1$ and $\pi$, there is a
    proof $\Pi'$ for sequent $\Gamma_{{\mathrm{1}}}  \Ellesym{,}  \Ellent{Y}  \Ellesym{,}  \Gamma_{{\mathrm{2}}}  \Ellesym{,}  \Delta  \Ellesym{,}  \Gamma_{{\mathrm{3}}}  \vdash_\mathcal{L}  \Ellent{A}$ s.t. $c(\Pi') \leq |B|$. Therefore,
    the proof $\Pi$ can be constructed as follows with $c(\Pi) = c(\Pi') \leq |B|$.
    \begin{center}
      \scriptsize
      \begin{math}
        $$\mprset{flushleft}
        \inferrule* [right={\tiny Fl}] {
          {
            \begin{array}{c}
              \Pi' \\
              {\Gamma_{{\mathrm{1}}}  \Ellesym{,}  \Ellent{Y}  \Ellesym{,}  \Gamma_{{\mathrm{2}}}  \Ellesym{,}  \Delta  \Ellesym{,}  \Gamma_{{\mathrm{3}}}  \vdash_\mathcal{L}  \Ellent{A}}
            \end{array}
          }
        }{\Gamma_{{\mathrm{1}}}  \Ellesym{,}   \mathsf{F} \Ellent{Y}   \Ellesym{,}  \Gamma_{{\mathrm{2}}}  \Ellesym{,}  \Delta  \Ellesym{,}  \Gamma_{{\mathrm{3}}}  \vdash_\mathcal{L}  \Ellent{A}}
      \end{math}
    \end{center}

  \item $\ElledruleTXXGrName$:
    \begin{center}
      \scriptsize
      \begin{math}
        \begin{array}{c}
          \Pi_1 \\
          {\Phi  \vdash_\mathcal{C}  \Ellent{X}}
        \end{array}
      \end{math}
      \qquad\qquad
      $\Pi_2$:
      \begin{math}
        $$\mprset{flushleft}
        \inferrule* [right={\tiny Gr}] {
          {
            \begin{array}{c}
              \pi \\
              {\Psi_{{\mathrm{1}}}  \Ellesym{,}  \Ellent{X}  \Ellesym{,}  \Psi_{{\mathrm{2}}}  \vdash_\mathcal{L}  \Ellent{A}}
            \end{array}
          }
        }{\Psi_{{\mathrm{1}}}  \Ellesym{,}  \Ellent{X}  \Ellesym{,}  \Psi_{{\mathrm{2}}}  \vdash_\mathcal{C}   \mathsf{G} \Ellent{A} }
      \end{math}
    \end{center}
    By assumption, $c(\Pi_1),c(\Pi_2)\leq |X|$. By induction on $\Pi_1$ and $\pi$, there is a
    proof $\Pi'$ for sequent $\Psi_{{\mathrm{1}}}  \Ellesym{,}  \Phi  \Ellesym{,}  \Psi_{{\mathrm{2}}}  \vdash_\mathcal{L}  \Ellent{A}$ s.t. $c(\Pi') \leq |X|$. Therefore, the
    proof $\Pi$ can be constructed as follows with $c(\Pi) = c(\Pi') \leq |X|$.
    \begin{center}
      \scriptsize
      \begin{math}
        $$\mprset{flushleft}
        \inferrule* [right={\tiny Gr}] {
          {
            \begin{array}{c}
              \Pi' \\
              {\Psi_{{\mathrm{1}}}  \Ellesym{,}  \Phi  \Ellesym{,}  \Psi_{{\mathrm{2}}}  \vdash_\mathcal{L}  \Ellent{A}}
            \end{array}
          }
        }{\Psi_{{\mathrm{1}}}  \Ellesym{,}  \Phi  \Ellesym{,}  \Psi_{{\mathrm{2}}}  \vdash_\mathcal{C}   \mathsf{G} \Ellent{A} }
      \end{math}
    \end{center}

  \item $\ElledruleSXXGlName$ Case 1:
    \begin{center}
      \scriptsize
      \begin{math}
        \begin{array}{c}
          \Pi_1 \\
          {\Phi  \vdash_\mathcal{C}  \Ellent{X}}
        \end{array}
      \end{math}
      \qquad\qquad
      $\Pi_2$:
      \begin{math}
        $$\mprset{flushleft}
        \inferrule* [right={\tiny Gl}] {
          {
            \begin{array}{c}
              \pi \\
              {\Gamma_{{\mathrm{1}}}  \Ellesym{,}  \Ellent{X}  \Ellesym{,}  \Gamma_{{\mathrm{2}}}  \Ellesym{,}  \Ellent{B}  \Ellesym{,}  \Gamma_{{\mathrm{3}}}  \vdash_\mathcal{L}  \Ellent{A}}
            \end{array}
          }
        }{\Gamma_{{\mathrm{1}}}  \Ellesym{,}  \Ellent{X}  \Ellesym{,}  \Gamma_{{\mathrm{2}}}  \Ellesym{,}   \mathsf{G} \Ellent{B}   \Ellesym{,}  \Gamma_{{\mathrm{3}}}  \vdash_\mathcal{L}  \Ellent{A}}
      \end{math}
    \end{center}
    By assumption, $c(\Pi_1),c(\Pi_2)\leq |X|$. By induction on $\Pi_1$ and $\pi$, there is a
    proof $\Pi'$ for sequent $\Gamma_{{\mathrm{1}}}  \Ellesym{,}  \Phi  \Ellesym{,}  \Gamma_{{\mathrm{2}}}  \Ellesym{,}  \Ellent{B}  \Ellesym{,}  \Gamma_{{\mathrm{3}}}  \vdash_\mathcal{L}  \Ellent{A}$ s.t. $c(\Pi') \leq |X|$. Therefore,
    the proof $\Pi$ can be constructed as follows with $c(\Pi) = c(\Pi') \leq |X|$.
    \begin{center}
      \scriptsize
      \begin{math}
        $$\mprset{flushleft}
        \inferrule* [right={\tiny Gl}] {
          {
            \begin{array}{c}
              \Pi' \\
              {\Gamma_{{\mathrm{1}}}  \Ellesym{,}  \Phi  \Ellesym{,}  \Gamma_{{\mathrm{2}}}  \Ellesym{,}  \Ellent{B}  \Ellesym{,}  \Gamma_{{\mathrm{3}}}  \vdash_\mathcal{L}  \Ellent{A}}
            \end{array}
          }
        }{\Gamma_{{\mathrm{1}}}  \Ellesym{,}  \Phi  \Ellesym{,}  \Gamma_{{\mathrm{2}}}  \Ellesym{,}   \mathsf{G} \Ellent{B}   \Ellesym{,}  \Gamma_{{\mathrm{3}}}  \vdash_\mathcal{L}  \Ellent{A}}
      \end{math}
    \end{center}

  \item $\ElledruleSXXGlName$ Case 2:
    \begin{center}
      \scriptsize
      \begin{math}
        \begin{array}{c}
          \Pi_1 \\
          {\Delta  \vdash_\mathcal{L}  \Ellent{B}}
        \end{array}
      \end{math}
      \qquad\qquad
      $\Pi_2$:
      \begin{math}
        $$\mprset{flushleft}
        \inferrule* [right={\tiny Gl}] {
          {
            \begin{array}{c}
              \pi \\
              {\Gamma_{{\mathrm{1}}}  \Ellesym{,}  \Ellent{B}  \Ellesym{,}  \Gamma_{{\mathrm{2}}}  \Ellesym{,}  \Ellent{C}  \Ellesym{,}  \Gamma_{{\mathrm{3}}}  \vdash_\mathcal{L}  \Ellent{A}}
            \end{array}
          }
        }{\Gamma_{{\mathrm{1}}}  \Ellesym{,}  \Ellent{B}  \Ellesym{,}  \Gamma_{{\mathrm{2}}}  \Ellesym{,}   \mathsf{G} \Ellent{C}   \Ellesym{,}  \Gamma_{{\mathrm{3}}}  \vdash_\mathcal{L}  \Ellent{A}}
      \end{math}
    \end{center}
    By assumption, $c(\Pi_1),c(\Pi_2)\leq |B|$. By induction on $\Pi_1$ and $\pi$, there is a
    proof $\Pi'$ for sequent $\Gamma_{{\mathrm{1}}}  \Ellesym{,}  \Delta  \Ellesym{,}  \Gamma_{{\mathrm{2}}}  \Ellesym{,}  \Ellent{C}  \Ellesym{,}  \Gamma_{{\mathrm{3}}}  \vdash_\mathcal{L}  \Ellent{A}$ s.t. $c(\Pi') \leq |B|$. Therefore,
    the proof $\Pi$ can be constructed as follows with $c(\Pi) = c(\Pi') \leq |B|$.
    \begin{center}
      \scriptsize
      \begin{math}
        $$\mprset{flushleft}
        \inferrule* [right={\tiny Gl}] {
          {
            \begin{array}{c}
              \Pi' \\
              {\Gamma_{{\mathrm{1}}}  \Ellesym{,}  \Delta  \Ellesym{,}  \Gamma_{{\mathrm{2}}}  \Ellesym{,}  \Ellent{C}  \Ellesym{,}  \Gamma_{{\mathrm{3}}}  \vdash_\mathcal{L}  \Ellent{A}}
            \end{array}
          }
        }{\Gamma_{{\mathrm{1}}}  \Ellesym{,}  \Delta  \Ellesym{,}  \Gamma_{{\mathrm{2}}}  \Ellesym{,}   \mathsf{G} \Ellent{C}   \Ellesym{,}  \Gamma_{{\mathrm{3}}}  \vdash_\mathcal{L}  \Ellent{A}}
      \end{math}
    \end{center}

  \item $\ElledruleSXXGlName$ Case 3:
    \begin{center}
      \scriptsize
      \begin{math}
        \begin{array}{c}
          \Pi_1 \\
          {\Phi  \vdash_\mathcal{C}  \Ellent{X}}
        \end{array}
      \end{math}
      \qquad\qquad
      $\Pi_2$:
      \begin{math}
        $$\mprset{flushleft}
        \inferrule* [right={\tiny Gl}] {
          {
            \begin{array}{c}
              \pi \\
              {\Gamma_{{\mathrm{1}}}  \Ellesym{,}  \Ellent{B}  \Ellesym{,}  \Gamma_{{\mathrm{2}}}  \Ellesym{,}  \Ellent{X}  \Ellesym{,}  \Gamma_{{\mathrm{3}}}  \vdash_\mathcal{L}  \Ellent{A}}
            \end{array}
          }
        }{\Gamma_{{\mathrm{1}}}  \Ellesym{,}   \mathsf{G} \Ellent{B}   \Ellesym{,}  \Gamma_{{\mathrm{2}}}  \Ellesym{,}  \Ellent{X}  \Ellesym{,}  \Gamma_{{\mathrm{3}}}  \vdash_\mathcal{L}  \Ellent{A}}
      \end{math}
    \end{center}
    By assumption, $c(\Pi_1),c(\Pi_2)\leq |X|$. By induction on $\Pi_1$ and $\pi$, there is a
    proof $\Pi'$ for sequent $\Gamma_{{\mathrm{1}}}  \Ellesym{,}  \Ellent{B}  \Ellesym{,}  \Gamma_{{\mathrm{2}}}  \Ellesym{,}  \Phi  \Ellesym{,}  \Gamma_{{\mathrm{3}}}  \vdash_\mathcal{L}  \Ellent{A}$ s.t. $c(\Pi') \leq |X|$. Therefore,
    the proof $\Pi$ can be constructed as follows with $c(\Pi) = c(\Pi') \leq |X|$.
    \begin{center}
      \scriptsize
      \begin{math}
        $$\mprset{flushleft}
        \inferrule* [right={\tiny Gl}] {
          {
            \begin{array}{c}
              \Pi' \\
              {\Gamma_{{\mathrm{1}}}  \Ellesym{,}  \Ellent{B}  \Ellesym{,}  \Gamma_{{\mathrm{2}}}  \Ellesym{,}  \Phi  \Ellesym{,}  \Gamma_{{\mathrm{3}}}  \vdash_\mathcal{L}  \Ellent{A}}
            \end{array}
          }
        }{\Gamma_{{\mathrm{1}}}  \Ellesym{,}   \mathsf{G} \Ellent{B}   \Ellesym{,}  \Gamma_{{\mathrm{2}}}  \Ellesym{,}  \Phi  \Ellesym{,}  \Gamma_{{\mathrm{3}}}  \vdash_\mathcal{L}  \Ellent{A}}
      \end{math}
    \end{center}

  \item $\ElledruleSXXGlName$ Case 4:
    \begin{center}
      \scriptsize
      \begin{math}
        \begin{array}{c}
          \Pi_1 \\
          {\Delta  \vdash_\mathcal{L}  \Ellent{B}}
        \end{array}
      \end{math}
      \qquad\qquad
      $\Pi_2$:
      \begin{math}
        $$\mprset{flushleft}
        \inferrule* [right={\tiny Gl}] {
          {
            \begin{array}{c}
              \pi \\
              {\Gamma_{{\mathrm{1}}}  \Ellesym{,}  \Ellent{C}  \Ellesym{,}  \Gamma_{{\mathrm{2}}}  \Ellesym{,}  \Ellent{B}  \Ellesym{,}  \Gamma_{{\mathrm{3}}}  \vdash_\mathcal{L}  \Ellent{A}}
            \end{array}
          }
        }{\Gamma_{{\mathrm{1}}}  \Ellesym{,}   \mathsf{G} \Ellent{C}   \Ellesym{,}  \Gamma_{{\mathrm{2}}}  \Ellesym{,}  \Ellent{B}  \Ellesym{,}  \Gamma_{{\mathrm{3}}}  \vdash_\mathcal{L}  \Ellent{A}}
      \end{math}
    \end{center}
    By assumption, $c(\Pi_1),c(\Pi_2)\leq |B|$. By induction on $\Pi_1$ and $\pi$, there is a
    proof $\Pi'$ for sequent $\Gamma_{{\mathrm{1}}}  \Ellesym{,}  \Ellent{C}  \Ellesym{,}  \Gamma_{{\mathrm{2}}}  \Ellesym{,}  \Delta  \Ellesym{,}  \Gamma_{{\mathrm{3}}}  \vdash_\mathcal{L}  \Ellent{A}$ s.t. $c(\Pi') \leq |B|$. Therefore,
    the proof $\Pi$ can be constructed as follows with $c(\Pi) = c(\Pi') \leq |B|$.
    \begin{center}
      \scriptsize
      \begin{math}
        $$\mprset{flushleft}
        \inferrule* [right={\tiny Gl}] {
          {
            \begin{array}{c}
              \Pi' \\
              {\Gamma_{{\mathrm{1}}}  \Ellesym{,}  \Ellent{C}  \Ellesym{,}  \Gamma_{{\mathrm{2}}}  \Ellesym{,}  \Delta  \Ellesym{,}  \Gamma_{{\mathrm{3}}}  \vdash_\mathcal{L}  \Ellent{A}}
            \end{array}
          }
        }{\Gamma_{{\mathrm{1}}}  \Ellesym{,}   \mathsf{G} \Ellent{C}   \Ellesym{,}  \Gamma_{{\mathrm{2}}}  \Ellesym{,}  \Delta  \Ellesym{,}  \Gamma_{{\mathrm{3}}}  \vdash_\mathcal{L}  \Ellent{A}}
      \end{math}
    \end{center}

    \item $\ElledruleSXXcutOneName$ Case 1:
      \begin{center}
        \scriptsize
        \begin{math}
          \begin{array}{c}
            \Pi_1 \\
            {\Phi_{{\mathrm{1}}}  \vdash_\mathcal{C}  \Ellent{X}}
          \end{array}
        \end{math}
        \qquad\qquad
        $\Pi_2$:
        \begin{math}
          $$\mprset{flushleft}
          \inferrule* [right={\tiny cut1}] {
            {
              \begin{array}{cc}
                \pi_1 & \pi_2 \\
                {\Phi_{{\mathrm{2}}}  \vdash_\mathcal{C}  \Ellent{Y}} & {\Gamma_{{\mathrm{1}}}  \Ellesym{,}  \Ellent{X}  \Ellesym{,}  \Gamma_{{\mathrm{2}}}  \Ellesym{,}  \Ellent{Y}  \Ellesym{,}  \Gamma_{{\mathrm{3}}}  \vdash_\mathcal{L}  \Ellent{A}}
              \end{array}
            }
          }{\Gamma_{{\mathrm{1}}}  \Ellesym{,}  \Ellent{X}  \Ellesym{,}  \Gamma_{{\mathrm{2}}}  \Ellesym{,}  \Phi_{{\mathrm{2}}}  \Ellesym{,}  \Gamma_{{\mathrm{3}}}  \vdash_\mathcal{L}  \Ellent{A}}
        \end{math}
      \end{center}
      By assumption, $c(\Pi_1),c(\Pi_2)\leq |X|$. So $|Y|+1 \leq |X|$. By induction on $\Pi_1$
      and $\pi_2$, there is a proof $\Pi'$ for sequent $\Gamma_{{\mathrm{1}}}  \Ellesym{,}  \Phi_{{\mathrm{1}}}  \Ellesym{,}  \Gamma_{{\mathrm{2}}}  \Ellesym{,}  \Ellent{Y}  \Ellesym{,}  \Gamma_{{\mathrm{3}}}  \vdash_\mathcal{L}  \Ellent{A}$ s.t.
      $c(\Pi') \leq |X|$. Therefore, the proof $\Pi$ can be constructed as follows with
      $c(\Pi) = max\{\Pi', |Y|+1\} \leq |X|$.
      \begin{center}
        \scriptsize
        \begin{math}
          $$\mprset{flushleft}
          \inferrule* [right={\tiny cut1}] {
            {
              \begin{array}{cc}
                \pi_1 & \Pi' \\
                {\Phi_{{\mathrm{2}}}  \vdash_\mathcal{C}  \Ellent{Y}} & {\Gamma_{{\mathrm{1}}}  \Ellesym{,}  \Phi_{{\mathrm{1}}}  \Ellesym{,}  \Gamma_{{\mathrm{2}}}  \Ellesym{,}  \Ellent{Y}  \Ellesym{,}  \Gamma_{{\mathrm{3}}}  \vdash_\mathcal{L}  \Ellent{A}}
              \end{array}
            }
          }{\Gamma_{{\mathrm{1}}}  \Ellesym{,}  \Phi_{{\mathrm{1}}}  \Ellesym{,}  \Gamma_{{\mathrm{2}}}  \Ellesym{,}  \Phi_{{\mathrm{2}}}  \Ellesym{,}  \Gamma_{{\mathrm{3}}}  \vdash_\mathcal{L}  \Ellent{A}}
        \end{math}
      \end{center}

    \item $\ElledruleSXXcutOneName$ Case 2:
      \begin{center}
        \scriptsize
        \begin{math}
          \begin{array}{c}
            \Pi_1 \\
            {\Delta  \vdash_\mathcal{L}  \Ellent{B}}
          \end{array}
        \end{math}
        \qquad\qquad
        $\Pi_2$:
        \begin{math}
          $$\mprset{flushleft}
          \inferrule* [right={\tiny cut1}] {
            {
              \begin{array}{cc}
                \pi_1 & \pi_2 \\
                {\Phi  \vdash_\mathcal{C}  \Ellent{Y}} & {\Gamma_{{\mathrm{1}}}  \Ellesym{,}  \Ellent{B}  \Ellesym{,}  \Gamma_{{\mathrm{2}}}  \Ellesym{,}  \Ellent{Y}  \Ellesym{,}  \Gamma_{{\mathrm{3}}}  \vdash_\mathcal{L}  \Ellent{A}}
              \end{array}
            }
          }{\Gamma_{{\mathrm{1}}}  \Ellesym{,}  \Ellent{B}  \Ellesym{,}  \Gamma_{{\mathrm{2}}}  \Ellesym{,}  \Phi  \Ellesym{,}  \Gamma_{{\mathrm{3}}}  \vdash_\mathcal{L}  \Ellent{A}}
        \end{math}
      \end{center}
      By assumption, $c(\Pi_1),c(\Pi_2)\leq |B|$. So $|Y|+1 \leq |X|$. By induction on $\Pi_1$
      and $\pi_2$, there is a proof $\Pi'$ for sequent $\Gamma_{{\mathrm{1}}}  \Ellesym{,}  \Delta  \Ellesym{,}  \Gamma_{{\mathrm{2}}}  \Ellesym{,}  \Ellent{Y}  \Ellesym{,}  \Gamma_{{\mathrm{3}}}  \vdash_\mathcal{L}  \Ellent{A}$ s.t.
      $c(\Pi') \leq |B|$. Therefore, the proof $\Pi$ can be constructed as follows with
      $c(\Pi) = max\{\Pi', |Y|+1\} \leq |B|$.
      \begin{center}
        \scriptsize
        \begin{math}
          $$\mprset{flushleft}
          \inferrule* [right={\tiny cut1}] {
            {
              \begin{array}{cc}
                \pi_1 & \Pi' \\
                {\Phi  \vdash_\mathcal{C}  \Ellent{Y}} & {\Gamma_{{\mathrm{1}}}  \Ellesym{,}  \Delta  \Ellesym{,}  \Gamma_{{\mathrm{2}}}  \Ellesym{,}  \Ellent{Y}  \Ellesym{,}  \Gamma_{{\mathrm{3}}}  \vdash_\mathcal{L}  \Ellent{A}}
              \end{array}
            }
          }{\Gamma_{{\mathrm{1}}}  \Ellesym{,}  \Delta  \Ellesym{,}  \Gamma_{{\mathrm{2}}}  \Ellesym{,}  \Phi  \Ellesym{,}  \Gamma_{{\mathrm{3}}}  \vdash_\mathcal{L}  \Ellent{A}}
        \end{math}
      \end{center}

    \item $\ElledruleSXXcutOneName$ Case 3:
      \begin{center}
        \scriptsize
        \begin{math}
          \begin{array}{c}
            \Pi_1 \\
            {\Phi_{{\mathrm{1}}}  \vdash_\mathcal{C}  \Ellent{X}}
          \end{array}
        \end{math}
        \qquad\qquad
        $\Pi_2$:
        \begin{math}
          $$\mprset{flushleft}
          \inferrule* [right={\tiny cut1}] {
            {
              \begin{array}{cc}
                \pi_1 & \pi_2 \\
                {\Phi_{{\mathrm{2}}}  \vdash_\mathcal{C}  \Ellent{Y}} & {\Gamma_{{\mathrm{1}}}  \Ellesym{,}  \Ellent{Y}  \Ellesym{,}  \Gamma_{{\mathrm{2}}}  \Ellesym{,}  \Ellent{X}  \Ellesym{,}  \Gamma_{{\mathrm{3}}}  \vdash_\mathcal{L}  \Ellent{A}}
              \end{array}
            }
          }{\Gamma_{{\mathrm{1}}}  \Ellesym{,}  \Phi_{{\mathrm{2}}}  \Ellesym{,}  \Gamma_{{\mathrm{2}}}  \Ellesym{,}  \Ellent{X}  \Ellesym{,}  \Gamma_{{\mathrm{3}}}  \vdash_\mathcal{L}  \Ellent{A}}
        \end{math}
      \end{center}
      By assumption, $c(\Pi_1),c(\Pi_2)\leq |X|$. So $|Y|+1 \leq |X|$. By induction on $\Pi_1$
      and $\pi_2$, there is a proof $\Pi'$ for sequent $\Gamma_{{\mathrm{1}}}  \Ellesym{,}  \Ellent{Y}  \Ellesym{,}  \Gamma_{{\mathrm{2}}}  \Ellesym{,}  \Phi_{{\mathrm{1}}}  \Ellesym{,}  \Gamma_{{\mathrm{3}}}  \vdash_\mathcal{L}  \Ellent{A}$ s.t.
      $c(\Pi') \leq |X|$. Therefore, the proof $\Pi$ can be constructed as follows with
      $c(\Pi) = max\{\Pi', |Y|+1\} \leq |X|$.
      \begin{center}
        \scriptsize
        \begin{math}
          $$\mprset{flushleft}
          \inferrule* [right={\tiny cut1}] {
            {
              \begin{array}{cc}
                \pi_1 & \Pi' \\
                {\Phi_{{\mathrm{2}}}  \vdash_\mathcal{C}  \Ellent{Y}} & {\Gamma_{{\mathrm{1}}}  \Ellesym{,}  \Phi_{{\mathrm{2}}}  \Ellesym{,}  \Gamma_{{\mathrm{2}}}  \Ellesym{,}  \Phi_{{\mathrm{1}}}  \Ellesym{,}  \Gamma_{{\mathrm{3}}}  \vdash_\mathcal{L}  \Ellent{A}}
              \end{array}
            }
          }{\Gamma_{{\mathrm{1}}}  \Ellesym{,}  \Phi_{{\mathrm{2}}}  \Ellesym{,}  \Gamma_{{\mathrm{2}}}  \Ellesym{,}  \Phi_{{\mathrm{1}}}  \Ellesym{,}  \Gamma_{{\mathrm{3}}}  \vdash_\mathcal{L}  \Ellent{A}}
        \end{math}
      \end{center}

    \item $\ElledruleSXXcutOneName$ Case 4:
      \begin{center}
        \scriptsize
        \begin{math}
          \begin{array}{c}
            \Pi_1 \\
            {\Delta  \vdash_\mathcal{L}  \Ellent{B}}
          \end{array}
        \end{math}
        \qquad\qquad
        $\Pi_2$:
        \begin{math}
          $$\mprset{flushleft}
          \inferrule* [right={\tiny cut1}] {
            {
              \begin{array}{cc}
                \pi_1 & \pi_2 \\
                {\Phi  \vdash_\mathcal{C}  \Ellent{Y}} & {\Gamma_{{\mathrm{1}}}  \Ellesym{,}  \Ellent{Y}  \Ellesym{,}  \Gamma_{{\mathrm{2}}}  \Ellesym{,}  \Ellent{B}  \Ellesym{,}  \Gamma_{{\mathrm{3}}}  \vdash_\mathcal{L}  \Ellent{A}}
              \end{array}
            }
          }{\Gamma_{{\mathrm{1}}}  \Ellesym{,}  \Phi  \Ellesym{,}  \Gamma_{{\mathrm{2}}}  \Ellesym{,}  \Ellent{B}  \Ellesym{,}  \Gamma_{{\mathrm{3}}}  \vdash_\mathcal{L}  \Ellent{A}}
        \end{math}
      \end{center}
      By assumption, $c(\Pi_1),c(\Pi_2)\leq |B|$. So $|Y|+1 \leq |X|$. By induction on $\Pi_1$
      and $\pi_2$, there is a proof $\Pi'$ for sequent $\Gamma_{{\mathrm{1}}}  \Ellesym{,}  \Ellent{Y}  \Ellesym{,}  \Gamma_{{\mathrm{2}}}  \Ellesym{,}  \Delta  \Ellesym{,}  \Gamma_{{\mathrm{3}}}  \vdash_\mathcal{L}  \Ellent{A}$ s.t.
      $c(\Pi') \leq |B|$. Therefore, the proof $\Pi$ can be constructed as follows with
      $c(\Pi) = max\{\Pi', |Y|+1\} \leq |B|$.
      \begin{center}
        \scriptsize
        \begin{math}
          $$\mprset{flushleft}
          \inferrule* [right={\tiny cut1}] {
            {
              \begin{array}{cc}
                \pi_1 & \Pi' \\
                {\Phi  \vdash_\mathcal{C}  \Ellent{Y}} & {\Gamma_{{\mathrm{1}}}  \Ellesym{,}  \Ellent{Y}  \Ellesym{,}  \Gamma_{{\mathrm{2}}}  \Ellesym{,}  \Delta  \Ellesym{,}  \Gamma_{{\mathrm{3}}}  \vdash_\mathcal{L}  \Ellent{A}}
              \end{array}
            }
          }{\Gamma_{{\mathrm{1}}}  \Ellesym{,}  \Phi  \Ellesym{,}  \Gamma_{{\mathrm{2}}}  \Ellesym{,}  \Delta  \Ellesym{,}  \Gamma_{{\mathrm{3}}}  \vdash_\mathcal{L}  \Ellent{A}}
        \end{math}
      \end{center}

    \item $\ElledruleSXXcutTwoName$ Case 1:
      \begin{center}
        \scriptsize
        \begin{math}
          \begin{array}{c}
            \Pi_1 \\
            {\Phi  \vdash_\mathcal{C}  \Ellent{X}}
          \end{array}
        \end{math}
        \qquad\qquad
        $\Pi_2$:
        \begin{math}
          $$\mprset{flushleft}
          \inferrule* [right={\tiny cut2}] {
            {
              \begin{array}{cc}
                \pi_1 & \pi_2 \\
                {\Delta  \vdash_\mathcal{L}  \Ellent{A}} & {\Gamma_{{\mathrm{1}}}  \Ellesym{,}  \Ellent{X}  \Ellesym{,}  \Gamma_{{\mathrm{2}}}  \Ellesym{,}  \Ellent{A}  \Ellesym{,}  \Gamma_{{\mathrm{3}}}  \vdash_\mathcal{L}  \Ellent{B}}
              \end{array}
            }
          }{\Gamma_{{\mathrm{1}}}  \Ellesym{,}  \Ellent{X}  \Ellesym{,}  \Gamma_{{\mathrm{2}}}  \Ellesym{,}  \Delta  \Ellesym{,}  \Gamma_{{\mathrm{3}}}  \vdash_\mathcal{L}  \Ellent{B}}
        \end{math}
      \end{center}
      By assumption, $c(\Pi_1),c(\Pi_2)\leq |X|$. So $|Y|+1 \leq |X|$. By induction on $\Pi_1$
      and $\pi_2$, there is a proof $\Pi'$ for sequent $\Gamma_{{\mathrm{1}}}  \Ellesym{,}  \Phi  \Ellesym{,}  \Gamma_{{\mathrm{2}}}  \Ellesym{,}  \Ellent{A}  \Ellesym{,}  \Gamma_{{\mathrm{3}}}  \vdash_\mathcal{L}  \Ellent{B}$ s.t.
      $c(\Pi') \leq |X|$. Therefore, the proof $\Pi$ can be constructed as follows with
      $c(\Pi) = max\{\Pi', |Y|+1\} \leq |X|$.
      \begin{center}
        \scriptsize
        \begin{math}
          $$\mprset{flushleft}
          \inferrule* [right={\tiny cut2}] {
            {
              \begin{array}{cc}
                \pi_1 & \Pi' \\
                {\Delta  \vdash_\mathcal{L}  \Ellent{A}} & {\Gamma_{{\mathrm{1}}}  \Ellesym{,}  \Phi  \Ellesym{,}  \Gamma_{{\mathrm{2}}}  \Ellesym{,}  \Ellent{A}  \Ellesym{,}  \Gamma_{{\mathrm{3}}}  \vdash_\mathcal{L}  \Ellent{B}}
              \end{array}
            }
          }{\Gamma_{{\mathrm{1}}}  \Ellesym{,}  \Phi  \Ellesym{,}  \Gamma_{{\mathrm{2}}}  \Ellesym{,}  \Delta  \Ellesym{,}  \Gamma_{{\mathrm{3}}}  \vdash_\mathcal{L}  \Ellent{B}}
        \end{math}
      \end{center}

    \item $\ElledruleSXXcutTwoName$ Case 2:
      \begin{center}
        \scriptsize
        \begin{math}
          \begin{array}{c}
            \Pi_1 \\
            {\Delta_{{\mathrm{1}}}  \vdash_\mathcal{L}  \Ellent{C}}
          \end{array}
        \end{math}
        \qquad\qquad
        $\Pi_2$:
        \begin{math}
          $$\mprset{flushleft}
          \inferrule* [right={\tiny cut2}] {
            {
              \begin{array}{cc}
                \pi_1 & \pi_2 \\
                {\Delta_{{\mathrm{2}}}  \vdash_\mathcal{L}  \Ellent{A}} & {\Gamma_{{\mathrm{1}}}  \Ellesym{,}  \Ellent{C}  \Ellesym{,}  \Gamma_{{\mathrm{2}}}  \Ellesym{,}  \Ellent{A}  \Ellesym{,}  \Gamma_{{\mathrm{3}}}  \vdash_\mathcal{L}  \Ellent{B}}
              \end{array}
            }
          }{\Gamma_{{\mathrm{1}}}  \Ellesym{,}  \Ellent{C}  \Ellesym{,}  \Gamma_{{\mathrm{2}}}  \Ellesym{,}  \Delta_{{\mathrm{2}}}  \Ellesym{,}  \Gamma_{{\mathrm{3}}}  \vdash_\mathcal{L}  \Ellent{B}}
        \end{math}
      \end{center}
      By assumption, $c(\Pi_1),c(\Pi_2)\leq |C|$. So $|Y|+1 \leq |X|$. By induction on $\Pi_1$
      and $\pi_2$, there is a proof $\Pi'$ for sequent $\Gamma_{{\mathrm{1}}}  \Ellesym{,}  \Delta_{{\mathrm{1}}}  \Ellesym{,}  \Gamma_{{\mathrm{2}}}  \Ellesym{,}  \Ellent{A}  \Ellesym{,}  \Gamma_{{\mathrm{3}}}  \vdash_\mathcal{L}  \Ellent{B}$ s.t.
      $c(\Pi') \leq |C|$. Therefore, the proof $\Pi$ can be constructed as follows with
      $c(\Pi) = max\{\Pi', |Y|+1\} \leq |C|$.
      \begin{center}
        \scriptsize
        \begin{math}
          $$\mprset{flushleft}
          \inferrule* [right={\tiny cut2}] {
            {
              \begin{array}{cc}
                \pi_1 & \Pi' \\
                {\Delta_{{\mathrm{2}}}  \vdash_\mathcal{L}  \Ellent{A}} & {\Gamma_{{\mathrm{1}}}  \Ellesym{,}  \Delta_{{\mathrm{1}}}  \Ellesym{,}  \Gamma_{{\mathrm{2}}}  \Ellesym{,}  \Ellent{A}  \Ellesym{,}  \Gamma_{{\mathrm{3}}}  \vdash_\mathcal{L}  \Ellent{B}}
              \end{array}
            }
          }{\Gamma_{{\mathrm{1}}}  \Ellesym{,}  \Delta_{{\mathrm{1}}}  \Ellesym{,}  \Gamma_{{\mathrm{2}}}  \Ellesym{,}  \Delta_{{\mathrm{2}}}  \Ellesym{,}  \Gamma_{{\mathrm{3}}}  \vdash_\mathcal{L}  \Ellent{B}}
        \end{math}
      \end{center}

    \item $\ElledruleSXXcutTwoName$ Case 3:
      \begin{center}
        \scriptsize
        \begin{math}
          \begin{array}{c}
            \Pi_1 \\
            {\Phi  \vdash_\mathcal{C}  \Ellent{X}}
          \end{array}
        \end{math}
        \qquad\qquad
        $\Pi_2$:
        \begin{math}
          $$\mprset{flushleft}
          \inferrule* [right={\tiny cut2}] {
            {
              \begin{array}{cc}
                \pi_1 & \pi_2 \\
                {\Delta  \vdash_\mathcal{L}  \Ellent{A}} & {\Gamma_{{\mathrm{1}}}  \Ellesym{,}  \Ellent{A}  \Ellesym{,}  \Gamma_{{\mathrm{2}}}  \Ellesym{,}  \Ellent{X}  \Ellesym{,}  \Gamma_{{\mathrm{3}}}  \vdash_\mathcal{L}  \Ellent{B}}
              \end{array}
            }
          }{\Gamma_{{\mathrm{1}}}  \Ellesym{,}  \Delta  \Ellesym{,}  \Gamma_{{\mathrm{2}}}  \Ellesym{,}  \Ellent{X}  \Ellesym{,}  \Gamma_{{\mathrm{3}}}  \vdash_\mathcal{L}  \Ellent{B}}
        \end{math}
      \end{center}
      By assumption, $c(\Pi_1),c(\Pi_2)\leq |X|$. So $|Y|+1 \leq |X|$. By induction on $\Pi_1$
      and $\pi_2$, there is a proof $\Pi'$ for sequent $\Gamma_{{\mathrm{1}}}  \Ellesym{,}  \Ellent{A}  \Ellesym{,}  \Gamma_{{\mathrm{2}}}  \Ellesym{,}  \Phi  \Ellesym{,}  \Gamma_{{\mathrm{3}}}  \vdash_\mathcal{L}  \Ellent{B}$ s.t.
      $c(\Pi') \leq |X|$. Therefore, the proof $\Pi$ can be constructed as follows with
      $c(\Pi) = max\{\Pi', |Y|+1\} \leq |X|$.
      \begin{center}
        \scriptsize
        \begin{math}
          $$\mprset{flushleft}
          \inferrule* [right={\tiny cut2}] {
            {
              \begin{array}{cc}
                \pi_1 & \Pi' \\
                {\Delta  \vdash_\mathcal{L}  \Ellent{A}} & {\Gamma_{{\mathrm{1}}}  \Ellesym{,}  \Ellent{A}  \Ellesym{,}  \Gamma_{{\mathrm{2}}}  \Ellesym{,}  \Phi  \Ellesym{,}  \Gamma_{{\mathrm{3}}}  \vdash_\mathcal{L}  \Ellent{B}}
              \end{array}
            }
          }{\Gamma_{{\mathrm{1}}}  \Ellesym{,}  \Delta  \Ellesym{,}  \Gamma_{{\mathrm{2}}}  \Ellesym{,}  \Phi  \Ellesym{,}  \Gamma_{{\mathrm{3}}}  \vdash_\mathcal{L}  \Ellent{B}}
        \end{math}
      \end{center}

    \item $\ElledruleSXXcutTwoName$ Case 4:
      \begin{center}
        \scriptsize
        \begin{math}
          \begin{array}{c}
            \Pi_1 \\
            {\Delta_{{\mathrm{1}}}  \vdash_\mathcal{L}  \Ellent{C}}
          \end{array}
        \end{math}
        \qquad\qquad
        $\Pi_2$:
        \begin{math}
          $$\mprset{flushleft}
          \inferrule* [right={\tiny cut2}] {
            {
              \begin{array}{cc}
                \pi_1 & \pi_2 \\
                {\Delta_{{\mathrm{2}}}  \vdash_\mathcal{L}  \Ellent{A}} & {\Gamma_{{\mathrm{1}}}  \Ellesym{,}  \Ellent{A}  \Ellesym{,}  \Gamma_{{\mathrm{2}}}  \Ellesym{,}  \Ellent{C}  \Ellesym{,}  \Gamma_{{\mathrm{3}}}  \vdash_\mathcal{L}  \Ellent{B}}
              \end{array}
            }
          }{\Gamma_{{\mathrm{1}}}  \Ellesym{,}  \Delta_{{\mathrm{2}}}  \Ellesym{,}  \Gamma_{{\mathrm{2}}}  \Ellesym{,}  \Ellent{C}  \Ellesym{,}  \Gamma_{{\mathrm{3}}}  \vdash_\mathcal{L}  \Ellent{B}}
        \end{math}
      \end{center}
      By assumption, $c(\Pi_1),c(\Pi_2)\leq |C|$. So $|Y|+1 \leq |X|$. By induction on $\Pi_1$
      and $\pi_2$, there is a proof $\Pi'$ for sequent $\Gamma_{{\mathrm{1}}}  \Ellesym{,}  \Ellent{A}  \Ellesym{,}  \Gamma_{{\mathrm{2}}}  \Ellesym{,}  \Delta_{{\mathrm{1}}}  \Ellesym{,}  \Gamma_{{\mathrm{3}}}  \vdash_\mathcal{L}  \Ellent{B}$ s.t.
      $c(\Pi') \leq |C|$. Therefore, the proof $\Pi$ can be constructed as follows with
      $c(\Pi) = max\{\Pi', |Y|+1\} \leq |C|$.
      \begin{center}
        \scriptsize
        \begin{math}
          $$\mprset{flushleft}
          \inferrule* [right={\tiny cut2}] {
            {
              \begin{array}{cc}
                \pi_1 & \Pi' \\
                {\Delta_{{\mathrm{2}}}  \vdash_\mathcal{L}  \Ellent{A}} & {\Gamma_{{\mathrm{1}}}  \Ellesym{,}  \Ellent{A}  \Ellesym{,}  \Gamma_{{\mathrm{2}}}  \Ellesym{,}  \Delta_{{\mathrm{1}}}  \Ellesym{,}  \Gamma_{{\mathrm{3}}}  \vdash_\mathcal{L}  \Ellent{B}}
              \end{array}
            }
          }{\Gamma_{{\mathrm{1}}}  \Ellesym{,}  \Delta_{{\mathrm{2}}}  \Ellesym{,}  \Gamma_{{\mathrm{2}}}  \Ellesym{,}  \Delta_{{\mathrm{1}}}  \Ellesym{,}  \Gamma_{{\mathrm{3}}}  \vdash_\mathcal{L}  \Ellent{B}}
        \end{math}
      \end{center}

    \item $\ElledruleSXXbetaName$ Case 1:
      \begin{center}
        \scriptsize
        \begin{math}
          \begin{array}{c}
            \Pi_1 \\
            {\Phi  \vdash_\mathcal{C}  \Ellent{X}}
          \end{array}
        \end{math}
        \qquad\qquad
        $\Pi_2$:
        \begin{math}
          $$\mprset{flushleft}
          \inferrule* [right={\tiny beta}] {
            {
              \begin{array}{c}
                \pi \\
                {\Gamma_{{\mathrm{1}}}  \Ellesym{,}  \Ellent{X}  \Ellesym{,}  \Gamma_{{\mathrm{2}}}  \Ellesym{,}  \Ellent{Y_{{\mathrm{1}}}}  \Ellesym{,}  \Ellent{Y_{{\mathrm{2}}}}  \Ellesym{,}  \Gamma_{{\mathrm{3}}}  \vdash_\mathcal{L}  \Ellent{A}}
              \end{array}
            }
          }{\Gamma_{{\mathrm{1}}}  \Ellesym{,}  \Ellent{X}  \Ellesym{,}  \Gamma_{{\mathrm{2}}}  \Ellesym{,}  \Ellent{Y_{{\mathrm{2}}}}  \Ellesym{,}  \Ellent{Y_{{\mathrm{1}}}}  \Ellesym{,}  \Gamma_{{\mathrm{3}}}  \vdash_\mathcal{L}  \Ellent{A}}
        \end{math}
      \end{center}
      By assumption, $c(\Pi_1),c(\Pi_2)\leq |X|$. By induction on $\Pi_1$ and $\pi$, there is
      a proof $\Pi'$ for sequent $\Gamma_{{\mathrm{1}}}  \Ellesym{,}  \Phi  \Ellesym{,}  \Gamma_{{\mathrm{2}}}  \Ellesym{,}  \Ellent{Y_{{\mathrm{1}}}}  \Ellesym{,}  \Ellent{Y_{{\mathrm{2}}}}  \Ellesym{,}  \Gamma_{{\mathrm{3}}}  \vdash_\mathcal{L}  \Ellent{A}$ s.t. $c(\Pi') \leq |X|$.
      Therefore, the proof $\Pi$ can be constructed as follows with
      $c(\Pi) = c(\Pi') \leq |X|$.
      \begin{center}
        \scriptsize
        \begin{math}
          $$\mprset{flushleft}
          \inferrule* [right={\tiny cut}] {
            {
              \begin{array}{cc}
                \Pi' \\
                {\Gamma_{{\mathrm{1}}}  \Ellesym{,}  \Phi  \Ellesym{,}  \Gamma_{{\mathrm{2}}}  \Ellesym{,}  \Ellent{Y_{{\mathrm{1}}}}  \Ellesym{,}  \Ellent{Y_{{\mathrm{2}}}}  \Ellesym{,}  \Gamma_{{\mathrm{3}}}  \vdash_\mathcal{L}  \Ellent{A}}
              \end{array}
            }
          }{\Gamma_{{\mathrm{1}}}  \Ellesym{,}  \Phi  \Ellesym{,}  \Gamma_{{\mathrm{2}}}  \Ellesym{,}  \Ellent{Y_{{\mathrm{2}}}}  \Ellesym{,}  \Ellent{Y_{{\mathrm{1}}}}  \Ellesym{,}  \Gamma_{{\mathrm{3}}}  \vdash_\mathcal{L}  \Ellent{A}}
        \end{math}
      \end{center}

    \item $\ElledruleSXXbetaName$ Case 2:
      \begin{center}
        \scriptsize
        \begin{math}
          \begin{array}{c}
            \Pi_1 \\
            {\Delta  \vdash_\mathcal{L}  \Ellent{B}}
          \end{array}
        \end{math}
        \qquad\qquad
        $\Pi_2$:
        \begin{math}
          $$\mprset{flushleft}
          \inferrule* [right={\tiny beta}] {
            {
              \begin{array}{c}
                \pi \\
                {\Gamma_{{\mathrm{1}}}  \Ellesym{,}  \Ellent{B}  \Ellesym{,}  \Gamma_{{\mathrm{2}}}  \Ellesym{,}  \Ellent{Y_{{\mathrm{1}}}}  \Ellesym{,}  \Ellent{Y_{{\mathrm{2}}}}  \Ellesym{,}  \Gamma_{{\mathrm{3}}}  \vdash_\mathcal{L}  \Ellent{A}}
              \end{array}
            }
          }{\Gamma_{{\mathrm{1}}}  \Ellesym{,}  \Ellent{B}  \Ellesym{,}  \Gamma_{{\mathrm{2}}}  \Ellesym{,}  \Ellent{Y_{{\mathrm{2}}}}  \Ellesym{,}  \Ellent{Y_{{\mathrm{1}}}}  \Ellesym{,}  \Gamma_{{\mathrm{3}}}  \vdash_\mathcal{L}  \Ellent{A}}
        \end{math}
      \end{center}
      By assumption, $c(\Pi_1),c(\Pi_2)\leq |X|$. By induction on $\Pi_1$ and $\pi$, there is
      a proof $\Pi'$ for sequent $\Gamma_{{\mathrm{1}}}  \Ellesym{,}  \Delta  \Ellesym{,}  \Gamma_{{\mathrm{2}}}  \Ellesym{,}  \Ellent{Y_{{\mathrm{1}}}}  \Ellesym{,}  \Ellent{Y_{{\mathrm{2}}}}  \Ellesym{,}  \Gamma_{{\mathrm{3}}}  \vdash_\mathcal{L}  \Ellent{A}$ s.t. $c(\Pi') \leq |X|$.
      Therefore, the proof $\Pi$ can be constructed as follows with
      $c(\Pi) = c(\Pi') \leq |X|$.
      \begin{center}
        \scriptsize
        \begin{math}
          $$\mprset{flushleft}
          \inferrule* [right={\tiny cut}] {
            {
              \begin{array}{cc}
                \Pi' \\
                {\Gamma_{{\mathrm{1}}}  \Ellesym{,}  \Delta  \Ellesym{,}  \Gamma_{{\mathrm{2}}}  \Ellesym{,}  \Ellent{Y_{{\mathrm{1}}}}  \Ellesym{,}  \Ellent{Y_{{\mathrm{2}}}}  \Ellesym{,}  \Gamma_{{\mathrm{3}}}  \vdash_\mathcal{L}  \Ellent{A}}
              \end{array}
            }
          }{\Gamma_{{\mathrm{1}}}  \Ellesym{,}  \Delta  \Ellesym{,}  \Gamma_{{\mathrm{2}}}  \Ellesym{,}  \Ellent{Y_{{\mathrm{2}}}}  \Ellesym{,}  \Ellent{Y_{{\mathrm{1}}}}  \Ellesym{,}  \Gamma_{{\mathrm{3}}}  \vdash_\mathcal{L}  \Ellent{A}}
        \end{math}
      \end{center}

    \item $\ElledruleSXXbetaName$ Case 3:
      \begin{center}
        \scriptsize
        \begin{math}
          \begin{array}{c}
            \Pi_1 \\
            {\Phi  \vdash_\mathcal{C}  \Ellent{X}}
          \end{array}
        \end{math}
        \qquad\qquad
        $\Pi_2$:
        \begin{math}
          $$\mprset{flushleft}
          \inferrule* [right={\tiny beta}] {
            {
              \begin{array}{c}
                \pi \\
                {\Gamma_{{\mathrm{1}}}  \Ellesym{,}  \Ellent{Y_{{\mathrm{1}}}}  \Ellesym{,}  \Ellent{Y_{{\mathrm{2}}}}  \Ellesym{,}  \Gamma_{{\mathrm{2}}}  \Ellesym{,}  \Ellent{X}  \Ellesym{,}  \Gamma_{{\mathrm{3}}}  \vdash_\mathcal{L}  \Ellent{A}}
              \end{array}
            }
          }{\Gamma_{{\mathrm{1}}}  \Ellesym{,}  \Ellent{X}  \Ellesym{,}  \Gamma_{{\mathrm{2}}}  \Ellesym{,}  \Ellent{Y_{{\mathrm{2}}}}  \Ellesym{,}  \Ellent{Y_{{\mathrm{1}}}}  \Ellesym{,}  \Gamma_{{\mathrm{3}}}  \vdash_\mathcal{L}  \Ellent{A}}
        \end{math}
      \end{center}
      By assumption, $c(\Pi_1),c(\Pi_2)\leq |X|$. By induction on $\Pi_1$ and $\pi$, there is
      a proof $\Pi'$ for sequent $\Gamma_{{\mathrm{1}}}  \Ellesym{,}  \Ellent{Y_{{\mathrm{1}}}}  \Ellesym{,}  \Ellent{Y_{{\mathrm{2}}}}  \Ellesym{,}  \Gamma_{{\mathrm{2}}}  \Ellesym{,}  \Phi  \Ellesym{,}  \Gamma_{{\mathrm{3}}}  \vdash_\mathcal{L}  \Ellent{A}$ s.t. $c(\Pi') \leq |X|$.
      Therefore, the proof $\Pi$ can be constructed as follows with
      $c(\Pi) = c(\Pi') \leq |X|$.
      \begin{center}
        \scriptsize
        \begin{math}
          $$\mprset{flushleft}
          \inferrule* [right={\tiny cut}] {
            {
              \begin{array}{cc}
                \Pi' \\
                {\Gamma_{{\mathrm{1}}}  \Ellesym{,}  \Ellent{Y_{{\mathrm{1}}}}  \Ellesym{,}  \Ellent{Y_{{\mathrm{2}}}}  \Ellesym{,}  \Gamma_{{\mathrm{2}}}  \Ellesym{,}  \Phi  \Ellesym{,}  \Gamma_{{\mathrm{3}}}  \vdash_\mathcal{L}  \Ellent{A}}
              \end{array}
            }
          }{\Gamma_{{\mathrm{1}}}  \Ellesym{,}  \Ellent{Y_{{\mathrm{2}}}}  \Ellesym{,}  \Ellent{Y_{{\mathrm{1}}}}  \Ellesym{,}  \Gamma_{{\mathrm{2}}}  \Ellesym{,}  \Phi  \Ellesym{,}  \Gamma_{{\mathrm{3}}}  \vdash_\mathcal{L}  \Ellent{A}}
        \end{math}
      \end{center}

    \item $\ElledruleSXXbetaName$ Case 4:
      \begin{center}
        \scriptsize
        \begin{math}
          \begin{array}{c}
            \Pi_1 \\
            {\Delta  \vdash_\mathcal{L}  \Ellent{B}}
          \end{array}
        \end{math}
        \qquad\qquad
        $\Pi_2$:
        \begin{math}
          $$\mprset{flushleft}
          \inferrule* [right={\tiny beta}] {
            {
              \begin{array}{c}
                \pi \\
                {\Gamma_{{\mathrm{1}}}  \Ellesym{,}  \Ellent{Y_{{\mathrm{1}}}}  \Ellesym{,}  \Ellent{Y_{{\mathrm{2}}}}  \Ellesym{,}  \Gamma_{{\mathrm{2}}}  \Ellesym{,}  \Ellent{B}  \Ellesym{,}  \Gamma_{{\mathrm{3}}}  \vdash_\mathcal{L}  \Ellent{A}}
              \end{array}
            }
          }{\Gamma_{{\mathrm{1}}}  \Ellesym{,}  \Ellent{Y_{{\mathrm{2}}}}  \Ellesym{,}  \Ellent{Y_{{\mathrm{1}}}}  \Ellesym{,}  \Gamma_{{\mathrm{2}}}  \Ellesym{,}  \Ellent{B}  \Ellesym{,}  \Gamma_{{\mathrm{3}}}  \vdash_\mathcal{L}  \Ellent{A}}
        \end{math}
      \end{center}
      By assumption, $c(\Pi_1),c(\Pi_2)\leq |X|$. By induction on $\Pi_1$ and $\pi$, there is
      a proof $\Pi'$ for sequent $\Gamma_{{\mathrm{1}}}  \Ellesym{,}  \Ellent{Y_{{\mathrm{1}}}}  \Ellesym{,}  \Ellent{Y_{{\mathrm{2}}}}  \Ellesym{,}  \Gamma_{{\mathrm{2}}}  \Ellesym{,}  \Delta  \Ellesym{,}  \Gamma_{{\mathrm{3}}}  \vdash_\mathcal{L}  \Ellent{A}$ s.t. $c(\Pi') \leq |X|$.
      Therefore, the proof $\Pi$ can be constructed as follows with
      $c(\Pi) = c(\Pi') \leq |X|$.
      \begin{center}
        \scriptsize
        \begin{math}
          $$\mprset{flushleft}
          \inferrule* [right={\tiny cut}] {
            {
              \begin{array}{cc}
                \Pi' \\
                {\Gamma_{{\mathrm{1}}}  \Ellesym{,}  \Ellent{Y_{{\mathrm{1}}}}  \Ellesym{,}  \Ellent{Y_{{\mathrm{2}}}}  \Ellesym{,}  \Gamma_{{\mathrm{2}}}  \Ellesym{,}  \Delta  \Ellesym{,}  \Gamma_{{\mathrm{3}}}  \vdash_\mathcal{L}  \Ellent{A}}
              \end{array}
            }
          }{\Gamma_{{\mathrm{1}}}  \Ellesym{,}  \Ellent{Y_{{\mathrm{2}}}}  \Ellesym{,}  \Ellent{Y_{{\mathrm{1}}}}  \Ellesym{,}  \Gamma_{{\mathrm{2}}}  \Ellesym{,}  \Delta  \Ellesym{,}  \Gamma_{{\mathrm{3}}}  \vdash_\mathcal{L}  \Ellent{A}}
        \end{math}
      \end{center}








  FOR COPYING:





  \end{itemize}






  \end{enumerate}
\end{proof}
















\begin{proof}
  We consider cases according to the classes of the last rules used in $\Pi_1$ and $\Pi_2$.
  \begin{enumerate}
  \item Both proofs end in logical rules which introduce the cut formula, i.e. $\Pi_1$ ends
        in a right rule and $\Pi_2$ in a corresponding left rule.
    \begin{itemize}
    \item $ \mathsf{UnitT} $:
      \begin{center}
        \scriptsize
        $\Pi_1:$
        \begin{math}
          $$\mprset{flushleft}
          \inferrule* [right={\tiny unitR}] {
            \,
          }{ \cdot   \vdash_\mathcal{C}   \mathsf{UnitT} }
        \end{math}
        \qquad\qquad
        $\Pi_2:$
        \begin{math}
          $$\mprset{flushleft}
          \inferrule* [right={\tiny unitL}] {
            {
              \begin{array}{c}
                \pi \\
                {\Phi  \vdash_\mathcal{C}  \Ellent{X}}
              \end{array}
            }
          }{ \mathsf{UnitT}   \Ellesym{,}  \Phi  \vdash_\mathcal{C}  \Ellent{X}}
        \end{math}
      \end{center}
      By assumption, $c(\Pi_1),c(\Pi_2)\leq | \mathsf{UnitT} |$. The proof $\Pi$ is the subproof $\pi$
      in $\Pi_2$ for sequent $\Phi  \vdash_\mathcal{C}  \Ellent{X}$. So $c(\Pi)=c(\Pi_2)\leq | \mathsf{UnitT} |$.

      \begin{center}
        \scriptsize
        $\Pi_1:$
        \begin{math}
          $$\mprset{flushleft}
          \inferrule* [right={\tiny unitR}] {
            \,
          }{ \cdot   \vdash_\mathcal{C}   \mathsf{UnitT} }
        \end{math}
        \qquad\qquad
        $\Pi_2:$
        \begin{math}
          $$\mprset{flushleft}
          \inferrule* [right={\tiny unitL1}] {
            {
              \begin{array}{c}
                \pi \\
                {\Gamma  \vdash_\mathcal{L}  \Ellent{A}}
              \end{array}
            }
          }{ \mathsf{UnitT}   \Ellesym{,}  \Gamma  \vdash_\mathcal{L}  \Ellent{A}}
        \end{math}
      \end{center}
      Similar as above, $\Pi$ is $\pi$.

    \item $\otimes$:
      \begin{center}
        \scriptsize
        $\Pi_1:$
        \begin{math}
          $$\mprset{flushleft}
          \inferrule* [right={\tiny tenR}] {
            {
              \begin{array}{cc}
                \pi_1 & \pi_2 \\
                {\Phi_{{\mathrm{1}}}  \vdash_\mathcal{C}  \Ellent{X}} & {\Phi_{{\mathrm{2}}}  \vdash_\mathcal{C}  \Ellent{Y}}
              \end{array}
            }
          }{\Phi_{{\mathrm{1}}}  \Ellesym{,}  \Phi_{{\mathrm{2}}}  \vdash_\mathcal{C}  \Ellent{X}  \otimes  \Ellent{Y}}
        \end{math}
        \qquad\qquad
        $\Pi_2:$
        \begin{math}
          $$\mprset{flushleft}
          \inferrule* [right={\tiny tenL}] {
            {
              \begin{array}{c}
                \pi_3 \\
                {\Psi_{{\mathrm{1}}}  \Ellesym{,}  \Ellent{X}  \Ellesym{,}  \Ellent{Y}  \Ellesym{,}  \Psi_{{\mathrm{2}}}  \vdash_\mathcal{C}  \Ellent{Z}}
              \end{array}
            }
          }{\Psi_{{\mathrm{1}}}  \Ellesym{,}  \Ellent{X}  \otimes  \Ellent{Y}  \Ellesym{,}  \Psi_{{\mathrm{2}}}  \vdash_\mathcal{C}  \Ellent{Z}}
        \end{math}
      \end{center}
      By assumption, $c(\Pi_1),c(\Pi_2)\leq |\Ellent{X}  \otimes  \Ellent{Y}| = |X|+|Y|+1$. The proof $\Pi$ can be
      constructed as follows, and
      $c(\Pi)\leq max\{c(\pi_1),c(\pi_2),c(\pi_3),|X|+1,|Y|+1\}\leq |X|+|Y|+1 = |\Ellent{X}  \otimes  \Ellent{Y}|$.
      \begin{center}
        \scriptsize
        \begin{math}
          $$\mprset{flushleft}
          \inferrule* [right={\tiny cut}] {
            {
              \begin{array}{c}
                \pi_1 \\
                {\Phi_{{\mathrm{1}}}  \vdash_\mathcal{C}  \Ellent{X}}
              \end{array}
            }
            $$\mprset{flushleft}
            \inferrule* [right={\tiny cut}] {
            {
              \begin{array}{cc}
                \pi_2 & \pi_3 \\
                {\Phi_{{\mathrm{2}}}  \vdash_\mathcal{C}  \Ellent{Y}} & {\Psi_{{\mathrm{1}}}  \Ellesym{,}  \Ellent{X}  \Ellesym{,}  \Ellent{Y}  \Ellesym{,}  \Psi_{{\mathrm{2}}}  \vdash_\mathcal{C}  \Ellent{Z}}
              \end{array}
            }
            }{\Psi_{{\mathrm{1}}}  \Ellesym{,}  \Ellent{X}  \Ellesym{,}  \Phi_{{\mathrm{2}}}  \Ellesym{,}  \Psi_{{\mathrm{2}}}  \vdash_\mathcal{C}  \Ellent{Z}}
          }{\Psi_{{\mathrm{1}}}  \Ellesym{,}  \Phi_{{\mathrm{1}}}  \Ellesym{,}  \Phi_{{\mathrm{2}}}  \Ellesym{,}  \Psi_{{\mathrm{2}}}  \vdash_\mathcal{C}  \Ellent{Z}}
        \end{math}
      \end{center}
    \item $\multimap$:
      \begin{center}
        \scriptsize
        $\Pi_1:$
        \begin{math}
          $$\mprset{flushleft}
          \inferrule* [right={\tiny tenR}] {
            {
              \begin{array}{c}
                \pi_1 \\
                {\Phi_{{\mathrm{1}}}  \Ellesym{,}  \Ellent{X}  \vdash_\mathcal{C}  \Ellent{Y}}
              \end{array}
            }
          }{\Phi_{{\mathrm{1}}}  \vdash_\mathcal{C}  \Ellent{X}  \multimap  \Ellent{Y}}
        \end{math}
        \qquad\qquad
        $\Pi_2:$
        \begin{math}
          $$\mprset{flushleft}
          \inferrule* [right={\tiny tenL}] {
            {
              \begin{array}{cc}
                \pi_2 & \pi_3 \\
                {\Phi_{{\mathrm{2}}}  \vdash_\mathcal{C}  \Ellent{X}} & {\Psi_{{\mathrm{1}}}  \Ellesym{,}  \Ellent{Y}  \Ellesym{,}  \Psi_{{\mathrm{2}}}  \vdash_\mathcal{C}  \Ellent{Z}}
              \end{array}
            }
          }{\Psi_{{\mathrm{1}}}  \Ellesym{,}  \Ellent{X}  \multimap  \Ellent{Y}  \Ellesym{,}  \Phi  \Ellesym{,}  \Psi_{{\mathrm{2}}}  \vdash_\mathcal{C}  \Ellent{Z}}
        \end{math}
      \end{center}
      By assumption, $c(\Pi_1),c(\Pi_2)\leq |\Ellent{X}  \multimap  \Ellent{Y}| = |X|+|Y|+1$. The proof $\Pi$ is
      constructed as follows
      $c(\Pi)\leq max\{c(\pi_1),c(\pi_2),c(\pi_3),|X|+1,|Y|+1\}\leq |X|+|Y|+1 = |\Ellent{X}  \multimap  \Ellent{Y}|$.
      \begin{center}
        \scriptsize
        \begin{math}
          $$\mprset{flushleft}
          \inferrule* [right={\tiny tenR}] {
            $$\mprset{flushleft}
            \inferrule* [right={\tiny tenR}] {
              {
                \begin{array}{cc}
                  \pi_1 & \pi_2 \\
                  {\Phi_{{\mathrm{1}}}  \Ellesym{,}  \Ellent{X}  \vdash_\mathcal{C}  \Ellent{Y}} & {\Phi_{{\mathrm{2}}}  \vdash_\mathcal{C}  \Ellent{X}}
                \end{array}
              }
            }{\Phi_{{\mathrm{1}}}  \Ellesym{,}  \Phi_{{\mathrm{2}}}  \vdash_\mathcal{C}  \Ellent{Y}} \\
             {
               \begin{array}{c}
                 \pi_3 \\
                 {\Psi_{{\mathrm{1}}}  \Ellesym{,}  \Ellent{Y}  \Ellesym{,}  \Psi_{{\mathrm{2}}}  \vdash_\mathcal{C}  \Ellent{Z}}
               \end{array}
             }
          }{\Psi_{{\mathrm{1}}}  \Ellesym{,}  \Phi_{{\mathrm{1}}}  \Ellesym{,}  \Phi_{{\mathrm{2}}}  \Ellesym{,}  \Psi_{{\mathrm{2}}}  \vdash_\mathcal{C}  \Ellent{Z}}
        \end{math}
      \end{center}
    \item $ \mathsf{UnitS} $:
      \begin{center}
        \scriptsize
        $\Pi_1:$
        \begin{math}
          $$\mprset{flushleft}
          \inferrule* [right={\tiny unitR}] {
            \,
          }{ \cdot   \vdash_\mathcal{L}   \mathsf{UnitS} }
        \end{math}
        \qquad\qquad
        $\Pi_2:$
        \begin{math}
          $$\mprset{flushleft}
          \inferrule* [right={\tiny unitL2}] {
            {
              \begin{array}{c}
                \pi \\
                {\Delta  \vdash_\mathcal{L}  \Ellent{A}}
              \end{array}
            }
          }{ \mathsf{UnitS}   \Ellesym{,}  \Delta  \vdash_\mathcal{L}  \Ellent{A}}
        \end{math}
      \end{center}
      By assumption, $c(\Pi_1),c(\Pi_2)\leq | \mathsf{UnitS} |$. The proof $\Pi$ is the subproof $\pi$
      in $\Pi_2$ for sequent $\Delta  \vdash_\mathcal{L}  \Ellent{A}$. So $c(\Pi)=c(\Pi_2)\leq | \mathsf{UnitS} |$.

    \item $\tri$:
      \begin{center}
        \scriptsize
        $\Pi_1:$
        \begin{math}
          $$\mprset{flushleft}
          \inferrule* [right={\tiny tenR}] {
            {
              \begin{array}{cc}
                \pi_1 & \pi_2 \\
                {\Gamma_{{\mathrm{1}}}  \vdash_\mathcal{L}  \Ellent{A}} & {\Gamma_{{\mathrm{2}}}  \vdash_\mathcal{L}  \Ellent{B}}
              \end{array}
            }
          }{\Gamma_{{\mathrm{1}}}  \Ellesym{,}  \Gamma_{{\mathrm{2}}}  \vdash_\mathcal{L}  \Ellent{A}  \triangleright  \Ellent{B}}
        \end{math}
        \qquad\qquad
        $\Pi_2:$
        \begin{math}
          $$\mprset{flushleft}
          \inferrule* [right={\tiny tenL1}] {
            {
              \begin{array}{c}
                \pi_3 \\
                {\Delta_{{\mathrm{1}}}  \Ellesym{,}  \Ellent{A}  \Ellesym{,}  \Ellent{B}  \Ellesym{,}  \Delta_{{\mathrm{2}}}  \vdash_\mathcal{L}  \Ellent{C}}
              \end{array}
            }
          }{\Delta_{{\mathrm{1}}}  \Ellesym{,}  \Ellent{A}  \triangleright  \Ellent{B}  \Ellesym{,}  \Delta_{{\mathrm{2}}}  \vdash_\mathcal{L}  \Ellent{C}}
        \end{math}
      \end{center}
      By assumption, $c(\Pi_1),c(\Pi_2)\leq |\Ellent{A}  \triangleright  \Ellent{B}| = |X|+|Y|+1$. The proof $\Pi$ can be
      constructed as follows, and
      $c(\Pi)\leq max\{c(\pi_1),c(\pi_2),c(\pi_3),|A|+1,|B|+1\}\leq |A|+|B|+1 = |\Ellent{A}  \triangleright  \Ellent{B}|$.
      \begin{center}
        \scriptsize
        \begin{math}
          $$\mprset{flushleft}
          \inferrule* [right={\tiny cut2}] {
            {
              \begin{array}{c}
                \pi_1 \\
                {\Gamma_{{\mathrm{1}}}  \vdash_\mathcal{L}  \Ellent{A}}
              \end{array}
            }
            $$\mprset{flushleft}
            \inferrule* [right={\tiny cut2}] {
            {
              \begin{array}{cc}
                \pi_2 & \pi_3 \\
                {\Gamma_{{\mathrm{2}}}  \vdash_\mathcal{L}  \Ellent{B}} & {\Delta_{{\mathrm{1}}}  \Ellesym{,}  \Ellent{A}  \Ellesym{,}  \Ellent{B}  \Ellesym{,}  \Delta_{{\mathrm{2}}}  \vdash_\mathcal{L}  \Ellent{C}}
              \end{array}
            }
            }{\Delta_{{\mathrm{1}}}  \Ellesym{,}  \Ellent{A}  \Ellesym{,}  \Gamma_{{\mathrm{2}}}  \Ellesym{,}  \Delta_{{\mathrm{2}}}  \vdash_\mathcal{L}  \Ellent{C}}
          }{\Delta_{{\mathrm{1}}}  \Ellesym{,}  \Gamma_{{\mathrm{1}}}  \Ellesym{,}  \Gamma_{{\mathrm{2}}}  \Ellesym{,}  \Psi_{{\mathrm{2}}}  \vdash_\mathcal{L}  \Ellent{C}}
        \end{math}
      \end{center}
    \item $\lto$:
      \begin{center}
        \scriptsize
        $\Pi_1:$
        \begin{math}
          $$\mprset{flushleft}
          \inferrule* [right={\tiny imprR}] {
            {
              \begin{array}{c}
                \pi_1 \\
                {\Gamma  \Ellesym{,}  \Ellent{A}  \vdash_\mathcal{L}  \Ellent{B}}
              \end{array}
            }
          }{\Gamma  \vdash_\mathcal{L}  \Ellent{A}  \rightharpoonup  \Ellent{B}}
        \end{math}
        \qquad\qquad
        $\Pi_2:$
        \begin{math}
          $$\mprset{flushleft}
          \inferrule* [right={\tiny imprL}] {
            {
              \begin{array}{cc}
                \pi_2 & \pi_3 \\
                {\Delta_{{\mathrm{1}}}  \vdash_\mathcal{L}  \Ellent{A}} & {\Delta_{{\mathrm{2}}}  \Ellesym{,}  \Ellent{B}  \vdash_\mathcal{L}  \Ellent{C}}
              \end{array}
            }
          }{\Delta_{{\mathrm{2}}}  \Ellesym{,}  \Ellent{A}  \rightharpoonup  \Ellent{B}  \Ellesym{,}  \Delta_{{\mathrm{1}}}  \vdash_\mathcal{L}  \Ellent{C}}
        \end{math}
      \end{center}
      By assumption, $c(\Pi_1),c(\Pi_2)\leq |\Ellent{A}  \rightharpoonup  \Ellent{B}| = |A|+|B|+1$. The proof $\Pi$ is
      constructed as follows, and
      $c(\Pi)\leq max\{c(\pi_1),c(\pi_2),c(\pi_3),|A|+1,|B|+1\}\leq |A|+|B|+1 = |\Ellent{A}  \rightharpoonup  \Ellent{B}|$.
      \begin{center}
        \scriptsize
        \begin{math}
          $$\mprset{flushleft}
          \inferrule* [right={\tiny cut2}] {
            $$\mprset{flushleft}
            \inferrule* [right={\tiny cut2}] {
              {
                \begin{array}{cc}
                  \pi_1 & \pi_2 \\
                  {\Gamma  \Ellesym{,}  \Ellent{A}  \vdash_\mathcal{L}  \Ellent{B}} & {\Delta_{{\mathrm{1}}}  \vdash_\mathcal{L}  \Ellent{A}}
                \end{array}
              }
            }{\Gamma  \Ellesym{,}  \Delta_{{\mathrm{1}}}  \vdash_\mathcal{L}  \Ellent{B}}
             {
               \begin{array}{c}
                 \pi_3 \\
                 {\Delta_{{\mathrm{2}}}  \Ellesym{,}  \Ellent{B}  \vdash_\mathcal{L}  \Ellent{C}}
               \end{array}
             }
          }{\Delta_{{\mathrm{2}}}  \Ellesym{,}  \Gamma  \Ellesym{,}  \Delta_{{\mathrm{1}}}  \vdash_\mathcal{L}  \Ellent{C}}
        \end{math}
      \end{center}

    \item $\rto$:
      \begin{center}
        \scriptsize
        $\Pi_1:$
        \begin{math}
          $$\mprset{flushleft}
          \inferrule* [right={\tiny implR}] {
            {
              \begin{array}{c}
                \pi_1 \\
                {\Ellent{A}  \Ellesym{,}  \Gamma  \vdash_\mathcal{L}  \Ellent{B}}
              \end{array}
            }
          }{\Gamma  \vdash_\mathcal{L}  \Ellent{B}  \leftharpoonup  \Ellent{A}}
        \end{math}
        \qquad\qquad
        $\Pi_2:$
        \begin{math}
          $$\mprset{flushleft}
          \inferrule* [right={\tiny implL}] {
            {
              \begin{array}{cc}
                \pi_2 & \pi_3 \\
                {\Delta_{{\mathrm{1}}}  \vdash_\mathcal{L}  \Ellent{A}} & {\Ellent{B}  \Ellesym{,}  \Delta_{{\mathrm{2}}}  \vdash_\mathcal{L}  \Ellent{C}}
              \end{array}
            }
          }{\Delta_{{\mathrm{1}}}  \Ellesym{,}  \Ellent{B}  \leftharpoonup  \Ellent{A}  \Ellesym{,}  \Delta_{{\mathrm{2}}}  \vdash_\mathcal{L}  \Ellent{C}}
        \end{math}
      \end{center}
      By assumption, $c(\Pi_1),c(\Pi_2)\leq |\Ellent{B}  \leftharpoonup  \Ellent{A}| = |A|+|B|+1$. The proof $\Pi$ is
      constructed as follows, and
      $c(\Pi)\leq max\{c(\pi_1),c(\pi_2),c(\pi_3),|A|+1,|B|+1\}\leq |A|+|B|+1 = |\Ellent{B}  \leftharpoonup  \Ellent{A}|$.
      \begin{center}
        \scriptsize
        \begin{math}
          $$\mprset{flushleft}
          \inferrule* [right={\tiny cut1}] {
            $$\mprset{flushleft}
            \inferrule* [right={\tiny cut2}] {
              {
                \begin{array}{cc}
                  \pi_1 & \pi_2 \\
                  {\Ellent{A}  \Ellesym{,}  \Gamma  \vdash_\mathcal{L}  \Ellent{B}} & {\Delta_{{\mathrm{1}}}  \vdash_\mathcal{L}  \Ellent{A}}
                \end{array}
              }
            }{\Delta_{{\mathrm{1}}}  \Ellesym{,}  \Gamma  \vdash_\mathcal{L}  \Ellent{B}}
             {
               \begin{array}{c}
                 \pi_3 \\
                 {\Ellent{B}  \Ellesym{,}  \Delta_{{\mathrm{2}}}  \vdash_\mathcal{L}  \Ellent{C}}
               \end{array}
             }
          }{\Delta_{{\mathrm{1}}}  \Ellesym{,}  \Gamma  \Ellesym{,}  \Delta_{{\mathrm{2}}}  \vdash_\mathcal{L}  \Ellent{C}}
        \end{math}
      \end{center}

    \item $F$:
      \begin{center}
        \scriptsize
        $\Pi_1:$
        \begin{math}
          $$\mprset{flushleft}
          \inferrule* [right={\tiny FR}] {
            {
              \begin{array}{c}
                \pi_1 \\
                {\Phi  \vdash_\mathcal{C}  \Ellent{X}}
              \end{array}
            }
          }{\Phi  \vdash_\mathcal{L}   \mathsf{F} \Ellent{X} }
        \end{math}
        \qquad\qquad
        $\Pi_2:$
        \begin{math}
          $$\mprset{flushleft}
          \inferrule* [right={\tiny FL}] {
            {
              \begin{array}{c}
                \pi_2 \\
                {\Gamma  \Ellesym{,}  \Ellent{X}  \Ellesym{,}  \Delta  \vdash_\mathcal{L}  \Ellent{A}}
              \end{array}
            }
          }{\Gamma  \Ellesym{,}   \mathsf{F} \Ellent{X}   \Ellesym{,}  \Delta  \vdash_\mathcal{L}  \Ellent{A}}
        \end{math}
      \end{center}
      By assumption, $c(\Pi_1),c(\Pi_2)\leq | \mathsf{F} \Ellent{X} | = |X|+1$. The proof $\Pi$ is
      constructed as follows, and $c(\Pi)\leq max\{c(\pi_1),c(\pi_2),|X|+1\}\leq | \mathsf{F} \Ellent{X} |$.
      \begin{center}
        \scriptsize
        \begin{math}
          $$\mprset{flushleft}
          \inferrule* [right={\tiny cut2}] {
            {
              \begin{array}{cc}
                \pi_1 & \pi_2 \\
                {\Phi  \vdash_\mathcal{C}  \Ellent{X}} & {\Gamma  \Ellesym{,}  \Ellent{X}  \Ellesym{,}  \Delta  \vdash_\mathcal{L}  \Ellent{A}}
              \end{array}
            }
          }{\Gamma  \Ellesym{,}  \Phi  \Ellesym{,}  \Delta  \vdash_\mathcal{L}  \Ellent{A}}
        \end{math}
      \end{center}

    \item $G$:
      \begin{center}
        \scriptsize
        $\Pi_1:$
        \begin{math}
          $$\mprset{flushleft}
          \inferrule* [right={\tiny GR}] {
            {
              \begin{array}{c}
                \pi_1 \\
                {\Phi  \vdash_\mathcal{L}  \Ellent{A}}
              \end{array}
            }
          }{\Phi  \vdash_\mathcal{C}   \mathsf{G} \Ellent{A} }
        \end{math}
        \qquad\qquad
        $\Pi_2:$
        \begin{math}
          $$\mprset{flushleft}
          \inferrule* [right={\tiny GL}] {
            {
              \begin{array}{c}
                \pi_2 \\
                {\Gamma  \Ellesym{,}  \Ellent{A}  \Ellesym{,}  \Delta  \vdash_\mathcal{L}  \Ellent{B}}
              \end{array}
            }
          }{\Gamma  \Ellesym{,}   \mathsf{G} \Ellent{A}   \Ellesym{,}  \Delta  \vdash_\mathcal{L}  \Ellent{B}}
        \end{math}
      \end{center}
      By assumption, $c(\Pi_1),c(\Pi_2)\leq | \mathsf{G} \Ellent{A} | = |A|+1$. The proof $\Pi$ is
      constructed as follows, and $c(\Pi)\leq max\{c(\pi_1),c(\pi_2),|A|+1\}\leq | \mathsf{G} \Ellent{A} |$.
      \begin{center}
        \scriptsize
        \begin{math}
          $$\mprset{flushleft}
          \inferrule* [right={\tiny GL}] {
            {
              \begin{array}{cc}
                \pi_1 & \pi_2 \\
                {\Phi  \vdash_\mathcal{L}  \Ellent{A}} & {\Gamma  \Ellesym{,}  \Ellent{A}  \Ellesym{,}  \Delta  \vdash_\mathcal{L}  \Ellent{B}}
              \end{array}
            }
          }{\Gamma  \Ellesym{,}  \Phi  \Ellesym{,}  \Delta  \vdash_\mathcal{L}  \Ellent{B}}
        \end{math}
      \end{center}
    \end{itemize}

  \item The last rule used in $\Pi_1$ is not a right logical rule.
    \begin{itemize}
    \item \ElledruleTXXcutName / $\cat{C}$-sequent:
      \begin{center}
        \scriptsize
        $\Pi_1$:
        \begin{math}
          $$\mprset{flushleft}
          \inferrule* [right={\tiny cut}] {
            {
              \begin{array}{cc}
                \pi_1 & \pi_2 \\
                {\Phi_{{\mathrm{1}}}  \vdash_\mathcal{C}  \Ellent{X}} & {\Phi_{{\mathrm{2}}}  \Ellesym{,}  \Ellent{X}  \Ellesym{,}  \Phi_{{\mathrm{3}}}  \vdash_\mathcal{C}  \Ellent{Y}}
              \end{array}
            }
          }{\Phi_{{\mathrm{2}}}  \Ellesym{,}  \Phi_{{\mathrm{1}}}  \Ellesym{,}  \Phi_{{\mathrm{3}}}  \vdash_\mathcal{C}  \Ellent{Y}}
        \end{math}
        \qquad\qquad
        \begin{math}
          \begin{array}{c}
            \Pi_2 \\
            {\Psi_{{\mathrm{1}}}  \Ellesym{,}  \Ellent{Y}  \Ellesym{,}  \Psi_{{\mathrm{2}}}  \vdash_\mathcal{C}  \Ellent{Z}}
          \end{array}
        \end{math}
      \end{center}
      By assumption, $c(\Pi_1),c(\Pi_2)\leq |Y|$. Since the cut rank of the last cut in
      $\Pi_1$ is $|X|+1$, then $|X|+1\leq |Y|$. By induction on the length of $\Pi_1$ and
      $\Pi_2$, the induction hypothesis states that there is a proof $\Pi'$ constructed from
      $\pi_2$ and $\Pi_2$ for sequent $\Psi_{{\mathrm{1}}}  \Ellesym{,}  \Phi_{{\mathrm{2}}}  \Ellesym{,}  \Ellent{X}  \Ellesym{,}  \Phi_{{\mathrm{3}}}  \Ellesym{,}  \Psi_{{\mathrm{2}}}  \vdash_\mathcal{C}  \Ellent{Z}$ s.t. $c(\Pi')\leq|Y|$.
      Therefore, the proof $\Pi$ can be constructed as follows, and
      $c(\Pi)\leq max\{c(\pi_1),c(\Pi'),|X|+1\}\leq |Y|$.
      \begin{center}
        \scriptsize
        \begin{math}
          $$\mprset{flushleft}
          \inferrule* [right={\tiny cut}] {
            {
              \begin{array}{c}
                \pi_1 \\
                {\Phi_{{\mathrm{1}}}  \vdash_\mathcal{C}  \Ellent{X}}
              \end{array}
            }
            $$\mprset{flushleft}
            \inferrule* [right={\tiny cut}] {
              {
                \begin{array}{cc}
                  \pi_2 & \Pi_2 \\
                  {\Phi_{{\mathrm{2}}}  \Ellesym{,}  \Ellent{X}  \Ellesym{,}  \Phi_{{\mathrm{3}}}  \vdash_\mathcal{C}  \Ellent{Y}} & {\Psi_{{\mathrm{1}}}  \Ellesym{,}  \Ellent{Y}  \Ellesym{,}  \Psi_{{\mathrm{2}}}  \vdash_\mathcal{C}  \Ellent{Z}}
                \end{array}
              }
            }{\Psi_{{\mathrm{1}}}  \Ellesym{,}  \Phi_{{\mathrm{2}}}  \Ellesym{,}  \Ellent{X}  \Ellesym{,}  \Phi_{{\mathrm{3}}}  \Ellesym{,}  \Psi_{{\mathrm{2}}}  \vdash_\mathcal{C}  \Ellent{Z}}
          }{\Psi_{{\mathrm{1}}}  \Ellesym{,}  \Phi_{{\mathrm{2}}}  \Ellesym{,}  \Phi_{{\mathrm{1}}}  \Ellesym{,}  \Phi_{{\mathrm{3}}}  \Ellesym{,}  \Psi_{{\mathrm{2}}}  \vdash_\mathcal{C}  \Ellent{Z}}
        \end{math}
      \end{center}

    \item \ElledruleTXXcutName / $\cat{L}$-sequent:
      \begin{center}
        \scriptsize
        $\Pi_1$:
        \begin{math}
          $$\mprset{flushleft}
          \inferrule* [right={\tiny cut}] {
            {
              \begin{array}{cc}
                \pi_1 & \pi_2 \\
                {\Phi  \vdash_\mathcal{C}  \Ellent{X}} & {\Psi_{{\mathrm{1}}}  \Ellesym{,}  \Ellent{X}  \Ellesym{,}  \Psi_{{\mathrm{2}}}  \vdash_\mathcal{C}  \Ellent{Y}}
              \end{array}
            }
          }{\Psi_{{\mathrm{1}}}  \Ellesym{,}  \Phi  \Ellesym{,}  \Psi_{{\mathrm{2}}}  \vdash_\mathcal{C}  \Ellent{Y}}
        \end{math}
        \qquad\qquad
        \begin{math}
          \begin{array}{c}
            \Pi_2 \\
            {\Gamma_{{\mathrm{1}}}  \Ellesym{,}  \Ellent{Y}  \Ellesym{,}  \Gamma_{{\mathrm{2}}}  \vdash_\mathcal{L}  \Ellent{A}}
          \end{array}
        \end{math}
      \end{center}
      By assumption, $c(\Pi_1),c(\Pi_2)\leq |Y|$. Similar as above, $|X|+1\leq |Y|$ and there
      is a proof $\Pi'$ constructed from $\pi_2$ and $\Pi_2$ for sequent
      $\Gamma_{{\mathrm{1}}}  \Ellesym{,}  \Psi_{{\mathrm{1}}}  \Ellesym{,}  \Ellent{X}  \Ellesym{,}  \Psi_{{\mathrm{2}}}  \Ellesym{,}  \Gamma_{{\mathrm{2}}}  \vdash_\mathcal{L}  \Ellent{A}$ s.t. $c(\Pi')\leq|Y|$. Therefore, the proof $\Pi$ can be
      constructed as follows, and $c(\Pi)\leq max\{c(\pi_1),c(\Pi'),|X|+1\}\leq |Y|$.
      \begin{center}
        \scriptsize
        \begin{math}
          $$\mprset{flushleft}
          \inferrule* [right={\tiny cut}] {
            {
              \begin{array}{c}
                \pi_1 \\
                {\Phi  \vdash_\mathcal{C}  \Ellent{X}}
              \end{array}
            }
            $$\mprset{flushleft}
            \inferrule* [right={\tiny cut}] {
              {
                \begin{array}{cc}
                  \pi_2 & \Pi_2 \\
                  {\Psi_{{\mathrm{1}}}  \Ellesym{,}  \Ellent{X}  \Ellesym{,}  \Psi_{{\mathrm{2}}}  \vdash_\mathcal{C}  \Ellent{Y}} & {\Gamma_{{\mathrm{1}}}  \Ellesym{,}  \Ellent{Y}  \Ellesym{,}  \Gamma_{{\mathrm{2}}}  \vdash_\mathcal{L}  \Ellent{A}}
                \end{array}
              }
            }{\Gamma_{{\mathrm{1}}}  \Ellesym{,}  \Psi_{{\mathrm{1}}}  \Ellesym{,}  \Ellent{X}  \Ellesym{,}  \Psi_{{\mathrm{2}}}  \Ellesym{,}  \Gamma_{{\mathrm{2}}}  \vdash_\mathcal{L}  \Ellent{A}}
          }{\Gamma_{{\mathrm{1}}}  \Ellesym{,}  \Psi_{{\mathrm{1}}}  \Ellesym{,}  \Phi  \Ellesym{,}  \Psi_{{\mathrm{2}}}  \Ellesym{,}  \Gamma_{{\mathrm{2}}}  \vdash_\mathcal{L}  \Ellent{A}}
        \end{math}
      \end{center}

    \item \ElledruleSXXcutOneName / $\cat{L}$-sequent:
      \begin{center}
        \scriptsize
        $\Pi_1$:
        \begin{math}
          $$\mprset{flushleft}
          \inferrule* [right={\tiny cut}] {
            {
              \begin{array}{cc}
                \pi_1 & \pi_2 \\
                {\Phi  \vdash_\mathcal{C}  \Ellent{X}} & {\Gamma_{{\mathrm{1}}}  \Ellesym{,}  \Ellent{X}  \Ellesym{,}  \Gamma_{{\mathrm{2}}}  \vdash_\mathcal{L}  \Ellent{A}}
              \end{array}
            }
          }{\Gamma_{{\mathrm{1}}}  \Ellesym{,}  \Phi  \Ellesym{,}  \Gamma_{{\mathrm{2}}}  \vdash_\mathcal{L}  \Ellent{A}}
        \end{math}
        \qquad\qquad
        \begin{math}
          \begin{array}{c}
            \Pi_2 \\
            {\Delta_{{\mathrm{1}}}  \Ellesym{,}  \Ellent{A}  \Ellesym{,}  \Delta_{{\mathrm{2}}}  \vdash_\mathcal{L}  \Ellent{B}}
          \end{array}
        \end{math}
      \end{center}
      By assumption, $c(\Pi_1),c(\Pi_2)\leq |A|$. Similar as above, $|X|+1\leq |A|$ and there
      is a proof $\Pi'$ constructed from $\pi_2$ and $\Pi_2$ for sequent
      $\Delta_{{\mathrm{1}}}  \Ellesym{,}  \Gamma_{{\mathrm{1}}}  \Ellesym{,}  \Ellent{X}  \Ellesym{,}  \Gamma_{{\mathrm{2}}}  \Ellesym{,}  \Delta_{{\mathrm{2}}}  \vdash_\mathcal{L}  \Ellent{B}$ s.t. $c(\Pi')\leq|A|$. Therefore, the proof $\Pi$ can be
      constructed as follows, and $c(\Pi)\leq max\{c(\pi_1),c(\Pi'),|X|+1\}\leq |A|$.
      \begin{center}
        \scriptsize
        \begin{math}
          $$\mprset{flushleft}
          \inferrule* [right={\tiny cut}] {
            {
              \begin{array}{c}
                \pi_1 \\
                {\Phi  \vdash_\mathcal{C}  \Ellent{X}}
              \end{array}
            }
            $$\mprset{flushleft}
            \inferrule* [right={\tiny cut}] {
              {
                \begin{array}{cc}
                  \pi_2 & \Pi_2 \\
                  {\Gamma_{{\mathrm{1}}}  \Ellesym{,}  \Ellent{X}  \Ellesym{,}  \Gamma_{{\mathrm{2}}}  \vdash_\mathcal{L}  \Ellent{A}} & {\Delta_{{\mathrm{1}}}  \Ellesym{,}  \Ellent{A}  \Ellesym{,}  \Delta_{{\mathrm{2}}}  \vdash_\mathcal{L}  \Ellent{B}}
                \end{array}
              }
            }{\Delta_{{\mathrm{1}}}  \Ellesym{,}  \Gamma_{{\mathrm{1}}}  \Ellesym{,}  \Ellent{X}  \Ellesym{,}  \Gamma_{{\mathrm{2}}}  \Ellesym{,}  \Delta_{{\mathrm{2}}}  \vdash_\mathcal{L}  \Ellent{B}}
          }{\Delta_{{\mathrm{1}}}  \Ellesym{,}  \Gamma_{{\mathrm{1}}}  \Ellesym{,}  \Phi  \Ellesym{,}  \Gamma_{{\mathrm{2}}}  \Ellesym{,}  \Delta_{{\mathrm{2}}}  \vdash_\mathcal{L}  \Ellent{B}}
        \end{math}
      \end{center}

    \item \ElledruleSXXcutTwoName / $\cat{L}$-sequent:
      \begin{center}
        \scriptsize
        $\Pi_1$:
        \begin{math}
          $$\mprset{flushleft}
          \inferrule* [right={\tiny cut}] {
            {
              \begin{array}{cc}
                \pi_1 & \pi_2 \\
                {\Gamma_{{\mathrm{1}}}  \vdash_\mathcal{L}  \Ellent{A}} & {\Gamma_{{\mathrm{2}}}  \Ellesym{,}  \Ellent{A}  \Ellesym{,}  \Gamma_{{\mathrm{3}}}  \vdash_\mathcal{L}  \Ellent{B}}
              \end{array}
            }
          }{\Gamma_{{\mathrm{2}}}  \Ellesym{,}  \Gamma_{{\mathrm{1}}}  \Ellesym{,}  \Gamma_{{\mathrm{3}}}  \vdash_\mathcal{L}  \Ellent{A}}
        \end{math}
        \qquad\qquad
        \begin{math}
          \begin{array}{c}
            \Pi_2 \\
            {\Delta_{{\mathrm{1}}}  \Ellesym{,}  \Ellent{B}  \Ellesym{,}  \Delta_{{\mathrm{2}}}  \vdash_\mathcal{L}  \Ellent{C}}
          \end{array}
        \end{math}
      \end{center}
      By assumption, $c(\Pi_1),c(\Pi_2)\leq |B|$. Similar as above, $|A|+1\leq |B|$ and there
      is a proof $\Pi'$ constructed from $\pi_2$ and $\Pi_2$ for sequent
      $\Delta_{{\mathrm{1}}}  \Ellesym{,}  \Gamma_{{\mathrm{2}}}  \Ellesym{,}  \Ellent{A}  \Ellesym{,}  \Gamma_{{\mathrm{3}}}  \Ellesym{,}  \Delta_{{\mathrm{2}}}  \vdash_\mathcal{L}  \Ellent{C}$ s.t. $c(\Pi')\leq|A|$. Therefore, the proof $\Pi$ can be
      constructed as follows, and $c(\Pi)\leq max\{c(\pi_1),c(\Pi'),|A|+1\}\leq |B|$.
      \begin{center}
        \scriptsize
        \begin{math}
          $$\mprset{flushleft}
          \inferrule* [right={\tiny cut}] {
            {
              \begin{array}{c}
                \pi_1 \\
                {\Gamma_{{\mathrm{1}}}  \vdash_\mathcal{L}  \Ellent{A}}
              \end{array}
            }
            $$\mprset{flushleft}
            \inferrule* [right={\tiny cut}] {
              {
                \begin{array}{cc}
                  \pi_2 & \Pi_2 \\
                  {\Gamma_{{\mathrm{2}}}  \Ellesym{,}  \Ellent{A}  \Ellesym{,}  \Gamma_{{\mathrm{3}}}  \vdash_\mathcal{L}  \Ellent{B}} & {\Delta_{{\mathrm{1}}}  \Ellesym{,}  \Ellent{B}  \Ellesym{,}  \Delta_{{\mathrm{2}}}  \vdash_\mathcal{L}  \Ellent{C}}
                \end{array}
              }
            }{\Delta_{{\mathrm{1}}}  \Ellesym{,}  \Gamma_{{\mathrm{2}}}  \Ellesym{,}  \Ellent{A}  \Ellesym{,}  \Gamma_{{\mathrm{3}}}  \Ellesym{,}  \Delta_{{\mathrm{2}}}  \vdash_\mathcal{L}  \Ellent{C}}
          }{\Delta_{{\mathrm{1}}}  \Ellesym{,}  \Gamma_{{\mathrm{2}}}  \Ellesym{,}  \Gamma_{{\mathrm{1}}}  \Ellesym{,}  \Gamma_{{\mathrm{3}}}  \Ellesym{,}  \Delta_{{\mathrm{2}}}  \vdash_\mathcal{L}  \Ellent{C}}
        \end{math}
      \end{center}

    \item \ElledruleTXXbetaName / $\cat{C}$-sequent:
      \begin{center}
        \scriptsize
        $\Pi_1$:
        \begin{math}
          $$\mprset{flushleft}
          \inferrule* [right={\tiny beta}] {
            {
              \begin{array}{c}
                \pi \\
                {\Phi_{{\mathrm{1}}}  \Ellesym{,}  \Ellent{X_{{\mathrm{1}}}}  \Ellesym{,}  \Ellent{X_{{\mathrm{2}}}}  \Ellesym{,}  \Phi_{{\mathrm{2}}}  \vdash_\mathcal{C}  \Ellent{Y}}
              \end{array}
            }
          }{\Phi_{{\mathrm{1}}}  \Ellesym{,}  \Ellent{X_{{\mathrm{2}}}}  \Ellesym{,}  \Ellent{X_{{\mathrm{1}}}}  \Ellesym{,}  \Phi_{{\mathrm{2}}}  \vdash_\mathcal{C}  \Ellent{Y}}
        \end{math}
        \qquad\qquad
        \begin{math}
          \begin{array}{c}
            \Pi_2 \\
            {\Psi_{{\mathrm{1}}}  \Ellesym{,}  \Ellent{Y}  \Ellesym{,}  \Psi_{{\mathrm{2}}}  \vdash_\mathcal{C}  \Ellent{Z}}
          \end{array}
        \end{math}
      \end{center}
      By assumption, $c(\Pi_1),c(\Pi_2)\leq |Y|$. By induction on the length of $\Pi_1$ and
      $\Pi_2$, the induction hypothesis states that there is a proof $\Pi'$ constructed from
      $\pi$ and $\Pi_2$ for sequent \\
      $\Psi_{{\mathrm{1}}}  \Ellesym{,}  \Phi_{{\mathrm{1}}}  \Ellesym{,}  \Ellent{X_{{\mathrm{1}}}}  \Ellesym{,}  \Ellent{X_{{\mathrm{2}}}}  \Ellesym{,}  \Phi_{{\mathrm{2}}}  \Ellesym{,}  \Psi_{{\mathrm{2}}}  \vdash_\mathcal{C}  \Ellent{Z}$ s.t. $c(\Pi')\leq|Y|$.
      Therefore, the proof $\Pi$ can be constructed as follows, and $c(\Pi)=c(\Pi')\leq|Y|$.
      \begin{center}
        \scriptsize
        \begin{math}
          $$\mprset{flushleft}
          \inferrule* [right={\tiny beta}] {
            $$\mprset{flushleft}
            \inferrule* [right={\tiny cut}] {
              {
                \begin{array}{cc}
                  \pi & \Pi_2 \\
                  {\Phi_{{\mathrm{1}}}  \Ellesym{,}  \Ellent{X_{{\mathrm{1}}}}  \Ellesym{,}  \Ellent{X_{{\mathrm{2}}}}  \Ellesym{,}  \Phi_{{\mathrm{2}}}  \vdash_\mathcal{C}  \Ellent{Y}} & {\Psi_{{\mathrm{1}}}  \Ellesym{,}  \Ellent{Y}  \Ellesym{,}  \Psi_{{\mathrm{2}}}  \vdash_\mathcal{C}  \Ellent{Z}}
                \end{array}
              }
            }{\Psi_{{\mathrm{1}}}  \Ellesym{,}  \Phi_{{\mathrm{1}}}  \Ellesym{,}  \Ellent{X_{{\mathrm{1}}}}  \Ellesym{,}  \Ellent{X_{{\mathrm{2}}}}  \Ellesym{,}  \Phi_{{\mathrm{2}}}  \Ellesym{,}  \Psi_{{\mathrm{2}}}  \vdash_\mathcal{C}  \Ellent{Z}}
          }{\Psi_{{\mathrm{1}}}  \Ellesym{,}  \Phi_{{\mathrm{1}}}  \Ellesym{,}  \Ellent{X_{{\mathrm{2}}}}  \Ellesym{,}  \Ellent{X_{{\mathrm{1}}}}  \Ellesym{,}  \Phi_{{\mathrm{2}}}  \Ellesym{,}  \Psi_{{\mathrm{2}}}  \vdash_\mathcal{C}  \Ellent{Z}}
        \end{math}
      \end{center}

    \item \ElledruleTXXbetaName / $\cat{L}$-sequent:
      \begin{center}
        \scriptsize
        $\Pi_1$:
        \begin{math}
          $$\mprset{flushleft}
          \inferrule* [right={\tiny beta}] {
            {
              \begin{array}{c}
                \pi \\
                {\Phi_{{\mathrm{1}}}  \Ellesym{,}  \Ellent{X}  \Ellesym{,}  \Ellent{Y}  \Ellesym{,}  \Phi_{{\mathrm{2}}}  \vdash_\mathcal{C}  \Ellent{Z}}
              \end{array}
            }
          }{\Phi_{{\mathrm{1}}}  \Ellesym{,}  \Ellent{Y}  \Ellesym{,}  \Ellent{X}  \Ellesym{,}  \Phi_{{\mathrm{2}}}  \vdash_\mathcal{C}  \Ellent{Z}}
        \end{math}
        \qquad\qquad
        \begin{math}
          \begin{array}{c}
            \Pi_2 \\
            {\Gamma_{{\mathrm{1}}}  \Ellesym{,}  \Ellent{Z}  \Ellesym{,}  \Gamma_{{\mathrm{2}}}  \vdash_\mathcal{L}  \Ellent{A}}
          \end{array}
        \end{math}
      \end{center}
      By assumption, $c(\Pi_1),c(\Pi_2)\leq |Z|$. Similar as above, there is a proof $\Pi'$
      constructed from $\pi$ and $\Pi_2$ for sequent $\Gamma_{{\mathrm{1}}}  \Ellesym{,}  \Phi_{{\mathrm{1}}}  \Ellesym{,}  \Ellent{X}  \Ellesym{,}  \Ellent{Y}  \Ellesym{,}  \Phi_{{\mathrm{2}}}  \Ellesym{,}  \Gamma_{{\mathrm{2}}}  \vdash_\mathcal{L}  \Ellent{A}$ s.t.
      $c(\Pi')\leq|Z|$. Therefore, the proof $\Pi$ can be constructed as follows, and
      $c(\Pi)=c(\Pi')\leq|Z|$.
      \begin{center}
        \scriptsize
        \begin{math}
          $$\mprset{flushleft}
          \inferrule* [right={\tiny beta}] {
            $$\mprset{flushleft}
            \inferrule* [right={\tiny cut1}] {
              {
                \begin{array}{cc}
                  \pi & \Pi_2 \\
                  {\Phi_{{\mathrm{1}}}  \Ellesym{,}  \Ellent{X}  \Ellesym{,}  \Ellent{Y}  \Ellesym{,}  \Phi_{{\mathrm{2}}}  \vdash_\mathcal{C}  \Ellent{Z}} & {\Gamma_{{\mathrm{1}}}  \Ellesym{,}  \Ellent{Z}  \Ellesym{,}  \Gamma_{{\mathrm{2}}}  \vdash_\mathcal{L}  \Ellent{A}}
                \end{array}
              }
            }{\Gamma_{{\mathrm{1}}}  \Ellesym{,}  \Phi_{{\mathrm{1}}}  \Ellesym{,}  \Ellent{X}  \Ellesym{,}  \Ellent{Y}  \Ellesym{,}  \Phi_{{\mathrm{2}}}  \Ellesym{,}  \Gamma_{{\mathrm{2}}}  \vdash_\mathcal{L}  \Ellent{A}}
          }{\Gamma_{{\mathrm{1}}}  \Ellesym{,}  \Phi_{{\mathrm{1}}}  \Ellesym{,}  \Ellent{Y}  \Ellesym{,}  \Ellent{X}  \Ellesym{,}  \Phi_{{\mathrm{2}}}  \Ellesym{,}  \Gamma_{{\mathrm{2}}}  \vdash_\mathcal{L}  \Ellent{A}}
        \end{math}
      \end{center}

    \item \ElledruleSXXbetaName / $\cat{L}$-sequent:
      \begin{center}
        \scriptsize
        $\Pi_1$:
        \begin{math}
          $$\mprset{flushleft}
          \inferrule* [right={\tiny beta}] {
            {
              \begin{array}{c}
                \pi \\
                {\Gamma_{{\mathrm{1}}}  \Ellesym{,}  \Ellent{X}  \Ellesym{,}  \Ellent{Y}  \Ellesym{,}  \Gamma_{{\mathrm{2}}}  \vdash_\mathcal{L}  \Ellent{A}}
              \end{array}
            }
          }{\Gamma_{{\mathrm{1}}}  \Ellesym{,}  \Ellent{Y}  \Ellesym{,}  \Ellent{X}  \Ellesym{,}  \Gamma_{{\mathrm{2}}}  \vdash_\mathcal{L}  \Ellent{A}}
        \end{math}
        \qquad\qquad
        \begin{math}
          \begin{array}{c}
            \Pi_2 \\
            {\Delta_{{\mathrm{1}}}  \Ellesym{,}  \Ellent{A}  \Ellesym{,}  \Delta_{{\mathrm{2}}}  \vdash_\mathcal{L}  \Ellent{B}}
          \end{array}
        \end{math}
      \end{center}
      By assumption, $c(\Pi_1),c(\Pi_2)\leq |A|$. Similar as above, there is a proof $\Pi'$
      constructed from $\pi$ and $\Pi_2$ for sequent $\Delta_{{\mathrm{1}}}  \Ellesym{,}  \Gamma_{{\mathrm{1}}}  \Ellesym{,}  \Ellent{X}  \Ellesym{,}  \Ellent{Y}  \Ellesym{,}  \Gamma_{{\mathrm{2}}}  \Ellesym{,}  \Delta_{{\mathrm{2}}}  \vdash_\mathcal{L}  \Ellent{B}$ s.t.
      $c(\Pi')\leq|A|$. Therefore, the proof $\Pi$ can be constructed as follows, and
      $c(\Pi)=c(\Pi')\leq|A|$.
      \begin{center}
        \scriptsize
        \begin{math}
          $$\mprset{flushleft}
          \inferrule* [right={\tiny beta}] {
            $$\mprset{flushleft}
            \inferrule* [right={\tiny cut2}] {
              {
                \begin{array}{cc}
                  \pi & \Pi_2 \\
                  {\Gamma_{{\mathrm{1}}}  \Ellesym{,}  \Ellent{X}  \Ellesym{,}  \Ellent{Y}  \Ellesym{,}  \Gamma_{{\mathrm{2}}}  \vdash_\mathcal{L}  \Ellent{A}} & {\Delta_{{\mathrm{1}}}  \Ellesym{,}  \Ellent{A}  \Ellesym{,}  \Delta_{{\mathrm{2}}}  \vdash_\mathcal{L}  \Ellent{B}}
                \end{array}
              }
            }{\Delta_{{\mathrm{1}}}  \Ellesym{,}  \Gamma_{{\mathrm{1}}}  \Ellesym{,}  \Ellent{X}  \Ellesym{,}  \Ellent{Y}  \Ellesym{,}  \Gamma_{{\mathrm{2}}}  \Ellesym{,}  \Delta_{{\mathrm{2}}}  \vdash_\mathcal{L}  \Ellent{B}}
          }{\Delta_{{\mathrm{1}}}  \Ellesym{,}  \Gamma_{{\mathrm{1}}}  \Ellesym{,}  \Ellent{Y}  \Ellesym{,}  \Ellent{X}  \Ellesym{,}  \Gamma_{{\mathrm{2}}}  \Ellesym{,}  \Delta_{{\mathrm{2}}}  \vdash_\mathcal{L}  \Ellent{B}}
        \end{math}
      \end{center}

    \item \ElledruleTXXaxName / $\cat{C}$-sequent:
      \begin{center}
        \scriptsize
        $\Pi_1$:
        \begin{math}
          $$\mprset{flushleft}
          \inferrule* [right={\tiny ax}] {
            \,
          }{\Ellent{X}  \vdash_\mathcal{C}  \Ellent{X}}
        \end{math}
        \qquad\qquad
        \begin{math}
          \begin{array}{c}
            \Pi_2 \\
            {\Phi_{{\mathrm{1}}}  \Ellesym{,}  \Ellent{X}  \Ellesym{,}  \Phi_{{\mathrm{2}}}  \vdash_\mathcal{C}  \Ellent{Y}}
          \end{array}
        \end{math}
      \end{center}
      By assumption, $c(\Pi_1),c(\Pi_2)\leq |X|$. The proof $\Pi$ is the same as $\Pi_2$.

    \item \ElledruleTXXaxName / $\cat{L}$-sequent:
      \begin{center}
        \scriptsize
        $\Pi_1$:
        \begin{math}
          $$\mprset{flushleft}
          \inferrule* [right={\tiny ax}] {
            \,
          }{\Ellent{X}  \vdash_\mathcal{C}  \Ellent{X}}
        \end{math}
        \qquad\qquad
        \begin{math}
          \begin{array}{c}
            \Pi_2 \\
            {\Gamma_{{\mathrm{1}}}  \Ellesym{,}  \Ellent{X}  \Ellesym{,}  \Gamma_{{\mathrm{2}}}  \vdash_\mathcal{L}  \Ellent{A}}
          \end{array}
        \end{math}
      \end{center}
      By assumption, $c(\Pi_1),c(\Pi_2)\leq |X|$. The proof $\Pi$ is the same as $\Pi_2$.

    \item \ElledruleSXXaxName / $\cat{L}$-sequent:
      \begin{center}
        \scriptsize
        $\Pi_1$:
        \begin{math}
          $$\mprset{flushleft}
          \inferrule* [right={\tiny ax}] {
            \,
          }{\Ellent{A}  \vdash_\mathcal{L}  \Ellent{A}}
        \end{math}
        \qquad\qquad
        \begin{math}
          \begin{array}{c}
            \Pi_2 \\
            {\Gamma_{{\mathrm{1}}}  \Ellesym{,}  \Ellent{A}  \Ellesym{,}  \Gamma_{{\mathrm{2}}}  \vdash_\mathcal{L}  \Ellent{B}}
          \end{array}
        \end{math}
      \end{center}
      By assumption, $c(\Pi_1),c(\Pi_2)\leq |A|$. The proof $\Pi$ is the same as $\Pi_2$.

    \item \ElledruleTXXunitLName / $\cat{C}$-sequent:
      \begin{center}
        \scriptsize
        $\Pi_1$:
        \begin{math}
          $$\mprset{flushleft}
          \inferrule* [right={\tiny unitL}] {
            {
              \begin{array}{c}
                \pi \\
                {\Phi  \vdash_\mathcal{C}  \Ellent{X}}
              \end{array}
            }
          }{ \mathsf{UnitT}   \Ellesym{,}  \Ellent{X}  \vdash_\mathcal{C}  \Ellent{X}}
        \end{math}
        \qquad\qquad
        \begin{math}
          \begin{array}{c}
            \Pi_2 \\
            {\Psi_{{\mathrm{1}}}  \Ellesym{,}  \Ellent{X}  \Ellesym{,}  \Psi_{{\mathrm{2}}}  \vdash_\mathcal{C}  \Ellent{Y}}
          \end{array}
        \end{math}
      \end{center}
      By assumption, $c(\Pi_1),c(\Pi_2)\leq |X|$. By induction, there is a proof $\Pi'$ from
      $\pi$ and $\Pi_2$ for sequent $\Psi_{{\mathrm{1}}}  \Ellesym{,}  \Phi  \Ellesym{,}  \Psi_{{\mathrm{2}}}  \vdash_\mathcal{C}  \Ellent{Y}$ s.t. $c(\Pi')\leq |X|$. Therefore,
      the proof $\Pi$ can be constructed as follows, and $c(\Pi)=c(\Pi')\leq |X|$.
      \begin{center}
        \scriptsize
        \begin{math}
          $$\mprset{flushleft}
          \inferrule* [right={\tiny unitL}] {
            $$\mprset{flushleft}
            \inferrule* [right={\tiny cut}] {
              {
                \begin{array}{cc}
                  \pi & \Pi_2 \\
                  {\Phi  \vdash_\mathcal{C}  \Ellent{X}} & {\Psi_{{\mathrm{1}}}  \Ellesym{,}  \Ellent{X}  \Ellesym{,}  \Psi_{{\mathrm{2}}}  \vdash_\mathcal{C}  \Ellent{Y}}
                \end{array}
              }
            }{\Psi_{{\mathrm{1}}}  \Ellesym{,}  \Phi  \Ellesym{,}  \Psi_{{\mathrm{2}}}  \vdash_\mathcal{C}  \Ellent{Y}}
          }{ \mathsf{UnitT}   \Ellesym{,}  \Psi_{{\mathrm{1}}}  \Ellesym{,}  \Phi  \Ellesym{,}  \Psi_{{\mathrm{2}}}  \vdash_\mathcal{C}  \Ellent{Y}}
        \end{math}
      \end{center}

    \item \ElledruleTXXunitLName / $\cat{L}$-sequent:
      \begin{center}
        \scriptsize
        $\Pi_1$:
        \begin{math}
          $$\mprset{flushleft}
          \inferrule* [right={\tiny unitL}] {
            {
              \begin{array}{c}
                \pi \\
                {\Phi  \vdash_\mathcal{C}  \Ellent{X}}
              \end{array}
            }
          }{ \mathsf{UnitT}   \Ellesym{,}  \Ellent{X}  \vdash_\mathcal{C}  \Ellent{X}}
        \end{math}
        \qquad\qquad
        \begin{math}
          \begin{array}{c}
            \Pi_2 \\
            {\Gamma_{{\mathrm{1}}}  \Ellesym{,}  \Ellent{X}  \Ellesym{,}  \Gamma_{{\mathrm{2}}}  \vdash_\mathcal{L}  \Ellent{A}}
          \end{array}
        \end{math}
      \end{center}
      By assumption, $c(\Pi_1),c(\Pi_2)\leq |X|$. Similar as above, the proof $\Pi$ can be
      constructed as follows with $c(\Pi)\leq |X|$.
      \begin{center}
        \scriptsize
        \begin{math}
          $$\mprset{flushleft}
          \inferrule* [right={\tiny unitL1}] {
            $$\mprset{flushleft}
            \inferrule* [right={\tiny cut1}] {
              {
                \begin{array}{cc}
                  \pi & \Pi_2 \\
                  {\Phi  \vdash_\mathcal{C}  \Ellent{X}} & {\Gamma_{{\mathrm{1}}}  \Ellesym{,}  \Ellent{X}  \Ellesym{,}  \Gamma_{{\mathrm{2}}}  \vdash_\mathcal{L}  \Ellent{A}}
                \end{array}
              }
            }{\Gamma_{{\mathrm{1}}}  \Ellesym{,}  \Phi  \Ellesym{,}  \Gamma_{{\mathrm{2}}}  \vdash_\mathcal{L}  \Ellent{A}}
          }{ \mathsf{UnitT}   \Ellesym{,}  \Gamma_{{\mathrm{1}}}  \Ellesym{,}  \Phi  \Ellesym{,}  \Gamma_{{\mathrm{2}}}  \vdash_\mathcal{L}  \Ellent{A}}
        \end{math}
      \end{center}

    \item \ElledruleSXXunitLOneName / $\cat{L}$-sequent:
      \begin{center}
        \scriptsize
        $\Pi_1$:
        \begin{math}
          $$\mprset{flushleft}
          \inferrule* [right={\tiny unitL1}] {
            {
              \begin{array}{c}
                \pi \\
                {\Delta  \vdash_\mathcal{L}  \Ellent{A}}
              \end{array}
            }
          }{ \mathsf{UnitT}   \Ellesym{,}  \Delta  \vdash_\mathcal{L}  \Ellent{A}}
        \end{math}
        \qquad\qquad
        \begin{math}
          \begin{array}{c}
            \Pi_2 \\
            {\Gamma_{{\mathrm{1}}}  \Ellesym{,}  \Ellent{A}  \Ellesym{,}  \Gamma_{{\mathrm{2}}}  \vdash_\mathcal{L}  \Ellent{B}}
          \end{array}
        \end{math}
      \end{center}
      By assumption, $c(\Pi_1),c(\Pi_2)\leq |A|$. By induction, there is a proof $\Pi'$ from
      $\pi$ and $\Pi_2$ for sequent $\Gamma_{{\mathrm{1}}}  \Ellesym{,}  \Delta  \Ellesym{,}  \Gamma_{{\mathrm{2}}}  \vdash_\mathcal{L}  \Ellent{B}$ s.t. $c(\Pi')\leq |A|$. Therefore,
      the proof $\Pi$ can be constructed as follows, and $c(\Pi)=c(\Pi')\leq |A|$.
      \begin{center}
        \scriptsize
        \begin{math}
          $$\mprset{flushleft}
          \inferrule* [right={\tiny unitL1}] {
            $$\mprset{flushleft}
            \inferrule* [right={\tiny cut2}] {
              {
                \begin{array}{cc}
                  \pi & \Pi_2 \\
                  {\Delta  \vdash_\mathcal{L}  \Ellent{A}} & {\Gamma_{{\mathrm{1}}}  \Ellesym{,}  \Ellent{A}  \Ellesym{,}  \Gamma_{{\mathrm{2}}}  \vdash_\mathcal{L}  \Ellent{B}}
                \end{array}
              }
            }{\Gamma_{{\mathrm{1}}}  \Ellesym{,}  \Delta  \Ellesym{,}  \Gamma_{{\mathrm{2}}}  \vdash_\mathcal{L}  \Ellent{B}}
          }{ \mathsf{UnitT}   \Ellesym{,}  \Gamma_{{\mathrm{1}}}  \Ellesym{,}  \Delta  \Ellesym{,}  \Gamma_{{\mathrm{2}}}  \vdash_\mathcal{L}  \Ellent{B}}
        \end{math}
      \end{center}

    \item \ElledruleSXXunitLTwoName / $\cat{L}$-sequent:
      \begin{center}
        \scriptsize
        $\Pi_1$:
        \begin{math}
          $$\mprset{flushleft}
          \inferrule* [right={\tiny unitL2}] {
            {
              \begin{array}{c}
                \pi \\
                {\Delta  \vdash_\mathcal{L}  \Ellent{A}}
              \end{array}
            }
          }{ \mathsf{UnitS}   \Ellesym{,}  \Delta  \vdash_\mathcal{L}  \Ellent{A}}
        \end{math}
        \qquad\qquad
        \begin{math}
          \begin{array}{c}
            \Pi_2 \\
            {\Gamma_{{\mathrm{1}}}  \Ellesym{,}  \Ellent{A}  \Ellesym{,}  \Gamma_{{\mathrm{2}}}  \vdash_\mathcal{L}  \Ellent{B}}
          \end{array}
        \end{math}
      \end{center}
      By assumption, $c(\Pi_1),c(\Pi_2)\leq |A|$. By induction, there is a proof $\Pi'$ from
      $\pi$ and $\Pi_2$ for sequent $\Gamma_{{\mathrm{1}}}  \Ellesym{,}  \Delta  \Ellesym{,}  \Gamma_{{\mathrm{2}}}  \vdash_\mathcal{L}  \Ellent{B}$ s.t. $c(\Pi')\leq |A|$. Therefore,
      the proof $\Pi$ can be constructed as follows, and $c(\Pi)=c(\Pi')\leq |A|$.
      \begin{center}
        \scriptsize
        \begin{math}
          $$\mprset{flushleft}
          \inferrule* [right={\tiny unitL2}] {
            $$\mprset{flushleft}
            \inferrule* [right={\tiny cut2}] {
              {
                \begin{array}{cc}
                  \pi & \Pi_2 \\
                  {\Delta  \vdash_\mathcal{L}  \Ellent{A}} & {\Gamma_{{\mathrm{1}}}  \Ellesym{,}  \Ellent{A}  \Ellesym{,}  \Gamma_{{\mathrm{2}}}  \vdash_\mathcal{L}  \Ellent{B}}
                \end{array}
              }
            }{\Gamma_{{\mathrm{1}}}  \Ellesym{,}  \Delta  \Ellesym{,}  \Gamma_{{\mathrm{2}}}  \vdash_\mathcal{L}  \Ellent{B}}
          }{ \mathsf{UnitS}   \Ellesym{,}  \Gamma_{{\mathrm{1}}}  \Ellesym{,}  \Delta  \Ellesym{,}  \Gamma_{{\mathrm{2}}}  \vdash_\mathcal{L}  \Ellent{B}}
        \end{math}
      \end{center}

    \item \ElledruleTXXtenLName / $\cat{C}$-sequent:
      \begin{center}
        \scriptsize
        $\Pi_1$:
        \begin{math}
          $$\mprset{flushleft}
          \inferrule* [right={\tiny tenL}] {
            {
              \begin{array}{c}
                \pi \\
                {\Phi_{{\mathrm{1}}}  \Ellesym{,}  \Ellent{X_{{\mathrm{1}}}}  \Ellesym{,}  \Ellent{X_{{\mathrm{2}}}}  \Ellesym{,}  \Phi_{{\mathrm{2}}}  \vdash_\mathcal{C}  \Ellent{Y}}
              \end{array}
            }
          }{\Phi_{{\mathrm{1}}}  \Ellesym{,}  \Ellent{X_{{\mathrm{1}}}}  \otimes  \Ellent{X_{{\mathrm{2}}}}  \Ellesym{,}  \Phi_{{\mathrm{2}}}  \vdash_\mathcal{C}  \Ellent{Y}}
        \end{math}
        \qquad\qquad
        \begin{math}
          \begin{array}{c}
            \Pi_2 \\
            {\Psi_{{\mathrm{1}}}  \Ellesym{,}  \Ellent{Y}  \Ellesym{,}  \Psi_{{\mathrm{2}}}  \vdash_\mathcal{C}  \Ellent{Z}}
          \end{array}
        \end{math}
      \end{center}
      By assumption, $c(\Pi_1),c(\Pi_2)\leq |Y|$. By induction, there is a proof $\Pi'$ from
      $\pi$ and $\Pi_2$ for $\Psi_{{\mathrm{1}}}  \Ellesym{,}  \Phi_{{\mathrm{1}}}  \Ellesym{,}  \Ellent{X_{{\mathrm{1}}}}  \Ellesym{,}  \Ellent{X_{{\mathrm{2}}}}  \Ellesym{,}  \Phi_{{\mathrm{2}}}  \Ellesym{,}  \Psi_{{\mathrm{2}}}  \vdash_\mathcal{C}  \Ellent{Z}$ s.t. $c(\Pi')\leq |Y|$.
      Therefore, the proof $\Pi$ can be constructed as follows with $c(\Pi)\leq |Y|$.
      \begin{center}
        \scriptsize
        \begin{math}
          $$\mprset{flushleft}
          \inferrule* [right={\tiny tenL}] {
            $$\mprset{flushleft}
            \inferrule* [right={\tiny cut}] {
              {
                \begin{array}{cc}
                  \pi & \Pi_2 \\
                  {\Phi_{{\mathrm{1}}}  \Ellesym{,}  \Ellent{X_{{\mathrm{1}}}}  \Ellesym{,}  \Ellent{X_{{\mathrm{2}}}}  \Ellesym{,}  \Phi_{{\mathrm{2}}}  \vdash_\mathcal{C}  \Ellent{Y}} & {\Psi_{{\mathrm{1}}}  \Ellesym{,}  \Ellent{Y}  \Ellesym{,}  \Psi_{{\mathrm{2}}}  \vdash_\mathcal{C}  \Ellent{Z}}
                \end{array}
              }
            }{\Psi_{{\mathrm{1}}}  \Ellesym{,}  \Phi_{{\mathrm{1}}}  \Ellesym{,}  \Ellent{X_{{\mathrm{1}}}}  \Ellesym{,}  \Ellent{X_{{\mathrm{2}}}}  \Ellesym{,}  \Phi_{{\mathrm{2}}}  \Ellesym{,}  \Psi_{{\mathrm{2}}}  \vdash_\mathcal{C}  \Ellent{Z}}
          }{\Psi_{{\mathrm{1}}}  \Ellesym{,}  \Phi_{{\mathrm{1}}}  \Ellesym{,}  \Ellent{X_{{\mathrm{1}}}}  \otimes  \Ellent{X_{{\mathrm{2}}}}  \Ellesym{,}  \Phi_{{\mathrm{2}}}  \Ellesym{,}  \Psi_{{\mathrm{2}}}  \vdash_\mathcal{C}  \Ellent{Z}}
        \end{math}
      \end{center}

    \item \ElledruleTXXtenLName / $\cat{L}$-sequent:
      \begin{center}
        \scriptsize
        $\Pi_1$:
        \begin{math}
          $$\mprset{flushleft}
          \inferrule* [right={\tiny tenL}] {
            {
              \begin{array}{c}
                \pi \\
                {\Phi_{{\mathrm{1}}}  \Ellesym{,}  \Ellent{X_{{\mathrm{1}}}}  \Ellesym{,}  \Ellent{X_{{\mathrm{2}}}}  \Ellesym{,}  \Phi_{{\mathrm{2}}}  \vdash_\mathcal{C}  \Ellent{Y}}
              \end{array}
            }
          }{\Phi_{{\mathrm{1}}}  \Ellesym{,}  \Ellent{X_{{\mathrm{1}}}}  \otimes  \Ellent{X_{{\mathrm{2}}}}  \Ellesym{,}  \Phi_{{\mathrm{2}}}  \vdash_\mathcal{C}  \Ellent{Y}}
        \end{math}
        \qquad\qquad
        \begin{math}
          \begin{array}{c}
            \Pi_2 \\
            {\Gamma_{{\mathrm{1}}}  \Ellesym{,}  \Ellent{Y}  \Ellesym{,}  \Gamma_{{\mathrm{2}}}  \vdash_\mathcal{L}  \Ellent{A}}
          \end{array}
        \end{math}
      \end{center}
      By assumption, $c(\Pi_1),c(\Pi_2)\leq |Y|$. By induction, there is a proof $\Pi'$ from
      $\pi$ and $\Pi_2$ for $\Gamma_{{\mathrm{1}}}  \Ellesym{,}  \Phi_{{\mathrm{1}}}  \Ellesym{,}  \Ellent{X_{{\mathrm{1}}}}  \Ellesym{,}  \Ellent{X_{{\mathrm{2}}}}  \Ellesym{,}  \Phi_{{\mathrm{2}}}  \Ellesym{,}  \Gamma_{{\mathrm{2}}}  \vdash_\mathcal{L}  \Ellent{A}$ s.t. $c(\Pi')\leq |Y|$.
      Therefore, the proof $\Pi$ can be constructed as follows with $c(\Pi)\leq |Y|$.
      \begin{center}
        \scriptsize
        \begin{math}
          $$\mprset{flushleft}
          \inferrule* [right={\tiny tenL1}] {
            $$\mprset{flushleft}
            \inferrule* [right={\tiny cut1}] {
              {
                \begin{array}{cc}
                  \pi & \Pi_2 \\
                  {\Phi_{{\mathrm{1}}}  \Ellesym{,}  \Ellent{X_{{\mathrm{1}}}}  \Ellesym{,}  \Ellent{X_{{\mathrm{2}}}}  \Ellesym{,}  \Phi_{{\mathrm{2}}}  \vdash_\mathcal{C}  \Ellent{Y}} & {\Gamma_{{\mathrm{1}}}  \Ellesym{,}  \Ellent{Y}  \Ellesym{,}  \Gamma_{{\mathrm{2}}}  \vdash_\mathcal{L}  \Ellent{A}}
                \end{array}
              }
            }{\Gamma_{{\mathrm{1}}}  \Ellesym{,}  \Phi_{{\mathrm{1}}}  \Ellesym{,}  \Ellent{X_{{\mathrm{1}}}}  \Ellesym{,}  \Ellent{X_{{\mathrm{2}}}}  \Ellesym{,}  \Phi_{{\mathrm{2}}}  \Ellesym{,}  \Gamma_{{\mathrm{2}}}  \vdash_\mathcal{L}  \Ellent{A}}
          }{\Gamma_{{\mathrm{1}}}  \Ellesym{,}  \Phi_{{\mathrm{1}}}  \Ellesym{,}  \Ellent{X_{{\mathrm{1}}}}  \otimes  \Ellent{X_{{\mathrm{2}}}}  \Ellesym{,}  \Phi_{{\mathrm{2}}}  \Ellesym{,}  \Gamma_{{\mathrm{2}}}  \vdash_\mathcal{L}  \Ellent{A}}
        \end{math}
      \end{center}

    \item \ElledruleSXXtenLOneName / $\cat{L}$-sequent:
      \begin{center}
        \scriptsize
        $\Pi_1$:
        \begin{math}
          $$\mprset{flushleft}
          \inferrule* [right={\tiny tenL}] {
            {
              \begin{array}{c}
                \pi \\
                {\Gamma_{{\mathrm{1}}}  \Ellesym{,}  \Ellent{X}  \Ellesym{,}  \Ellent{Y}  \Ellesym{,}  \Gamma_{{\mathrm{2}}}  \vdash_\mathcal{L}  \Ellent{A}}
              \end{array}
            }
          }{\Gamma_{{\mathrm{1}}}  \Ellesym{,}  \Ellent{X}  \otimes  \Ellent{Y}  \Ellesym{,}  \Gamma_{{\mathrm{2}}}  \vdash_\mathcal{L}  \Ellent{A}}
        \end{math}
        \qquad\qquad
        \begin{math}
          \begin{array}{c}
            \Pi_2 \\
            {\Delta_{{\mathrm{1}}}  \Ellesym{,}  \Ellent{A}  \Ellesym{,}  \Delta_{{\mathrm{2}}}  \vdash_\mathcal{L}  \Ellent{B}}
          \end{array}
        \end{math}
      \end{center}
      By assumption, $c(\Pi_1),c(\Pi_2)\leq |A|$. By induction, there is a proof $\Pi'$ from
      $\pi$ and $\Pi_2$ for $\Delta_{{\mathrm{1}}}  \Ellesym{,}  \Ellent{X}  \Ellesym{,}  \Ellent{Y}  \Ellesym{,}  \Gamma_{{\mathrm{2}}}  \Ellesym{,}  \Delta_{{\mathrm{2}}}  \vdash_\mathcal{L}  \Ellent{B}$ s.t. $c(\Pi')\leq |A|$.
      Therefore, the proof $\Pi$ can be constructed as follows with $c(\Pi)\leq |A|$.
      \begin{center}
        \scriptsize
        \begin{math}
          $$\mprset{flushleft}
          \inferrule* [right={\tiny tenL1}] {
            $$\mprset{flushleft}
            \inferrule* [right={\tiny cut2}] {
              {
                \begin{array}{cc}
                  \pi & \Pi_2 \\
                  {\Gamma_{{\mathrm{1}}}  \Ellesym{,}  \Ellent{X}  \Ellesym{,}  \Ellent{Y}  \Ellesym{,}  \Gamma_{{\mathrm{2}}}  \vdash_\mathcal{L}  \Ellent{A}} & {\Delta_{{\mathrm{1}}}  \Ellesym{,}  \Ellent{A}  \Ellesym{,}  \Delta_{{\mathrm{2}}}  \vdash_\mathcal{L}  \Ellent{B}}
                \end{array}
              }
            }{\Delta_{{\mathrm{1}}}  \Ellesym{,}  \Gamma_{{\mathrm{1}}}  \Ellesym{,}  \Ellent{X}  \Ellesym{,}  \Ellent{Y}  \Ellesym{,}  \Gamma_{{\mathrm{2}}}  \Ellesym{,}  \Delta_{{\mathrm{2}}}  \vdash_\mathcal{L}  \Ellent{B}}
          }{\Delta_{{\mathrm{1}}}  \Ellesym{,}  \Gamma_{{\mathrm{1}}}  \Ellesym{,}  \Ellent{X}  \otimes  \Ellent{Y}  \Ellesym{,}  \Gamma_{{\mathrm{2}}}  \Ellesym{,}  \Delta_{{\mathrm{2}}}  \vdash_\mathcal{L}  \Ellent{B}}
        \end{math}
      \end{center}

    \item \ElledruleSXXtenLTwoName / $\cat{L}$-sequent:
      \begin{center}
        \scriptsize
        $\Pi_1$:
        \begin{math}
          $$\mprset{flushleft}
          \inferrule* [right={\tiny tenL2}] {
            {
              \begin{array}{c}
                \pi \\
                {\Gamma_{{\mathrm{1}}}  \Ellesym{,}  \Ellent{A_{{\mathrm{1}}}}  \Ellesym{,}  \Ellent{A_{{\mathrm{2}}}}  \Ellesym{,}  \Gamma_{{\mathrm{2}}}  \vdash_\mathcal{L}  \Ellent{B}}
              \end{array}
            }
          }{\Gamma_{{\mathrm{1}}}  \Ellesym{,}  \Ellent{A_{{\mathrm{1}}}}  \triangleright  \Ellent{A_{{\mathrm{2}}}}  \Ellesym{,}  \Gamma_{{\mathrm{2}}}  \vdash_\mathcal{L}  \Ellent{B}}
        \end{math}
        \qquad\qquad
        \begin{math}
          \begin{array}{c}
            \Pi_2 \\
            {\Delta_{{\mathrm{1}}}  \Ellesym{,}  \Ellent{B}  \Ellesym{,}  \Delta_{{\mathrm{2}}}  \vdash_\mathcal{L}  \Ellent{C}}
          \end{array}
        \end{math}
      \end{center}
      By assumption, $c(\Pi_1),c(\Pi_2)\leq |B|$. By induction, there is a proof $\Pi'$ from
      $\pi$ and $\Pi_2$ for $\Delta_{{\mathrm{1}}}  \Ellesym{,}  \Gamma_{{\mathrm{1}}}  \Ellesym{,}  \Ellent{A_{{\mathrm{1}}}}  \Ellesym{,}  \Ellent{A_{{\mathrm{2}}}}  \Ellesym{,}  \Gamma_{{\mathrm{2}}}  \Ellesym{,}  \Delta_{{\mathrm{2}}}  \vdash_\mathcal{L}  \Ellent{C}$ s.t. $c(\Pi')\leq |B|$.
      Therefore, the proof $\Pi$ can be constructed as follows with $c(\Pi)\leq |B|$.
      \begin{center}
        \scriptsize
        \begin{math}
          $$\mprset{flushleft}
          \inferrule* [right={\tiny tenL2}] {
            $$\mprset{flushleft}
            \inferrule* [right={\tiny cut2}] {
              {
                \begin{array}{cc}
                  \pi & \Pi_2 \\
                  {\Gamma_{{\mathrm{1}}}  \Ellesym{,}  \Ellent{A_{{\mathrm{1}}}}  \Ellesym{,}  \Ellent{A_{{\mathrm{2}}}}  \Ellesym{,}  \Gamma_{{\mathrm{2}}}  \vdash_\mathcal{L}  \Ellent{B}} & {\Delta_{{\mathrm{1}}}  \Ellesym{,}  \Ellent{B}  \Ellesym{,}  \Delta_{{\mathrm{2}}}  \vdash_\mathcal{L}  \Ellent{C}}
                \end{array}
              }
            }{\Delta_{{\mathrm{1}}}  \Ellesym{,}  \Gamma_{{\mathrm{1}}}  \Ellesym{,}  \Ellent{A_{{\mathrm{1}}}}  \Ellesym{,}  \Ellent{A_{{\mathrm{2}}}}  \Ellesym{,}  \Gamma_{{\mathrm{2}}}  \Ellesym{,}  \Delta_{{\mathrm{2}}}  \vdash_\mathcal{L}  \Ellent{C}}
          }{\Delta_{{\mathrm{1}}}  \Ellesym{,}  \Gamma_{{\mathrm{1}}}  \Ellesym{,}  \Ellent{A_{{\mathrm{1}}}}  \triangleright  \Ellent{A_{{\mathrm{2}}}}  \Ellesym{,}  \Gamma_{{\mathrm{2}}}  \Ellesym{,}  \Delta_{{\mathrm{2}}}  \vdash_\mathcal{L}  \Ellent{C}}
        \end{math}
      \end{center}

    \item \ElledruleTXXimpLName / $\cat{C}$-sequent:
      \begin{center}
        \scriptsize
        $\Pi_1$:
        \begin{math}
          $$\mprset{flushleft}
          \inferrule* [right={\tiny impL}] {
            {
              \begin{array}{cc}
                \pi_1 & \pi_2 \\
                {\Phi_{{\mathrm{1}}}  \vdash_\mathcal{C}  \Ellent{X_{{\mathrm{1}}}}} & {\Phi_{{\mathrm{2}}}  \Ellesym{,}  \Ellent{X_{{\mathrm{2}}}}  \Ellesym{,}  \Phi_{{\mathrm{3}}}  \vdash_\mathcal{C}  \Ellent{Y}}
              \end{array}
            }
          }{\Phi_{{\mathrm{2}}}  \Ellesym{,}  \Ellent{X_{{\mathrm{1}}}}  \multimap  \Ellent{X_{{\mathrm{2}}}}  \Ellesym{,}  \Phi_{{\mathrm{1}}}  \Ellesym{,}  \Phi_{{\mathrm{3}}}  \vdash_\mathcal{C}  \Ellent{Y}}
        \end{math}
        \qquad\qquad
        \begin{math}
          \begin{array}{c}
            \Pi_2 \\
            {\Psi_{{\mathrm{1}}}  \Ellesym{,}  \Ellent{Y}  \Ellesym{,}  \Psi_{{\mathrm{2}}}  \vdash_\mathcal{C}  \Ellent{Z}}
          \end{array}
        \end{math}
      \end{center}
      By assumption, $c(\Pi_1),c(\Pi_2)\leq |Y|$. By induction, there is a proof $\Pi'$ from
      $\pi_2$ and $\Pi_2$ for $\Psi_{{\mathrm{1}}}  \Ellesym{,}  \Phi_{{\mathrm{2}}}  \Ellesym{,}  \Ellent{X_{{\mathrm{2}}}}  \Ellesym{,}  \Phi_{{\mathrm{3}}}  \Ellesym{,}  \Psi_{{\mathrm{2}}}  \vdash_\mathcal{C}  \Ellent{Z}$ s.t. $c(\Pi')\leq |Y|$.
      Therefore, the proof $\Pi$ can be constructed as follows with $c(\Pi)\leq |Y|$.
      \begin{center}
        \scriptsize
        \begin{math}
          $$\mprset{flushleft}
          \inferrule* [right={\tiny impL}] {
            {
              \begin{array}{c}
                \pi_1 \\
                {\Phi_{{\mathrm{1}}}  \vdash_\mathcal{C}  \Ellent{X_{{\mathrm{1}}}}}
              \end{array}
            }
            $$\mprset{flushleft}
            \inferrule* [right={\tiny cut}] {
              {
                \begin{array}{cc}
                  \pi_2 & \Pi_2 \\
                  {\Phi_{{\mathrm{2}}}  \Ellesym{,}  \Ellent{X_{{\mathrm{2}}}}  \Ellesym{,}  \Phi_{{\mathrm{3}}}  \vdash_\mathcal{C}  \Ellent{Y}} & {\Psi_{{\mathrm{1}}}  \Ellesym{,}  \Ellent{Y}  \Ellesym{,}  \Psi_{{\mathrm{2}}}  \vdash_\mathcal{C}  \Ellent{Z}}
                \end{array}
              }
            }{\Psi_{{\mathrm{1}}}  \Ellesym{,}  \Phi_{{\mathrm{2}}}  \Ellesym{,}  \Ellent{X_{{\mathrm{2}}}}  \Ellesym{,}  \Phi_{{\mathrm{3}}}  \Ellesym{,}  \Psi_{{\mathrm{2}}}  \vdash_\mathcal{C}  \Ellent{Z}}
          }{\Psi_{{\mathrm{1}}}  \Ellesym{,}  \Phi_{{\mathrm{2}}}  \Ellesym{,}  \Ellent{X_{{\mathrm{1}}}}  \multimap  \Ellent{X_{{\mathrm{2}}}}  \Ellesym{,}  \Phi_{{\mathrm{1}}}  \Ellesym{,}  \Phi_{{\mathrm{3}}}  \Ellesym{,}  \Psi_{{\mathrm{2}}}  \vdash_\mathcal{C}  \Ellent{Z}}
        \end{math}
      \end{center}

    \item \ElledruleTXXimpLName / $\cat{L}$-sequent:
      \begin{center}
        \scriptsize
        $\Pi_1$:
        \begin{math}
          $$\mprset{flushleft}
          \inferrule* [right={\tiny impL}] {
            {
              \begin{array}{cc}
                \pi_1 & \pi_2 \\
                {\Phi_{{\mathrm{1}}}  \vdash_\mathcal{C}  \Ellent{X_{{\mathrm{1}}}}} & {\Phi_{{\mathrm{2}}}  \Ellesym{,}  \Ellent{X_{{\mathrm{2}}}}  \Ellesym{,}  \Phi_{{\mathrm{3}}}  \vdash_\mathcal{C}  \Ellent{Y}}
              \end{array}
            }
          }{\Phi_{{\mathrm{2}}}  \Ellesym{,}  \Ellent{X_{{\mathrm{1}}}}  \multimap  \Ellent{X_{{\mathrm{2}}}}  \Ellesym{,}  \Phi_{{\mathrm{1}}}  \Ellesym{,}  \Phi_{{\mathrm{3}}}  \vdash_\mathcal{C}  \Ellent{Y}}
        \end{math}
        \qquad\qquad
        \begin{math}
          \begin{array}{c}
            \Pi_2 \\
            {\Gamma_{{\mathrm{1}}}  \Ellesym{,}  \Ellent{Y}  \Ellesym{,}  \Gamma_{{\mathrm{2}}}  \vdash_\mathcal{L}  \Ellent{A}}
          \end{array}
        \end{math}
      \end{center}
      By assumption, $c(\Pi_1),c(\Pi_2)\leq |Y|$. By induction, there is a proof $\Pi'$ from
      $\pi_2$ and $\Pi_2$ for $\Gamma_{{\mathrm{1}}}  \Ellesym{,}  \Phi_{{\mathrm{2}}}  \Ellesym{,}  \Ellent{X_{{\mathrm{2}}}}  \Ellesym{,}  \Phi_{{\mathrm{3}}}  \Ellesym{,}  \Gamma_{{\mathrm{2}}}  \vdash_\mathcal{L}  \Ellent{A}$ s.t. $c(\Pi')\leq |Y|$.
      Therefore, the proof $\Pi$ can be constructed as follows with $c(\Pi)\leq |Y|$.
      \begin{center}
        \scriptsize
        \begin{math}
          $$\mprset{flushleft}
          \inferrule* [right={\tiny impL}] {
            {
              \begin{array}{c}
                \pi_1 \\
                {\Phi_{{\mathrm{1}}}  \vdash_\mathcal{C}  \Ellent{X_{{\mathrm{1}}}}}
              \end{array}
            }
            $$\mprset{flushleft}
            \inferrule* [right={\tiny cut}] {
              {
                \begin{array}{cc}
                  \pi_2 & \Pi_2 \\
                  {\Phi_{{\mathrm{2}}}  \Ellesym{,}  \Ellent{X_{{\mathrm{2}}}}  \Ellesym{,}  \Phi_{{\mathrm{3}}}  \vdash_\mathcal{C}  \Ellent{Y}} & {\Gamma_{{\mathrm{1}}}  \Ellesym{,}  \Ellent{Y}  \Ellesym{,}  \Gamma_{{\mathrm{2}}}  \vdash_\mathcal{L}  \Ellent{A}}
                \end{array}
              }
            }{\Gamma_{{\mathrm{1}}}  \Ellesym{,}  \Phi_{{\mathrm{2}}}  \Ellesym{,}  \Ellent{X_{{\mathrm{2}}}}  \Ellesym{,}  \Phi_{{\mathrm{3}}}  \Ellesym{,}  \Gamma_{{\mathrm{2}}}  \vdash_\mathcal{L}  \Ellent{A}}
          }{\Gamma_{{\mathrm{1}}}  \Ellesym{,}  \Phi_{{\mathrm{2}}}  \Ellesym{,}  \Ellent{X_{{\mathrm{1}}}}  \multimap  \Ellent{X_{{\mathrm{2}}}}  \Ellesym{,}  \Phi_{{\mathrm{1}}}  \Ellesym{,}  \Phi_{{\mathrm{3}}}  \Ellesym{,}  \Gamma_{{\mathrm{2}}}  \vdash_\mathcal{L}  \Ellent{A}}
        \end{math}
      \end{center}

    \item \ElledruleSXXimprLName / $\cat{L}$-sequent:
      \begin{center}
        \scriptsize
        $\Pi_1$:
        \begin{math}
          $$\mprset{flushleft}
          \inferrule* [right={\tiny impL}] {
            {
              \begin{array}{cc}
                \pi_1 & \pi_2 \\
                {\Gamma_{{\mathrm{1}}}  \vdash_\mathcal{L}  \Ellent{A_{{\mathrm{1}}}}} & {\Gamma_{{\mathrm{2}}}  \Ellesym{,}  \Ellent{A_{{\mathrm{2}}}}  \vdash_\mathcal{L}  \Ellent{B}}
              \end{array}
            }
          }{\Gamma_{{\mathrm{2}}}  \Ellesym{,}  \Ellent{A_{{\mathrm{1}}}}  \rightharpoonup  \Ellent{B_{{\mathrm{2}}}}  \Ellesym{,}  \Gamma_{{\mathrm{1}}}  \vdash_\mathcal{L}  \Ellent{B}}
        \end{math}
        \qquad\qquad
        \begin{math}
          \begin{array}{c}
            \Pi_2 \\
            {\Delta_{{\mathrm{1}}}  \Ellesym{,}  \Ellent{B}  \Ellesym{,}  \Delta_{{\mathrm{2}}}  \vdash_\mathcal{L}  \Ellent{C}}
          \end{array}
        \end{math}
      \end{center}
      By assumption, $c(\Pi_1),c(\Pi_2)\leq |B|$. By induction, there is a proof $\Pi'$ from
      $\pi_2$ and $\Pi_2$ for $\Delta_{{\mathrm{1}}}  \Ellesym{,}  \Gamma_{{\mathrm{2}}}  \Ellesym{,}  \Ellent{A_{{\mathrm{2}}}}  \Ellesym{,}  \Delta_{{\mathrm{2}}}  \vdash_\mathcal{L}  \Ellent{C}$ s.t. $c(\Pi')\leq |B|$.
      Therefore, the proof $\Pi$ can be constructed as follows with $c(\Pi)\leq |B|$.
      \begin{center}
        \scriptsize
        \begin{math}
          $$\mprset{flushleft}
          \inferrule* [right={\tiny impL}] {
            {
              \begin{array}{c}
                \pi_1 \\
                {\Gamma_{{\mathrm{1}}}  \vdash_\mathcal{L}  \Ellent{A_{{\mathrm{1}}}}}
              \end{array}
            }
            $$\mprset{flushleft}
            \inferrule* [right={\tiny cut}] {
              {
                \begin{array}{cc}
                  \pi_2 & \Pi_2 \\
                  {\Gamma_{{\mathrm{2}}}  \Ellesym{,}  \Ellent{A_{{\mathrm{2}}}}  \vdash_\mathcal{L}  \Ellent{B}} & {\Delta_{{\mathrm{1}}}  \Ellesym{,}  \Ellent{B}  \Ellesym{,}  \Delta_{{\mathrm{2}}}  \vdash_\mathcal{L}  \Ellent{C}}
                \end{array}
              }
            }{\Delta_{{\mathrm{1}}}  \Ellesym{,}  \Gamma_{{\mathrm{2}}}  \Ellesym{,}  \Ellent{A_{{\mathrm{2}}}}  \Ellesym{,}  \Delta_{{\mathrm{2}}}  \vdash_\mathcal{L}  \Ellent{C}}
          }{?}
        \end{math}
      \end{center}

    \item \ElledruleSXXFlName / $\cat{L}$-sequent:
      \begin{center}
        \scriptsize
        $\Pi_1$:
        \begin{math}
          $$\mprset{flushleft}
          \inferrule* [right={\tiny FL}] {
            {
              \begin{array}{c}
                \pi_1 \\
                {\Gamma_{{\mathrm{1}}}  \Ellesym{,}  \Ellent{X}  \Ellesym{,}  \Gamma_{{\mathrm{2}}}  \vdash_\mathcal{L}  \Ellent{A}}
              \end{array}
            }
          }{\Gamma_{{\mathrm{1}}}  \Ellesym{,}   \mathsf{F} \Ellent{X}   \Ellesym{,}  \Gamma_{{\mathrm{2}}}  \vdash_\mathcal{L}  \Ellent{A}}
        \end{math}
        \qquad\qquad
        \begin{math}
          \begin{array}{c}
            \Pi_2 \\
            {\Delta_{{\mathrm{1}}}  \Ellesym{,}  \Ellent{A}  \Ellesym{,}  \Delta_{{\mathrm{2}}}  \vdash_\mathcal{L}  \Ellent{B}}
          \end{array}
        \end{math}
      \end{center}
      By assumption, $c(\Pi_1),c(\Pi_2)\leq |A|$. By induction, there is a proof $\Pi'$ from
      $\pi_2$ and $\Pi_2$ for $\Delta_{{\mathrm{1}}}  \Ellesym{,}  \Gamma_{{\mathrm{1}}}  \Ellesym{,}  \Ellent{X}  \Ellesym{,}  \Gamma_{{\mathrm{2}}}  \Ellesym{,}  \Delta_{{\mathrm{2}}}  \vdash_\mathcal{L}  \Ellent{B}$ s.t. $c(\Pi')\leq |A|$.
      Therefore, the proof $\Pi$ can be constructed as follows with $c(\Pi)\leq |A|$.
      \begin{center}
        \scriptsize
        \begin{math}
          $$\mprset{flushleft}
          \inferrule* [right={\tiny FL}] {
            $$\mprset{flushleft}
            \inferrule* [right={\tiny cut2}] {
              {
                \begin{array}{cc}
                  \pi_2 & \Pi_2 \\
                  {\Gamma_{{\mathrm{1}}}  \Ellesym{,}  \Ellent{X}  \Ellesym{,}  \Gamma_{{\mathrm{2}}}  \vdash_\mathcal{L}  \Ellent{A}} & {\Delta_{{\mathrm{1}}}  \Ellesym{,}  \Ellent{A}  \Ellesym{,}  \Delta_{{\mathrm{2}}}  \vdash_\mathcal{L}  \Ellent{B}}
                \end{array}
              }
            }{\Delta_{{\mathrm{1}}}  \Ellesym{,}  \Gamma_{{\mathrm{1}}}  \Ellesym{,}  \Ellent{X}  \Ellesym{,}  \Gamma_{{\mathrm{2}}}  \Ellesym{,}  \Delta_{{\mathrm{2}}}  \vdash_\mathcal{L}  \Ellent{B}}
          }{\Delta_{{\mathrm{1}}}  \Ellesym{,}  \Gamma_{{\mathrm{1}}}  \Ellesym{,}   \mathsf{F} \Ellent{X}   \Ellesym{,}  \Gamma_{{\mathrm{2}}}  \Ellesym{,}  \Delta_{{\mathrm{2}}}  \vdash_\mathcal{L}  \Ellent{B}}
        \end{math}
      \end{center}

    \item \ElledruleSXXGlName / $\cat{L}$-sequent:
      \begin{center}
        \scriptsize
        $\Pi_1$:
        \begin{math}
          $$\mprset{flushleft}
          \inferrule* [right={\tiny GL}] {
            {
              \begin{array}{c}
                \pi_1 \\
                {\Gamma_{{\mathrm{1}}}  \Ellesym{,}  \Ellent{A}  \Ellesym{,}  \Gamma_{{\mathrm{2}}}  \vdash_\mathcal{L}  \Ellent{B}}
              \end{array}
            }
          }{\Gamma_{{\mathrm{1}}}  \Ellesym{,}   \mathsf{G} \Ellent{A}   \Ellesym{,}  \Gamma_{{\mathrm{2}}}  \vdash_\mathcal{L}  \Ellent{B}}
        \end{math}
        \qquad\qquad
        \begin{math}
          \begin{array}{c}
            \Pi_2 \\
            {\Delta_{{\mathrm{1}}}  \Ellesym{,}  \Ellent{B}  \Ellesym{,}  \Delta_{{\mathrm{2}}}  \vdash_\mathcal{L}  \Ellent{C}}
          \end{array}
        \end{math}
      \end{center}
      By assumption, $c(\Pi_1),c(\Pi_2)\leq |B|$. By induction, there is a proof $\Pi'$ from
      $\pi_2$ and $\Pi_2$ for $\Delta_{{\mathrm{1}}}  \Ellesym{,}  \Gamma_{{\mathrm{1}}}  \Ellesym{,}  \Ellent{A}  \Ellesym{,}  \Gamma_{{\mathrm{2}}}  \Ellesym{,}  \Delta_{{\mathrm{2}}}  \vdash_\mathcal{L}  \Ellent{C}$ s.t. $c(\Pi')\leq |B|$.
      Therefore, the proof $\Pi$ can be constructed as follows with $c(\Pi)\leq |B|$.
      \begin{center}
        \scriptsize
        \begin{math}
          $$\mprset{flushleft}
          \inferrule* [right={\tiny GL}] {
            $$\mprset{flushleft}
            \inferrule* [right={\tiny cut2}] {
              {
                \begin{array}{cc}
                  \pi_2 & \Pi_2 \\
                  {\Gamma_{{\mathrm{1}}}  \Ellesym{,}  \Ellent{A}  \Ellesym{,}  \Gamma_{{\mathrm{2}}}  \vdash_\mathcal{L}  \Ellent{B}} & {\Delta_{{\mathrm{1}}}  \Ellesym{,}  \Ellent{B}  \Ellesym{,}  \Delta_{{\mathrm{2}}}  \vdash_\mathcal{L}  \Ellent{C}}
                \end{array}
              }
            }{\Delta_{{\mathrm{1}}}  \Ellesym{,}  \Gamma_{{\mathrm{1}}}  \Ellesym{,}  \Ellent{A}  \Ellesym{,}  \Gamma_{{\mathrm{2}}}  \Ellesym{,}  \Delta_{{\mathrm{2}}}  \vdash_\mathcal{L}  \Ellent{C}}
          }{\Delta_{{\mathrm{1}}}  \Ellesym{,}  \Gamma_{{\mathrm{1}}}  \Ellesym{,}   \mathsf{G} \Ellent{A}   \Ellesym{,}  \Gamma_{{\mathrm{2}}}  \Ellesym{,}  \Delta_{{\mathrm{2}}}  \vdash_\mathcal{L}  \Ellent{C}}
        \end{math}
      \end{center}

    \end{itemize}

  \item The cut formula is a minor formula of the last rule in $\Pi_2$. 

  \item $\Pi_2$ is an axiom on the cut formula. The case is trivial. The proof $\Pi$ is the
        same as $\Pi_1$.

  \end{enumerate}
\end{proof}

\begin{lemma}
  \label{lem:less-cut-rank}
  Let $\Pi$ be a proof of a sequent $\Phi  \vdash_\mathcal{C}  \Ellent{X}$ or $\Gamma  \vdash_\mathcal{L}  \Ellent{A}$ s.t. $c(\Pi)>0$. Then there
  is a proof $\Pi'$ of the same sequent with $c(\Pi')<c(\Pi)$.
\end{lemma}
\begin{proof}
  We prove the lemma by induction on $d(\Pi)$. We denote the proof $\Pi$ by $\pi+r$, where $r$
  is the last inference of $\Pi$ and $\pi$ denotes the rest of the proof. If $r$ is not a cut,
  then by induction hypothesis on $\pi$, there is a proof $\pi'$ s.t. $c(\pi')>c(\pi)$ and
  $\Pi'=\pi'+r$. Otherwise, we assume $r$ is a cut on a formula $X$. If $c(\Pi)>|X|+1$, then
  there is a cut on $|Y|$ in $\pi$ with $|Y|>|X|$. So we can apply the induction hypothesis
  on $\pi$ to get $\Pi'$ with $c(\Pi')<c(\Pi)$. The last case to consider is when
  $c(\Pi)=|X|+1$ (note that $c(\Pi)$ cannot be less than $|X|+1$). In this case, $\Pi$ is in
  the form of
  \begin{center}
    \scriptsize
    \begin{math}
      $$\mprset{flushleft}
      \inferrule* [right={\tiny cut}] {
        {
          \begin{array}{cc}
            \Pi_1 & \Pi_2 \\
            {\Phi  \vdash_\mathcal{C}  \Ellent{X}} & {\Psi_{{\mathrm{1}}}  \Ellesym{,}  \Ellent{X}  \Ellesym{,}  \Psi_{{\mathrm{2}}}  \vdash_\mathcal{C}  \Ellent{Y}}
          \end{array}
        }
      }{\Psi_{{\mathrm{1}}}  \Ellesym{,}  \Phi  \Ellesym{,}  \Psi_{{\mathrm{2}}}  \vdash_\mathcal{C}  \Ellent{Y}}
    \end{math}
    \qquad\qquad
    or,
    \begin{math}
      $$\mprset{flushleft}
      \inferrule* [right={\tiny cut1}] {
        {
          \begin{array}{cc}
            \Pi_1 & \Pi_2 \\
            {\Phi  \vdash_\mathcal{C}  \Ellent{X}} & {\Gamma_{{\mathrm{1}}}  \Ellesym{,}  \Ellent{X}  \Ellesym{,}  \Gamma_{{\mathrm{2}}}  \vdash_\mathcal{L}  \Ellent{A}}
          \end{array}
        }
      }{\Gamma_{{\mathrm{1}}}  \Ellesym{,}  \Phi  \Ellesym{,}  \Gamma_{{\mathrm{2}}}  \vdash_\mathcal{L}  \Ellent{A}}
    \end{math}
  \end{center}
  By assumption, $c(\Pi_1),c(\Pi_2)\leq |X|+1$. By induction, we can construct $c(\Pi_1')$
  proving $\Phi  \vdash_\mathcal{C}  \Ellent{X}$ and $c(\Pi_2')$ proving $\Psi_{{\mathrm{1}}}  \Ellesym{,}  \Ellent{X}  \Ellesym{,}  \Psi_{{\mathrm{2}}}  \vdash_\mathcal{C}  \Ellent{Y}$ (or
  $\Gamma_{{\mathrm{1}}}  \Ellesym{,}  \Ellent{X}  \Ellesym{,}  \Gamma_{{\mathrm{2}}}  \vdash_\mathcal{L}  \Ellent{A}$) with $c(\Pi_1'),c(\Pi_2')\leq |X|$. Then by
  Lemma~\ref{lem:cut-reduction}, we can construct $\Pi'$ proving $\Psi_{{\mathrm{1}}}  \Ellesym{,}  \Phi  \Ellesym{,}  \Psi_{{\mathrm{2}}}  \vdash_\mathcal{C}  \Ellent{Y}$ (or
  $\Gamma_{{\mathrm{1}}}  \Ellesym{,}  \Phi  \Ellesym{,}  \Gamma_{{\mathrm{2}}}  \vdash_\mathcal{L}  \Ellent{A}$) with $c(\Pi')\leq |X|$. 

  The case where the last inference is a cut on a formula $A$ is similar as when it is a cut
  on $X$.

\end{proof}

\begin{theorem}[Cut Elimination]
  Let $\Pi$ be a proof of a sequent $\Phi  \vdash_\mathcal{C}  \Ellent{X}$ or $\Gamma  \vdash_\mathcal{L}  \Ellent{A}$ s.t. $c(\Pi)>0$. Then there
  is an algorithm which yields a cut-free proof $\Pi'$ of the same sequent.
\end{theorem}
\begin{proof}
  This follows immediately by induction on $c(\Pi)$ and Lemma~\ref{lem:less-cut-rank}.
\end{proof}



%%%%%%%%%%%%%%%%%%%%%%%%%%%%%%%%%%%%%%%%%%%%%%%%%%
\subsection{Natural Deduction}
\label{subsec:elle-nd}

The term assignment for natural deduction of the commutative part of the model, i.e. the SMCC
of the adjunction, is defined in Figure~\ref{fig:elle-nd-smcc}. And the term assignme for the
non-commutative part, i.e. the Lambek category of the adjunction, is defined in
Figure~\ref{fig:elle-nd-lambek}.

\begin{figure}[!h]
  \scriptsize
  \begin{mdframed}
    \begin{mathpar}
      \NDdruleTXXid{} \qquad\qquad \NDdruleTXXunitI{} \qquad\qquad \NDdruleTXXunitE{} \\
      \NDdruleTXXtenI{} \qquad\qquad \NDdruleTXXtenE{} \\
      \NDdruleTXXimpI{} \qquad\qquad \NDdruleTXXimpE{} \qquad\qquad \NDdruleTXXGI{} \\
      \NDdruleSXXbeta{}
    \end{mathpar}
  \end{mdframed}
\caption{Natural Deduction: Commutative Part}
\label{fig:elle-nd-smcc}
\end{figure}

\begin{figure}[!h]
 \scriptsize
  \begin{mdframed}
    \begin{mathpar}
      \NDdruleSXXid{} \qquad\qquad \NDdruleSXXunitI{} \qquad\qquad \NDdruleSXXunitEOne{} \\
      \NDdruleSXXunitEOne{} \qquad\qquad \NDdruleSXXunitETwo{} \\
      \NDdruleSXXtenI{} \qquad\qquad \NDdruleSXXtenEOne{} \\
      \NDdruleSXXtenETwo{} \qquad\qquad \NDdruleSXXimprI{} \\
      \NDdruleSXXimprE{} \qquad\qquad \NDdruleSXXimplI{} \\
      \NDdruleSXXimplE{} \qquad\qquad \NDdruleSXXGE{} \qquad\qquad \NDdruleSXXFI{} \\
      \NDdruleSXXFE{}
    \end{mathpar}
  \end{mdframed}
\caption{Natural Deduction: Non-Commutative Part}
\label{fig:elle-nd-lambek}
\end{figure}

We could derive exchange comonadically as follows:

\begin{center}
  \tiny
  \begin{math}
  $$\mprset{flushleft}
  \inferrule* [right={\tiny imprI}] {
    $$\mprset{flushleft}
    \inferrule* [right={\tiny tenE2}] {
      $$\mprset{flushleft}
      \inferrule* [right={\tiny id}] {
        \,
      }{\NDmv{z}  \NDsym{:}    \mathsf{F}  \mathsf{G} \NDnt{A}     \triangleright   \mathsf{F}  \mathsf{G} \NDnt{B}    \vdash_\mathcal{L}  \NDmv{z}  \NDsym{:}    \mathsf{F}  \mathsf{G} \NDnt{A}     \triangleright   \mathsf{F}  \mathsf{G} \NDnt{B}  }
        $$\mprset{flushleft}
        \inferrule* [right={\tiny FE}] {
          $$\mprset{flushleft}
          \inferrule* [right={\tiny id}] {
            \,
          }{\NDmv{x_{{\mathrm{2}}}}  \NDsym{:}   \mathsf{F}  \mathsf{G} \NDnt{A}    \vdash_\mathcal{L}  \NDmv{x_{{\mathrm{2}}}}  \NDsym{:}   \mathsf{F}  \mathsf{G} \NDnt{A}  }
            $$\mprset{flushleft}
            \inferrule* [right={\tiny FE}] {
              $$\mprset{flushleft}
              \inferrule* [right={\tiny id}] {
                \,
              }{\NDmv{y_{{\mathrm{2}}}}  \NDsym{:}   \mathsf{F}  \mathsf{G} \NDnt{B}    \vdash_\mathcal{L}  \NDmv{y_{{\mathrm{2}}}}  \NDsym{:}   \mathsf{F}  \mathsf{G} \NDnt{B}  }
              \inferrule* [right={\tiny beta}] {
                $$\mprset{flushleft}
                \inferrule* [right={\tiny FE}] {
                  $$\mprset{flushleft}
                  \inferrule* [right={\tiny FI}] {
                    $$\mprset{flushleft}
                    \inferrule* [right={\tiny id}] {
                      \,
                    }{\NDmv{y_{{\mathrm{0}}}}  \NDsym{:}   \mathsf{G} \NDnt{B}   \vdash_\mathcal{C}  \NDmv{y_{{\mathrm{0}}}}  \NDsym{:}   \mathsf{G} \NDnt{B} }
                  }{\NDmv{y_{{\mathrm{0}}}}  \NDsym{:}   \mathsf{G} \NDnt{B}   \vdash_\mathcal{L}   \mathsf{F} \NDmv{y_{{\mathrm{0}}}}   \NDsym{:}   \mathsf{F}  \mathsf{G} \NDnt{B}  }
                  $$\mprset{flushleft}
                  \inferrule* [right={\tiny FI}] {
                    $$\mprset{flushleft}
                    \inferrule* [right={\tiny id}] {
                      \,
                    }{\NDmv{x_{{\mathrm{0}}}}  \NDsym{:}   \mathsf{G} \NDnt{A}   \vdash_\mathcal{C}  \NDmv{x_{{\mathrm{0}}}}  \NDsym{:}   \mathsf{G} \NDnt{A} }
                  }{\NDmv{x_{{\mathrm{0}}}}  \NDsym{:}   \mathsf{G} \NDnt{A}   \vdash_\mathcal{L}   \mathsf{F} \NDmv{x_{{\mathrm{0}}}}   \NDsym{:}   \mathsf{F}  \mathsf{G} \NDnt{A}  }
                }{\NDmv{y_{{\mathrm{0}}}}  \NDsym{:}   \mathsf{G} \NDnt{B}   \NDsym{,}  \NDmv{x_{{\mathrm{0}}}}  \NDsym{:}   \mathsf{G} \NDnt{A}   \vdash_\mathcal{L}    \mathsf{F} \NDmv{y_{{\mathrm{0}}}}    \triangleright   \mathsf{F} \NDmv{x_{{\mathrm{0}}}}   \NDsym{:}    \mathsf{F}  \mathsf{G} \NDnt{B}     \triangleright   \mathsf{F}  \mathsf{G} \NDnt{A}  }
              }{\NDmv{x_{{\mathrm{1}}}}  \NDsym{:}   \mathsf{G} \NDnt{A}   \NDsym{,}  \NDmv{y_{{\mathrm{1}}}}  \NDsym{:}   \mathsf{G} \NDnt{B}   \vdash_\mathcal{L}   \mathsf{ex}\, \NDmv{y_{{\mathrm{1}}}} , \NDmv{x_{{\mathrm{1}}}} \,\mathsf{with}\, \NDmv{y_{{\mathrm{0}}}} , \NDmv{x_{{\mathrm{0}}}} \,\mathsf{in}\, \NDsym{(}    \mathsf{F} \NDmv{y_{{\mathrm{0}}}}    \triangleright   \mathsf{F} \NDmv{x_{{\mathrm{0}}}}   \NDsym{)}   \NDsym{:}    \mathsf{F}  \mathsf{G} \NDnt{B}     \triangleright   \mathsf{F}  \mathsf{G} \NDnt{A}  }
            }{\NDmv{x_{{\mathrm{1}}}}  \NDsym{:}   \mathsf{G} \NDnt{A}   \NDsym{,}  \NDmv{y_{{\mathrm{2}}}}  \NDsym{:}   \mathsf{F}  \mathsf{G} \NDnt{B}    \vdash_\mathcal{L}   \mathsf{let}\,  \mathsf{F} \NDmv{y_{{\mathrm{1}}}}   :   \mathsf{F}  \mathsf{G} \NDnt{B}   \,\mathsf{be}\, \NDmv{y_{{\mathrm{2}}}} \,\mathsf{in}\, \NDsym{(}   \mathsf{ex}\, \NDmv{y_{{\mathrm{1}}}} , \NDmv{x_{{\mathrm{1}}}} \,\mathsf{with}\, \NDmv{y_{{\mathrm{0}}}} , \NDmv{x_{{\mathrm{0}}}} \,\mathsf{in}\, \NDsym{(}    \mathsf{F} \NDmv{y_{{\mathrm{0}}}}    \triangleright   \mathsf{F} \NDmv{x_{{\mathrm{0}}}}   \NDsym{)}   \NDsym{)}   \NDsym{:}    \mathsf{F}  \mathsf{G} \NDnt{B}     \triangleright   \mathsf{F}  \mathsf{G} \NDnt{A}  }
          }{\NDmv{x_{{\mathrm{2}}}}  \NDsym{:}   \mathsf{F}  \mathsf{G} \NDnt{A}    \NDsym{,}  \NDmv{y_{{\mathrm{2}}}}  \NDsym{:}   \mathsf{F}  \mathsf{G} \NDnt{B}    \vdash_\mathcal{L}   \mathsf{let}\,  \mathsf{F} \NDmv{x_{{\mathrm{1}}}}   :   \mathsf{F}  \mathsf{G} \NDnt{A}   \,\mathsf{be}\, \NDmv{x_{{\mathrm{2}}}} \,\mathsf{in}\, \NDsym{(}   \mathsf{let}\,  \mathsf{F} \NDmv{y_{{\mathrm{1}}}}   :   \mathsf{F}  \mathsf{G} \NDnt{B}   \,\mathsf{be}\, \NDmv{y_{{\mathrm{2}}}} \,\mathsf{in}\, \NDsym{(}   \mathsf{ex}\, \NDmv{y_{{\mathrm{1}}}} , \NDmv{x_{{\mathrm{1}}}} \,\mathsf{with}\, \NDmv{y_{{\mathrm{0}}}} , \NDmv{x_{{\mathrm{0}}}} \,\mathsf{in}\, \NDsym{(}    \mathsf{F} \NDmv{y_{{\mathrm{0}}}}    \triangleright   \mathsf{F} \NDmv{x_{{\mathrm{0}}}}   \NDsym{)}   \NDsym{)}   \NDsym{)}   \NDsym{:}    \mathsf{F}  \mathsf{G} \NDnt{B}     \triangleright   \mathsf{F}  \mathsf{G} \NDnt{A}  }
        }{\NDmv{z}  \NDsym{:}    \mathsf{F}  \mathsf{G} \NDnt{A}     \triangleright   \mathsf{F}  \mathsf{G} \NDnt{B}    \vdash_\mathcal{L}   \mathsf{let}\, \NDmv{z}  :    \mathsf{F}  \mathsf{G} \NDnt{A}     \triangleright   \mathsf{F}  \mathsf{G} \NDnt{B}   \,\mathsf{be}\, \NDmv{x_{{\mathrm{2}}}}  \triangleright  \NDmv{y_{{\mathrm{2}}}} \,\mathsf{in}\, \NDsym{(}   \mathsf{let}\,  \mathsf{F} \NDmv{x_{{\mathrm{1}}}}   :   \mathsf{F}  \mathsf{G} \NDnt{A}   \,\mathsf{be}\, \NDmv{x_{{\mathrm{2}}}} \,\mathsf{in}\, \NDsym{(}   \mathsf{let}\,  \mathsf{F} \NDmv{y_{{\mathrm{1}}}}   :   \mathsf{F}  \mathsf{G} \NDnt{B}   \,\mathsf{be}\, \NDmv{y_{{\mathrm{2}}}} \,\mathsf{in}\, \NDsym{(}   \mathsf{ex}\, \NDmv{y_{{\mathrm{1}}}} , \NDmv{x_{{\mathrm{1}}}} \,\mathsf{with}\, \NDmv{y_{{\mathrm{0}}}} , \NDmv{x_{{\mathrm{0}}}} \,\mathsf{in}\, \NDsym{(}    \mathsf{F} \NDmv{y_{{\mathrm{0}}}}    \triangleright   \mathsf{F} \NDmv{x_{{\mathrm{0}}}}   \NDsym{)}   \NDsym{)}   \NDsym{)}   \NDsym{)}   \NDsym{:}    \mathsf{F}  \mathsf{G} \NDnt{B}     \triangleright   \mathsf{F}  \mathsf{G} \NDnt{A}  }
      }{ \cdot   \vdash_\mathcal{L}   \lambda_r  \NDmv{z}  :    \mathsf{F}  \mathsf{G} \NDnt{A}     \triangleright   \mathsf{F}  \mathsf{G} \NDnt{B}   .  \mathsf{let}\, \NDmv{z}  :    \mathsf{F}  \mathsf{G} \NDnt{A}     \triangleright   \mathsf{F}  \mathsf{G} \NDnt{B}   \,\mathsf{be}\, \NDmv{x_{{\mathrm{2}}}}  \triangleright  \NDmv{y_{{\mathrm{2}}}} \,\mathsf{in}\, \NDsym{(}   \mathsf{let}\,  \mathsf{F} \NDmv{x_{{\mathrm{1}}}}   :   \mathsf{F}  \mathsf{G} \NDnt{A}   \,\mathsf{be}\, \NDmv{x_{{\mathrm{2}}}} \,\mathsf{in}\, \NDsym{(}   \mathsf{let}\,  \mathsf{F} \NDmv{y_{{\mathrm{1}}}}   :   \mathsf{F}  \mathsf{G} \NDnt{B}   \,\mathsf{be}\, \NDmv{y_{{\mathrm{2}}}} \,\mathsf{in}\, \NDsym{(}   \mathsf{ex}\, \NDmv{y_{{\mathrm{1}}}} , \NDmv{x_{{\mathrm{1}}}} \,\mathsf{with}\, \NDmv{y_{{\mathrm{0}}}} , \NDmv{x_{{\mathrm{0}}}} \,\mathsf{in}\, \NDsym{(}    \mathsf{F} \NDmv{y_{{\mathrm{0}}}}    \triangleright   \mathsf{F} \NDmv{x_{{\mathrm{0}}}}   \NDsym{)}   \NDsym{)}   \NDsym{)}   \NDsym{)}    \NDsym{:}  \NDsym{(}    \mathsf{F}  \mathsf{G} \NDnt{A}     \triangleright   \mathsf{F}  \mathsf{G} \NDnt{B}    \NDsym{)}  \rightharpoonup  \NDsym{(}    \mathsf{F}  \mathsf{G} \NDnt{B}     \triangleright   \mathsf{F}  \mathsf{G} \NDnt{A}    \NDsym{)}}
  \end{math}
\end{center}

We also have the three cut rules derivable in the natural deduction:
(NOTE: Don't know how to prove the third one S\_cut2.)

\begin{figure}[!h]
  \scriptsize
  \begin{mathpar}
    \NDdruleTXXcut{} \qquad\qquad \NDdruleSXXcutOne{} \qquad\qquad \NDdruleSXXcutTwo{}
  \end{mathpar}
\end{figure}



%%%%%%%%%%%%%%%%%%%%%%%%%%%%%%%%%%%%%%%%%%%%%%%%%%
\subsubsection{One Step $\beta$-Reduction}

We define the normalization procedure by considering the following pairs of introduction and
elimination rules:

\begin{itemize}

\item (\NDdruleTXXunitIName, \NDdruleTXXunitEName):
  \begin{center}
    \tiny
    \begin{math}
      $$\mprset{flushleft}
      \inferrule* [right={\tiny unitE}] {
        $$\mprset{flushleft}
        \inferrule* [right={\tiny unitI}] {
          \,
        }{ \cdot   \vdash_\mathcal{C}   \mathsf{trivT}   \NDsym{:}   \mathsf{UnitT} } \\
         {\Phi  \vdash_\mathcal{C}  \NDnt{t}  \NDsym{:}  \NDnt{X}}
      }{\Phi  \vdash_\mathcal{C}   \mathsf{let}\,  \mathsf{trivT}   :   \mathsf{UnitT}  \,\mathsf{be}\,  \mathsf{trivT}  \,\mathsf{in}\, \NDnt{t}   \NDsym{:}  \NDnt{X}}
    \end{math}
  \end{center}
  normalizes to 
  \begin{center}
    \tiny
    $\Phi  \vdash_\mathcal{C}  \NDnt{t}  \NDsym{:}  \NDnt{X}$
  \end{center}

\item (\NDdruleTXXunitIName, \NDdruleSXXunitEOneName):
  \begin{center}
    \tiny
    \begin{math}
     $$\mprset{flushleft}
     \inferrule* [right={\tiny unitE2}] {
       $$\mprset{flushleft}
       \inferrule* [right={\tiny unitI}] {
         \,
        }{ \cdot   \vdash_\mathcal{C}   \mathsf{trivT}   \NDsym{:}   \mathsf{UnitT} } \\
         {\Delta  \vdash_\mathcal{L}  \NDnt{s}  \NDsym{:}  \NDnt{A}}
      }{\Delta  \vdash_\mathcal{L}   \mathsf{let}\,  \mathsf{trivT}   :   \mathsf{UnitT}  \,\mathsf{be}\,  \mathsf{trivT}  \,\mathsf{in}\, \NDnt{s}   \NDsym{:}  \NDnt{A}}
    \end{math}
  \end{center}
  normalizes to
  \begin{center}
    \tiny
    $\Delta  \vdash_\mathcal{L}  \NDnt{s}  \NDsym{:}  \NDnt{A}$
  \end{center}

\item (\NDdruleTXXtenIName, \NDdruleTXXtenEName):
  \begin{center}
    \tiny
    \begin{math}
      $$\mprset{flushleft}
      \inferrule* [right={\tiny tenE}] {
        $$\mprset{flushleft}
        \inferrule* [right={\tiny tenI}] {
          {\Phi_{{\mathrm{1}}}  \vdash_\mathcal{C}  \NDnt{t_{{\mathrm{1}}}}  \NDsym{:}  \NDnt{X}} \\
          {\Phi_{{\mathrm{2}}}  \vdash_\mathcal{C}  \NDnt{t_{{\mathrm{2}}}}  \NDsym{:}  \NDnt{Y}}
        }{\Phi_{{\mathrm{1}}}  \NDsym{,}  \Phi_{{\mathrm{2}}}  \vdash_\mathcal{C}  \NDnt{t_{{\mathrm{1}}}}  \otimes  \NDnt{t_{{\mathrm{2}}}}  \NDsym{:}  \NDnt{X}  \otimes  \NDnt{Y}} \\
         {\Psi_{{\mathrm{1}}}  \NDsym{,}  \NDmv{x}  \NDsym{:}  \NDnt{X}  \NDsym{,}  \NDmv{y}  \NDsym{:}  \NDnt{Y}  \NDsym{,}  \Psi_{{\mathrm{2}}}  \vdash_\mathcal{C}  \NDnt{t_{{\mathrm{3}}}}  \NDsym{:}  \NDnt{Z}}
      }{\Psi_{{\mathrm{1}}}  \NDsym{,}  \Phi_{{\mathrm{1}}}  \NDsym{,}  \Phi_{{\mathrm{2}}}  \NDsym{,}  \Psi_{{\mathrm{2}}}  \vdash_\mathcal{C}   \mathsf{let}\, \NDnt{t_{{\mathrm{1}}}}  \otimes  \NDnt{t_{{\mathrm{2}}}}  :  \NDnt{X}  \otimes  \NDnt{Y} \,\mathsf{be}\, \NDmv{x}  \otimes  \NDmv{y} \,\mathsf{in}\, \NDnt{t_{{\mathrm{3}}}}   \NDsym{:}  \NDnt{Z}}
    \end{math}
  \end{center}
  normalizes to
  \begin{center}
    \tiny
    \begin{math}
      $$\mprset{flushleft}
      \inferrule* [right={\tiny cut}] {
        {\Phi_{{\mathrm{1}}}  \vdash_\mathcal{C}  \NDnt{t_{{\mathrm{1}}}}  \NDsym{:}  \NDnt{X}} \\
        $$\mprset{flushleft}
        \inferrule* [right={\tiny cut}] {
          {\Phi_{{\mathrm{2}}}  \vdash_\mathcal{C}  \NDnt{t_{{\mathrm{2}}}}  \NDsym{:}  \NDnt{Y}} \\
          {\Psi_{{\mathrm{1}}}  \NDsym{,}  \NDmv{x}  \NDsym{:}  \NDnt{X}  \NDsym{,}  \NDmv{y}  \NDsym{:}  \NDnt{Y}  \NDsym{,}  \Psi_{{\mathrm{2}}}  \vdash_\mathcal{C}  \NDnt{t_{{\mathrm{3}}}}  \NDsym{:}  \NDnt{Z}}
        }{\Psi_{{\mathrm{1}}}  \NDsym{,}  \NDmv{x}  \NDsym{:}  \NDnt{X}  \NDsym{,}  \Phi_{{\mathrm{2}}}  \NDsym{,}  \Psi_{{\mathrm{2}}}  \vdash_\mathcal{C}  \NDsym{[}  \NDnt{t_{{\mathrm{2}}}}  \NDsym{/}  \NDmv{y}  \NDsym{]}  \NDnt{t_{{\mathrm{3}}}}  \NDsym{:}  \NDnt{Z}}
      }{\Psi_{{\mathrm{1}}}  \NDsym{,}  \Phi_{{\mathrm{1}}}  \NDsym{,}  \Phi_{{\mathrm{2}}}  \NDsym{,}  \Psi_{{\mathrm{2}}}  \vdash_\mathcal{C}  \NDsym{[}  \NDnt{t_{{\mathrm{1}}}}  \NDsym{/}  \NDmv{x}  \NDsym{]}  \NDsym{[}  \NDnt{t_{{\mathrm{2}}}}  \NDsym{/}  \NDmv{y}  \NDsym{]}  \NDnt{t_{{\mathrm{3}}}}  \NDsym{:}  \NDnt{Z}}
    \end{math}
  \end{center}
  
\item (\NDdruleTXXtenIName, \NDdruleSXXtenEOneName):
  \begin{center}
    \tiny
    \begin{math}
      $$\mprset{flushleft}
      \inferrule* [right={\tiny tenE1}] {
        $$\mprset{flushleft}
        \inferrule* [right={\tiny tenI}] {
          {\Phi  \vdash_\mathcal{C}  \NDnt{t_{{\mathrm{1}}}}  \NDsym{:}  \NDnt{X}} \\
          {\Psi  \vdash_\mathcal{C}  \NDnt{t_{{\mathrm{2}}}}  \NDsym{:}  \NDnt{Y}}
        }{\Phi  \NDsym{,}  \Psi  \vdash_\mathcal{C}  \NDnt{t_{{\mathrm{1}}}}  \otimes  \NDnt{t_{{\mathrm{2}}}}  \NDsym{:}  \NDnt{X}  \otimes  \NDnt{Y}} \\
         {\Gamma  \NDsym{,}  \NDmv{x}  \NDsym{:}  \NDnt{X}  \NDsym{,}  \NDmv{y}  \NDsym{:}  \NDnt{Y}  \NDsym{,}  \Delta  \vdash_\mathcal{L}  \NDnt{s}  \NDsym{:}  \NDnt{A}}
      }{\Gamma  \NDsym{,}  \Phi  \NDsym{,}  \Psi  \NDsym{,}  \Delta  \vdash_\mathcal{L}   \mathsf{let}\, \NDnt{t_{{\mathrm{1}}}}  \otimes  \NDnt{t_{{\mathrm{2}}}}  :  \NDnt{X}  \otimes  \NDnt{Y} \,\mathsf{be}\, \NDmv{x}  \otimes  \NDmv{y} \,\mathsf{in}\, \NDnt{s}   \NDsym{:}  \NDnt{A}}
    \end{math}
  \end{center}
  normalizes to
  \begin{center}
    \tiny
    \begin{math}
      $$\mprset{flushleft}
      \inferrule* [right={\tiny cut2}] {
        {\Phi  \vdash_\mathcal{C}  \NDnt{t_{{\mathrm{1}}}}  \NDsym{:}  \NDnt{X}} \\
        $$\mprset{flushleft}
        \inferrule* [right={\tiny cut2}] {
          {\Psi  \vdash_\mathcal{C}  \NDnt{t_{{\mathrm{2}}}}  \NDsym{:}  \NDnt{Y}} \\
          {\Gamma  \NDsym{,}  \NDmv{x}  \NDsym{:}  \NDnt{X}  \NDsym{,}  \NDmv{y}  \NDsym{:}  \NDnt{Y}  \NDsym{,}  \Delta  \vdash_\mathcal{L}  \NDnt{s}  \NDsym{:}  \NDnt{A}}
        }{\Gamma  \NDsym{,}  \NDmv{x}  \NDsym{:}  \NDnt{X}  \NDsym{,}  \Psi  \NDsym{,}  \Delta  \vdash_\mathcal{L}  \NDsym{[}  \NDnt{t_{{\mathrm{2}}}}  \NDsym{/}  \NDmv{y}  \NDsym{]}  \NDnt{s}  \NDsym{:}  \NDnt{A}}
      }{\Gamma  \NDsym{,}  \Phi  \NDsym{,}  \Psi  \NDsym{,}  \Delta  \vdash_\mathcal{L}  \NDsym{[}  \NDnt{t_{{\mathrm{1}}}}  \NDsym{/}  \NDmv{x}  \NDsym{]}  \NDsym{[}  \NDnt{t_{{\mathrm{2}}}}  \NDsym{/}  \NDmv{y}  \NDsym{]}  \NDnt{s}  \NDsym{:}  \NDnt{A}}
    \end{math}
  \end{center}
  
\item (\NDdruleTXXimpIName, \NDdruleTXXimpEName):
  \begin{center}
    \tiny
    \begin{math}
      $$\mprset{flushleft}
      \inferrule* [right={\tiny impE}] {
        $$\mprset{flushleft}
        \inferrule* [right={\tiny impI}] {
          {\Phi  \NDsym{,}  \NDmv{x}  \NDsym{:}  \NDnt{X}  \vdash_\mathcal{C}  \NDnt{t_{{\mathrm{1}}}}  \NDsym{:}  \NDnt{Y}}
        }{\Phi  \vdash_\mathcal{C}   \lambda  \NDmv{x}  :  \NDnt{X} . \NDnt{t_{{\mathrm{1}}}}   \NDsym{:}  \NDnt{X}  \multimap  \NDnt{Y}} \\
         {\Psi  \vdash_\mathcal{C}  \NDnt{t_{{\mathrm{2}}}}  \NDsym{:}  \NDnt{X}}
      }{\Phi  \NDsym{,}  \Psi  \vdash_\mathcal{C}   \mathsf{app}\, \NDsym{(}   \lambda  \NDmv{x}  :  \NDnt{X} . \NDnt{t_{{\mathrm{1}}}}   \NDsym{)} \, \NDnt{t_{{\mathrm{2}}}}   \NDsym{:}  \NDnt{Y}}
    \end{math}
  \end{center}
  normalizes to
  \begin{center}
    \tiny
    \begin{math}
      $$\mprset{flushleft}
      \inferrule* [right={\tiny cut}] {
        {\Phi  \NDsym{,}  \NDmv{x}  \NDsym{:}  \NDnt{X}  \vdash_\mathcal{C}  \NDnt{t_{{\mathrm{1}}}}  \NDsym{:}  \NDnt{Y}} \\
        {\Psi  \vdash_\mathcal{C}  \NDnt{t_{{\mathrm{2}}}}  \NDsym{:}  \NDnt{X}}
      }{\Phi  \NDsym{,}  \Psi  \vdash_\mathcal{C}  \NDsym{[}  \NDnt{t_{{\mathrm{2}}}}  \NDsym{/}  \NDmv{x}  \NDsym{]}  \NDnt{t_{{\mathrm{1}}}}  \NDsym{:}  \NDnt{Y}}
    \end{math}
  \end{center}

\item (\NDdruleSXXunitIName, \NDdruleSXXunitETwoName):
  \begin{center}
    \tiny
    \begin{math}
     $$\mprset{flushleft}
     \inferrule* [right={\tiny unitE2}] {
       $$\mprset{flushleft}
       \inferrule* [right={\tiny unitI}] {
         \,
        }{ \cdot   \vdash_\mathcal{L}   \mathsf{trivS}   \NDsym{:}   \mathsf{UnitS} } \\
         {\Delta  \vdash_\mathcal{L}  \NDnt{s}  \NDsym{:}  \NDnt{A}}
      }{\Delta  \vdash_\mathcal{L}   \mathsf{let}\,  \mathsf{trivS}   :   \mathsf{UnitS}  \,\mathsf{be}\,  \mathsf{trivS}  \,\mathsf{in}\, \NDnt{s}   \NDsym{:}  \NDnt{A}}
    \end{math}
  \end{center}
  normalizes to
  \begin{center}
    \tiny
    $\Delta  \vdash_\mathcal{L}  \NDnt{s}  \NDsym{:}  \NDnt{A}$
  \end{center}

\item (\NDdruleSXXtenIName, \NDdruleSXXtenETwoName):
  \begin{center}
    \tiny
    \begin{math}
     $$\mprset{flushleft}
     \inferrule* [right={\tiny tenE2}] {
       $$\mprset{flushleft}
       \inferrule* [right={\tiny tenI}] {
         {\Gamma_{{\mathrm{1}}}  \vdash_\mathcal{L}  \NDnt{s_{{\mathrm{1}}}}  \NDsym{:}  \NDnt{A}} \\
         {\Gamma_{{\mathrm{2}}}  \vdash_\mathcal{L}  \NDnt{s_{{\mathrm{2}}}}  \NDsym{:}  \NDnt{B}}
        }{\Gamma_{{\mathrm{1}}}  \NDsym{,}  \Gamma_{{\mathrm{2}}}  \vdash_\mathcal{L}  \NDnt{s_{{\mathrm{1}}}}  \triangleright  \NDnt{s_{{\mathrm{2}}}}  \NDsym{:}  \NDnt{A}  \triangleright  \NDnt{B}} \\
         {\Delta_{{\mathrm{1}}}  \NDsym{,}  \NDmv{x}  \NDsym{:}  \NDnt{A}  \NDsym{,}  \NDmv{y}  \NDsym{:}  \NDnt{B}  \NDsym{,}  \Delta_{{\mathrm{2}}}  \vdash_\mathcal{L}  \NDnt{s_{{\mathrm{3}}}}  \NDsym{:}  \NDnt{C}}
      }{\Delta_{{\mathrm{1}}}  \NDsym{,}  \Gamma_{{\mathrm{1}}}  \NDsym{,}  \Gamma_{{\mathrm{2}}}  \NDsym{,}  \Delta_{{\mathrm{2}}}  \vdash_\mathcal{L}   \mathsf{let}\, \NDnt{s_{{\mathrm{1}}}}  \triangleright  \NDnt{s_{{\mathrm{2}}}}  :  \NDnt{A}  \triangleright  \NDnt{B} \,\mathsf{be}\, \NDmv{x}  \triangleright  \NDmv{y} \,\mathsf{in}\, \NDnt{s_{{\mathrm{3}}}}   \NDsym{:}  \NDnt{C}}
    \end{math}
  \end{center}
  normalizes to
  \begin{center}
    \tiny
    \begin{math}
      $$\mprset{flushleft}
      \inferrule* [right={\tiny cut2}] {
        {\Gamma_{{\mathrm{1}}}  \vdash_\mathcal{L}  \NDnt{s_{{\mathrm{1}}}}  \NDsym{:}  \NDnt{A}} \\
        $$\mprset{flushleft}
        \inferrule* [right={\tiny cut2}] {
          {\Gamma_{{\mathrm{2}}}  \vdash_\mathcal{L}  \NDnt{s_{{\mathrm{2}}}}  \NDsym{:}  \NDnt{B}} \\
          {\Delta_{{\mathrm{1}}}  \NDsym{,}  \NDmv{x}  \NDsym{:}  \NDnt{X}  \NDsym{,}  \NDmv{y}  \NDsym{:}  \NDnt{Y}  \NDsym{,}  \Delta_{{\mathrm{2}}}  \vdash_\mathcal{L}  \NDnt{s_{{\mathrm{3}}}}  \NDsym{:}  \NDnt{C}}
        }{\Delta_{{\mathrm{1}}}  \NDsym{,}  \NDmv{x}  \NDsym{:}  \NDnt{X}  \NDsym{,}  \Gamma_{{\mathrm{2}}}  \NDsym{,}  \Delta_{{\mathrm{2}}}  \vdash_\mathcal{L}  \NDsym{[}  \NDnt{s_{{\mathrm{2}}}}  \NDsym{/}  \NDmv{y}  \NDsym{]}  \NDnt{s_{{\mathrm{3}}}}  \NDsym{:}  \NDnt{C}}
      }{\Delta_{{\mathrm{1}}}  \NDsym{,}  \Gamma_{{\mathrm{1}}}  \NDsym{,}  \Gamma_{{\mathrm{2}}}  \NDsym{,}  \Delta_{{\mathrm{2}}}  \vdash_\mathcal{L}  \NDsym{[}  \NDnt{s_{{\mathrm{1}}}}  \NDsym{/}  \NDmv{x}  \NDsym{]}  \NDsym{[}  \NDnt{s_{{\mathrm{2}}}}  \NDsym{/}  \NDmv{y}  \NDsym{]}  \NDnt{s_{{\mathrm{3}}}}  \NDsym{:}  \NDnt{C}}
    \end{math}
  \end{center}
        
\item (\NDdruleSXXimprIName, \NDdruleSXXimprEName):
  \begin{center}
    \tiny
    \begin{math}
     $$\mprset{flushleft}
     \inferrule* [right={\tiny unitE2}] {
       $$\mprset{flushleft}
       \inferrule* [right={\tiny imprI}] {
         {\Gamma  \NDsym{,}  \NDmv{x}  \NDsym{:}  \NDnt{A}  \vdash_\mathcal{L}  \NDnt{s_{{\mathrm{1}}}}  \NDsym{:}  \NDnt{B}}
        }{\Gamma  \vdash_\mathcal{L}   \lambda_r  \NDmv{x}  :  \NDnt{A} . \NDnt{s_{{\mathrm{1}}}}   \NDsym{:}  \NDnt{A}  \rightharpoonup  \NDnt{B}} \\
         {\Delta  \vdash_\mathcal{L}  \NDnt{s_{{\mathrm{2}}}}  \NDsym{:}  \NDnt{A}}
      }{\Gamma  \NDsym{,}  \Delta  \vdash_\mathcal{L}   \mathsf{app}_r\, \NDsym{(}   \lambda_r  \NDmv{x}  :  \NDnt{A} . \NDnt{s_{{\mathrm{1}}}}   \NDsym{)} \, \NDnt{s_{{\mathrm{2}}}}   \NDsym{:}  \NDnt{B}}
    \end{math}
  \end{center}
  normalizes to
  \begin{center}
    \tiny
    \begin{math}
      $$\mprset{flushleft}
      \inferrule* [right={\tiny cut2}] {
        {\Gamma  \NDsym{,}  \NDmv{x}  \NDsym{:}  \NDnt{A}  \vdash_\mathcal{L}  \NDnt{s_{{\mathrm{1}}}}  \NDsym{:}  \NDnt{B}} \\
        {\Delta  \vdash_\mathcal{L}  \NDnt{s_{{\mathrm{2}}}}  \NDsym{:}  \NDnt{A}}
      }{\Gamma  \NDsym{,}  \Delta  \vdash_\mathcal{L}  \NDsym{[}  \NDnt{s_{{\mathrm{2}}}}  \NDsym{/}  \NDmv{x}  \NDsym{]}  \NDnt{s_{{\mathrm{1}}}}  \NDsym{:}  \NDnt{B}}
    \end{math}
  \end{center}
        
\item (\NDdruleSXXimplIName, \NDdruleSXXimplEName):
  \begin{center}
    \tiny
    \begin{math}
     $$\mprset{flushleft}
     \inferrule* [right={\tiny unitE2}] {
       $$\mprset{flushleft}
       \inferrule* [right={\tiny implI}] {
         {\NDmv{x}  \NDsym{:}  \NDnt{A}  \NDsym{,}  \Gamma  \vdash_\mathcal{L}  \NDnt{s_{{\mathrm{1}}}}  \NDsym{:}  \NDnt{B}}
        }{\Gamma  \vdash_\mathcal{L}   \lambda_l  \NDmv{x}  :  \NDnt{A} . \NDnt{s_{{\mathrm{1}}}}   \NDsym{:}  \NDnt{B}  \leftharpoonup  \NDnt{A}} \\
         {\Delta  \vdash_\mathcal{L}  \NDnt{s_{{\mathrm{2}}}}  \NDsym{:}  \NDnt{A}}
      }{\Delta  \NDsym{,}  \Gamma  \vdash_\mathcal{L}   \mathsf{app}_l\, \NDsym{(}   \lambda_l  \NDmv{x}  :  \NDnt{A} . \NDnt{s_{{\mathrm{1}}}}   \NDsym{)} \, \NDnt{s_{{\mathrm{2}}}}   \NDsym{:}  \NDnt{B}}
    \end{math}
  \end{center}
  normalizes to
  \begin{center}
    \tiny
    \begin{math}
      $$\mprset{flushleft}
      \inferrule* [right={\tiny cut2}] {
        {\NDmv{x}  \NDsym{:}  \NDnt{A}  \NDsym{,}  \Gamma  \vdash_\mathcal{L}  \NDnt{s_{{\mathrm{1}}}}  \NDsym{:}  \NDnt{B}} \\
        {\Delta  \vdash_\mathcal{L}  \NDnt{s_{{\mathrm{2}}}}  \NDsym{:}  \NDnt{A}}
      }{\Delta  \NDsym{,}  \Gamma  \vdash_\mathcal{L}  \NDsym{[}  \NDnt{s_{{\mathrm{2}}}}  \NDsym{/}  \NDmv{x}  \NDsym{]}  \NDnt{s_{{\mathrm{1}}}}  \NDsym{:}  \NDnt{B}}
    \end{math}
  \end{center}
        
\item (\NDdruleSXXFIName, \NDdruleSXXFEName):
  \begin{center}
    \tiny
    \begin{math}
      $$\mprset{flushleft}
      \inferrule* [right={\tiny FE}] {
        $$\mprset{flushleft}
        \inferrule* [right={\tiny FI}] {
          {\Phi  \vdash_\mathcal{C}  \NDmv{y}  \NDsym{:}  \NDnt{X}}
        }{\Phi  \vdash_\mathcal{L}   \mathsf{F} \NDmv{y}   \NDsym{:}   \mathsf{F} \NDnt{X} } \\
         {\Delta_{{\mathrm{1}}}  \NDsym{,}  \NDmv{x}  \NDsym{:}  \NDnt{X}  \NDsym{,}  \Delta_{{\mathrm{2}}}  \vdash_\mathcal{L}  \NDnt{s}  \NDsym{:}  \NDnt{A}}
      }{\Delta_{{\mathrm{1}}}  \NDsym{,}  \Phi  \NDsym{,}  \Delta_{{\mathrm{2}}}  \vdash_\mathcal{L}   \mathsf{let}\,  \mathsf{F} \NDmv{x}   :   \mathsf{F} \NDnt{X}  \,\mathsf{be}\,  \mathsf{F}\, \NDmv{y}  \,\mathsf{in}\, \NDnt{s}   \NDsym{:}  \NDnt{A}}
    \end{math}
  \end{center}
  normalizes to
  \begin{center}
    \tiny
    \begin{math}
      $$\mprset{flushleft}
      \inferrule* [right={\tiny cut1}] {
        {\Phi  \vdash_\mathcal{C}  \NDmv{y}  \NDsym{:}  \NDnt{X}} \\
        {\Delta_{{\mathrm{1}}}  \NDsym{,}  \NDmv{x}  \NDsym{:}  \NDnt{X}  \NDsym{,}  \Delta_{{\mathrm{2}}}  \vdash_\mathcal{L}  \NDnt{s}  \NDsym{:}  \NDnt{A}}
      }{\Delta_{{\mathrm{1}}}  \NDsym{,}  \Phi  \NDsym{,}  \Delta_{{\mathrm{2}}}  \vdash_\mathcal{L}  \NDsym{[}  \NDmv{y}  \NDsym{/}  \NDmv{x}  \NDsym{]}  \NDnt{s}  \NDsym{:}  \NDnt{A}}
    \end{math}
  \end{center}

\item (\NDdruleTXXGIName, \NDdruleSXXGEName):
  \begin{center}
    \tiny
    \begin{math}
      $$\mprset{flushleft}
      \inferrule* [right={\tiny GE}] {
        $$\mprset{flushleft}
        \inferrule* [right={\tiny GI}] {
          {\Phi  \vdash_\mathcal{L}  \NDnt{s}  \NDsym{:}  \NDnt{A}}
        }{\Phi  \vdash_\mathcal{C}   \mathsf{G} \NDnt{s}   \NDsym{:}   \mathsf{G} \NDnt{A} }
      }{\Phi  \vdash_\mathcal{L}   \mathsf{derelict}\, \NDsym{(}   \mathsf{G} \NDnt{s}   \NDsym{)}   \NDsym{:}  \NDnt{A}}
    \end{math}
  \end{center}
  normalizes to
  \begin{center}
    \tiny
    $\Phi  \vdash_\mathcal{L}  \NDnt{s}  \NDsym{:}  \NDnt{A}$
  \end{center}

\end{itemize}

\begin{theorem}[Normalization]
  For a cut-free deduction $\Pi$, there is a deduction which is in normal form.
\end{theorem}
\begin{proof}
  By induction on the structure of $\Pi$.
\end{proof}



  \begin{center}
    \tiny
    $\texttt{\textcolor{red}{\texttt{\textcolor{red}{<<no parses (char 2): no*** parses (char 2): Th***eta \mbox{$\backslash{}$}mbox\{\$\mbox{$\backslash{}$}mid\$\}-c X  >>}}}}$
  \end{center}



%%%%%%%%%%%%%%%%%%%%%%%%%%%%%%%%%%%%%%%%%%%%%%%%%%
\subsubsection{Commuting Conversions}

\begin{itemize}

\item Commutation of $\mathrm{UnitT}_E$:
  \begin{itemize}

  \item (\NDdruleTXXunitEName, \NDdruleTXXunitEName):
    \begin{center}
      \tiny
      \begin{math}
        $$\mprset{flushleft}
        \inferrule* [right={\tiny unitE}] {
          $$\mprset{flushleft}
          \inferrule* [right={\tiny unitE}] {
            {\Phi_{{\mathrm{1}}}  \vdash_\mathcal{C}  \NDnt{t_{{\mathrm{1}}}}  \NDsym{:}   \mathsf{UnitT} } \\
            {\Phi_{{\mathrm{2}}}  \vdash_\mathcal{C}  \NDnt{t_{{\mathrm{2}}}}  \NDsym{:}   \mathsf{UnitT} }
          }{\Phi_{{\mathrm{2}}}  \NDsym{,}  \Phi_{{\mathrm{1}}}  \vdash_\mathcal{C}   \mathsf{let}\, \NDnt{t_{{\mathrm{2}}}}  :   \mathsf{UnitT}  \,\mathsf{be}\,  \mathsf{trivT}  \,\mathsf{in}\, \NDnt{t_{{\mathrm{1}}}}   \NDsym{:}   \mathsf{UnitT} } \\
          {\Phi_{{\mathrm{3}}}  \vdash_\mathcal{C}  \NDnt{t_{{\mathrm{3}}}}  \NDsym{:}  \NDnt{X}}
        }{\Phi_{{\mathrm{2}}}  \NDsym{,}  \Phi_{{\mathrm{1}}}  \NDsym{,}  \Phi_{{\mathrm{3}}}  \vdash_\mathcal{C}   \mathsf{let}\, \NDsym{(}   \mathsf{let}\, \NDnt{t_{{\mathrm{2}}}}  :   \mathsf{UnitT}  \,\mathsf{be}\,  \mathsf{trivT}  \,\mathsf{in}\, \NDnt{t_{{\mathrm{1}}}}   \NDsym{)}  :   \mathsf{UnitT}  \,\mathsf{be}\,  \mathsf{trivT}  \,\mathsf{in}\, \NDnt{t_{{\mathrm{3}}}}   \NDsym{:}  \NDnt{X}}
      \end{math}
    \end{center}
    commutes to
    \begin{center}
      \tiny
      \begin{math}
        $$\mprset{flushleft}
        \inferrule* [right={\tiny unitE}] {
          $$\mprset{flushleft}
          \inferrule* [right={\tiny unitE}] {
            {\Phi_{{\mathrm{1}}}  \vdash_\mathcal{C}  \NDnt{t_{{\mathrm{1}}}}  \NDsym{:}   \mathsf{UnitT} } \\
            {\Phi_{{\mathrm{3}}}  \vdash_\mathcal{C}  \NDnt{t_{{\mathrm{3}}}}  \NDsym{:}  \NDnt{X}}
          }{\Phi_{{\mathrm{1}}}  \NDsym{,}  \Phi_{{\mathrm{3}}}  \vdash_\mathcal{C}   \mathsf{let}\, \NDnt{t_{{\mathrm{1}}}}  :   \mathsf{UnitT}  \,\mathsf{be}\,  \mathsf{trivT}  \,\mathsf{in}\, \NDnt{t_{{\mathrm{3}}}}   \NDsym{:}  \NDnt{X}} \\
           {\Phi_{{\mathrm{2}}}  \vdash_\mathcal{C}  \NDnt{t_{{\mathrm{2}}}}  \NDsym{:}   \mathsf{UnitT} }
        }{\Phi_{{\mathrm{2}}}  \NDsym{,}  \Phi_{{\mathrm{1}}}  \NDsym{,}  \Phi_{{\mathrm{3}}}  \vdash_\mathcal{C}   \mathsf{let}\, \NDnt{t_{{\mathrm{2}}}}  :   \mathsf{UnitT}  \,\mathsf{be}\,  \mathsf{trivT}  \,\mathsf{in}\, \NDsym{(}   \mathsf{let}\, \NDnt{t_{{\mathrm{1}}}}  :   \mathsf{UnitT}  \,\mathsf{be}\,  \mathsf{trivT}  \,\mathsf{in}\, \NDnt{t_{{\mathrm{3}}}}   \NDsym{)}   \NDsym{:}  \NDnt{X}}
      \end{math}
    \end{center}

  \item (\NDdruleTXXunitEName, \NDdruleTXXtenEName) need multiple exchanges at the end:
    \begin{center}
      \tiny
      \begin{math}
        $$\mprset{flushleft}
        \inferrule* [right={\tiny tenE}] {
          $$\mprset{flushleft}
          \inferrule* [right={\tiny unitE}] {
            {\Phi_{{\mathrm{1}}}  \vdash_\mathcal{C}  \NDnt{t_{{\mathrm{1}}}}  \NDsym{:}  \NDnt{X}  \otimes  \NDnt{Y}} \\
            {\Phi_{{\mathrm{2}}}  \vdash_\mathcal{C}  \NDnt{t_{{\mathrm{2}}}}  \NDsym{:}   \mathsf{UnitT} }
          }{\Phi_{{\mathrm{2}}}  \NDsym{,}  \Phi_{{\mathrm{1}}}  \vdash_\mathcal{C}   \mathsf{let}\, \NDnt{t_{{\mathrm{2}}}}  :   \mathsf{UnitT}  \,\mathsf{be}\,  \mathsf{trivT}  \,\mathsf{in}\, \NDnt{t_{{\mathrm{1}}}}   \NDsym{:}  \NDnt{X}  \otimes  \NDnt{Y}} \\
           {\Psi_{{\mathrm{1}}}  \NDsym{,}  \NDmv{x}  \NDsym{:}  \NDnt{X}  \NDsym{,}  \NDmv{y}  \NDsym{:}  \NDnt{Y}  \NDsym{,}  \Psi_{{\mathrm{2}}}  \vdash_\mathcal{C}  \NDnt{t_{{\mathrm{3}}}}  \NDsym{:}  \NDnt{Z}}
        }{\Psi_{{\mathrm{1}}}  \NDsym{,}  \Phi_{{\mathrm{2}}}  \NDsym{,}  \Phi_{{\mathrm{1}}}  \NDsym{,}  \Psi_{{\mathrm{2}}}  \vdash_\mathcal{C}   \mathsf{let}\, \NDsym{(}   \mathsf{let}\, \NDnt{t_{{\mathrm{2}}}}  :   \mathsf{UnitT}  \,\mathsf{be}\,  \mathsf{trivT}  \,\mathsf{in}\, \NDnt{t_{{\mathrm{1}}}}   \NDsym{)}  :  \NDnt{X}  \otimes  \NDnt{Y} \,\mathsf{be}\, \NDmv{x}  \otimes  \NDmv{y} \,\mathsf{in}\, \NDnt{t_{{\mathrm{3}}}}   \NDsym{:}  \NDnt{Z}}
      \end{math}
    \end{center}
    commutes to
    \begin{center}
      \tiny
      \begin{math}
        $$\mprset{flushleft}
        \inferrule* [right={\tiny unitE}] {
          $$\mprset{flushleft}
          \inferrule* [right={\tiny tenE}] {
            {\Phi_{{\mathrm{1}}}  \vdash_\mathcal{C}  \NDnt{t_{{\mathrm{1}}}}  \NDsym{:}  \NDnt{X}  \otimes  \NDnt{Y}} \\
            {\Psi_{{\mathrm{1}}}  \NDsym{,}  \NDmv{x}  \NDsym{:}  \NDnt{X}  \NDsym{,}  \NDmv{y}  \NDsym{:}  \NDnt{Y}  \NDsym{,}  \Psi_{{\mathrm{2}}}  \vdash_\mathcal{C}  \NDnt{t_{{\mathrm{3}}}}  \NDsym{:}  \NDnt{Z}}
          }{\Psi_{{\mathrm{1}}}  \NDsym{,}  \Phi_{{\mathrm{1}}}  \NDsym{,}  \Psi_{{\mathrm{2}}}  \vdash_\mathcal{C}   \mathsf{let}\, \NDnt{t_{{\mathrm{1}}}}  :  \NDnt{X}  \otimes  \NDnt{Y} \,\mathsf{be}\, \NDmv{x}  \otimes  \NDmv{y} \,\mathsf{in}\, \NDnt{t_{{\mathrm{3}}}}   \NDsym{:}  \NDnt{Z}} \\
           {\Phi_{{\mathrm{2}}}  \vdash_\mathcal{C}  \NDnt{t_{{\mathrm{2}}}}  \NDsym{:}   \mathsf{UnitT} }
        }{\Phi_{{\mathrm{2}}}  \NDsym{,}  \Psi_{{\mathrm{1}}}  \NDsym{,}  \Phi_{{\mathrm{1}}}  \NDsym{,}  \Psi_{{\mathrm{2}}}  \vdash_\mathcal{C}   \mathsf{let}\, \NDnt{t_{{\mathrm{2}}}}  :   \mathsf{UnitT}  \,\mathsf{be}\,  \mathsf{trivT}  \,\mathsf{in}\, \NDsym{(}   \mathsf{let}\, \NDnt{t_{{\mathrm{1}}}}  :  \NDnt{X}  \otimes  \NDnt{Y} \,\mathsf{be}\, \NDmv{x}  \otimes  \NDmv{y} \,\mathsf{in}\, \NDnt{t_{{\mathrm{3}}}}   \NDsym{)}   \NDsym{:}  \NDnt{Z}}
      \end{math}
    \end{center}

  \item (\NDdruleTXXunitEName, \NDdruleTXXimpEName):
    \begin{center}
      \tiny
      \begin{math}
        $$\mprset{flushleft}
        \inferrule* [right={\tiny unitE}] {
          $$\mprset{flushleft}
          \inferrule* [right={\tiny tenE}] {
            {\Phi_{{\mathrm{1}}}  \vdash_\mathcal{C}  \NDnt{t_{{\mathrm{1}}}}  \NDsym{:}  \NDnt{X}  \multimap  \NDnt{Y}} \\
            {\Phi_{{\mathrm{2}}}  \vdash_\mathcal{C}  \NDnt{t_{{\mathrm{2}}}}  \NDsym{:}   \mathsf{UnitT} }
          }{\Phi_{{\mathrm{2}}}  \NDsym{,}  \Phi_{{\mathrm{1}}}  \vdash_\mathcal{C}   \mathsf{let}\, \NDnt{t_{{\mathrm{2}}}}  :   \mathsf{UnitT}  \,\mathsf{be}\,  \mathsf{trivT}  \,\mathsf{in}\, \NDnt{t_{{\mathrm{1}}}}   \NDsym{:}  \NDnt{X}  \multimap  \NDnt{Y}} \\
           {\Phi_{{\mathrm{3}}}  \vdash_\mathcal{C}  \NDnt{t_{{\mathrm{3}}}}  \NDsym{:}  \NDnt{X}}
        }{\Phi_{{\mathrm{2}}}  \NDsym{,}  \Phi_{{\mathrm{1}}}  \NDsym{,}  \Phi_{{\mathrm{3}}}  \vdash_\mathcal{C}   \mathsf{app}\, \NDsym{(}   \mathsf{let}\, \NDnt{t_{{\mathrm{2}}}}  :   \mathsf{UnitT}  \,\mathsf{be}\,  \mathsf{trivT}  \,\mathsf{in}\, \NDnt{t_{{\mathrm{1}}}}   \NDsym{)} \, \NDnt{t_{{\mathrm{3}}}}   \NDsym{:}  \NDnt{Y}}
      \end{math}
    \end{center}
    commutes to
    \begin{center}
      \tiny
      \begin{math}
        $$\mprset{flushleft}
        \inferrule* [right={\tiny tenE}] {
          $$\mprset{flushleft}
          \inferrule* [right={\tiny unitE}] {
            {\Phi_{{\mathrm{1}}}  \vdash_\mathcal{C}  \NDnt{t_{{\mathrm{1}}}}  \NDsym{:}  \NDnt{X}  \multimap  \NDnt{Y}} \\
            {\Phi_{{\mathrm{3}}}  \vdash_\mathcal{C}  \NDnt{t_{{\mathrm{3}}}}  \NDsym{:}  \NDnt{X}}
          }{\Phi_{{\mathrm{1}}}  \NDsym{,}  \Phi_{{\mathrm{3}}}  \vdash_\mathcal{C}   \mathsf{app}\, \NDnt{t_{{\mathrm{1}}}} \, \NDnt{t_{{\mathrm{3}}}}   \NDsym{:}  \NDnt{Y}} \\
           {\Phi_{{\mathrm{2}}}  \vdash_\mathcal{C}  \NDnt{t_{{\mathrm{2}}}}  \NDsym{:}   \mathsf{UnitT} }
        }{\Phi_{{\mathrm{2}}}  \NDsym{,}  \Phi_{{\mathrm{1}}}  \NDsym{,}  \Phi_{{\mathrm{3}}}  \vdash_\mathcal{C}   \mathsf{let}\, \NDnt{t_{{\mathrm{2}}}}  :   \mathsf{UnitT}  \,\mathsf{be}\,  \mathsf{trivT}  \,\mathsf{in}\, \NDsym{(}   \mathsf{app}\, \NDnt{t_{{\mathrm{1}}}} \, \NDnt{t_{{\mathrm{3}}}}   \NDsym{)}   \NDsym{:}  \NDnt{Y}}
      \end{math}
    \end{center}
  \end{itemize}


\item Commutation of $\otimes_E$:

  \begin{itemize}
  \item (\NDdruleTXXtenEName, \NDdruleTXXunitEName):
    \begin{center}
      \tiny
      \begin{math}
        $$\mprset{flushleft}
        \inferrule* [right={\tiny unitE}] {
          $$\mprset{flushleft}
          \inferrule* [right={\tiny tenE}] {
            {\Phi_{{\mathrm{1}}}  \NDsym{,}  \NDmv{x}  \NDsym{:}  \NDnt{X}  \NDsym{,}  \NDmv{y}  \NDsym{:}  \NDnt{Y}  \NDsym{,}  \Phi_{{\mathrm{2}}}  \vdash_\mathcal{C}  \NDnt{t_{{\mathrm{1}}}}  \NDsym{:}   \mathsf{UnitT} } \\
            {\Psi_{{\mathrm{1}}}  \vdash_\mathcal{C}  \NDnt{t_{{\mathrm{2}}}}  \NDsym{:}  \NDnt{X}  \otimes  \NDnt{Y}}
          }{\Phi_{{\mathrm{1}}}  \NDsym{,}  \Psi_{{\mathrm{1}}}  \NDsym{,}  \Phi_{{\mathrm{2}}}  \vdash_\mathcal{C}   \mathsf{let}\, \NDnt{t_{{\mathrm{2}}}}  :  \NDnt{X}  \otimes  \NDnt{Y} \,\mathsf{be}\, \NDmv{x}  \otimes  \NDmv{y} \,\mathsf{in}\, \NDnt{t_{{\mathrm{1}}}}   \NDsym{:}   \mathsf{UnitT} } \\
           {\Psi_{{\mathrm{2}}}  \vdash_\mathcal{C}  \NDnt{t_{{\mathrm{3}}}}  \NDsym{:}  \NDnt{Z}}
        }{\Phi_{{\mathrm{1}}}  \NDsym{,}  \Psi_{{\mathrm{1}}}  \NDsym{,}  \Phi_{{\mathrm{2}}}  \NDsym{,}  \Psi_{{\mathrm{2}}}  \vdash_\mathcal{C}   \mathsf{let}\, \NDsym{(}   \mathsf{let}\, \NDnt{t_{{\mathrm{2}}}}  :  \NDnt{X}  \otimes  \NDnt{Y} \,\mathsf{be}\, \NDmv{x}  \otimes  \NDmv{y} \,\mathsf{in}\, \NDnt{t_{{\mathrm{1}}}}   \NDsym{)}  :   \mathsf{UnitT}  \,\mathsf{be}\,  \mathsf{trivT}  \,\mathsf{in}\, \NDnt{t_{{\mathrm{3}}}}   \NDsym{:}  \NDnt{Z}}
      \end{math}
    \end{center}
    commutes to
    \begin{center}
      \tiny
      \begin{math}
        $$\mprset{flushleft}
        \inferrule* [right={\tiny tenE}] {
          $$\mprset{flushleft}
          \inferrule* [right={\tiny unitE}] {
            {\Phi_{{\mathrm{1}}}  \NDsym{,}  \NDmv{x}  \NDsym{:}  \NDnt{X}  \NDsym{,}  \NDmv{y}  \NDsym{:}  \NDnt{Y}  \NDsym{,}  \Phi_{{\mathrm{2}}}  \vdash_\mathcal{C}  \NDnt{t_{{\mathrm{1}}}}  \NDsym{:}   \mathsf{UnitT} } \\
            {\Psi_{{\mathrm{2}}}  \vdash_\mathcal{C}  \NDnt{t_{{\mathrm{3}}}}  \NDsym{:}  \NDnt{Z}}
          }{\Phi_{{\mathrm{1}}}  \NDsym{,}  \NDmv{x}  \NDsym{:}  \NDnt{X}  \NDsym{,}  \NDmv{y}  \NDsym{:}  \NDnt{Y}  \NDsym{,}  \Phi_{{\mathrm{2}}}  \NDsym{,}  \Psi_{{\mathrm{2}}}  \vdash_\mathcal{C}   \mathsf{let}\, \NDnt{t_{{\mathrm{1}}}}  :   \mathsf{UnitT}  \,\mathsf{be}\,  \mathsf{trivT}  \,\mathsf{in}\, \NDnt{t_{{\mathrm{3}}}}   \NDsym{:}  \NDnt{Z}} \\
           {\Psi_{{\mathrm{1}}}  \vdash_\mathcal{C}  \NDnt{t_{{\mathrm{2}}}}  \NDsym{:}  \NDnt{X}  \otimes  \NDnt{Y}}
        }{\Phi_{{\mathrm{1}}}  \NDsym{,}  \Psi_{{\mathrm{1}}}  \NDsym{,}  \Phi_{{\mathrm{2}}}  \NDsym{,}  \Psi_{{\mathrm{2}}}  \vdash_\mathcal{C}   \mathsf{let}\, \NDnt{t_{{\mathrm{2}}}}  :  \NDnt{X}  \otimes  \NDnt{Y} \,\mathsf{be}\, \NDmv{x}  \otimes  \NDmv{y} \,\mathsf{in}\, \NDsym{(}   \mathsf{let}\, \NDnt{t_{{\mathrm{1}}}}  :   \mathsf{UnitT}  \,\mathsf{be}\,  \mathsf{trivT}  \,\mathsf{in}\, \NDnt{t_{{\mathrm{3}}}}   \NDsym{)}   \NDsym{:}  \NDnt{Z}}
      \end{math}
    \end{center}

  \item (\NDdruleTXXtenEName, \NDdruleTXXtenEName):
    \begin{center}
      \tiny
      \begin{math}
        $$\mprset{flushleft}
        \inferrule* [right={\tiny tenE}] {
          $$\mprset{flushleft}
          \inferrule* [right={\tiny tenE}] {
            {\Phi_{{\mathrm{1}}}  \NDsym{,}  \NDmv{x}  \NDsym{:}  \NDnt{X_{{\mathrm{2}}}}  \NDsym{,}  \NDmv{y}  \NDsym{:}  \NDnt{Y_{{\mathrm{2}}}}  \NDsym{,}  \Phi_{{\mathrm{2}}}  \vdash_\mathcal{C}  \NDnt{t_{{\mathrm{1}}}}  \NDsym{:}  \NDnt{X_{{\mathrm{1}}}}  \otimes  \NDnt{Y_{{\mathrm{1}}}}} \\
            {\Psi  \vdash_\mathcal{C}  \NDnt{t_{{\mathrm{2}}}}  \NDsym{:}  \NDnt{X_{{\mathrm{2}}}}  \otimes  \NDnt{Y_{{\mathrm{2}}}}}
          }{\Phi_{{\mathrm{1}}}  \NDsym{,}  \Psi  \NDsym{,}  \Phi_{{\mathrm{2}}}  \vdash_\mathcal{C}   \mathsf{let}\, \NDnt{t_{{\mathrm{2}}}}  :  \NDnt{X_{{\mathrm{2}}}}  \otimes  \NDnt{Y_{{\mathrm{2}}}} \,\mathsf{be}\, \NDmv{x}  \otimes  \NDmv{y} \,\mathsf{in}\, \NDnt{t_{{\mathrm{1}}}}   \NDsym{:}  \NDnt{X_{{\mathrm{1}}}}  \otimes  \NDnt{Y_{{\mathrm{1}}}}} \\
           {\Psi_{{\mathrm{1}}}  \NDsym{,}  \NDmv{w}  \NDsym{:}  \NDnt{X_{{\mathrm{1}}}}  \NDsym{,}  \NDmv{z}  \NDsym{:}  \NDnt{Y_{{\mathrm{1}}}}  \NDsym{,}  \Psi_{{\mathrm{2}}}  \vdash_\mathcal{C}  \NDnt{t_{{\mathrm{3}}}}  \NDsym{:}  \NDnt{Z}}
        }{\Psi_{{\mathrm{1}}}  \NDsym{,}  \Phi_{{\mathrm{1}}}  \NDsym{,}  \Psi  \NDsym{,}  \Phi_{{\mathrm{2}}}  \NDsym{,}  \Psi_{{\mathrm{2}}}  \vdash_\mathcal{C}   \mathsf{let}\, \NDsym{(}   \mathsf{let}\, \NDnt{t_{{\mathrm{2}}}}  :  \NDnt{X_{{\mathrm{2}}}}  \otimes  \NDnt{Y_{{\mathrm{2}}}} \,\mathsf{be}\, \NDmv{x}  \otimes  \NDmv{y} \,\mathsf{in}\, \NDnt{t_{{\mathrm{1}}}}   \NDsym{)}  :  \NDnt{X_{{\mathrm{1}}}}  \otimes  \NDnt{Y_{{\mathrm{1}}}} \,\mathsf{be}\, \NDmv{w}  \otimes  \NDmv{z} \,\mathsf{in}\, \NDnt{t_{{\mathrm{3}}}}   \NDsym{:}  \NDnt{Z}}
      \end{math}
    \end{center}
    commutes to
    \begin{center}
      \tiny
      \begin{math}
        $$\mprset{flushleft}
        \inferrule* [right={\tiny tenE}] {
          $$\mprset{flushleft}
          \inferrule* [right={\tiny tenE}] {
            {\Phi_{{\mathrm{1}}}  \NDsym{,}  \NDmv{x}  \NDsym{:}  \NDnt{X_{{\mathrm{2}}}}  \NDsym{,}  \NDmv{y}  \NDsym{:}  \NDnt{Y_{{\mathrm{2}}}}  \NDsym{,}  \Phi_{{\mathrm{2}}}  \vdash_\mathcal{C}  \NDnt{t_{{\mathrm{1}}}}  \NDsym{:}  \NDnt{X_{{\mathrm{1}}}}  \otimes  \NDnt{Y_{{\mathrm{1}}}}} \\
            {\Psi_{{\mathrm{1}}}  \NDsym{,}  \NDmv{w}  \NDsym{:}  \NDnt{X_{{\mathrm{1}}}}  \NDsym{,}  \NDmv{z}  \NDsym{:}  \NDnt{Y_{{\mathrm{1}}}}  \NDsym{,}  \Psi_{{\mathrm{2}}}  \vdash_\mathcal{C}  \NDnt{t_{{\mathrm{3}}}}  \NDsym{:}  \NDnt{Z}}
          }{\Psi_{{\mathrm{1}}}  \NDsym{,}  \Phi_{{\mathrm{1}}}  \NDsym{,}  \NDmv{x}  \NDsym{:}  \NDnt{X_{{\mathrm{2}}}}  \NDsym{,}  \NDmv{y}  \NDsym{:}  \NDnt{Y_{{\mathrm{2}}}}  \NDsym{,}  \Phi_{{\mathrm{2}}}  \NDsym{,}  \Psi_{{\mathrm{2}}}  \vdash_\mathcal{C}   \mathsf{let}\, \NDnt{t_{{\mathrm{1}}}}  :  \NDnt{X_{{\mathrm{1}}}}  \otimes  \NDnt{Y_{{\mathrm{1}}}} \,\mathsf{be}\, \NDmv{w}  \otimes  \NDmv{z} \,\mathsf{in}\, \NDnt{t_{{\mathrm{3}}}}   \NDsym{:}  \NDnt{Z}} \\
           {\Psi  \vdash_\mathcal{C}  \NDnt{t_{{\mathrm{2}}}}  \NDsym{:}  \NDnt{X_{{\mathrm{2}}}}  \otimes  \NDnt{Y_{{\mathrm{2}}}}}
        }{\Psi_{{\mathrm{1}}}  \NDsym{,}  \Phi_{{\mathrm{1}}}  \NDsym{,}  \Psi  \NDsym{,}  \Phi_{{\mathrm{2}}}  \NDsym{,}  \Psi_{{\mathrm{2}}}  \vdash_\mathcal{C}   \mathsf{let}\, \NDnt{t_{{\mathrm{2}}}}  :  \NDnt{X_{{\mathrm{2}}}}  \otimes  \NDnt{Y_{{\mathrm{2}}}} \,\mathsf{be}\, \NDmv{x}  \otimes  \NDmv{y} \,\mathsf{in}\, \NDsym{(}   \mathsf{let}\, \NDnt{t_{{\mathrm{1}}}}  :  \NDnt{X_{{\mathrm{1}}}}  \otimes  \NDnt{Y_{{\mathrm{1}}}} \,\mathsf{be}\, \NDmv{w}  \otimes  \NDmv{z} \,\mathsf{in}\, \NDnt{t_{{\mathrm{3}}}}   \NDsym{)}   \NDsym{:}  \NDnt{Z}}
      \end{math}
    \end{center}

  \item (\NDdruleTXXtenEName, \NDdruleTXXimpEName):
    \begin{center}
      \tiny
      \begin{math}
        $$\mprset{flushleft}
        \inferrule* [right={\tiny tenE}] {
          $$\mprset{flushleft}
          \inferrule* [right={\tiny impE}] {
            {\Phi_{{\mathrm{1}}}  \NDsym{,}  \NDmv{x}  \NDsym{:}  \NDnt{X_{{\mathrm{2}}}}  \NDsym{,}  \NDmv{y}  \NDsym{:}  \NDnt{Y_{{\mathrm{2}}}}  \NDsym{,}  \Phi_{{\mathrm{2}}}  \vdash_\mathcal{C}  \NDnt{t_{{\mathrm{1}}}}  \NDsym{:}  \NDnt{X_{{\mathrm{1}}}}  \multimap  \NDnt{Y_{{\mathrm{1}}}}} \\
            {\Psi_{{\mathrm{1}}}  \vdash_\mathcal{C}  \NDnt{t_{{\mathrm{2}}}}  \NDsym{:}  \NDnt{X_{{\mathrm{2}}}}  \otimes  \NDnt{Y_{{\mathrm{2}}}}}
          }{\Phi_{{\mathrm{1}}}  \NDsym{,}  \Psi_{{\mathrm{1}}}  \NDsym{,}  \Phi_{{\mathrm{2}}}  \vdash_\mathcal{C}   \mathsf{let}\, \NDnt{t_{{\mathrm{2}}}}  :  \NDnt{X_{{\mathrm{2}}}}  \otimes  \NDnt{Y_{{\mathrm{2}}}} \,\mathsf{be}\, \NDmv{x}  \otimes  \NDmv{y} \,\mathsf{in}\, \NDnt{t_{{\mathrm{1}}}}   \NDsym{:}  \NDnt{X_{{\mathrm{1}}}}  \multimap  \NDnt{Y_{{\mathrm{1}}}}} \\
           {\Psi_{{\mathrm{2}}}  \vdash_\mathcal{C}  \NDnt{t_{{\mathrm{3}}}}  \NDsym{:}  \NDnt{X_{{\mathrm{1}}}}}
        }{\Phi_{{\mathrm{1}}}  \NDsym{,}  \Psi_{{\mathrm{1}}}  \NDsym{,}  \Phi_{{\mathrm{2}}}  \NDsym{,}  \Psi_{{\mathrm{2}}}  \vdash_\mathcal{C}   \mathsf{app}\, \NDsym{(}   \mathsf{let}\, \NDnt{t_{{\mathrm{2}}}}  :  \NDnt{X_{{\mathrm{2}}}}  \otimes  \NDnt{Y_{{\mathrm{2}}}} \,\mathsf{be}\, \NDmv{x}  \otimes  \NDmv{y} \,\mathsf{in}\, \NDnt{t_{{\mathrm{1}}}}   \NDsym{)} \, \NDnt{t_{{\mathrm{3}}}}   \NDsym{:}  \NDnt{Y_{{\mathrm{1}}}}}
      \end{math}
    \end{center}
    commutes to
    \begin{center}
      \tiny
      \begin{math}
        $$\mprset{flushleft}
        \inferrule* [right={\tiny impE}] {
          $$\mprset{flushleft}
          \inferrule* [right={\tiny tenE}] {
            {\Phi_{{\mathrm{1}}}  \NDsym{,}  \NDmv{x}  \NDsym{:}  \NDnt{X_{{\mathrm{2}}}}  \NDsym{,}  \NDmv{y}  \NDsym{:}  \NDnt{Y_{{\mathrm{2}}}}  \NDsym{,}  \Phi_{{\mathrm{2}}}  \vdash_\mathcal{C}  \NDnt{t_{{\mathrm{1}}}}  \NDsym{:}  \NDnt{X_{{\mathrm{1}}}}  \multimap  \NDnt{Y_{{\mathrm{1}}}}} \\
            {\Psi_{{\mathrm{2}}}  \vdash_\mathcal{C}  \NDnt{t_{{\mathrm{3}}}}  \NDsym{:}  \NDnt{X_{{\mathrm{1}}}}}
          }{\Phi_{{\mathrm{1}}}  \NDsym{,}  \NDmv{x}  \NDsym{:}  \NDnt{X_{{\mathrm{2}}}}  \NDsym{,}  \NDmv{y}  \NDsym{:}  \NDnt{Y_{{\mathrm{2}}}}  \NDsym{,}  \Phi_{{\mathrm{2}}}  \NDsym{,}  \Psi_{{\mathrm{2}}}  \vdash_\mathcal{C}   \mathsf{app}\, \NDnt{t_{{\mathrm{1}}}} \, \NDnt{t_{{\mathrm{3}}}}   \NDsym{:}  \NDnt{Y_{{\mathrm{1}}}}} \\
           {\Psi_{{\mathrm{1}}}  \vdash_\mathcal{C}  \NDnt{t_{{\mathrm{2}}}}  \NDsym{:}  \NDnt{X_{{\mathrm{2}}}}  \otimes  \NDnt{Y_{{\mathrm{2}}}}}
        }{\Phi_{{\mathrm{1}}}  \NDsym{,}  \Psi_{{\mathrm{1}}}  \NDsym{,}  \Phi_{{\mathrm{2}}}  \NDsym{,}  \Psi_{{\mathrm{2}}}  \vdash_\mathcal{C}   \mathsf{let}\, \NDnt{t_{{\mathrm{2}}}}  :  \NDnt{X_{{\mathrm{2}}}}  \otimes  \NDnt{Y_{{\mathrm{2}}}} \,\mathsf{be}\, \NDmv{x}  \otimes  \NDmv{y} \,\mathsf{in}\, \NDsym{(}   \mathsf{app}\, \NDnt{t_{{\mathrm{1}}}} \, \NDnt{t_{{\mathrm{3}}}}   \NDsym{)}   \NDsym{:}  \NDnt{Y_{{\mathrm{1}}}}}
      \end{math}
    \end{center}

  \end{itemize}

\item Commutation of $\multimap_E$:

  \begin{itemize}

  \item (\NDdruleTXXimpEName, \NDdruleTXXunitEName):
    \begin{center}
      \tiny
      \begin{math}
        $$\mprset{flushleft}
        \inferrule* [right={\tiny tenE}] {
          $$\mprset{flushleft}
          \inferrule* [right={\tiny impE}] {
            {\Phi_{{\mathrm{1}}}  \vdash_\mathcal{C}  \NDnt{t_{{\mathrm{1}}}}  \NDsym{:}   \mathsf{UnitT} } \\
            {\Phi_{{\mathrm{2}}}  \vdash_\mathcal{C}  \NDnt{t_{{\mathrm{2}}}}  \NDsym{:}   \mathsf{UnitT}   \multimap   \mathsf{UnitT} }
          }{\Phi_{{\mathrm{2}}}  \NDsym{,}  \Phi_{{\mathrm{1}}}  \vdash_\mathcal{C}   \mathsf{app}\, \NDnt{t_{{\mathrm{2}}}} \, \NDnt{t_{{\mathrm{1}}}}   \NDsym{:}   \mathsf{UnitT} } \\
           {\Phi_{{\mathrm{3}}}  \vdash_\mathcal{C}  \NDnt{t_{{\mathrm{3}}}}  \NDsym{:}   \mathsf{UnitT} }
        }{\Phi_{{\mathrm{2}}}  \NDsym{,}  \Phi_{{\mathrm{1}}}  \NDsym{,}  \Phi_{{\mathrm{3}}}  \vdash_\mathcal{C}   \mathsf{let}\, \NDsym{(}   \mathsf{app}\, \NDnt{t_{{\mathrm{2}}}} \, \NDnt{t_{{\mathrm{1}}}}   \NDsym{)}  :   \mathsf{UnitT}  \,\mathsf{be}\,  \mathsf{trivT}  \,\mathsf{in}\, \NDnt{t_{{\mathrm{3}}}}   \NDsym{:}   \mathsf{UnitT} }
      \end{math}
    \end{center}
    commutes to
    \begin{center}
      \tiny
      \begin{math}
        $$\mprset{flushleft}
        \inferrule* [right={\tiny impE}] {
          $$\mprset{flushleft}
          \inferrule* [right={\tiny tenE}] {
            {\Phi_{{\mathrm{1}}}  \vdash_\mathcal{C}  \NDnt{t_{{\mathrm{1}}}}  \NDsym{:}   \mathsf{UnitT} } \\
            {\Phi_{{\mathrm{3}}}  \vdash_\mathcal{C}  \NDnt{t_{{\mathrm{3}}}}  \NDsym{:}   \mathsf{UnitT} }
          }{\Phi_{{\mathrm{1}}}  \NDsym{,}  \Phi_{{\mathrm{3}}}  \vdash_\mathcal{C}   \mathsf{let}\, \NDnt{t_{{\mathrm{1}}}}  :   \mathsf{UnitT}  \,\mathsf{be}\,  \mathsf{trivT}  \,\mathsf{in}\, \NDnt{t_{{\mathrm{3}}}}   \NDsym{:}   \mathsf{UnitT} }
           {\Phi_{{\mathrm{2}}}  \vdash_\mathcal{C}  \NDnt{t_{{\mathrm{2}}}}  \NDsym{:}   \mathsf{UnitT}   \multimap   \mathsf{UnitT} }
        }{\Phi_{{\mathrm{2}}}  \NDsym{,}  \Phi_{{\mathrm{1}}}  \NDsym{,}  \Phi_{{\mathrm{3}}}  \vdash_\mathcal{C}   \mathsf{app}\, \NDnt{t_{{\mathrm{2}}}} \, \NDsym{(}   \mathsf{let}\, \NDnt{t_{{\mathrm{1}}}}  :   \mathsf{UnitT}  \,\mathsf{be}\,  \mathsf{trivT}  \,\mathsf{in}\, \NDnt{t_{{\mathrm{3}}}}   \NDsym{)}   \NDsym{:}   \mathsf{UnitT} }
      \end{math}
    \end{center}
  \item (\NDdruleTXXimpEName, \NDdruleTXXtenEName): ?
  \item (\NDdruleTXXimpEName, \NDdruleTXXimpEName): ?
  \end{itemize}

\item Commutation of $\tri_E$:

  \begin{itemize}

  \item (\NDdruleSXXunitETwoName, \NDdruleSXXunitETwoName):
    \begin{center}
      \tiny
      \begin{math}
        $$\mprset{flushleft}
        \inferrule* [right={\tiny unitE}] {
          $$\mprset{flushleft}
          \inferrule* [right={\tiny unitE}] {
            {\Gamma_{{\mathrm{1}}}  \vdash_\mathcal{L}  \NDnt{s_{{\mathrm{1}}}}  \NDsym{:}   \mathsf{UnitS} } \\
            {\Gamma_{{\mathrm{2}}}  \vdash_\mathcal{L}  \NDnt{s_{{\mathrm{2}}}}  \NDsym{:}   \mathsf{UnitS} }
          }{\Gamma_{{\mathrm{2}}}  \NDsym{,}  \Gamma_{{\mathrm{1}}}  \vdash_\mathcal{L}   \mathsf{let}\, \NDnt{s_{{\mathrm{2}}}}  :   \mathsf{UnitS}  \,\mathsf{be}\,  \mathsf{trivS}  \,\mathsf{in}\, \NDnt{s_{{\mathrm{1}}}}   \NDsym{:}   \mathsf{UnitS} } \\
           {\Gamma_{{\mathrm{3}}}  \vdash_\mathcal{L}  \NDnt{s_{{\mathrm{3}}}}  \NDsym{:}  \NDnt{A}}
        }{\Gamma_{{\mathrm{2}}}  \NDsym{,}  \Gamma_{{\mathrm{1}}}  \NDsym{,}  \Gamma_{{\mathrm{3}}}  \vdash_\mathcal{L}   \mathsf{let}\, \NDsym{(}   \mathsf{let}\, \NDnt{s_{{\mathrm{2}}}}  :   \mathsf{UnitS}  \,\mathsf{be}\,  \mathsf{trivS}  \,\mathsf{in}\, \NDnt{s_{{\mathrm{1}}}}   \NDsym{)}  :   \mathsf{UnitS}  \,\mathsf{be}\,  \mathsf{trivS}  \,\mathsf{in}\, \NDnt{s_{{\mathrm{3}}}}   \NDsym{:}  \NDnt{A}}
      \end{math}
    \end{center}
    commutes to
    \begin{center}
      \tiny
      \begin{math}
        $$\mprset{flushleft}
        \inferrule* [right={\tiny unitE}] {
          $$\mprset{flushleft}
          \inferrule* [right={\tiny unitE}] {
            {\Gamma_{{\mathrm{1}}}  \vdash_\mathcal{L}  \NDnt{s_{{\mathrm{1}}}}  \NDsym{:}   \mathsf{UnitS} } \\
            {\Gamma_{{\mathrm{3}}}  \vdash_\mathcal{L}  \NDnt{s_{{\mathrm{3}}}}  \NDsym{:}  \NDnt{A}}
          }{\Gamma_{{\mathrm{1}}}  \NDsym{,}  \Gamma_{{\mathrm{3}}}  \vdash_\mathcal{L}   \mathsf{let}\, \NDnt{s_{{\mathrm{1}}}}  :   \mathsf{UnitS}  \,\mathsf{be}\,  \mathsf{trivS}  \,\mathsf{in}\, \NDnt{s_{{\mathrm{3}}}}   \NDsym{:}  \NDnt{A}} \\
           {\Gamma_{{\mathrm{2}}}  \vdash_\mathcal{L}  \NDnt{s_{{\mathrm{2}}}}  \NDsym{:}   \mathsf{UnitS} }
        }{\Gamma_{{\mathrm{2}}}  \NDsym{,}  \Gamma_{{\mathrm{1}}}  \NDsym{,}  \Gamma_{{\mathrm{3}}}  \vdash_\mathcal{L}   \mathsf{let}\, \NDnt{s_{{\mathrm{2}}}}  :   \mathsf{UnitS}  \,\mathsf{be}\,  \mathsf{trivS}  \,\mathsf{in}\, \NDsym{(}   \mathsf{let}\, \NDnt{s_{{\mathrm{1}}}}  :   \mathsf{UnitS}  \,\mathsf{be}\,  \mathsf{trivS}  \,\mathsf{in}\, \NDnt{s_{{\mathrm{3}}}}   \NDsym{)}   \NDsym{:}  \NDnt{A}}
      \end{math}
    \end{center}

  \item (\NDdruleSXXunitETwoName, \NDdruleSXXtenETwoName): Does NOT commute
    \begin{center}
      \tiny
      \begin{math}
        $$\mprset{flushleft}
        \inferrule* [right={\tiny tenE2}] {
          $$\mprset{flushleft}
          \inferrule* [right={\tiny unitE}] {
            {\Gamma_{{\mathrm{1}}}  \vdash_\mathcal{L}  \NDnt{s_{{\mathrm{1}}}}  \NDsym{:}  \NDnt{A}  \triangleright  \NDnt{B}} \\
            {\Gamma_{{\mathrm{2}}}  \vdash_\mathcal{L}  \NDnt{s_{{\mathrm{2}}}}  \NDsym{:}   \mathsf{UnitS} }
          }{\Gamma_{{\mathrm{2}}}  \NDsym{,}  \Gamma_{{\mathrm{1}}}  \vdash_\mathcal{L}   \mathsf{let}\, \NDnt{s_{{\mathrm{2}}}}  :   \mathsf{UnitS}  \,\mathsf{be}\,  \mathsf{trivS}  \,\mathsf{in}\, \NDnt{s_{{\mathrm{1}}}}   \NDsym{:}  \NDnt{A}  \triangleright  \NDnt{B}} \\
           {\Delta_{{\mathrm{1}}}  \NDsym{,}  \NDmv{x}  \NDsym{:}  \NDnt{A}  \NDsym{,}  \NDmv{y}  \NDsym{:}  \NDnt{B}  \NDsym{,}  \Delta_{{\mathrm{2}}}  \vdash_\mathcal{L}  \NDnt{s_{{\mathrm{3}}}}  \NDsym{:}  \NDnt{C}}
        }{\Delta_{{\mathrm{1}}}  \NDsym{,}  \Gamma_{{\mathrm{2}}}  \NDsym{,}  \Gamma_{{\mathrm{1}}}  \NDsym{,}  \Delta_{{\mathrm{2}}}  \vdash_\mathcal{L}   \mathsf{let}\, \NDsym{(}   \mathsf{let}\, \NDnt{s_{{\mathrm{2}}}}  :   \mathsf{UnitS}  \,\mathsf{be}\,  \mathsf{trivS}  \,\mathsf{in}\, \NDnt{s_{{\mathrm{1}}}}   \NDsym{)}  :  \NDnt{A}  \triangleright  \NDnt{B} \,\mathsf{be}\, \NDmv{x}  \triangleright  \NDmv{y} \,\mathsf{in}\, \NDnt{s_{{\mathrm{3}}}}   \NDsym{:}  \NDnt{C}}
      \end{math}
    \end{center}
    commutes to
    \begin{center}
      \tiny
      \begin{math}
        $$\mprset{flushleft}
        \inferrule* [right={\tiny unitE}] {
          $$\mprset{flushleft}
          \inferrule* [right={\tiny tenE2}] {
            {\Gamma_{{\mathrm{1}}}  \vdash_\mathcal{L}  \NDnt{s_{{\mathrm{1}}}}  \NDsym{:}  \NDnt{A}  \triangleright  \NDnt{B}} \\
            {\Delta_{{\mathrm{1}}}  \NDsym{,}  \NDmv{x}  \NDsym{:}  \NDnt{A}  \NDsym{,}  \NDmv{y}  \NDsym{:}  \NDnt{B}  \NDsym{,}  \Delta_{{\mathrm{2}}}  \vdash_\mathcal{L}  \NDnt{s_{{\mathrm{3}}}}  \NDsym{:}  \NDnt{C}}
          }{\Delta_{{\mathrm{1}}}  \NDsym{,}  \Gamma_{{\mathrm{1}}}  \NDsym{,}  \Delta_{{\mathrm{2}}}  \vdash_\mathcal{L}   \mathsf{let}\, \NDnt{s_{{\mathrm{1}}}}  :  \NDnt{A}  \triangleright  \NDnt{B} \,\mathsf{be}\, \NDmv{x}  \triangleright  \NDmv{y} \,\mathsf{in}\, \NDnt{s_{{\mathrm{3}}}}   \NDsym{:}  \NDnt{C}} \\
           {\Gamma_{{\mathrm{2}}}  \vdash_\mathcal{L}  \NDnt{s_{{\mathrm{2}}}}  \NDsym{:}   \mathsf{UnitS} }
        }{\Gamma_{{\mathrm{2}}}  \NDsym{,}  \Delta_{{\mathrm{1}}}  \NDsym{,}  \Gamma_{{\mathrm{1}}}  \NDsym{,}  \Delta_{{\mathrm{2}}}  \vdash_\mathcal{L}   \mathsf{let}\, \NDnt{s_{{\mathrm{2}}}}  :   \mathsf{UnitS}  \,\mathsf{be}\,  \mathsf{trivS}  \,\mathsf{in}\, \NDsym{(}   \mathsf{let}\, \NDnt{s_{{\mathrm{1}}}}  :  \NDnt{A}  \triangleright  \NDnt{B} \,\mathsf{be}\, \NDmv{x}  \triangleright  \NDmv{y} \,\mathsf{in}\, \NDnt{s_{{\mathrm{3}}}}   \NDsym{)}   \NDsym{:}  \NDnt{C}}
      \end{math}
    \end{center}

  \item (\NDdruleSXXunitETwoName, \NDdruleSXXimprEName):
    \begin{center}
      \tiny
      \begin{math}
        $$\mprset{flushleft}
        \inferrule* [right={\tiny imprE}] {
          $$\mprset{flushleft}
          \inferrule* [right={\tiny unitE}] {
            {\Gamma_{{\mathrm{1}}}  \vdash_\mathcal{L}  \NDnt{s_{{\mathrm{1}}}}  \NDsym{:}  \NDnt{A}  \rightharpoonup  \NDnt{B}} \\
            {\Gamma_{{\mathrm{2}}}  \vdash_\mathcal{L}  \NDnt{s_{{\mathrm{2}}}}  \NDsym{:}   \mathsf{UnitS} }
          }{\Gamma_{{\mathrm{2}}}  \NDsym{,}  \Gamma_{{\mathrm{1}}}  \vdash_\mathcal{L}   \mathsf{let}\, \NDnt{s_{{\mathrm{2}}}}  :   \mathsf{UnitS}  \,\mathsf{be}\,  \mathsf{trivS}  \,\mathsf{in}\, \NDnt{s_{{\mathrm{1}}}}   \NDsym{:}  \NDnt{A}  \rightharpoonup  \NDnt{B}} \\
           {\Gamma_{{\mathrm{3}}}  \vdash_\mathcal{L}  \NDnt{s_{{\mathrm{3}}}}  \NDsym{:}  \NDnt{A}}
        }{\Gamma_{{\mathrm{2}}}  \NDsym{,}  \Gamma_{{\mathrm{1}}}  \NDsym{,}  \Gamma_{{\mathrm{3}}}  \vdash_\mathcal{L}   \mathsf{app}_r\, \NDsym{(}   \mathsf{let}\, \NDnt{s_{{\mathrm{2}}}}  :   \mathsf{UnitS}  \,\mathsf{be}\,  \mathsf{trivS}  \,\mathsf{in}\, \NDnt{s_{{\mathrm{1}}}}   \NDsym{)} \, \NDnt{s_{{\mathrm{3}}}}   \NDsym{:}  \NDnt{B}}
      \end{math}
    \end{center}
    commutes to
    \begin{center}
      \tiny
      \begin{math}
        $$\mprset{flushleft}
        \inferrule* [right={\tiny unitE}] {
          $$\mprset{flushleft}
          \inferrule* [right={\tiny imprE}] {
            {\Gamma_{{\mathrm{1}}}  \vdash_\mathcal{L}  \NDnt{s_{{\mathrm{1}}}}  \NDsym{:}  \NDnt{A}  \rightharpoonup  \NDnt{B}} \\
            {\Gamma_{{\mathrm{3}}}  \vdash_\mathcal{L}  \NDnt{s_{{\mathrm{3}}}}  \NDsym{:}  \NDnt{A}}
          }{\Gamma_{{\mathrm{1}}}  \NDsym{,}  \Gamma_{{\mathrm{3}}}  \vdash_\mathcal{L}   \mathsf{app}_r\, \NDnt{s_{{\mathrm{1}}}} \, \NDnt{s_{{\mathrm{3}}}}   \NDsym{:}  \NDnt{B}} \\
           {\Gamma_{{\mathrm{2}}}  \vdash_\mathcal{L}  \NDnt{s_{{\mathrm{2}}}}  \NDsym{:}   \mathsf{UnitS} }
        }{\Gamma_{{\mathrm{2}}}  \NDsym{,}  \Gamma_{{\mathrm{1}}}  \NDsym{,}  \Gamma_{{\mathrm{3}}}  \vdash_\mathcal{L}   \mathsf{let}\, \NDnt{s_{{\mathrm{2}}}}  :   \mathsf{UnitS}  \,\mathsf{be}\,  \mathsf{trivS}  \,\mathsf{in}\, \NDsym{(}   \mathsf{app}_r\, \NDnt{s_{{\mathrm{1}}}} \, \NDnt{s_{{\mathrm{3}}}}   \NDsym{)}   \NDsym{:}  \NDnt{B}}
      \end{math}
    \end{center}

  \item (\NDdruleSXXunitETwoName, \NDdruleSXXimplEName): Does NOT commute.
    \begin{center}
      \tiny
      \begin{math}
        $$\mprset{flushleft}
        \inferrule* [right={\tiny imprE}] {
          $$\mprset{flushleft}
          \inferrule* [right={\tiny unitE}] {
            {\Gamma_{{\mathrm{1}}}  \vdash_\mathcal{L}  \NDnt{s_{{\mathrm{1}}}}  \NDsym{:}  \NDnt{B}  \leftharpoonup  \NDnt{A}} \\
            {\Gamma_{{\mathrm{2}}}  \vdash_\mathcal{L}  \NDnt{s_{{\mathrm{2}}}}  \NDsym{:}   \mathsf{UnitS} }
          }{\Gamma_{{\mathrm{2}}}  \NDsym{,}  \Gamma_{{\mathrm{1}}}  \vdash_\mathcal{L}   \mathsf{let}\, \NDnt{s_{{\mathrm{2}}}}  :   \mathsf{UnitS}  \,\mathsf{be}\,  \mathsf{trivS}  \,\mathsf{in}\, \NDnt{s_{{\mathrm{1}}}}   \NDsym{:}  \NDnt{B}  \leftharpoonup  \NDnt{A}} \\
           {\Gamma_{{\mathrm{3}}}  \vdash_\mathcal{L}  \NDnt{s_{{\mathrm{3}}}}  \NDsym{:}  \NDnt{A}}
        }{\Gamma_{{\mathrm{3}}}  \NDsym{,}  \Gamma_{{\mathrm{2}}}  \NDsym{,}  \Gamma_{{\mathrm{1}}}  \vdash_\mathcal{L}   \mathsf{app}_l\, \NDsym{(}   \mathsf{let}\, \NDnt{s_{{\mathrm{2}}}}  :   \mathsf{UnitS}  \,\mathsf{be}\,  \mathsf{trivS}  \,\mathsf{in}\, \NDnt{s_{{\mathrm{1}}}}   \NDsym{)} \, \NDnt{s_{{\mathrm{3}}}}   \NDsym{:}  \NDnt{B}}
      \end{math}
    \end{center}
    commutes to
    \begin{center}
      \tiny
      \begin{math}
        $$\mprset{flushleft}
        \inferrule* [right={\tiny unitE}] {
          $$\mprset{flushleft}
          \inferrule* [right={\tiny imprE}] {
            {\Gamma_{{\mathrm{1}}}  \vdash_\mathcal{L}  \NDnt{s_{{\mathrm{1}}}}  \NDsym{:}  \NDnt{B}  \leftharpoonup  \NDnt{A}} \\
            {\Gamma_{{\mathrm{3}}}  \vdash_\mathcal{L}  \NDnt{s_{{\mathrm{3}}}}  \NDsym{:}  \NDnt{A}}
          }{\Gamma_{{\mathrm{3}}}  \NDsym{,}  \Gamma_{{\mathrm{1}}}  \vdash_\mathcal{L}   \mathsf{app}_l\, \NDnt{s_{{\mathrm{1}}}} \, \NDnt{s_{{\mathrm{3}}}}   \NDsym{:}  \NDnt{B}} \\
           {\Gamma_{{\mathrm{2}}}  \vdash_\mathcal{L}  \NDnt{s_{{\mathrm{2}}}}  \NDsym{:}   \mathsf{UnitS} }
        }{\Gamma_{{\mathrm{2}}}  \NDsym{,}  \Gamma_{{\mathrm{3}}}  \NDsym{,}  \Gamma_{{\mathrm{1}}}  \vdash_\mathcal{L}   \mathsf{let}\, \NDnt{s_{{\mathrm{2}}}}  :   \mathsf{UnitS}  \,\mathsf{be}\,  \mathsf{trivS}  \,\mathsf{in}\, \NDsym{(}   \mathsf{app}_l\, \NDnt{s_{{\mathrm{1}}}} \, \NDnt{s_{{\mathrm{3}}}}   \NDsym{)}   \NDsym{:}  \NDnt{B}}
      \end{math}
    \end{center}

  \item (\NDdruleSXXtenETwoName, \NDdruleSXXunitETwoName):
    \begin{center}
      \tiny
      \begin{math}
        $$\mprset{flushleft}
        \inferrule* [right={\tiny unitE2}] {
          $$\mprset{flushleft}
          \inferrule* [right={\tiny tenE2}] {
            {\Gamma_{{\mathrm{1}}}  \NDsym{,}  \NDmv{x}  \NDsym{:}  \NDnt{A}  \NDsym{,}  \NDmv{y}  \NDsym{:}  \NDnt{B}  \NDsym{,}  \Gamma_{{\mathrm{2}}}  \vdash_\mathcal{L}  \NDnt{s_{{\mathrm{1}}}}  \NDsym{:}   \mathsf{UnitS} } \\
            {\Delta_{{\mathrm{1}}}  \vdash_\mathcal{L}  \NDnt{s_{{\mathrm{2}}}}  \NDsym{:}  \NDnt{A}  \triangleright  \NDnt{B}}
          }{\Gamma_{{\mathrm{1}}}  \NDsym{,}  \Delta_{{\mathrm{1}}}  \NDsym{,}  \Gamma_{{\mathrm{2}}}  \vdash_\mathcal{L}   \mathsf{let}\, \NDnt{s_{{\mathrm{2}}}}  :  \NDnt{A}  \triangleright  \NDnt{B} \,\mathsf{be}\, \NDmv{x}  \triangleright  \NDmv{y} \,\mathsf{in}\, \NDnt{s_{{\mathrm{1}}}}   \NDsym{:}   \mathsf{UnitS} } \\
           {\Delta_{{\mathrm{2}}}  \vdash_\mathcal{L}  \NDnt{s_{{\mathrm{3}}}}  \NDsym{:}  \NDnt{C}}
        }{\Gamma_{{\mathrm{1}}}  \NDsym{,}  \Delta_{{\mathrm{1}}}  \NDsym{,}  \Gamma_{{\mathrm{2}}}  \NDsym{,}  \Delta_{{\mathrm{2}}}  \vdash_\mathcal{L}   \mathsf{let}\, \NDsym{(}   \mathsf{let}\, \NDnt{s_{{\mathrm{2}}}}  :  \NDnt{A}  \triangleright  \NDnt{B} \,\mathsf{be}\, \NDmv{x}  \triangleright  \NDmv{y} \,\mathsf{in}\, \NDnt{s_{{\mathrm{1}}}}   \NDsym{)}  :   \mathsf{UnitS}  \,\mathsf{be}\,  \mathsf{trivS}  \,\mathsf{in}\, \NDnt{s_{{\mathrm{3}}}}   \NDsym{:}  \NDnt{C}}
      \end{math}
    \end{center}
    commutes to
    \begin{center}
      \tiny
      \begin{math}
        $$\mprset{flushleft}
        \inferrule* [right={\tiny tenE2}] {
          $$\mprset{flushleft}
          \inferrule* [right={\tiny unitE2}] {
            {\Gamma_{{\mathrm{1}}}  \NDsym{,}  \NDmv{x}  \NDsym{:}  \NDnt{A}  \NDsym{,}  \NDmv{y}  \NDsym{:}  \NDnt{B}  \NDsym{,}  \Gamma_{{\mathrm{2}}}  \vdash_\mathcal{L}  \NDnt{s_{{\mathrm{1}}}}  \NDsym{:}   \mathsf{UnitS} } \\
            {\Delta_{{\mathrm{2}}}  \vdash_\mathcal{L}  \NDnt{s_{{\mathrm{3}}}}  \NDsym{:}  \NDnt{C}}
          }{\Gamma_{{\mathrm{1}}}  \NDsym{,}  \NDmv{x}  \NDsym{:}  \NDnt{A}  \NDsym{,}  \NDmv{y}  \NDsym{:}  \NDnt{B}  \NDsym{,}  \Gamma_{{\mathrm{2}}}  \NDsym{,}  \Delta_{{\mathrm{2}}}  \vdash_\mathcal{L}   \mathsf{let}\, \NDnt{s_{{\mathrm{1}}}}  :   \mathsf{UnitS}  \,\mathsf{be}\,  \mathsf{trivS}  \,\mathsf{in}\, \NDnt{s_{{\mathrm{3}}}}   \NDsym{:}  \NDnt{C}} \\
           {\Delta_{{\mathrm{1}}}  \vdash_\mathcal{L}  \NDnt{s_{{\mathrm{2}}}}  \NDsym{:}  \NDnt{A}  \triangleright  \NDnt{B}}
        }{\Gamma_{{\mathrm{1}}}  \NDsym{,}  \Delta_{{\mathrm{1}}}  \NDsym{,}  \Gamma_{{\mathrm{2}}}  \NDsym{,}  \Delta_{{\mathrm{2}}}  \vdash_\mathcal{L}   \mathsf{let}\, \NDnt{s_{{\mathrm{2}}}}  :  \NDnt{A}  \triangleright  \NDnt{B} \,\mathsf{be}\, \NDmv{x}  \triangleright  \NDmv{y} \,\mathsf{in}\, \NDsym{(}   \mathsf{let}\, \NDnt{s_{{\mathrm{1}}}}  :   \mathsf{UnitS}  \,\mathsf{be}\,  \mathsf{trivS}  \,\mathsf{in}\, \NDnt{s_{{\mathrm{3}}}}   \NDsym{)}   \NDsym{:}  \NDnt{C}}
      \end{math}
    \end{center}

  \item (\NDdruleSXXtenETwoName, \NDdruleSXXtenETwoName):
    \begin{center}
      \tiny
      \begin{math}
        $$\mprset{flushleft}
        \inferrule* [right={\tiny tenE2}] {
          $$\mprset{flushleft}
          \inferrule* [right={\tiny tenE2}] {
            {\Gamma_{{\mathrm{1}}}  \NDsym{,}  \NDmv{x}  \NDsym{:}  \NDnt{A_{{\mathrm{2}}}}  \NDsym{,}  \NDmv{y}  \NDsym{:}  \NDnt{B_{{\mathrm{2}}}}  \NDsym{,}  \Gamma_{{\mathrm{2}}}  \vdash_\mathcal{L}  \NDnt{s_{{\mathrm{1}}}}  \NDsym{:}  \NDnt{A_{{\mathrm{1}}}}  \triangleright  \NDnt{B_{{\mathrm{1}}}}} \\
            {\Gamma  \vdash_\mathcal{L}  \NDnt{s_{{\mathrm{2}}}}  \NDsym{:}  \NDnt{A_{{\mathrm{2}}}}  \triangleright  \NDnt{B_{{\mathrm{2}}}}}
          }{\Gamma_{{\mathrm{1}}}  \NDsym{,}  \Gamma  \NDsym{,}  \Gamma_{{\mathrm{2}}}  \vdash_\mathcal{L}   \mathsf{let}\, \NDnt{s_{{\mathrm{2}}}}  :  \NDnt{A_{{\mathrm{2}}}}  \triangleright  \NDnt{B_{{\mathrm{2}}}} \,\mathsf{be}\, \NDmv{x}  \triangleright  \NDmv{y} \,\mathsf{in}\, \NDnt{s_{{\mathrm{1}}}}   \NDsym{:}  \NDnt{A_{{\mathrm{1}}}}  \triangleright  \NDnt{B_{{\mathrm{1}}}}} \\
           {\Delta_{{\mathrm{1}}}  \NDsym{,}  \NDmv{w}  \NDsym{:}  \NDnt{A_{{\mathrm{1}}}}  \NDsym{,}  \NDmv{z}  \NDsym{:}  \NDnt{B_{{\mathrm{1}}}}  \NDsym{,}  \Delta_{{\mathrm{2}}}  \vdash_\mathcal{L}  \NDnt{s_{{\mathrm{3}}}}  \NDsym{:}  \NDnt{C}}
        }{\Delta_{{\mathrm{1}}}  \NDsym{,}  \Gamma_{{\mathrm{1}}}  \NDsym{,}  \Gamma  \NDsym{,}  \Gamma_{{\mathrm{2}}}  \NDsym{,}  \Delta_{{\mathrm{2}}}  \vdash_\mathcal{L}   \mathsf{let}\, \NDsym{(}   \mathsf{let}\, \NDnt{s_{{\mathrm{2}}}}  :  \NDnt{A_{{\mathrm{2}}}}  \triangleright  \NDnt{B_{{\mathrm{2}}}} \,\mathsf{be}\, \NDmv{x}  \triangleright  \NDmv{y} \,\mathsf{in}\, \NDnt{s_{{\mathrm{1}}}}   \NDsym{)}  :  \NDnt{A_{{\mathrm{1}}}}  \triangleright  \NDnt{B_{{\mathrm{1}}}} \,\mathsf{be}\, \NDmv{w}  \triangleright  \NDmv{z} \,\mathsf{in}\, \NDnt{s_{{\mathrm{3}}}}   \NDsym{:}  \NDnt{C}}
      \end{math}
    \end{center}
    commutes to
    \begin{center}
      \tiny
      \begin{math}
        $$\mprset{flushleft}
        \inferrule* [right={\tiny tenE2}] {
          $$\mprset{flushleft}
          \inferrule* [right={\tiny tenE2}] {
            {\Gamma_{{\mathrm{1}}}  \NDsym{,}  \NDmv{x}  \NDsym{:}  \NDnt{A_{{\mathrm{2}}}}  \NDsym{,}  \NDmv{y}  \NDsym{:}  \NDnt{B_{{\mathrm{2}}}}  \NDsym{,}  \Gamma_{{\mathrm{2}}}  \vdash_\mathcal{L}  \NDnt{s_{{\mathrm{1}}}}  \NDsym{:}  \NDnt{A_{{\mathrm{1}}}}  \triangleright  \NDnt{B_{{\mathrm{1}}}}} \\
            {\Delta_{{\mathrm{1}}}  \NDsym{,}  \NDmv{w}  \NDsym{:}  \NDnt{A_{{\mathrm{1}}}}  \NDsym{,}  \NDmv{z}  \NDsym{:}  \NDnt{B_{{\mathrm{1}}}}  \NDsym{,}  \Delta_{{\mathrm{2}}}  \vdash_\mathcal{L}  \NDnt{s_{{\mathrm{3}}}}  \NDsym{:}  \NDnt{C}}
          }{\Delta_{{\mathrm{1}}}  \NDsym{,}  \Gamma_{{\mathrm{1}}}  \NDsym{,}  \NDmv{x}  \NDsym{:}  \NDnt{A_{{\mathrm{2}}}}  \NDsym{,}  \NDmv{y}  \NDsym{:}  \NDnt{B_{{\mathrm{2}}}}  \NDsym{,}  \Gamma_{{\mathrm{2}}}  \NDsym{,}  \Delta_{{\mathrm{2}}}  \vdash_\mathcal{L}   \mathsf{let}\, \NDnt{s_{{\mathrm{1}}}}  :  \NDnt{A_{{\mathrm{1}}}}  \triangleright  \NDnt{B_{{\mathrm{1}}}} \,\mathsf{be}\, \NDmv{w}  \triangleright  \NDmv{z} \,\mathsf{in}\, \NDnt{s_{{\mathrm{3}}}}   \NDsym{:}  \NDnt{C}}
            {\Gamma  \vdash_\mathcal{L}  \NDnt{s_{{\mathrm{2}}}}  \NDsym{:}  \NDnt{A_{{\mathrm{2}}}}  \triangleright  \NDnt{B_{{\mathrm{2}}}}}
        }{\Delta_{{\mathrm{1}}}  \NDsym{,}  \Gamma_{{\mathrm{1}}}  \NDsym{,}  \Gamma  \NDsym{,}  \Gamma_{{\mathrm{2}}}  \NDsym{,}  \Delta_{{\mathrm{2}}}  \vdash_\mathcal{L}   \mathsf{let}\, \NDnt{s_{{\mathrm{2}}}}  :  \NDnt{A_{{\mathrm{2}}}}  \triangleright  \NDnt{B_{{\mathrm{2}}}} \,\mathsf{be}\, \NDmv{x}  \triangleright  \NDmv{y} \,\mathsf{in}\, \NDsym{(}   \mathsf{let}\, \NDnt{s_{{\mathrm{1}}}}  :  \NDnt{A_{{\mathrm{1}}}}  \triangleright  \NDnt{B_{{\mathrm{1}}}} \,\mathsf{be}\, \NDmv{x}  \triangleright  \NDmv{z} \,\mathsf{in}\, \NDnt{s_{{\mathrm{3}}}}   \NDsym{)}   \NDsym{:}  \NDnt{C}}
      \end{math}
    \end{center}

  \item (\NDdruleSXXtenETwoName, \NDdruleSXXimprEName):
    \begin{center}
      \tiny
      \begin{math}
        $$\mprset{flushleft}
        \inferrule* [right={\tiny imprE}] {
          $$\mprset{flushleft}
          \inferrule* [right={\tiny tenE2}] {
            {\Gamma_{{\mathrm{1}}}  \NDsym{,}  \NDmv{x}  \NDsym{:}  \NDnt{A_{{\mathrm{2}}}}  \NDsym{,}  \NDmv{y}  \NDsym{:}  \NDnt{B_{{\mathrm{2}}}}  \NDsym{,}  \Gamma_{{\mathrm{2}}}  \vdash_\mathcal{L}  \NDnt{s_{{\mathrm{1}}}}  \NDsym{:}  \NDnt{A_{{\mathrm{1}}}}  \rightharpoonup  \NDnt{B_{{\mathrm{1}}}}} \\
            {\Delta_{{\mathrm{1}}}  \vdash_\mathcal{L}  \NDnt{s_{{\mathrm{2}}}}  \NDsym{:}  \NDnt{A_{{\mathrm{2}}}}  \triangleright  \NDnt{B_{{\mathrm{2}}}}}
          }{\Gamma_{{\mathrm{1}}}  \NDsym{,}  \Delta_{{\mathrm{1}}}  \NDsym{,}  \Gamma_{{\mathrm{2}}}  \vdash_\mathcal{L}   \mathsf{let}\, \NDnt{s_{{\mathrm{2}}}}  :  \NDnt{A_{{\mathrm{2}}}}  \triangleright  \NDnt{B_{{\mathrm{2}}}} \,\mathsf{be}\, \NDmv{x}  \triangleright  \NDmv{y} \,\mathsf{in}\, \NDnt{s_{{\mathrm{1}}}}   \NDsym{:}  \NDnt{A_{{\mathrm{1}}}}  \rightharpoonup  \NDnt{B_{{\mathrm{1}}}}} \\
           {\Delta_{{\mathrm{2}}}  \vdash_\mathcal{L}  \NDnt{s_{{\mathrm{3}}}}  \NDsym{:}  \NDnt{A_{{\mathrm{1}}}}}
        }{\Gamma_{{\mathrm{1}}}  \NDsym{,}  \Delta_{{\mathrm{1}}}  \NDsym{,}  \Gamma_{{\mathrm{2}}}  \NDsym{,}  \Delta_{{\mathrm{2}}}  \vdash_\mathcal{L}   \mathsf{app}_r\, \NDsym{(}   \mathsf{let}\, \NDnt{s_{{\mathrm{2}}}}  :  \NDnt{A_{{\mathrm{2}}}}  \triangleright  \NDnt{B_{{\mathrm{2}}}} \,\mathsf{be}\, \NDmv{x}  \triangleright  \NDmv{y} \,\mathsf{in}\, \NDnt{s_{{\mathrm{1}}}}   \NDsym{)} \, \NDnt{s_{{\mathrm{3}}}}   \NDsym{:}  \NDnt{B_{{\mathrm{1}}}}}
      \end{math}
    \end{center}
    commutes to
    \begin{center}
      \tiny
      \begin{math}
        $$\mprset{flushleft}
        \inferrule* [right={\tiny tenE2}] {
          $$\mprset{flushleft}
          \inferrule* [right={\tiny imprE}] {
            {\Gamma_{{\mathrm{1}}}  \NDsym{,}  \NDmv{x}  \NDsym{:}  \NDnt{A_{{\mathrm{2}}}}  \NDsym{,}  \NDmv{y}  \NDsym{:}  \NDnt{B_{{\mathrm{2}}}}  \NDsym{,}  \Gamma_{{\mathrm{2}}}  \vdash_\mathcal{L}  \NDnt{s_{{\mathrm{1}}}}  \NDsym{:}  \NDnt{A_{{\mathrm{1}}}}  \rightharpoonup  \NDnt{B_{{\mathrm{1}}}}} \\
            {\Delta_{{\mathrm{2}}}  \vdash_\mathcal{L}  \NDnt{s_{{\mathrm{3}}}}  \NDsym{:}  \NDnt{A_{{\mathrm{1}}}}}
          }{\Gamma_{{\mathrm{1}}}  \NDsym{,}  \NDmv{x}  \NDsym{:}  \NDnt{A_{{\mathrm{2}}}}  \NDsym{,}  \NDmv{y}  \NDsym{:}  \NDnt{B_{{\mathrm{2}}}}  \NDsym{,}  \Gamma_{{\mathrm{2}}}  \NDsym{,}  \Delta_{{\mathrm{2}}}  \vdash_\mathcal{L}   \mathsf{app}_r\, \NDnt{s_{{\mathrm{1}}}} \, \NDnt{s_{{\mathrm{3}}}}   \NDsym{:}  \NDnt{B_{{\mathrm{1}}}}} \\
            {\Delta_{{\mathrm{1}}}  \vdash_\mathcal{L}  \NDnt{s_{{\mathrm{2}}}}  \NDsym{:}  \NDnt{A_{{\mathrm{2}}}}  \triangleright  \NDnt{B_{{\mathrm{2}}}}}
        }{\Gamma_{{\mathrm{1}}}  \NDsym{,}  \Delta_{{\mathrm{1}}}  \NDsym{,}  \Gamma_{{\mathrm{2}}}  \NDsym{,}  \Delta_{{\mathrm{2}}}  \vdash_\mathcal{L}   \mathsf{let}\, \NDnt{s_{{\mathrm{2}}}}  :  \NDnt{A_{{\mathrm{2}}}}  \triangleright  \NDnt{B_{{\mathrm{2}}}} \,\mathsf{be}\, \NDmv{x}  \triangleright  \NDmv{y} \,\mathsf{in}\, \NDsym{(}   \mathsf{app}_r\, \NDnt{s_{{\mathrm{1}}}} \, \NDnt{s_{{\mathrm{3}}}}   \NDsym{)}   \NDsym{:}  \NDnt{B_{{\mathrm{1}}}}}
      \end{math}
    \end{center}

  \item (\NDdruleSXXtenETwoName, \NDdruleSXXimplEName):
    \begin{center}
      \tiny
      \begin{math}
        $$\mprset{flushleft}
        \inferrule* [right={\tiny imprE}] {
          $$\mprset{flushleft}
          \inferrule* [right={\tiny tenE2}] {
            {\Gamma_{{\mathrm{1}}}  \NDsym{,}  \NDmv{x}  \NDsym{:}  \NDnt{A_{{\mathrm{2}}}}  \NDsym{,}  \NDmv{y}  \NDsym{:}  \NDnt{B_{{\mathrm{2}}}}  \NDsym{,}  \Gamma_{{\mathrm{2}}}  \vdash_\mathcal{L}  \NDnt{s_{{\mathrm{1}}}}  \NDsym{:}  \NDnt{B_{{\mathrm{1}}}}  \leftharpoonup  \NDnt{A_{{\mathrm{1}}}}} \\
            {\Delta_{{\mathrm{1}}}  \vdash_\mathcal{L}  \NDnt{s_{{\mathrm{2}}}}  \NDsym{:}  \NDnt{A_{{\mathrm{2}}}}  \triangleright  \NDnt{B_{{\mathrm{2}}}}}
          }{\Gamma_{{\mathrm{1}}}  \NDsym{,}  \Delta_{{\mathrm{1}}}  \NDsym{,}  \Gamma_{{\mathrm{2}}}  \vdash_\mathcal{L}   \mathsf{let}\, \NDnt{s_{{\mathrm{2}}}}  :  \NDnt{A_{{\mathrm{2}}}}  \triangleright  \NDnt{B_{{\mathrm{2}}}} \,\mathsf{be}\, \NDmv{x}  \triangleright  \NDmv{y} \,\mathsf{in}\, \NDnt{s_{{\mathrm{1}}}}   \NDsym{:}  \NDnt{B_{{\mathrm{1}}}}  \leftharpoonup  \NDnt{A_{{\mathrm{1}}}}} \\
           {\Delta_{{\mathrm{2}}}  \vdash_\mathcal{L}  \NDnt{s_{{\mathrm{3}}}}  \NDsym{:}  \NDnt{A_{{\mathrm{1}}}}}
        }{\Delta_{{\mathrm{2}}}  \NDsym{,}  \Gamma_{{\mathrm{1}}}  \NDsym{,}  \Delta_{{\mathrm{1}}}  \NDsym{,}  \Gamma_{{\mathrm{2}}}  \vdash_\mathcal{L}   \mathsf{app}_l\, \NDsym{(}   \mathsf{let}\, \NDnt{s_{{\mathrm{2}}}}  :  \NDnt{A_{{\mathrm{2}}}}  \triangleright  \NDnt{B_{{\mathrm{2}}}} \,\mathsf{be}\, \NDmv{x}  \triangleright  \NDmv{y} \,\mathsf{in}\, \NDnt{s_{{\mathrm{1}}}}   \NDsym{)} \, \NDnt{s_{{\mathrm{3}}}}   \NDsym{:}  \NDnt{B_{{\mathrm{1}}}}}
      \end{math}
    \end{center}
    commutes to
    \begin{center}
      \tiny
      \begin{math}
        $$\mprset{flushleft}
        \inferrule* [right={\tiny tenE2}] {
          $$\mprset{flushleft}
          \inferrule* [right={\tiny imprE}] {
            {\Gamma_{{\mathrm{1}}}  \NDsym{,}  \NDmv{x}  \NDsym{:}  \NDnt{A_{{\mathrm{2}}}}  \NDsym{,}  \NDmv{y}  \NDsym{:}  \NDnt{B_{{\mathrm{2}}}}  \NDsym{,}  \Gamma_{{\mathrm{2}}}  \vdash_\mathcal{L}  \NDnt{s_{{\mathrm{1}}}}  \NDsym{:}  \NDnt{B_{{\mathrm{1}}}}  \leftharpoonup  \NDnt{A_{{\mathrm{1}}}}} \\
            {\Delta_{{\mathrm{2}}}  \vdash_\mathcal{L}  \NDnt{s_{{\mathrm{3}}}}  \NDsym{:}  \NDnt{A_{{\mathrm{1}}}}}
          }{\Delta_{{\mathrm{2}}}  \NDsym{,}  \Gamma_{{\mathrm{1}}}  \NDsym{,}  \NDmv{x}  \NDsym{:}  \NDnt{A_{{\mathrm{2}}}}  \NDsym{,}  \NDmv{y}  \NDsym{:}  \NDnt{B_{{\mathrm{2}}}}  \NDsym{,}  \Gamma_{{\mathrm{2}}}  \vdash_\mathcal{L}   \mathsf{app}_l\, \NDnt{s_{{\mathrm{1}}}} \, \NDnt{s_{{\mathrm{3}}}}   \NDsym{:}  \NDnt{B_{{\mathrm{1}}}}} \\
            {\Delta_{{\mathrm{1}}}  \vdash_\mathcal{L}  \NDnt{s_{{\mathrm{2}}}}  \NDsym{:}  \NDnt{A_{{\mathrm{2}}}}  \triangleright  \NDnt{B_{{\mathrm{2}}}}}
        }{\Delta_{{\mathrm{2}}}  \NDsym{,}  \Gamma_{{\mathrm{1}}}  \NDsym{,}  \Delta_{{\mathrm{1}}}  \NDsym{,}  \Gamma_{{\mathrm{2}}}  \vdash_\mathcal{L}   \mathsf{let}\, \NDnt{s_{{\mathrm{2}}}}  :  \NDnt{A_{{\mathrm{2}}}}  \triangleright  \NDnt{B_{{\mathrm{2}}}} \,\mathsf{be}\, \NDmv{x}  \triangleright  \NDmv{y} \,\mathsf{in}\, \NDsym{(}   \mathsf{app}_l\, \NDnt{s_{{\mathrm{1}}}} \, \NDnt{s_{{\mathrm{3}}}}   \NDsym{)}   \NDsym{:}  \NDnt{B_{{\mathrm{1}}}}}
      \end{math}
    \end{center}
  
  \end{itemize}

\item Commutation of $F_E$:
  \begin{itemize}
  \item (\NDdruleSXXFEName, \NDdruleSXXunitETwoName):
    \begin{center}
      \tiny
      \begin{math}
        $$\mprset{flushleft}
        \inferrule* [right={\tiny unitE2}] {
          $$\mprset{flushleft}
          \inferrule* [right={\tiny FE}] {
            {\Gamma_{{\mathrm{1}}}  \NDsym{,}  \NDmv{x}  \NDsym{:}  \NDnt{X}  \NDsym{,}  \Gamma_{{\mathrm{2}}}  \vdash_\mathcal{L}  \NDnt{s_{{\mathrm{1}}}}  \NDsym{:}   \mathsf{UnitS} } \\
            {\Delta_{{\mathrm{1}}}  \vdash_\mathcal{L}  \NDmv{y}  \NDsym{:}   \mathsf{F} \NDnt{X} }
          }{\Gamma_{{\mathrm{1}}}  \NDsym{,}  \Delta_{{\mathrm{1}}}  \NDsym{,}  \Gamma_{{\mathrm{2}}}  \vdash_\mathcal{L}   \mathsf{let}\,  \mathsf{F} \NDmv{x}   :   \mathsf{F} \NDnt{X}  \,\mathsf{be}\, \NDmv{y} \,\mathsf{in}\, \NDnt{s_{{\mathrm{1}}}}   \NDsym{:}   \mathsf{UnitS} } \\
           {\Delta_{{\mathrm{2}}}  \vdash_\mathcal{L}  \NDnt{s_{{\mathrm{2}}}}  \NDsym{:}  \NDnt{A}}
        }{\Gamma_{{\mathrm{1}}}  \NDsym{,}  \Delta_{{\mathrm{1}}}  \NDsym{,}  \Gamma_{{\mathrm{2}}}  \NDsym{,}  \Delta_{{\mathrm{2}}}  \vdash_\mathcal{L}   \mathsf{let}\, \NDsym{(}   \mathsf{let}\,  \mathsf{F} \NDmv{x}   :   \mathsf{F} \NDnt{X}  \,\mathsf{be}\, \NDmv{y} \,\mathsf{in}\, \NDnt{s_{{\mathrm{1}}}}   \NDsym{)}  :   \mathsf{UnitS}  \,\mathsf{be}\,  \mathsf{trivS}  \,\mathsf{in}\, \NDnt{s_{{\mathrm{2}}}}   \NDsym{:}  \NDnt{A}}
      \end{math}
    \end{center}
    commutes to
    \begin{center}
      \tiny
      \begin{math}
        $$\mprset{flushleft}
        \inferrule* [right={\tiny FE}] {
          $$\mprset{flushleft}
          \inferrule* [right={\tiny unitE2}] {
            {\Gamma_{{\mathrm{1}}}  \NDsym{,}  \NDmv{x}  \NDsym{:}  \NDnt{X}  \NDsym{,}  \Gamma_{{\mathrm{2}}}  \vdash_\mathcal{L}  \NDnt{s_{{\mathrm{1}}}}  \NDsym{:}   \mathsf{UnitS} } \\
            {\Delta_{{\mathrm{2}}}  \vdash_\mathcal{L}  \NDnt{s_{{\mathrm{2}}}}  \NDsym{:}  \NDnt{A}}
          }{\Gamma_{{\mathrm{1}}}  \NDsym{,}  \NDmv{x}  \NDsym{:}  \NDnt{X}  \NDsym{,}  \Gamma_{{\mathrm{2}}}  \NDsym{,}  \Delta_{{\mathrm{2}}}  \vdash_\mathcal{L}   \mathsf{let}\, \NDnt{s_{{\mathrm{1}}}}  :   \mathsf{UnitS}  \,\mathsf{be}\,  \mathsf{trivS}  \,\mathsf{in}\, \NDnt{s_{{\mathrm{2}}}}   \NDsym{:}  \NDnt{A}} \\
           {\Delta_{{\mathrm{1}}}  \vdash_\mathcal{L}  \NDmv{y}  \NDsym{:}   \mathsf{F} \NDnt{X} }
        }{\Gamma_{{\mathrm{1}}}  \NDsym{,}  \Delta_{{\mathrm{1}}}  \NDsym{,}  \Gamma_{{\mathrm{2}}}  \NDsym{,}  \Delta_{{\mathrm{2}}}  \vdash_\mathcal{L}   \mathsf{let}\,  \mathsf{F} \NDmv{x}   :   \mathsf{F} \NDnt{X}  \,\mathsf{be}\, \NDmv{y} \,\mathsf{in}\, \NDsym{(}   \mathsf{let}\, \NDnt{s_{{\mathrm{1}}}}  :   \mathsf{UnitS}  \,\mathsf{be}\,  \mathsf{trivS}  \,\mathsf{in}\, \NDnt{s_{{\mathrm{2}}}}   \NDsym{)}   \NDsym{:}  \NDnt{A}}
      \end{math}
    \end{center}
  \item (\NDdruleSXXFEName, \NDdruleSXXtenETwoName):
    \begin{center}
      \tiny
      \begin{math}
        $$\mprset{flushleft}
        \inferrule* [right={\tiny tenE2}] {
          $$\mprset{flushleft}
          \inferrule* [right={\tiny FE}] {
            {\Gamma_{{\mathrm{1}}}  \NDsym{,}  \NDmv{x}  \NDsym{:}  \NDnt{X}  \NDsym{,}  \Gamma_{{\mathrm{2}}}  \vdash_\mathcal{L}  \NDnt{s_{{\mathrm{1}}}}  \NDsym{:}  \NDnt{A}  \triangleright  \NDnt{B}} \\
            {\Delta  \vdash_\mathcal{L}  \NDmv{y}  \NDsym{:}   \mathsf{F} \NDnt{X} }
          }{\Gamma_{{\mathrm{1}}}  \NDsym{,}  \Delta  \NDsym{,}  \Gamma_{{\mathrm{2}}}  \vdash_\mathcal{L}   \mathsf{let}\,  \mathsf{F} \NDmv{x}   :   \mathsf{F} \NDnt{X}  \,\mathsf{be}\, \NDmv{y} \,\mathsf{in}\, \NDnt{s_{{\mathrm{1}}}}   \NDsym{:}  \NDnt{A}  \triangleright  \NDnt{B}} \\
           {\Delta_{{\mathrm{1}}}  \NDsym{,}  \NDmv{x}  \NDsym{:}  \NDnt{A}  \NDsym{,}  \NDmv{y}  \NDsym{:}  \NDnt{B}  \NDsym{,}  \Delta_{{\mathrm{2}}}  \vdash_\mathcal{L}  \NDnt{s_{{\mathrm{2}}}}  \NDsym{:}  \NDnt{C}}
        }{\Delta_{{\mathrm{1}}}  \NDsym{,}  \Gamma_{{\mathrm{1}}}  \NDsym{,}  \Delta  \NDsym{,}  \Gamma_{{\mathrm{2}}}  \NDsym{,}  \Delta_{{\mathrm{2}}}  \vdash_\mathcal{L}   \mathsf{let}\, \NDsym{(}   \mathsf{let}\,  \mathsf{F} \NDmv{x}   :   \mathsf{F} \NDnt{X}  \,\mathsf{be}\, \NDmv{y} \,\mathsf{in}\, \NDnt{s_{{\mathrm{1}}}}   \NDsym{)}  :  \NDnt{A}  \triangleright  \NDnt{B} \,\mathsf{be}\, \NDmv{x}  \triangleright  \NDmv{y} \,\mathsf{in}\, \NDnt{s_{{\mathrm{2}}}}   \NDsym{:}  \NDnt{C}}
      \end{math}
    \end{center}
    commutes to
    \begin{center}
      \tiny
      \begin{math}
        $$\mprset{flushleft}
        \inferrule* [right={\tiny FE}] {
          $$\mprset{flushleft}
          \inferrule* [right={\tiny tenE2}] {
            {\Gamma_{{\mathrm{1}}}  \NDsym{,}  \NDmv{x}  \NDsym{:}  \NDnt{X}  \NDsym{,}  \Gamma_{{\mathrm{2}}}  \vdash_\mathcal{L}  \NDnt{s_{{\mathrm{1}}}}  \NDsym{:}  \NDnt{A}  \triangleright  \NDnt{B}} \\
            {\Delta_{{\mathrm{1}}}  \NDsym{,}  \NDmv{x}  \NDsym{:}  \NDnt{A}  \NDsym{,}  \NDmv{y}  \NDsym{:}  \NDnt{B}  \NDsym{,}  \Delta_{{\mathrm{2}}}  \vdash_\mathcal{L}  \NDnt{s_{{\mathrm{2}}}}  \NDsym{:}  \NDnt{C}}
          }{\Delta_{{\mathrm{1}}}  \NDsym{,}  \Gamma_{{\mathrm{1}}}  \NDsym{,}  \NDmv{x}  \NDsym{:}  \NDnt{X}  \NDsym{,}  \Gamma_{{\mathrm{2}}}  \NDsym{,}  \Delta_{{\mathrm{2}}}  \vdash_\mathcal{L}   \mathsf{let}\, \NDnt{s_{{\mathrm{1}}}}  :  \NDnt{A}  \triangleright  \NDnt{B} \,\mathsf{be}\, \NDmv{x}  \triangleright  \NDmv{y} \,\mathsf{in}\, \NDnt{s_{{\mathrm{2}}}}   \NDsym{:}  \NDnt{C}} \\
           {\Delta  \vdash_\mathcal{L}  \NDmv{y}  \NDsym{:}   \mathsf{F} \NDnt{X} }
        }{\Delta_{{\mathrm{1}}}  \NDsym{,}  \Gamma_{{\mathrm{1}}}  \NDsym{,}  \Delta  \NDsym{,}  \Gamma_{{\mathrm{2}}}  \NDsym{,}  \Delta_{{\mathrm{2}}}  \vdash_\mathcal{L}   \mathsf{let}\,  \mathsf{F} \NDmv{x}   :   \mathsf{F} \NDnt{X}  \,\mathsf{be}\, \NDmv{y} \,\mathsf{in}\, \NDsym{(}   \mathsf{let}\, \NDnt{s_{{\mathrm{1}}}}  :  \NDnt{A}  \triangleright  \NDnt{B} \,\mathsf{be}\, \NDmv{x}  \triangleright  \NDmv{y} \,\mathsf{in}\, \NDnt{s_{{\mathrm{2}}}}   \NDsym{)}   \NDsym{:}  \NDnt{C}}
      \end{math}
    \end{center}
  \item (\NDdruleSXXFEName, \NDdruleSXXimprEName):
    \begin{center}
      \tiny
      \begin{math}
        $$\mprset{flushleft}
        \inferrule* [right={\tiny imprE}] {
          $$\mprset{flushleft}
          \inferrule* [right={\tiny FE}] {
            {\Gamma_{{\mathrm{1}}}  \NDsym{,}  \NDmv{x}  \NDsym{:}  \NDnt{X}  \NDsym{,}  \Gamma_{{\mathrm{2}}}  \vdash_\mathcal{L}  \NDnt{s_{{\mathrm{1}}}}  \NDsym{:}  \NDnt{A}  \rightharpoonup  \NDnt{B}} \\
            {\Delta_{{\mathrm{1}}}  \vdash_\mathcal{L}  \NDmv{y}  \NDsym{:}   \mathsf{F} \NDnt{X} }
          }{\Gamma_{{\mathrm{1}}}  \NDsym{,}  \Delta_{{\mathrm{1}}}  \NDsym{,}  \Gamma_{{\mathrm{2}}}  \vdash_\mathcal{L}   \mathsf{let}\,  \mathsf{F} \NDmv{x}   :   \mathsf{F} \NDnt{X}  \,\mathsf{be}\, \NDmv{y} \,\mathsf{in}\, \NDnt{s_{{\mathrm{1}}}}   \NDsym{:}  \NDnt{A}  \rightharpoonup  \NDnt{B}} \\
           {\Delta_{{\mathrm{2}}}  \vdash_\mathcal{L}  \NDnt{s_{{\mathrm{2}}}}  \NDsym{:}  \NDnt{A}}
        }{\Gamma_{{\mathrm{1}}}  \NDsym{,}  \Delta_{{\mathrm{1}}}  \NDsym{,}  \Gamma_{{\mathrm{2}}}  \NDsym{,}  \Delta_{{\mathrm{2}}}  \vdash_\mathcal{L}   \mathsf{app}_r\, \NDsym{(}   \mathsf{let}\,  \mathsf{F} \NDmv{x}   :   \mathsf{F} \NDnt{X}  \,\mathsf{be}\, \NDmv{y} \,\mathsf{in}\, \NDnt{s_{{\mathrm{1}}}}   \NDsym{)} \, \NDnt{s_{{\mathrm{2}}}}   \NDsym{:}  \NDnt{B}}
      \end{math}
    \end{center}
    commutes to
    \begin{center}
      \tiny
      \begin{math}
        $$\mprset{flushleft}
        \inferrule* [right={\tiny FE}] {
          $$\mprset{flushleft}
          \inferrule* [right={\tiny imprE}] {
            {\Gamma_{{\mathrm{1}}}  \NDsym{,}  \NDmv{x}  \NDsym{:}  \NDnt{X}  \NDsym{,}  \Gamma_{{\mathrm{2}}}  \vdash_\mathcal{L}  \NDnt{s_{{\mathrm{1}}}}  \NDsym{:}  \NDnt{A}  \rightharpoonup  \NDnt{B}} \\
            {\Delta_{{\mathrm{2}}}  \vdash_\mathcal{L}  \NDnt{s_{{\mathrm{2}}}}  \NDsym{:}  \NDnt{A}}
          }{\Gamma_{{\mathrm{1}}}  \NDsym{,}  \NDmv{x}  \NDsym{:}  \NDnt{X}  \NDsym{,}  \Gamma_{{\mathrm{2}}}  \NDsym{,}  \Delta_{{\mathrm{2}}}  \vdash_\mathcal{L}   \mathsf{app}_r\, \NDnt{s_{{\mathrm{1}}}} \, \NDnt{s_{{\mathrm{2}}}}   \NDsym{:}  \NDnt{B}} \\
           {\Delta_{{\mathrm{1}}}  \vdash_\mathcal{L}  \NDmv{y}  \NDsym{:}   \mathsf{F} \NDnt{X} }
        }{\Gamma_{{\mathrm{1}}}  \NDsym{,}  \Delta_{{\mathrm{1}}}  \NDsym{,}  \Gamma_{{\mathrm{2}}}  \NDsym{,}  \Delta_{{\mathrm{2}}}  \vdash_\mathcal{L}   \mathsf{let}\,  \mathsf{F} \NDmv{x}   :   \mathsf{F} \NDnt{X}  \,\mathsf{be}\, \NDmv{y} \,\mathsf{in}\, \NDsym{(}   \mathsf{app}_r\, \NDnt{s_{{\mathrm{1}}}} \, \NDnt{s_{{\mathrm{2}}}}   \NDsym{)}   \NDsym{:}  \NDnt{B}}
      \end{math}
    \end{center}
  \item (\NDdruleSXXFEName, \NDdruleSXXimplEName):
    \begin{center}
      \tiny
      \begin{math}
        $$\mprset{flushleft}
        \inferrule* [right={\tiny implE}] {
          $$\mprset{flushleft}
          \inferrule* [right={\tiny FE}] {
            {\Gamma_{{\mathrm{1}}}  \NDsym{,}  \NDmv{x}  \NDsym{:}  \NDnt{X}  \NDsym{,}  \Gamma_{{\mathrm{2}}}  \vdash_\mathcal{L}  \NDnt{s_{{\mathrm{1}}}}  \NDsym{:}  \NDnt{A}  \leftharpoonup  \NDnt{B}} \\
            {\Delta_{{\mathrm{1}}}  \vdash_\mathcal{L}  \NDmv{y}  \NDsym{:}   \mathsf{F} \NDnt{X} }
          }{\Gamma_{{\mathrm{1}}}  \NDsym{,}  \Delta_{{\mathrm{1}}}  \NDsym{,}  \Gamma_{{\mathrm{2}}}  \vdash_\mathcal{L}   \mathsf{let}\,  \mathsf{F} \NDmv{x}   :   \mathsf{F} \NDnt{X}  \,\mathsf{be}\, \NDmv{y} \,\mathsf{in}\, \NDnt{s_{{\mathrm{1}}}}   \NDsym{:}  \NDnt{A}  \leftharpoonup  \NDnt{B}} \\
           {\Delta_{{\mathrm{2}}}  \vdash_\mathcal{L}  \NDnt{s_{{\mathrm{2}}}}  \NDsym{:}  \NDnt{A}}
        }{\Delta_{{\mathrm{2}}}  \NDsym{,}  \Gamma_{{\mathrm{1}}}  \NDsym{,}  \Delta_{{\mathrm{1}}}  \NDsym{,}  \Gamma_{{\mathrm{2}}}  \vdash_\mathcal{L}   \mathsf{app}_l\, \NDsym{(}   \mathsf{let}\,  \mathsf{F} \NDmv{x}   :   \mathsf{F} \NDnt{X}  \,\mathsf{be}\, \NDmv{y} \,\mathsf{in}\, \NDnt{s_{{\mathrm{1}}}}   \NDsym{)} \, \NDnt{s_{{\mathrm{2}}}}   \NDsym{:}  \NDnt{B}}
      \end{math}
    \end{center}
    commutes to
    \begin{center}
      \tiny
      \begin{math}
        $$\mprset{flushleft}
        \inferrule* [right={\tiny FE}] {
          $$\mprset{flushleft}
          \inferrule* [right={\tiny imprE}] {
            {\Gamma_{{\mathrm{1}}}  \NDsym{,}  \NDmv{x}  \NDsym{:}  \NDnt{X}  \NDsym{,}  \Gamma_{{\mathrm{2}}}  \vdash_\mathcal{L}  \NDnt{s_{{\mathrm{1}}}}  \NDsym{:}  \NDnt{A}  \leftharpoonup  \NDnt{B}} \\
            {\Delta_{{\mathrm{2}}}  \vdash_\mathcal{L}  \NDnt{s_{{\mathrm{2}}}}  \NDsym{:}  \NDnt{A}}
          }{\Delta_{{\mathrm{2}}}  \NDsym{,}  \Gamma_{{\mathrm{1}}}  \NDsym{,}  \NDmv{x}  \NDsym{:}  \NDnt{X}  \NDsym{,}  \Gamma_{{\mathrm{2}}}  \vdash_\mathcal{L}   \mathsf{app}_l\, \NDnt{s_{{\mathrm{1}}}} \, \NDnt{s_{{\mathrm{2}}}}   \NDsym{:}  \NDnt{B}} \\
           {\Delta_{{\mathrm{1}}}  \vdash_\mathcal{L}  \NDmv{y}  \NDsym{:}   \mathsf{F} \NDnt{X} }
        }{\Delta_{{\mathrm{2}}}  \NDsym{,}  \Gamma_{{\mathrm{1}}}  \NDsym{,}  \Delta_{{\mathrm{1}}}  \NDsym{,}  \Gamma_{{\mathrm{2}}}  \vdash_\mathcal{L}   \mathsf{let}\,  \mathsf{F} \NDmv{x}   :   \mathsf{F} \NDnt{X}  \,\mathsf{be}\, \NDmv{y} \,\mathsf{in}\, \NDsym{(}   \mathsf{app}_l\, \NDnt{s_{{\mathrm{1}}}} \, \NDnt{s_{{\mathrm{2}}}}   \NDsym{)}   \NDsym{:}  \NDnt{B}}
      \end{math}
    \end{center}

  \end{itemize}

\end{itemize}



%%%%%%%%%%%%%%%%%%%%%%%%%%%%%%%%%%%%%%%%%%%%%%%%%%
\subsection{Mappings Between Sequent Calculus and Natural Deduction}

Function $S:ND\rightarrow SE$ maps a proof in the natural deduction to a proof of the same
sequent in the sequent calculus. The function is defined as follows:

\begin{itemize}
\item The axioms map to axioms.
\item Introduction rules map to right rules.
\item Elimination rules map to combinations of left rules with cuts:
  \begin{itemize}
  \item \NDdruleTXXunitEName:
    \begin{center}
      \tiny
      $\NDdruleTXXunitE{}$
    \end{center}
    maps to
    \begin{center}
      \tiny
      \begin{math}
        $$\mprset{flushleft}
        \inferrule* [right={\tiny cut}] {
          {\Phi  \vdash_\mathcal{C}  \NDnt{t_{{\mathrm{1}}}}  \NDsym{:}   \mathsf{UnitT} } \\
          $$\mprset{flushleft}
          \inferrule* [right={\tiny unitL}] {
            {\Psi  \vdash_\mathcal{C}  \NDnt{t_{{\mathrm{2}}}}  \NDsym{:}  \NDnt{Y}}
          }{\NDmv{x}  \NDsym{:}   \mathsf{UnitT}   \NDsym{,}  \Psi  \vdash_\mathcal{C}   \mathsf{let}\, \NDmv{x}  :   \mathsf{UnitT}  \,\mathsf{be}\,  \mathsf{trivT}  \,\mathsf{in}\, \NDnt{t_{{\mathrm{2}}}}   \NDsym{:}  \NDnt{Y}}
        }{\Phi  \NDsym{,}  \Psi  \vdash_\mathcal{C}  \NDsym{[}  \NDnt{t_{{\mathrm{1}}}}  \NDsym{/}  \NDmv{x}  \NDsym{]}  \NDsym{(}   \mathsf{let}\, \NDmv{x}  :   \mathsf{UnitT}  \,\mathsf{be}\,  \mathsf{trivT}  \,\mathsf{in}\, \NDnt{t_{{\mathrm{2}}}}   \NDsym{)}  \NDsym{:}  \NDnt{Y}}
      \end{math}
    \end{center}
  \item \NDdruleTXXtenEName:
    \begin{center}
      \tiny
      $\NDdruleTXXtenE{}$
    \end{center}
    maps to
    \begin{center}
      \tiny
      \begin{math}
        $$\mprset{flushleft}
        \inferrule* [right={\tiny cut}] {
          {\Phi  \vdash_\mathcal{C}  \NDnt{t_{{\mathrm{1}}}}  \NDsym{:}  \NDnt{X}  \otimes  \NDnt{Y}} \\
          $$\mprset{flushleft}
          \inferrule* [right={\tiny unitL}] {
            {\Psi_{{\mathrm{1}}}  \NDsym{,}  \NDmv{x}  \NDsym{:}  \NDnt{X}  \NDsym{,}  \NDmv{y}  \NDsym{:}  \NDnt{Y}  \NDsym{,}  \Psi_{{\mathrm{2}}}  \vdash_\mathcal{C}  \NDnt{t_{{\mathrm{2}}}}  \NDsym{:}  \NDnt{Z}}
          }{\Psi_{{\mathrm{1}}}  \NDsym{,}  \NDmv{z}  \NDsym{:}  \NDnt{X}  \otimes  \NDnt{Y}  \NDsym{,}  \Psi_{{\mathrm{2}}}  \vdash_\mathcal{C}   \mathsf{let}\, \NDmv{z}  :  \NDnt{X}  \otimes  \NDnt{Y} \,\mathsf{be}\, \NDmv{x}  \otimes  \NDmv{y} \,\mathsf{in}\, \NDnt{t_{{\mathrm{2}}}}   \NDsym{:}  \NDnt{Z}}
        }{\Psi_{{\mathrm{1}}}  \NDsym{,}  \Phi  \NDsym{,}  \Psi_{{\mathrm{2}}}  \vdash_\mathcal{C}  \NDsym{[}  \NDnt{t_{{\mathrm{1}}}}  \NDsym{/}  \NDmv{z}  \NDsym{]}  \NDsym{(}   \mathsf{let}\, \NDmv{z}  :  \NDnt{X}  \otimes  \NDnt{Y} \,\mathsf{be}\, \NDmv{x}  \otimes  \NDmv{y} \,\mathsf{in}\, \NDnt{t_{{\mathrm{2}}}}   \NDsym{)}  \NDsym{:}  \NDnt{Z}}
      \end{math}
    \end{center}
  \item \NDdruleTXXimpEName:
    \begin{center}
      \tiny
      $\NDdruleTXXimpE{}$
    \end{center}
    maps to
    \begin{center}
      \tiny
      \begin{math}
        $$\mprset{flushleft}
        \inferrule* [right={\tiny cut}] {
          {\Phi  \vdash_\mathcal{C}  \NDnt{t_{{\mathrm{1}}}}  \NDsym{:}  \NDnt{X}  \multimap  \NDnt{Y}} \\
          $$\mprset{flushleft}
          \inferrule* [right={\tiny unitL}] {
            {\Psi  \vdash_\mathcal{C}  \NDnt{t_{{\mathrm{2}}}}  \NDsym{:}  \NDnt{X}} \\
            {\NDmv{x}  \NDsym{:}  \NDnt{Y}  \vdash_\mathcal{C}  \NDmv{x}  \NDsym{:}  \NDnt{Y}}
          }{\NDmv{y}  \NDsym{:}  \NDnt{X}  \multimap  \NDnt{Y}  \NDsym{,}  \Psi  \vdash_\mathcal{C}  \NDsym{[}   \mathsf{app}\, \NDmv{y} \, \NDnt{t_{{\mathrm{2}}}}   \NDsym{/}  \NDmv{x}  \NDsym{]}  \NDmv{x}  \NDsym{:}  \NDnt{Y}}
        }{\Phi  \NDsym{,}  \Psi  \vdash_\mathcal{C}  \NDsym{[}  \NDnt{t_{{\mathrm{1}}}}  \NDsym{/}  \NDmv{y}  \NDsym{]}  \NDsym{[}   \mathsf{app}\, \NDmv{y} \, \NDnt{t_{{\mathrm{2}}}}   \NDsym{/}  \NDmv{x}  \NDsym{]}  \NDmv{x}  \NDsym{:}  \NDnt{Y}}
      \end{math}
    \end{center}
  \item \NDdruleSXXunitEOneName:
    \begin{center}
      \tiny
      $\NDdruleSXXunitEOne{}$
    \end{center}
    maps to
    \begin{center}
      \tiny
      \begin{math}
        $$\mprset{flushleft}
        \inferrule* [right={\tiny cut1}] {
          {\Phi  \vdash_\mathcal{C}  \NDnt{t}  \NDsym{:}   \mathsf{UnitT} } \\
          $$\mprset{flushleft}
          \inferrule* [right={\tiny unitL1}] {
            {\Gamma  \vdash_\mathcal{L}  \NDnt{s}  \NDsym{:}  \NDnt{A}}
          }{\NDmv{x}  \NDsym{:}   \mathsf{UnitT}   \NDsym{,}  \Gamma  \vdash_\mathcal{L}   \mathsf{let}\, \NDmv{x}  :   \mathsf{UnitT}  \,\mathsf{be}\,  \mathsf{trivT}  \,\mathsf{in}\, \NDnt{s}   \NDsym{:}  \NDnt{A}}
        }{\Phi  \NDsym{,}  \Psi  \vdash_\mathcal{L}  \NDsym{[}  \NDnt{t}  \NDsym{/}  \NDmv{x}  \NDsym{]}  \NDsym{(}   \mathsf{let}\, \NDmv{x}  :   \mathsf{UnitT}  \,\mathsf{be}\,  \mathsf{trivT}  \,\mathsf{in}\, \NDnt{s}   \NDsym{)}  \NDsym{:}  \NDnt{A}}
      \end{math}
    \end{center}
  \item \NDdruleSXXunitETwoName:
    \begin{center}
      \tiny
      $\NDdruleSXXunitETwo{}$
    \end{center}
    maps to
    \begin{center}
      \tiny
      \begin{math}
        $$\mprset{flushleft}
        \inferrule* [right={\tiny cut2}] {
          {\Gamma  \vdash_\mathcal{L}  \NDnt{s_{{\mathrm{1}}}}  \NDsym{:}   \mathsf{UnitS} } \\
          $$\mprset{flushleft}
          \inferrule* [right={\tiny unitL2}] {
            {\Delta  \vdash_\mathcal{L}  \NDnt{s_{{\mathrm{2}}}}  \NDsym{:}  \NDnt{A}}
          }{\NDmv{x}  \NDsym{:}   \mathsf{UnitS}   \NDsym{,}  \Delta  \vdash_\mathcal{L}   \mathsf{let}\, \NDmv{x}  :   \mathsf{UnitS}  \,\mathsf{be}\,  \mathsf{trivS}  \,\mathsf{in}\, \NDnt{s_{{\mathrm{2}}}}   \NDsym{:}  \NDnt{A}}
        }{\Gamma  \NDsym{,}  \Delta  \vdash_\mathcal{L}  \NDsym{[}  \NDnt{s_{{\mathrm{1}}}}  \NDsym{/}  \NDmv{x}  \NDsym{]}  \NDsym{(}   \mathsf{let}\, \NDmv{x}  :   \mathsf{UnitS}  \,\mathsf{be}\,  \mathsf{trivS}  \,\mathsf{in}\, \NDnt{s_{{\mathrm{2}}}}   \NDsym{)}  \NDsym{:}  \NDnt{A}}
      \end{math}
    \end{center}
  \item \NDdruleSXXtenEOneName:
    \begin{center}
      \tiny
      $\NDdruleSXXtenEOne{}$
    \end{center}
    maps to
    \begin{center}
      \tiny
      \begin{math}
        $$\mprset{flushleft}
        \inferrule* [right={\tiny cut1}] {
          {\Phi  \vdash_\mathcal{C}  \NDnt{t}  \NDsym{:}  \NDnt{X}  \otimes  \NDnt{Y}} \\
          $$\mprset{flushleft}
          \inferrule* [right={\tiny tenL1}] {
            {\Gamma_{{\mathrm{1}}}  \NDsym{,}  \NDmv{x}  \NDsym{:}  \NDnt{X}  \NDsym{,}  \NDmv{y}  \NDsym{:}  \NDnt{Y}  \NDsym{,}  \Gamma_{{\mathrm{2}}}  \vdash_\mathcal{L}  \NDnt{s}  \NDsym{:}  \NDnt{A}}
          }{\Gamma_{{\mathrm{1}}}  \NDsym{,}  \NDmv{z}  \NDsym{:}  \NDnt{X}  \otimes  \NDnt{Y}  \NDsym{,}  \Gamma_{{\mathrm{2}}}  \vdash_\mathcal{L}   \mathsf{let}\, \NDmv{z}  :  \NDnt{X}  \otimes  \NDnt{Y} \,\mathsf{be}\, \NDmv{x}  \otimes  \NDmv{y} \,\mathsf{in}\, \NDnt{s}   \NDsym{:}  \NDnt{A}}
        }{\Gamma_{{\mathrm{1}}}  \NDsym{,}  \Phi  \NDsym{,}  \Gamma_{{\mathrm{2}}}  \vdash_\mathcal{L}  \NDsym{[}  \NDnt{t}  \NDsym{/}  \NDmv{z}  \NDsym{]}  \NDsym{(}   \mathsf{let}\, \NDmv{z}  :  \NDnt{X}  \otimes  \NDnt{Y} \,\mathsf{be}\, \NDmv{x}  \otimes  \NDmv{y} \,\mathsf{in}\, \NDnt{s}   \NDsym{)}  \NDsym{:}  \NDnt{A}}
      \end{math}
    \end{center}
  \item \NDdruleSXXtenETwoName:
    \begin{center}
      \tiny
      $\NDdruleSXXtenETwo{}$
    \end{center}
    maps to
    \begin{center}
      \tiny
      \begin{math}
        $$\mprset{flushleft}
        \inferrule* [right={\tiny cut2}] {
          {\Gamma  \vdash_\mathcal{L}  \NDnt{s_{{\mathrm{1}}}}  \NDsym{:}  \NDnt{A}  \triangleright  \NDnt{B}} \\
          $$\mprset{flushleft}
          \inferrule* [right={\tiny tenL2}] {
            {\Delta_{{\mathrm{1}}}  \NDsym{,}  \NDmv{x}  \NDsym{:}  \NDnt{A}  \NDsym{,}  \NDmv{y}  \NDsym{:}  \NDnt{B}  \NDsym{,}  \Delta_{{\mathrm{2}}}  \vdash_\mathcal{L}  \NDnt{s_{{\mathrm{2}}}}  \NDsym{:}  \NDnt{C}}
          }{\Delta_{{\mathrm{1}}}  \NDsym{,}  \NDmv{z}  \NDsym{:}  \NDnt{A}  \triangleright  \NDnt{B}  \NDsym{,}  \Delta_{{\mathrm{2}}}  \vdash_\mathcal{L}   \mathsf{let}\, \NDmv{z}  :  \NDnt{A}  \triangleright  \NDnt{B} \,\mathsf{be}\, \NDmv{x}  \triangleright  \NDmv{y} \,\mathsf{in}\, \NDnt{s_{{\mathrm{2}}}}   \NDsym{:}  \NDnt{C}}
        }{\Delta_{{\mathrm{1}}}  \NDsym{,}  \Gamma  \NDsym{,}  \Delta_{{\mathrm{2}}}  \vdash_\mathcal{L}  \NDsym{[}  \NDnt{s_{{\mathrm{1}}}}  \NDsym{/}  \NDmv{z}  \NDsym{]}  \NDsym{(}   \mathsf{let}\, \NDmv{z}  :  \NDnt{A}  \triangleright  \NDnt{B} \,\mathsf{be}\, \NDmv{x}  \triangleright  \NDmv{y} \,\mathsf{in}\, \NDnt{s_{{\mathrm{2}}}}   \NDsym{)}  \NDsym{:}  \NDnt{C}}
      \end{math}
    \end{center}
  \item \NDdruleSXXimprEName: (NOT SURE)
    \begin{center}
      \tiny
      $\NDdruleSXXimprE{}$
    \end{center}
    maps to
    \begin{center}
      \tiny
      \begin{math}
        $$\mprset{flushleft}
        \inferrule* [right={\tiny cut2}] {
          {\Gamma  \vdash_\mathcal{L}  \NDnt{s_{{\mathrm{1}}}}  \NDsym{:}  \NDnt{A}  \rightharpoonup  \NDnt{B}} \\
          $$\mprset{flushleft}
          \inferrule* [right={\tiny imprL}] {
            {\Delta  \vdash_\mathcal{L}  \NDnt{s_{{\mathrm{2}}}}  \NDsym{:}  \NDnt{A}} \\
            {\NDmv{x}  \NDsym{:}  \NDnt{B}  \vdash_\mathcal{L}  \NDmv{x}  \NDsym{:}  \NDnt{B}}
          }{\NDmv{y}  \NDsym{:}  \NDnt{A}  \rightharpoonup  \NDnt{B}  \NDsym{,}  \Delta  \vdash_\mathcal{L}  \NDsym{[}   \mathsf{app}_r\, \NDmv{y} \, \NDnt{s_{{\mathrm{2}}}}   \NDsym{/}  \NDmv{x}  \NDsym{]}  \NDmv{x}  \NDsym{:}  \NDnt{B}}
        }{\Gamma  \NDsym{,}  \Delta  \vdash_\mathcal{L}  \NDsym{[}  \NDnt{s_{{\mathrm{1}}}}  \NDsym{/}  \NDmv{y}  \NDsym{]}  \NDsym{[}   \mathsf{app}_r\, \NDmv{y} \, \NDnt{s_{{\mathrm{2}}}}   \NDsym{/}  \NDmv{x}  \NDsym{]}  \NDmv{x}  \NDsym{:}  \NDnt{B}}
      \end{math}
    \end{center}
  \item \NDdruleSXXimplEName: (NOT SURE)
    \begin{center}
      \tiny
      $\NDdruleSXXimplE{}$
    \end{center}
    maps to
    \begin{center}
      \tiny
      \begin{math}
        $$\mprset{flushleft}
        \inferrule* [right={\tiny cut2}] {
          {\Gamma  \vdash_\mathcal{L}  \NDnt{s_{{\mathrm{1}}}}  \NDsym{:}  \NDnt{B}  \leftharpoonup  \NDnt{A}} \\
          $$\mprset{flushleft}
          \inferrule* [right={\tiny implL}] {
            {\Delta  \vdash_\mathcal{L}  \NDnt{s_{{\mathrm{2}}}}  \NDsym{:}  \NDnt{A}} \\
            {\NDmv{x}  \NDsym{:}  \NDnt{B}  \vdash_\mathcal{L}  \NDmv{x}  \NDsym{:}  \NDnt{B}}
          }{\Delta  \NDsym{,}  \NDmv{y}  \NDsym{:}  \NDnt{B}  \leftharpoonup  \NDnt{A}  \vdash_\mathcal{L}  \NDsym{[}   \mathsf{app}_l\, \NDmv{y} \, \NDnt{s_{{\mathrm{2}}}}   \NDsym{/}  \NDmv{x}  \NDsym{]}  \NDmv{x}  \NDsym{:}  \NDnt{B}}
        }{\Delta  \NDsym{,}  \Gamma  \vdash_\mathcal{L}  \NDsym{[}  \NDnt{s_{{\mathrm{1}}}}  \NDsym{/}  \NDmv{y}  \NDsym{]}  \NDsym{[}   \mathsf{app}_l\, \NDmv{y} \, \NDnt{s_{{\mathrm{2}}}}   \NDsym{/}  \NDmv{x}  \NDsym{]}  \NDmv{x}  \NDsym{:}  \NDnt{B}}
      \end{math}
    \end{center}
  \item \NDdruleSXXFEName:
    \begin{center}
      \tiny
      $\NDdruleSXXFE{}$
    \end{center}
    maps to
    \begin{center}
      \tiny
      \begin{math}
        $$\mprset{flushleft}
        \inferrule* [right={\tiny cut2}] {
          {\Gamma  \vdash_\mathcal{L}  \NDmv{y}  \NDsym{:}   \mathsf{F} \NDnt{X} } \\
          $$\mprset{flushleft}
          \inferrule* [right={\tiny FL}] {
            {\Delta_{{\mathrm{1}}}  \NDsym{,}  \NDmv{x}  \NDsym{:}  \NDnt{X}  \NDsym{,}  \Delta_{{\mathrm{2}}}  \vdash_\mathcal{L}  \NDnt{s}  \NDsym{:}  \NDnt{A}}
          }{\Delta_{{\mathrm{1}}}  \NDsym{,}  \NDmv{z}  \NDsym{:}   \mathsf{F} \NDnt{X}   \NDsym{,}  \Delta_{{\mathrm{2}}}  \vdash_\mathcal{L}   \mathsf{let}\, \NDmv{z}  :   \mathsf{F} \NDnt{X}  \,\mathsf{be}\,  \mathsf{F}\, \NDmv{x}  \,\mathsf{in}\, \NDnt{s}   \NDsym{:}  \NDnt{A}}
        }{\Delta_{{\mathrm{1}}}  \NDsym{,}  \Gamma  \NDsym{,}  \Delta_{{\mathrm{2}}}  \vdash_\mathcal{L}  \NDsym{[}  \NDmv{y}  \NDsym{/}  \NDmv{z}  \NDsym{]}  \NDsym{(}   \mathsf{let}\, \NDmv{y}  :   \mathsf{F} \NDnt{X}  \,\mathsf{be}\,  \mathsf{F}\, \NDmv{x}  \,\mathsf{in}\, \NDnt{s}   \NDsym{)}  \NDsym{:}  \NDnt{A}}
      \end{math}
    \end{center}
  \item \NDdruleSXXGEName:
    \begin{center}
      \tiny
      $\NDdruleSXXGE{}$
    \end{center}
    maps to
    \begin{center}
      \tiny
      \begin{math}
        $$\mprset{flushleft}
        \inferrule* [right={\tiny cut1}] {
          $$\mprset{flushleft}
          \inferrule* [right={\tiny GL}] {
            {\NDmv{x}  \NDsym{:}  \NDnt{A}  \vdash_\mathcal{L}  \NDmv{x}  \NDsym{:}  \NDnt{A}}
          }{\NDmv{y}  \NDsym{:}   \mathsf{G} \NDnt{A}   \vdash_\mathcal{L}   \mathsf{let}\, \NDmv{y}  :   \mathsf{G} \NDnt{A}  \,\mathsf{be}\,  \mathsf{G}\, \NDmv{x}  \,\mathsf{in}\, \NDmv{x}   \NDsym{:}  \NDnt{A}} \\
           {\Phi  \vdash_\mathcal{C}  \NDnt{t}  \NDsym{:}   \mathsf{G} \NDnt{A} }
        }{\Phi  \vdash_\mathcal{L}  \NDsym{[}  \NDnt{t}  \NDsym{/}  \NDmv{y}  \NDsym{]}  \NDsym{(}   \mathsf{let}\, \NDmv{y}  :   \mathsf{G} \NDnt{A}  \,\mathsf{be}\,  \mathsf{G}\, \NDmv{x}  \,\mathsf{in}\, \NDmv{x}   \NDsym{)}  \NDsym{:}  \NDnt{A}}
      \end{math}
    \end{center}
  \end{itemize}
\end{itemize}

Function $N:SE\rightarrow ND$ maps a proof in the sequent calculus to a proof of the same
sequent in the natural deduction. The function is defined as follows:

\begin{itemize}
\item Axioms map to axioms.
\item Instances of cut rules map to the admissible substitution rules.
\item Right rules map to introductions.
\item Left rules map to eliminations modulo some structural fiddling.
  \begin{itemize}
  \item \ElledruleTXXunitLName:
    \begin{center}
      \tiny
      $\ElledruleTXXunitL{}$
    \end{center}
    maps to
    \begin{center}
      \tiny
      \begin{math}
        $$\mprset{flushleft}
        \inferrule* [right={\tiny unitE}] {
          {\NDmv{x}  \NDsym{:}   \mathsf{UnitT}   \vdash_\mathcal{C}  \NDmv{x}  \NDsym{:}   \mathsf{UnitT} } \\
          {\Psi  \vdash_\mathcal{C}  \NDnt{t}  \NDsym{:}  \NDnt{X}}
        }{\NDmv{x}  \NDsym{:}   \mathsf{UnitT}   \NDsym{,}  \Psi  \vdash_\mathcal{C}   \mathsf{let}\, \NDmv{x}  :   \mathsf{UnitT}  \,\mathsf{be}\,  \mathsf{trivT}  \,\mathsf{in}\, \NDnt{t}   \NDsym{:}  \NDnt{X}}
      \end{math}
    \end{center}
  \item \ElledruleTXXtenLName:
    \begin{center}
      \tiny
      $\ElledruleTXXtenL{}$
    \end{center}
    maps to
    \begin{center}
      \tiny
      \begin{math}
        $$\mprset{flushleft}
        \inferrule* [right={\tiny tenE}] {
          {\NDmv{z}  \NDsym{:}  \NDnt{X}  \otimes  \NDnt{Y}  \vdash_\mathcal{C}  \NDmv{z}  \NDsym{:}  \NDnt{X}  \otimes  \NDnt{Y}} \\
          {\Phi  \NDsym{,}  \NDmv{x}  \NDsym{:}  \NDnt{X}  \NDsym{,}  \NDmv{y}  \NDsym{:}  \NDnt{Y}  \NDsym{,}  \Psi  \vdash_\mathcal{C}  \NDnt{t}  \NDsym{:}  \NDnt{Z}}
        }{\Phi  \NDsym{,}  \NDmv{z}  \NDsym{:}  \NDnt{X}  \otimes  \NDnt{Y}  \NDsym{,}  \Psi  \vdash_\mathcal{C}   \mathsf{let}\, \NDmv{z}  :  \NDnt{X}  \otimes  \NDnt{Y} \,\mathsf{be}\, \NDmv{x}  \otimes  \NDmv{y} \,\mathsf{in}\, \NDnt{t}   \NDsym{:}  \NDnt{Z}}
      \end{math}
    \end{center}
  \item \ElledruleTXXimpLName:
    \begin{center}
      \tiny
      $\ElledruleTXXimpL{}$
    \end{center}
    maps to
    \begin{center}
      \tiny
      \begin{math}
        $$\mprset{flushleft}
        \inferrule* [right={\tiny cut1}] {
          $$\mprset{flushleft}
          \inferrule* [right={\tiny impE}] {
            {\NDmv{z}  \NDsym{:}  \NDnt{X}  \multimap  \NDnt{Y}  \vdash_\mathcal{C}  \NDmv{z}  \NDsym{:}  \NDnt{X}  \multimap  \NDnt{Y}} \\
            {\Phi  \vdash_\mathcal{C}  \NDnt{t_{{\mathrm{1}}}}  \NDsym{:}  \NDnt{X}}
          }{\NDmv{z}  \NDsym{:}  \NDnt{X}  \multimap  \NDnt{Y}  \NDsym{,}  \Phi  \vdash_\mathcal{C}   \mathsf{app}\, \NDmv{z} \, \NDnt{t_{{\mathrm{1}}}}   \NDsym{:}  \NDnt{Y}} \\
           {\Psi_{{\mathrm{1}}}  \NDsym{,}  \NDmv{x}  \NDsym{:}  \NDnt{Y}  \NDsym{,}  \Psi_{{\mathrm{2}}}  \vdash_\mathcal{C}  \NDnt{t_{{\mathrm{2}}}}  \NDsym{:}  \NDnt{Z}}
        }{\Psi_{{\mathrm{1}}}  \NDsym{,}  \NDmv{z}  \NDsym{:}  \NDnt{X}  \multimap  \NDnt{Y}  \NDsym{,}  \Phi  \NDsym{,}  \Psi_{{\mathrm{2}}}  \vdash_\mathcal{C}  \NDsym{[}   \mathsf{app}\, \NDmv{z} \, \NDnt{t_{{\mathrm{2}}}}   \NDsym{/}  \NDmv{x}  \NDsym{]}  \NDnt{t_{{\mathrm{2}}}}  \NDsym{:}  \NDnt{Z}}
      \end{math}
    \end{center}
  \item \ElledruleSXXunitLOneName:
    \begin{center}
      \tiny
      $\ElledruleSXXunitLOne{}$
    \end{center}
    maps to
    \begin{center}
      \tiny
      \begin{math}
        $$\mprset{flushleft}
        \inferrule* [right={\tiny unitE1}] {
          {\NDmv{x}  \NDsym{:}   \mathsf{UnitT}   \vdash_\mathcal{C}  \NDmv{x}  \NDsym{:}   \mathsf{UnitT} } \\
          {\Delta  \vdash_\mathcal{L}  \NDnt{s}  \NDsym{:}  \NDnt{A}}
        }{\NDmv{x}  \NDsym{:}   \mathsf{UnitT}   \NDsym{,}  \Delta  \vdash_\mathcal{L}   \mathsf{let}\, \NDmv{x}  :   \mathsf{UnitT}  \,\mathsf{be}\,  \mathsf{trivT}  \,\mathsf{in}\, \NDnt{s}   \NDsym{:}  \NDnt{A}}
      \end{math}
    \end{center}
  \item \ElledruleSXXunitLTwoName:
    \begin{center}
      \tiny
      $\ElledruleSXXunitLTwo{}$
    \end{center}
    maps to
    \begin{center}
      \tiny
      \begin{math}
        $$\mprset{flushleft}
        \inferrule* [right={\tiny unitE2}] {
          {\NDmv{x}  \NDsym{:}   \mathsf{UnitS}   \vdash_\mathcal{L}  \NDmv{x}  \NDsym{:}   \mathsf{UnitS} } \\
          {\Delta  \vdash_\mathcal{L}  \NDnt{s}  \NDsym{:}  \NDnt{A}}
        }{\NDmv{x}  \NDsym{:}   \mathsf{UnitS}   \NDsym{,}  \Delta  \vdash_\mathcal{L}   \mathsf{let}\, \NDmv{x}  :   \mathsf{UnitS}  \,\mathsf{be}\,  \mathsf{trivS}  \,\mathsf{in}\, \NDnt{s}   \NDsym{:}  \NDnt{A}}
      \end{math}
    \end{center}
  \item \ElledruleSXXtenLOneName:
    \begin{center}
      \tiny
      $\ElledruleSXXtenLOne{}$
    \end{center}
    maps to
    \begin{center}
      \tiny
      \begin{math}
        $$\mprset{flushleft}
        \inferrule* [right={\tiny tenE1}] {
          {\NDmv{z}  \NDsym{:}  \NDnt{X}  \otimes  \NDnt{Y}  \vdash_\mathcal{C}  \NDmv{z}  \NDsym{:}  \NDnt{X}  \otimes  \NDnt{Y}} \\
          {\Gamma  \NDsym{,}  \NDmv{x}  \NDsym{:}  \NDnt{X}  \NDsym{,}  \NDmv{y}  \NDsym{:}  \NDnt{Y}  \NDsym{,}  \Delta  \vdash_\mathcal{L}  \NDnt{s}  \NDsym{:}  \NDnt{A}}
        }{\Gamma  \NDsym{,}  \NDmv{z}  \NDsym{:}  \NDnt{X}  \otimes  \NDnt{Y}  \NDsym{,}  \Delta  \vdash_\mathcal{L}   \mathsf{let}\, \NDmv{z}  :  \NDnt{X}  \otimes  \NDnt{Y} \,\mathsf{be}\, \NDmv{x}  \otimes  \NDmv{y} \,\mathsf{in}\, \NDnt{s}   \NDsym{:}  \NDnt{A}}
      \end{math}
    \end{center}
  \item \ElledruleSXXtenLTwoName:
    \begin{center}
      \tiny
      $\ElledruleSXXtenLTwo{}$
    \end{center}
    maps to
    \begin{center}
      \tiny
      \begin{math}
        $$\mprset{flushleft}
        \inferrule* [right={\tiny tenE2}] {
          {\NDmv{z}  \NDsym{:}  \NDnt{A}  \triangleright  \NDnt{B}  \vdash_\mathcal{L}  \NDmv{z}  \NDsym{:}  \NDnt{A}  \triangleright  \NDnt{B}} \\
          {\Gamma  \NDsym{,}  \NDmv{x}  \NDsym{:}  \NDnt{A}  \NDsym{,}  \NDmv{y}  \NDsym{:}  \NDnt{B}  \NDsym{,}  \Delta  \vdash_\mathcal{L}  \NDnt{s}  \NDsym{:}  \NDnt{C}}
        }{\Gamma  \NDsym{,}  \NDmv{z}  \NDsym{:}  \NDnt{A}  \triangleright  \NDnt{B}  \NDsym{,}  \Delta  \vdash_\mathcal{L}   \mathsf{let}\, \NDmv{z}  :  \NDnt{A}  \triangleright  \NDnt{B} \,\mathsf{be}\, \NDmv{x}  \triangleright  \NDmv{y} \,\mathsf{in}\, \NDnt{s}   \NDsym{:}  \NDnt{C}}
      \end{math}
    \end{center}
  \item \ElledruleSXXimpLName:
    \begin{center}
      \tiny
      $\ElledruleSXXimpL{}$
    \end{center}
    maps to
    \begin{center}
      \tiny
      \begin{math}
        $$\mprset{flushleft}
        \inferrule* [right={\tiny cut1}] {
          $$\mprset{flushleft}
          \inferrule* [right={\tiny impE}] {
            {\NDmv{z}  \NDsym{:}  \NDnt{X}  \multimap  \NDnt{Y}  \vdash_\mathcal{C}  \NDmv{z}  \NDsym{:}  \NDnt{X}  \multimap  \NDnt{Y}} \\
            {\Phi  \vdash_\mathcal{C}  \NDnt{t}  \NDsym{:}  \NDnt{X}}
          }{\NDmv{z}  \NDsym{:}  \NDnt{X}  \multimap  \NDnt{Y}  \NDsym{,}  \Phi  \vdash_\mathcal{C}   \mathsf{app}\, \NDmv{z} \, \NDnt{t}   \NDsym{:}  \NDnt{Y}} \\
           {\Gamma  \NDsym{,}  \NDmv{x}  \NDsym{:}  \NDnt{Y}  \NDsym{,}  \Delta  \vdash_\mathcal{L}  \NDnt{s}  \NDsym{:}  \NDnt{A}}
        }{\Gamma  \NDsym{,}  \NDmv{z}  \NDsym{:}  \NDnt{X}  \multimap  \NDnt{Y}  \NDsym{,}  \Phi  \NDsym{,}  \Delta  \vdash_\mathcal{L}  \NDsym{[}   \mathsf{app}\, \NDmv{z} \, \NDnt{t}   \NDsym{/}  \NDmv{x}  \NDsym{]}  \NDnt{s}  \NDsym{:}  \NDnt{A}}
      \end{math}
    \end{center}
  \item \ElledruleSXXimprLName:
    \begin{center}
      \tiny
      $\ElledruleSXXimprL{}$
    \end{center}
    maps to
    \begin{center}
      \tiny
      \begin{math}
        $$\mprset{flushleft}
        \inferrule* [right={\tiny cut2}] {
          $$\mprset{flushleft}
          \inferrule* [right={\tiny imprE}] {
            {\NDmv{z}  \NDsym{:}  \NDnt{A}  \rightharpoonup  \NDnt{B}  \vdash_\mathcal{L}  \NDmv{z}  \NDsym{:}  \NDnt{A}  \rightharpoonup  \NDnt{B}} \\
            {\Gamma  \vdash_\mathcal{L}  \NDnt{s_{{\mathrm{1}}}}  \NDsym{:}  \NDnt{A}}
          }{\NDmv{z}  \NDsym{:}  \NDnt{A}  \rightharpoonup  \NDnt{B}  \NDsym{,}  \Gamma  \vdash_\mathcal{L}   \mathsf{app}_r\, \NDmv{z} \, \NDnt{s_{{\mathrm{1}}}}   \NDsym{:}  \NDnt{B}} \\
           {\Delta  \NDsym{,}  \NDmv{x}  \NDsym{:}  \NDnt{B}  \vdash_\mathcal{L}  \NDnt{s_{{\mathrm{2}}}}  \NDsym{:}  \NDnt{C}}
        }{\Delta  \NDsym{,}  \NDmv{z}  \NDsym{:}  \NDnt{A}  \rightharpoonup  \NDnt{B}  \NDsym{,}  \Gamma  \vdash_\mathcal{L}  \NDsym{[}   \mathsf{app}_r\, \NDmv{z} \, \NDnt{s_{{\mathrm{1}}}}   \NDsym{/}  \NDmv{x}  \NDsym{]}  \NDnt{s_{{\mathrm{2}}}}  \NDsym{:}  \NDnt{C}}
      \end{math}
    \end{center}
  \item \ElledruleSXXimplLName:
    \begin{center}
      \tiny
      $\ElledruleSXXimplL{}$
    \end{center}
    maps to
    \begin{center}
      \tiny
      \begin{math}
        $$\mprset{flushleft}
        \inferrule* [right={\tiny cut2}] {
          $$\mprset{flushleft}
          \inferrule* [right={\tiny implE}] {
            {\NDmv{z}  \NDsym{:}  \NDnt{B}  \leftharpoonup  \NDnt{A}  \vdash_\mathcal{L}  \NDmv{z}  \NDsym{:}  \NDnt{B}  \leftharpoonup  \NDnt{A}} \\
            {\Gamma  \vdash_\mathcal{L}  \NDnt{s_{{\mathrm{1}}}}  \NDsym{:}  \NDnt{A}}
          }{\Gamma  \NDsym{,}  \NDmv{z}  \NDsym{:}  \NDnt{B}  \leftharpoonup  \NDnt{A}  \vdash_\mathcal{L}   \mathsf{app}_l\, \NDmv{z} \, \NDnt{s_{{\mathrm{1}}}}   \NDsym{:}  \NDnt{B}} \\
           {\NDmv{x}  \NDsym{:}  \NDnt{B}  \NDsym{,}  \Delta  \vdash_\mathcal{L}  \NDnt{s_{{\mathrm{2}}}}  \NDsym{:}  \NDnt{C}}
        }{\Gamma  \NDsym{,}  \NDmv{z}  \NDsym{:}  \NDnt{B}  \leftharpoonup  \NDnt{A}  \NDsym{,}  \Delta  \vdash_\mathcal{L}  \NDsym{[}   \mathsf{app}_l\, \NDmv{z} \, \NDnt{s_{{\mathrm{1}}}}   \NDsym{/}  \NDmv{x}  \NDsym{]}  \NDnt{s_{{\mathrm{2}}}}  \NDsym{:}  \NDnt{C}}
      \end{math}
    \end{center}
  \item \ElledruleSXXFlName:
    \begin{center}
      \tiny
      $\ElledruleSXXFl{}$
    \end{center}
    maps to
    \begin{center}
      \tiny
      \begin{math}
        $$\mprset{flushleft}
        \inferrule* [right={\tiny FE}] {
          {\NDmv{z}  \NDsym{:}   \mathsf{F} \NDnt{X}   \vdash_\mathcal{L}  \NDmv{z}  \NDsym{:}   \mathsf{F} \NDnt{X} } \\
          {\Gamma  \NDsym{,}  \NDmv{x}  \NDsym{:}  \NDnt{X}  \NDsym{,}  \Delta  \vdash_\mathcal{L}  \NDnt{s}  \NDsym{:}  \NDnt{A}}
        }{\Gamma  \NDsym{,}  \NDmv{z}  \NDsym{:}   \mathsf{F} \NDnt{X}   \NDsym{,}  \Delta  \vdash_\mathcal{L}   \mathsf{let}\,  \mathsf{F} \NDmv{x}   :   \mathsf{F} \NDnt{X}  \,\mathsf{be}\, \NDmv{z} \,\mathsf{in}\, \NDnt{s}   \NDsym{:}  \NDnt{A}}
      \end{math}
    \end{center}
  \item \ElledruleSXXGlName:
    \begin{center}
      \tiny
      $\ElledruleSXXGl{}$
    \end{center}
    maps to
    \begin{center}
      \tiny
      \begin{math}
        $$\mprset{flushleft}
        \inferrule* [right={\tiny cut2}] {
          $$\mprset{flushleft}
          \inferrule* [right={\tiny GE}] {
            {\NDmv{y}  \NDsym{:}   \mathsf{G} \NDnt{A}   \vdash_\mathcal{C}  \NDmv{y}  \NDsym{:}   \mathsf{G} \NDnt{A} }
          }{\NDmv{y}  \NDsym{:}   \mathsf{G} \NDnt{A}   \vdash_\mathcal{L}   \mathsf{derelict}\, \NDmv{y}   \NDsym{:}  \NDnt{A}} \\
           {\Gamma  \NDsym{,}  \NDmv{x}  \NDsym{:}  \NDnt{A}  \NDsym{,}  \Delta  \vdash_\mathcal{L}  \NDnt{s}  \NDsym{:}  \NDnt{B}}
        }{\Gamma  \NDsym{,}  \NDmv{y}  \NDsym{:}   \mathsf{G} \NDnt{A}   \NDsym{,}  \Delta  \vdash_\mathcal{L}  \NDsym{[}   \mathsf{derelict}\, \NDmv{y}   \NDsym{/}  \NDmv{x}  \NDsym{]}  \NDnt{s}  \NDsym{:}  \NDnt{B}}
      \end{math}
    \end{center}
    
  \end{itemize}
\end{itemize}

\subsection{Strong Normalization of LAM Logic}
\label{subsec:strong_normalization_of_lam_logic}
\input{Elle-to-LNL-ott}
% subsection strong_normalization_of_lam_logic (end)
