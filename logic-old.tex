In a LAM, the SMCC $\cat{C}$ models the commutative linear logic and the Lambeck category
$\cat{L}$ models the non-commutative variant. In Section~\ref{subsec:elle}, we will present the
term assignment for sequent calculus of both sides and prove the cut elimination theorem. In
Section~\ref{subsec:elle-nd}, we present the term assignment for natural deduction of both sides
and prove the logic is strongly normalizing.

A sequent in the commutative side is of the form $[[P,I |-c t : X]]$. The types must be $X$,
$Y$, $Z$, etc., which are objects in the SMCC $\cat{C}$. Tye typing contexts are multisets.
Suppose $[[P]]$ is the set $x_1:X_1, x_2:X_2, ..., x_m:X_m$ and $[[I]]$ is the set
$y_1:Y_1, y_2:Y_2,...,y_n:Y_n$, then the categorical interpretation of the sequent is the
morphism $(X_1\otimes X_2\otimes...\otimes X_m)\otimes(Y_1\otimes Y_2\otimes...\otimes Y_n)\rightarrow X$.

A sequent in the non-commutative side is of the form $[[G,D |-l s : A]]$. The types must be $A$,
$B$, $C$, etc., which are objects in the Lambek category $\cat{L}$. Tye typing contexts are
lists instead of multisets. The typing contexts are mixed in the sense that they could include
contexts from the commutative side. When a commutative context $[[P]]=\{x_1:X_1,...,x_m:X_m\}$
is included, it is interpreted as the object $F(X_1\otimes...\otimes X_m)$. Therefore, the
interpretation of the sequent $[[P,G |-l s : A]]$, where $[[P]]$ is defined as above and $[[G]]$
is the list $y_1:A_1,...,y_n:A_n$, is the morphism
$F(X_1\otimes...\otimes X_m)\tri(A_1\tri...\tri A_n)\rightarrow A$.

For the commutative side, since the contexts are multisets, the following exchange rule
is implicit in both sequent calculus and natural deduction:

\begin{center}
  \scriptsize
  $\ElledruleTXXbeta{}$
\end{center}



%%%%%%%%%%%%%%%%%%%%%%%%%%%%%%%%%%%%%%%%%%%%%%%%%%
\subsection{Sequent Calculus}
\label{subsec:elle}

The term assignment for sequent calculus of the commutative part of the model, i.e. the SMCC of
the adjunction, is defined in Figure~\ref{fig:elle-smcc}. And the term assignme for the
non-commutative part, i.e. the Lambek category of the adjunction, is defined in
Figure~\ref{fig:elle-lambek}. We do not have the structural rules except for exchange because
the calculus is for linear logic. 

\begin{figure}[!h]
 \scriptsize
  \begin{mdframed}
    \begin{mathpar}
      \ElledruleTXXax{} \qquad\qquad \ElledruleTXXunitL{} \qquad\qquad \ElledruleTXXunitR{} \\
      \ElledruleTXXtenL{} \qquad\qquad \ElledruleTXXtenR{} \\
      \ElledruleTXXimpL{} \qquad\qquad \ElledruleTXXimpR{} \\
      \ElledruleTXXGr{} \qquad\qquad \ElledruleTXXcut{}
    \end{mathpar}
  \end{mdframed}
\caption{Sequent Calculus: Commutative Part}
\label{fig:elle-smcc}
\end{figure}

\begin{figure}[!h]
 \scriptsize
  \begin{mdframed}
    \begin{mathpar}
      \ElledruleSXXax{} \qquad\qquad \ElledruleSXXunitR{} \qquad\qquad \ElledruleSXXunitLOne{} \\
      \ElledruleSXXunitLTwo{} \qquad\qquad \ElledruleSXXbeta{} \\
      \ElledruleSXXtenLOne{} \qquad\qquad \ElledruleSXXtenLTwo{} \\
      \ElledruleSXXtenR{} \qquad\qquad \ElledruleSXXimpL{} \\
      \ElledruleSXXimprL{} \qquad\qquad \ElledruleSXXimplL{} \\
      \ElledruleSXXimprR{} \qquad\qquad \ElledruleSXXimplR{} \qquad\qquad \ElledruleSXXFr{} \\
      \ElledruleSXXFl{} \qquad\qquad \ElledruleSXXGl{} \\
      \ElledruleSXXcutOne{} \qquad\qquad \ElledruleSXXcutTwo{} \\
    \end{mathpar}
  \end{mdframed}
\caption{Sequent Calculus: Non-Commutative Part}
\label{fig:elle-lambek}
\end{figure}

Next, we prove cut elimination for the sequent calculus. We define the \textit{degree $|X|$
(or $|A|$) of a commutative (or non-commutative) proposition} to be the number of logical
connectives in the proposition. For instance, $|[[X (x) Y]]| = |[[X]]| + |[[Y]]| + 1$. The
\textit{cut rank} $c(\Pi)$ of a proof $\Pi$ is one more than the maximum of the ranks of all
the cut formulae in $\Pi$, and $0$ if $\Pi$ is cut-free. Then \textit{depth} $d(\Pi)$ of a
proof $\Pi$ is the length of the longest path in the proof tree (so the depth of an axiom is
$0$). The key to the proof of cut elimination is the following lemma, which shows how to
transform a single cut, either by removing it or by replacing it with one or more simpler cuts.

%\textbf{degree of a cut rule} is the degree of the cut formula. The following key cases
%demonstrate how we can replace a cut with at most two cuts with lower degree. The
%\textbf{degree $|\Pi|$ of a proof} $\Pi$ is the maximum of the degrees of all cut fules in the
%proof and $|\Pi|=0$ if $\Pi$ is cut-free. Finally, the \textbf{height $h(\Pi)$ of a proof
%$\Pi$} is the length of the longest path in the proof tree and the height of an axiom is $0$.

\begin{lemma}[Cut Reduction]
  \label{lem:cut-reduction}
  \begin{enumerate}
  \item If $\Pi_1$ is a proof of $<<I |-c X>>$ and $\Pi_2$ is a proof of $<<P1,X,P2 |-c Y>>$
        with $c(\Pi_1)$, $c(\Pi_2)\leq |X|$, then there exists a proof $\Pi$ of
        $<<P1, I, P2 |-c Y>>$ with $c(\Pi)\leq |X|$.
  \item If $\Pi_1$ is a proof of $<<I |-c X>>$ and $\Pi_2$ is a proof of $<<G1,X,G2 |-l A>>$
        with $c(\Pi_1)$, $c(\Pi_2)\leq |X|$, then there exists a proof $\Pi$ of
        $<<G1, I, G2 |-l A>>$ with $c(\Pi)\leq |X|$.
  \item If $\Pi_1$ is a proof of $<<G |-l A>>$ and $\Pi_2$ is a proof of $<<D1,A,D2 |-l B>>$
        with $c(\Pi_1)$, $c(\Pi_2)\leq |A|$, then there exists a proof $\Pi$ of
        $<<D1, G, D2 |-l B>>$ with $c(\Pi)\leq |A|$.
  \end{enumerate}
\end{lemma}
\begin{proof}
  We consider cases according to the classes of the last rules used in $\Pi_1$ and $\Pi_2$.
  \begin{enumerate}
  \item Both proofs end in logical rules which introduce the cut formula, i.e. $\Pi_1$ ends
        in a right rule and $\Pi_2$ in a corresponding left rule.
    \begin{itemize}
    \item $<<UnitT>>$:
      \begin{center}
        \scriptsize
        $\Pi_1:$
        \begin{math}
          $$\mprset{flushleft}
          \inferrule* [right={\tiny unitR}] {
            \,
          }{<<. |-c UnitT>>}
        \end{math}
        \qquad\qquad
        $\Pi_2:$
        \begin{math}
          $$\mprset{flushleft}
          \inferrule* [right={\tiny unitL}] {
            {
              \begin{array}{c}
                \pi \\
                {<<I |-c X>>}
              \end{array}
            }
          }{<<UnitT, I |-c X>>}
        \end{math}
      \end{center}
      By assumption, $c(\Pi_1),c(\Pi_2)\leq |<<UnitT>>|$. The proof $\Pi$ is the subproof $\pi$
      in $\Pi_2$ for sequent $<<I |-c X>>$. So $c(\Pi)=c(\Pi_2)\leq |<<UnitT>>|$.

      \begin{center}
        \scriptsize
        $\Pi_1:$
        \begin{math}
          $$\mprset{flushleft}
          \inferrule* [right={\tiny unitR}] {
            \,
          }{<<. |-c UnitT>>}
        \end{math}
        \qquad\qquad
        $\Pi_2:$
        \begin{math}
          $$\mprset{flushleft}
          \inferrule* [right={\tiny unitL1}] {
            {
              \begin{array}{c}
                \pi \\
                {<<G |-l A>>}
              \end{array}
            }
          }{<<UnitT, G |-l A>>}
        \end{math}
      \end{center}
      Similar as above, $\Pi$ is $\pi$.

    \item $\otimes$:
      \begin{center}
        \scriptsize
        $\Pi_1:$
        \begin{math}
          $$\mprset{flushleft}
          \inferrule* [right={\tiny tenR}] {
            {
              \begin{array}{cc}
                \pi_1 & \pi_2 \\
                {<<I1 |-c X>>} & {<<I2 |-c Y>>}
              \end{array}
            }
          }{<<I1, I2 |-c X (x) Y>>}
        \end{math}
        \qquad\qquad
        $\Pi_2:$
        \begin{math}
          $$\mprset{flushleft}
          \inferrule* [right={\tiny tenL}] {
            {
              \begin{array}{c}
                \pi_3 \\
                {<<P1, X, Y, P2 |-c Z>>}
              \end{array}
            }
          }{<<P1, X (x) Y, P2 |-c Z>>}
        \end{math}
      \end{center}
      By assumption, $c(\Pi_1),c(\Pi_2)\leq |<<X (x) Y>>| = |X|+|Y|+1$. The proof $\Pi$ can be
      constructed as follows, and
      $c(\Pi)\leq max\{c(\pi_1),c(\pi_2),c(\pi_3),|X|+1,|Y|+1\}\leq |X|+|Y|+1 = |<<X (x) Y>>|$.
      \begin{center}
        \scriptsize
        \begin{math}
          $$\mprset{flushleft}
          \inferrule* [right={\tiny cut}] {
            {
              \begin{array}{c}
                \pi_1 \\
                {<<I1 |-c X>>}
              \end{array}
            }
            $$\mprset{flushleft}
            \inferrule* [right={\tiny cut}] {
            {
              \begin{array}{cc}
                \pi_2 & \pi_3 \\
                {<<I2 |-c Y>>} & {<<P1, X, Y, P2 |-c Z>>}
              \end{array}
            }
            }{<<P1, X, I2, P2 |-c Z>>}
          }{<<P1, I1, I2, P2 |-c Z>>}
        \end{math}
      \end{center}
    \item $\multimap$:
      \begin{center}
        \scriptsize
        $\Pi_1:$
        \begin{math}
          $$\mprset{flushleft}
          \inferrule* [right={\tiny tenR}] {
            {
              \begin{array}{c}
                \pi_1 \\
                {<<I1, X |-c Y>>}
              \end{array}
            }
          }{<<I1 |-c X -o Y>>}
        \end{math}
        \qquad\qquad
        $\Pi_2:$
        \begin{math}
          $$\mprset{flushleft}
          \inferrule* [right={\tiny tenL}] {
            {
              \begin{array}{cc}
                \pi_2 & \pi_3 \\
                {<<I2 |-c X>>} & {<<P1, Y, P2 |-c Z>>}
              \end{array}
            }
          }{<<P1, X -o Y, I, P2 |-c Z>>}
        \end{math}
      \end{center}
      By assumption, $c(\Pi_1),c(\Pi_2)\leq |<<X -o Y>>| = |X|+|Y|+1$. The proof $\Pi$ is
      constructed as follows
      $c(\Pi)\leq max\{c(\pi_1),c(\pi_2),c(\pi_3),|X|+1,|Y|+1\}\leq |X|+|Y|+1 = |<<X -o Y>>|$.
      \begin{center}
        \scriptsize
        \begin{math}
          $$\mprset{flushleft}
          \inferrule* [right={\tiny tenR}] {
            $$\mprset{flushleft}
            \inferrule* [right={\tiny tenR}] {
              {
                \begin{array}{cc}
                  \pi_1 & \pi_2 \\
                  {<<I1, X |-c Y>>} & {<<I2 |-c X>>}
                \end{array}
              }
            }{<<I1, I2 |-c Y>>} \\
             {
               \begin{array}{c}
                 \pi_3 \\
                 {<<P1, Y, P2 |-c Z>>}
               \end{array}
             }
          }{<<P1, I1, I2, P2 |-c Z>>}
        \end{math}
      \end{center}
    \item $<<UnitS>>$:
      \begin{center}
        \scriptsize
        $\Pi_1:$
        \begin{math}
          $$\mprset{flushleft}
          \inferrule* [right={\tiny unitR}] {
            \,
          }{<<. |-l UnitS>>}
        \end{math}
        \qquad\qquad
        $\Pi_2:$
        \begin{math}
          $$\mprset{flushleft}
          \inferrule* [right={\tiny unitL2}] {
            {
              \begin{array}{c}
                \pi \\
                {<<D |-l A>>}
              \end{array}
            }
          }{<<UnitS, D |-l A>>}
        \end{math}
      \end{center}
      By assumption, $c(\Pi_1),c(\Pi_2)\leq |<<UnitS>>|$. The proof $\Pi$ is the subproof $\pi$
      in $\Pi_2$ for sequent $<<D |-l A>>$. So $c(\Pi)=c(\Pi_2)\leq |<<UnitS>>|$.

    \item $\tri$:
      \begin{center}
        \scriptsize
        $\Pi_1:$
        \begin{math}
          $$\mprset{flushleft}
          \inferrule* [right={\tiny tenR}] {
            {
              \begin{array}{cc}
                \pi_1 & \pi_2 \\
                {<<G1 |-l A>>} & {<<G2 |-l B>>}
              \end{array}
            }
          }{<<G1, G2 |-l A (>) B>>}
        \end{math}
        \qquad\qquad
        $\Pi_2:$
        \begin{math}
          $$\mprset{flushleft}
          \inferrule* [right={\tiny tenL1}] {
            {
              \begin{array}{c}
                \pi_3 \\
                {<<D1, A, B, D2 |-l C>>}
              \end{array}
            }
          }{<<D1, A (>) B, D2 |-l C>>}
        \end{math}
      \end{center}
      By assumption, $c(\Pi_1),c(\Pi_2)\leq |<<A (>) B>>| = |X|+|Y|+1$. The proof $\Pi$ can be
      constructed as follows, and
      $c(\Pi)\leq max\{c(\pi_1),c(\pi_2),c(\pi_3),|A|+1,|B|+1\}\leq |A|+|B|+1 = |<<A (>) B>>|$.
      \begin{center}
        \scriptsize
        \begin{math}
          $$\mprset{flushleft}
          \inferrule* [right={\tiny cut2}] {
            {
              \begin{array}{c}
                \pi_1 \\
                {<<G1 |-l A>>}
              \end{array}
            }
            $$\mprset{flushleft}
            \inferrule* [right={\tiny cut2}] {
            {
              \begin{array}{cc}
                \pi_2 & \pi_3 \\
                {<<G2 |-l B>>} & {<<D1, A, B, D2 |-l C>>}
              \end{array}
            }
            }{<<D1, A, G2, D2 |-l C>>}
          }{<<D1, G1, G2, P2 |-l C>>}
        \end{math}
      \end{center}
    \item $\lto$:
      \begin{center}
        \scriptsize
        $\Pi_1:$
        \begin{math}
          $$\mprset{flushleft}
          \inferrule* [right={\tiny imprR}] {
            {
              \begin{array}{c}
                \pi_1 \\
                {<<G, A |-l B>>}
              \end{array}
            }
          }{<<G |-l A -> B>>}
        \end{math}
        \qquad\qquad
        $\Pi_2:$
        \begin{math}
          $$\mprset{flushleft}
          \inferrule* [right={\tiny imprL}] {
            {
              \begin{array}{cc}
                \pi_2 & \pi_3 \\
                {<<D1 |-l A>>} & {<<D2, B |-l C>>}
              \end{array}
            }
          }{<<D2, A -> B, D1 |-l C>>}
        \end{math}
      \end{center}
      By assumption, $c(\Pi_1),c(\Pi_2)\leq |<<A -> B>>| = |A|+|B|+1$. The proof $\Pi$ is
      constructed as follows, and
      $c(\Pi)\leq max\{c(\pi_1),c(\pi_2),c(\pi_3),|A|+1,|B|+1\}\leq |A|+|B|+1 = |<<A -> B>>|$.
      \begin{center}
        \scriptsize
        \begin{math}
          $$\mprset{flushleft}
          \inferrule* [right={\tiny cut2}] {
            $$\mprset{flushleft}
            \inferrule* [right={\tiny cut2}] {
              {
                \begin{array}{cc}
                  \pi_1 & \pi_2 \\
                  {<<G, A |-l B>>} & {<<D1 |-l A>>}
                \end{array}
              }
            }{<<G, D1 |-l B>>}
             {
               \begin{array}{c}
                 \pi_3 \\
                 {<<D2, B |-l C>>}
               \end{array}
             }
          }{<<D2, G, D1 |-l C>>}
        \end{math}
      \end{center}

    \item $\rto$:
      \begin{center}
        \scriptsize
        $\Pi_1:$
        \begin{math}
          $$\mprset{flushleft}
          \inferrule* [right={\tiny implR}] {
            {
              \begin{array}{c}
                \pi_1 \\
                {<<A, G |-l B>>}
              \end{array}
            }
          }{<<G |-l B <- A>>}
        \end{math}
        \qquad\qquad
        $\Pi_2:$
        \begin{math}
          $$\mprset{flushleft}
          \inferrule* [right={\tiny implL}] {
            {
              \begin{array}{cc}
                \pi_2 & \pi_3 \\
                {<<D1 |-l A>>} & {<<B, D2 |-l C>>}
              \end{array}
            }
          }{<<D1, B <- A, D2 |-l C>>}
        \end{math}
      \end{center}
      By assumption, $c(\Pi_1),c(\Pi_2)\leq |<<B <- A>>| = |A|+|B|+1$. The proof $\Pi$ is
      constructed as follows, and
      $c(\Pi)\leq max\{c(\pi_1),c(\pi_2),c(\pi_3),|A|+1,|B|+1\}\leq |A|+|B|+1 = |<<B <- A>>|$.
      \begin{center}
        \scriptsize
        \begin{math}
          $$\mprset{flushleft}
          \inferrule* [right={\tiny cut1}] {
            $$\mprset{flushleft}
            \inferrule* [right={\tiny cut2}] {
              {
                \begin{array}{cc}
                  \pi_1 & \pi_2 \\
                  {<<A, G |-l B>>} & {<<D1 |-l A>>}
                \end{array}
              }
            }{<<D1, G |-l B>>}
             {
               \begin{array}{c}
                 \pi_3 \\
                 {<<B, D2 |-l C>>}
               \end{array}
             }
          }{<<D1, G, D2 |-l C>>}
        \end{math}
      \end{center}

    \item $F$:
      \begin{center}
        \scriptsize
        $\Pi_1:$
        \begin{math}
          $$\mprset{flushleft}
          \inferrule* [right={\tiny FR}] {
            {
              \begin{array}{c}
                \pi_1 \\
                {<<I |-c X>>}
              \end{array}
            }
          }{<<I |-l F X>>}
        \end{math}
        \qquad\qquad
        $\Pi_2:$
        \begin{math}
          $$\mprset{flushleft}
          \inferrule* [right={\tiny FL}] {
            {
              \begin{array}{c}
                \pi_2 \\
                {<<G, X, D |-l A>>}
              \end{array}
            }
          }{<<G, F X, D |-l A>>}
        \end{math}
      \end{center}
      By assumption, $c(\Pi_1),c(\Pi_2)\leq |<<F X>>| = |X|+1$. The proof $\Pi$ is
      constructed as follows, and $c(\Pi)\leq max\{c(\pi_1),c(\pi_2),|X|+1\}\leq |<<F X>>|$.
      \begin{center}
        \scriptsize
        \begin{math}
          $$\mprset{flushleft}
          \inferrule* [right={\tiny cut2}] {
            {
              \begin{array}{cc}
                \pi_1 & \pi_2 \\
                {<<I |-c X>>} & {<<G, X, D |-l A>>}
              \end{array}
            }
          }{<<G, I, D |-l A>>}
        \end{math}
      \end{center}

    \item $G$:
      \begin{center}
        \scriptsize
        $\Pi_1:$
        \begin{math}
          $$\mprset{flushleft}
          \inferrule* [right={\tiny GR}] {
            {
              \begin{array}{c}
                \pi_1 \\
                {<<I |-l A>>}
              \end{array}
            }
          }{<<I |-c Gf A>>}
        \end{math}
        \qquad\qquad
        $\Pi_2:$
        \begin{math}
          $$\mprset{flushleft}
          \inferrule* [right={\tiny GL}] {
            {
              \begin{array}{c}
                \pi_2 \\
                {<<G, A, D |-l B>>}
              \end{array}
            }
          }{<<G, Gf A, D |-l B>>}
        \end{math}
      \end{center}
      By assumption, $c(\Pi_1),c(\Pi_2)\leq |<<Gf A>>| = |A|+1$. The proof $\Pi$ is
      constructed as follows, and $c(\Pi)\leq max\{c(\pi_1),c(\pi_2),|A|+1\}\leq |<<Gf A>>|$.
      \begin{center}
        \scriptsize
        \begin{math}
          $$\mprset{flushleft}
          \inferrule* [right={\tiny GL}] {
            {
              \begin{array}{cc}
                \pi_1 & \pi_2 \\
                {<<I |-l A>>} & {<<G, A, D |-l B>>}
              \end{array}
            }
          }{<<G, I, D |-l B>>}
        \end{math}
      \end{center}
    \end{itemize}

  \item The last rule used in $\Pi_1$ is not a right logical rule.
    \begin{itemize}
    \item \ElledruleTXXcutName / $\cat{C}$-sequent:
      \begin{center}
        \scriptsize
        $\Pi_1$:
        \begin{math}
          $$\mprset{flushleft}
          \inferrule* [right={\tiny cut}] {
            {
              \begin{array}{cc}
                \pi_1 & \pi_2 \\
                {<<I1 |-c X>>} & {<<I2, X, I3 |-c Y>>}
              \end{array}
            }
          }{<<I2, I1, I3 |-c Y>>}
        \end{math}
        \qquad\qquad
        \begin{math}
          \begin{array}{c}
            \Pi_2 \\
            {<<P1, Y, P2 |-c Z>>}
          \end{array}
        \end{math}
      \end{center}
      By assumption, $c(\Pi_1),c(\Pi_2)\leq |Y|$. Since the cut rank of the last cut in
      $\Pi_1$ is $|X|+1$, then $|X|+1\leq |Y|$. By induction on the length of $\Pi_1$ and
      $\Pi_2$, the induction hypothesis states that there is a proof $\Pi'$ constructed from
      $\pi_2$ and $\Pi_2$ for sequent $<<P1, I2, X, I3, P2 |-c Z>>$ s.t. $c(\Pi')\leq|Y|$.
      Therefore, the proof $\Pi$ can be constructed as follows, and
      $c(\Pi)\leq max\{c(\pi_1),c(\Pi'),|X|+1\}\leq |Y|$.
      \begin{center}
        \scriptsize
        \begin{math}
          $$\mprset{flushleft}
          \inferrule* [right={\tiny cut}] {
            {
              \begin{array}{c}
                \pi_1 \\
                {<<I1 |-c X>>}
              \end{array}
            }
            $$\mprset{flushleft}
            \inferrule* [right={\tiny cut}] {
              {
                \begin{array}{cc}
                  \pi_2 & \Pi_2 \\
                  {<<I2, X, I3 |-c Y>>} & {<<P1, Y, P2 |-c Z>>}
                \end{array}
              }
            }{<<P1, I2, X, I3, P2 |-c Z>>}
          }{<<P1, I2, I1, I3, P2 |-c Z>>}
        \end{math}
      \end{center}

    \item \ElledruleTXXcutName / $\cat{L}$-sequent:
      \begin{center}
        \scriptsize
        $\Pi_1$:
        \begin{math}
          $$\mprset{flushleft}
          \inferrule* [right={\tiny cut}] {
            {
              \begin{array}{cc}
                \pi_1 & \pi_2 \\
                {<<I |-c X>>} & {<<P1, X, P2 |-c Y>>}
              \end{array}
            }
          }{<<P1, I, P2 |-c Y>>}
        \end{math}
        \qquad\qquad
        \begin{math}
          \begin{array}{c}
            \Pi_2 \\
            {<<G1, Y, G2 |-l A>>}
          \end{array}
        \end{math}
      \end{center}
      By assumption, $c(\Pi_1),c(\Pi_2)\leq |Y|$. Similar as above, $|X|+1\leq |Y|$ and there
      is a proof $\Pi'$ constructed from $\pi_2$ and $\Pi_2$ for sequent
      $<<G1, P1, X, P2, G2 |-l A>>$ s.t. $c(\Pi')\leq|Y|$. Therefore, the proof $\Pi$ can be
      constructed as follows, and $c(\Pi)\leq max\{c(\pi_1),c(\Pi'),|X|+1\}\leq |Y|$.
      \begin{center}
        \scriptsize
        \begin{math}
          $$\mprset{flushleft}
          \inferrule* [right={\tiny cut}] {
            {
              \begin{array}{c}
                \pi_1 \\
                {<<I |-c X>>}
              \end{array}
            }
            $$\mprset{flushleft}
            \inferrule* [right={\tiny cut}] {
              {
                \begin{array}{cc}
                  \pi_2 & \Pi_2 \\
                  {<<P1, X, P2 |-c Y>>} & {<<G1, Y, G2 |-l A>>}
                \end{array}
              }
            }{<<G1, P1, X, P2, G2 |-l A>>}
          }{<<G1, P1, I, P2, G2 |-l A>>}
        \end{math}
      \end{center}

    \item \ElledruleSXXcutOneName / $\cat{L}$-sequent:
      \begin{center}
        \scriptsize
        $\Pi_1$:
        \begin{math}
          $$\mprset{flushleft}
          \inferrule* [right={\tiny cut}] {
            {
              \begin{array}{cc}
                \pi_1 & \pi_2 \\
                {<<I |-c X>>} & {<<G1, X, G2 |-l A>>}
              \end{array}
            }
          }{<<G1, I, G2 |-l A>>}
        \end{math}
        \qquad\qquad
        \begin{math}
          \begin{array}{c}
            \Pi_2 \\
            {<<D1, A, D2 |-l B>>}
          \end{array}
        \end{math}
      \end{center}
      By assumption, $c(\Pi_1),c(\Pi_2)\leq |A|$. Similar as above, $|X|+1\leq |A|$ and there
      is a proof $\Pi'$ constructed from $\pi_2$ and $\Pi_2$ for sequent
      $<<D1, G1, X, G2, D2 |-l B>>$ s.t. $c(\Pi')\leq|A|$. Therefore, the proof $\Pi$ can be
      constructed as follows, and $c(\Pi)\leq max\{c(\pi_1),c(\Pi'),|X|+1\}\leq |A|$.
      \begin{center}
        \scriptsize
        \begin{math}
          $$\mprset{flushleft}
          \inferrule* [right={\tiny cut}] {
            {
              \begin{array}{c}
                \pi_1 \\
                {<<I |-c X>>}
              \end{array}
            }
            $$\mprset{flushleft}
            \inferrule* [right={\tiny cut}] {
              {
                \begin{array}{cc}
                  \pi_2 & \Pi_2 \\
                  {<<G1, X, G2 |-l A>>} & {<<D1, A, D2 |-l B>>}
                \end{array}
              }
            }{<<D1, G1, X, G2, D2 |-l B>>}
          }{<<D1, G1, I, G2, D2 |-l B>>}
        \end{math}
      \end{center}

    \item \ElledruleSXXcutTwoName / $\cat{L}$-sequent:
      \begin{center}
        \scriptsize
        $\Pi_1$:
        \begin{math}
          $$\mprset{flushleft}
          \inferrule* [right={\tiny cut}] {
            {
              \begin{array}{cc}
                \pi_1 & \pi_2 \\
                {<<G1 |-l A>>} & {<<G2, A, G3 |-l B>>}
              \end{array}
            }
          }{<<G2, G1, G3 |-l A>>}
        \end{math}
        \qquad\qquad
        \begin{math}
          \begin{array}{c}
            \Pi_2 \\
            {<<D1, B, D2 |-l C>>}
          \end{array}
        \end{math}
      \end{center}
      By assumption, $c(\Pi_1),c(\Pi_2)\leq |B|$. Similar as above, $|A|+1\leq |B|$ and there
      is a proof $\Pi'$ constructed from $\pi_2$ and $\Pi_2$ for sequent
      $<<D1, G2, A, G3, D2 |-l C>>$ s.t. $c(\Pi')\leq|A|$. Therefore, the proof $\Pi$ can be
      constructed as follows, and $c(\Pi)\leq max\{c(\pi_1),c(\Pi'),|A|+1\}\leq |B|$.
      \begin{center}
        \scriptsize
        \begin{math}
          $$\mprset{flushleft}
          \inferrule* [right={\tiny cut}] {
            {
              \begin{array}{c}
                \pi_1 \\
                {<<G1 |-l A>>}
              \end{array}
            }
            $$\mprset{flushleft}
            \inferrule* [right={\tiny cut}] {
              {
                \begin{array}{cc}
                  \pi_2 & \Pi_2 \\
                  {<<G2, A, G3 |-l B>>} & {<<D1, B, D2 |-l C>>}
                \end{array}
              }
            }{<<D1, G2, A, G3, D2 |-l C>>}
          }{<<D1, G2, G1, G3, D2 |-l C>>}
        \end{math}
      \end{center}

    \item \ElledruleTXXbetaName / $\cat{C}$-sequent:
      \begin{center}
        \scriptsize
        $\Pi_1$:
        \begin{math}
          $$\mprset{flushleft}
          \inferrule* [right={\tiny beta}] {
            {
              \begin{array}{c}
                \pi \\
                {<<I1, X1, X2, I2 |-c Y>>}
              \end{array}
            }
          }{<<I1, X2, X1, I2 |-c Y>>}
        \end{math}
        \qquad\qquad
        \begin{math}
          \begin{array}{c}
            \Pi_2 \\
            {<<P1, Y, P2 |-c Z>>}
          \end{array}
        \end{math}
      \end{center}
      By assumption, $c(\Pi_1),c(\Pi_2)\leq |Y|$. By induction on the length of $\Pi_1$ and
      $\Pi_2$, the induction hypothesis states that there is a proof $\Pi'$ constructed from
      $\pi$ and $\Pi_2$ for sequent \\
      $<<P1, I1, X1, X2, I2, P2 |-c Z>>$ s.t. $c(\Pi')\leq|Y|$.
      Therefore, the proof $\Pi$ can be constructed as follows, and $c(\Pi)=c(\Pi')\leq|Y|$.
      \begin{center}
        \scriptsize
        \begin{math}
          $$\mprset{flushleft}
          \inferrule* [right={\tiny beta}] {
            $$\mprset{flushleft}
            \inferrule* [right={\tiny cut}] {
              {
                \begin{array}{cc}
                  \pi & \Pi_2 \\
                  {<<I1, X1, X2, I2 |-c Y>>} & {<<P1, Y, P2 |-c Z>>}
                \end{array}
              }
            }{<<P1, I1, X1, X2, I2, P2 |-c Z>>}
          }{<<P1, I1, X2, X1, I2, P2 |-c Z>>}
        \end{math}
      \end{center}

    \item \ElledruleTXXbetaName / $\cat{L}$-sequent:
      \begin{center}
        \scriptsize
        $\Pi_1$:
        \begin{math}
          $$\mprset{flushleft}
          \inferrule* [right={\tiny beta}] {
            {
              \begin{array}{c}
                \pi \\
                {<<I1, X, Y, I2 |-c Z>>}
              \end{array}
            }
          }{<<I1, Y, X, I2 |-c Z>>}
        \end{math}
        \qquad\qquad
        \begin{math}
          \begin{array}{c}
            \Pi_2 \\
            {<<G1, Z, G2 |-l A>>}
          \end{array}
        \end{math}
      \end{center}
      By assumption, $c(\Pi_1),c(\Pi_2)\leq |Z|$. Similar as above, there is a proof $\Pi'$
      constructed from $\pi$ and $\Pi_2$ for sequent $<<G1, I1, X, Y, I2, G2 |-l A>>$ s.t.
      $c(\Pi')\leq|Z|$. Therefore, the proof $\Pi$ can be constructed as follows, and
      $c(\Pi)=c(\Pi')\leq|Z|$.
      \begin{center}
        \scriptsize
        \begin{math}
          $$\mprset{flushleft}
          \inferrule* [right={\tiny beta}] {
            $$\mprset{flushleft}
            \inferrule* [right={\tiny cut1}] {
              {
                \begin{array}{cc}
                  \pi & \Pi_2 \\
                  {<<I1, X, Y, I2 |-c Z>>} & {<<G1, Z, G2 |-l A>>}
                \end{array}
              }
            }{<<G1, I1, X, Y, I2, G2 |-l A>>}
          }{<<G1, I1, Y, X, I2, G2 |-l A>>}
        \end{math}
      \end{center}

    \item \ElledruleSXXbetaName / $\cat{L}$-sequent:
      \begin{center}
        \scriptsize
        $\Pi_1$:
        \begin{math}
          $$\mprset{flushleft}
          \inferrule* [right={\tiny beta}] {
            {
              \begin{array}{c}
                \pi \\
                {<<G1, X, Y, G2 |-l A>>}
              \end{array}
            }
          }{<<G1, Y, X, G2 |-l A>>}
        \end{math}
        \qquad\qquad
        \begin{math}
          \begin{array}{c}
            \Pi_2 \\
            {<<D1, A, D2 |-l B>>}
          \end{array}
        \end{math}
      \end{center}
      By assumption, $c(\Pi_1),c(\Pi_2)\leq |A|$. Similar as above, there is a proof $\Pi'$
      constructed from $\pi$ and $\Pi_2$ for sequent $<<D1, G1, X, Y, G2, D2 |-l B>>$ s.t.
      $c(\Pi')\leq|A|$. Therefore, the proof $\Pi$ can be constructed as follows, and
      $c(\Pi)=c(\Pi')\leq|A|$.
      \begin{center}
        \scriptsize
        \begin{math}
          $$\mprset{flushleft}
          \inferrule* [right={\tiny beta}] {
            $$\mprset{flushleft}
            \inferrule* [right={\tiny cut2}] {
              {
                \begin{array}{cc}
                  \pi & \Pi_2 \\
                  {<<G1, X, Y, G2 |-l A>>} & {<<D1, A, D2 |-l B>>}
                \end{array}
              }
            }{<<D1, G1, X, Y, G2, D2 |-l B>>}
          }{<<D1, G1, Y, X, G2, D2 |-l B>>}
        \end{math}
      \end{center}

    \item \ElledruleTXXaxName / $\cat{C}$-sequent:
      \begin{center}
        \scriptsize
        $\Pi_1$:
        \begin{math}
          $$\mprset{flushleft}
          \inferrule* [right={\tiny ax}] {
            \,
          }{<<X |-c X>>}
        \end{math}
        \qquad\qquad
        \begin{math}
          \begin{array}{c}
            \Pi_2 \\
            {<<I1, X, I2 |-c Y>>}
          \end{array}
        \end{math}
      \end{center}
      By assumption, $c(\Pi_1),c(\Pi_2)\leq |X|$. The proof $\Pi$ is the same as $\Pi_2$.

    \item \ElledruleTXXaxName / $\cat{L}$-sequent:
      \begin{center}
        \scriptsize
        $\Pi_1$:
        \begin{math}
          $$\mprset{flushleft}
          \inferrule* [right={\tiny ax}] {
            \,
          }{<<X |-c X>>}
        \end{math}
        \qquad\qquad
        \begin{math}
          \begin{array}{c}
            \Pi_2 \\
            {<<G1, X, G2 |-l A>>}
          \end{array}
        \end{math}
      \end{center}
      By assumption, $c(\Pi_1),c(\Pi_2)\leq |X|$. The proof $\Pi$ is the same as $\Pi_2$.

    \item \ElledruleSXXaxName / $\cat{L}$-sequent:
      \begin{center}
        \scriptsize
        $\Pi_1$:
        \begin{math}
          $$\mprset{flushleft}
          \inferrule* [right={\tiny ax}] {
            \,
          }{<<A |-l A>>}
        \end{math}
        \qquad\qquad
        \begin{math}
          \begin{array}{c}
            \Pi_2 \\
            {<<G1, A, G2 |-l B>>}
          \end{array}
        \end{math}
      \end{center}
      By assumption, $c(\Pi_1),c(\Pi_2)\leq |A|$. The proof $\Pi$ is the same as $\Pi_2$.

    \item \ElledruleTXXunitLName / $\cat{C}$-sequent:
      \begin{center}
        \scriptsize
        $\Pi_1$:
        \begin{math}
          $$\mprset{flushleft}
          \inferrule* [right={\tiny unitL}] {
            {
              \begin{array}{c}
                \pi \\
                {<<I |-c X>>}
              \end{array}
            }
          }{<<UnitT, X |-c X>>}
        \end{math}
        \qquad\qquad
        \begin{math}
          \begin{array}{c}
            \Pi_2 \\
            {<<P1, X, P2 |-c Y>>}
          \end{array}
        \end{math}
      \end{center}
      By assumption, $c(\Pi_1),c(\Pi_2)\leq |X|$. By induction, there is a proof $\Pi'$ from
      $\pi$ and $\Pi_2$ for sequent $<<P1, I, P2 |-c Y>>$ s.t. $c(\Pi')\leq |X|$. Therefore,
      the proof $\Pi$ can be constructed as follows, and $c(\Pi)=c(\Pi')\leq |X|$.
      \begin{center}
        \scriptsize
        \begin{math}
          $$\mprset{flushleft}
          \inferrule* [right={\tiny unitL}] {
            $$\mprset{flushleft}
            \inferrule* [right={\tiny cut}] {
              {
                \begin{array}{cc}
                  \pi & \Pi_2 \\
                  {<<I |-c X>>} & {<<P1, X, P2 |-c Y>>}
                \end{array}
              }
            }{<<P1, I, P2 |-c Y>>}
          }{<<UnitT, P1, I, P2 |-c Y>>}
        \end{math}
      \end{center}

    \item \ElledruleTXXunitLName / $\cat{L}$-sequent:
      \begin{center}
        \scriptsize
        $\Pi_1$:
        \begin{math}
          $$\mprset{flushleft}
          \inferrule* [right={\tiny unitL}] {
            {
              \begin{array}{c}
                \pi \\
                {<<I |-c X>>}
              \end{array}
            }
          }{<<UnitT, X |-c X>>}
        \end{math}
        \qquad\qquad
        \begin{math}
          \begin{array}{c}
            \Pi_2 \\
            {<<G1, X, G2 |-l A>>}
          \end{array}
        \end{math}
      \end{center}
      By assumption, $c(\Pi_1),c(\Pi_2)\leq |X|$. Similar as above, the proof $\Pi$ can be
      constructed as follows with $c(\Pi)\leq |X|$.
      \begin{center}
        \scriptsize
        \begin{math}
          $$\mprset{flushleft}
          \inferrule* [right={\tiny unitL1}] {
            $$\mprset{flushleft}
            \inferrule* [right={\tiny cut1}] {
              {
                \begin{array}{cc}
                  \pi & \Pi_2 \\
                  {<<I |-c X>>} & {<<G1, X, G2 |-l A>>}
                \end{array}
              }
            }{<<G1, I, G2 |-l A>>}
          }{<<UnitT, G1, I, G2 |-l A>>}
        \end{math}
      \end{center}

    \item \ElledruleSXXunitLOneName / $\cat{L}$-sequent:
      \begin{center}
        \scriptsize
        $\Pi_1$:
        \begin{math}
          $$\mprset{flushleft}
          \inferrule* [right={\tiny unitL1}] {
            {
              \begin{array}{c}
                \pi \\
                {<<D |-l A>>}
              \end{array}
            }
          }{<<UnitT, D |-l A>>}
        \end{math}
        \qquad\qquad
        \begin{math}
          \begin{array}{c}
            \Pi_2 \\
            {<<G1, A, G2 |-l B>>}
          \end{array}
        \end{math}
      \end{center}
      By assumption, $c(\Pi_1),c(\Pi_2)\leq |A|$. By induction, there is a proof $\Pi'$ from
      $\pi$ and $\Pi_2$ for sequent $<<G1, D, G2 |-l B>>$ s.t. $c(\Pi')\leq |A|$. Therefore,
      the proof $\Pi$ can be constructed as follows, and $c(\Pi)=c(\Pi')\leq |A|$.
      \begin{center}
        \scriptsize
        \begin{math}
          $$\mprset{flushleft}
          \inferrule* [right={\tiny unitL1}] {
            $$\mprset{flushleft}
            \inferrule* [right={\tiny cut2}] {
              {
                \begin{array}{cc}
                  \pi & \Pi_2 \\
                  {<<D |-l A>>} & {<<G1, A, G2 |-l B>>}
                \end{array}
              }
            }{<<G1, D, G2 |-l B>>}
          }{<<UnitT, G1, D, G2 |-l B>>}
        \end{math}
      \end{center}

    \item \ElledruleSXXunitLTwoName / $\cat{L}$-sequent:
      \begin{center}
        \scriptsize
        $\Pi_1$:
        \begin{math}
          $$\mprset{flushleft}
          \inferrule* [right={\tiny unitL2}] {
            {
              \begin{array}{c}
                \pi \\
                {<<D |-l A>>}
              \end{array}
            }
          }{<<UnitS, D |-l A>>}
        \end{math}
        \qquad\qquad
        \begin{math}
          \begin{array}{c}
            \Pi_2 \\
            {<<G1, A, G2 |-l B>>}
          \end{array}
        \end{math}
      \end{center}
      By assumption, $c(\Pi_1),c(\Pi_2)\leq |A|$. By induction, there is a proof $\Pi'$ from
      $\pi$ and $\Pi_2$ for sequent $<<G1, D, G2 |-l B>>$ s.t. $c(\Pi')\leq |A|$. Therefore,
      the proof $\Pi$ can be constructed as follows, and $c(\Pi)=c(\Pi')\leq |A|$.
      \begin{center}
        \scriptsize
        \begin{math}
          $$\mprset{flushleft}
          \inferrule* [right={\tiny unitL2}] {
            $$\mprset{flushleft}
            \inferrule* [right={\tiny cut2}] {
              {
                \begin{array}{cc}
                  \pi & \Pi_2 \\
                  {<<D |-l A>>} & {<<G1, A, G2 |-l B>>}
                \end{array}
              }
            }{<<G1, D, G2 |-l B>>}
          }{<<UnitS, G1, D, G2 |-l B>>}
        \end{math}
      \end{center}

    \item \ElledruleTXXtenLName / $\cat{C}$-sequent:
      \begin{center}
        \scriptsize
        $\Pi_1$:
        \begin{math}
          $$\mprset{flushleft}
          \inferrule* [right={\tiny tenL}] {
            {
              \begin{array}{c}
                \pi \\
                {<<I1, X1, X2, I2 |-c Y>>}
              \end{array}
            }
          }{<<I1, X1 (x) X2, I2 |-c Y>>}
        \end{math}
        \qquad\qquad
        \begin{math}
          \begin{array}{c}
            \Pi_2 \\
            {<<P1, Y, P2 |-c Z>>}
          \end{array}
        \end{math}
      \end{center}
      By assumption, $c(\Pi_1),c(\Pi_2)\leq |Y|$. By induction, there is a proof $\Pi'$ from
      $\pi$ and $\Pi_2$ for $<<P1, I1, X1, X2, I2, P2 |-c Z>>$ s.t. $c(\Pi')\leq |Y|$.
      Therefore, the proof $\Pi$ can be constructed as follows with $c(\Pi)\leq |Y|$.
      \begin{center}
        \scriptsize
        \begin{math}
          $$\mprset{flushleft}
          \inferrule* [right={\tiny tenL}] {
            $$\mprset{flushleft}
            \inferrule* [right={\tiny cut}] {
              {
                \begin{array}{cc}
                  \pi & \Pi_2 \\
                  {<<I1, X1, X2, I2 |-c Y>>} & {<<P1, Y, P2 |-c Z>>}
                \end{array}
              }
            }{<<P1, I1, X1, X2, I2, P2 |-c Z>>}
          }{<<P1, I1, X1 (x) X2, I2, P2 |-c Z>>}
        \end{math}
      \end{center}

    \item \ElledruleTXXtenLName / $\cat{L}$-sequent:
      \begin{center}
        \scriptsize
        $\Pi_1$:
        \begin{math}
          $$\mprset{flushleft}
          \inferrule* [right={\tiny tenL}] {
            {
              \begin{array}{c}
                \pi \\
                {<<I1, X1, X2, I2 |-c Y>>}
              \end{array}
            }
          }{<<I1, X1 (x) X2, I2 |-c Y>>}
        \end{math}
        \qquad\qquad
        \begin{math}
          \begin{array}{c}
            \Pi_2 \\
            {<<G1, Y, G2 |-l A>>}
          \end{array}
        \end{math}
      \end{center}
      By assumption, $c(\Pi_1),c(\Pi_2)\leq |Y|$. By induction, there is a proof $\Pi'$ from
      $\pi$ and $\Pi_2$ for $<<G1, I1, X1, X2, I2, G2 |-l A>>$ s.t. $c(\Pi')\leq |Y|$.
      Therefore, the proof $\Pi$ can be constructed as follows with $c(\Pi)\leq |Y|$.
      \begin{center}
        \scriptsize
        \begin{math}
          $$\mprset{flushleft}
          \inferrule* [right={\tiny tenL1}] {
            $$\mprset{flushleft}
            \inferrule* [right={\tiny cut1}] {
              {
                \begin{array}{cc}
                  \pi & \Pi_2 \\
                  {<<I1, X1, X2, I2 |-c Y>>} & {<<G1, Y, G2 |-l A>>}
                \end{array}
              }
            }{<<G1, I1, X1, X2, I2, G2 |-l A>>}
          }{<<G1, I1, X1 (x) X2, I2, G2 |-l A>>}
        \end{math}
      \end{center}

    \item \ElledruleSXXtenLOneName / $\cat{L}$-sequent:
      \begin{center}
        \scriptsize
        $\Pi_1$:
        \begin{math}
          $$\mprset{flushleft}
          \inferrule* [right={\tiny tenL}] {
            {
              \begin{array}{c}
                \pi \\
                {<<G1, X, Y, G2 |-l A>>}
              \end{array}
            }
          }{<<G1, X (x) Y, G2 |-l A>>}
        \end{math}
        \qquad\qquad
        \begin{math}
          \begin{array}{c}
            \Pi_2 \\
            {<<D1, A, D2 |-l B>>}
          \end{array}
        \end{math}
      \end{center}
      By assumption, $c(\Pi_1),c(\Pi_2)\leq |A|$. By induction, there is a proof $\Pi'$ from
      $\pi$ and $\Pi_2$ for $<<D1, X, Y, G2, D2 |-l B>>$ s.t. $c(\Pi')\leq |A|$.
      Therefore, the proof $\Pi$ can be constructed as follows with $c(\Pi)\leq |A|$.
      \begin{center}
        \scriptsize
        \begin{math}
          $$\mprset{flushleft}
          \inferrule* [right={\tiny tenL1}] {
            $$\mprset{flushleft}
            \inferrule* [right={\tiny cut2}] {
              {
                \begin{array}{cc}
                  \pi & \Pi_2 \\
                  {<<G1, X, Y, G2 |-l A>>} & {<<D1, A, D2 |-l B>>}
                \end{array}
              }
            }{<<D1, G1, X, Y, G2, D2 |-l B>>}
          }{<<D1, G1, X (x) Y, G2, D2 |-l B>>}
        \end{math}
      \end{center}

    \item \ElledruleSXXtenLTwoName / $\cat{L}$-sequent:
      \begin{center}
        \scriptsize
        $\Pi_1$:
        \begin{math}
          $$\mprset{flushleft}
          \inferrule* [right={\tiny tenL2}] {
            {
              \begin{array}{c}
                \pi \\
                {<<G1, A1, A2, G2 |-l B>>}
              \end{array}
            }
          }{<<G1, A1 (>) A2, G2 |-l B>>}
        \end{math}
        \qquad\qquad
        \begin{math}
          \begin{array}{c}
            \Pi_2 \\
            {<<D1, B, D2 |-l C>>}
          \end{array}
        \end{math}
      \end{center}
      By assumption, $c(\Pi_1),c(\Pi_2)\leq |B|$. By induction, there is a proof $\Pi'$ from
      $\pi$ and $\Pi_2$ for $<<D1, G1, A1, A2, G2, D2 |-l C>>$ s.t. $c(\Pi')\leq |B|$.
      Therefore, the proof $\Pi$ can be constructed as follows with $c(\Pi)\leq |B|$.
      \begin{center}
        \scriptsize
        \begin{math}
          $$\mprset{flushleft}
          \inferrule* [right={\tiny tenL2}] {
            $$\mprset{flushleft}
            \inferrule* [right={\tiny cut2}] {
              {
                \begin{array}{cc}
                  \pi & \Pi_2 \\
                  {<<G1, A1, A2, G2 |-l B>>} & {<<D1, B, D2 |-l C>>}
                \end{array}
              }
            }{<<D1, G1, A1, A2, G2, D2 |-l C>>}
          }{<<D1, G1, A1 (>) A2, G2, D2 |-l C>>}
        \end{math}
      \end{center}

    \item \ElledruleTXXimpLName / $\cat{C}$-sequent:
      \begin{center}
        \scriptsize
        $\Pi_1$:
        \begin{math}
          $$\mprset{flushleft}
          \inferrule* [right={\tiny impL}] {
            {
              \begin{array}{cc}
                \pi_1 & \pi_2 \\
                {<<I1 |-c X1>>} & {<<I2, X2, I3 |-c Y>>}
              \end{array}
            }
          }{<<I2, X1 -o X2, I1, I3 |-c Y>>}
        \end{math}
        \qquad\qquad
        \begin{math}
          \begin{array}{c}
            \Pi_2 \\
            {<<P1, Y, P2 |-c Z>>}
          \end{array}
        \end{math}
      \end{center}
      By assumption, $c(\Pi_1),c(\Pi_2)\leq |Y|$. By induction, there is a proof $\Pi'$ from
      $\pi_2$ and $\Pi_2$ for $<<P1, I2, X2, I3, P2 |-c Z>>$ s.t. $c(\Pi')\leq |Y|$.
      Therefore, the proof $\Pi$ can be constructed as follows with $c(\Pi)\leq |Y|$.
      \begin{center}
        \scriptsize
        \begin{math}
          $$\mprset{flushleft}
          \inferrule* [right={\tiny impL}] {
            {
              \begin{array}{c}
                \pi_1 \\
                {<<I1 |-c X1>>}
              \end{array}
            }
            $$\mprset{flushleft}
            \inferrule* [right={\tiny cut}] {
              {
                \begin{array}{cc}
                  \pi_2 & \Pi_2 \\
                  {<<I2, X2, I3 |-c Y>>} & {<<P1, Y, P2 |-c Z>>}
                \end{array}
              }
            }{<<P1, I2, X2, I3, P2 |-c Z>>}
          }{<<P1, I2, X1 -o X2, I1, I3, P2 |-c Z>>}
        \end{math}
      \end{center}

    \item \ElledruleTXXimpLName / $\cat{L}$-sequent:
      \begin{center}
        \scriptsize
        $\Pi_1$:
        \begin{math}
          $$\mprset{flushleft}
          \inferrule* [right={\tiny impL}] {
            {
              \begin{array}{cc}
                \pi_1 & \pi_2 \\
                {<<I1 |-c X1>>} & {<<I2, X2, I3 |-c Y>>}
              \end{array}
            }
          }{<<I2, X1 -o X2, I1, I3 |-c Y>>}
        \end{math}
        \qquad\qquad
        \begin{math}
          \begin{array}{c}
            \Pi_2 \\
            {<<G1, Y, G2 |-l A>>}
          \end{array}
        \end{math}
      \end{center}
      By assumption, $c(\Pi_1),c(\Pi_2)\leq |Y|$. By induction, there is a proof $\Pi'$ from
      $\pi_2$ and $\Pi_2$ for $<<G1, I2, X2, I3, G2 |-l A>>$ s.t. $c(\Pi')\leq |Y|$.
      Therefore, the proof $\Pi$ can be constructed as follows with $c(\Pi)\leq |Y|$.
      \begin{center}
        \scriptsize
        \begin{math}
          $$\mprset{flushleft}
          \inferrule* [right={\tiny impL}] {
            {
              \begin{array}{c}
                \pi_1 \\
                {<<I1 |-c X1>>}
              \end{array}
            }
            $$\mprset{flushleft}
            \inferrule* [right={\tiny cut}] {
              {
                \begin{array}{cc}
                  \pi_2 & \Pi_2 \\
                  {<<I2, X2, I3 |-c Y>>} & {<<G1, Y, G2 |-l A>>}
                \end{array}
              }
            }{<<G1, I2, X2, I3, G2 |-l A>>}
          }{<<G1, I2, X1 -o X2, I1, I3, G2 |-l A>>}
        \end{math}
      \end{center}

    \item \ElledruleSXXimprLName / $\cat{L}$-sequent:
      \begin{center}
        \scriptsize
        $\Pi_1$:
        \begin{math}
          $$\mprset{flushleft}
          \inferrule* [right={\tiny impL}] {
            {
              \begin{array}{cc}
                \pi_1 & \pi_2 \\
                {<<G1 |-l A1>>} & {<<G2, A2 |-l B>>}
              \end{array}
            }
          }{<<G2, A1 -> B2, G1 |-l B>>}
        \end{math}
        \qquad\qquad
        \begin{math}
          \begin{array}{c}
            \Pi_2 \\
            {<<D1, B, D2 |-l C>>}
          \end{array}
        \end{math}
      \end{center}
      By assumption, $c(\Pi_1),c(\Pi_2)\leq |B|$. By induction, there is a proof $\Pi'$ from
      $\pi_2$ and $\Pi_2$ for $<<D1, G2, A2, D2 |-l C>>$ s.t. $c(\Pi')\leq |B|$.
      Therefore, the proof $\Pi$ can be constructed as follows with $c(\Pi)\leq |B|$.
      \begin{center}
        \scriptsize
        \begin{math}
          $$\mprset{flushleft}
          \inferrule* [right={\tiny impL}] {
            {
              \begin{array}{c}
                \pi_1 \\
                {<<G1 |-l A1 >>}
              \end{array}
            }
            $$\mprset{flushleft}
            \inferrule* [right={\tiny cut}] {
              {
                \begin{array}{cc}
                  \pi_2 & \Pi_2 \\
                  {<<G2, A2 |-l B>>} & {<<D1, B, D2 |-l C>>}
                \end{array}
              }
            }{<<D1, G2, A2, D2 |-l C>>}
          }{?}
        \end{math}
      \end{center}

    \item \ElledruleSXXFlName / $\cat{L}$-sequent:
      \begin{center}
        \scriptsize
        $\Pi_1$:
        \begin{math}
          $$\mprset{flushleft}
          \inferrule* [right={\tiny FL}] {
            {
              \begin{array}{c}
                \pi_1 \\
                {<<G1, X, G2 |-l A>>}
              \end{array}
            }
          }{<<G1, F X, G2 |-l A>>}
        \end{math}
        \qquad\qquad
        \begin{math}
          \begin{array}{c}
            \Pi_2 \\
            {<<D1, A, D2 |-l B>>}
          \end{array}
        \end{math}
      \end{center}
      By assumption, $c(\Pi_1),c(\Pi_2)\leq |A|$. By induction, there is a proof $\Pi'$ from
      $\pi_2$ and $\Pi_2$ for $<<D1, G1, X, G2, D2 |-l B>>$ s.t. $c(\Pi')\leq |A|$.
      Therefore, the proof $\Pi$ can be constructed as follows with $c(\Pi)\leq |A|$.
      \begin{center}
        \scriptsize
        \begin{math}
          $$\mprset{flushleft}
          \inferrule* [right={\tiny FL}] {
            $$\mprset{flushleft}
            \inferrule* [right={\tiny cut2}] {
              {
                \begin{array}{cc}
                  \pi_2 & \Pi_2 \\
                  {<<G1, X, G2 |-l A>>} & {<<D1, A, D2 |-l B>>}
                \end{array}
              }
            }{<<D1, G1, X, G2, D2 |-l B>>}
          }{<<D1, G1, F X, G2, D2 |-l B>>}
        \end{math}
      \end{center}

    \item \ElledruleSXXGlName / $\cat{L}$-sequent:
      \begin{center}
        \scriptsize
        $\Pi_1$:
        \begin{math}
          $$\mprset{flushleft}
          \inferrule* [right={\tiny GL}] {
            {
              \begin{array}{c}
                \pi_1 \\
                {<<G1, A, G2 |-l B>>}
              \end{array}
            }
          }{<<G1, Gf A, G2 |-l B>>}
        \end{math}
        \qquad\qquad
        \begin{math}
          \begin{array}{c}
            \Pi_2 \\
            {<<D1, B, D2 |-l C>>}
          \end{array}
        \end{math}
      \end{center}
      By assumption, $c(\Pi_1),c(\Pi_2)\leq |B|$. By induction, there is a proof $\Pi'$ from
      $\pi_2$ and $\Pi_2$ for $<<D1, G1, A, G2, D2 |-l C>>$ s.t. $c(\Pi')\leq |B|$.
      Therefore, the proof $\Pi$ can be constructed as follows with $c(\Pi)\leq |B|$.
      \begin{center}
        \scriptsize
        \begin{math}
          $$\mprset{flushleft}
          \inferrule* [right={\tiny GL}] {
            $$\mprset{flushleft}
            \inferrule* [right={\tiny cut2}] {
              {
                \begin{array}{cc}
                  \pi_2 & \Pi_2 \\
                  {<<G1, A, G2 |-l B>>} & {<<D1, B, D2 |-l C>>}
                \end{array}
              }
            }{<<D1, G1, A, G2, D2 |-l C>>}
          }{<<D1, G1, Gf A, G2, D2 |-l C>>}
        \end{math}
      \end{center}

    \end{itemize}

  \item The cut formula is a minor formula of the last rule in $\Pi_2$. 

  \item $\Pi_2$ is an axiom on the cut formula. The case is trivial. The proof $\Pi$ is the
        same as $\Pi_1$.

  \end{enumerate}
\end{proof}

\begin{lemma}
  \label{lem:less-cut-rank}
  Let $\Pi$ be a proof of a sequent $<<I |-c X>>$ or $<<G |-l A>>$ s.t. $c(\Pi)>0$. Then there
  is a proof $\Pi'$ of the same sequent with $c(\Pi')<c(\Pi)$.
\end{lemma}
\begin{proof}
  We prove the lemma by induction on $d(\Pi)$. We denote the proof $\Pi$ by $\pi+r$, where $r$
  is the last inference of $\Pi$ and $\pi$ denotes the rest of the proof. If $r$ is not a cut,
  then by induction hypothesis on $\pi$, there is a proof $\pi'$ s.t. $c(\pi')>c(\pi)$ and
  $\Pi'=\pi'+r$. Otherwise, we assume $r$ is a cut on a formula $X$. If $c(\Pi)>|X|+1$, then
  there is a cut on $|Y|$ in $\pi$ with $|Y|>|X|$. So we can apply the induction hypothesis
  on $\pi$ to get $\Pi'$ with $c(\Pi')<c(\Pi)$. The last case to consider is when
  $c(\Pi)=|X|+1$ (note that $c(\Pi)$ cannot be less than $|X|+1$). In this case, $\Pi$ is in
  the form of
  \begin{center}
    \scriptsize
    \begin{math}
      $$\mprset{flushleft}
      \inferrule* [right={\tiny cut}] {
        {
          \begin{array}{cc}
            \Pi_1 & \Pi_2 \\
            {<<I |-c X>>} & {<<P1, X, P2 |-c Y>>}
          \end{array}
        }
      }{<<P1, I, P2 |-c Y>>}
    \end{math}
    \qquad\qquad
    or,
    \begin{math}
      $$\mprset{flushleft}
      \inferrule* [right={\tiny cut1}] {
        {
          \begin{array}{cc}
            \Pi_1 & \Pi_2 \\
            {<<I |-c X>>} & {<<G1, X, G2 |-l A>>}
          \end{array}
        }
      }{<<G1, I, G2 |-l A>>}
    \end{math}
  \end{center}
  By assumption, $c(\Pi_1),c(\Pi_2)\leq |X|+1$. By induction, we can construct $c(\Pi_1')$
  proving $<<I |-c X>>$ and $c(\Pi_2')$ proving $<<P1, X, P2 |-c Y>>$ (or
  $<<G1, X, G2 |-l A>>$) with $c(\Pi_1'),c(\Pi_2')\leq |X|$. Then by
  Lemma~\ref{lem:cut-reduction}, we can construct $\Pi'$ proving $<<P1, I, P2 |-c Y>>$ (or
  $<<G1, I, G2 |-l A>>$) with $c(\Pi')\leq |X|$. 

  The case where the last inference is a cut on a formula $A$ is similar as when it is a cut
  on $X$.

\end{proof}

\begin{theorem}[Cut Elimination]
  Let $\Pi$ be a proof of a sequent $<<I |-c X>>$ or $<<G |-l A>>$ s.t. $c(\Pi)>0$. Then there
  is an algorithm which yields a cut-free proof $\Pi'$ of the same sequent.
\end{theorem}
\begin{proof}
  This follows immediately by induction on $c(\Pi)$ and Lemma~\ref{lem:less-cut-rank}.
\end{proof}



%%%%%%%%%%%%%%%%%%%%%%%%%%%%%%%%%%%%%%%%%%%%%%%%%%
\subsection{Natural Deduction}
\label{subsec:elle-nd}

The term assignment for natural deduction of the commutative part of the model, i.e. the SMCC
of the adjunction, is defined in Figure~\ref{fig:elle-nd-smcc}. And the term assignme for the
non-commutative part, i.e. the Lambek category of the adjunction, is defined in
Figure~\ref{fig:elle-nd-lambek}.

\begin{figure}[!h]
  \scriptsize
  \begin{mdframed}
    \begin{mathpar}
      \NDdruleTXXid{} \qquad\qquad \NDdruleTXXunitI{} \qquad\qquad \NDdruleTXXunitE{} \\
      \NDdruleTXXtenI{} \qquad\qquad \NDdruleTXXtenE{} \\
      \NDdruleTXXimpI{} \qquad\qquad \NDdruleTXXimpE{} \qquad\qquad \NDdruleTXXGI{} \\
      \NDdruleSXXbeta{}
    \end{mathpar}
  \end{mdframed}
\caption{Natural Deduction: Commutative Part}
\label{fig:elle-nd-smcc}
\end{figure}

\begin{figure}[!h]
 \scriptsize
  \begin{mdframed}
    \begin{mathpar}
      \NDdruleSXXid{} \qquad\qquad \NDdruleSXXunitI{} \qquad\qquad \NDdruleSXXunitEOne{} \\
      \NDdruleSXXunitEOne{} \qquad\qquad \NDdruleSXXunitETwo{} \\
      \NDdruleSXXtenI{} \qquad\qquad \NDdruleSXXtenEOne{} \\
      \NDdruleSXXtenETwo{} \qquad\qquad \NDdruleSXXimprI{} \\
      \NDdruleSXXimprE{} \qquad\qquad \NDdruleSXXimplI{} \\
      \NDdruleSXXimplE{} \qquad\qquad \NDdruleSXXGE{} \qquad\qquad \NDdruleSXXFI{} \\
      \NDdruleSXXFE{}
    \end{mathpar}
  \end{mdframed}
\caption{Natural Deduction: Non-Commutative Part}
\label{fig:elle-nd-lambek}
\end{figure}

We could derive exchange comonadically as follows:

\begin{center}
  \tiny
  \begin{math}
  $$\mprset{flushleft}
  \inferrule* [right={\tiny imprI}] {
    $$\mprset{flushleft}
    \inferrule* [right={\tiny tenE2}] {
      $$\mprset{flushleft}
      \inferrule* [right={\tiny id}] {
        \,
      }{[[z : h(F Gf A) (>) F Gf B |-l z : h(F Gf A) (>) F Gf B]]}
        $$\mprset{flushleft}
        \inferrule* [right={\tiny FE}] {
          $$\mprset{flushleft}
          \inferrule* [right={\tiny id}] {
            \,
          }{[[x2 : F Gf A |-l x2 : F Gf A]]}
            $$\mprset{flushleft}
            \inferrule* [right={\tiny FE}] {
              $$\mprset{flushleft}
              \inferrule* [right={\tiny id}] {
                \,
              }{[[y2 : F Gf B |-l y2 : F Gf B]]}
              \inferrule* [right={\tiny beta}] {
                $$\mprset{flushleft}
                \inferrule* [right={\tiny FE}] {
                  $$\mprset{flushleft}
                  \inferrule* [right={\tiny FI}] {
                    $$\mprset{flushleft}
                    \inferrule* [right={\tiny id}] {
                      \,
                    }{[[y0 : Gf B |-c y0 : Gf B]]}
                  }{[[y0 : Gf B |-l F y0 : F Gf B]]}
                  $$\mprset{flushleft}
                  \inferrule* [right={\tiny FI}] {
                    $$\mprset{flushleft}
                    \inferrule* [right={\tiny id}] {
                      \,
                    }{[[x0 : Gf A |-c x0 : Gf A]]}
                  }{[[x0 : Gf A |-l F x0 : F Gf A]]}
                }{[[y0 : Gf B, x0 : Gf A |-l h(F y0) (>) F x0 : h(F Gf B) (>) F Gf A]]}
              }{[[x1 : Gf A, y1 : Gf B |-l ex y1 , x1 with y0 , x0 in (h(F y0) (>) F x0) : h(F Gf B) (>) F Gf A]]}
            }{[[x1 : Gf A , y2 : F Gf B |-l let F y1 : F Gf B be y2 in (ex y1 , x1 with y0 , x0 in (h(F y0) (>) F x0)) : h(F Gf B) (>) F Gf A]]}
          }{[[x2 : F Gf A , y2 : F Gf B |-l let F x1 : F Gf A be x2 in (let F y1 : F Gf B be y2 in (ex y1 , x1 with y0 , x0 in (h(F y0) (>) F x0))) : h(F Gf B) (>) F Gf A]]}
        }{[[z : h(F Gf A) (>) F Gf B |-l let z : h(F Gf A) (>) F Gf B be x2 (>) y2 in (let F x1 : F Gf A be x2 in (let F y1 : F Gf B be y2 in (ex y1 , x1 with y0 , x0 in (h(F y0) (>) F x0)))) : h(F Gf B) (>) F Gf A]]}
      }{[[ . |-l \r z : h(F Gf A) (>) F Gf B.let z : h(F Gf A) (>) F Gf B be x2 (>) y2 in (let F x1 : F Gf A be x2 in (let F y1 : F Gf B be y2 in (ex y1 , x1 with y0 , x0 in (h(F y0) (>) F x0)))) : (h(F Gf A) (>) F Gf B) -> (h(F Gf B) (>) F Gf A)]]}
  \end{math}
\end{center}

We also have the three cut rules derivable in the natural deduction:
(NOTE: Don't know how to prove the third one S\_cut2.)

\begin{figure}[!h]
  \scriptsize
  \begin{mathpar}
    \NDdruleTXXcut{} \qquad\qquad \NDdruleSXXcutOne{} \qquad\qquad \NDdruleSXXcutTwo{}
  \end{mathpar}
\end{figure}



%%%%%%%%%%%%%%%%%%%%%%%%%%%%%%%%%%%%%%%%%%%%%%%%%%
\subsubsection{One Step $\beta$-Reduction}

We define the normalization procedure by considering the following pairs of introduction and
elimination rules:

\begin{itemize}

\item (\NDdruleTXXunitIName, \NDdruleTXXunitEName):
  \begin{center}
    \tiny
    \begin{math}
      $$\mprset{flushleft}
      \inferrule* [right={\tiny unitE}] {
        $$\mprset{flushleft}
        \inferrule* [right={\tiny unitI}] {
          \,
        }{[[. |-c trivT : UnitT]]} \\
         {[[I |-c t : X]]}
      }{[[I |-c let trivT : UnitT be trivT in t : X]]}
    \end{math}
  \end{center}
  normalizes to 
  \begin{center}
    \tiny
    $[[I |-c t : X]]$
  \end{center}

\item (\NDdruleTXXunitIName, \NDdruleSXXunitEOneName):
  \begin{center}
    \tiny
    \begin{math}
     $$\mprset{flushleft}
     \inferrule* [right={\tiny unitE2}] {
       $$\mprset{flushleft}
       \inferrule* [right={\tiny unitI}] {
         \,
        }{[[. |-c trivT : UnitT]]} \\
         {[[D |-l s : A]]}
      }{[[D |-l let trivT : UnitT be trivT in s : A]]}
    \end{math}
  \end{center}
  normalizes to
  \begin{center}
    \tiny
    $[[D |-l s : A]]$
  \end{center}

\item (\NDdruleTXXtenIName, \NDdruleTXXtenEName):
  \begin{center}
    \tiny
    \begin{math}
      $$\mprset{flushleft}
      \inferrule* [right={\tiny tenE}] {
        $$\mprset{flushleft}
        \inferrule* [right={\tiny tenI}] {
          {[[I1 |-c t1 : X]]} \\
          {[[I2 |-c t2 : Y]]}
        }{[[I1, I2 |-c t1 (x) t2 : X (x) Y]]} \\
         {[[P1, x : X, y : Y, P2 |-c t3 : Z]]}
      }{[[P1, I1, I2, P2 |-c let t1 (x) t2 : X (x) Y be x (x) y in t3 : Z]]}
    \end{math}
  \end{center}
  normalizes to
  \begin{center}
    \tiny
    \begin{math}
      $$\mprset{flushleft}
      \inferrule* [right={\tiny cut}] {
        {[[I1 |-c t1 : X]]} \\
        $$\mprset{flushleft}
        \inferrule* [right={\tiny cut}] {
          {[[I2 |-c t2 : Y]]} \\
          {[[P1, x : X, y : Y, P2 |-c t3 : Z]]}
        }{[[P1, x : X, I2, P2 |-c [t2 / y]t3 : Z]]}
      }{[[P1, I1, I2, P2 |-c [t1 / x][t2 / y]t3 : Z]]}
    \end{math}
  \end{center}
  
\item (\NDdruleTXXtenIName, \NDdruleSXXtenEOneName):
  \begin{center}
    \tiny
    \begin{math}
      $$\mprset{flushleft}
      \inferrule* [right={\tiny tenE1}] {
        $$\mprset{flushleft}
        \inferrule* [right={\tiny tenI}] {
          {[[I |-c t1 : X]]} \\
          {[[P |-c t2 : Y]]}
        }{[[I, P |-c t1 (x) t2 : X (x) Y]]} \\
         {[[G, x : X, y : Y, D |-l s : A]]}
      }{[[G, I, P, D |-l let t1 (x) t2 : X (x) Y be x (x) y in s : A]]}
    \end{math}
  \end{center}
  normalizes to
  \begin{center}
    \tiny
    \begin{math}
      $$\mprset{flushleft}
      \inferrule* [right={\tiny cut2}] {
        {[[I |-c t1 : X]]} \\
        $$\mprset{flushleft}
        \inferrule* [right={\tiny cut2}] {
          {[[P |-c t2 : Y]]} \\
          {[[G, x : X, y : Y, D |-l s : A]]}
        }{[[G, x : X, P, D |-l [t2 / y]s : A]]}
      }{[[G, I, P, D |-l [t1 / x][t2 / y]s : A]]}
    \end{math}
  \end{center}
  
\item (\NDdruleTXXimpIName, \NDdruleTXXimpEName):
  \begin{center}
    \tiny
    \begin{math}
      $$\mprset{flushleft}
      \inferrule* [right={\tiny impE}] {
        $$\mprset{flushleft}
        \inferrule* [right={\tiny impI}] {
          {[[I, x : X |-c t1 : Y]]}
        }{[[I |-c \ x : X . t1 : X -o Y]]} \\
         {[[P |-c t2 : X]]}
      }{[[I, P |-c app (\ x : X . t1) t2 : Y]]}
    \end{math}
  \end{center}
  normalizes to
  \begin{center}
    \tiny
    \begin{math}
      $$\mprset{flushleft}
      \inferrule* [right={\tiny cut}] {
        {[[I, x : X |-c t1 : Y]]} \\
        {[[P |-c t2 : X]]}
      }{[[I, P |-c [t2 / x]t1 : Y]]}
    \end{math}
  \end{center}

\item (\NDdruleSXXunitIName, \NDdruleSXXunitETwoName):
  \begin{center}
    \tiny
    \begin{math}
     $$\mprset{flushleft}
     \inferrule* [right={\tiny unitE2}] {
       $$\mprset{flushleft}
       \inferrule* [right={\tiny unitI}] {
         \,
        }{[[. |-l trivS : UnitS]]} \\
         {[[D |-l s : A]]}
      }{[[D |-l let trivS : UnitS be trivS in s : A]]}
    \end{math}
  \end{center}
  normalizes to
  \begin{center}
    \tiny
    $[[D |-l s : A]]$
  \end{center}

\item (\NDdruleSXXtenIName, \NDdruleSXXtenETwoName):
  \begin{center}
    \tiny
    \begin{math}
     $$\mprset{flushleft}
     \inferrule* [right={\tiny tenE2}] {
       $$\mprset{flushleft}
       \inferrule* [right={\tiny tenI}] {
         {[[G1 |-l s1 : A]]} \\
         {[[G2 |-l s2 : B]]}
        }{[[G1, G2 |-l s1 (>) s2 : A (>) B]]} \\
         {[[D1, x : A, y : B, D2 |-l s3 : C]]}
      }{[[D1, G1, G2, D2 |-l let s1 (>) s2 : A (>) B be x (>) y in s3 : C]]}
    \end{math}
  \end{center}
  normalizes to
  \begin{center}
    \tiny
    \begin{math}
      $$\mprset{flushleft}
      \inferrule* [right={\tiny cut2}] {
        {[[G1 |-l s1 : A]]} \\
        $$\mprset{flushleft}
        \inferrule* [right={\tiny cut2}] {
          {[[G2 |-l s2 : B]]} \\
          {[[D1, x : X, y : Y, D2 |-l s3 : C]]}
        }{[[D1, x : X, G2, D2 |-l [s2 / y]s3 : C]]}
      }{[[D1, G1, G2, D2 |-l [s1 / x][s2 / y]s3 : C]]}
    \end{math}
  \end{center}
        
\item (\NDdruleSXXimprIName, \NDdruleSXXimprEName):
  \begin{center}
    \tiny
    \begin{math}
     $$\mprset{flushleft}
     \inferrule* [right={\tiny unitE2}] {
       $$\mprset{flushleft}
       \inferrule* [right={\tiny imprI}] {
         {[[G, x : A |-l s1 : B]]}
        }{[[G |-l \r x : A . s1 : A -> B]]} \\
         {[[D |-l s2 : A]]}
      }{[[G, D |-l appr (\r x : A . s1) s2 : B]]}
    \end{math}
  \end{center}
  normalizes to
  \begin{center}
    \tiny
    \begin{math}
      $$\mprset{flushleft}
      \inferrule* [right={\tiny cut2}] {
        {[[G, x : A |-l s1 : B]]} \\
        {[[D |-l s2 : A]]}
      }{[[G, D |-l [s2 / x]s1 : B]]}
    \end{math}
  \end{center}
        
\item (\NDdruleSXXimplIName, \NDdruleSXXimplEName):
  \begin{center}
    \tiny
    \begin{math}
     $$\mprset{flushleft}
     \inferrule* [right={\tiny unitE2}] {
       $$\mprset{flushleft}
       \inferrule* [right={\tiny implI}] {
         {[[x : A, G |-l s1 : B]]}
        }{[[G |-l \l x : A . s1 : B <- A]]} \\
         {[[D |-l s2 : A]]}
      }{[[D, G |-l appl (\l x : A . s1) s2 : B]]}
    \end{math}
  \end{center}
  normalizes to
  \begin{center}
    \tiny
    \begin{math}
      $$\mprset{flushleft}
      \inferrule* [right={\tiny cut2}] {
        {[[x : A, G |-l s1 : B]]} \\
        {[[D |-l s2 : A]]}
      }{[[D, G |-l [s2 / x]s1 : B]]}
    \end{math}
  \end{center}
        
\item (\NDdruleSXXFIName, \NDdruleSXXFEName):
  \begin{center}
    \tiny
    \begin{math}
      $$\mprset{flushleft}
      \inferrule* [right={\tiny FE}] {
        $$\mprset{flushleft}
        \inferrule* [right={\tiny FI}] {
          {[[I |-c y : X]]}
        }{[[I |-l F y : F X]]} \\
         {[[D1, x : X, D2 |-l s : A]]}
      }{[[D1, I, D2 |-l let F x : F X be F y in s : A]]}
    \end{math}
  \end{center}
  normalizes to
  \begin{center}
    \tiny
    \begin{math}
      $$\mprset{flushleft}
      \inferrule* [right={\tiny cut1}] {
        {[[I |-c y : X]]} \\
        {[[D1, x : X, D2 |-l s : A]]}
      }{[[D1, I, D2 |-l [y / x]s : A]]}
    \end{math}
  \end{center}

\item (\NDdruleTXXGIName, \NDdruleSXXGEName):
  \begin{center}
    \tiny
    \begin{math}
      $$\mprset{flushleft}
      \inferrule* [right={\tiny GE}] {
        $$\mprset{flushleft}
        \inferrule* [right={\tiny GI}] {
          {[[I |-l s : A]]}
        }{[[I |-c Gf s : Gf A]]}
      }{[[I |-l derelict (Gf s) : A]]}
    \end{math}
  \end{center}
  normalizes to
  \begin{center}
    \tiny
    $[[I |-l s : A]]$
  \end{center}

\end{itemize}

\begin{theorem}[Normalization]
  For a cut-free deduction $\Pi$, there is a deduction which is in normal form.
\end{theorem}
\begin{proof}
  By induction on the structure of $\Pi$.
\end{proof}



  \begin{center}
    \tiny
    $[[Theta |-c X]]$
  \end{center}



%%%%%%%%%%%%%%%%%%%%%%%%%%%%%%%%%%%%%%%%%%%%%%%%%%
\subsubsection{Commuting Conversions}

\begin{itemize}

\item Commutation of $\mathrm{UnitT}_E$:
  \begin{itemize}

  \item (\NDdruleTXXunitEName, \NDdruleTXXunitEName):
    \begin{center}
      \tiny
      \begin{math}
        $$\mprset{flushleft}
        \inferrule* [right={\tiny unitE}] {
          $$\mprset{flushleft}
          \inferrule* [right={\tiny unitE}] {
            {[[I1 |-c t1 : UnitT]]} \\
            {[[I2 |-c t2 : UnitT]]}
          }{[[I2, I1 |-c let t2 : UnitT be trivT in t1 : UnitT]]} \\
          {[[I3 |-c t3 : X]]}
        }{[[I2, I1, I3 |-c let (let t2 : UnitT be trivT in t1) : UnitT be trivT in t3 : X]]}
      \end{math}
    \end{center}
    commutes to
    \begin{center}
      \tiny
      \begin{math}
        $$\mprset{flushleft}
        \inferrule* [right={\tiny unitE}] {
          $$\mprset{flushleft}
          \inferrule* [right={\tiny unitE}] {
            {[[I1 |-c t1 : UnitT]]} \\
            {[[I3 |-c t3 : X]]}
          }{[[I1, I3 |-c let t1 : UnitT be trivT in t3 : X]]} \\
           {[[I2 |-c t2 : UnitT]]}
        }{[[I2, I1, I3 |-c let t2 : UnitT be trivT in (let t1 : UnitT be trivT in t3) : X]]}
      \end{math}
    \end{center}

  \item (\NDdruleTXXunitEName, \NDdruleTXXtenEName) need multiple exchanges at the end:
    \begin{center}
      \tiny
      \begin{math}
        $$\mprset{flushleft}
        \inferrule* [right={\tiny tenE}] {
          $$\mprset{flushleft}
          \inferrule* [right={\tiny unitE}] {
            {[[I1 |-c t1 : X (x) Y]]} \\
            {[[I2 |-c t2 : UnitT]]}
          }{[[I2, I1 |-c let t2 : UnitT be trivT in t1 : X (x) Y]]} \\
           {[[P1, x : X, y : Y, P2 |-c t3 : Z]]}
        }{[[P1, I2, I1, P2 |-c let (let t2 : UnitT be trivT in t1) : X (x) Y be x (x) y in t3 : Z]]}
      \end{math}
    \end{center}
    commutes to
    \begin{center}
      \tiny
      \begin{math}
        $$\mprset{flushleft}
        \inferrule* [right={\tiny unitE}] {
          $$\mprset{flushleft}
          \inferrule* [right={\tiny tenE}] {
            {[[I1 |-c t1 : X (x) Y]]} \\
            {[[P1, x : X, y : Y, P2 |-c t3 : Z]]}
          }{[[P1, I1, P2 |-c let t1 : X (x) Y be x (x) y in t3 : Z]]} \\
           {[[I2 |-c t2 : UnitT]]}
        }{[[I2, P1, I1, P2 |-c let t2 : UnitT be trivT in (let t1 : X (x) Y be x (x) y in t3) : Z]]}
      \end{math}
    \end{center}

  \item (\NDdruleTXXunitEName, \NDdruleTXXimpEName):
    \begin{center}
      \tiny
      \begin{math}
        $$\mprset{flushleft}
        \inferrule* [right={\tiny unitE}] {
          $$\mprset{flushleft}
          \inferrule* [right={\tiny tenE}] {
            {[[I1 |-c t1 : X -o Y]]} \\
            {[[I2 |-c t2 : UnitT]]}
          }{[[I2, I1 |-c let t2 : UnitT be trivT in t1 : X -o Y]]} \\
           {[[I3 |-c t3 : X]]}
        }{[[I2, I1, I3 |-c app (let t2 : UnitT be trivT in t1) t3 : Y]]}
      \end{math}
    \end{center}
    commutes to
    \begin{center}
      \tiny
      \begin{math}
        $$\mprset{flushleft}
        \inferrule* [right={\tiny tenE}] {
          $$\mprset{flushleft}
          \inferrule* [right={\tiny unitE}] {
            {[[I1 |-c t1 : X -o Y]]} \\
            {[[I3 |-c t3 : X]]}
          }{[[I1, I3 |-c app t1 t3 : Y]]} \\
           {[[I2 |-c t2 : UnitT]]}
        }{[[I2, I1, I3 |-c let t2 : UnitT be trivT in (app t1 t3) : Y]]}
      \end{math}
    \end{center}
  \end{itemize}


\item Commutation of $\otimes_E$:

  \begin{itemize}
  \item (\NDdruleTXXtenEName, \NDdruleTXXunitEName):
    \begin{center}
      \tiny
      \begin{math}
        $$\mprset{flushleft}
        \inferrule* [right={\tiny unitE}] {
          $$\mprset{flushleft}
          \inferrule* [right={\tiny tenE}] {
            {[[I1, x : X, y : Y, I2 |-c t1 : UnitT]]} \\
            {[[P1 |-c t2 : X (x) Y]]}
          }{[[I1, P1, I2 |-c let t2 : X (x) Y be x (x) y in t1 : UnitT]]} \\
           {[[P2 |-c t3 : Z]]}
        }{[[I1, P1, I2, P2 |-c let (let t2 : X (x) Y be x (x) y in t1) : UnitT be trivT in t3 : Z]]}
      \end{math}
    \end{center}
    commutes to
    \begin{center}
      \tiny
      \begin{math}
        $$\mprset{flushleft}
        \inferrule* [right={\tiny tenE}] {
          $$\mprset{flushleft}
          \inferrule* [right={\tiny unitE}] {
            {[[I1, x : X, y : Y, I2 |-c t1 : UnitT]]} \\
            {[[P2 |-c t3 : Z]]}
          }{[[I1, x : X, y : Y, I2, P2 |-c let t1 : UnitT be trivT in t3 : Z]]} \\
           {[[P1 |-c t2 : X (x) Y]]}
        }{[[I1, P1, I2, P2 |-c let t2 : X (x) Y be x (x) y in (let t1 : UnitT be trivT in t3) : Z]]}
      \end{math}
    \end{center}

  \item (\NDdruleTXXtenEName, \NDdruleTXXtenEName):
    \begin{center}
      \tiny
      \begin{math}
        $$\mprset{flushleft}
        \inferrule* [right={\tiny tenE}] {
          $$\mprset{flushleft}
          \inferrule* [right={\tiny tenE}] {
            {[[I1, x : X2, y : Y2, I2 |-c t1 : X1 (x) Y1]]} \\
            {[[P |-c t2 : X2 (x) Y2]]}
          }{[[I1, P, I2 |-c let t2 : X2 (x) Y2 be x (x) y in t1 : X1 (x) Y1]]} \\
           {[[P1, w : X1, z : Y1, P2 |-c t3 : Z]]}
        }{[[P1, I1, P, I2, P2 |-c let (let t2 : X2 (x) Y2 be x (x) y in t1) : X1 (x) Y1 be w (x) z in t3 : Z]]}
      \end{math}
    \end{center}
    commutes to
    \begin{center}
      \tiny
      \begin{math}
        $$\mprset{flushleft}
        \inferrule* [right={\tiny tenE}] {
          $$\mprset{flushleft}
          \inferrule* [right={\tiny tenE}] {
            {[[I1, x : X2, y : Y2, I2 |-c t1 : X1 (x) Y1]]} \\
            {[[P1, w : X1, z : Y1, P2 |-c t3 : Z]]}
          }{[[P1, I1, x : X2, y : Y2, I2, P2 |-c let t1 : X1 (x) Y1 be w (x) z in t3 : Z]]} \\
           {[[P |-c t2 : X2 (x) Y2]]}
        }{[[P1, I1, P, I2, P2 |-c let t2 : X2 (x) Y2 be x (x) y in (let t1 : X1 (x) Y1 be w (x) z in t3) : Z]]}
      \end{math}
    \end{center}

  \item (\NDdruleTXXtenEName, \NDdruleTXXimpEName):
    \begin{center}
      \tiny
      \begin{math}
        $$\mprset{flushleft}
        \inferrule* [right={\tiny tenE}] {
          $$\mprset{flushleft}
          \inferrule* [right={\tiny impE}] {
            {[[I1, x : X2, y : Y2, I2 |-c t1 : X1 -o Y1]]} \\
            {[[P1 |-c t2 : X2 (x) Y2]]}
          }{[[I1, P1, I2 |-c let t2 : X2 (x) Y2 be x (x) y in t1 : X1 -o Y1]]} \\
           {[[P2 |-c t3 : X1]]}
        }{[[I1, P1, I2, P2 |-c app (let t2 : X2 (x) Y2 be x (x) y in t1) t3 : Y1]]}
      \end{math}
    \end{center}
    commutes to
    \begin{center}
      \tiny
      \begin{math}
        $$\mprset{flushleft}
        \inferrule* [right={\tiny impE}] {
          $$\mprset{flushleft}
          \inferrule* [right={\tiny tenE}] {
            {[[I1, x : X2, y : Y2, I2 |-c t1 : X1 -o Y1]]} \\
            {[[P2 |-c t3 : X1]]}
          }{[[I1, x : X2, y : Y2, I2, P2 |-c app t1 t3 : Y1]]} \\
           {[[P1 |-c t2 : X2 (x) Y2]]}
        }{[[I1, P1, I2, P2 |-c let t2 : X2 (x) Y2 be x (x) y in (app t1 t3) : Y1]]}
      \end{math}
    \end{center}

  \end{itemize}

\item Commutation of $\multimap_E$:

  \begin{itemize}

  \item (\NDdruleTXXimpEName, \NDdruleTXXunitEName):
    \begin{center}
      \tiny
      \begin{math}
        $$\mprset{flushleft}
        \inferrule* [right={\tiny tenE}] {
          $$\mprset{flushleft}
          \inferrule* [right={\tiny impE}] {
            {[[I1 |-c t1 : UnitT]]} \\
            {[[I2 |-c t2 : UnitT -o UnitT]]}
          }{[[I2, I1 |-c app t2 t1 : UnitT]]} \\
           {[[I3 |-c t3 : UnitT]]}
        }{[[I2, I1, I3 |-c let (app t2 t1) : UnitT be trivT in t3 : UnitT]]}
      \end{math}
    \end{center}
    commutes to
    \begin{center}
      \tiny
      \begin{math}
        $$\mprset{flushleft}
        \inferrule* [right={\tiny impE}] {
          $$\mprset{flushleft}
          \inferrule* [right={\tiny tenE}] {
            {[[I1 |-c t1 : UnitT]]} \\
            {[[I3 |-c t3 : UnitT]]}
          }{[[I1, I3 |-c let t1 : UnitT be trivT in t3 : UnitT]]}
           {[[I2 |-c t2 : UnitT -o UnitT]]}
        }{[[I2, I1, I3 |-c app t2 (let t1 : UnitT be trivT in t3) : UnitT]]}
      \end{math}
    \end{center}
  \item (\NDdruleTXXimpEName, \NDdruleTXXtenEName): ?
  \item (\NDdruleTXXimpEName, \NDdruleTXXimpEName): ?
  \end{itemize}

\item Commutation of $\tri_E$:

  \begin{itemize}

  \item (\NDdruleSXXunitETwoName, \NDdruleSXXunitETwoName):
    \begin{center}
      \tiny
      \begin{math}
        $$\mprset{flushleft}
        \inferrule* [right={\tiny unitE}] {
          $$\mprset{flushleft}
          \inferrule* [right={\tiny unitE}] {
            {[[G1 |-l s1 : UnitS]]} \\
            {[[G2 |-l s2 : UnitS]]}
          }{[[G2, G1 |-l let s2 : UnitS be trivS in s1 : UnitS]]} \\
           {[[G3 |-l s3 : A]]}
        }{[[G2, G1, G3 |-l let (let s2 : UnitS be trivS in s1) : UnitS be trivS in s3 : A]]}
      \end{math}
    \end{center}
    commutes to
    \begin{center}
      \tiny
      \begin{math}
        $$\mprset{flushleft}
        \inferrule* [right={\tiny unitE}] {
          $$\mprset{flushleft}
          \inferrule* [right={\tiny unitE}] {
            {[[G1 |-l s1 : UnitS]]} \\
            {[[G3 |-l s3 : A]]}
          }{[[G1, G3 |-l let s1 : UnitS be trivS in s3 : A]]} \\
           {[[G2 |-l s2 : UnitS]]}
        }{[[G2, G1, G3 |-l let s2 : UnitS be trivS in (let s1 : UnitS be trivS in s3) : A]]}
      \end{math}
    \end{center}

  \item (\NDdruleSXXunitETwoName, \NDdruleSXXtenETwoName): Does NOT commute
    \begin{center}
      \tiny
      \begin{math}
        $$\mprset{flushleft}
        \inferrule* [right={\tiny tenE2}] {
          $$\mprset{flushleft}
          \inferrule* [right={\tiny unitE}] {
            {[[G1 |-l s1 : A (>) B]]} \\
            {[[G2 |-l s2 : UnitS]]}
          }{[[G2, G1 |-l let s2 : UnitS be trivS in s1 : A (>) B]]} \\
           {[[D1, x : A, y : B, D2 |-l s3 : C]]}
        }{[[D1, G2, G1, D2 |-l let (let s2 : UnitS be trivS in s1) : A (>) B be x (>) y in s3 : C]]}
      \end{math}
    \end{center}
    commutes to
    \begin{center}
      \tiny
      \begin{math}
        $$\mprset{flushleft}
        \inferrule* [right={\tiny unitE}] {
          $$\mprset{flushleft}
          \inferrule* [right={\tiny tenE2}] {
            {[[G1 |-l s1 : A (>) B]]} \\
            {[[D1, x : A, y : B, D2 |-l s3 : C]]}
          }{[[D1, G1, D2 |-l let s1 : A (>) B be x (>) y in s3 : C]]} \\
           {[[G2 |-l s2 : UnitS]]}
        }{[[G2, D1, G1, D2 |-l let s2 : UnitS be trivS in (let s1 : A (>) B be x (>) y in s3) : C]]}
      \end{math}
    \end{center}

  \item (\NDdruleSXXunitETwoName, \NDdruleSXXimprEName):
    \begin{center}
      \tiny
      \begin{math}
        $$\mprset{flushleft}
        \inferrule* [right={\tiny imprE}] {
          $$\mprset{flushleft}
          \inferrule* [right={\tiny unitE}] {
            {[[G1 |-l s1 : A -> B]]} \\
            {[[G2 |-l s2 : UnitS]]}
          }{[[G2, G1 |-l let s2 : UnitS be trivS in s1 : A -> B]]} \\
           {[[G3 |-l s3 : A]]}
        }{[[G2, G1, G3 |-l appr (let s2 : UnitS be trivS in s1) s3 : B]]}
      \end{math}
    \end{center}
    commutes to
    \begin{center}
      \tiny
      \begin{math}
        $$\mprset{flushleft}
        \inferrule* [right={\tiny unitE}] {
          $$\mprset{flushleft}
          \inferrule* [right={\tiny imprE}] {
            {[[G1 |-l s1 : A -> B]]} \\
            {[[G3 |-l s3 : A]]}
          }{[[G1, G3 |-l appr s1 s3 : B]]} \\
           {[[G2 |-l s2 : UnitS]]}
        }{[[G2, G1, G3 |-l let s2 : UnitS be trivS in (appr s1 s3) : B]]}
      \end{math}
    \end{center}

  \item (\NDdruleSXXunitETwoName, \NDdruleSXXimplEName): Does NOT commute.
    \begin{center}
      \tiny
      \begin{math}
        $$\mprset{flushleft}
        \inferrule* [right={\tiny imprE}] {
          $$\mprset{flushleft}
          \inferrule* [right={\tiny unitE}] {
            {[[G1 |-l s1 : B <- A]]} \\
            {[[G2 |-l s2 : UnitS]]}
          }{[[G2, G1 |-l let s2 : UnitS be trivS in s1 : B <- A]]} \\
           {[[G3 |-l s3 : A]]}
        }{[[G3, G2, G1 |-l appl (let s2 : UnitS be trivS in s1) s3 : B]]}
      \end{math}
    \end{center}
    commutes to
    \begin{center}
      \tiny
      \begin{math}
        $$\mprset{flushleft}
        \inferrule* [right={\tiny unitE}] {
          $$\mprset{flushleft}
          \inferrule* [right={\tiny imprE}] {
            {[[G1 |-l s1 : B <- A]]} \\
            {[[G3 |-l s3 : A]]}
          }{[[G3, G1 |-l appl s1 s3 : B]]} \\
           {[[G2 |-l s2 : UnitS]]}
        }{[[G2, G3, G1 |-l let s2 : UnitS be trivS in (appl s1 s3) : B]]}
      \end{math}
    \end{center}

  \item (\NDdruleSXXtenETwoName, \NDdruleSXXunitETwoName):
    \begin{center}
      \tiny
      \begin{math}
        $$\mprset{flushleft}
        \inferrule* [right={\tiny unitE2}] {
          $$\mprset{flushleft}
          \inferrule* [right={\tiny tenE2}] {
            {[[G1, x : A, y : B, G2 |-l s1 : UnitS]]} \\
            {[[D1 |-l s2 : A (>) B]]}
          }{[[G1, D1, G2 |-l let s2 : A (>) B be x (>) y in s1 : UnitS]]} \\
           {[[D2 |-l s3 : C]]}
        }{[[G1, D1, G2, D2 |-l let (let s2 : A (>) B be x (>) y in s1) : UnitS be trivS in s3 : C]]}
      \end{math}
    \end{center}
    commutes to
    \begin{center}
      \tiny
      \begin{math}
        $$\mprset{flushleft}
        \inferrule* [right={\tiny tenE2}] {
          $$\mprset{flushleft}
          \inferrule* [right={\tiny unitE2}] {
            {[[G1, x : A, y : B, G2 |-l s1 : UnitS]]} \\
            {[[D2 |-l s3 : C]]}
          }{[[G1, x : A, y : B, G2, D2 |-l let s1 : UnitS be trivS in s3 : C]]} \\
           {[[D1 |-l s2 : A (>) B]]}
        }{[[G1, D1, G2, D2 |-l let s2 : A (>) B be x (>) y in (let s1 : UnitS be trivS in s3) : C]]}
      \end{math}
    \end{center}

  \item (\NDdruleSXXtenETwoName, \NDdruleSXXtenETwoName):
    \begin{center}
      \tiny
      \begin{math}
        $$\mprset{flushleft}
        \inferrule* [right={\tiny tenE2}] {
          $$\mprset{flushleft}
          \inferrule* [right={\tiny tenE2}] {
            {[[G1, x : A2, y : B2, G2 |-l s1 : A1 (>) B1]]} \\
            {[[G |-l s2 : A2 (>) B2]]}
          }{[[G1, G, G2 |-l let s2 : A2 (>) B2 be x (>) y in s1 : A1 (>) B1]]} \\
           {[[D1, w : A1, z : B1, D2 |-l s3 : C]]}
        }{[[D1, G1, G, G2, D2 |-l let (let s2 : A2 (>) B2 be x (>) y in s1) : A1 (>) B1 be w (>) z in s3 : C]]}
      \end{math}
    \end{center}
    commutes to
    \begin{center}
      \tiny
      \begin{math}
        $$\mprset{flushleft}
        \inferrule* [right={\tiny tenE2}] {
          $$\mprset{flushleft}
          \inferrule* [right={\tiny tenE2}] {
            {[[G1, x : A2, y : B2, G2 |-l s1 : A1 (>) B1]]} \\
            {[[D1, w : A1, z : B1, D2 |-l s3 : C]]}
          }{[[D1, G1, x : A2, y : B2, G2, D2 |-l let s1 : A1 (>) B1 be w (>) z in s3 : C]]}
            {[[G |-l s2 : A2 (>) B2]]}
        }{[[D1, G1, G, G2, D2 |-l let s2 : A2 (>) B2 be x (>) y in (let s1 : A1 (>) B1 be x (>) z in s3) : C]]}
      \end{math}
    \end{center}

  \item (\NDdruleSXXtenETwoName, \NDdruleSXXimprEName):
    \begin{center}
      \tiny
      \begin{math}
        $$\mprset{flushleft}
        \inferrule* [right={\tiny imprE}] {
          $$\mprset{flushleft}
          \inferrule* [right={\tiny tenE2}] {
            {[[G1, x : A2, y : B2, G2 |-l s1 : A1 -> B1]]} \\
            {[[D1 |-l s2 : A2 (>) B2]]}
          }{[[G1, D1, G2 |-l let s2 : A2 (>) B2 be x (>) y in s1 : A1 -> B1]]} \\
           {[[D2 |-l s3 : A1]]}
        }{[[G1, D1, G2, D2 |-l appr (let s2 : A2 (>) B2 be x (>) y in s1) s3 : B1]]}
      \end{math}
    \end{center}
    commutes to
    \begin{center}
      \tiny
      \begin{math}
        $$\mprset{flushleft}
        \inferrule* [right={\tiny tenE2}] {
          $$\mprset{flushleft}
          \inferrule* [right={\tiny imprE}] {
            {[[G1, x : A2, y : B2, G2 |-l s1 : A1 -> B1]]} \\
            {[[D2 |-l s3 : A1]]}
          }{[[G1, x : A2, y : B2, G2, D2 |-l appr s1 s3 : B1]]} \\
            {[[D1 |-l s2 : A2 (>) B2]]}
        }{[[G1, D1, G2, D2 |-l let s2 : A2 (>) B2 be x (>) y in (appr s1 s3) : B1]]}
      \end{math}
    \end{center}

  \item (\NDdruleSXXtenETwoName, \NDdruleSXXimplEName):
    \begin{center}
      \tiny
      \begin{math}
        $$\mprset{flushleft}
        \inferrule* [right={\tiny imprE}] {
          $$\mprset{flushleft}
          \inferrule* [right={\tiny tenE2}] {
            {[[G1, x : A2, y : B2, G2 |-l s1 : B1 <- A1]]} \\
            {[[D1 |-l s2 : A2 (>) B2]]}
          }{[[G1, D1, G2 |-l let s2 : A2 (>) B2 be x (>) y in s1 : B1 <- A1]]} \\
           {[[D2 |-l s3 : A1]]}
        }{[[D2, G1, D1, G2 |-l appl (let s2 : A2 (>) B2 be x (>) y in s1) s3 : B1]]}
      \end{math}
    \end{center}
    commutes to
    \begin{center}
      \tiny
      \begin{math}
        $$\mprset{flushleft}
        \inferrule* [right={\tiny tenE2}] {
          $$\mprset{flushleft}
          \inferrule* [right={\tiny imprE}] {
            {[[G1, x : A2, y : B2, G2 |-l s1 : B1 <- A1]]} \\
            {[[D2 |-l s3 : A1]]}
          }{[[D2, G1, x : A2, y : B2, G2 |-l appl s1 s3 : B1]]} \\
            {[[D1 |-l s2 : A2 (>) B2]]}
        }{[[D2, G1, D1, G2 |-l let s2 : A2 (>) B2 be x (>) y in (appl s1 s3) : B1]]}
      \end{math}
    \end{center}
  
  \end{itemize}

\item Commutation of $F_E$:
  \begin{itemize}
  \item (\NDdruleSXXFEName, \NDdruleSXXunitETwoName):
    \begin{center}
      \tiny
      \begin{math}
        $$\mprset{flushleft}
        \inferrule* [right={\tiny unitE2}] {
          $$\mprset{flushleft}
          \inferrule* [right={\tiny FE}] {
            {[[G1, x : X, G2 |-l s1 : UnitS]]} \\
            {[[D1 |-l y : F X]]}
          }{[[G1, D1, G2 |-l let F x : F X be y in s1 : UnitS]]} \\
           {[[D2 |-l s2 : A]]}
        }{[[G1, D1, G2, D2 |-l let (let F x : F X be y in s1) : UnitS be trivS in s2 : A]]}
      \end{math}
    \end{center}
    commutes to
    \begin{center}
      \tiny
      \begin{math}
        $$\mprset{flushleft}
        \inferrule* [right={\tiny FE}] {
          $$\mprset{flushleft}
          \inferrule* [right={\tiny unitE2}] {
            {[[G1, x : X, G2 |-l s1 : UnitS]]} \\
            {[[D2 |-l s2 : A]]}
          }{[[G1, x : X, G2, D2 |-l let s1 : UnitS be trivS in s2 : A]]} \\
           {[[D1 |-l y : F X]]}
        }{[[G1, D1, G2, D2 |-l let F x : F X be y in (let s1 : UnitS be trivS in s2) : A]]}
      \end{math}
    \end{center}
  \item (\NDdruleSXXFEName, \NDdruleSXXtenETwoName):
    \begin{center}
      \tiny
      \begin{math}
        $$\mprset{flushleft}
        \inferrule* [right={\tiny tenE2}] {
          $$\mprset{flushleft}
          \inferrule* [right={\tiny FE}] {
            {[[G1, x : X, G2 |-l s1 : A (>) B]]} \\
            {[[D |-l y : F X]]}
          }{[[G1, D, G2 |-l let F x : F X be y in s1 : A (>) B]]} \\
           {[[D1, x : A, y : B, D2 |-l s2 : C]]}
        }{[[D1, G1, D, G2, D2 |-l let (let F x : F X be y in s1) : A (>) B be x (>) y in s2 : C]]}
      \end{math}
    \end{center}
    commutes to
    \begin{center}
      \tiny
      \begin{math}
        $$\mprset{flushleft}
        \inferrule* [right={\tiny FE}] {
          $$\mprset{flushleft}
          \inferrule* [right={\tiny tenE2}] {
            {[[G1, x : X, G2 |-l s1 : A (>) B]]} \\
            {[[D1, x : A, y : B, D2 |-l s2 : C]]}
          }{[[D1, G1, x : X, G2, D2 |-l let s1 : A (>) B be x (>) y in s2 : C]]} \\
           {[[D |-l y : F X]]}
        }{[[D1, G1, D, G2, D2 |-l let F x : F X be y in (let s1 : A (>) B be x (>) y in s2) : C]]}
      \end{math}
    \end{center}
  \item (\NDdruleSXXFEName, \NDdruleSXXimprEName):
    \begin{center}
      \tiny
      \begin{math}
        $$\mprset{flushleft}
        \inferrule* [right={\tiny imprE}] {
          $$\mprset{flushleft}
          \inferrule* [right={\tiny FE}] {
            {[[G1, x : X, G2 |-l s1 : A -> B]]} \\
            {[[D1 |-l y : F X]]}
          }{[[G1, D1, G2 |-l let F x : F X be y in s1 : A -> B]]} \\
           {[[D2 |-l s2 : A]]}
        }{[[G1, D1, G2, D2 |-l appr (let F x : F X be y in s1) s2 : B]]}
      \end{math}
    \end{center}
    commutes to
    \begin{center}
      \tiny
      \begin{math}
        $$\mprset{flushleft}
        \inferrule* [right={\tiny FE}] {
          $$\mprset{flushleft}
          \inferrule* [right={\tiny imprE}] {
            {[[G1, x : X, G2 |-l s1 : A -> B]]} \\
            {[[D2 |-l s2 : A]]}
          }{[[G1, x : X, G2, D2 |-l appr s1 s2 : B]]} \\
           {[[D1 |-l y : F X]]}
        }{[[G1, D1, G2, D2 |-l let F x : F X be y in (appr s1 s2) : B]]}
      \end{math}
    \end{center}
  \item (\NDdruleSXXFEName, \NDdruleSXXimplEName):
    \begin{center}
      \tiny
      \begin{math}
        $$\mprset{flushleft}
        \inferrule* [right={\tiny implE}] {
          $$\mprset{flushleft}
          \inferrule* [right={\tiny FE}] {
            {[[G1, x : X, G2 |-l s1 : A <- B]]} \\
            {[[D1 |-l y : F X]]}
          }{[[G1, D1, G2 |-l let F x : F X be y in s1 : A <- B]]} \\
           {[[D2 |-l s2 : A]]}
        }{[[D2, G1, D1, G2 |-l appl (let F x : F X be y in s1) s2 : B]]}
      \end{math}
    \end{center}
    commutes to
    \begin{center}
      \tiny
      \begin{math}
        $$\mprset{flushleft}
        \inferrule* [right={\tiny FE}] {
          $$\mprset{flushleft}
          \inferrule* [right={\tiny imprE}] {
            {[[G1, x : X, G2 |-l s1 : A <- B]]} \\
            {[[D2 |-l s2 : A]]}
          }{[[D2, G1, x : X, G2 |-l appl s1 s2 : B]]} \\
           {[[D1 |-l y : F X]]}
        }{[[D2, G1, D1, G2 |-l let F x : F X be y in (appl s1 s2) : B]]}
      \end{math}
    \end{center}

  \end{itemize}

\end{itemize}



%%%%%%%%%%%%%%%%%%%%%%%%%%%%%%%%%%%%%%%%%%%%%%%%%%
\subsection{Mappings Between Sequent Calculus and Natural Deduction}

Function $S:ND\rightarrow SE$ maps a proof in the natural deduction to a proof of the same
sequent in the sequent calculus. The function is defined as follows:

\begin{itemize}
\item The axioms map to axioms.
\item Introduction rules map to right rules.
\item Elimination rules map to combinations of left rules with cuts:
  \begin{itemize}
  \item \NDdruleTXXunitEName:
    \begin{center}
      \tiny
      $\NDdruleTXXunitE{}$
    \end{center}
    maps to
    \begin{center}
      \tiny
      \begin{math}
        $$\mprset{flushleft}
        \inferrule* [right={\tiny cut}] {
          {[[I |-c t1 : UnitT]]} \\
          $$\mprset{flushleft}
          \inferrule* [right={\tiny unitL}] {
            {[[P |-c t2 : Y]]}
          }{[[x : UnitT, P |-c let x : UnitT be trivT in t2 : Y]]}
        }{[[I, P |-c [t1 / x](let x : UnitT be trivT in t2) : Y]]}
      \end{math}
    \end{center}
  \item \NDdruleTXXtenEName:
    \begin{center}
      \tiny
      $\NDdruleTXXtenE{}$
    \end{center}
    maps to
    \begin{center}
      \tiny
      \begin{math}
        $$\mprset{flushleft}
        \inferrule* [right={\tiny cut}] {
          {[[I |-c t1 : X (x) Y]]} \\
          $$\mprset{flushleft}
          \inferrule* [right={\tiny unitL}] {
            {[[P1, x : X, y : Y, P2 |-c t2 : Z]]}
          }{[[P1, z : X (x) Y, P2 |-c let z : X (x) Y be x (x) y in t2 : Z]]}
        }{[[P1, I, P2 |-c [t1 / z](let z : X (x) Y be x (x) y in t2) : Z]]}
      \end{math}
    \end{center}
  \item \NDdruleTXXimpEName:
    \begin{center}
      \tiny
      $\NDdruleTXXimpE{}$
    \end{center}
    maps to
    \begin{center}
      \tiny
      \begin{math}
        $$\mprset{flushleft}
        \inferrule* [right={\tiny cut}] {
          {[[I |-c t1 : X -o Y]]} \\
          $$\mprset{flushleft}
          \inferrule* [right={\tiny unitL}] {
            {[[P |-c t2 : X]]} \\
            {[[x : Y |-c x : Y]]}
          }{[[y : X -o Y, P |-c [app y t2 / x]x : Y]]}
        }{[[I, P |-c [t1 / y][app y t2 / x]x : Y]]}
      \end{math}
    \end{center}
  \item \NDdruleSXXunitEOneName:
    \begin{center}
      \tiny
      $\NDdruleSXXunitEOne{}$
    \end{center}
    maps to
    \begin{center}
      \tiny
      \begin{math}
        $$\mprset{flushleft}
        \inferrule* [right={\tiny cut1}] {
          {[[I |-c t : UnitT]]} \\
          $$\mprset{flushleft}
          \inferrule* [right={\tiny unitL1}] {
            {[[G |-l s : A]]}
          }{[[x : UnitT, G |-l let x : UnitT be trivT in s : A]]}
        }{[[I, P |-l [t / x](let x : UnitT be trivT in s) : A]]}
      \end{math}
    \end{center}
  \item \NDdruleSXXunitETwoName:
    \begin{center}
      \tiny
      $\NDdruleSXXunitETwo{}$
    \end{center}
    maps to
    \begin{center}
      \tiny
      \begin{math}
        $$\mprset{flushleft}
        \inferrule* [right={\tiny cut2}] {
          {[[G |-l s1 : UnitS]]} \\
          $$\mprset{flushleft}
          \inferrule* [right={\tiny unitL2}] {
            {[[D |-l s2 : A]]}
          }{[[x : UnitS, D |-l let x : UnitS be trivS in s2 : A]]}
        }{[[G, D |-l [s1 / x](let x : UnitS be trivS in s2) : A]]}
      \end{math}
    \end{center}
  \item \NDdruleSXXtenEOneName:
    \begin{center}
      \tiny
      $\NDdruleSXXtenEOne{}$
    \end{center}
    maps to
    \begin{center}
      \tiny
      \begin{math}
        $$\mprset{flushleft}
        \inferrule* [right={\tiny cut1}] {
          {[[I |-c t : X (x) Y]]} \\
          $$\mprset{flushleft}
          \inferrule* [right={\tiny tenL1}] {
            {[[G1, x : X, y : Y, G2 |-l s : A]]}
          }{[[G1, z : X (x) Y, G2 |-l let z : X (x) Y be x (x) y in s : A]]}
        }{[[G1, I, G2 |-l [t / z](let z : X (x) Y be x (x) y in s) : A]]}
      \end{math}
    \end{center}
  \item \NDdruleSXXtenETwoName:
    \begin{center}
      \tiny
      $\NDdruleSXXtenETwo{}$
    \end{center}
    maps to
    \begin{center}
      \tiny
      \begin{math}
        $$\mprset{flushleft}
        \inferrule* [right={\tiny cut2}] {
          {[[G |-l s1 : A (>) B]]} \\
          $$\mprset{flushleft}
          \inferrule* [right={\tiny tenL2}] {
            {[[D1, x : A, y : B, D2 |-l s2 : C]]}
          }{[[D1, z : A (>) B, D2 |-l let z : A (>) B be x (>) y in s2 : C]]}
        }{[[D1, G, D2 |-l [s1 / z](let z : A (>) B be x (>) y in s2) : C]]}
      \end{math}
    \end{center}
  \item \NDdruleSXXimprEName: (NOT SURE)
    \begin{center}
      \tiny
      $\NDdruleSXXimprE{}$
    \end{center}
    maps to
    \begin{center}
      \tiny
      \begin{math}
        $$\mprset{flushleft}
        \inferrule* [right={\tiny cut2}] {
          {[[G |-l s1 : A -> B]]} \\
          $$\mprset{flushleft}
          \inferrule* [right={\tiny imprL}] {
            {[[D |-l s2 : A]]} \\
            {[[x : B |-l x : B]]}
          }{[[y : A -> B, D |-l [appr y s2 / x]x : B]]}
        }{[[G, D |-l [s1 / y][appr y s2 / x]x : B]]}
      \end{math}
    \end{center}
  \item \NDdruleSXXimplEName: (NOT SURE)
    \begin{center}
      \tiny
      $\NDdruleSXXimplE{}$
    \end{center}
    maps to
    \begin{center}
      \tiny
      \begin{math}
        $$\mprset{flushleft}
        \inferrule* [right={\tiny cut2}] {
          {[[G |-l s1 : B <- A]]} \\
          $$\mprset{flushleft}
          \inferrule* [right={\tiny implL}] {
            {[[D |-l s2 : A]]} \\
            {[[x : B |-l x : B]]}
          }{[[D, y : B <- A |-l [appl y s2 / x]x : B]]}
        }{[[D, G |-l [s1 / y][appl y s2 / x]x : B]]}
      \end{math}
    \end{center}
  \item \NDdruleSXXFEName:
    \begin{center}
      \tiny
      $\NDdruleSXXFE{}$
    \end{center}
    maps to
    \begin{center}
      \tiny
      \begin{math}
        $$\mprset{flushleft}
        \inferrule* [right={\tiny cut2}] {
          {[[G |-l y : F X]]} \\
          $$\mprset{flushleft}
          \inferrule* [right={\tiny FL}] {
            {[[D1, x : X, D2 |-l s : A]]}
          }{[[D1, z : F X, D2 |-l let z : F X be F x in s : A]]}
        }{[[D1, G, D2 |-l [y / z](let y : F X be F x in s) : A]]}
      \end{math}
    \end{center}
  \item \NDdruleSXXGEName:
    \begin{center}
      \tiny
      $\NDdruleSXXGE{}$
    \end{center}
    maps to
    \begin{center}
      \tiny
      \begin{math}
        $$\mprset{flushleft}
        \inferrule* [right={\tiny cut1}] {
          $$\mprset{flushleft}
          \inferrule* [right={\tiny GL}] {
            {[[x : A |-l x : A]]}
          }{[[y : Gf A |-l let y : Gf A be Gf x in x : A]]} \\
           {[[I |-c t : Gf A]]}
        }{[[I |-l [t / y](let y : Gf A be Gf x in x) : A]]}
      \end{math}
    \end{center}
  \end{itemize}
\end{itemize}

Function $N:SE\rightarrow ND$ maps a proof in the sequent calculus to a proof of the same
sequent in the natural deduction. The function is defined as follows:

\begin{itemize}
\item Axioms map to axioms.
\item Instances of cut rules map to the admissible substitution rules.
\item Right rules map to introductions.
\item Left rules map to eliminations modulo some structural fiddling.
  \begin{itemize}
  \item \ElledruleTXXunitLName:
    \begin{center}
      \tiny
      $\ElledruleTXXunitL{}$
    \end{center}
    maps to
    \begin{center}
      \tiny
      \begin{math}
        $$\mprset{flushleft}
        \inferrule* [right={\tiny unitE}] {
          {[[x : UnitT |-c x : UnitT]]} \\
          {[[P |-c t : X]]}
        }{[[x : UnitT, P |-c let x : UnitT be trivT in t : X]]}
      \end{math}
    \end{center}
  \item \ElledruleTXXtenLName:
    \begin{center}
      \tiny
      $\ElledruleTXXtenL{}$
    \end{center}
    maps to
    \begin{center}
      \tiny
      \begin{math}
        $$\mprset{flushleft}
        \inferrule* [right={\tiny tenE}] {
          {[[z : X (x) Y |-c z : X (x) Y]]} \\
          {[[I, x : X, y : Y, P |-c t : Z]]}
        }{[[I, z : X (x) Y, P |-c let z : X (x) Y be x (x) y in t : Z]]}
      \end{math}
    \end{center}
  \item \ElledruleTXXimpLName:
    \begin{center}
      \tiny
      $\ElledruleTXXimpL{}$
    \end{center}
    maps to
    \begin{center}
      \tiny
      \begin{math}
        $$\mprset{flushleft}
        \inferrule* [right={\tiny cut1}] {
          $$\mprset{flushleft}
          \inferrule* [right={\tiny impE}] {
            {[[z : X -o Y |-c z : X -o Y]]} \\
            {[[I |-c t1 : X]]}
          }{[[z : X -o Y, I |-c app z t1 : Y]]} \\
           {[[P1, x : Y, P2 |-c t2 : Z]]}
        }{[[P1, z : X -o Y, I, P2 |-c [app z t2 / x]t2 : Z]]}
      \end{math}
    \end{center}
  \item \ElledruleSXXunitLOneName:
    \begin{center}
      \tiny
      $\ElledruleSXXunitLOne{}$
    \end{center}
    maps to
    \begin{center}
      \tiny
      \begin{math}
        $$\mprset{flushleft}
        \inferrule* [right={\tiny unitE1}] {
          {[[x : UnitT |-c x : UnitT]]} \\
          {[[D |-l s : A]]}
        }{[[x : UnitT, D |-l let x : UnitT be trivT in s : A]]}
      \end{math}
    \end{center}
  \item \ElledruleSXXunitLTwoName:
    \begin{center}
      \tiny
      $\ElledruleSXXunitLTwo{}$
    \end{center}
    maps to
    \begin{center}
      \tiny
      \begin{math}
        $$\mprset{flushleft}
        \inferrule* [right={\tiny unitE2}] {
          {[[x : UnitS |-l x : UnitS]]} \\
          {[[D |-l s : A]]}
        }{[[x : UnitS, D |-l let x : UnitS be trivS in s : A]]}
      \end{math}
    \end{center}
  \item \ElledruleSXXtenLOneName:
    \begin{center}
      \tiny
      $\ElledruleSXXtenLOne{}$
    \end{center}
    maps to
    \begin{center}
      \tiny
      \begin{math}
        $$\mprset{flushleft}
        \inferrule* [right={\tiny tenE1}] {
          {[[z : X (x) Y |-c z : X (x) Y]]} \\
          {[[G, x : X, y : Y, D |-l s : A]]}
        }{[[G, z : X (x) Y, D |-l let z : X (x) Y be x (x) y in s : A]]}
      \end{math}
    \end{center}
  \item \ElledruleSXXtenLTwoName:
    \begin{center}
      \tiny
      $\ElledruleSXXtenLTwo{}$
    \end{center}
    maps to
    \begin{center}
      \tiny
      \begin{math}
        $$\mprset{flushleft}
        \inferrule* [right={\tiny tenE2}] {
          {[[z : A (>) B |-l z : A (>) B]]} \\
          {[[G, x : A, y : B, D |-l s : C]]}
        }{[[G, z : A (>) B, D |-l let z : A (>) B be x (>) y in s : C]]}
      \end{math}
    \end{center}
  \item \ElledruleSXXimpLName:
    \begin{center}
      \tiny
      $\ElledruleSXXimpL{}$
    \end{center}
    maps to
    \begin{center}
      \tiny
      \begin{math}
        $$\mprset{flushleft}
        \inferrule* [right={\tiny cut1}] {
          $$\mprset{flushleft}
          \inferrule* [right={\tiny impE}] {
            {[[z : X -o Y |-c z : X -o Y]]} \\
            {[[I |-c t : X]]}
          }{[[z : X -o Y, I |-c app z t : Y]]} \\
           {[[G, x : Y, D |-l s : A]]}
        }{[[G, z : X -o Y, I, D |-l [app z t / x]s : A]]}
      \end{math}
    \end{center}
  \item \ElledruleSXXimprLName:
    \begin{center}
      \tiny
      $\ElledruleSXXimprL{}$
    \end{center}
    maps to
    \begin{center}
      \tiny
      \begin{math}
        $$\mprset{flushleft}
        \inferrule* [right={\tiny cut2}] {
          $$\mprset{flushleft}
          \inferrule* [right={\tiny imprE}] {
            {[[z : A -> B |-l z : A -> B]]} \\
            {[[G |-l s1 : A]]}
          }{[[z : A -> B, G |-l appr z s1 : B]]} \\
           {[[D, x : B |-l s2 : C]]}
        }{[[D, z : A -> B, G |-l [appr z s1 / x]s2 : C]]}
      \end{math}
    \end{center}
  \item \ElledruleSXXimplLName:
    \begin{center}
      \tiny
      $\ElledruleSXXimplL{}$
    \end{center}
    maps to
    \begin{center}
      \tiny
      \begin{math}
        $$\mprset{flushleft}
        \inferrule* [right={\tiny cut2}] {
          $$\mprset{flushleft}
          \inferrule* [right={\tiny implE}] {
            {[[z : B <- A |-l z : B <- A]]} \\
            {[[G |-l s1 : A]]}
          }{[[G, z : B <- A |-l appl z s1 : B]]} \\
           {[[x : B, D |-l s2 : C]]}
        }{[[G, z : B <- A, D |-l [appl z s1 / x]s2 : C]]}
      \end{math}
    \end{center}
  \item \ElledruleSXXFlName:
    \begin{center}
      \tiny
      $\ElledruleSXXFl{}$
    \end{center}
    maps to
    \begin{center}
      \tiny
      \begin{math}
        $$\mprset{flushleft}
        \inferrule* [right={\tiny FE}] {
          {[[z : F X |-l z : F X]]} \\
          {[[G, x : X, D |-l s : A]]}
        }{[[G, z : F X, D |-l let F x : F X be z in s : A]]}
      \end{math}
    \end{center}
  \item \ElledruleSXXGlName:
    \begin{center}
      \tiny
      $\ElledruleSXXGl{}$
    \end{center}
    maps to
    \begin{center}
      \tiny
      \begin{math}
        $$\mprset{flushleft}
        \inferrule* [right={\tiny cut2}] {
          $$\mprset{flushleft}
          \inferrule* [right={\tiny GE}] {
            {[[y : Gf A |-c y : Gf A]]}
          }{[[y : Gf A |-l derelict y : A]]} \\
           {[[G, x : A, D |-l s : B]]}
        }{[[G, y : Gf A, D |-l [derelict y / x]s : B]]}
      \end{math}
    \end{center}
    
  \end{itemize}
\end{itemize}

\subsection{Strong Normalization of LAM Logic}
\label{subsec:strong_normalization_of_lam_logic}
\input{Elle-to-LNL-ott}
% subsection strong_normalization_of_lam_logic (end)
