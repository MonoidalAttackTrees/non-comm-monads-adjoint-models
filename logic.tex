In a LAM, the SMCC $\cat{C}$ models the commutative linear logic and the Lambeck category
$\cat{L}$ models the non-commutative variant. In Section~\ref{subsec:elle}, we will present the
term assignment for sequent calculus of both sides and prove the cut elimination theorem. In
Section~\ref{subsec:elle-nd}, we present the term assignment for natural deduction of both sides
and prove the logic is strongly normalizing.

A sequent in the commutative side is of the form $[[P,I |-c t : X]]$. The types must be $X$,
$Y$, $Z$, etc., which are objects in the SMCC $\cat{C}$. Tye typing contexts are multisets.
Suppose $[[P]]$ is the set $x_1:X_1, x_2:X_2, ..., x_m:X_m$ and $[[I]]$ is the set
$y_1:Y_1, y_2:Y_2,...,y_n:Y_n$, then the categorical interpretation of the sequent is the
morphism $(X_1\otimes X_2\otimes...\otimes X_m)\otimes(Y_1\otimes Y_2\otimes...\otimes Y_n)\rightarrow X$.

A sequent in the non-commutative side is of the form $[[G,D |-l s : A]]$. The types must be $A$,
$B$, $C$, etc., which are objects in the Lambek category $\cat{L}$. Tye typing contexts are
lists instead of multisets. The typing contexts are mixed in the sense that they could include
contexts from the commutative side. When a commutative context $[[P]]=\{x_1:X_1,...,x_m:X_m\}$
is included, it is interpreted as the object $F(X_1\otimes...\otimes X_m)$. Therefore, the
interpretation of the sequent $[[P,G |-l s : A]]$, where $[[P]]$ is defined as above and $[[G]]$
is the list $y_1:A_1,...,y_n:A_n$, is the morphism
$F(X_1\otimes...\otimes X_m)\tri(A_1\tri...\tri A_n)\rightarrow A$.

For the commutative side, since the contexts are multisets, the following exchange rule
is implicit in both sequent calculus and natural deduction:

\begin{figure}[!h]
  \scriptsize
  \begin{mathpar}
    \ElledruleTXXbeta{}
  \end{mathpar}
\end{figure}



%%%%%%%%%%%%%%%%%%%%%%%%%%%%%%%%%%%%%%%%%%%%%%%%%%
\subsection{Sequent Calculus}
\label{subsec:elle}

The term assignment for sequent calculus of the commutative part of the model, i.e. the SMCC of
the adjunction, is defined in Figure~\ref{fig:elle-smcc}. And the term assignme for the
non-commutative part, i.e. the Lambek category of the adjunction, is defined in
Figure~\ref{fig:elle-lambek}. We do not have the structural rules except for exchange because
the calculus is for linear logic. 


\begin{figure}[!h]
 \scriptsize
  \begin{mdframed}
    \begin{mathpar}
      \ElledruleTXXvar{} \qquad\qquad \ElledruleTXXunitL{} \qquad\qquad \ElledruleTXXunitR{} \\
      \ElledruleTXXtenL{} \qquad\qquad \ElledruleTXXtenR{} \\
      \ElledruleTXXimpL{} \qquad\qquad \ElledruleTXXimpR{} \\
      \ElledruleTXXGr{} \qquad\qquad \ElledruleTXXcut{}
    \end{mathpar}
  \end{mdframed}
\caption{Sequent Calculus: Commutative Part}
\label{fig:elle-smcc}
\end{figure}

\begin{figure}[!h]
 \scriptsize
  \begin{mdframed}
    \begin{mathpar}
      \ElledruleSXXax{} \qquad\qquad \ElledruleSXXunitR{} \qquad\qquad \ElledruleSXXunitLOne{} \\
      \ElledruleSXXunitLTwo{} \qquad\qquad \ElledruleSXXbeta{} \\
      \ElledruleSXXtenLOne{} \qquad\qquad \ElledruleSXXtenLTwo{} \\
      \ElledruleSXXtenR{} \qquad\qquad \ElledruleSXXimpL{} \\
      \ElledruleSXXimprL{} \qquad\qquad \ElledruleSXXimplL{} \\
      \ElledruleSXXimprR{} \qquad\qquad \ElledruleSXXimplR{} \qquad\qquad \ElledruleSXXFr{} \\
      \ElledruleSXXFl{} \qquad\qquad \ElledruleSXXGl{} \\
      \ElledruleSXXcutOne{} \qquad\qquad \ElledruleSXXcutTwo{} \\
    \end{mathpar}
  \end{mdframed}
\caption{Sequent Calculus: Non-Commutative Part}
\label{fig:elle-lambek}
\end{figure}

We prove cut elimination for the sequent calculus using a similar method as Benton's \cite{}.
First, we replace the cut rules $\ElledruleSXXGlName{}$ by the following
n-ary forms respectively:



%%%%%%%%%%%%%%%%%%%%%%%%%%%%%%%%%%%%%%%%%%%%%%%%%%
\subsection{Natural Deduction}
\label{subsec:elle-nd}

The term assignment for natural deduction of the commutative part of the model, i.e. the SMCC of
the adjunction, is defined in Figure~\ref{fig:elle-nd-smcc}. And the term assignme for the
non-commutative part, i.e. the Lambek category of the adjunction, is defined in
Figure~\ref{fig:elle-nd-lambek}.

\begin{figure}[!h]
  \scriptsize
  \begin{mdframed}
    \begin{mathpar}
      \NDdruleTXXid{} \qquad\qquad \NDdruleTXXunitI{} \qquad\qquad \NDdruleTXXunitE{} \\
      \NDdruleTXXtenI{} \qquad\qquad \NDdruleTXXtenE{} \\
      \NDdruleTXXimpI{} \qquad\qquad \NDdruleTXXimpE{} \qquad\qquad \NDdruleTXXGI{} \\
      \NDdruleSXXbeta{}
    \end{mathpar}
  \end{mdframed}
\caption{Natural Deduction: Commutative Part}
\label{fig:elle-nd-smcc}
\end{figure}

\begin{figure}[!h]
 \scriptsize
  \begin{mdframed}
    \begin{mathpar}
      \NDdruleSXXid{} \qquad\qquad \NDdruleSXXunitI{} \qquad\qquad \NDdruleSXXunitEOne{} \\
      \NDdruleSXXunitEOne{} \qquad\qquad \NDdruleSXXunitETwo{} \\
      \NDdruleSXXtenI{} \qquad\qquad \NDdruleSXXtenEOne{} \\
      \NDdruleSXXtenETwo{} \qquad\qquad \NDdruleSXXimprI{} \\
      \NDdruleSXXimprE{} \qquad\qquad \NDdruleSXXimplI{} \\
      \NDdruleSXXimplE{} \qquad\qquad \NDdruleSXXGE{} \qquad\qquad \NDdruleSXXFI{} \\
      \NDdruleSXXFE{}
    \end{mathpar}
  \end{mdframed}
\caption{Natural Deduction: Non-Commutative Part}
\label{fig:elle-nd-lambek}
\end{figure}

We could derive exchange comonadically as follows:

\begin{center}
  \tiny
  \begin{math}
  $$\mprset{flushleft}
  \inferrule* [right={\tiny imprI}] {
    $$\mprset{flushleft}
    \inferrule* [right={\tiny tenE2}] {
      $$\mprset{flushleft}
      \inferrule* [right={\tiny id}] {
        \,
      }{[[z : h(F Gf A) (>) F Gf B |-l z : h(F Gf A) (>) F Gf B]]}
        $$\mprset{flushleft}
        \inferrule* [right={\tiny FE}] {
          $$\mprset{flushleft}
          \inferrule* [right={\tiny id}] {
            \,
          }{[[x2 : F Gf A |-l x2 : F Gf A]]}
            $$\mprset{flushleft}
            \inferrule* [right={\tiny FE}] {
              $$\mprset{flushleft}
              \inferrule* [right={\tiny id}] {
                \,
              }{[[y2 : F Gf B |-l y2 : F Gf B]]}
              \inferrule* [right={\tiny beta}] {
                $$\mprset{flushleft}
                \inferrule* [right={\tiny FE}] {
                  $$\mprset{flushleft}
                  \inferrule* [right={\tiny FI}] {
                    $$\mprset{flushleft}
                    \inferrule* [right={\tiny id}] {
                      \,
                    }{[[y0 : Gf B |-c y0 : Gf B]]}
                  }{[[y0 : Gf B |-l F y0 : F Gf B]]}
                  $$\mprset{flushleft}
                  \inferrule* [right={\tiny FI}] {
                    $$\mprset{flushleft}
                    \inferrule* [right={\tiny id}] {
                      \,
                    }{[[x0 : Gf A |-c x0 : Gf A]]}
                  }{[[x0 : Gf A |-l F x0 : F Gf A]]}
                }{[[y0 : Gf B, x0 : Gf A |-l h(F y0) (>) F x0 : h(F Gf B) (>) F Gf A]]}
              }{[[x1 : Gf A, y1 : Gf B |-l ex y1 , x1 with y0 , x0 in (h(F y0) (>) F x0) : h(F Gf B) (>) F Gf A]]}
            }{[[x1 : Gf A , y2 : F Gf B |-l let F y1 : F Gf B be y2 in (ex y1 , x1 with y0 , x0 in (h(F y0) (>) F x0)) : h(F Gf B) (>) F Gf A]]}
          }{[[x2 : F Gf A , y2 : F Gf B |-l let F x1 : F Gf A be x2 in (let F y1 : F Gf B be y2 in (ex y1 , x1 with y0 , x0 in (h(F y0) (>) F x0))) : h(F Gf B) (>) F Gf A]]}
        }{[[z : h(F Gf A) (>) F Gf B |-l let z : h(F Gf A) (>) F Gf B be x2 (>) y2 in (let F x1 : F Gf A be x2 in (let F y1 : F Gf B be y2 in (ex y1 , x1 with y0 , x0 in (h(F y0) (>) F x0)))) : h(F Gf B) (>) F Gf A]]}
      }{[[ . |-l \r z : h(F Gf A) (>) F Gf B.let z : h(F Gf A) (>) F Gf B be x2 (>) y2 in (let F x1 : F Gf A be x2 in (let F y1 : F Gf B be y2 in (ex y1 , x1 with y0 , x0 in (h(F y0) (>) F x0)))) : (h(F Gf A) (>) F Gf B) -> (h(F Gf B) (>) F Gf A)]]}
  \end{math}
\end{center}

We also have the three cut rules derivable in the natural deduction:

\begin{figure}[!h]
  \scriptsize
  \begin{mathpar}
    \NDdruleTXXcut{} \qquad\qquad \NDdruleSXXcutOne{} \qquad\qquad \NDdruleSXXcutTwo{}
  \end{mathpar}
\end{figure}

\begin{center}
  \tiny
  \begin{math}
  $$\mprset{flushleft}
  \inferrule* [right={\tiny FI}] {
    {[[I |-c t : X]]}
  }{[[I |-l F t : F X]]}
  \end{math}
\end{center}
