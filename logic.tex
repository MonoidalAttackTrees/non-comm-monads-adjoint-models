In a LAM, the SMCC $\cat{C}$ models the commutative linear logic and the Lambeck category
$\cat{L}$ models the non-commutative variant. In Section~\ref{subsec:elle}, we will present the
term assignment for sequent calculus of both sides and prove the cut elimination theorem. In
Section~\ref{subsec:elle-nd}, we present the term assignment for natural deduction of both sides
and prove the logic is strongly normalizing.

A sequent in the commutative side is of the form $[[P,I |-c t : X]]$. The types must be $X$,
$Y$, $Z$, etc., which are objects in the SMCC $\cat{C}$. Tye typing contexts are multisets.
Suppose $[[P]]$ is the set $x_1:X_1, x_2:X_2, ..., x_m:X_m$ and $[[I]]$ is the set
$y_1:Y_1, y_2:Y_2,...,y_n:Y_n$, then the categorical interpretation of the sequent is the
morphism $(X_1\otimes X_2\otimes...\otimes X_m)\otimes(Y_1\otimes Y_2\otimes...\otimes Y_n)\rightarrow X$.

A sequent in the non-commutative side is of the form $[[G,D |-l s : A]]$. The types must be $A$,
$B$, $C$, etc., which are objects in the Lambek category $\cat{L}$. Tye typing contexts are
lists instead of multisets. The typing contexts are mixed in the sense that they could include
contexts from the commutative side. When a commutative context $[[P]]=\{x_1:X_1,...,x_m:X_m\}$
is included, it is interpreted as the object $F(X_1\otimes...\otimes X_m)$. Therefore, the
interpretation of the sequent $[[P,G |-l s : A]]$, where $[[P]]$ is defined as above and $[[G]]$
is the list $y_1:A_1,...,y_n:A_n$, is the morphism
$F(X_1\otimes...\otimes X_m)\tri(A_1\tri...\tri A_n)\rightarrow A$.

For the commutative side, since the contexts are multisets, the following exchange rule
is implicit in both sequent calculus and natural deduction:

\begin{center}
  \scriptsize
  $\ElledruleTXXbeta{}$
\end{center}



%%%%%%%%%%%%%%%%%%%%%%%%%%%%%%%%%%%%%%%%%%%%%%%%%%
\subsection{Sequent Calculus}
\label{subsec:elle}

The term assignment for sequent calculus of the commutative part of the model, i.e. the SMCC of
the adjunction, is defined in Figure~\ref{fig:elle-smcc}. And the term assignme for the
non-commutative part, i.e. the Lambek category of the adjunction, is defined in
Figure~\ref{fig:elle-lambek}. We do not have the structural rules except for exchange because
the calculus is for linear logic. 

\begin{figure}[!h]
 \scriptsize
  \begin{mdframed}
    \begin{mathpar}
      \ElledruleTXXax{} \qquad\qquad \ElledruleTXXunitL{} \qquad\qquad \ElledruleTXXunitR{} \\
      \ElledruleTXXtenL{} \qquad\qquad \ElledruleTXXtenR{} \\
      \ElledruleTXXimpL{} \qquad\qquad \ElledruleTXXimpR{} \\
      \ElledruleTXXGr{} \qquad\qquad \ElledruleTXXcut{}
    \end{mathpar}
  \end{mdframed}
\caption{Sequent Calculus: Commutative Part}
\label{fig:elle-smcc}
\end{figure}

\begin{figure}[!h]
 \scriptsize
  \begin{mdframed}
    \begin{mathpar}
      \ElledruleSXXax{} \qquad\qquad \ElledruleSXXunitR{} \qquad\qquad \ElledruleSXXunitLOne{} \\
      \ElledruleSXXunitLTwo{} \qquad\qquad \ElledruleSXXbeta{} \\
      \ElledruleSXXtenLOne{} \qquad\qquad \ElledruleSXXtenLTwo{} \\
      \ElledruleSXXtenR{} \qquad\qquad \ElledruleSXXimpL{} \\
      \ElledruleSXXimprL{} \qquad\qquad \ElledruleSXXimplL{} \\
      \ElledruleSXXimprR{} \qquad\qquad \ElledruleSXXimplR{} \qquad\qquad \ElledruleSXXFr{} \\
      \ElledruleSXXFl{} \qquad\qquad \ElledruleSXXGl{} \\
      \ElledruleSXXcutOne{} \qquad\qquad \ElledruleSXXcutTwo{} \\
    \end{mathpar}
  \end{mdframed}
\caption{Sequent Calculus: Non-Commutative Part}
\label{fig:elle-lambek}
\end{figure}

Next, we prove cut elimination for the sequent calculus. We define the \textbf{degree $|X|$
(or $|A|$) of a commutative (or non-commutative) formula} to be the number of logical
connectives $|X|$ plus $1$. For instance, $|[[X (x) Y]]| = |[[X]]| + |[[Y]]| + 1$. And the
\textbf{degree of a cut rule} is the degree of the cut formula. The following key cases
demonstrate how we can replace a cut with at most two cuts with lower degree. The
\textbf{degree $|\Pi|$ of a proof} $\Pi$ is the maximum of the degrees of all cut fules in the
proof and $|\Pi|=0$ if $\Pi$ is cut-free. Finally, the \textbf{height $h(\Pi)$ of a proof
$\Pi$} is the length of the longest path in the proof tree and the height of an axiom is $0$.

We consider the following $11$ key cases in proving cut elimination, each of which is a
$(R, L)$ pair for the same connective.

\begin{itemize}

\item $(\ElledruleTXXunitRName, \ElledruleTXXunitLName)$:
  \begin{center}
    \tiny
    \begin{math}
      $$\mprset{flushleft}
      \inferrule* [right={\tiny cut}] {
        $$\mprset{flushleft}
        \inferrule* [right={\tiny unitR}] {
          \,
        }{[[. |-c trivT : UnitT]]}
        \\
        $$\mprset{flushleft}
        \inferrule* [right={\tiny unitL}] {
          {[[I, P |-c t : X]]}
        }{[[I, x : UnitT, P |-c let x : UnitT be trivT in t : X]]}
      }{[[I, P |-c [trivT / x] (let x : UnitT be trivT in t) : X]]}
    \end{math}
  \end{center}
  is transformed to 
  \begin{center}
    \tiny
    $[[I, P |-c t : X]]$
  \end{center}

\item $(\ElledruleTXXunitRName, \ElledruleSXXunitLOneName)$:
  \begin{center}
    \tiny
    \begin{math}
      $$\mprset{flushleft}
      \inferrule* [right={\tiny cut}] {
        $$\mprset{flushleft}
        \inferrule* [right={\tiny unitR}] {
          \,
        }{[[. |-c trivT : UnitT]]}
        \\
        $$\mprset{flushleft}
        \inferrule* [right={\tiny unitL}] {
          {[[G, D |-l s : A]]}
        }{[[G, x : UnitT, D |-l let x : UnitT be trivT in s : A]]}
      }{[[G, D |-l [trivT / x] (let x : UnitT be trivT in s) : A]]}
    \end{math}
  \end{center}
  is transformed to
  \begin{center}
    \tiny
    $[[G, D |-l s : A]]$
  \end{center}

\item $(\ElledruleTXXtenRName, \ElledruleTXXtenLName)$:
  \begin{center}
    \tiny
    \begin{math}
      $$\mprset{flushleft}
      \inferrule* [right={\tiny cut}] {
        $$\mprset{flushleft}
        \inferrule* [right={\tiny tenR}] {
          {[[I1 |-c t1 : X]]} \\
          {[[I2 |-c t2 : Y]]}
        }{[[I1, I2 |-c t1 (x) t2 : X (x) Y]]}
        \\
        $$\mprset{flushleft}
        \inferrule* [right={\tiny tenL}] {
          {[[P1, x : X, y : Y, P2 |-c t3 : Z]]}
        }{[[P1, z : X (x) Y, P2 |-c let z : X (x) Y be x (x) y in t3 : Z]]}
      }{[[P1, I1, I2, P2 |-c [t1 (x) t2 / z](let z : X (x) Y be x (x) y in t3) : Z]]}
    \end{math}
  \end{center}
  is transformed to
  \begin{center}
    \tiny
    \begin{math}
      $$\mprset{flushleft}
      \inferrule* [right={\tiny cut}] {
        $$\mprset{flushleft}
        \inferrule* [right={\tiny cut}] {
          {[[I1 |-c t1 : X]]} \\
          {[[P1, x : X, y : Y, P2 |-c t3 : Z]]}
        }{[[P1, I1, y : Y, P2 |-c [t1 / x] t3 : Z]]} \\
        {[[I2 |-c t2 : Y]]}
      }{[[P1, I1, I2, P2 |-c [t2/y][t1/x]t3 : Z]]}
    \end{math}
  \end{center}

\item $(\ElledruleTXXtenRName, \ElledruleSXXtenLOneName)$:
  \begin{center}
    \tiny
    \begin{math}
      $$\mprset{flushleft}
      \inferrule* [right={\tiny cut1}] {
        $$\mprset{flushleft}
        \inferrule* [right={\tiny tenR}] {
          {[[I1 |-c t1 : X]]} \\
          {[[I2 |-c t2 : Y]]}
        }{[[I1, I2 |-c t1 (x) t2 : X (x) Y]]}
        \\
        $$\mprset{flushleft}
        \inferrule* [right={\tiny tenL1}] {
          {[[G, x : X, y : Y, D |-l s : A]]}
        }{[[G, z : X (x) Y, D |-l let z : X (x) Y be x (x) y in s : A]]}
      }{[[G, I1, I2, D |-l [t1 (x) t2 / z](let z : X (x) Y be x (x) y in s) : A]]}
    \end{math}
  \end{center}
  is transformed to
  \begin{center}
    \tiny
    \begin{math}
      $$\mprset{flushleft}
      \inferrule* [right={\tiny cut1}] {
        $$\mprset{flushleft}
        \inferrule* [right={\tiny cut1}] {
          {[[I1 |-c t1 : X]]} \\
          {[[G, x : X, y : Y, D |-l s : A]]}
        }{[[G, I1, y : Y, D |-l [t1 / x] s : A]]} \\
        {[[I2 |-c t2 : Y]]}
      }{[[G, I1, I2, D |-l [t2/y][t1/x]s : A]]}
    \end{math}
  \end{center}
  
\item $(\ElledruleTXXimpRName, \ElledruleTXXimpLName)$
  \begin{center}
    \tiny
    \begin{math}
      $$\mprset{flushleft}
      \inferrule* [right={\tiny cut}] {
        $$\mprset{flushleft}
        \inferrule* [right={\tiny impR}] {
          {[[I1, x : X, I2 |-c t1 : Y]]}
        }{[[I1, I2 |-c \ x : X . t1 : X -o Y]]}
        \\
        $$\mprset{flushleft}
        \inferrule* [right={\tiny impL}] {
          {[[I |-c t2 : X]]} \\
          {[[P1, y : Y, P2 |-c t3 : Z]]}
        }{[[P1, I, z : X -o Y, P2 |-c [app z t2 / y]t3 : Z]]}
      }{[[P1, I, I1, I2, P2 |-c [(\ x : X . t1) / z][app z t2 / y]t3 : Z]]}
    \end{math}
  \end{center}
  is transformed to
  \begin{center}
    \tiny
    \begin{math}
      $$\mprset{flushleft}
      \inferrule* [right={\tiny cut}] {
        $$\mprset{flushleft}
        \inferrule* [right={\tiny cut}] {
          {[[I1, x : X, I2 |-c t1 : Y]]} \\
          {[[I |-c t2 : X]]}
        }{[[I1, I, I2 |-c [t2 / x]t1 : Y]]} \\
        {[[P1, y : Y, P2 |-c t3 : Z]]}
      }{[[P1, I1, I, I2, P2 |-c [([t2 / x]t1) / y]t3 : Z]]}
    \end{math}
  \end{center}

\item $(\ElledruleSXXunitRName, \ElledruleSXXunitLTwoName)$:
  \begin{center}
    \tiny
    \begin{math}
      $$\mprset{flushleft}
      \inferrule* [right={\tiny cut2}] {
        $$\mprset{flushleft}
        \inferrule* [right={\tiny unitR}] {
          \,
        }{[[. |-l trivS : UnitS]]}
        \\
        $$\mprset{flushleft}
        \inferrule* [right={\tiny unitL2}] {
          {[[G, D |-l s : A]]}
        }{[[G, x : UnitS, D |-l let x : UnitS be trivS in s : A]]}
      }{[[G, D |-l [trivS / x] (let x : UnitS be trivS in s) : A]]}
    \end{math}
  \end{center}
  is transformed to 
  \begin{center}
    \tiny
    $[[G, D |-l s : A]]$
  \end{center}

\item $(\ElledruleSXXtenRName, \ElledruleSXXtenLTwoName)$:
  \begin{center}
    \tiny
    \begin{math}
      $$\mprset{flushleft}
      \inferrule* [right={\tiny cut2}] {
        $$\mprset{flushleft}
        \inferrule* [right={\tiny tenR}] {
          {[[G1 |-l s1 : A]]} \\
          {[[G2 |-l s2 : B]]}
        }{[[G1, G2 |-l s1 (>) s2 : A (>) B]]}
        \\
        $$\mprset{flushleft}
        \inferrule* [right={\tiny tenL1}] {
          {[[D1, x : A, y : B, D2 |-l s3 : C]]}
        }{[[D1, z : A (>) B, D2 |-l let z : A (>) B be x (>) y in s : C]]}
      }{[[D1, G1, G2, D2 |-l [s1 (>) s2 / z](let z : A (>) B be x (>) y in s) : C]]}
    \end{math}
  \end{center}
  is transformed to
  \begin{center}
    \tiny
    \begin{math}
      $$\mprset{flushleft}
      \inferrule* [right={\tiny cut2}] {
        $$\mprset{flushleft}
        \inferrule* [right={\tiny cut2}] {
          {[[G1 |-l s1 : A]]} \\
          {[[D1, x : A, y : B, D2 |-l s3 : C]]}
        }{[[D1, G1, y : B, D2 |-l [s1 / x]s3 : C]]} \\
        {[[G2 |-l s2 : B]]}
      }{[[D1, G1, G2, D2 |-l [s2 / y][s1 / x]s3 : C]]}
    \end{math}
  \end{center}

\item $(\ElledruleSXXimprRName, \ElledruleSXXimprLName)$:
  \begin{center}
    \tiny
    \begin{math}
      $$\mprset{flushleft}
      \inferrule* [right={\tiny cut2}] {
        $$\mprset{flushleft}
        \inferrule* [right={\tiny imprR}] {
          {[[G, x : A |-l s1 : B]]}
        }{[[G |-l \r x : A . s1 : A -> B]]}
        \\
        $$\mprset{flushleft}
        \inferrule* [right={\tiny imprL}] {
          {[[D1 |-l s2 : A]]} \\
          {[[D2 , y : B |-l s3 : C]]}
        }{[[D2, z : A -> B, D1 |-l [appr z s2 / y]s3 : C]]}
      }{[[D2, G, D1 |-l [(\r x : A . s1) / z][appr z s2 / y]s3 : C]]}
    \end{math}
  \end{center}
  is transformed to
  \begin{center}
    \tiny
    \begin{math}
      $$\mprset{flushleft}
      \inferrule* [right={\tiny cut2}] {
        $$\mprset{flushleft}
        \inferrule* [right={\tiny cut2}] {
          {[[G, x : A |-l s1 : B]]} \\
          {[[D1 |-l s2 : A]]}
        }{[[G, D1 |-l [s2 / x]s1 : B]]} \\
        {[[D2, y : B |-l s3 : C]]}
      }{[[D2, G, D1 |-l [([s2 / x]s1) / y]s3 : C]]}
    \end{math}
  \end{center}

\item $(\ElledruleSXXimplRName, \ElledruleSXXimplLName)$:
  \begin{center}
    \tiny
    \begin{math}
      $$\mprset{flushleft}
      \inferrule* [right={\tiny cut2}] {
        $$\mprset{flushleft}
        \inferrule* [right={\tiny implR}] {
          {[[x : A, G |-l s1 : B]]}
        }{[[G |-l \l x : A . s1 : B <- A]]}
        \\
        $$\mprset{flushleft}
        \inferrule* [right={\tiny implL}] {
          {[[D1 |-l s2 : A]]} \\
          {[[y : B, D2 |-l s3 : C]]}
        }{[[D1, z : B <- A, D2 |-l [appl z s2 / y]s3 : C]]}
      }{[[D1, G, D2 |-l [(\l x : A . s1) / z][appl z s2 / y]s3 : C]]}
    \end{math}
  \end{center}
  is transformed to
  \begin{center}
    \tiny
    \begin{math}
      $$\mprset{flushleft}
      \inferrule* [right={\tiny cut2}] {
        $$\mprset{flushleft}
        \inferrule* [right={\tiny cut2}] {
          {[[x : A, G |-l s1 : B]]} \\
          {[[D1 |-l s2 : A]]}
        }{[[ D1, G |-l [s2 / x]s1 : B]]} \\
        {[[y : B, D2 |-l s3 : C]]}
      }{[[D1, G, D2 |-l [([s2 / x]s1) / y]s3 : C]]}
    \end{math}
  \end{center}

\item $(\ElledruleSXXFrName, \ElledruleSXXFlName)$:
  \begin{center}
    \tiny
    \begin{math}
      $$\mprset{flushleft}
      \inferrule* [right={\tiny cut2}] {
        $$\mprset{flushleft}
        \inferrule* [right={\tiny FR}] {
          {[[I |-c t : X]]}
        }{[[I |-l F t : F X]]}
        \\
        $$\mprset{flushleft}
        \inferrule* [right={\tiny FL}] {
          {[[G, x : X, D |-l s : A]]}
        }{[[G, y : F X, D |-l let y : F X be F x in s : A]]}
      }{[[G, I, D |-l [F t / y](let y : F X be F x in s) : A]]}
    \end{math}
  \end{center}
  is transformed to
  \begin{center}
    \tiny
    \begin{math}
      $$\mprset{flushleft}
      \inferrule* [right={\tiny cut1}] {
        {[[I |-c t : X]]} \\
        {[[G, x : A, D |-l s : A]]}
      }{[[G, I, D |-l [t / x]s : A]]}
    \end{math}
  \end{center}

\item $(\ElledruleTXXGrName, \ElledruleSXXGlName)$:
  \begin{center}
    \tiny
    \begin{math}
      $$\mprset{flushleft}
      \inferrule* [right={\tiny cut1}] {
        $$\mprset{flushleft}
        \inferrule* [right={\tiny GR}] {
          {[[I |-l s1 : A]]}
        }{[[I |-c Gf s1 : Gf A]]}
        \\
        $$\mprset{flushleft}
        \inferrule* [right={\tiny GL}] {
          {[[G, x : A, D |-l s2 : B]]}
        }{[[G, y : Gf A, D |-l let y : Gf A be Gf x in s2 : B]]}
      }{[[G, I, D |-l [Gf s1 / y](let y : Gf A be Gf x in s2) : B]]}
    \end{math}
  \end{center}
  is transformed to
  \begin{center}
    \tiny
    \begin{math}
      $$\mprset{flushleft}
      \inferrule* [right={\tiny cut}] {
        {[[I |-l s1 : A]]} \\
        {[[G, x : A, D |-l s2 : B]]}
      }{[[G, I, D |-l [s1 / x]s2 : B]]}
    \end{math}
  \end{center}

\end{itemize}

Based on the key cases, given a formula (either commutative or non-commutative) $L$ and proofs
$\Pi$, $\Pi'$, of sequents $M\vdash N$ and $M'\vdash N'$ respectively with degrees less than
$|L|$, there is a proof of $M,N\vdash M',N'$ with degree less than $|L|$, s.t. all currences of
formula $L$ is removed. This can be proved by induction on $h(\Pi)+h(\Pi')$. Therefore, we have
the result that given a proof of a sequent with degree $d>0$, there is a proof of the same
sequent. As a result, we have the cut elimination theorem.

\begin{theorem}[Cut Elimination]
  Let $\Pi$ be a proof of a sequent $[[I |-c t:X]]$ or a sequent $[[G |-l s:A]]$ s.t. $|\Pi|>0$.
  Then there is a cut-free proof of the same sequent.
\end{theorem}



%%%%%%%%%%%%%%%%%%%%%%%%%%%%%%%%%%%%%%%%%%%%%%%%%%
\subsection{Natural Deduction}
\label{subsec:elle-nd}

The term assignment for natural deduction of the commutative part of the model, i.e. the SMCC of
the adjunction, is defined in Figure~\ref{fig:elle-nd-smcc}. And the term assignme for the
non-commutative part, i.e. the Lambek category of the adjunction, is defined in
Figure~\ref{fig:elle-nd-lambek}.

\begin{figure}[!h]
  \scriptsize
  \begin{mdframed}
    \begin{mathpar}
      \NDdruleTXXid{} \qquad\qquad \NDdruleTXXunitI{} \qquad\qquad \NDdruleTXXunitE{} \\
      \NDdruleTXXtenI{} \qquad\qquad \NDdruleTXXtenE{} \\
      \NDdruleTXXimpI{} \qquad\qquad \NDdruleTXXimpE{} \qquad\qquad \NDdruleTXXGI{} \\
      \NDdruleSXXbeta{}
    \end{mathpar}
  \end{mdframed}
\caption{Natural Deduction: Commutative Part}
\label{fig:elle-nd-smcc}
\end{figure}

\begin{figure}[!h]
 \scriptsize
  \begin{mdframed}
    \begin{mathpar}
      \NDdruleSXXid{} \qquad\qquad \NDdruleSXXunitI{} \qquad\qquad \NDdruleSXXunitEOne{} \\
      \NDdruleSXXunitEOne{} \qquad\qquad \NDdruleSXXunitETwo{} \\
      \NDdruleSXXtenI{} \qquad\qquad \NDdruleSXXtenEOne{} \\
      \NDdruleSXXtenETwo{} \qquad\qquad \NDdruleSXXimprI{} \\
      \NDdruleSXXimprE{} \qquad\qquad \NDdruleSXXimplI{} \\
      \NDdruleSXXimplE{} \qquad\qquad \NDdruleSXXGE{} \qquad\qquad \NDdruleSXXFI{} \\
      \NDdruleSXXFE{}
    \end{mathpar}
  \end{mdframed}
\caption{Natural Deduction: Non-Commutative Part}
\label{fig:elle-nd-lambek}
\end{figure}

We could derive exchange comonadically as follows:

\begin{center}
  \tiny
  \begin{math}
  $$\mprset{flushleft}
  \inferrule* [right={\tiny imprI}] {
    $$\mprset{flushleft}
    \inferrule* [right={\tiny tenE2}] {
      $$\mprset{flushleft}
      \inferrule* [right={\tiny id}] {
        \,
      }{[[z : h(F Gf A) (>) F Gf B |-l z : h(F Gf A) (>) F Gf B]]}
        $$\mprset{flushleft}
        \inferrule* [right={\tiny FE}] {
          $$\mprset{flushleft}
          \inferrule* [right={\tiny id}] {
            \,
          }{[[x2 : F Gf A |-l x2 : F Gf A]]}
            $$\mprset{flushleft}
            \inferrule* [right={\tiny FE}] {
              $$\mprset{flushleft}
              \inferrule* [right={\tiny id}] {
                \,
              }{[[y2 : F Gf B |-l y2 : F Gf B]]}
              \inferrule* [right={\tiny beta}] {
                $$\mprset{flushleft}
                \inferrule* [right={\tiny FE}] {
                  $$\mprset{flushleft}
                  \inferrule* [right={\tiny FI}] {
                    $$\mprset{flushleft}
                    \inferrule* [right={\tiny id}] {
                      \,
                    }{[[y0 : Gf B |-c y0 : Gf B]]}
                  }{[[y0 : Gf B |-l F y0 : F Gf B]]}
                  $$\mprset{flushleft}
                  \inferrule* [right={\tiny FI}] {
                    $$\mprset{flushleft}
                    \inferrule* [right={\tiny id}] {
                      \,
                    }{[[x0 : Gf A |-c x0 : Gf A]]}
                  }{[[x0 : Gf A |-l F x0 : F Gf A]]}
                }{[[y0 : Gf B, x0 : Gf A |-l h(F y0) (>) F x0 : h(F Gf B) (>) F Gf A]]}
              }{[[x1 : Gf A, y1 : Gf B |-l ex y1 , x1 with y0 , x0 in (h(F y0) (>) F x0) : h(F Gf B) (>) F Gf A]]}
            }{[[x1 : Gf A , y2 : F Gf B |-l let F y1 : F Gf B be y2 in (ex y1 , x1 with y0 , x0 in (h(F y0) (>) F x0)) : h(F Gf B) (>) F Gf A]]}
          }{[[x2 : F Gf A , y2 : F Gf B |-l let F x1 : F Gf A be x2 in (let F y1 : F Gf B be y2 in (ex y1 , x1 with y0 , x0 in (h(F y0) (>) F x0))) : h(F Gf B) (>) F Gf A]]}
        }{[[z : h(F Gf A) (>) F Gf B |-l let z : h(F Gf A) (>) F Gf B be x2 (>) y2 in (let F x1 : F Gf A be x2 in (let F y1 : F Gf B be y2 in (ex y1 , x1 with y0 , x0 in (h(F y0) (>) F x0)))) : h(F Gf B) (>) F Gf A]]}
      }{[[ . |-l \r z : h(F Gf A) (>) F Gf B.let z : h(F Gf A) (>) F Gf B be x2 (>) y2 in (let F x1 : F Gf A be x2 in (let F y1 : F Gf B be y2 in (ex y1 , x1 with y0 , x0 in (h(F y0) (>) F x0)))) : (h(F Gf A) (>) F Gf B) -> (h(F Gf B) (>) F Gf A)]]}
  \end{math}
\end{center}

We also have the three cut rules derivable in the natural deduction:
(NOTE: Don't know how to prove the third one S\_cut2.)

\begin{figure}[!h]
  \scriptsize
  \begin{mathpar}
    \NDdruleTXXcut{} \qquad\qquad \NDdruleSXXcutOne{} \qquad\qquad \NDdruleSXXcutTwo{}
  \end{mathpar}
\end{figure}

We define the normalization procedure by considering the following pairs of introduction and
elimination rules:

\begin{itemize}

\item (\NDdruleTXXunitIName, \NDdruleTXXunitEName):
  \begin{center}
    \tiny
    \begin{math}
      $$\mprset{flushleft}
      \inferrule* [right={\tiny unitE}] {
        $$\mprset{flushleft}
        \inferrule* [right={\tiny unitI}] {
          \,
        }{[[. |-c trivT : UnitT]]} \\
         {[[I |-c t : X]]}
      }{[[I |-c let trivT : UnitT be trivT in t : X]]}
    \end{math}
  \end{center}
  normalizes to 
  \begin{center}
    \tiny
    $[[I |-c t : X]]$
  \end{center}

\item (\NDdruleTXXunitIName, \NDdruleSXXunitEOneName):
  \begin{center}
    \tiny
    \begin{math}
     $$\mprset{flushleft}
     \inferrule* [right={\tiny unitE2}] {
       $$\mprset{flushleft}
       \inferrule* [right={\tiny unitI}] {
         \,
        }{[[. |-c trivT : UnitT]]} \\
         {[[D |-l s : A]]}
      }{[[D |-l let trivT : UnitT be trivT in s : A]]}
    \end{math}
  \end{center}
  normalizes to
  \begin{center}
    \tiny
    $[[D |-l s : A]]$
  \end{center}

\item (\NDdruleTXXtenIName, \NDdruleTXXtenEName):
  \begin{center}
    \tiny
    \begin{math}
      $$\mprset{flushleft}
      \inferrule* [right={\tiny tenE}] {
        $$\mprset{flushleft}
        \inferrule* [right={\tiny tenI}] {
          {[[I1 |-c t1 : X]]} \\
          {[[I2 |-c t2 : Y]]}
        }{[[I1, I2 |-c t1 (x) t2 : X (x) Y]]} \\
         {[[P1, x : X, y : Y, P2 |-c t3 : Z]]}
      }{[[P1, I1, I2, P2 |-c let t1 (x) t2 : X (x) Y be x (x) y in t3 : Z]]}
    \end{math}
  \end{center}
  normalizes to
  \begin{center}
    \tiny
    \begin{math}
      $$\mprset{flushleft}
      \inferrule* [right={\tiny cut}] {
        {[[I1 |-c t1 : X]]} \\
        $$\mprset{flushleft}
        \inferrule* [right={\tiny cut}] {
          {[[I2 |-c t2 : Y]]} \\
          {[[P1, x : X, y : Y, P2 |-c t3 : Z]]}
        }{[[P1, x : X, I2, P2 |-c [t2 / y]t3 : Z]]}
      }{[[P1, I1, I2, P2 |-c [t1 / x][t2 / y]t3 : Z]]}
    \end{math}
  \end{center}
  
\item (\NDdruleTXXtenIName, \NDdruleSXXtenEOneName):
  \begin{center}
    \tiny
    \begin{math}
      $$\mprset{flushleft}
      \inferrule* [right={\tiny tenE1}] {
        $$\mprset{flushleft}
        \inferrule* [right={\tiny tenI}] {
          {[[I |-c t1 : X]]} \\
          {[[P |-c t2 : Y]]}
        }{[[I, P |-c t1 (x) t2 : X (x) Y]]} \\
         {[[G, x : X, y : Y, D |-l s : A]]}
      }{[[G, I, P, D |-l let t1 (x) t2 : X (x) Y be x (x) y in s : A]]}
    \end{math}
  \end{center}
  normalizes to
  \begin{center}
    \tiny
    \begin{math}
      $$\mprset{flushleft}
      \inferrule* [right={\tiny cut2}] {
        {[[I |-c t1 : X]]} \\
        $$\mprset{flushleft}
        \inferrule* [right={\tiny cut2}] {
          {[[P |-c t2 : Y]]} \\
          {[[G, x : X, y : Y, D |-l s : A]]}
        }{[[G, x : X, P, D |-l [t2 / y]s : A]]}
      }{[[G, I, P, D |-l [t1 / x][t2 / y]s : A]]}
    \end{math}
  \end{center}
  
\item (\NDdruleTXXimpIName, \NDdruleTXXimpEName):
  \begin{center}
    \tiny
    \begin{math}
      $$\mprset{flushleft}
      \inferrule* [right={\tiny impE}] {
        $$\mprset{flushleft}
        \inferrule* [right={\tiny impI}] {
          {[[I, x : X |-c t1 : Y]]}
        }{[[I |-c \ x : X . t1 : X -o Y]]} \\
         {[[P |-c t2 : X]]}
      }{[[I, P |-c app (\ x : X . t1) t2 : Y]]}
    \end{math}
  \end{center}
  normalizes to
  \begin{center}
    \tiny
    \begin{math}
      $$\mprset{flushleft}
      \inferrule* [right={\tiny cut}] {
        {[[I, x : X |-c t1 : Y]]} \\
        {[[P |-c t2 : X]]}
      }{[[I, P |-c [t2 / x]t1 : Y]]}
    \end{math}
  \end{center}

\item (\NDdruleSXXunitIName, \NDdruleSXXunitETwoName):
  \begin{center}
    \tiny
    \begin{math}
     $$\mprset{flushleft}
     \inferrule* [right={\tiny unitE2}] {
       $$\mprset{flushleft}
       \inferrule* [right={\tiny unitI}] {
         \,
        }{[[. |-l trivS : UnitS]]} \\
         {[[D |-l s : A]]}
      }{[[D |-l let trivS : UnitS be trivS in s : A]]}
    \end{math}
  \end{center}
  normalizes to
  \begin{center}
    \tiny
    $[[D |-l s : A]]$
  \end{center}

\item (\NDdruleSXXtenIName, \NDdruleSXXtenETwoName):
  \begin{center}
    \tiny
    \begin{math}
     $$\mprset{flushleft}
     \inferrule* [right={\tiny tenE2}] {
       $$\mprset{flushleft}
       \inferrule* [right={\tiny tenI}] {
         {[[G1 |-l s1 : A]]} \\
         {[[G2 |-l s2 : B]]}
        }{[[G1, G2 |-l s1 (>) s2 : A (>) B]]} \\
         {[[D1, x : A, y : B, D2 |-l s3 : C]]}
      }{[[D1, G1, G2, D2 |-l let s1 (>) s2 : A (>) B be x (>) y in s3 : C]]}
    \end{math}
  \end{center}
  normalizes to
  \begin{center}
    \tiny
    \begin{math}
      $$\mprset{flushleft}
      \inferrule* [right={\tiny cut2}] {
        {[[G1 |-l s1 : A]]} \\
        $$\mprset{flushleft}
        \inferrule* [right={\tiny cut2}] {
          {[[G2 |-l s2 : B]]} \\
          {[[D1, x : X, y : Y, D2 |-l s3 : C]]}
        }{[[D1, x : X, G2, D2 |-l [s2 / y]s3 : C]]}
      }{[[D1, G1, G2, D2 |-l [s1 / x][s2 / y]s3 : C]]}
    \end{math}
  \end{center}
        
\item (\NDdruleSXXimprIName, \NDdruleSXXimprEName):
  \begin{center}
    \tiny
    \begin{math}
     $$\mprset{flushleft}
     \inferrule* [right={\tiny unitE2}] {
       $$\mprset{flushleft}
       \inferrule* [right={\tiny imprI}] {
         {[[G, x : A |-l s1 : B]]}
        }{[[G |-l \r x : A . s1 : A -> B]]} \\
         {[[D |-l s2 : A]]}
      }{[[G, D |-l appr (\r x : A . s1) s2 : B]]}
    \end{math}
  \end{center}
  normalizes to
  \begin{center}
    \tiny
    \begin{math}
      $$\mprset{flushleft}
      \inferrule* [right={\tiny cut2}] {
        {[[G, x : A |-l s1 : B]]} \\
        {[[D |-l s2 : A]]}
      }{[[G, D |-l [s2 / x]s1 : B]]}
    \end{math}
  \end{center}
        
\item (\NDdruleSXXimplIName, \NDdruleSXXimplEName):
  \begin{center}
    \tiny
    \begin{math}
     $$\mprset{flushleft}
     \inferrule* [right={\tiny unitE2}] {
       $$\mprset{flushleft}
       \inferrule* [right={\tiny implI}] {
         {[[x : A, G |-l s1 : B]]}
        }{[[G |-l \l x : A . s1 : B <- A]]} \\
         {[[D |-l s2 : A]]}
      }{[[D, G |-l appl (\l x : A . s1) s2 : B]]}
    \end{math}
  \end{center}
  normalizes to
  \begin{center}
    \tiny
    \begin{math}
      $$\mprset{flushleft}
      \inferrule* [right={\tiny cut2}] {
        {[[x : A, G |-l s1 : B]]} \\
        {[[D |-l s2 : A]]}
      }{[[D, G |-l [s2 / x]s1 : B]]}
    \end{math}
  \end{center}
        
\item (\NDdruleSXXFIName, \NDdruleSXXFEName):
  \begin{center}
    \tiny
    \begin{math}
      $$\mprset{flushleft}
      \inferrule* [right={\tiny FE}] {
        $$\mprset{flushleft}
        \inferrule* [right={\tiny FI}] {
          {[[I |-c y : X]]}
        }{[[I |-l F y : F X]]} \\
         {[[D1, x : X, D2 |-l s : A]]}
      }{[[D1, I, D2 |-l let F x : F X be F y in s : A]]}
    \end{math}
  \end{center}
  normalizes to
  \begin{center}
    \tiny
    \begin{math}
      $$\mprset{flushleft}
      \inferrule* [right={\tiny cut1}] {
        {[[I |-c y : X]]} \\
        {[[D1, x : X, D2 |-l s : A]]}
      }{[[D1, I, D2 |-l [y / x]s : A]]}
    \end{math}
  \end{center}

\item (\NDdruleTXXGIName, \NDdruleSXXGEName):
  \begin{center}
    \tiny
    \begin{math}
      $$\mprset{flushleft}
      \inferrule* [right={\tiny GE}] {
        $$\mprset{flushleft}
        \inferrule* [right={\tiny GI}] {
          {[[I |-l s : A]]}
        }{[[I |-c Gf s : Gf A]]}
      }{[[I |-l derelict Gf s : A]]}
    \end{math}
  \end{center}
  normalizes to
  \begin{center}
    \tiny
    $[[I |-l s : A]]$
  \end{center}

\end{itemize}





































