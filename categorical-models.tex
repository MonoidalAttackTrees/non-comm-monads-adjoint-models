\begin{definition}
  A \textbf{SMCC-Lambek model} consists of
  \begin{itemize}
  \item a symmetric monoidal closed category $(\cat{C},\otimes,I,\alpha,\lambda,\rho)$;
  \item a Lambek category $(\cat{L},\otimes',I',\alpha',\lambda',\rho')$;
  \item a monoidal adjunction $F:\cat{C}\dashv\cat{L}:G$, where $F:\cat{C}\rightarrow\cat{L}$
        and $G:\cat{L}\rightarrow\cat{C}$ are monoidal functors.
  \end{itemize}
\end{definition}

Thus, in a SMCC-Lambek model, the following four diagrams commute because $\eta$ and
$\varepsilon$ are monoidal natural transformations:
\begin{mathpar}
\bfig
  \square/=`->`->`/<1600,400>[
    A\otimes B`A\otimes B`GFA\otimes GFB`GF(A\otimes B);
    id_{A\otimes B}`\eta_A\otimes\eta_B`\eta_{A\otimes B}`]
  \morphism<800,0>[GFA\otimes GFB`G(FA\otimes FB);\n{FA,FB}]
  \morphism(800,0)<800,0>[G(FA\otimes FB)`GF(A\otimes B);G\m{A,B}]
\efig
\and
\bfig
  \square/->`=`<-`->/<400,400>[I`GFI`I`GI';\eta_I``G\m{I}`\n{I'}]
\efig
\end{mathpar}
\begin{mathpar}
\bfig
  \square/`->`->`=/<1600,400>[
    FGA\otimes FGB`FG(A\otimes B)`A\otimes B`A\otimes B;
    `
    \varepsilon_A\otimes\varepsilon_B`\varepsilon_{A\otimes B}`]
  \morphism(0,400)<800,0>[FGA\otimes FGB`F(GA\otimes GB);\m{GA,GB}]
  \morphism(800,400)<800,0>[F(GA\otimes GB)`FG(A\otimes B);F\n{A,B}]
\efig
\and
\bfig
  \square/->`<-`=`<-/<400,400>[FGI'`I'`FI`I';\varepsilon_{I'}`F\n{I'}``\m{I}]
\efig
\end{mathpar}
And the following two diagrams commute because of the adjunction:
\begin{mathpar}
\bfig
  \Vtriangle/->`=`->/<400,400>[FA`FGFA`FA;F\eta_A``\varepsilon_{FA}]
\efig
\and
\bfig
  \Vtriangle/->`=`->/<400,400>[GB`GFGB`GB;\eta_{GX}``G\varepsilon_B]
\efig
\end{mathpar}

\begin{lemma}
  \label{lem:monoidal-monad}
  The monad on the SMCC $\cat{C}$ in a SMCC-Lambek model is monoidal.
\end{lemma}
\begin{proof}
  We define the monad $T$ on the $\cat{C}$ in the adjunction of a SMCC-Lambek model as $T=GF$,
  and the two corresponding natural transformations $\eta:id_\cat{C}\rightarrow T$ and
  $\mu:T^2\rightarrow T$ are defined as
  $$\eta_A:A\rightarrow GFA$$
  $$\mu_A=G\varepsilon_{FA}:GFGFA\rightarrow GFA$$
  where $\eta$ is the unit and $\varepsilon:FG\rightarrow id_\cat{L}$ is the counit in the
  adjunction $F:\cat{C}\dashv\cat{L}:G$, and $(F,\m{})$ and $(G,\n{})$ are monoidal functors. \\
  Thus, we have
  $$\t{A,B}=G\m{A,B}\circ\n{FA,FB}:TA\otimes TB\rightarrow T(A\otimes B)$$
  $$\t{I}=G\m{I}\circ\n{I'}:I\rightarrow TI$$
  The monad $T$ being monoidal means
  \begin{enumerate}
  \item $T$ is a monoidal functor, i.e. the folllowing diagrams commute:
        \begin{mathpar}
        \bfig
          \hSquares/->`->`->``->`->`->/<400>[
            (TA\otimes TB)\otimes TC`TA\otimes(TB\otimes TC)`TA\otimes T(B\otimes C)`
            T(A\otimes B)\otimes TC`T((A\otimes B)\otimes C)`T(A\otimes(B\otimes C));
            \alpha_{TA,TB,TC}`id_{TA}\otimes\t{B,C}`\t{A,B}\otimes id_{TC}``
            \t{A,B\otimes C}`\t{A\otimes B,C}`T\alpha_{A,B,C}]
        \efig
        \and
        \bfig
          \square/->`->`<-`->/<600,400>[
            I\otimes TA`TA`TI\otimes TA`T(I\otimes A);
            \lambda_{TA}`\t{I}\otimes id_{TA}`T\lambda_A`\t{I,A}]
        \efig
        \and
        \bfig
          \square/->`->`<-`->/<600,400>[
            TA\otimes I`TA`TA\otimes TI`T(A\otimes I);
            \rho_{TA}`id_{TA}\otimes\t{I}`T\rho_A`\t{A,I}]
        \efig
        \end{mathpar}
        We write $GF$ instead of $T$ in the proof for clarity. \\
        By replacing $\t{}$ with its definition, the first diagram above commutes by the
        following commutative diagram, where the two hexagons commute because $G$ and $F$ are
        monoidal functors, and the two quadrilaterals commute by the naturality of $\n{}$.
        \begin{mathpar}
        \bfig
          \iiixiii/->`->`->``->```->`<-`->``/<1400,400>[
            (GFA\otimes GFB)\otimes GFC`GFA\otimes(GFB\otimes GFC)`GFA\otimes G(FB\otimes'FC)`
            G(FA\otimes'FB)\otimes GFC`G(FA\otimes'(FB\otimes'FC))`GFA\otimes GF(B\otimes C)`
            GF(A\otimes B)\otimes GFC`G((FA\otimes'FB)\otimes'FC)`G(FA\otimes'F(B\otimes C));
            \alpha_{GFA,GFB,GFC}`id_{GFA}\otimes\n{FB,FC}`\n{FA,FB}\otimes id_{GFC}``
            id_{GFA}\otimes G\m{B,C}```G\m{A,B}\otimes id_{GFC}`G\alpha'_{FA,FB,FC}`
            \n{FA,F(B\otimes C)}``]
          \morphism(2800,800)|m|<-1400,-400>[
            GFA\otimes G(FB\otimes'FC)`G(FA\otimes'(FB\otimes'FC));\n{FA,FB\otimes'FC}]
          \morphism(0,400)|m|<1400,-400>[
            G(FA\otimes'FB)\otimes GFC`G((FA\otimes'FB)\otimes'FC);\n{FA\otimes'FB,FC}]
          \morphism(1400,400)|m|<1400,-400>[
            G(FA\otimes'(FB\otimes'FC))`G(FA\otimes'F(B\otimes C));G(id_{FA}\otimes'\m{B,C})]
          \ptriangle(0,-400)|mlm|/`->`->/<1400,400>[
            GF(A\otimes B)\otimes GFC`G((FA\otimes'FB)\otimes'FC)`G(F(A\otimes B)\otimes'FC);
            `\n{F(A\otimes B),FC}`G(\m{A,B}\otimes id_{FC})]
          \morphism(0,-400)|b|<1400,0>[
            G(F(A\otimes B)\otimes'FC)`GF((A\otimes B)\otimes C);G\m{A\otimes B,C}]
          \dtriangle(1400,-400)|mrb|/`->`->/<1400,400>[
            G(FA\otimes'F(B\otimes C))`GF((A\otimes B)\otimes C)`GF(A\otimes(B\otimes C));
            `G\m{A,B\otimes C}`GF\alpha_{A,B,C}]
        \efig
        \end{mathpar}
        The first square above commutes by the following commutative diagrams, in which the top
        quadrilateral commutes because $G$ is monoidal, the right quadrilateral commutes because
        $F$ is monoidal, and the left square commutes by the naturality of $\n{}$.
        \begin{mathpar}
        \bfig
          \ptriangle/->`->`/<1600,400>[
            I\otimes GFA`GFA`GI'\otimes GFA;\lambda_{GFA}`\n{I'}\otimes id_{GFA}`]
          \square(0,-400)|lmmb|<800,400>[
            GI'\otimes GFA`G(I'\otimes'FA)`GFI\otimes GFA`G(FI\otimes'FA);
            \n{I',FA}`G\m{I}\otimes id_{GFA}`G(\m{I}\otimes'id_{FA})`\n{FI,FA}]
          \morphism(800,0)|m|<800,400>[G(I'\otimes'FA)`GFA;G\lambda'_{FA}]
          \dtriangle(800,-400)/`<-`->/<800,800>[
            GFA`G(FI\otimes'FA)`GF(I\otimes A);
            `GF\lambda_A`G\m{I,A}]
        \efig
        \end{mathpar}
        Similarly, the second square above commutes by the following commutative diagram:
        \begin{mathpar}
        \bfig
          \ptriangle/->`->`/<1600,400>[
            GFA\otimes I`GFA`GFA\otimes GI';\rho_{GFA}`id_{GFA}\otimes\n{I'}`]
          \square(0,-400)|lmmb|<800,400>[
            GFA\otimes GI'`G(FA\otimes'I')`GFA\otimes GFI`G(FA\otimes'FI);
            \n{FA,I'}`id_{GFA}\otimes G\m{I}`G(id_{FA}\otimes\m{I})`\n{FA,FI}]
          \morphism(800,0)|m|<800,400>[G(FA\otimes'I')`GFA;G\rho'_{FA}]
          \dtriangle(800,-400)/`<-`->/<800,800>[
            GFA`G(FA\otimes'FI)`GF(A\otimes I);
            `GF\rho_A`G\m{A,I}]
        \efig
        \end{mathpar}
  \item $\eta$ is a monoidal natural transformation. In fact, since $\eta$ is the unit of the
        monoidal adjunction, $\eta$ is monoidal and thus the following two diagrams commute.
        \begin{mathpar}
        \bfig
          \square/=`->`->`->/<600,400>[
            A\otimes B`A\otimes B`TA\otimes TB`T(A\otimes B);
            `\eta_A\otimes\eta_B`\eta_{A\otimes B}`\t{A,B}]
        \efig
        \and
        \bfig
          \Vtriangle/->`=`<-/<400,400>[I`TI`I;\eta_I``\t{I}]
        \efig
        \end{mathpar}
  \item $\mu$ is a monoidal natural transformation. It is obvious that since
        $\mu=G\varepsilon_{FA}$ and $\varepsilon$ is monoidal, so is $\mu$. Thus the following
        diagrams commute.
        \begin{mathpar}
        \bfig
          \square/`->`->`->/<1500,400>[
            T^2A\otimes T^2B`T^2(A\otimes B)`TA\otimes TB`T(A\otimes B);
            `\mu_A\otimes\mu_B`\mu_{A\otimes B}`\t{A,B}]
          \morphism(0,400)<800,0>[T^2A\otimes T^2B`T(TA\otimes TB);\t{TA,TB}]
          \morphism(800,400)<700,0>[T(TA\otimes TB)`T^2(A\otimes B);T\t{A,B}]
        \efig
        \and
        \bfig
          \square/->`<-`<-`<-/<400,400>[T^2I`TI`TI`I;\mu_I`T\t{I}`\t{I}`\t{I}]
        \efig
        \end{mathpar}
  \end{enumerate}
\end{proof}

However, the monad is not symmetric becauase the following diagram does not commute, for the
lambek category $\cat{L}$ is not symmetric.
\begin{mathpar}
\bfig
  \ptriangle/->`->`/<900,400>[
    GFA\otimes GFB`GFB\otimes GFA`G(FA\otimes'FB);\e{GFA,GFB}`\n{FA,FB}`]
  \morphism(900,400)<900,0>[GFB\otimes GFA`G(FB\otimes' FA);\n{FB,FA}]
  \dtriangle(900,0)/`->`->/<900,400>[
    G(FB\otimes'FA)`GF(A\otimes B)`GF(B\otimes A);`G\m{B,A}`GF\e{A,B}]
  \morphism|b|<900,0>[G(FA\otimes'FB)`GF(A\otimes B);G\m{A,B}]
\efig
\end{mathpar}

\begin{lemma}
  \label{lem:strong-monad}
  The monad on the SMCC in a SMCC-Lambek model is strong.
\end{lemma}
\begin{proof}
  Let $F:\cat{C}\vdash\cat{L}:G$ be a SMCC-Lambek model, where
  $(\cat{C},\otimes,I,\alpha,\lambda,\rho)$ is an SMCC,
  $(\cat{L},\otimes',I',\alpha',\lambda',\rho')$ is a Lambek category, and $(F,\m{})$ and
  $(G,\n{})$ are monoidal functors. Let $(T,\eta,\mu)$ be the monad on $\cat{C}$ where
  $T=GF$. We have proved that $T$ is monoidal with the natural transformation
  $\t{A,B}:TA\otimes TB\rightarrow T(A\otimes B)$ and the morphism $\t{I}:I\rightarrow TI$
  defined as in Lemma~\ref{lem:monoidal-monad}. \\
  We define the tensorial strength $\tau_{A,B}:A\otimes TB\rightarrow T(A\otimes B)$ as
  $\tau_{A,B}=\t{A,B}\circ\eta_A\otimes id_{TB}$. \\
  Since $\eta$ is a monoidal natural transformation, we have $\eta_I=G\m{I}\circ\n{I'}$.
  Therefore $\eta_I=\t{I}$. Thus the following diagram commutes because $T$ is monoidal,
  where the composition $\t{I,A}\circ\t{I}\otimes id_{TA}$ is the definition of $\tau_{I,A}$. So
  the first triangle in Defition~\ref{def:strong-monad} commutes.
  \begin{mathpar}
  \bfig
    \square/->`->`->`<-/<600,400>[
      I\otimes TA`TI\otimes TA`TA`T(I\otimes A);
      \t{I}\otimes id_{TA}`\lambda_{TA}`\t{I,A}`T\lambda_A]
  \efig
  \end{mathpar}
  Similarly, by using the definition of $\tau$, the the second triangle in the definition is
  equivalent to the following diagram, which commutes because $\eta$ is a monoidal natural
  transformation:
  \begin{mathpar}
  \bfig
    \square/->`->`->`<-/<600,400>[
      A\otimes B`A\otimes TB`T(A\otimes B)`TA\otimes TB;
      id_A\otimes\eta_B`\eta_{A\otimes B}`\eta_A\otimes id_{TB}`\t{A,B}]
    \morphism(0,400)|m|<600,-400>[A\otimes B`TA\otimes TB;\eta_A\otimes\eta_B]
  \efig
  \end{mathpar}
  The first pentagon in the definition commutes by the following commutative diagrams, because
  $\eta$ are $\alpha$ natural transformations and $T$ is monoidal:
  \begin{mathpar}
  \bfig
    \qtriangle|amm|/->`->`<-/<1000,400>[
      (A\otimes B)\otimes TC`T(A\otimes B)\otimes TC`(TA\otimes TB)\otimes TC;
      \eta_{A\otimes B}\otimes id_{TC}`
      (\eta_A\otimes\eta_B)\otimes id_{TC}`
      \t{A,B}\otimes id_{TC}]
    \morphism(0,400)<0,-400>[(A\otimes B)\otimes TC`A\otimes(B\otimes TC);\alpha_{A,B,TC}]
    \morphism(1000,0)|m|<0,-400>[
      (TA\otimes TB)\otimes TC`TA\otimes(TB\otimes TC);\alpha_{TA,TB,TC}]
    \Dtriangle(0,-800)|lmm|/->`->`<-/<1000,400>[
      A\otimes(B\otimes TC)`TA\otimes(TB\otimes TC)`A\otimes(TB\otimes TC);
      id_A\otimes(\eta_B\otimes id_{TC})`
      \eta_A\otimes(\eta_B\otimes id_{TC})`
      \eta_A\otimes id_{TB\otimes TC}]
    \morphism(0,-800)|b|<1000,0>[
      A\otimes(TB\otimes TC)`A\otimes T(B\otimes C);id_A\otimes\t{B,C}]
    \qtriangle(1000,0)|amr|/->``->/<1000,400>[
      T(A\otimes B)\otimes TC`T((A\otimes B)\otimes C)`T(A\otimes(B\otimes C));
      \t{A\otimes B,C}``T\alpha_{A,B,C}]
    \morphism(2000,-800)<0,800>[
      TA\otimes T(B\otimes C)`T(A\otimes(B\otimes C));\t{A,B\otimes C}]
    \btriangle(1000,-800)|mmb|/`->`->/<1000,400>[
      TA\otimes(TB\otimes TC)`A\otimes T(B\otimes C)`TA\otimes T(B\otimes C);
      `id_{TA}\otimes\t{B,C}`\eta_A\otimes id_{T(B\otimes C)}]
  \efig
  \end{mathpar}
  The last diagram in the definition commtues by the following commutative diagram, because
  $T$ is a monad, $\t{}$ is a natural transformation, and $\mu$ is a monoidal natural
  transformation:
  \begin{mathpar}
  \bfig
    \ptriangle/->`->`/<700,400>[
      A\otimes T^2B`TA\otimes T^2B`A\otimes TB;\eta_A\otimes id_{T^2B}`id_A\otimes\mu_B`]
    \btriangle(0,-400)/->``->/<700,400>[
      A\otimes TB`TA\otimes TB`T(A\otimes B);\eta_A\otimes id_{TB}``\t{A,B}]
    \morphism(700,400)|m|<-700,-800>[TA\otimes T^2B`TA\otimes TB;id_{TA}\otimes\mu_B]
    \morphism(700,0)|m|<-700,-400>[TA\otimes T^2B`TA\otimes TB;id_{TA}\otimes\mu_B]
    \qtriangle(700,0)/->``->/<1800,400>[
      TA\otimes T^2B`T(A\otimes TB)`T(TA\otimes TB);\t{A,TB}``T(\eta_A\otimes id_{TB})]
    \btriangle(700,0)|mmm|/=`->`<-/<900,400>[
      TA\otimes T^2B`TA\otimes T^2B`T^2A\otimes T^2B;
      `T\eta_A\otimes id_{T^2B}`\mu_A\otimes id_{T^2B}]
    \morphism(1600,0)|m|<900,0>[T^2A\otimes T^2B`T(TA\otimes TB);\t{TA,TB}]
    \morphism(1600,0)|m|<-1600,-400>[T^2A\otimes T^2B`TA\otimes TB;\mu_A\otimes\mu_B]
    \dtriangle(700,-400)/`->`<-/<1800,400>[
      T(TA\otimes TB)`T(A\otimes B)`T^2(A\otimes B);`T\t{A,B}`\mu_{A\otimes B}]
  \efig
  \end{mathpar}
\end{proof}

\begin{lemma}[\cite{kock1970monads}]
  Let $\cat{M}$ be a symmetric monoidal category and $T$ be a strong monad on $\cat{M}$. Then
  $T$ is a symmetric monoidal functor iff it is commutative.
\end{lemma}

\begin{theorem}
  The monad on the SMCC in a SMCC-Lambek model is monoidal and non-commutative.
\end{theorem}

\begin{lemma}
  The comonad on the Lambek category in a SMCC-Lambek model is monoidal.
\end{lemma}
\begin{proof}
  We define the comonad $S$ on the Lambek category $\cat{L}$ in the adjunction
  $F:\cat{C}\vdash\cat{L}:G$ of a SMCC-Lambek model as $S=FG$, and the two corresponding natural
  transformations $\varepsilon:S\rightarrow id_\cat{L}$ and $\delta:S\rightarrow S^2$ are
  defined as
  $$\varepsilon_A:SA\rightarrow A$$
  $$\delta_A=F\eta_{GA}:SA\rightarrow S^2A$$
  where $\varepsilon$ is the counit and $\eta:id_\cat{L}\rightarrow GF$ is the unit in the
  adjunction, and $(F,\m{})$ and $(G,\n{})$ are monoidal functors. Thus, we have
  $$\s{A,B}=F\n{A,B}\circ\m{GA,GB}:SA\otimes'SB\rightarrow SA\otimes'SB$$
  $$\s{I}=F\n{I'}\circ\m{I}:I'\rightarrow SI'$$
  The comonad $S$ being monoidal means
  \begin{enumerate}
  \item $S$ is a monoidal functor, i.e. the following diagrams commute:
        \begin{mathpar}
        \bfig
          \hSquares/->`->`->``->`->`->/<400>[
            (SA\otimes'SB)\otimes'SC`SA\otimes'(SB\otimes'SC)`SA\otimes'S(B\otimes'C)`
            S(A\otimes'B)\otimes'SC`S((A\otimes'B)\otimes'C)`S(A\otimes'(B\otimes'C));
            \alpha_{SA,SB,SC}'`id_{SA}\otimes'\s{B,C}`\s{A,B}\otimes'id_{SC}``
            \s{A,B\otimes'C}`\s{A\otimes'B,C}`S\alpha_{A,B,C}']
        \efig
        \and
        \bfig
          \square/->`->`<-`->/<600,400>[
            I'\otimes'SA`SA`SI'\otimes'SA`S(I'\otimes'A);
            \lambda_{SA}'`\s{I'}\otimes'id_{SA}`S\lambda_A'`\s{I',A}]
        \efig
        \and
        \bfig
          \square/->`->`<-`->/<600,400>[
            SA\otimes'I'`SA`SA\otimes'SI'`S(A\otimes'I');
            \rho_{SA}'`id_{SA}'\otimes'\s{I'}`S\rho_A'`\s{A,I'}]
        \efig
        \end{mathpar}
  \item $\varepsilon$ is a monoidal natural transformation:
        \begin{mathpar}
        \bfig
          \square/->`->`->`=/<600,400>[
            SA\otimes'SB`S(A\otimes'B)`A\otimes'B`A\otimes'B;
            \s{A,B}`\varepsilon_A\otimes'\varepsilon_B`\varepsilon_{A\otimes'B}`]
        \efig
        \and
        \bfig
          \Vtriangle/->`<-`=/<400,400>[SI'`I'`I';\varepsilon_{I'}`\s{I'}`]
        \efig
        \end{mathpar}
  \item $\delta$ is a monoidal natural transformation:
        \begin{mathpar}
        \bfig
          \square/->`->`->`/<1500,400>[
            SA\otimes'SA`S(A\otimes'B)`S^2A\otimes'S^2B`S^2(A\otimes'B);
            \s{A,B}`\delta_A\otimes'\delta_B`\delta_{A\otimes'B}`]
          \morphism<800,0>[S^2A\otimes'S^2B`S(SA\otimes'SB);\s{SA,SB}]
          \morphism(800,0)<700,0>[S(SA\otimes'SB)`S^2(A\otimes'B);S\s{A,B}]
        \efig
        \and
        \bfig
          \square/->`<-`<-`->/<400,400>[
            SI'`S^2I'`I'`SI';\delta_{I'}`\s{I'}`S\s{I'}`\s{I'}]
        \efig
        \end{mathpar}
  \end{enumerate}
  The proof for the commutativity of the diagrams are similar as the proof in
  Lemma~\ref{lem:monoidal-monad}. We do not include the proof here for simplicity.
\end{proof}


%%% Local Variables: 
%%% mode: latex
%%% TeX-master: main.tex
%%% End: 






















