\begin{definition}
\label{def:mc}
  A \textbf{monoidal category} $(\cat{M},\otimes,I,\alpha,\lambda,\rho)$ is a category $\cat{M}$
  consists of
  \begin{itemize}
  \item a bifunctor $\otimes:\cat{M}\times\cat{M}\rightarrow\cat{M}$, called the tensor product;
  \item an object $I$, called the unit object;
  \item three natural isomorphisms $\alpha$, $\lambda$, and $\rho$ with components
        $$\alpha_{A,B,C}:(A\otimes B)\otimes C\rightarrow A\otimes(B\otimes C)$$
        $$\lambda_A:I\otimes A\rightarrow A$$
        $$\rho_A:A\otimes I\rightarrow A$$
        where $\alpha$ is called associator, $\lambda$ is left unitor, and $\rho$ is right
        unitor,
  \end{itemize}
  such that the following diagrams commute for any objects $A$, $B$, $C$ in $\cat{M}$:
  \begin{mathpar}
  \bfig
    \square/`->`->`->/<2100,400>[
      ((A\otimes B)\otimes C)\otimes D`
      A\otimes((B\otimes C)\otimes D)`
      (A\otimes B)\otimes(C\otimes D)`
      A\otimes(B\otimes(C\otimes D));
      `
      \alpha_{A\otimes B,C,D}`
      id_A\otimes\alpha_{B,C,D}`
      \alpha_{A,B,C\otimes D}]
    \morphism(0,400)<1100,0>[
      ((A\otimes B)\otimes C)\otimes D`
      (A\otimes(B\otimes C))\otimes D;
      \alpha_{A,B,C}\otimes id_D]
    \morphism(1100,400)<1000,0>[
      (A\otimes(B\otimes C))\otimes D`
      A\otimes((B\otimes C)\otimes D);
      \alpha_{A,B\otimes C,D}]
  \efig
  \and
  \bfig
    \Vtriangle<400,400>[
      (A\otimes I)\otimes B`
      A\otimes(I\otimes B)`
      A\otimes B;
      \alpha_{A,I,B}`
      \rho_A\otimes id_B`
      id_A\otimes\lambda_B]
  \efig
  \end{mathpar}
\end{definition}

\begin{definition}
  A \textbf{Lambek category} (or a \textbf{biclosed monoidal category}) is a monoidal category
  $(\cat{M},\otimes,I,\alpha,\lambda,\rho)$ equipped with two bifunctors
  $\rightharpoonup:\cat{M}^{op}\times\cat{M}\rightarrow\cat{M}$ and
  $\leftharpoonup:\cat{M}\times\cat{M}^{op}\rightarrow\cat{M}$ that are both right adjoint to
  the tensor product. That is, the following natural bijections hold:
  \begin{center}
  \begin{math}
  \begin{array}{lll}
    \Hom{L}{X\otimes A}{B}\cong\Hom{L}{X}{A\lto B} & \quad\quad\quad\quad & 
    \Hom{L}{A\otimes X}{B}\cong\Hom{L}{X}{B\rto A}
  \end{array}
  \end{math}
  \end{center}
\end{definition}

\begin{definition}
  A \textbf{symmetric monoidal category} is a monoidal category
  $(\cat{M},\otimes,I,\alpha,\lambda,\rho)$ together with a natural transformation with
  components $\e{A,B}:A\otimes B\rightarrow B\otimes A$, called exchange, such that the
  following diagrams commute:
  \begin{mathpar}
  \bfig
    \Vtriangle<300,400>[A\otimes I`I\otimes A`A;\e{A,I}`\rho_A`\lambda_A]
  \efig
  \and
  \bfig
    \Vtriangle/=`->`<-/<300,400>[
      A\otimes B`A\otimes B`B\otimes A;
      id_{A\otimes B}`\e{A,B}`\e{B,A}]
  \efig
  \and
  \bfig
    \hSquares/->`->`->``->`->`->/<400>[
      (A\otimes B)\otimes C`A\otimes(B\otimes C)`(B\otimes C)\otimes A`
      (B\otimes A)\otimes C`B\otimes(A\otimes C)`B\otimes(C\otimes A);
      \alpha_{A,B,C}`\e{A,B\otimes C}`\e{A,B}\otimes id_C``
      \alpha_{B,A,C}`\alpha_{B,A,C}`id_B\otimes\e{A,C}]
  \efig
  \end{mathpar}
\end{definition}

\begin{definition}
  A \textbf{symmetric monoidal closed category} $(\cat{M},\otimes,I,\alpha,\lambda,\rho)$ is a
  symmetric monoidal category equipped with a bifunctor
  $\limp:\cat{M}^{op}\times\cat{M}\rightarrow\cat{M}$ that is right adjoint to the tensor
  product. That is, the following natural bijection
  $\Hom{\cat{M}}{X\otimes A}{B}\cong\Hom{\cat{M}}{X}{A\limp B}$ holds.
\end{definition}

\begin{lemma}
  \label{lemma:internal-homs-collapse}
  Let $A$ and $B$ be two objects in a Lambek category with the exchange natural transformation.
  Then $(A \lto B) \cong (B \rto A)$.
\end{lemma}
\begin{proof}
  First, notice that for any object $C$ we have
  \begin{center}
  \begin{math}
  \small
  \begin{array}{lllll}
    Hom[C,A\lto B]
    & \cong & Hom[C\otimes A,B] & \cat{L}\text{ is a Lambek category}\\
    & \cong & Hom[A\otimes C,B] & \text{By the exchange }\e{C,A}\\
    & \cong & Hom[C,B\rto A]    & \cat{L}\text{ is a Lambek category}
  \end{array}
  \end{math}
  \end{center}  
  Thus, $A\lto B\cong B\rto A$ by the Yoneda lemma.
\end{proof}
\begin{corollary}
  \label{corollary:LC-with-ex-mc}
  A Lambek category with exchange is symmetric monoidal closed.
\end{corollary}

\begin{definition}
  Let $(\cat{M},\otimes,I,\alpha,\lambda,\rho)$ and
  $(\cat{M'},\otimes',I',\alpha',\lambda',\rho')$ be monoidal categories. A \textbf{monoidal
  functor} $(F,\m{})$ from $\cat{M}$ to $\cat{M'}$ is a functor $F:\cat{M}\rightarrow\cat{M'}$
  together with a morphism $\m{I}:I'\rightarrow F(I)$ and a natural transformation
  $\m{A,B}:FA'\otimes FB'\rightarrow F(A\otimes B)$, such that the following diagrams commute
  for any objects $A$, $B$, and $C$ in $\cat{M}$:
  \begin{mathpar}
  \bfig
    \hSquares/->`->`->``->`->`->/<400>[
      (FA\otimes'FB)\otimes'FC`FA\otimes'(FB\otimes'FC)`FA\otimes'F(B\otimes C)`
      F(A\otimes B)\otimes'FC`F((A\otimes B)\otimes C)`F(A\otimes(B\otimes C));
      \alpha'_{FA,FB,FC}`id_{FA}\otimes'\m{A,B}`\m{A,B}\otimes'id_{FC}``
      \m{A,B\otimes C}`\m{A\otimes B,C}`F\alpha_{A,B,C}]
  \efig
  \and
  \bfig
    \square/->`->`<-`->/<600,400>[
      I'\otimes'FA`FA`FI\otimes'FA`F(I\otimes A);
      \lambda'_{FA}`\m{I}\otimes id_{FA}`F\lambda_A`\m{I,A}]
  \efig
  \and
  \bfig
    \square/->`->`<-`->/<600,400>[
      FA\otimes'I'`FA`FA\otimes'FI`F(A\otimes I);
      \rho'_{FA}`\id_{FA}\otimes\m{I}`F\rho_A`\m{A,I}]
  \efig
  \end{mathpar}
\end{definition}

\begin{definition}
  Let $(\cat{M},\otimes,I,\alpha,\lambda,\rho)$ and
  $(\cat{M'},\otimes',I',\alpha',\lambda',\rho')$ be monoidal categories. A \textbf{symmetric
  monoidal functor} $F:\cat{M}\rightarrow\cat{M'}$ is a monoidal functor $(F,m)$ that satisfies
  the following coherence diagram:
  \begin{mathpar}
  \bfig
    \square<700,400>[
      FA\otimes'FB`FB\otimes'FA`F(A\otimes B)`F(B\otimes A);
      \e{FA,FB}`\m{A,B}`\m{B,A}`F\e{A,B}]
  \efig
  \end{mathpar}
\end{definition}

\begin{definition}
  An \textbf{adjunction} between categories $\cat{C}$ and $\cat{D}$ consists of two functors
  $F:\cat{D}\rightarrow\cat{C}$, called the \textbf{left adjoint}, and
  $G:\cat{C}\rightarrow\cat{D}$, called the \textbf{right adjoint}, and two natural
  transformations $\eta:id_\cat{D}\rightarrow GF$, called the \textbf{unit}, and
  $\varepsilon:FG\rightarrow id_\cat{C}$, called the \textbf{counit}, such that the following
  diagrams commute for any object $A$ in $\cat{C}$ and $B$ in $\cat{D}$:
  \begin{mathpar}
  \bfig
    \Vtriangle/->`=`->/<400,400>[FB`FGFB`FB;F\eta_B``\varepsilon_{FB}]
  \efig
  \and
  \bfig
    \Vtriangle/->`=`->/<400,400>[GA`GFGA`GA;\eta_{GA}``G\varepsilon_A]
  \efig
  \end{mathpar}
\end{definition}

\begin{definition}
  Let $(F,\m{})$ and $(G,\n{})$ be monoidal functors from a monoidal category $\cat{M}$ to a
  monoidal category $\cat{M'}$. A \textbf{monoidal natural transformation} from $(F,\m{})$ to 
  $(G,\n{})$ is a natural transformation $\theta:(F,\m{})\rightarrow(G,\n{})$ such that the
  following diagrams commute for any objects $A$ and $B$ in $\cat{M}$:
  \begin{mathpar}
  \bfig
    \square<700,400>[
      FA\otimes'FB`F(A\otimes B)`GA\otimes'GB`G(A\otimes B);
      \m{A,B}`\theta_A\otimes'\theta_B`\theta_{A\otimes B}`\n{A,B}]
  \efig
  \and
  \bfig
    \Vtriangle/->`<-`<-/<400,400>[FI`GI`I';\theta_I`\m{I}`\n{I}]
  \efig
  \end{mathpar}
\end{definition}

\begin{definition}
  Let $(\cat{M},\otimes,I,\alpha,\lambda,\rho)$ and
  $(\cat{M'},\otimes',I',\alpha',\lambda',\rho')$ be monoidal categories,
  $F:\cat{M}\rightarrow\cat{M'}$ and $G:\cat{M}'\rightarrow\cat{M}$ be functors. The adjunction
  $F:\cat{M}\dashv\cat{M'}:G$ is a \textbf{monoidal adjunction} if $F$ and $G$ are monoidal
  functors, and the unit $\eta$ and the counit $\varepsilon$ are monoidal natural
  transformations.
\end{definition}

\begin{definition}
  A \textbf{SMCC-Lambek model} consists of
  \begin{itemize}
  \item a symmetric monoidal closed category $(\cat{C},\otimes,I,\alpha,\lambda,\rho)$;
  \item a Lambek category $(\cat{L},\otimes',I',\alpha',\lambda',\rho')$;
  \item a monoidal adjunction $F:\cat{C}\dashv\cat{L}:G$, where $F:\cat{C}\rightarrow\cat{L}$
        and $G:\cat{L}\rightarrow\cat{C}$ are monoidal functors.
  \end{itemize}
\end{definition}

Thus, in a SMCC-Lambek model, the following four diagrams commute because $\eta$ and
$\varepsilon$ are monoidal natural transformations:
\begin{mathpar}
\bfig
  \square/=`->`->`/<1600,400>[
    A\otimes B`A\otimes B`GFA\otimes GFB`GF(A\otimes B);
    id_{A\otimes B}`\eta_A\otimes\eta_B`\eta_{A\otimes B}`]
  \morphism<800,0>[GFA\otimes GFB`G(FA\otimes FB);\n{FA,FB}]
  \morphism(800,0)<800,0>[G(FA\otimes FB)`GF(A\otimes B);G\m{A,B}]
\efig
\and
\bfig
  \square/->`=`<-`->/<400,400>[I`GFI`I`GI';\eta_I``G\m{I}`\n{I'}]
\efig
\end{mathpar}
\begin{mathpar}
\bfig
  \square/`->`->`=/<1600,400>[
    FGA\otimes FGB`FG(A\otimes B)`A\otimes B`A\otimes B;
    `
    \varepsilon_A\otimes\varepsilon_B`\varepsilon_{A\otimes B}`]
  \morphism(0,400)<800,0>[FGA\otimes FGB`F(GA\otimes GB);\m{GA,GB}]
  \morphism(800,400)<800,0>[F(GA\otimes GB)`FG(A\otimes B);F\n{A,B}]
\efig
\and
\bfig
  \square/->`<-`=`<-/<400,400>[FGI'`I'`FI`I';\varepsilon_{I'}`F\n{I'}``\m{I}]
\efig
\end{mathpar}
And the following two diagrams commute because of the adjunction:
\begin{mathpar}
\bfig
  \Vtriangle/->`=`->/<400,400>[FA`FGFA`FA;F\eta_A``\varepsilon_{FA}]
\efig
\and
\bfig
  \Vtriangle/->`=`->/<400,400>[GB`GFGB`GB;\eta_{GX}``G\varepsilon_B]
\efig
\end{mathpar}

\begin{definition}
  Let $\cat{C}$ be a category. A \textbf{monad} on $\cat{C}$ consists of an endofunctor
  $T:\cat{C}\rightarrow\cat{C}$ together with two natural transformations
  $\eta:id_\cat{C}\rightarrow T$ and $\mu:T^2\rightarrow id_\cat{C}$, where $id_\cat{C}$
  is the identity functor on $\cat{C}$, such that the following diagrams commute:
  \begin{mathpar}
  \bfig
    \square<400,400>[T^3`T^2`T^2`T;T\mu`\mu_T`\mu`\mu]
  \efig
  \and
  \bfig
    \square<400,400>[T`T^2`T^2`T;\eta_T`T\eta`\mu`\mu]
    \morphism(0,400)/=/<400,-400>[T`T;]
  \efig
  \end{mathpar}
\end{definition}

\begin{lemma}
  The monad on the SMCC $\cat{C}$ in a SMCC-Lambek model is monoidal.
\end{lemma}
\begin{proof}
  We define the monad $T$ on the $\cat{C}$ in the adjunction of a SMCC-Lambek model as $T=GF$,
  and the two corresponding natural transformations $\eta:id_\cat{C}\rightarrow T$ and
  $\mu:T^2\rightarrow T$ are defined as
  $$\eta:id_\cat{C}\rightarrow GF$$
  $$\mu=GF\varepsilon_A=\varepsilon_{GFA}:GFGF\rightarrow GF$$
  where $\eta$ is the unit and $\mu$ is the counit in the adjunction $F:\cat{C}\dashv\cat{L}:G$,
  and $(F,m)$ and $(G,n)$ are monoidal functors. \\
  Thus, we have
  $$\q{A,B}=G\m{A,B}\circ\n{FA,FB}:TA\otimes TB\rightarrow T(A\otimes B)$$
  $$\q{I}=G\m{I}\circ\n{I'}:I\rightarrow TI$$
  The monad $T$ being monoidal means
  \begin{enumerate}
  \item $T$ is a monoidal functor i.e. the folllowing diagrams commute:
        \begin{mathpar}
        \bfig
          \hSquares/->`->`->``->`->`->/<400>[
            (TA\otimes TB)\otimes TC`TA\otimes(TB\otimes TC)`TA\otimes T(B\otimes C)`
            T(A\otimes B)\otimes TC`T((A\otimes B)\otimes C)`T(A\otimes(B\otimes C));
            \alpha_{TA,TB,TC}`id_{TA}\otimes\q{B,C}`\q{A,B}\otimes id_{TC}``
            \q{A,B\otimes C}`\q{A\otimes B,C}`T\alpha_{A,B,C}]
        \efig
        \and
        \bfig
          \square/->`->`<-`->/<600,400>[
            I\otimes TA`TA`TI\otimes TA`T(I\otimes A);
            \lambda_{TA}`\q{I}\otimes id_{TA}`T\lambda_A`\q{I,A}]
        \efig
        \and
        \bfig
          \square/->`->`<-`->/<600,400>[
            TA\otimes I`TA`TA\otimes TI`T(A\otimes I);
            \rho_{TA}`id_{TA}\otimes\q{I}`T\rho_A`\q{A,I}]
        \efig
        \end{mathpar}
        We write $GF$ instead of $T$ in the diagram chasings for clarity. \\
        By replacing $\q{}$ with its definition, the first diagram above commutes by the
        following diagram chasing, where the two hexagons commute because $G$ and $F$ are
        monoidal functors, and the two quadrilaterals commute by the naturality of $\n{}$.
        \begin{mathpar}
        \bfig
          \iiixiii/->`->`->``->```->`<-`->``/<1400,400>[
            (GFA\otimes GFB)\otimes GFC`GFA\otimes(GFB\otimes GFC)`GFA\otimes G(FB\otimes'FC)`
            G(FA\otimes'FB)\otimes GFC`G(FA\otimes'(FB\otimes'FC))`GFA\otimes GF(B\otimes C)`
            GF(A\otimes B)\otimes GFC`G((FA\otimes'FB)\otimes'FC)`G(FA\otimes'F(B\otimes C));
            \alpha_{GFA,GFB,GFC}`id_{GFA}\otimes\n{FB,FC}`\n{FA,FB}\otimes id_{GFC}``
            id_{GFA}\otimes G\m{B,C}```G\m{A,B}\otimes id_{GFC}`G\alpha'_{FA,FB,FC}`
            \n{FA,F(B\otimes C)}``]
          \morphism(2800,800)|m|<-1400,-400>[
            GFA\otimes G(FB\otimes'FC)`G(FA\otimes'(FB\otimes'FC));\n{FA,FB\otimes'FC}]
          \morphism(0,400)|m|<1400,-400>[
            G(FA\otimes'FB)\otimes GFC`G((FA\otimes'FB)\otimes'FC);\n{FA\otimes'FB,FC}]
          \morphism(1400,400)|m|<1400,-400>[
            G(FA\otimes'(FB\otimes'FC))`G(FA\otimes'F(B\otimes C));G(id_{FA}\otimes'\m{B,C})]
          \ptriangle(0,-400)|mlm|/`->`->/<1400,400>[
            GF(A\otimes B)\otimes GFC`G((FA\otimes'FB)\otimes'FC)`G(F(A\otimes B)\otimes'FC);
            `\n{F(A\otimes B),FC}`G(\m{A,B}\otimes id_{FC})]
          \morphism(0,-400)|b|<1400,0>[
            G(F(A\otimes B)\otimes'FC)`GF((A\otimes B)\otimes C);G\m{A\otimes B,C}]
          \dtriangle(1400,-400)|mrb|/`->`->/<1400,400>[
            G(FA\otimes'F(B\otimes C))`GF((A\otimes B)\otimes C)`GF(A\otimes(B\otimes C));
            `G\m{A,B\otimes C}`GF\alpha_{A,B,C}]
        \efig
        \end{mathpar}
        The first square above commutes by the following diagram chasing, in which the top
        quadrilateral commutes because $G$ is monoidal, the right quadrilateral commutes because
        $F$ is monoidal, and the left square commutes by the naturality of $\n{}$.
        \begin{mathpar}
        \bfig
          \ptriangle/->`->`/<1600,400>[
            I\otimes GFA`GFA`GI'\otimes GFA;\lambda_{GFA}`\n{I'}\otimes id_{GFA}`]
          \square(0,-400)|lmmb|<800,400>[
            GI'\otimes GFA`G(I'\otimes'FA)`GFI\otimes GFA`G(FI\otimes'FA);
            \n{I',FA}`G\m{I}\otimes id_{GFA}`G(\m{I}\otimes'id_{FA})`\n{FI,FA}]
          \morphism(800,0)|m|<800,400>[G(I'\otimes'FA)`GFA;G\lambda'_{FA}]
          \dtriangle(800,-400)/`<-`->/<800,800>[
            GFA`G(FI\otimes'FA)`GF(I\otimes A);
            `GF\lambda_A`G\m{I,A}]
        \efig
        \end{mathpar}
        Similarly, the second square above commutes by the following diagram chasing:
        \begin{mathpar}
        \bfig
          \ptriangle/->`->`/<1600,400>[
            GFA\otimes I`GFA`GFA\otimes GI';\rho_{GFA}`id_{GFA}\otimes\n{I'}`]
          \square(0,-400)|lmmb|<800,400>[
            GFA\otimes GI'`G(FA\otimes'I')`GFA\otimes GFI`G(FA\otimes'FI);
            \n{FA,I'}`id_{GFA}\otimes G\m{I}`G(id_{FA}\otimes\m{I})`\n{FA,FI}]
          \morphism(800,0)|m|<800,400>[G(FA\otimes'I')`GFA;G\rho'_{FA}]
          \dtriangle(800,-400)/`<-`->/<800,800>[
            GFA`G(FA\otimes'FI)`GF(A\otimes I);
            `GF\rho_A`G\m{A,I}]
        \efig
        \end{mathpar}
  \item $\eta$ is a monoidal natural transformation, i.e. the following diagrams commute. In
        fact, since $\eta$ is the unit of the monoidal adjunction, $\eta$ is monoidal and thus
        the following two diagrams commute.
        \begin{mathpar}
        \bfig
          \square/=`->`->`->/<600,400>[
            A\otimes B`A\otimes B`TA\otimes TB`T(A\otimes B);
            `\eta_A\otimes\eta_B`\eta_{A\otimes B}`\q{A,B}]
        \efig
        \and
        \bfig
          \Vtriangle/->`=`<-/<400,400>[I`TI`I;\eta_I``\q{I}]
        \efig
        \end{mathpar}
  \item $\mu$ is a monoidal natural transformation, i.e. the following diagrams commute. Since
        $\mu=\varepsilon_{GFA}$, $\mu$ is obviously also monoidal since $\varepsilon$ also is.
        Thus the following diagrams commute.
        \begin{mathpar}
        \bfig
          \square/`->`->`->/<1500,400>[
            T^2A\otimes T^2B`T^2(A\otimes B)`TA\otimes TB`T(A\otimes B);
            `\mu_A\otimes\mu_B`\mu_{A\otimes B}`\q{A,B}]
          \morphism(0,400)<800,0>[T^2A\otimes T^2B`T(TA\otimes TB);\q{TA,TB}]
          \morphism(800,400)<700,0>[T(TA\otimes TB)`T^2(A\otimes B);T\q{A,B}]
        \efig
        \and
        \bfig
          \square/->`<-`<-`<-/<400,400>[T^2I`TI`TI`I;\mu_I`T\q{I}`\q{I}`\q{I}]
        \efig
        \end{mathpar}
  \end{enumerate}
\end{proof}

However, the monad $T$ we get from the SMCC-Lambek model is not symmetric because the following
diagram does not commute:
\begin{mathpar}
\bfig
  \hSquares/->`->`->``->`->`->/<400>[
    GFA\otimes GFB`GFB\otimes GFA`G(FB\otimes'FA)`G(FA\otimes'FB)`GF(A\otimes B)`GF(B\otimes A);
    \e{GFA,GFB}`\n{FB,FA}`\n{FA,FB}``G\m{B,A}`G\m{A,B}`GF\e{A,B}]
\efig
\end{mathpar}

Therefore, the monad is non-commutative.



%%% Local Variables: 
%%% mode: latex
%%% TeX-master: main.tex
%%% End: 






















