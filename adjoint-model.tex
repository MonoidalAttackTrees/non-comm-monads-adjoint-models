Our adjoint model, SMCC-Lambek model, has a similar structure as Benton's LNL model \cite{}.
Benton's LNL model consists of a symmetric monoidal adjunction $F:\cat{C}\dashv\cat{L}:G$
between a cartesian closed category $\cat{C}$ and a symmetric monoidal closed category
$\cat{L}$.

\begin{definition}
  A \textbf{SMCC-Lambek model} consists of
  \begin{itemize}
  \item a symmetric monoidal closed category $(\cat{C},\otimes,I,\alpha,\lambda,\rho)$;
  \item a Lambek category $(\cat{L},\tri,I',\alpha',\lambda',\rho')$;
  \item a monoidal adjunction $F:\cat{C}\dashv\cat{L}:G$, where $F:\cat{C}\rightarrow\cat{L}$
        and $G:\cat{L}\rightarrow\cat{C}$ are monoidal functors.
  \end{itemize}
\end{definition}

Thus, in a SMCC-Lambek model, the following four diagrams commute because $\eta$ and
$\varepsilon$ are monoidal natural transformations:
\begin{mathpar}
\bfig
  \square/=`->`->`/<1600,400>[
    X\otimes Y`X\otimes Y`GFX\otimes GFY`GF(X\otimes Y);
    id_{X\otimes Y}`\eta_X\otimes\eta_Y`\eta_{X\otimes Y}`]
  \morphism<800,0>[GFX\otimes GFY`G(FX\otimes FY);\n{FX,FY}]
  \morphism(800,0)<800,0>[G(FX\otimes FY)`GF(X\otimes Y);G\m{X,Y}]
\efig
\and
\bfig
  \square/->`=`<-`->/<400,400>[I`GFI`I`GI';\eta_I``G\m{I}`\n{I'}]
\efig
\end{mathpar}
\begin{mathpar}
\bfig
  \square/`->`->`=/<1600,400>[
    FGA\otimes FGB`FG(A\otimes B)`A\otimes B`A\otimes B;
    `
    \varepsilon_A\otimes\varepsilon_B`\varepsilon_{A\otimes B}`]
  \morphism(0,400)<800,0>[FGA\otimes FGB`F(GA\otimes GB);\m{GA,GB}]
  \morphism(800,400)<800,0>[F(GA\otimes GB)`FG(A\otimes B);F\n{A,B}]
\efig
\and
\bfig
  \square/->`<-`=`<-/<400,400>[FGI'`I'`FI`I';\varepsilon_{I'}`F\n{I'}``\m{I}]
\efig
\end{mathpar}
And the following two diagrams commute because of the adjunction:
\begin{mathpar}
\bfig
  \Vtriangle/->`=`->/<400,400>[FX`FGFX`FX;F\eta_X``\varepsilon_{FX}]
\efig
\and
\bfig
  \Vtriangle/->`=`->/<400,400>[GA`GFGA`GA;\eta_{GA}``G\varepsilon_A]
\efig
\end{mathpar}
Following the tradition, we use letters $X$, $Y$, $Z$ for objects in $\cat{C}$ and $A$, $B$, $C$
for objects in $\cat{L}$. The following lemmas and theorems establish the essential properties
of the monad and the comomad derived from the adjunction.

\begin{lemma}
  \label{lem:monoidal-monad}
  The monad on the symmetric monoidal closed category $\cat{C}$ in a SMCC-Lambek model is
  monoidal.
\end{lemma}
\begin{proof}
  We define the monad $T$ on the $\cat{C}$ in the adjunction of a SMCC-Lambek model as $T=GF$,
  and the two corresponding natural transformations $\eta:id_\cat{C}\rightarrow T$ and
  $\mu:T^2\rightarrow T$ as
  $$\eta_X:X\rightarrow GFX \qquad\qquad\qquad \mu_X=G\varepsilon_{FX}:GFGFX\rightarrow GFX$$
  where $\eta$ is the unit and $\varepsilon:FG\rightarrow id_\cat{L}$ is the counit of the
  adjunction $F:\cat{C}\dashv\cat{L}:G$. Since the adjunction is monoidal, then $(F,\m{})$ and
  $(G,\n{})$ are monoidal functors. Thus, we have
  $$\t{X,Y}=G\m{X,Y}\circ\n{FX,FY}:TX\otimes TY\rightarrow T(X\otimes Y)
  \qquad\qquad\qquad\t{I}=G\m{I}\circ\n{I'}:I\rightarrow TI$$
  The monad $T$ being monoidal means
  \begin{enumerate}
  \item $T$ is a monoidal functor, i.e. the folllowing diagrams commute:
        \begin{mathpar}
        \bfig
          \hSquares/->`->`->``->`->`->/<400>[
            (TX\otimes TY)\otimes TZ`TX\otimes(TY\otimes TZ)`TX\otimes T(Y\otimes Z)`
            T(X\otimes Y)\otimes TZ`T((X\otimes Y)\otimes Z)`T(X\otimes(Y\otimes Z));
            \alpha_{TX,TY,TZ}`id_{TX}\otimes\t{Y,Z}`\t{X,Y}\otimes id_{TZ}``
            \t{X,Y\otimes Z}`\t{X\otimes Y,Z}`T\alpha_{X,Y,Z}]
          \morphism(1300,200)//<0,0>[`;(1)]
        \efig
        \and
        \bfig
          \square/->`->`<-`->/<600,400>[
            I\otimes TX`TX`TI\otimes TX`T(I\otimes X);
            \lambda_{TX}`\t{I}\otimes id_{TX}`T\lambda_X`\t{I,X}]
          \morphism(350,200)//<0,0>[`;(2)]
        \efig
        \and
        \bfig
          \square/->`->`<-`->/<600,400>[
            TX\otimes I`TX`TX\otimes TI`T(X\otimes I);
            \rho_{TX}`id_{TX}\otimes\t{I}`T\rho_X`\t{X,I}]
          \morphism(350,200)//<0,0>[`;(3)]
        \efig
        \end{mathpar}
        We write $GF$ instead of $T$ in the proof for clarity. \\
        By replacing $\t{X,Y}$ with its definition, diagram (1) above commutes by the following
        commutative diagram, in which the two hexagons commute because $G$ and $F$ are monoidal
        functors, and the two quadrilaterals commute by the naturality of $\n{}$.
        \begin{mathpar}
        \bfig
          \iiixiii/->`->`->``->```->`<-`->``/<1400,400>[
            (GFX\otimes GFY)\otimes GFZ`GFX\otimes(GFY\otimes GFZ)`GFX\otimes G(FY\tri FZ)`
            G(FX\tri FY)\otimes GFZ`G(FX\tri(FY\tri FZ))`GFX\otimes GF(Y\otimes Z)`
            GF(X\otimes Y)\otimes GFZ`G((FX\tri FY)\tri FZ)`G(FX\tri F(Y\otimes Z));
            \alpha_{GFX,GFY,GFZ}`id_{GFX}\otimes\n{FY,FZ}`\n{FX,FY}\otimes id_{GFZ}``
            id_{GFX}\otimes G\m{Y,Z}```G\m{X,Y}\otimes id_{GFZ}`G\alpha'_{FX,FY,FZ}`
            \n{FX,F(Y\otimes Z)}``]
          \morphism(2800,800)|m|<-1400,-400>[
            GFX\otimes G(FY\tri FZ)`G(FX\tri(FY\tri FZ));\n{FX,FY\tri FZ}]
          \morphism(0,400)|m|<1400,-400>[
            G(FX\tri FY)\otimes GFZ`G((FX\tri FY)\tri FZ);\n{FX\tri FY,FZ}]
          \morphism(1400,400)|m|<1400,-400>[
            G(FX\tri(FY\tri FZ))`G(FX\tri F(Y\otimes Z));G(id_{FX}\tri\m{Y,Z})]
          \ptriangle(0,-400)|mlm|/`->`->/<1400,400>[
            GF(X\otimes Y)\otimes GFZ`G((FX\tri FY)\tri FZ)`G(F(X\otimes Y)\tri FZ);
            `\n{F(X\otimes Y),FZ}`G(\m{X,Y}\otimes id_{FZ})]
          \morphism(0,-400)|b|<1400,0>[
            G(F(X\otimes Y)\tri FZ)`GF((X\otimes Y)\otimes Z);G\m{X\otimes Y,Z}]
          \dtriangle(1400,-400)|mrb|/`->`->/<1400,400>[
            G(FX\tri F(Y\otimes Z))`GF((X\otimes Y)\otimes Z)`GF(X\otimes(Y\otimes Z));
            `G\m{X,Y\otimes Z}`GF\alpha_{X,Y,Z}]
        \efig
        \end{mathpar}
        Diagram (2) commutes by the following commutative diagrams, in which the top
        quadrilateral commutes because $G$ is monoidal, the right quadrilateral commutes because
        $F$ is monoidal, and the left square commutes by the naturality of $\n{}$.
        \begin{mathpar}
        \bfig
          \ptriangle/->`->`/<1600,400>[
            I\otimes GFX`GFX`GI'\otimes GFX;\lambda_{GFX}`\n{I'}\otimes id_{GFX}`]
          \square(0,-400)|lmmb|<800,400>[
            GI'\otimes GFX`G(I'\tri FX)`GFI\otimes GFX`G(FI\tri FX);
            \n{I',FX}`G\m{I}\otimes id_{GFX}`G(\m{I}\tri id_{FX})`\n{FI,FX}]
          \morphism(800,0)|m|<800,400>[G(I'\tri FX)`GFX;G\lambda'_{FX}]
          \dtriangle(800,-400)/`<-`->/<800,800>[
            GFX`G(FI\tri FX)`GF(I\otimes X);
            `GF\lambda_X`G\m{I,X}]
        \efig
        \end{mathpar}
        Similarly, diagram (3) commutes as follows:
        \begin{mathpar}
        \bfig
          \ptriangle/->`->`/<1600,400>[
            GFX\otimes I`GFX`GFX\otimes GI';\rho_{GFX}`id_{GFX}\otimes\n{I'}`]
          \square(0,-400)|lmmb|<800,400>[
            GFX\otimes GI'`G(FX\tri I')`GFX\otimes GFI`G(FX\tri FI);
            \n{FX,I'}`id_{GFX}\otimes G\m{I}`G(id_{FX}\otimes\m{I})`\n{FX,FI}]
          \morphism(800,0)|m|<800,400>[G(FX\tri I')`GFX;G\rho'_{FX}]
          \dtriangle(800,-400)/`<-`->/<800,800>[
            GFX`G(FX\tri FI)`GF(X\otimes I);
            `GF\rho_X`G\m{X,I}]
        \efig
        \end{mathpar}
  \item $\eta$ is a monoidal natural transformation. In fact, since $\eta$ is the unit of the
        monoidal adjunction, $\eta$ is monoidal by definition and thus the following two
        diagrams commute.
        \begin{mathpar}
        \bfig
          \square/=`->`->`->/<600,400>[
            X\otimes Y`X\otimes Y`TX\otimes TY`T(X\otimes Y);
            `\eta_X\otimes\eta_Y`\eta_{X\otimes Y}`\t{X,Y}]
        \efig
        \and
        \bfig
          \Vtriangle/->`=`<-/<400,400>[I`TI`I;\eta_I``\t{I}]
        \efig
        \end{mathpar}
  \item $\mu$ is a monoidal natural transformation. It is obvious that since
        $\mu=G\varepsilon_{FA}$ and $\varepsilon$ is monoidal, so is $\mu$. Thus the following
        diagrams commute.
        \begin{mathpar}
        \bfig
          \square/`->`->`->/<1500,400>[
            T^2X\otimes T^2Y`T^2(X\otimes Y)`TX\otimes TY`T(X\otimes Y);
            `\mu_X\otimes\mu_Y`\mu_{X\otimes Y}`\t{X,Y}]
          \morphism(0,400)<800,0>[T^2X\otimes T^2Y`T(TX\otimes TY);\t{TX,TY}]
          \morphism(800,400)<700,0>[T(TX\otimes TY)`T^2(X\otimes Y);T\t{X,Y}]
        \efig
        \and
        \bfig
          \square/->`<-`<-`<-/<400,400>[T^2I`TI`TI`I;\mu_I`T\t{I}`\t{I}`\t{I}]
        \efig
        \end{mathpar}
  \end{enumerate}
\end{proof}

However, the monad is not symmetric becauase the following diagram does not commute, for the
lambek category $\cat{L}$ is not symmetric.
\begin{mathpar}
\bfig
  \ptriangle/->`->`/<900,400>[
    GFX\otimes GFY`GFY\otimes GFX`G(FX\tri FY);\e{GFX,GFY}`\n{FX,FY}`]
  \morphism(900,400)<900,0>[GFY\otimes GFX`G(FY\tri FX);\n{FY,FX}]
  \dtriangle(900,0)/`->`->/<900,400>[
    G(FY\tri FX)`GF(X\otimes Y)`GF(Y\otimes X);`G\m{Y,X}`GF\e{X,Y}]
  \morphism|b|<900,0>[G(FX\tri FY)`GF(X\otimes Y);G\m{X,Y}]
\efig
\end{mathpar}

\begin{lemma}
  \label{lem:strong-monad}
  The monad on the symmetric monoidal closed category in a SMCC-Lambek model is strong.
\end{lemma}
\begin{proof}
  Let $F:\cat{C}\vdash\cat{L}:G$ be a SMCC-Lambek model, where
  $(\cat{C},\otimes,I,\alpha,\lambda,\rho)$ is symmetric monoidal closed,
  $(\cat{L},\tri,I',\alpha',\lambda',\rho')$ is a Lambek category, and $(F,\m{})$ and
  $(G,\n{})$ are monoidal functors. We have proved that the monad $(T=GF,\eta,\mu)$ is monoidal
  with the natural transformation $\t{X,Y}:TX\otimes TY\rightarrow T(X\otimes Y)$ and the
  morphism $\t{I}:I\rightarrow TI$ defined as in Lemma~\ref{lem:monoidal-monad}. \\
  We define the tensorial strength $\tau_{X,Y}:X\otimes TY\rightarrow T(X\otimes Y)$ as
  $\tau_{X,Y}=\t{X,Y}\circ(\eta_X\otimes id_{TY})$. \\
  Since $\eta$ is a monoidal natural transformation, we have $\eta_I=G\m{I}\circ\n{I'}$.
  Therefore $\eta_I=\t{I}$. Thus the following diagram commutes because $T$ is monoidal,
  where the composition $\t{I,X}\circ(\t{I}\otimes id_{TX})$ is the definition of $\tau_{I,X}$.
  So the first triangle in Defition~\ref{def:strong-monad} commutes.
  \begin{mathpar}
  \bfig
    \square/->`->`->`<-/<600,400>[
      I\otimes TX`TI\otimes TX`TX`T(I\otimes X);
      \t{I}\otimes id_{TX}`\lambda_{TX}`\t{I,X}`T\lambda_X]
  \efig
  \end{mathpar}
  Similarly, by using the definition of $\tau$, the the second triangle in the definition is
  equivalent to the following diagram, which commutes because $\eta$ is a monoidal natural
  transformation:
  \begin{mathpar}
  \bfig
    \square/->`->`->`<-/<600,400>[
      X\otimes Y`X\otimes TY`T(X\otimes Y)`TX\otimes TY;
      id_X\otimes\eta_Y`\eta_{X\otimes Y}`\eta_X\otimes id_{TY}`\t{X,Y}]
    \morphism(0,400)|m|<600,-400>[X\otimes Y`TX\otimes TY;\eta_X\otimes\eta_Y]
  \efig
  \end{mathpar}
  The first pentagon in the definition commutes by the following commutative diagrams, because
  $\eta$ are $\alpha$ natural transformations and $T$ is monoidal:
  \begin{mathpar}
  \bfig
    \qtriangle|amm|/->`->`<-/<1000,400>[
      (X\otimes Y)\otimes TZ`T(X\otimes Y)\otimes TZ`(TX\otimes TY)\otimes TZ;
      \eta_{X\otimes Y}\otimes id_{TZ}`
      (\eta_X\otimes\eta_Y)\otimes id_{TZ}`
      \t{X,Y}\otimes id_{TZ}]
    \morphism(0,400)<0,-400>[(X\otimes Y)\otimes TZ`X\otimes(Y\otimes TZ);\alpha_{X,Y,TZ}]
    \morphism(1000,0)|m|<0,-400>[
      (TX\otimes TY)\otimes TZ`TX\otimes(TY\otimes TZ);\alpha_{TX,TY,TZ}]
    \Dtriangle(0,-800)|lmm|/->`->`<-/<1000,400>[
      X\otimes(Y\otimes TZ)`TX\otimes(TY\otimes TZ)`X\otimes(TY\otimes TZ);
      id_X\otimes(\eta_Y\otimes id_{TZ})`
      \eta_X\otimes(\eta_Y\otimes id_{TZ})`
      \eta_X\otimes id_{TY\otimes TZ}]
    \morphism(0,-800)|b|<1000,0>[
      X\otimes(TY\otimes TZ)`X\otimes T(Y\otimes Z);id_X\otimes\t{Y,Z}]
    \qtriangle(1000,0)|amr|/->``->/<1000,400>[
      T(X\otimes Y)\otimes TZ`T((X\otimes Y)\otimes Z)`T(X\otimes(Y\otimes Z));
      \t{X\otimes Y,Z}``T\alpha_{X,Y,Z}]
    \morphism(2000,-800)<0,800>[
      TX\otimes T(Y\otimes Z)`T(X\otimes(Y\otimes Z));\t{X,Y\otimes Z}]
    \btriangle(1000,-800)|mmb|/`->`->/<1000,400>[
      TX\otimes(TY\otimes TZ)`X\otimes T(Y\otimes Z)`TX\otimes T(Y\otimes Z);
      `id_{TX}\otimes\t{Y,Z}`\eta_X\otimes id_{T(Y\otimes Z)}]
  \efig
  \end{mathpar}
  The last diagram in the definition commtues by the following commutative diagram, because
  $T$ is a monad, $\t{}$ is a natural transformation, and $\mu$ is a monoidal natural
  transformation:
  \begin{mathpar}
  \bfig
    \ptriangle/->`->`/<700,400>[
      X\otimes T^2Y`TX\otimes T^2Y`X\otimes TY;\eta_X\otimes id_{T^2Y}`id_X\otimes\mu_Y`]
    \btriangle(0,-400)/->``->/<700,400>[
      X\otimes TY`TX\otimes TY`T(X\otimes Y);\eta_X\otimes id_{TY}``\t{X,Y}]
    \morphism(700,400)|m|<-700,-800>[TX\otimes T^2Y`TX\otimes TY;id_{TX}\otimes\mu_Y]
    \morphism(700,0)|m|<-700,-400>[TX\otimes T^2Y`TX\otimes TY;id_{TX}\otimes\mu_Y]
    \qtriangle(700,0)/->``->/<1800,400>[
      TX\otimes T^2Y`T(X\otimes TY)`T(TX\otimes TY);\t{X,TY}``T(\eta_X\otimes id_{TY})]
    \btriangle(700,0)|mmm|/=`->`<-/<900,400>[
      TX\otimes T^2Y`TX\otimes T^2Y`T^2X\otimes T^2Y;
      `T\eta_X\otimes id_{T^2Y}`\mu_X\otimes id_{T^2Y}]
    \morphism(1600,0)|m|<900,0>[T^2X\otimes T^2Y`T(TX\otimes TY);\t{TX,TY}]
    \morphism(1600,0)|m|<-1600,-400>[T^2X\otimes T^2Y`TX\otimes TY;\mu_X\otimes\mu_Y]
    \dtriangle(700,-400)/`->`<-/<1800,400>[
      T(TX\otimes TY)`T(X\otimes Y)`T^2(X\otimes Y);`T\t{X,Y}`\mu_{X\otimes Y}]
  \efig
  \end{mathpar}
\end{proof}

\begin{lemma}[\cite{kock1970monads}]
  Let $\cat{M}$ be a symmetric monoidal category and $T$ be a strong monad on $\cat{M}$. Then
  $T$ is symmetric iff it is commutative.
\end{lemma}

\begin{theorem}
  The monad on the SMCC in a SMCC-Lambek model is monoidal and non-commutative.
\end{theorem}

\begin{lemma}
  The comonad on the Lambek category in a SMCC-Lambek model is monoidal.
\end{lemma}
\begin{proof}
  We define the comonad $S$ on the Lambek category $\cat{L}$ in the adjunction
  $F:\cat{C}\vdash\cat{L}:G$ of a SMCC-Lambek model as $S=FG$. The two corresponding natural
  transformations $\varepsilon:S\rightarrow id_\cat{L}$ and $\delta:S\rightarrow S^2$ are
  defined as
  $$\varepsilon_A:SA\rightarrow A \qquad\qquad\qquad \delta_A=F\eta_{GA}:SA\rightarrow S^2A$$
  where $\varepsilon$ is the counit and $\eta:id_\cat{L}\rightarrow GF$ is the unit of the
  adjunction, and $(F,\m{})$ and $(G,\n{})$ are monoidal functors. Thus, we have
  $$\s{A,B}=F\n{A,B}\circ\m{GA,GB}:SA\tri SB\rightarrow SA\tri SB \qquad\qquad\qquad
  \s{I}=F\n{I'}\circ\m{I}:I'\rightarrow SI'$$
  The comonad $S$ being monoidal means
  \begin{enumerate}
  \item $S$ is a monoidal functor, i.e. the following diagrams commute:
        \begin{mathpar}
        \bfig
          \hSquares/->`->`->``->`->`->/<400>[
            (SA\tri SB)\tri SC`SA\tri(SB\tri SC)`SA\tri S(B\tri C)`
            S(A\tri B)\tri SC`S((A\tri B)\tri C)`S(A\tri(B\tri C));
            \alpha_{SA,SB,SC}'`id_{SA}\tri\s{B,C}`\s{A,B}\tri id_{SC}``
            \s{A,B\tri C}`\s{A\tri B,C}`S\alpha_{A,B,C}']
        \efig
        \and
        \bfig
          \square/->`->`<-`->/<600,400>[
            I'\tri SA`SA`SI'\tri SA`S(I'\tri A);
            \lambda_{SA}'`\s{I'}\tri id_{SA}`S\lambda_A'`\s{I',A}]
        \efig
        \and
        \bfig
          \square/->`->`<-`->/<600,400>[
            SA\tri I'`SA`SA\tri SI'`S(A\tri I');
            \rho_{SA}'`id_{SA}'\tri\s{I'}`S\rho_A'`\s{A,I'}]
        \efig
        \end{mathpar}
  \item $\varepsilon$ is a monoidal natural transformation:
        \begin{mathpar}
        \bfig
          \square/->`->`->`=/<600,400>[
            SA\tri SB`S(A\tri B)`A\tri B`A\tri B;
            \s{A,B}`\varepsilon_A\tri\varepsilon_B`\varepsilon_{A\tri B}`]
        \efig
        \and
        \bfig
          \Vtriangle/->`<-`=/<400,400>[SI'`I'`I';\varepsilon_{I'}`\s{I'}`]
        \efig
        \end{mathpar}
  \item $\delta$ is a monoidal natural transformation:
        \begin{mathpar}
        \bfig
          \square/->`->`->`/<1500,400>[
            SA\tri SA`S(A\tri B)`S^2A\tri S^2B`S^2(A\tri B);
            \s{A,B}`\delta_A\tri\delta_B`\delta_{A\tri B}`]
          \morphism<800,0>[S^2A\tri S^2B`S(SA\tri SB);\s{SA,SB}]
          \morphism(800,0)<700,0>[S(SA\tri SB)`S^2(A\tri B);S\s{A,B}]
        \efig
        \and
        \bfig
          \square/->`<-`<-`->/<400,400>[
            SI'`S^2I'`I'`SI';\delta_{I'}`\s{I'}`S\s{I'}`\s{I'}]
        \efig
        \end{mathpar}
  \end{enumerate}
  The proof for the commutativity of the diagrams are similar as the proof in
  Lemma~\ref{lem:monoidal-monad}. We do not include the proof here for simplicity.
\end{proof}

The comonad $S$ on the Lambek category $\cat{L}$ of the adjunction is clearly not symmetric
because $\cat{L}$ is not. However, it is symmetric on the co-Eilenberg-Moore category of the
comonad.

\begin{definition}
  Let $(S,\varepsilon,\delta)$ be a comonad on a category $\cat{L}$. Then the
  \textbf{co-Eilenberg-Moore category $\cat{L}^S$} of the comonad has
  \begin{itemize}
  \item as objects the S-coalgebras $(A,h_A:A\rightarrow SA)$, where $A$ is an object in
        $\cat{L}$, s.t. the following diagrams commute:
        \begin{mathpar}
        \bfig
          \square<400,400>[A`SA`SA`S^2A;h_A`h_A`\delta_A`Sh_A]
        \efig
        \and
        \bfig
          \Atriangle/<-`->`=/<200,400>[SA`A`A;h_A`\varepsilon_A`]
        \efig
        \end{mathpar}
  \item as morphisms the coalgebra morphisms, i.e. morphisms $f:(A,h_A)\rightarrow(B,h_B)$
        between coalgebras s.t. the diagram commutes:
        $$\bfig
          \square<400,400>[A`B`SA`SB;f`h_A`h_B`Sf]
        \efig$$
  \end{itemize}
\end{definition}

\begin{lemma}
  \label{lem:em-exchange}
  Given a SMCC-Lambek model $F:\cat{C}\dashv\cat{L}:G$ and the comonad $S$ on $\cat{L}$, the
  co-Eilenberg-Moore category $\cat{L}^S$ of has an exchange natural transformation
  $\e{A,B}^S:A\tri B\rightarrow B\tri A$.
\end{lemma}
\begin{proof}
  We define the exchange $\e{A,B}^S:A\tri B\rightarrow B\tri A$ as
  $$\bfig
    \morphism<600,0>[A\tri B`FGA\tri FGB;h_A\tri h_B]
    \morphism(600,0)<800,0>[FGA\tri FGB`F(GA\otimes GB);\m{GA,GB}]
    \morphism(1400,0)<800,0>[F(GA\otimes GB)`F(GB\otimes GA);F\e{GA,GB}]
    \morphism(2200,0)<700,0>[F(GB\otimes GA)`FG(B\tri A);F\n{B,A}]
    \morphism(2900,0)<500,0>[FG(B\tri A)`B\tri A;\varepsilon_{B\tri A}]
  \efig$$
  in which $(F,\m{})$ and $(G,\n{})$ are monoidal functors, and $\e{}$ is the exchange for
  $\cat{C}$. $\e{}^S$ is a natural transformation because the following diagrams commute for
  morphisms $f:A\rightarrow A'$ and $g:B\rightarrow B'$:
  \begin{mathpar}
  \bfig
    \square|almb|<700,400>[
      A\tri B`FGA\tri FGB`A'\tri B'`FGA'\tri FGB';
      h_A\tri h_B`f\tri g`FGf\tri FGg`h_{A'}\tri h_{B'}]
    \square(700,0)|ammb|/->``->`->/<800,400>[
      FGA\tri FGB`F(GA\otimes GB)`FGA'\tri FGB'`F(GA'\otimes GB');
      \m{GA,GB}``F(Gf\otimes Gg)`\m{GA',GB'}]
    \square(1500,0)|ammb|/->``->`->/<800,400>[
      F(GA\otimes GB)`F(GB\otimes GA)`F(GA'\otimes GB')`F(GB'\otimes GA');
      F\e{A,B}``F(Gg\otimes Gf)`F\e{A',B'}]
    \square(2300,0)|ammb|/->``->`->/<800,400>[
      F(GB\otimes GA)`FG(B\tri A)`F(GB'\otimes GA')`FG(B'\tri A');
      F\n{B,A}``FG(g\tri f)`F\n{B',A'}]
    \square(3100,0)|amrb|/->``->`->/<600,400>[
      FG(B\tri A)`B\tri A`FG(B'\tri A')`B'\tri A';
      \varepsilon_{B\tri A}``g\tri f`\varepsilon_{B'\tri A'}]
  \efig
  \end{mathpar}
\end{proof}

\begin{lemma}
  \label{lem:em-ex}
  The following diagrams commute in the co-Eilenberg-Moore category $\cat{L}^S$:
  \begin{mathpar}
  \bfig
    \iiixiii/->`->`->``->```->``->`->`->/<1300,400>[
      F((GA\otimes GB)\otimes GC)`F(G(A\tri B)\otimes GC)`FG((A\tri B)\tri C)`
      F(G(B\tri A)\otimes GC)``(A\tri B)\tri C`
      F(G(B\tri A)\otimes GC)`FG((B\tri A)\tri C)`(B\tri A)\tri C;
      F(\n{A,B}\otimes id_{GC})`F(\e{A,B}\otimes id_{GC})`
      F(\e{A,B}\otimes id_{GC})``\varepsilon_{(A\tri B)\tri C}```
      F(\n{B,A}\otimes id_{GC})``\e{A,B}^S\tri id_C`
      F\n{B\tri A,C}`\varepsilon_{(B\tri A)\tri C}]
  \efig
  \end{mathpar}
  \begin{mathpar}
  \bfig
    \iiixiii/->`->`->``->```->``->`->`->/<1300,400>[
      F(GB\otimes(GC\otimes GA))`F(GB\otimes G(C\tri A))`FG(B\tri(C\tri A))`
      F(GB\otimes(GA\otimes GC))``B\tri(C\tri A)`
      F(GB\otimes G(A\tri C))`FG(B\tri(A\tri C))`B\tri(A\tri C);
      F(id_{GB}\otimes\n{C,A})`F\n{B,C\tri A}`
      F(id_{GB}\otimes\e{C,A})``\varepsilon_{B\tri(C\tri A)}```
      F(id_{GB}\otimes\n{A,C})``id_A\tri\e{C,A}^S`
      F\n{B,A\tri C}`\varepsilon_{B\tri(A\tri C)}]
  \efig
  \end{mathpar}
\end{lemma}
\begin{proof}
  We only write the proof for the first diagram. The proof for the second one is similar.
  (1), (2), (3)--naturality of $\m{}$; (4)--F is monoidal; (5), (12)--$\varepsilon$ is monoidal;
  (6), (7), (8), (9), (10)--obvious; (11)--coalgebra.
  \begin{mathpar}
  \bfig
    \Vtrianglepair|aammm|/->`->`<-``<-/<1000,400>[
      F(G(A\tri B)\otimes GC)`FG((A\tri B)\tri C)`(A\tri B)\tri C`FG(A\tri B)\tri FGC;
      F\n{A\tri B,C}`\varepsilon_{(A\tri B)\tri C}`
      \m{G(A\tri B),GC}``\varepsilon_{A\tri B}\tri\varepsilon_C]
    \qtriangle(2000,0)|amr|/=`<-`->/<1000,400>[
      (A\tri B)\tri C`(A\tri B)\tri C`(FGA\tri FGB)\tri C;
      `(\varepsilon_A\tri\varepsilon_B)\tri id_C`(h_A\tri h_B)\tri id_C]
    \morphism|l|<0,400>[
      F((GA\otimes GB)\otimes GC)`F(G(A\tri B)\otimes GC);F(\n{A,B}\otimes id_{GC})]
    \dtriangle(1000,0)|mmm|/`<-`->/<1000,400>[
      (A\tri B)\tri C`FG(A\tri B)\tri FGC`FG(A\tri B)\tri C;
      `\varepsilon_{A\tri B}\tri id_C`id\tri\varepsilon_C]
    \btriangle(0,-400)|lmm|/->`<-`/<1000,400>[
      F((GA\otimes GB)\otimes GC)`F((GB\otimes GA)\otimes GC)`F(GA\otimes GB)\tri FGC;
      F(\e{A,B}\otimes id_{GC})`\m{GA\otimes GB,GC}`]
    \btriangle(1000,-400)|mmm|/<-``->/<2000,400>[
      FG(A\tri B)\tri FGC`F(GA\otimes GB)\tri FGC`F(GA\otimes GB)\tri C;
      F\n{A,B}\tri id_{FGC}``id_{F(GA\otimes GB)}\tri\varepsilon_C]
    \qtriangle(2000,-400)|mmr|/`<-`->/<1000,400>[
      FG(A\tri B)\tri C`(FGA\tri FGB)\tri C`F(GA\otimes GB)\tri C;
      `F\n{A,B}\otimes id_C`\m{GA,GB}\tri id_C]
    \btriangle(0,-800)|lmm|/->`<-`/<2000,400>[
      F((GB\otimes GA)\otimes GC)`F(G(B\tri A)\otimes GC)`F(GB\otimes GA)\tri FGC;
      F(\n{B,A}\otimes id_{GC})`\m{GB\otimes GA,GC}`]
    \morphism(1000,-400)|m|<1000,-400>[
      F(GA\otimes GB)\tri FGC`F(GB\otimes GA)\tri FGC;F\e{A,B}\tri id_{FGC}]
    \dtriangle(2000,-800)|mra|/`->`->/<1000,400>[
      F(GA\otimes GB)\tri C`F(GB\otimes GA)\tri FGC`F(GB\otimes GA)\tri C;
      `F\e{A,B}\tri id_C`id\tri\varepsilon_C]
    \square(0,-1200)|almb|/<-`->`->`->/<1000,400>[
      F(G(B\tri A)\otimes GC)`FG(B\tri A)\tri FGC`FG((B\tri A)\tri C)`(B\tri A)\tri C;
      \m{G(B\tri A),GC}`F\n{B\tri A,C}`
      \varepsilon_{B\tri A}\tri\varepsilon_C`\varepsilon_{(B\tri A)\tri C}]
    \morphism(1000,-800)|b|/<-/<1000,0>[
      FG(B\tri A)\tri FGC`F(GB\otimes GA)\tri FGC;F\n{B,A}\tri id]
    \dtriangle(1000,-1200)/`->`<-/<2000,400>[
      F(GB\otimes GA)\tri C`(B\tri A)\tri C`FG(B\tri A)\tri C;
      `F\n{B,A}\tri id_C`\varepsilon_{B\tri A}\tri id_C]
    \morphism(1000,-800)|m|<2000,-400>[
      FG(B\tri A)\tri FGC`FG(B\tri A)\tri C;id_{FG(B\tri A)}\tri\varepsilon_C]
    \morphism(500,50)//<0,0>[`;(1)]
    \morphism(500,-300)//<0,0>[`;(2)]
    \morphism(400,-550)//<0,0>[`;(3)]
    \morphism(500,-950)//<0,0>[`;(4)]
    \morphism(1000,300)//<0,0>[`;(5)]
    \morphism(1600,200)//<0,0>[`;(6)]
    \morphism(1800,-150)//<0,0>[`;(7)]
    \morphism(2400,-550)//<0,0>[`;(8)]
    \morphism(1300,-950)//<0,0>[`;(9)]
    \morphism(2700,-900)//<0,0>[`;(10)]
    \morphism(2800,300)//<0,0>[`;(11)]
    \morphism(2500,50)//<0,0>[`;(12)]
  \efig
  \end{mathpar}
\end{proof}

\begin{theorem}
  The co-Eilenberg-Moore category $\cat{L}^S$ of $S$ is symmetric monoidal closed.
\end{theorem}
\begin{proof}
  Let $(\cat{L},\tri,I',\alpha',\lambda',\rho')$ be the Lambek cateogry in a SMCC-Lambek model
  and $S$ be the comonad on $\cat{L}$. Since $\cat{L}$ is a Lambek category, it is obvious that
  $\cat{L}$ is also Lambek. By Corollary~\ref{corollary:LC-with-ex-mc}, we only need to prove
  the exchange defined in Lemma~\ref{lem:em-exchange} satisfies the three commutative diagrams
  in Definition~\ref{def:smcc}.

  The first triangle in Definition~\ref{def:smcc} commutes as follows:
  (1)--coalgebra; (2)--$\varepsilon$ is monoidal; (3)--naturality of $\rho$; (4)--naturality of
  $\varepsilon$; (5)--naturality of $\m{}$; (6)--$F$ is monoidal; (7)--$\cat{C}$ is symmetric;
  (8)--naturality of $\e{}$; (9)--$G$ is monoidal.
  \begin{mathpar}
  \bfig
    \Vtriangle/->`=`/<800,800>[A\tri I'`FGA\tri FGI'`A\tri I';h_A\tri h_{I'}``]
    \Ctrianglepair(1600,0)|mmmmm|/->`<-``=`<-/<800,400>[
      FGA\tri FGI'`FGA\tri I'`FGA\tri FI`FGA\tri I';
      id_{FGA}\tri\varepsilon_{I'}`id_{FGA}\tri F\n{I'}```id_{FGA}\tri\m{I}]
    \morphism(800,400)|m|<0,-400>[FGA\tri I'`A\tri I';\varepsilon_{FGA}\tri id_{I'}]
    \morphism(1600,400)|m|<800,0>[FGA\tri FI`F(GA\otimes I);\m{GA,I}]
    \morphism(1600,0)|m|<0,-400>[FGA\tri I'`FGA;\rho_{FGA}']
    \dtriangle(1600,-400)|mmm|/->`->`<-/<800,800>[
      F(GA\otimes I)`FGA`F(I\otimes GA);
      F\rho_{GA}`F\e{GA,I}`F\lambda_{GA}]
    \square(1600,-800)/->``->`<-/<1600,1600>[
      FGA\tri FGI'`F(GA\otimes GI')`FG(I'\tri A)`F(GI'\otimes GA);
      \m{GA,GI'}``F\e{GA,GI'}`F\n{I',A}]
    \morphism(2400,400)|m|<800,400>[F(GA\otimes I)`F(GA\otimes GI');F(id_{GA}\otimes\n{I'})]
    \morphism(2400,-400)|m|<800,-400>[F(I\otimes GA)`F(GI'\otimes GA);F(\n{I'}\otimes id_{GA})]
    \morphism(0,800)|l|<0,-1200>[A\tri I'`A;\rho_A']
    \square(0,-800)|mlmb|/<-`->`<-`<-/<1600,400>[
      A`FGA`I'\tri A`FG(I'\tri A);\varepsilon_A`\lambda_A'`FG\lambda_A'`\varepsilon_{I'\tri A}]
    \morphism(800,600)//<0,0>[`;(1)]
    \morphism(1300,400)//<0,0>[`;(2)]
    \morphism(800,-200)//<0,0>[`;(3)]
    \morphism(800,-600)//<0,0>[`;(4)]
    \morphism(2150,600)//<0,0>[`;(5)]
    \morphism(1950,200)//<0,0>[`;(6)]
    \morphism(2200,-200)//<0,0>[`;(7)]
    \morphism(2800,0)//<0,0>[`;(8)]
    \morphism(2200,-600)//<0,0>[`;(9)]
  \efig
  \end{mathpar}
  The second triangle in the proof commutes as follows: (1) and (5)--coalgebra; (2) and
  (4)--$\varepsilon$ is monoidal; (3)--$\cat{C}$ is symmetric.
  \begin{mathpar}
  \bfig
    \ptriangle|amm|/->`=`->/<600,600>[
      A\tri B`FGA\tri FGB`A\tri B;h_A\tri h_B``\varepsilon_A\tri\varepsilon_B]
    \morphism(600,0)|b|<-600,0>[FG(A\tri B)`A\tri B;\varepsilon_{A\tri B}]
    \morphism(600,600)<700,0>[FGA\tri FGB`F(GA\otimes GB);\m{GA,GB}]
    \morphism(1300,0)|b|<-700,0>[F(GA\otimes GB)`FG(A\tri B);F\n{A,B}]
    \square(1300,0)/->`=`=`<-/<800,600>[
      F(GA\otimes GB)`F(GB\otimes GA)`F(GA\otimes GB)`F(GB\otimes GA);F\e{A,B}```F\e{B,A}]
    \qtriangle(2100,0)|amr|/->``->/<1400,600>[F(GB\otimes GA)`FG(B\tri A)`B\tri A;F\n{B,A}``\varepsilon_{B\tri A}]
    \morphism(2900,0)|b|<-800,0>[FGB\tri FGA`F(GB\otimes GA);\m{GB,GA}]
    \btriangle(2900,0)|mmb|/<-`=`<-/<600,400>[
      B\tri A`FGB\tri FGA`B\tri A;
      \varepsilon_B\tri\varepsilon_A``h_A\tri h_A])
    \morphism(200,500)//<0,0>[`;(1)]
    \morphism(800,300)//<0,0>[`;(2)]
    \morphism(1700,300)//<0,0>[`;(3)]
    \morphism(2500,300)//<0,0>[`;(4)]
    \morphism(3100,200)//<0,0>[`;(5)]
  \efig
  \end{mathpar}
  The third diagram commutes as follows, which uses Lemma~\ref{lem:emex}.
  \begin{mathpar}
  \bfig
    \morphism(0,400)<1600,0>[(A\tri B)\tri C`A\tri(B\tri C);\alpha_{A,B,C}']
    \Vtriangle(1600,0)|amm|/->`=`->/<800,400>[
      A\tri(B\tri C)`FGA\tri FG(B\tri C)`A\tri(B\tri C);
      h_A\tri h_{B\tri C}``\varepsilon_A\tri\varepsilon_{B\tri C}]
    \morphism(800,0)|m|<-800,400>[
      FG((A\tri B)\tri C)`(A\tri B)\tri C;\varepsilon_{(A\tri B)\tri C}]
    \qtriangle(800,-400)|mmm|/`->`<-/<1600,400>[
      FG((A\tri B)\tri C)`A\tri(B\tri C)`FG(A\tri(B\tri C));
      `FG\alpha_{A,B,C}'`\varepsilon_{A\tri(B\tri C)}]
    \Ctriangle(2400,-1200)|mrm|/`->`<-/<800,800>[
      FGA\tri FG(B\tri C)`FG(A\tri(B\tri C))`F(GA\otimes G(B\tri C));
      `\m{GA,G(B\tri C)}`F\n{A,B\tri C}]
    \morphism(800,-400)<0,400>[F(G(A\tri B)\otimes GC)`FG((A\tri B)\tri C);F\n{A\tri B,C}]
    \btriangle(800,-800)|mmm|/<-``->/<1400,400>[
      F(G(A\tri B)\otimes GC)`F((GA\otimes GB)\otimes GC)`F(GA\otimes(GB\otimes GC));
      F(\n{A,B}\otimes id_{GC})``F\alpha_{GA,GB,GC}]
    \morphism(800,-800)|m|<0,-400>[
      F((GA\otimes GB)\otimes GC)`F((GB\otimes GA)\otimes GC);F(\e{A,B}\otimes id_{GC})]
    \morphism(2200,-800)|m|<1000,-400>[
      F(GA\otimes(GB\otimes GC))`F(GA\otimes G(B\tri C));F(id_{GA}\otimes\n{B,C})]
    \morphism(2200,-800)|m|<0,-400>[
      F(GA\otimes(GB\otimes GC))`F((GB\otimes GC)\otimes GA);F\e{GA,GB\otimes GC}]
    \qtriangle(2200,-1600)|mmr|/`->`->/<1000,400>[
      F((GB\otimes GC)\otimes GA)`F(GA\otimes G(B\tri C))`F(G(B\tri C)\otimes GA);
      `F(\n{B,C}\otimes id_{GA})`F\e{A,B\tri C}]
    \morphism(800,-1200)|m|<0,-400>[
      F((GB\otimes GA)\otimes GC)`F(G(B\tri A)\otimes GC);F(\n{B,A}\otimes id_{GC})]
    \morphism(2200,-1200)|m|<0,-300>[
      F((GB\otimes GC)\otimes GA)`F(GB\otimes(GC\otimes GA));F\alpha_{GB,GC,GA}]
    \morphism(800,-1200)|m|<700,-600>[
      F((GB\otimes GA)\otimes GC)`F(GB\otimes(GA\otimes GC));F\alpha]
    \morphism(2200,-1500)|m|<-700,-300>[
      F(GB\otimes(GC\otimes GA))`F(GB\otimes(GA\otimes GC));F(id_{GC}\otimes\e{C,A})]
    \morphism(800,-1600)|m|<0,-400>[F(G(B\tri A)\otimes GC)`FG((B\tri A)\tri C);F\n{B\tri A,C}]
    \morphism(2200,-1500)|m|<0,-500>[
      F(GB\otimes(GC\otimes GA))`F(GB\otimes G(C\tri A));F(id_{GB}\otimes\n{C,A})]
    \morphism(2200,-2000)|m|<0,-400>[
      F(GB\otimes G(C\tri A))`FG(B\tri(C\tri A));F\n{B,C\tri A}]
    \morphism(1500,-1800)|m|<0,-400>[
      F(GB\otimes(GA\otimes GC))`F(GB\otimes G(A\tri C));F(id_{GB}\otimes\n{A,C})]
    \morphism(1500,-2200)|m|<0,-400>[F(GB\otimes G(A\tri C))`FG(B\tri(A\tri C));F\n{B,A\tri C}]
    \morphism(0,400)|l|<0,-2000>[(A\tri B)\tri C`(B\tri A)\tri C;\e{A,B}^S\tri id_C]
    \morphism(800,-2000)|m|<-800,400>[
      FG((B\tri A)\tri C)`(B\tri A)\tri C;\varepsilon_{(B\tri A)\tri C}]
    \morphism(0,-1600)|l|<0,-1000>[(B\tri A)\tri C`B\tri(A\tri C);\alpha_{B,A,C}']
    \morphism(1500,-2600)|m|<-1500,0>[
      FG(B\tri(A\tri C))`B\tri(A\tri C);\varepsilon_{B\tri(A\tri C)}]
    \morphism(0,-2600)|l|<0,-400>[B\tri(A\tri C)`B\tri(C\tri A);\id_B\tri\e{A,C}^S]
    \morphism(1600,-3000)|b|<-1600,0>[(B\tri C)\tri A`B\tri(C\tri A);\alpha_{B,C,A}']
    \morphism(3200,-1600)|r|<0,-1400>[
      F(G(B\tri C)\otimes GA)`FG((B\tri C)\tri A);F\n{B\tri C,A}]
    \morphism(3200,-3000)|m|<-1000,600>[FG((B\tri C)\tri A)`FG(B\tri(C\tri A));FG\alpha_{B,C,A}]
    \morphism(3200,-3000)|b|<-1600,0>[
      FG((B\tri C)\tri A)`(B\tri C)\tri A;\varepsilon_{(B\tri C)\tri A}]
  \efig
  \end{mathpar}
\end{proof}


%%% Local Variables: 
%%% mode: latex
%%% TeX-master: main.tex
%%% End: 






















