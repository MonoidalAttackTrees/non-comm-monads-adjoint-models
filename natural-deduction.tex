The term assignment for natural deduction of the non-commutative part of the model, i.e. the
SMCC of the adjunction, is defined in Figure~\ref{fig:elle-nd-smcc}. And the term assignme for
the commutative part, i.e. the Lambek category of the adjunction, is defined in
Figure~\ref{fig:elle-nd-lambek}. $[[P]]$ and $[[I]]$ are contexts for the non-commutative part
and they are lists. $[[G]]$ and $[[D]]$ are contexts for the commutative part and they are
multisets, therefore the following exchange rules are implicit.
\scriptsize
\begin{mathpar}
  \ottdruleTXXbeta{} \qquad\qquad \ottdruleSXXbeta{}
\end{mathpar}

\begin{figure}
 \scriptsize
  \begin{mdframed}
    \begin{mathpar}
      \ottdruleTXXidentity{} \qquad\qquad \ottdruleTXXunitI{} \qquad\qquad \ottdruleTXXunitE{} \\
      \ottdruleTXXtenI{} \qquad\qquad \ottdruleTXXtenE{} \\
      \ottdruleTXXimpI{} \qquad\qquad \ottdruleTXXimpE{} \qquad\qquad \ottdruleTXXGI{}
    \end{mathpar}
  \end{mdframed}
\caption{Commutative Part}
\label{fig:elle-nd-smcc}
\end{figure}

\begin{figure}
 \scriptsize
  \begin{mdframed}
    \begin{mathpar}
      \ottdruleSXXidentity{} \qquad\qquad \ottdruleSXXunitI{} \qquad\qquad \ottdruleSXXunitE{} \\
      \ottdruleSXXtenI{} \qquad\qquad \ottdruleSXXtenEOne{} \\
      \ottdruleSXXtenETwo{} \qquad\qquad \ottdruleSXXimplI{} \\
      \ottdruleSXXimplE{} \qquad\qquad \ottdruleSXXimprI{} \\
      \ottdruleSXXimprE{} \qquad\qquad \ottdruleSXXFI{} \\
      \ottdruleSXXFE{} \qquad\qquad \ottdruleSXXGE{}
    \end{mathpar}
  \end{mdframed}
\caption{Non-Commutative Part}
\label{fig:elle-nd-lambek}
\end{figure}

\begin{center}
  \tiny
  \begin{math}
    $$\mprset{flushleft}
    \inferrule* [right={\tiny impR}] {
      $$\mprset{flushleft}
    \inferrule* [right={\tiny tenL}] {
      $$\mprset{flushleft}
    \inferrule* [right={\tiny Fl}] {
      $$\mprset{flushleft}
      \inferrule* [right={\tiny FI}] {
        $$\mprset{flushleft}
        \inferrule* [right={\tiny beta}] {
          $$\mprset{flushleft}
          \inferrule* [right={\tiny tenI}] {
            $$\mprset{flushleft}
            \inferrule* [right={\tiny FI}] {
              $$\mprset{flushleft}
              \inferrule* [right={\tiny identity}] {
                \,
              }{[[y0 : Gf B |- y0 : Gf B]]}
            }{[[y0 : Gf B; . |- F y0 : F Gf B]]}
            \\
            $$\mprset{flushleft}
            \inferrule* [right={\tiny FI}] {
              $$\mprset{flushleft}
              \inferrule* [right={\tiny identity}] {
                \,
              }{[[x0 : Gf A |- x0 : Gf A]]}
            }{[[x0 : Gf A; . |- F x0 : F Gf A]]}
          }{[[y0 : Gf B, x0 : Gf A ; . |- h(F y0) (>) F x0 : h(F Gf B) (>) F Gf A]]}
        }{[[x1 : Gf A, y1 : Gf B ; . |- ex y1 , x1 with y0 , x0 in (h(F y0) (>) F x0) : h(F Gf B) (>) F Gf A]]}
      }{[[x1 : Gf A ; y2 : F Gf B |- let y2 : F Gf B be F y1 in (ex y1 , x1 with y0 , x0 in (h(F y0) (x) F x0)) : h(F Gf B) (x) F Gf A]]}
    }{[[. ; x2 : F Gf A, y2 : F Gf B |- let x2 : F Gf A be F x1 in (let y2 : F Gf B be F y1 in (ex y1 , x1 with y0 , x0 in (h(F y0) (x) F x0))) : h(F Gf B) (x) F Gf A]]}
    }{[[. ; z : h(F Gf A) (x) F Gf B |- let z : h(F Gf A) (x) F Gf B be x2 (x) y2 in (let x2 : F Gf A be F x1 in (let y2 : F Gf B be F y1 in (ex y1 , x1 with y0 , x0 in (h(F y0) (x) F x0)))) : (h(F Gf B) (x) F Gf A)]]}
    }{[[. ; . |- \l z : h(F Gf A) (x) F Gf B . let z : h(F Gf A) (x) F Gf B be x2 (x) y2 in (let x2 : F Gf A be F x1 in (let y2 : F Gf B be F y1 in (ex y1 , x1 with y0 , x0 in (h(F y0) (x) F x0)))) : (h(F Gf A) (x) F Gf B) -> (h(F Gf B) (x) F Gf A)]]}
  \end{math}
\end{center}


\subsection{Categorical Interpretation of Natural Deductions}


$T$ rules: in the symmetric monoidal closed category of the adjunction model

T\_identity: $id_X:X\rightarrow X$

T\_unitI: 

T\_unitE: given $t_1:\Delta\rightarrow Unit$ and $t_2:\Gamma\rightarrow Y$, returns
$\lambda_Y\circ(t_1\otimes t_2):\Gamma\otimes\Delta\rightarrow Unit\otimes Y\rightarrow Y$

T\_tenI: given $t_1:\Gamma\rightarrow X$ and $t_2:\Delta\rightarrow Y$, returns
$t_1\otimes t_2:\Gamma\otimes\Delta\rightarrow X\otimes Y$

T\_tenE: given $t_1:\Gamma\rightarrow X\otimes Y$ and
$t_2:\Delta\otimes X\otimes Y\rightarrow Z$, returns \\
$t_2\circ\e{X\otimes Y,\Delta}\circ t_1\otimes id_\Delta:\Gamma\otimes\Delta\rightarrow(X\otimes Y)\otimes\Delta\rightarrow\Delta\otimes(X\otimes Y)\rightarrow Z$

T\_implI:

T\_implE:

T\_imprI:

T\_imprE:

T\_GI: given $s:FX_1\otimes'...\otimes' FX_n\rightarrow A$, returns \\
$Gs\circ G\m{}^{-1}\circ\eta:X_1\otimes...\otimes X_n\rightarrow GF(X_1\otimes...\otimes X_n)\rightarrow G(FX_1\otimes'...\otimes'FX_n)\rightarrow GA$

S rules: in the Lambek category of the adjunction model

S\_identity: $id_A:A\rightarrow A$

S\_unitI:

S\_unitE:



\subsection{Normaalization and Reduction}


