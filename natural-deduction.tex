The term assignment for natural deduction of the commutative part of the model, i.e. the SMCC of
the adjunction, is defined in Figure~\ref{fig:elle-nd-smcc}. And the term assignme for the
non-commutative part, i.e. the Lambek category of the adjunction, is defined in
Figure~\ref{fig:elle-nd-lambek}. $[[P]]$ and $[[I]]$ are contexts for the commutative part
and they are multisets. $[[G]]$ and $[[D]]$ are contexts for the mix of the commutative part and
the non-commutative part, and they are lists. Therefore the following exchange rule is implicit.

\begin{figure}[!h]
  \scriptsize
  \begin{mathpar}
    \NDdruleTXXbeta{}
  \end{mathpar}
\end{figure}

\begin{figure}[!h]
  \scriptsize
  \begin{mdframed}
    \begin{mathpar}
      \NDdruleTXXid{} \qquad\qquad \NDdruleTXXunitI{} \qquad\qquad \NDdruleTXXunitE{} \\
      \NDdruleTXXtenI{} \qquad\qquad \NDdruleTXXtenE{} \\
      \NDdruleTXXimpI{} \qquad\qquad \NDdruleTXXimpE{} \qquad\qquad \NDdruleTXXGI{} \\
      \NDdruleSXXbeta{}
    \end{mathpar}
  \end{mdframed}
\caption{Commutative Part}
\label{fig:elle-nd-smcc}
\end{figure}

\begin{figure}[!h]
 \scriptsize
  \begin{mdframed}
    \begin{mathpar}
      \NDdruleSXXid{} \qquad\qquad \NDdruleSXXunitI{} \qquad\qquad \NDdruleSXXunitEOne{} \\
      \NDdruleSXXunitEOne{} \qquad\qquad \NDdruleSXXunitETwo{} \\
      \NDdruleSXXtenI{} \qquad\qquad \NDdruleSXXtenEOne{} \\
      \NDdruleSXXtenETwo{} \qquad\qquad \NDdruleSXXimprI{} \\
      \NDdruleSXXimprE{} \qquad\qquad \NDdruleSXXimplI{} \\
      \NDdruleSXXimplE{} \qquad\qquad \NDdruleSXXGE{} \qquad\qquad \NDdruleSXXFI{} \\
      \NDdruleSXXFE{}
    \end{mathpar}
  \end{mdframed}
\caption{Non-Commutative Part}
\label{fig:elle-nd-lambek}
\end{figure}

We could derive exchange comonadically as follows:

\begin{center}
  \tiny
  \begin{math}
  $$\mprset{flushleft}
  \inferrule* [right={\tiny imprI}] {
    $$\mprset{flushleft}
    \inferrule* [right={\tiny tenE2}] {
      $$\mprset{flushleft}
      \inferrule* [right={\tiny id}] {
        \,
      }{[[z : h(F Gf A) (>) F Gf B |-l z : h(F Gf A) (>) F Gf B]]}
        $$\mprset{flushleft}
        \inferrule* [right={\tiny FE}] {
          $$\mprset{flushleft}
          \inferrule* [right={\tiny id}] {
            \,
          }{[[x2 : F Gf A |-l x2 : F Gf A]]}
            $$\mprset{flushleft}
            \inferrule* [right={\tiny FE}] {
              $$\mprset{flushleft}
              \inferrule* [right={\tiny id}] {
                \,
              }{[[y2 : F Gf B |-l y2 : F Gf B]]}
              \inferrule* [right={\tiny beta}] {
                $$\mprset{flushleft}
                \inferrule* [right={\tiny FE}] {
                  $$\mprset{flushleft}
                  \inferrule* [right={\tiny FI}] {
                    $$\mprset{flushleft}
                    \inferrule* [right={\tiny id}] {
                      \,
                    }{[[y0 : Gf B |-c y0 : Gf B]]}
                  }{[[y0 : Gf B |-l F y0 : F Gf B]]}
                  $$\mprset{flushleft}
                  \inferrule* [right={\tiny FI}] {
                    $$\mprset{flushleft}
                    \inferrule* [right={\tiny id}] {
                      \,
                    }{[[x0 : Gf A |-c x0 : Gf A]]}
                  }{[[x0 : Gf A |-l F x0 : F Gf A]]}
                }{[[y0 : Gf B, x0 : Gf A |-l h(F y0) (>) F x0 : h(F Gf B) (>) F Gf A]]}
              }{[[x1 : Gf A, y1 : Gf B |-l ex y1 , x1 with y0 , x0 in (h(F y0) (>) F x0) : h(F Gf B) (>) F Gf A]]}
            }{[[x1 : Gf A , y2 : F Gf B |-l let F y1 : F Gf B be y2 in (ex y1 , x1 with y0 , x0 in (h(F y0) (>) F x0)) : h(F Gf B) (>) F Gf A]]}
          }{[[x2 : F Gf A , y2 : F Gf B |-l let F x1 : F Gf A be x2 in (let F y1 : F Gf B be y2 in (ex y1 , x1 with y0 , x0 in (h(F y0) (>) F x0))) : h(F Gf B) (>) F Gf A]]}
        }{[[z : h(F Gf A) (>) F Gf B |-l let z : h(F Gf A) (>) F Gf B be x2 (>) y2 in (let F x1 : F Gf A be x2 in (let F y1 : F Gf B be y2 in (ex y1 , x1 with y0 , x0 in (h(F y0) (>) F x0)))) : h(F Gf B) (>) F Gf A]]}
      }{[[ . |-l \r z : h(F Gf A) (>) F Gf B.let z : h(F Gf A) (>) F Gf B be x2 (>) y2 in (let F x1 : F Gf A be x2 in (let F y1 : F Gf B be y2 in (ex y1 , x1 with y0 , x0 in (h(F y0) (>) F x0)))) : (h(F Gf A) (>) F Gf B) -> (h(F Gf B) (>) F Gf A)]]}
  \end{math}
\end{center}

We also have the three cut rules derivable in the natural deduction:

\begin{figure}[!h]
  \scriptsize
  \begin{mathpar}
    \NDdruleTXXcut{} \qquad\qquad \NDdruleSXXcutOne{} \qquad\qquad \NDdruleSXXcutTwo{}
  \end{mathpar}
\end{figure}

\begin{center}
  \tiny
  \begin{math}
  $$\mprset{flushleft}
  \inferrule* [right={\tiny FI}] {
    {[[I |-c t : X]]}
  }{[[I |-l F t : F X]]}
  \end{math}
\end{center}
