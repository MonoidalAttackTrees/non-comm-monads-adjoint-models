We now introduce Commutative/Non-commutative (CNC) logic in the form of a
sequent calculus. One should view this logic as cmposed of two logics; one
sitting to the left of the other. On the left, there is intuitionistic
linear logic, denoted by $\cat{C}$ and on the right is the Lambek calculus
denoted by $\cat{L}$. Then we connect these two systems by a pair of
monoidal adjoint functors $\cat{C} : \func{F} \dashv \func{G} : \cat{L}$.
Keeping this intuition in mind we now define the syntax for CNC logic.

\begin{definition}
  \label{def:Lambek-syntax}
  The following grammar describes the syntax of the sequent calculus of
  CNC logic:
  \begin{center}\vspace{-3px}\small
    \begin{math}
      \begin{array}{lll}
        \text{($\cat{C}$-Types)} & [[W]],[[X]],[[Y]],[[Z]] ::= [[UnitT]] \mid [[X (*) Y]] \mid [[X -o Y]] \mid [[Gf A]]\\
        \text{($\cat{L}$-Types)} & [[A]],[[B]],[[C]],[[DT]] ::= [[UnitS]] \mid [[A (>) B]] \mid [[A -> B]] \mid [[B <- A]] \mid [[F X]]\\
        \text{($\cat{C}$-Contexts)} & [[I]],[[P]] ::= [[.]] \mid [[X]] \mid [[I,P]]\\
        \text{($\cat{L}$-Contexts)} & [[G]],[[D]] ::= [[.]] \mid [[A]] \mid [[X]] \mid [[G;D]]\\
      \end{array}
    \end{math}
  \end{center}
\end{definition}

The syntax for $\cat{C}$-types are the standard types for intuitionistic
linear logic. We have a constant $[[UnitT]]$, tensor product $[[X (*) Y]]$,
and linear implication $[[X -o Y]]$, but just as in LNL logic we also have a
type $[[Gf A]]$ where $[[A]]$ is an $\cat{L}$-type; that is, a type from the
non-commutative side corresponding to the right-adjoint functor between
$\cat{L}$ and $\cat{C}$. This functor can be used to import types from the
non-commutative side into the commutative side. Now a sequent in the the
commutative side is denoted by $[[I |-c X]]$ where $[[I]]$ is a
$\cat{C}$-context, which is a sequence of types $[[X]]$.

The non-commutative side is a bit more interesting than the commutative side
just introduced. Sequents in the non-commutative side are denoted by
$[[G |-l A]]$ where $[[G]]$ is now a $\cat{L}$-context. These contexts are
ordered sequences of types from \emph{both} sides denoted by $[[B]]$ and
$[[X]]$ respectively. Given two contexts $[[G]]$ and $[[D]]$ we denote their
concatenation by $[[G;D]]$; we use a semicolon here to emphasize the fact
that the contexts are ordered.

The context consisting of hypotheses from both sides goes back to
Benton~\cite{Benton:1994} and is a property unique to adjoint logics such as
Benton's LNL logic and CNC logic. This is also a very useful property
because it allows one to make use of both sides within the Lambek calculus
without the need to annotate every formula with a modality.

The reader familiar with LNL logic will notice that our sequent,
$[[G |-l A]]$, differs from Benton's. His is of the form $[[G; D |-l A]]$,
where $[[G]]$ contains non-linear types, and $[[D]]$ contains linear
formulas. Just as Benton remarks, the splitting of his contexts was a
presentational device. One should view his contexts as merged, and hence,
linear formulas were fully mixed with non-linear formulas. Now why did we
not use this presentational device? Because, when contexts from LNL logic
become out of order Benton could use the exchange rule to put them back in
order again, but we no longer have general exchange. Thus, we are not able
to keep the context organized in this way.

The syntax for $\cat{L}$-types are of the typical form for the Lambek
Calculus. We have a unit type $[[UnitS]]$, a non-commutative tensor product
$[[A (>) B]]$, right implication $[[A -> B]]$, and left implication
$[[B <- A]]$.

The sequenct calculus for CNC logic can be found in
Figure~\ref{fig:CNC-sequent-calculus}.
\begin{figure}[!h]
  \footnotesize
  \begin{tabular}{|c|}
    \hline\\
    \begin{mathpar}
    \SCdruleTXXax{} \and
    \SCdruleTXXunitL{} \and
    \SCdruleTXXunitR{} \and
    \SCdruleTXXtenL{} \and
    \SCdruleTXXtenR{} \and
    \SCdruleTXXimpL{} \and
    \SCdruleTXXimpR{} \and
    \SCdruleTXXGr{} \and
    \SCdruleTXXex{} \and
    \SCdruleTXXcut{}
    \end{mathpar}\\\\
    \hline
    \\[5px]
    \begin{mathpar}
    \SCdruleSXXax{} \and
    \SCdruleSXXunitLOne{} \and
    \SCdruleSXXunitLTwo{} \and
    \SCdruleSXXunitR{} \and
    \SCdruleSXXex{} \and
    \SCdruleSXXtenLOne{} \and
    \SCdruleSXXtenLTwo{} \and
    \SCdruleSXXtenR{} \and
    \SCdruleSXXimpL{} \and
    \SCdruleSXXimprL{} \and
    \SCdruleSXXimprR{} \and
    \SCdruleSXXimplL{} \and
    \SCdruleSXXimplR{} \and
    \SCdruleSXXFl{} \and
    \SCdruleSXXFr{} \and
    \SCdruleSXXGl{} \and
    \SCdruleSXXcutOne{} \and
    \SCdruleSXXcutTwo{} \and
    \end{mathpar}\\\\
    \hline
  \end{tabular}
  \caption{Sequent Calculus for CNC Logic}
  \label{fig:CNC-sequent-calculus}
\end{figure}
We split the figure in two: the top of the figure are the rules of
intuitionistic linear logic whose sequents are the $\mathcal{C}$-sequents
denoted by $[[P |-c X]]$, and the bottom of the figure are the rules for the
mixed commutative/non-commutative Lambek calculus whose sequents are the
$\mathcal{L}$-sequents denoted by $[[G |-l A]]$, but the two halves are
connected via the functor rules $\SCdruleTXXGrName{}$,
$\SCdruleSXXGlName{}$, $\SCdruleSXXFlName{}$, and $\SCdruleSXXFrName{}$, and
the rules $\SCdruleSXXunitLOneName{}$, $\SCdruleSXXexName{}$,
$\SCdruleSXXtenLOneName{}$, $\SCdruleSXXimpLName{}$,
$\NDdruleSXXcutOneName{}$.

We prove cut elimination for the sequent calculus. We define the
\textit{degree} $|X|$ (resp. $|A|$) of a commutative (resp. non-commutative)
formula to be the number of logical connectives in the proposition. For
instance, $|[[X (*) Y]]| = |[[X]]| + |[[Y]]| + 1$. The \textit{cut rank}
$c(\Pi)$ of a proof $\Pi$ is one more than the maximum of the ranks of all
the cut formulae in $\Pi$, and $0$ if $\Pi$ is cut-free. Then the
\textit{depth} $d(\Pi)$ of a proof $\Pi$ is the length of the longest path
in the proof tree (so the depth of an axiom is $0$). The key to the proof
of cut elimination is the following lemma, which shows how to transform a
single cut, either by removing it or by replacing it with one or more
simpler cuts.

\begin{lemma}[Cut Reduction]
  \label{lem:cut-reduction}
  \begin{enumerate}
  \item If $\Pi_1$ is a proof of $[[I |-c X]]$ and $\Pi_2$ is a proof of
  $[[P1,X,P2 |-c Y]]$ with $c(\Pi_1)$, $c(\Pi_2)\leq |X|$, then there exists
  a proof $\Pi$ of $[[P1, I, P2 |-c Y]]$ with $c(\Pi)\leq |X|$.
  \item If $\Pi_1$ is a proof of $[[I |-c X]]$ and $\Pi_2$ is a proof of
  $[[G1; X; G2 |-l A]]$ with $c(\Pi_1)$, $c(\Pi_2)\leq |X|$, then there
  exists a proof $\Pi$ of $[[G1; I; G2 |-l A]]$ with $c(\Pi)\leq |X|$.
  \item If $\Pi_1$ is a proof of $[[G |-l A]]$ and $\Pi_2$ is a proof of
  $[[D1; A; D2 |-l B]]$ with $c(\Pi_1)$, $c(\Pi_2)\leq |A|$, then there
  exists a proof $\Pi$ of $[[D1; G; D2 |-l B]]$ with $c(\Pi)\leq |A|$.
  \end{enumerate}
\end{lemma}
\begin{proof}
  The proof is done case by case similar as in \cite{Mellies:2009}.
\end{proof}

\begin{lemma}
  \label{lem:less-cut-rank}
  Let $\Pi$ be a proof of a sequent $[[I |-c X]]$ or $[[G |-l A]]$ s.t.
  $c(\Pi)>0$. Then there is a proof $\Pi'$ of the same sequent with
  $c(\Pi')<c(\Pi)$.
\end{lemma}
\begin{proof}
  We prove the lemma by induction on $d(\Pi)$. We denote the proof $\Pi$ by 
  $\pi+r$, where $r$ is the last inference of $\Pi$ and $\pi$ denotes the
  rest of the proof. If $r$ is not a cut, then by induction hypothesis on
  $\pi$, there is a proof $\pi'$ s.t. $c(\pi')>c(\pi)$ and $\Pi'=\pi'+r$.
  Otherwise, we assume $r$ is a cut on a formula $X$. If $c(\Pi)>|X|+1$,
  then there is a cut on $|Y|$ in $\pi$ with $|Y|>|X|$. So we can apply
  the induction hypothesis on $\pi$ to get $\Pi'$ with $c(\Pi')<c(\Pi)$. The
  last case to consider is when $c(\Pi)=|X|+1$ (note that $c(\Pi)$ cannot be
  less than $|X|+1$). In this case, $\Pi$ is in the form of
  \begin{center}
    \scriptsize
    \begin{math}
      $$\mprset{flushleft}
      \inferrule* [right={\tiny cut}] {
        {
          \begin{array}{cc}
            \Pi_1 & \Pi_2 \\
            {[[I |-c X]]} & {[[P1, X, P2 |-c Y]]}
          \end{array}
        }
      }{[[P1, I, P2 |-c Y]]}
    \end{math}
    \qquad\qquad
    or,
    \begin{math}
      $$\mprset{flushleft}
      \inferrule* [right={\tiny cut1}] {
        {
          \begin{array}{cc}
            \Pi_1 & \Pi_2 \\
            {[[I |-c X]]} & {[[G1; X; G2 |-l A]]}
          \end{array}
        }
      }{[[G1; I; G2 |-l A]]}
    \end{math}
  \end{center}
  By assumption, $c(\Pi_1),c(\Pi_2)\leq |X|+1$. By induction, we can
  construct $c(\Pi_1')$ proving $[[I |-c X]]$ and $c(\Pi_2')$ proving
  $[[P1, X, P2 |-c Y]]$ (or $<[[G1; X; G2 |-l A]]$) with
  $c(\Pi_1'), c(\Pi_2')\leq |X|$. Then by Lemma~\ref{lem:cut-reduction}, we 
  can construct $\Pi'$ proving $[[P1, I, P2 |-c Y]]$ (or
  $[[G1; I; G2 |-l A]]$) with $c(\Pi')\leq |X|$. \\
  The case where the last inference is a cut on a formula $A$ is similar as 
  when it is a cut on $X$.
\end{proof}

\begin{theorem}[Cut Elimination]
  Let $\Pi$ be a proof of a sequent $[[I |-c X]]$ or $[[G |-l A]]$ s.t.
  $c(\Pi)>0$. Then there is an algorithm which yields a cut-free proof
  $\Pi'$ of the same sequent.
\end{theorem}
\begin{proof}
  This follows immediately by induction on $c(\Pi)$ and
  Lemma~\ref{lem:less-cut-rank}.
\end{proof}
