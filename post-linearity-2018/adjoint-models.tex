In this section we introduce Lambek Adjoint Models (LAMs). Benton's
LNL model consists of a symmetric monoidal adjunction
$F:\cat{C}\dashv\cat{L}:G$ between a cartesian closed category
$\cat{C}$ and a symmetric monoidal closed category $\cat{L}$. LAM
consists of a monoidal adjunction between a symmetric monoidal closed
category and a Lambek category.
\begin{definition}
  \label{def:lambek-category}
  A \textbf{Lambek category} is a monoidal category $(\cat{L},\tri,I',\alpha',\lambda',\rho')$
  with two functors $- \rightharpoonup - : \cat{L}^{\mathsf{op}} \times \cat{L} \mto \cat{L}$ and
  $- \leftharpoonup - : \cat{L} \times \cat{L}^{\mathsf{op}} \mto \cat{L}$ such that the following
  two natural bijections hold:
  \[
  \begin{array}{lllll}
    \Hom{L}{A \tri B}{C} \cong \Hom{L}{A}{B \rightharpoonup C} & \quad &
    \Hom{L}{A \tri B}{C} \cong \Hom{L}{B}{C \leftharpoonup A}\\
  \end{array}
  \]  
\end{definition}

\begin{definition}
  A \textbf{Lambek Adjoint Model (LAM)}, $(\cat{C},\cat{L},F,G,\eta,\varepsilon)$, consists of
  \begin{itemize}
  \item a symmetric monoidal closed category $(\cat{C},\otimes,I,\alpha,\lambda,\rho)$;
  \item a Lambek category $(\cat{L},\tri,I',\alpha',\lambda',\rho')$;
  \item a monoidal adjunction $F:\cat{C}\dashv\cat{L}:G$ with unit $\eta:\Id_{\cat{C}} \rightarrow GF$ and
        counit $\varepsilon:FG\rightarrow \Id_\cat{L}$, where $(F:\cat{C}\rightarrow\cat{L}, m)$
        and $(G:\cat{L}\rightarrow\cat{C}, n)$ are monoidal functors.
  \end{itemize}
\end{definition}
\noindent
Following the tradition, we use letters $X$, $Y$, $Z$ for objects in
$\cat{C}$ and $A$, $B$, $C$ for objects in $\cat{L}$. The rest of this
section proves essential properties of any LAM.

\textbf{An isomorphism}. Let $(\cat{C},\cat{L},F,G,\eta,\varepsilon)$
be a LAM, where $(F,\m{})$ and $(G,\n{})$ are monoidal
functors. Similarly as in Benton's LNL model, $\m{X,Y} : FX \tri FY
\mto F(X \otimes Y)$ are components of a natural isomorphism, and
$\m{I} : I' \mto FI$ is an isomorphism. This is essential for modeling
certain rules of CNC logic, such as tensor elimination in natural
deduction.  We define the inverses of $\m{X,Y}:FX\tri FY\rightarrow
F(X\otimes Y)$ and $\m{I}:I'\rightarrow FI$ as:
\begin{mathpar}
\bfig
  \morphism<1000,0>[\p{X,Y}:F(X\otimes Y)`F(GFX\otimes GFY);F(\eta_X\otimes\eta_Y)]
  \morphism(1000,0)<900,0>[F(GFX\otimes GFY)`FG(FX\tri FY);F\n{FX,FY}]
  \morphism(1900,0)<750,0>[FG(FX\tri FY)`FX\tri FY;\varepsilon_{FX\tri FX}]
\efig
\and
\bfig
  \morphism<500,0>[\p{I}:FI`FGI';F\n{I'}]
  \morphism(500,0)<400,0>[FGI'`I';\varepsilon_{I'}]
\efig
\end{mathpar}
We can now see that this straightforwardly follows from Benton's proof.

\begin{theorem}
\label{thm:m-natural-iso}
  $\m{X,Y}$ are components of a natural isomorphism and their inverses are $\p{X,Y}$. 
\end{theorem}
\begin{proof}
  This proof follows by a diagram chase using the adjunction in the
  definition of a LAM and naturality. The detail of the proof is in
  Appendix~\ref{app:m-natural-iso}.
\end{proof}

\begin{theorem}
\label{thm:m-iso}
  $\m{I}$ is an isomorphism and its inverse is $\p{I}$.
\end{theorem}
\begin{proof}
  This is equivalent to equations $\m{I}\circ\p{I}=id_{FI}$ and
  $\p{I}\circ\m{I}=id_{I'}$, equivalent to the following diagrams, which
  commute because $\varepsilon$ is a monoidal natural transformation.
  \begin{mathpar}
  \bfig
    \square/->`=`->`<-/<400,400>[FI`FGI'`FI`I';F\n{I'}``\varepsilon_{I'}`\m{I}]
  \efig
  \and
  \bfig
    \square/->`=`->`<-/<400,400>[I'`FI`I'`FGI';\m{I}``F\n{I'}`\varepsilon_{I'}]
  \efig
  \end{mathpar}
\end{proof}

\textbf{Strong non-commutative monad}. Next we show that the monad on
$\cat{C}$ in LAM is strong but non-commutative. In Benton's LNL model,
the monad on the cartesian closed category is commutative, but later
Benton and Wadler \cite{Benton:1996} wonder, is it possible to model
non-commutative monads using adjoint models similar to LNL models? The
following answers their question in the positive.
\begin{lemma}
\label{lem:monoidal-monad}
The monad induced by any LAM, $GF : \cat{C} \mto \cat{C}$, is monoidal.
\end{lemma}
\begin{proof}
  The proof is done by checking the conditions for a functor being monoidal.
  The detail of the proof is in Appendix~\ref{app:monoidal-monad}.
\end{proof}
\noindent
However, the monad is not symmetric because the following diagram does
not commute.
\begin{mathpar}
\bfig
  \ptriangle/->`->`/<900,400>[
    GFX\otimes GFY`GFY\otimes GFX`G(FX\tri FY);\e{GFX,GFY}`\n{FX,FY}`]
  \morphism(900,400)<900,0>[GFY\otimes GFX`G(FY\tri FX);\n{FY,FX}]
  \dtriangle(900,0)/`->`->/<900,400>[
    G(FY\tri FX)`GF(X\otimes Y)`GF(Y\otimes X);`G\m{Y,X}`GF\e{X,Y}]
  \morphism|b|<900,0>[G(FX\tri FY)`GF(X\otimes Y);G\m{X,Y}]
\efig
\end{mathpar}
Commutativity fails, because the functors defining the monad are not
symmetric monoidal, but only monoidal.  This means that the diagram
\[
\bfig
\square<700,400>[
  FA\otimes'FB`FB\otimes'FA`F(A\otimes B)`F(B\otimes A);
  \e{FA,FB}`\m{A,B}`\m{B,A}`F\e{A,B}]
\efig
\]
does not hold for $G$ nor $F$.  However, we can prove the monad is
strong.
\begin{lemma}
  \label{lem:strong-monad}
  The monad, $GF : \cat{C} \mto \cat{C}$, on the symmetric monoidal
  closed category in LAM is strong.
\end{lemma}
\begin{proof}
The proof is done by frist defining a natural transformation $\tau$, called
the \textbf{tensorial strength}, with components
$\tau_{A,B}:A\tri TB\rightarrow T(A\tri B)$, and then proving the
commutativity of several diagrams through diagram chasing. The formal
definition for a strong monad and the full proof are in
Appendix~\ref{app:strong-monad}.
\end{proof}
\noindent
Finally, we obtain the non-communativity of the monad induced by any
LAM as follows.
\begin{lemma}[Due to Kock~\cite{kock1972strong}]
\label{lem:monad-com-iff-sym}
  Let $\cat{M}$ be a symmetric monoidal category and $T$ be a strong monad on $\cat{M}$. Then
  $T$ is commutative iff it is symmetric monoidal.
\end{lemma}

\begin{theorem}
  The monad, $GF : \cat{C} \mto \cat{C}$, on the SMCC in LAM is strong
  but non-commutative.
\end{theorem}
\begin{proof}
  This proof follows from Lemma~\ref{lem:strong-monad} and
  Lemma~\ref{lem:monad-com-iff-sym}.
\end{proof}

\textbf{Comonad for exchange}.  We conclude this section by showing
that the comonad induced by any LAM is monoidal and extends $\cat{L}$
with exchange.  The latter is shown by proving that its corresponding
co-Eilenberg-Moore category is symmetric monoidal closed.

\begin{lemma}
  The comonad, $FG : \cat{L} \mto \cat{L}$, on the Lambek category in any LAM is monoidal.
\end{lemma}
\begin{proof}
  This proof follows by several diagram chases, but really does not
  depart much from Benton's proof \cite{Benton:1994}.
\end{proof}

\begin{theorem}
  \label{thm:em-exchange}
  Given a LAM $(\cat{C},\cat{L},F,G,\eta,\varepsilon)$ and the comonad
  $FG : \cat{L} \mto \cat{L}$, the co-Eilenberg-Moore category
  $\cat{L}^{FG}$ has an exchange natural transformation $\e{A,B}^{FG}:A\tri
  B\rightarrow B\tri A$.
\end{theorem}
\begin{proof}
  The natural transformation $\e{A,B}^{FG}:A\tri B\rightarrow B\tri A$ is defied
  as follows:
  $$\bfig
    \morphism<600,0>[A\tri B`FGA\tri FGB;h_A\tri h_B]
    \morphism(600,0)<800,0>[FGA\tri FGB`F(GA\otimes GB);\m{GA,GB}]
    \morphism(1400,0)<800,0>[F(GA\otimes GB)`F(GB\otimes GA);F\e{GA,GB}]
    \morphism(2200,0)<700,0>[F(GB\otimes GA)`FG(B\tri A);F\n{B,A}]
    \morphism(2900,0)<500,0>[FG(B\tri A)`B\tri A;\varepsilon_{B\tri A}]
  \efig$$
  in which $\e{}$ is the exchange for $\cat{C}$. Then $\e{}^{FG}$ is a
  natural transformation because the following diagrams commute for
  morphisms $f:A\rightarrow A'$ and $g:B\rightarrow B'$:
  \begin{mathpar}
  \bfig
    \square|almb|<700,400>[
      A\tri B`FGA\tri FGB`A'\tri B'`FGA'\tri FGB';
      h_A\tri h_B`f\tri g`FGf\tri FGg`h_{A'}\tri h_{B'}]
    \square(700,0)|ammb|/->``->`->/<800,400>[
      FGA\tri FGB`F(GA\otimes GB)`FGA'\tri FGB'`F(GA'\otimes GB');
      \m{GA,GB}``F(Gf\otimes Gg)`\m{GA',GB'}]
    \square(1500,0)|ammb|/->``->`->/<800,400>[
      F(GA\otimes GB)`F(GB\otimes GA)`F(GA'\otimes GB')`F(GB'\otimes GA');
      F\e{A,B}``F(Gg\otimes Gf)`F\e{A',B'}]
    \square(2300,0)|ammb|/->``->`->/<800,400>[
      F(GB\otimes GA)`FG(B\tri A)`F(GB'\otimes GA')`FG(B'\tri A');
      F\n{B,A}``FG(g\tri f)`F\n{B',A'}]
    \square(3100,0)|amrb|/->``->`->/<600,400>[
      FG(B\tri A)`B\tri A`FG(B'\tri A')`B'\tri A';
      \varepsilon_{B\tri A}``g\tri f`\varepsilon_{B'\tri A'}]
  \efig
  \end{mathpar}
\end{proof}

\begin{corollary}
  The subcategory, $\mathsf{Exp}(\cat{L}^{FG})$, of the
  co-Eilenberg-Moore category $\cat{L}^{FG}$ consisting of the free
  exponential coalgebras is symmetric monoidal closed.
\end{corollary}
\begin{proof}
  The proof that $\mathsf{Exp}(\cat{L}^{FG})$ is monoidal closed
  follows similarly to Benton~\cite{Benton:1994}.
\end{proof}

%%% Local Variables: 
%%% mode: latex
%%% TeX-master: main.tex
%%% End:
