Similar as the sequent calculus, the term assignment for CNC logic is also
composed of two logics; intuitionistic linear logic on the left, denoted by
$\cat{C}$, and the Lambek calculus on the right, denoted by $\cat{L}$. The
two systems by a pair of monoidal adjoint functors
$\cat{C}:\func{F} \dashv \func{G}:\cat{L}$. The syntax for types and
contexts we use in the term assignment is the same as in the sequent
calculus. The rest of the syntax for the term assignment is defined as
follows.
\begin{definition}
  \label{def:Lambek-syntax}
  The following grammar describes the syntax of the term assignment of the
  CNC logic:
  \begin{center}\vspace{-3px}\small
    \begin{math}
      \begin{array}{lll}        
        \text{($\cat{C}$-Terms)} & [[t]] ::= [[x]] \mid [[trivT]] \mid [[t1 (*) t2]] \mid [[let t1 : X be q in t2]] \mid [[\x:X.t]] \mid [[t1 t2]] \mid [[ex t1 , t2 with x1 , x2 in t3]] \mid [[Gf s]]\\
        \text{($\cat{L}$-Terms)} & [[s]] ::= [[x]] \mid [[trivS]] \mid [[s1 (>) s2]] \mid [[let s1 : A be p in s2]] \mid [[let t : X be q in s]] \mid [[\l x : A.s]] \mid [[\r x : A . s]] \\
        & \,\,\,\,\,\,\,\,\,\mid [[appl s1 s2]] \mid [[appr s1 s2]] \mid [[F t]]\\        
        \text{($\cat{C}$-Patterns)} & [[q]] ::= [[trivTp]] \mid [[x]] \mid [[q1 (*) q2]] \mid [[Gf p]]\\
        \text{($\cat{L}$-Patterns)} & [[p]] ::= [[trivSp]] \mid [[x]] \mid [[p1 (>) p2]] \mid [[F q]]\\        
        \text{($\cat{C}$-Typing Judgment)} & [[I |-c t : X]]\\
        \text{($\cat{L}$-Typing Judgment)} & [[G |-l s : A]]\\
      \end{array}
    \end{math}
  \end{center}
\end{definition}

Now $\cat{C}$-typing judgments are denoted by $[[P |-c t : X]]$ where
$[[P]]$ is a sequence of pairs of variables and their types, denoted by
$[[x : X]]$, $[[t]]$ is a $\cat{C}$-term, and $[[X]]$ is a $\cat{C}$-type.  
The $\cat{C}$-terms are all standard, but $[[Gf s]]$ corresponds to the
morphism part of the right-adjoint of the adjunction between both logics,
and $[[ex t1 , t2 with x1 , x2 in t3]]$ is the introduction form for the
structural rule exchange.

The $\cat{L}$-typing judgment has the form $[[G |-l s : A]]$ where $[[G]]$
is now a $\cat{L}$-context, denoted by $[[G]]$ or $[[D]]$. These contexts
are ordered sequences of pairs of free variables with their types from
\emph{both} sides denoted by $[[x : B]]$ and $[[x : X]]$ respectively.
Finally, the term $[[s]]$ is a $\cat{L}$-term, and $[[A]]$ is a
$\cat{L}$-type.  Given two typing contexts $[[G]]$ and $[[D]]$ we denote
their concatenation by $[[G;D]]$; we use a semicolon here to emphasize the
fact that the contexts are ordered. $\cat{L}$-terms correspond to
introduction and elimination forms for each of the previous types. For
example, $[[s1 (>) s2]]$ introduces a tensor, and
$[[let s1 : A (>) B be x (>) y in s2]]$ eliminates a tensor.

The typing rules for CNC logic can be found in
Figure~\ref{fig:CNC-typing-rules}.
\begin{figure}
  \footnotesize
  \begin{tabular}{|c|}
    \hline\\
      \begin{mathpar}
      \NDdruleTXXid{} \and
      \NDdruleTXXunitI{} \and
      \NDdruleTXXunitE{} \and
      \NDdruleTXXtenI{} \and
      \NDdruleTXXtenE{} \and
      \NDdruleTXXimpI{} \and
      \NDdruleTXXimpE{} \and
      \NDdruleTXXGI{} \and
      \NDdruleTXXbeta{} \and
      \NDdruleTXXcut{}      
      \end{mathpar}
      \\
      \\
      \hline
      \\[5px]
    \begin{mathpar}
      \NDdruleSXXid{} \and
      \NDdruleSXXunitI{} \and
      \NDdruleSXXunitETwo{} \and
      \NDdruleSXXunitEOne{} \and
      \NDdruleSXXtenI{} \and
      \NDdruleSXXtenETwo{} \and
      \NDdruleSXXtenEOne{} \and
      \NDdruleSXXimprI{} \and
      \NDdruleSXXimprE{} \and
      \NDdruleSXXimplI{} \and
      \NDdruleSXXimplE{} \and
      \NDdruleSXXFI{} \and
      \NDdruleSXXFE{} \and
      \NDdruleSXXGE{} \and
      \NDdruleSXXbeta{} \and
      \NDdruleSXXcutTwo{} \and
      \NDdruleSXXcutOne{}
    \end{mathpar}\\\\
    \hline
  \end{tabular}  
  \caption{Typing Rules for CNC Logic}
  \label{fig:CNC-typing-rules}
\end{figure}
We split the figure in two: the top of the figure are the rules of
intuitionistic linear logic whose judgment is the $\mathcal{C}$-typing
judgment denoted by $[[P |-c t : X]]$, and the bottom of the figure
are the rules for the mixed commutative/non-commutative Lambek
calculus whose judgment is the $\mathcal{L}$-judgment denoted by
$[[G |-l s : A]]$, and the two halves are connected via the rules 
rules $\NDdruleTXXGIName{}$, $\NDdruleSXXGEName{}$,
$\NDdruleSXXFIName{}$, and $\NDdruleSXXFEName{}$,
$\NDdruleSXXunitEOneName{}$, $\NDdruleSXXtenEOneName{}$, and
$\NDdruleSXXcutOneName{}$.

The one step $\beta$-reduction rules are listed in
Figure~\ref{fig:CNC-beta-reductions}. Similarly to the typing rules,
the figure is split in two: the top lists the rules of the
intuitionistic linear logic, and the bottom are those of the mixed
commutative/non-commutative Lambek calculus. 
\renewcommand{\NDdruleTbetaXXletUName}{}
\renewcommand{\NDdruleTbetaXXletTName}{}
\renewcommand{\NDdruleTbetaXXlamName}{}
\renewcommand{\NDdruleTbetaXXappOneName}{}
\renewcommand{\NDdruleTbetaXXappTwoName}{}
\renewcommand{\NDdruleTbetaXXappLetName}{}
\renewcommand{\NDdruleTbetaXXletLetName}{}
\renewcommand{\NDdruleTbetaXXletAppName}{}
\renewcommand{\NDdruleSbetaXXletUOneName}{}
\renewcommand{\NDdruleSbetaXXletTOneName}{}
\renewcommand{\NDdruleSbetaXXletTTwoName}{}
\renewcommand{\NDdruleSbetaXXletFName}{}
\renewcommand{\NDdruleSbetaXXlamLName}{}
\renewcommand{\NDdruleSbetaXXlamRName}{}
\renewcommand{\NDdruleSbetaXXapplOneName}{}
\renewcommand{\NDdruleSbetaXXapplTwoName}{}
\renewcommand{\NDdruleSbetaXXapprOneName}{}
\renewcommand{\NDdruleSbetaXXapprTwoName}{}
\renewcommand{\NDdruleSbetaXXderelictName}{}
\renewcommand{\NDdruleSbetaXXapplLetName}{}
\renewcommand{\NDdruleSbetaXXapprLetName}{}
\renewcommand{\NDdruleSbetaXXletLetName}{}
\renewcommand{\NDdruleSbetaXXletApplName}{}
\renewcommand{\NDdruleSbetaXXletApprName}{}
\renewcommand{\NDdruleTcomXXunitEXXunitEName}{}
\renewcommand{\NDdruleTcomXXunitEXXtenEName}{}
\renewcommand{\NDdruleTcomXXunitEXXimpEName}{}
\renewcommand{\NDdruleTcomXXtenEXXunitEName}{}
\renewcommand{\NDdruleTcomXXtenEXXtenEName}{}
\renewcommand{\NDdruleTcomXXtenEXXimpEName}{}
\renewcommand{\NDdruleTcomXXimpEXXunitEName}{}
\renewcommand{\NDdruleScomXXunitEXXunitEName}{}
\renewcommand{\NDdruleScomXXunitETwoXXunitEName}{}
\renewcommand{\NDdruleScomXXunitEXXimprEName}{}
\renewcommand{\NDdruleScomXXunitETwoXXimprEName}{}
\renewcommand{\NDdruleScomXXunitEXXFEName}{}
\renewcommand{\NDdruleScomXXunitETwoXXFEName}{}
\renewcommand{\NDdruleScomXXtenEXXunitEName}{}
\renewcommand{\NDdruleScomXXtenETwoXXunitEName}{}
\renewcommand{\NDdruleScomXXtenEXXtenEName}{}
\renewcommand{\NDdruleScomXXtenETwoXXtenEName}{}
\renewcommand{\NDdruleScomXXtenEXXimprEName}{}
\renewcommand{\NDdruleScomXXtenETwoXXimprEName}{}
\renewcommand{\NDdruleScomXXtenEXXimplEName}{}
\renewcommand{\NDdruleScomXXtenETwoXXimplEName}{}
\renewcommand{\NDdruleScomXXtenEXXFEName}{}
\renewcommand{\NDdruleScomXXtenETwoXXFEName}{}
\renewcommand{\NDdruleScomXXFEXXunitEName}{}
\renewcommand{\NDdruleScomXXFEXXtenEName}{}
\renewcommand{\NDdruleScomXXFEXXimprEName}{}
\renewcommand{\NDdruleScomXXFEXXimplEName}{}
\renewcommand{\NDdruleScomXXFEXXFEName}{}
\begin{figure}[!h]
  \footnotesize
  \begin{tabular}{|c|}
    \hline\\
      \begin{mathpar}
      \NDdruleTbetaXXletU{} \and
      \NDdruleTbetaXXletT{} \and
      \NDdruleTbetaXXlam{}
      \end{mathpar}
      \\
      \\
      \hline
      \\
    \begin{mathpar}
      \NDdruleSbetaXXletUOne{} \and
      \NDdruleSbetaXXletTOne{} \and
      \NDdruleSbetaXXletTTwo{} \and
      \NDdruleSbetaXXletF{} \and
      \NDdruleSbetaXXlamL{} \and
      \NDdruleSbetaXXlamR{} \and
      \NDdruleSbetaXXderelict{}
    \end{mathpar}\\\\
    \hline
  \end{tabular}  
  \caption{$\beta$-reductions for CNC Logic}
  \label{fig:CNC-beta-reductions}
\end{figure}


The commuting conversions can be found in
Figures~\ref{fig:CNC-commutating-conversions-intuitionistic}-\ref{fig:CNC-commutating-conversions-both}. We
divide the rules into three parts due to the length. The first part,
Figure~\ref{fig:CNC-commutating-conversions-intuitionistic}, includes
the rules for the intuitionistic linear logic. The second,
Figure~\ref{fig:CNC-commutating-conversions-mixed}, includes the rules
for the commutative/non-commutative Lambek calculus. The third,
Figure~\ref{fig:CNC-commutating-conversions-both}, includes the mixed
rules $\NDdruleSXXunitEOneName{}$ and $\NDdruleSXXtenEOneName{}$.

\begin{figure}[!h]
  \footnotesize
  \begin{tabular}{|c|}
    \hline\\
    \begin{mathpar}
      \NDdruleTcomXXunitEXXunitE{} \and
      \NDdruleTcomXXunitEXXtenE{} \and
      \NDdruleTcomXXunitEXXimpE{} \and
      \NDdruleTcomXXtenEXXunitE{} \and
      \NDdruleTcomXXtenEXXtenE{} \and
      \NDdruleTcomXXtenEXXimpE{} \and
      \NDdruleTcomXXimpEXXunitE{}
    \end{mathpar}
    \\
    \\
    \hline
  \end{tabular}  
  \caption{Commuting Conversions: Intuitionistic Linear Logic}
  \label{fig:CNC-commutating-conversions-intuitionistic}
\end{figure}
\begin{figure}[!h]
  \footnotesize
  \begin{tabular}{|c|}
    \hline\\
    \begin{mathpar}
      \NDdruleScomXXunitEXXunitE{} \and
      \NDdruleScomXXunitEXXimprE{} \and
      \NDdruleScomXXunitEXXFE{} \and
      \NDdruleScomXXtenEXXunitE{} \and
      \NDdruleScomXXtenEXXtenE{} \and
      \NDdruleScomXXtenEXXimprE{} \and
      \NDdruleScomXXtenEXXimplE{} \and
      \NDdruleScomXXtenEXXFE{} \and
      \NDdruleScomXXFEXXunitE{} \and
      \NDdruleScomXXFEXXtenE{} \and
      \NDdruleScomXXFEXXimprE{} \and
      \NDdruleScomXXFEXXimplE{} \and
      \NDdruleScomXXFEXXFE{}
    \end{mathpar}\\\\
    \hline
  \end{tabular}  
  \caption{Commuting Conversions: Commutative/Non-commutative Lambek Calculus}
  \label{fig:CNC-commutating-conversions-mixed}
\end{figure}
\begin{figure}[!h]
  \footnotesize
  \begin{tabular}{|c|}
    \hline\\
    \begin{mathpar}
      \NDdruleScomXXunitETwoXXunitE{} \and
      \NDdruleScomXXunitETwoXXimprE{} \and
      \NDdruleScomXXunitETwoXXFE{} \and
      \NDdruleScomXXtenETwoXXunitE{} \and
      \NDdruleScomXXtenETwoXXtenE{} \and
      \NDdruleScomXXtenETwoXXimprE{} \and
      \NDdruleScomXXtenETwoXXimplE{} \and
      \NDdruleScomXXtenETwoXXFE{}
    \end{mathpar}\\\\
    \hline
  \end{tabular}  
  \caption{Commuting Conversions: Mixed Rules}
  \label{fig:CNC-commutating-conversions-both}
\end{figure}

We also proved that the sequent calculus formalization given in
Figure~\ref{fig:CNC-sequent-calculus} is equivalent to the typing rules (or
else called the natural deduction formalization) given in
Figure~\ref{fig:CNC-typing-rules} are equivalent, as stated in the following
theorem.
\begin{theorem}
  \label{thm:sc-nd-equiv}
  The sequent calculus ($\mathit{SC}$) and natural deduction ($\mathit{ND}$)
  formalizations for CNC logic are equivalent in the sense that there are
  two mappins $N:\mathit{SC}\rightarrow\mathit{ND}$ and
  $S:\mathit{ND}\rightarrow\mathit{SC}$ that map each rule in $\mathit{SC}$
  to a proof in $\mathit{ND}$, and each rule in $\mathit{ND}$ to a proof
  in $\mathit{SC}$, respectively.
\end{theorem}
\begin{proof}
  The proof is done case by case on each rule in the sequence calculus and
  natural deduction formalizations. It is obvious that the axioms in one
  formalization can be mapped to the axioms in the other. The introduction
  rules in $\mathit{ND}$ are mapped to the right rules in $\mathit{SC}$, and
  vice versa. The elimination rules and lefts rules are mapped to each other
  with some fiddling. For instance, the elimination rule for the
  non-commutative tensor is mapped to the following proof in $\mathit{SC}$:
  \begin{center}
    \scriptsize
    \begin{math}
      $$\mprset{flushleft}
      \inferrule* [right={\scriptsize $\ElledruleTXXcutName$}] {
        {[[I |-c t1 : X (*) Y]]} \\
        $$\mprset{flushleft}
        \inferrule* [right={\scriptsize $\ElledruleTXXtenLName$}] {
          {[[P1, x : X, y : Y, P2 |-c t2 : Z]]}
        }{[[P1, z : X (*) Y, P2 |-c let z : X (*) Y be x (*) y in t2 : Z]]}
      }{[[P1, I, P2 |-c [t1 / z](let z : X (*) Y be x (*) y in t2) : Z]]}
    \end{math}
  \end{center}
  The full proof is in Appendix~\ref{app:sc-nd-equiv}.
\end{proof}

\noindent
Finally, we have strong normalization of CNC logic.

\begin{theorem}[Strong Normalization]
  \label{thm:strong-normalization}
  CNC logic is strongly normalizing.
\end{theorem}
\begin{proof}
  The proof is via a translation to LNL logic. The proof is in
  Appendix~\ref{app:strong-normalization}. We first define a function
  $CL$ that maps our term assignment for natural deduction to Benton's
  term assignment, and then shows that if there is reduction $[[t1 ~>  t2]]$ (resp. $[[s1 ~> s2]]$) in CNC, then there is a reduction
  $CL(t_1)\leadsto CL(t_2)$ (resp. $CL(s_1)\leadsto CL(s_2)$) in
  LNL. Recall that a LNL model consists of an adjunction
  $F:\cat{C}\dashv\cat{L}:G$ in which $\cat{C}$ is a cartesian closed
  category and $\cat{L}$ is a SMCC.  The basic idea of mapping a LAM
  model to a LNL model is mapping the SMCC in LAM to the cartesian
  closed category in LNL, and the Lambek category in LAM to the SMCC in
  LNL. For instance, $s_1\triangleright s_2$ in CNC, where $[[s1]]$ and
  $[[s2]]$ are atomic terms, maps to $CL(s_1)\otimes CL(s_2)$ in LNL.
\end{proof}

%%% Local Variables: 
%%% mode: latex
%%% TeX-master: main.tex
%%% End:
