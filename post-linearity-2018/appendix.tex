\section{Proof For Lemma~\ref{lem:cut-reduction}}
\label{app:cut-reduction}


\subsection{Commuting Conversion Cut vs. Cut}

\subsubsection{$\SCdruleTXXcutName$ vs. $\SCdruleTXXcutName$}
\begin{itemize}
% C-Cut vs. C-Cut Case 1
\item Case 1:
      \begin{center}
        \scriptsize
        \begin{math}
          \begin{array}{c}
            \Pi_1 \\
            {[[I |-c X]]}
          \end{array}
        \end{math}
        \qquad\qquad
        $\Pi_2:$
        \begin{math}
          $$\mprset{flushleft}
          \inferrule* [right={\tiny cut}] {
            {
              \begin{array}{cc}
                \pi_1 & \pi_2 \\
                {[[P2, X, P3 |-c Y]]} & {[[P1, Y, P4 |-c Z]]}
              \end{array}
            }
          }{[[P1, P2, X, P3, P4 |-c Z]]}
        \end{math}
      \end{center}
      By assumption, $c(\Pi_1),c(\Pi_2)\leq |X|$. Therefore, $c(\pi_1)$,
      $c(\pi_2)\leq |X|$. Since $Y$ is the cut formula on $\pi_1$ and
      $\pi_2$, we have $|Y|+1\leq|X|$. By induction on $\Pi_1$ and $\pi_1$
      there exists a proof $\Pi'$ for sequent $[[P2, I, P3 |-c Y]]$ s.t.
      $c(\Pi')\leq|X|$. So $\Pi$ can be constructed as follows, with
      $c(\Pi)\leq max\{c(\Pi'),c(\pi_2),|Y|+1\}\leq |X|$.
      \begin{center}
        \scriptsize
        \begin{math}
          $$\mprset{flushleft}
          \inferrule* [right={\tiny cut}] {
            {
              \begin{array}{cc}
                \Pi' & \pi_2 \\
                {[[P2, I, P3 |-c Y]]} & {[[P1, Y, P4 |-c Z]]}
              \end{array}
            }
          }{[[P1, P2, I, P3, P4 |-c Z]]}
        \end{math}
      \end{center}

% C-Cut vs. C-Cut Case 2
\item Case 2:
      \begin{center}
        \scriptsize
        $\Pi_1$:
        \begin{math}
          $$\mprset{flushleft}
          \inferrule* [right={\tiny cut}] {
            {
              \begin{array}{cc}
                \pi_1 & \pi_2 \\
                {[[I |-c X]]} & {[[P2, X, P3 |-c Y]]}
              \end{array}
            }
          }{[[P2, I, P3 |-c Y]]}
        \end{math}
        \qquad\qquad
        \begin{math}
          \begin{array}{c}
            \Pi_2 \\
            {[[P1, Y, P4 |-c Z]]}
          \end{array}
        \end{math}
      \end{center}
      By assumption, $c(\Pi_1),c(\Pi_2)\leq |Y|$. Since the cut rank of the last cut in
      $\Pi_1$ is $|X|+1$, then $|X|+1\leq |Y|$. By induction on $\Pi_1$ and $\Pi_2$, there is
      a proof $\Pi'$ for sequent $[[P1, P2, X, P3, P4 |-c Z]]$ s.t. $c(\Pi')\leq|Y|$.
      Therefore, the proof $\Pi$ can be constructed as follows, and
      $c(\Pi)\leq max\{c(\pi_1),c(\Pi'),|X|+1\}\leq |Y|$.
      \begin{center}
        \scriptsize
        \begin{math}
          $$\mprset{flushleft}
          \inferrule* [right={\tiny cut}] {
            {
              \begin{array}{cc}
                \pi_1 & \Pi' \\
                {[[I |-c X]]} & {[[P1, P2, X, P3, P4 |-c Z]]}
              \end{array}
            }
          }{[[P1, P2, I, P3, P4 |-c Z]]}
        \end{math}
      \end{center}
\end{itemize}



% C-Cut vs. LC-Cut Case 1
\subsubsection{$\SCdruleTXXcutName$ vs. $\SCdruleSXXcutOneName$}
\begin{itemize}
\item Case 1:
      \begin{center}
        \scriptsize
        \begin{math}
          \begin{array}{c}
            \Pi_1 \\
            {[[I |-c X]]}
          \end{array}
        \end{math}
        \qquad\qquad
        $\Pi_2:$
        \begin{math}
          $$\mprset{flushleft}
          \inferrule* [right={\tiny cut1}] {
            {
              \begin{array}{cc}
                \pi_2 & \pi_3 \\
                {[[P1, X, P2 |-c Y]]} & {[[G1; Y; G2 |-l A]]}
              \end{array}
            }
          }{[[G1; P1; X; P2; G2 |-l A]]}
        \end{math}
      \end{center}
      By assumption, $c(\Pi_1),c(\Pi_2)\leq |X|$. Therefore, $c(\pi_1)$,
      $c(\pi_2)\leq |X|$. Since $Y$ is the cut formula on $\pi_1$ and
      $\pi_2$, we have $|Y|+1\leq|X|$. By induction on $\Pi_1$ and $\pi_1$,
      there exists a proof $\Pi'$ for sequent $[[P1, I, P2 |-c Y]]$ s.t.
      $c(\Pi')\leq|X|$. So $\Pi$ can be constructed as follows, with
      $c(\Pi)\leq max\{c(\Pi'),c(\pi_2),|Y|+1\}\leq |X|$.
      \begin{center}
        \scriptsize
        \begin{math}
          $$\mprset{flushleft}
          \inferrule* [right={\tiny cut1}] {
            {
              \begin{array}{cc}
                \Pi' & \pi_2 \\
                {[[P1, I, P2 |-c Y]]} & {[[G1; Y; G2 |-l A]]}
              \end{array}
            }
          }{[[G1; P1; I; P2; G2 |-l A]]}
        \end{math}
      \end{center}

% C-Cut vs. LC-Cut Case 2
\item Case 2:
      \begin{center}
        \scriptsize
        $\Pi_1$:
        \begin{math}
          $$\mprset{flushleft}
          \inferrule* [right={\tiny cut}] {
            {
              \begin{array}{cc}
                \pi_1 & \pi_2 \\
                {[[I |-c X]]} & {[[P1, X, P2 |-c Y]]}
              \end{array}
            }
          }{[[P1, I, P2 |-c Y]]}
        \end{math}
        \qquad\qquad
        \begin{math}
          \begin{array}{c}
            \Pi_2 \\
            {[[G1; Y; G2 |-l A]]}
          \end{array}
        \end{math}
      \end{center}
      By assumption, $c(\Pi_1),c(\Pi_2)\leq |Y|$. Similar as above,
      $|X|+1\leq |Y|$ and there is a proof $\Pi'$ constructed from $\pi_2$
      and $\Pi_2$ for sequent $[[G1; P1; X; P2; G2 |-l A]]$ s.t.
      $c(\Pi')\leq|Y|$. Therefore, the proof $\Pi$ can be constructed as
      follows, and $c(\Pi)\leq max\{c(\pi_1),c(\Pi'),|X|+1\}\leq |Y|$.
      \begin{center}
        \scriptsize
        \begin{math}
          $$\mprset{flushleft}
          \inferrule* [right={\tiny cut}] {
            {
              \begin{array}{cc}
                \pi_1 & \Pi'\\
                {[[I |-c X]]} & {[[G1; P1; X; P2; G2 |-l A]]}
              \end{array}
            }
          }{[[G1; P1; I; P2; G2 |-l A]]}
        \end{math}
      \end{center}
\end{itemize}

% LC-Cut vs. L-Cut Case 1
\subsubsection{$\SCdruleSXXcutOneName$ vs. $\SCdruleSXXcutTwoName$}
\begin{itemize}
\item Case 1:
      \begin{center}
        \scriptsize
        \begin{math}
          \begin{array}{c}
            \Pi_1 \\
            {[[I |-c X]]}
          \end{array}
        \end{math}
        \qquad\qquad
        $\Pi_2:$
        \begin{math}
          $$\mprset{flushleft}
          \inferrule* [right={\tiny cut2}] {
            {
              \begin{array}{cc}
                \pi_1 & \pi_2 \\
                {[[G2; X; G3 |-l A]]} & {[[G1; A; G4 |-l B]]}
              \end{array}
            }
          }{[[G1; G2; X; G3; G4 |-l B]]}
        \end{math}
      \end{center}
      By assumption, $c(\Pi_1),c(\Pi_2)\leq |X|$. Therefore, $c(\pi_1)$,
      $c(\pi_2)\leq |X|$. Since $A$ is the cut formula on $\pi_1$ and
      $\pi_2$, we have $|A|+1\leq|X|$. By induction on $\Pi_1$ and $\pi_1$,
      there exists a proof $\Pi'$ for sequent $[[G2; I; G3 |-l A]]$ s.t.
      $c(\Pi')\leq|X|$. So $\Pi$ can be constructed as follows, with
      $c(\Pi)\leq max\{c(\Pi'),c(\pi_2),|A|+1\}\leq |X|$.
      \begin{center}
        \scriptsize
        \begin{math}
          $$\mprset{flushleft}
          \inferrule* [right={\tiny cut2}] {
            {
              \begin{array}{cc}
                \Pi' & \pi_2 \\
                {[[G2; I; G3 |-l A]]} & {[[G1; A; G4 |-l B]]}
              \end{array}
            }
          }{[[G1; G2; I; G3; G4 |-l B]]}
        \end{math}
      \end{center}

% LC-Cut vs. L-Cut Case 2
\item Case 2:
      \begin{center}
        \scriptsize
        $\Pi_1$:
        \begin{math}
          $$\mprset{flushleft}
          \inferrule* [right={\tiny cut}] {
            {
              \begin{array}{cc}
                \pi_1 & \pi_2 \\
                {[[I |-c X]]} & {[[G2; X; G3 |-l A]]}
              \end{array}
            }
          }{[[G2; I; G3 |-l A]]}
        \end{math}
        \qquad\qquad
        \begin{math}
          \begin{array}{c}
            \Pi_2 \\
            {[[G1; A; G4 |-l B]]}
          \end{array}
        \end{math}
      \end{center}
      By assumption, $c(\Pi_1),c(\Pi_2)\leq |A|$. Similar as above,
      $|X|+1\leq |A|$ and there is a proof $\Pi'$ constructed from'
      $\pi_2$ and $\Pi_2$ for sequent $[[G1; G2; X; G3; G4 |-l B]]$ s.t.
      $c(\Pi')\leq|A|$. Therefore, the proof $\Pi$ can be constructed as
      follows, and $c(\Pi)\leq max\{c(\pi_1),c(\Pi'),|X|+1\}\leq |A|$.
      \begin{center}
        \scriptsize
        \begin{math}
          $$\mprset{flushleft}
          \inferrule* [right={\tiny cut}] {
            {
              \begin{array}{cc}
                \pi_1  & \Pi' \\
                {[[I |-c X]]} & {[[G1; G2; X; G3; G4 |-l B]]}
              \end{array}
            }
          }{[[G1; G2; I; G3; G4 |-l B]]}
        \end{math}
      \end{center}
\end{itemize}

% L-Cut vs. L-Cut Case 1
\subsubsection{$\SCdruleSXXcutTwoName$ vs. $\SCdruleSXXcutTwoName$}
\begin{itemize}
\item Case 1:
      \begin{center}
        \scriptsize
        \begin{math}
          \begin{array}{c}
            \Pi_1 \\
            {[[G |-l A]]}
          \end{array}
        \end{math}
        \qquad\qquad
        $\Pi_2:$
        \begin{math}
          $$\mprset{flushleft}
          \inferrule* [right={\tiny cut2}] {
            {
              \begin{array}{cc}
                \pi_1 & \pi_2 \\
                {[[D2; A; D3 |-l B]]} & {[[D1; B; D4 |-l C]]}
              \end{array}
            }
          }{[[D1; D2; A; D3; D4 |-l C]]}
        \end{math}
      \end{center}
      By assumption, $c(\Pi_1),c(\Pi_2)\leq |A|$. Therefore, $c(\pi_1)$,
      $c(\pi_2)\leq |A|$. Since $B$ is the cut formula on $\pi_1$ and
      $\pi_3$, we have $|B|+1\leq|A|$. By induction on $\Pi_1$ and
      $\pi_1$, there exists a proof $\Pi'$ for sequent
      $[[D2; G; D3 |-l B]]$ s.t. $c(\Pi')\leq|A|$. So $\Pi$ can be
      constructed as follows,  with
      $c(\Pi)\leq max\{c(\Pi'),c(\pi_2),|B|+1\}\leq |A|$.
      \begin{center}
        \scriptsize
        \begin{math}
          $$\mprset{flushleft}
          \inferrule* [right={\tiny cut}] {
            {
              \begin{array}{cc}
                \Pi' & \pi_2 \\
                {[[D2; G; D3 |-l B]]} & {[[D1; B; D4 |-l C]]}
              \end{array}
            }
          }{[[D1; D2; G; D3; D4 |-l C]]}
        \end{math}
      \end{center}

% L-Cut vs. L-Cut Case 2
\item Case 2:
      \begin{center}
        \scriptsize
        $\Pi_1$:
        \begin{math}
          $$\mprset{flushleft}
          \inferrule* [right={\tiny cut}] {
            {
              \begin{array}{cc}
                \pi_1 & \pi_2 \\
                {[[D |-l A]]} & {[[D2; A; D3 |-l B]]}
              \end{array}
            }
          }{[[D2; D; D3 |-l A]]}
        \end{math}
        \qquad\qquad
        \begin{math}
          \begin{array}{c}
            \Pi_2 \\
            {[[D1; B; D4 |-l C]]}
          \end{array}
        \end{math}
      \end{center}
      By assumption, $c(\Pi_1),c(\Pi_2)\leq |B|$. Similar as above,
      $|A|+1\leq |B|$ and there is a proof $\Pi'$ constructed from $\pi_2$ 
      and $\Pi_2$ for sequent $[[D1; D2; A; D3; D4 |-l C]]$ s.t.
      $c(\Pi')\leq|A|$. Therefore, the proof $\Pi$ can be constructed as
      follows, and $c(\Pi)\leq max\{c(\pi_1),c(\Pi'),|A|+1\}\leq |B|$.
      \begin{center}
        \scriptsize
        \begin{math}
          $$\mprset{flushleft}
          \inferrule* [right={\tiny cut}] {
            {
              \begin{array}{cc}
                \pi_1 & \Pi' \\
                {[[G |-l A]]} & {[[D1; D2; A; D3; D4 |-l C]]}
              \end{array}
            }
          }{[[D1; D2; G; D3; D4 |-l C]]}
        \end{math}
      \end{center}

\end{itemize}
% End of subsubsection Commuting conversion cut vs. cut



\subsection{The Axiom Steps}

\subsubsection{$\SCdruleTXXaxName$}
\begin{itemize}
% C-id Case 1
\item Case 1:
      \begin{center}
        \scriptsize
        $\Pi_1$:
        \begin{math}
          $$\mprset{flushleft}
          \inferrule* [right={\tiny ax}] {
            \,
          }{[[X |-c X]]}
        \end{math}
        \qquad\qquad
        \begin{math}
          \begin{array}{c}
            \Pi_2 \\
            {[[I1, X, I2 |-c Y]]}
          \end{array}
        \end{math}
      \end{center}
      By assumption, $c(\Pi_1),c(\Pi_2)\leq |X|$. The proof $\Pi$ is the
      same as $\Pi_2$.

% C-id Case 2
\item Case 2:
      \begin{center}
        \scriptsize
        $\Pi_1$:
        \begin{math}
          \begin{array}{c}
            \Pi_1 \\
            {[[I |-c X]]}
          \end{array}
        \end{math}
        \qquad\qquad
        $\Pi_2$:
        \begin{math}
          $$\mprset{flushleft}
          \inferrule* [right={\tiny ax}] {
            \,
          }{[[X |-c X]]}
        \end{math}
      \end{center}
      By assumption, $c(\Pi_1),c(\Pi_2)\leq |X|$. The proof $\Pi$ is the
      same as $\Pi_1$.

% C-id Case 3
\item Case 3:
      \begin{center}
        \scriptsize
        $\Pi_1$:
        \begin{math}
          $$\mprset{flushleft}
          \inferrule* [right={\tiny ax}] {
            \,
          }{[[X |-c X]]}
        \end{math}
        \qquad\qquad
        \begin{math}
          \begin{array}{c}
            \Pi_2 \\
            {[[G1; X; G2 |-l A]]}
          \end{array}
        \end{math}
      \end{center}
      By assumption, $c(\Pi_1),c(\Pi_2)\leq |X|$. The proof $\Pi$ is the
      same as $\Pi_2$.
\end{itemize}

% L-id Case 1
\subsubsection{$\SCdruleTXXaxName$}
\begin{itemize}
\item Case 1:
      \begin{center}
        \scriptsize
        $\Pi_1$:
        \begin{math}
          $$\mprset{flushleft}
          \inferrule* [right={\tiny ax}] {
            \,
          }{[[A |-l A]]}
        \end{math}
        \qquad\qquad
        \begin{math}
          \begin{array}{c}
            \Pi_2 \\
            {[[G1; A; G2 |-l B]]}
          \end{array}
        \end{math}
      \end{center}
      By assumption, $c(\Pi_1),c(\Pi_2)\leq |A|$. The proof $\Pi$ is the
      same as $\Pi_2$.

% L-id Case 2
\item Case 2:
      \begin{center}
        \scriptsize
        $\Pi_1$:
        \begin{math}
          \begin{array}{c}
            \Pi_1 \\
            {[[D |-l A]]}
          \end{array}
        \end{math}
        \qquad\qquad
        $\Pi_2$:
        \begin{math}
          $$\mprset{flushleft}
          \inferrule* [right={\tiny ax}] {
            \,
          }{[[A |-l A]]}
        \end{math}
      \end{center}
      By assumption, $c(\Pi_1),c(\Pi_2)\leq |X|$. The proof $\Pi$ is the
      same as $\Pi_1$.
\end{itemize}
% End of subsubsection Axiom steps



\subsection{The Exchange Steps}

\subsubsection{$\SCdruleTXXexName$}

\begin{itemize}
% Conclusion vs. C-ex Case 1
\item Case 1:
      \begin{center}
        \scriptsize
        \begin{math}
          \begin{array}{c}
            \Pi_1 \\
            {[[P |-c X1]]}
          \end{array}
        \end{math}
        \qquad\qquad
        $\Pi_2$:
        \begin{math}
          $$\mprset{flushleft}
          \inferrule* [right={\tiny ex}] {
            {
              \begin{array}{c}
                \pi \\
                {[[I1, X1, X2, I2 |-c Y]]}
              \end{array}
            }
          }{[[I1, X2, X1, I2 |-c Y]]}
        \end{math}
      \end{center}
      By assumption, $c(\Pi_1),c(\Pi_2)\leq |X_1|$. By induction on $\pi$
      and $\Pi_1$, there is a proof $\Pi'$ for sequent
      $[[I1, P, X2, I2 |-c Y]]$ s.t. $c(\Pi')\leq|X_1|$. Therefore, the
      proof $\Pi$ can be constructed as follows, and
      $c(\Pi)=c(\Pi')\leq|X_1|$.
      \begin{center}
        \scriptsize
        \begin{math}
          $$\mprset{flushleft}
          \inferrule* [right={\tiny series of ex}] {
            {
              \begin{array}{c}
                \Pi' \\
                {[[I1, P, X2, I2 |-c Y]]}
              \end{array}
            }
          }{[[I1, X2, P, I2 |-c Y]]}
        \end{math}
      \end{center}

% Conclusion vs. C-ex Case 2
\item Case 2:
      \begin{center}
        \scriptsize
        \begin{math}
          \begin{array}{c}
            \Pi_1 \\
            {[[P |-c X2]]}
          \end{array}
        \end{math}
        \qquad\qquad
        $\Pi_2$:
        \begin{math}
          $$\mprset{flushleft}
          \inferrule* [right={\tiny ex}] {
            {
              \begin{array}{c}
                \pi \\
                {[[I1, X1, X2, I2 |-c Y]]}
              \end{array}
            }
          }{[[I1, X2, X1, I2 |-c Y]]}
        \end{math}
      \end{center}
      By assumption, $c(\Pi_1),c(\Pi_2)\leq |X_2|$. By induction on $\pi$
      and $\Pi_1$, there is a proof $\Pi'$ for sequent
      $[[I1, X1, P, I2 |-c Y]]$ s.t. $c(\Pi')\leq|X_2|$. Therefore, the
      proof $\Pi$ can be constructed as follows, and
      $c(\Pi)=c(\Pi')\leq|X_2|$.
      \begin{center}
        \scriptsize
        \begin{math}
          $$\mprset{flushleft}
          \inferrule* [right={\tiny series of ex}] {
            {
              \begin{array}{c}
                \Pi' \\
                {[[I1, X1, P, I2 |-c Y]]}
              \end{array}
            }
          }{[[I1, P, X1, I2 |-c Y]]}
        \end{math}
      \end{center}
\end{itemize}

% Conclusion vs. LC-ex Case 1
\subsubsection{$\SCdruleSXXexName$}
\begin{itemize}
\item Case 1:
      \begin{center}
        \scriptsize
        \begin{math}
          \begin{array}{c}
            \Pi_1 \\
            {[[P |-c X1]]}
          \end{array}
        \end{math}
        \qquad\qquad
        $\Pi_2$:
        \begin{math}
          $$\mprset{flushleft}
          \inferrule* [right={\tiny ex}] {
            {
              \begin{array}{c}
                \pi \\
                {[[D1; X1; X2; D2 |-l A]]}
              \end{array}
            }
          }{[[D1; X2; X1; D2 |-l A]]}
        \end{math}
      \end{center}
      By assumption, $c(\Pi_1),c(\Pi_2)\leq |X_1|$. By induction on $\pi$
      and $\Pi_1$, there is a proof $\Pi'$ for sequent
      $[[D1; P; X2; D2 |-l A]]$ s.t. $c(\Pi')\leq|X_1|$. Therefore, the
      proof $\Pi$ can be constructed as follows, and
      $c(\Pi)=c(\Pi')\leq|X_1|$.
      \begin{center}
        \scriptsize
        \begin{math}
          $$\mprset{flushleft}
          \inferrule* [right={\tiny series of ex}] {
            {
              \begin{array}{c}
                \Pi' \\
                {[[D1; P; X2; D2 |-l A]]}
              \end{array}
            }
          }{[[D1; X2; P; D2 |-l A]]}
        \end{math}
      \end{center}

% Conclusion vs. LC-ex Case 2
\item Case 2:
      \begin{center}
        \scriptsize
        \begin{math}
          \begin{array}{c}
            \Pi_1 \\
            {[[P |-c X2]]}
          \end{array}
        \end{math}
        \qquad\qquad
        $\Pi_2$:
        \begin{math}
          $$\mprset{flushleft}
          \inferrule* [right={\tiny ex}] {
            {
              \begin{array}{c}
                \pi \\
                {[[D1; X1; X2; D2 |-l A]]}
              \end{array}
            }
          }{[[D1; X2; X1; D2 |-l A]]}
        \end{math}
      \end{center}
      By assumption, $c(\Pi_1),c(\Pi_2)\leq |X_2|$. By induction on $\pi$
      and $\Pi_1$, there is a proof $\Pi'$ for sequent
      $[[D1; X1; P; D2 |-l A]]$ s.t. $c(\Pi')\leq|X_2|$. Therefore, the
      proof $\Pi$ can be constructed as follows, and
      $c(\Pi)=c(\Pi')\leq|X_2|$.
      \begin{center}
        \scriptsize
        \begin{math}
          $$\mprset{flushleft}
          \inferrule* [right={\tiny series of ex}] {
            {
              \begin{array}{c}
                \Pi' \\
                {[[D1; X1; P; I2 |-l A]]}
              \end{array}
            }
          }{[[I1; P; X1; I2 |-l A]]}
        \end{math}
      \end{center}
\end{itemize}



\subsection{Principal Formula vs. Principal Formula} 

\subsubsection{The Commutative Tensor Product $\otimes$}
\begin{center}
  \scriptsize
  $\Pi_1:$
  \begin{math}
    $$\mprset{flushleft}
    \inferrule* [right={\tiny tenR}] {
      {
        \begin{array}{cc}
          \pi_1 & \pi_2 \\
          {[[I1 |-c X]]} & {[[I2 |-c Y]]}
        \end{array}
      }
    }{[[I1, I2 |-c X (*) Y]]}
  \end{math}
  \qquad\qquad
  $\Pi_2:$
  \begin{math}
    $$\mprset{flushleft}
    \inferrule* [right={\tiny tenL}] {
      {
        \begin{array}{c}
          \pi_3 \\
          {[[P1, X, Y, P2 |-c Z]]}
        \end{array}
      }
    }{[[P1, X (*) Y, P2 |-c Z]]}
  \end{math}
\end{center}
By assumption, $c(\Pi_1),c(\Pi_2)\leq |[[X (*) Y]]| = |X|+|Y|+1$. The proof
$\Pi$ can be constructed as follows, and
$c(\Pi)\leq max\{c(\pi_1),c(\pi_2),c(\pi_3),|X|+1,|Y|+1\}\leq |X|+|Y|+1 = |[[X (*) Y]]|$.
\begin{center}
  \scriptsize
  \begin{math}
    $$\mprset{flushleft}
    \inferrule* [right={\tiny cut}] {
      {
        \begin{array}{c}
          \pi_1 \\
          {[[I1 |-c X]]}
        \end{array}
      }
      $$\mprset{flushleft}
      \inferrule* [right={\tiny cut}] {
      {
        \begin{array}{cc}
          \pi_2 & \pi_3 \\
          {[[I2 |-c Y]]} & {[[P1, X, Y, P2 |-c Z]]}
        \end{array}
      }
      }{[[P1, X, I2, P2 |-c Z]]}
    }{[[P1, I1, I2, P2 |-c Z]]}
  \end{math}
\end{center}

\subsubsection{The Non-commutative Tensor Product $\tri$}
\begin{center}
  \scriptsize
  $\Pi_1:$
  \begin{math}
    $$\mprset{flushleft}
    \inferrule* [right={\tiny tenR}] {
      {
        \begin{array}{cc}
          \pi_1 & \pi_2 \\
          {[[G1 |-l A]]} & {[[G2 |-l B]]}
        \end{array}
      }
    }{[[G1; G2 |-l A (>) B]]}
  \end{math}
  \qquad\qquad
  $\Pi_2:$
  \begin{math}
    $$\mprset{flushleft}
    \inferrule* [right={\tiny tenL1}] {
      {
        \begin{array}{c}
          \pi_3 \\
          {[[D1; A; B; D2 |-l C]]}
        \end{array}
      }
    }{[[D1; A (>) B; D2 |-l C]]}
  \end{math}
\end{center}
By assumption, $c(\Pi_1),c(\Pi_2)\leq |[[A (>) B]]| = |X|+|Y|+1$. The proof
$\Pi$ can be constructed as follows, and
$c(\Pi)\leq max\{c(\pi_1),c(\pi_2),c(\pi_3),|A|+1,|B|+1\}\leq |A|+|B|+1 = |[[A (>) B]]|$.
\begin{center}
  \scriptsize
  \begin{math}
    $$\mprset{flushleft}
    \inferrule* [right={\tiny cut2}] {
      {
        \begin{array}{c}
          \pi_1 \\
          {[[G1 |-l A]]}
        \end{array}
      }
      $$\mprset{flushleft}
      \inferrule* [right={\tiny cut2}] {
      {
        \begin{array}{cc}
          \pi_2 & \pi_3 \\
          {[[G2 |-l B]]} & {[[D1; A; B; D2 |-l C]]}
        \end{array}
      }
      }{[[D1; A; G2; D2 |-l C]]}
    }{[[D1; G1; G2; P2 |-l C]]}
  \end{math}
\end{center}

\subsubsection{The Commutative Implication $\multimap$}
\begin{center}
  \scriptsize
  $\Pi_1:$
  \begin{math}
    $$\mprset{flushleft}
    \inferrule* [right={\tiny tenR}] {
      {
        \begin{array}{c}
          \pi_1 \\
          {[[I1, X |-c Y]]}
        \end{array}
      }
    }{[[I1 |-c X -o Y]]}
  \end{math}
  \qquad\qquad
  $\Pi_2:$
  \begin{math}
    $$\mprset{flushleft}
    \inferrule* [right={\tiny tenL}] {
      {
        \begin{array}{cc}
          \pi_2 & \pi_3 \\
          {[[I2 |-c X]]} & {[[P1, Y, P2 |-c Z]]}
        \end{array}
      }
    }{[[P1, X -o Y, I, P2 |-c Z]]}
  \end{math}
\end{center}
By assumption, $c(\Pi_1),c(\Pi_2)\leq |[[X -o Y]]| = |X|+|Y|+1$. The proof 
$\Pi$ is constructed as follows
$c(\Pi)\leq max\{c(\pi_1),c(\pi_2),c(\pi_3),|X|+1,|Y|+1\}\leq |X|+|Y|+1 = |[[X -o Y]]|$.
\begin{center}
  \scriptsize
  \begin{math}
    $$\mprset{flushleft}
    \inferrule* [right={\tiny tenR}] {
      $$\mprset{flushleft}
      \inferrule* [right={\tiny tenR}] {
        {
          \begin{array}{cc}
            \pi_1 & \pi_2 \\
            {[[I1, X |-c Y]]} & {[[I2 |-c X]]}
          \end{array}
        }
      }{[[I1, I2 |-c Y]]} \\
       {
         \begin{array}{c}
           \pi_3 \\
           {[[P1, Y, P2 |-c Z]]}
         \end{array}
       }
    }{[[P1, I1, I2, P2 |-c Z]]}
  \end{math}
\end{center}

\subsubsection{The Non-commutative Right Implication $\lto$}
\begin{center}
  \scriptsize
  $\Pi_1:$
  \begin{math}
    $$\mprset{flushleft}
    \inferrule* [right={\tiny imprR}] {
      {
        \begin{array}{c}
          \pi_1 \\
          {[[G; A |-l B]]}
        \end{array}
      }
    }{[[G |-l A -> B]]}
  \end{math}
  \qquad\qquad
  $\Pi_2:$
  \begin{math}
    $$\mprset{flushleft}
    \inferrule* [right={\tiny imprL}] {
      {
        \begin{array}{cc}
          \pi_2 & \pi_3 \\
          {[[D1 |-l A]]} & {[[D2; B |-l C]]}
        \end{array}
      }
    }{[[D2; A -> B; D1 |-l C]]}
  \end{math}
\end{center}
By assumption, $c(\Pi_1),c(\Pi_2)\leq |[[A -> B]]| = |A|+|B|+1$. The proof
$\Pi$ is constructed as follows, and
$c(\Pi)\leq max\{c(\pi_1),c(\pi_2),c(\pi_3),|A|+1,|B|+1\}\leq |A|+|B|+1 = |[[A -> B]]|$.
\begin{center}
  \scriptsize
  \begin{math}
    $$\mprset{flushleft}
    \inferrule* [right={\tiny cut2}] {
      $$\mprset{flushleft}
      \inferrule* [right={\tiny cut2}] {
        {
          \begin{array}{cc}
            \pi_1 & \pi_2 \\
            {[[G; A |-l B]]} & {[[D1 |-l A]]}
          \end{array}
        }
      }{[[G; D1 |-l B]]}
       {
         \begin{array}{c}
           \pi_3 \\
           {[[D2; B |-l C]]}
         \end{array}
       }
    }{[[D2; G; D1 |-l C]]}
  \end{math}
\end{center}

\subsubsection{The Non-commutative Left Implication $\rto$}
\begin{center}
  \scriptsize
  $\Pi_1:$
  \begin{math}
    $$\mprset{flushleft}
    \inferrule* [right={\tiny implR}] {
      {
        \begin{array}{c}
          \pi_1 \\
          {[[A; G |-l B]]}
        \end{array}
      }
    }{[[G |-l B <- A]]}
  \end{math}
  \qquad\qquad
  $\Pi_2:$
  \begin{math}
    $$\mprset{flushleft}
    \inferrule* [right={\tiny implL}] {
      {
        \begin{array}{cc}
          \pi_2 & \pi_3 \\
          {[[D1 |-l A]]} & {[[B; D2 |-l C]]}
        \end{array}
      }
    }{[[D1; B <- A; D2 |-l C]]}
  \end{math}
\end{center}
By assumption, $c(\Pi_1),c(\Pi_2)\leq |[[B <- A]]| = |A|+|B|+1$. The
proof $\Pi$ is constructed as follows, and
$c(\Pi)\leq max\{c(\pi_1),c(\pi_2),c(\pi_3),|A|+1,|B|+1\}\leq |A|+|B|+1 = |[[B <- A]]|$.
\begin{center}
  \scriptsize
  \begin{math}
    $$\mprset{flushleft}
    \inferrule* [right={\tiny cut1}] {
      $$\mprset{flushleft}
      \inferrule* [right={\tiny cut2}] {
        {
          \begin{array}{cc}
            \pi_1 & \pi_2 \\
            {[[A; G |-l B]]} & {[[D1 |-l A]]}
          \end{array}
        }
      }{[[D1; G |-l B]]}
       {
         \begin{array}{c}
           \pi_3 \\
           {[[B; D2 |-l C]]}
         \end{array}
       }
    }{[[D1; G; D2 |-l C]]}
  \end{math}
\end{center}



\subsubsection{The Commutative Unit $[[UnitT]]$}
\begin{itemize}
\item Case 1:
      \begin{center}
        \scriptsize
        $\Pi_1:$
        \begin{math}
          $$\mprset{flushleft}
          \inferrule* [right={\tiny unitR}] {
            \,
          }{[[. |-c UnitT]]}
        \end{math}
        \qquad\qquad
        $\Pi_2:$
        \begin{math}
          $$\mprset{flushleft}
          \inferrule* [right={\tiny unitL}] {
            {
              \begin{array}{c}
                \pi \\
                {[[I, P |-c X]]}
              \end{array}
            }
          }{[[I, UnitT, P |-c X]]}
        \end{math}
      \end{center}
      By assumption, $c(\Pi_1),c(\Pi_2)\leq |[[UnitT]]|$. The proof $\Pi$
      is the subproof $\pi$ in $\Pi_2$ for sequent $[[I |-c X]]$. So
      $c(\Pi)=c(\Pi_2)\leq |[[UnitT]]|$.

\item Case 2:
      \begin{center}
        \scriptsize
        $\Pi_1:$
        \begin{math}
          $$\mprset{flushleft}
          \inferrule* [right={\tiny unitR}] {
            \,
          }{[[. |-c UnitT]]}
        \end{math}
        \qquad\qquad
        $\Pi_2:$
        \begin{math}
          $$\mprset{flushleft}
          \inferrule* [right={\tiny unitL1}] {
            {
              \begin{array}{c}
                \pi \\
                {[[G; D |-l A]]}
              \end{array}
            }
          }{[[G; UnitT; D |-l A]]}
        \end{math}
      \end{center}
      Similar as above, $\Pi$ is $\pi$.
\end{itemize}


\subsubsection{The Non-commutative Unit $[[UnitS]]$}
\begin{center}
  \scriptsize
  $\Pi_1:$
  \begin{math}
    $$\mprset{flushleft}
    \inferrule* [right={\tiny unitR}] {
      \,
    }{[[. |-l UnitS]]}
  \end{math}
  \qquad\qquad
  $\Pi_2:$
  \begin{math}
    $$\mprset{flushleft}
    \inferrule* [right={\tiny unitL2}] {
      {
        \begin{array}{c}
          \pi \\
          {[[G; D |-l A]]}
        \end{array}
      }
    }{[[G; UnitS; D |-l A]]}
  \end{math}
\end{center}
By assumption, $c(\Pi_1),c(\Pi_2)\leq |[[UnitS]]|$. The proof $\Pi$ is the
subproof $\pi$ in $\Pi_2$ for sequent $[[D |-l A]]$. So
$c(\Pi)=c(\Pi_2)\leq |[[UnitS]]|$.

\subsubsection{The Functor $F$}
\begin{center}
  \scriptsize
  $\Pi_1:$
  \begin{math}
    $$\mprset{flushleft}
    \inferrule* [right={\tiny FR}] {
      {
        \begin{array}{c}
          \pi_1 \\
          {[[I |-c X]]}
        \end{array}
      }
    }{[[I |-l F X]]}
  \end{math}
  \qquad\qquad
  $\Pi_2:$
  \begin{math}
    $$\mprset{flushleft}
    \inferrule* [right={\tiny FL}] {
      {
        \begin{array}{c}
          \pi_2 \\
          {[[G; X; D |-l A]]}
        \end{array}
      }
    }{[[G; F X; D |-l A]]}
  \end{math}
\end{center}
By assumption, $c(\Pi_1),c(\Pi_2)\leq |[[F X]]| = |X|+1$. The proof
$\Pi$ is constructed as follows, and \\
$c(\Pi)\leq max\{c(\pi_1),c(\pi_2),|X|+1\}\leq |[[F X]]|$.
\begin{center}
  \scriptsize
  \begin{math}
    $$\mprset{flushleft}
    \inferrule* [right={\tiny cut2}] {
      {
        \begin{array}{cc}
          \pi_1 & \pi_2 \\
          {[[I |-c X]]} & {[[G; X; D |-l A]]}
        \end{array}
      }
    }{[[G; I; D |-l A]]}
  \end{math}
\end{center}

\subsubsection{The Functor $G$}
\begin{center}
  \scriptsize
  $\Pi_1:$
  \begin{math}
    $$\mprset{flushleft}
    \inferrule* [right={\tiny GR}] {
      {
        \begin{array}{c}
          \pi_1 \\
          {[[I |-l A]]}
        \end{array}
      }
    }{[[I |-c Gf A]]}
  \end{math}
  \qquad\qquad
  $\Pi_2:$
  \begin{math}
    $$\mprset{flushleft}
    \inferrule* [right={\tiny GL}] {
      {
        \begin{array}{c}
          \pi_2 \\
          {[[G; A; D |-l B]]}
        \end{array}
      }
    }{[[G; Gf A; D |-l B]]}
  \end{math}
\end{center}
By assumption, $c(\Pi_1),c(\Pi_2)\leq |[[Gf A]]| = |A|+1$. The proof $\Pi$ 
is constructed as follows, and \\
$c(\Pi)\leq max\{c(\pi_1),c(\pi_2),|A|+1\}\leq |[[Gf A]]|$.
\begin{center}
  \scriptsize
  \begin{math}
    $$\mprset{flushleft}
    \inferrule* [right={\tiny GL}] {
      {
        \begin{array}{cc}
          \pi_1 & \pi_2 \\
          {[[I |-l A]]} & {[[G; A; D |-l B]]}
        \end{array}
      }
    }{[[G; I; D |-l B]]}
  \end{math}
\end{center}



\subsection{Secondary Conclusion}

\subsubsection{Left introduction of the commutative implication $\multimap$}
\begin{itemize}
\item Case 1:
      \begin{center}
        \scriptsize
        $\Pi_1$:
        \begin{math}
          $$\mprset{flushleft}
          \inferrule* [right={\tiny impL}] {
            {
              \begin{array}{cc}
                \pi_1 & \pi_2 \\
                {[[I1 |-c X1]]} & {[[I2, X2, I3 |-c Y]]}
              \end{array}
            }
          }{[[I2, X1 -o X2, I1, I3 |-c Y]]}
        \end{math}
        \qquad\qquad
        \begin{math}
          \begin{array}{c}
            \Pi_2 \\
            {[[P1, Y, P2 |-c Z]]}
          \end{array}
        \end{math}
      \end{center}
      By assumption, $c(\Pi_1),c(\Pi_2)\leq |Y|$. By induction, there is a
      proof $\Pi'$ from $\pi_2$ and $\Pi_2$ for sequent
      $[[P1, I2, X2, I3, P2 |-c Z]]$ s.t. $c(\Pi')\leq |Y|$. Therefore,
      the proof $\Pi$ can be constructed as follows with $c(\Pi)\leq |Y|$.
      \begin{center}
        \scriptsize
        \begin{math}
          $$\mprset{flushleft}
          \inferrule* [right={\tiny impL}] {
            {
              \begin{array}{c}
                \pi_1 \\
                {[[I1 |-c X1]]}
              \end{array}
            }
            $$\mprset{flushleft}
            \inferrule* [right={\tiny cut}] {
              {
                \begin{array}{cc}
                  \pi_2 & \Pi_2 \\
                  {[[I2, X2, I3 |-c Y]]} & {[[P1, Y, P2 |-c Z]]}
                \end{array}
              }
            }{[[P1, I2, X2, I3, P2 |-c Z]]}
          }{[[P1, I2, X1 -o X2, I1, I3, P2 |-c Z]]}
        \end{math}
      \end{center}

\item Case 2:
      \begin{center}
        \scriptsize
        $\Pi_1$:
        \begin{math}
          $$\mprset{flushleft}
          \inferrule* [right={\tiny impL}] {
            {
              \begin{array}{cc}
                \pi_1 & \pi_2 \\
                {[[I1 |-c X1]]} & {[[I2, X2, I3 |-c Y]]}
              \end{array}
            }
          }{[[I2, X1 -o X2, I1, I3 |-c Y]]}
        \end{math}
        \qquad\qquad
        \begin{math}
          \begin{array}{c}
            \Pi_2 \\
            {[[G1; Y; G2 |-l A]]}
          \end{array}
        \end{math}
      \end{center}
      By assumption, $c(\Pi_1),c(\Pi_2)\leq |Y|$. By induction, there is a
      proof $\Pi'$ from $\pi_2$ and $\Pi_2$ for sequent
      $[[G1; I2; X2; I3; G2 |-l A]]$ s.t. $c(\Pi')\leq |Y|$. Therefore, the
      proof $\Pi$ can be constructed as follows with $c(\Pi)\leq |Y|$.
      \begin{center}
        \scriptsize
        \begin{math}
          $$\mprset{flushleft}
          \inferrule* [right={\tiny impL}] {
            {
              \begin{array}{c}
                \pi_1 \\
                {[[I1 |-c X1]]}
              \end{array}
            }
            $$\mprset{flushleft}
            \inferrule* [right={\tiny cut}] {
              {
                \begin{array}{cc}
                  \pi_2 & \Pi_2 \\
                  {[[I2, X2, I3 |-c Y]]} & {[[G1; Y; G2 |-l A]]}
                \end{array}
              }
            }{[[G1; I2; X2; I3; G2 |-l A]]}
          }{[[G1; I2; X1 -o X2; I1; I3; G2 |-l A]]}
        \end{math}
      \end{center}
\end{itemize}



\subsubsection{Left introduction of the non-commutative left implication $\lto$}
\begin{center}
\scriptsize
  $\Pi_1$:
  \begin{math}
    $$\mprset{flushleft}
    \inferrule* [right={\tiny impL}] {
      {
        \begin{array}{cc}
          \pi_1 & \pi_2 \\
          {[[G1 |-l A1]]} & {[[G2; A2; G3 |-l B]]}
        \end{array}
      }
    }{[[G2; A1 -> A2; G1; G3 |-l B]]}
  \end{math}
  \qquad\qquad
  \begin{math}
    \begin{array}{c}
      \Pi_2 \\
      {[[D1; B; D2 |-l C]]}
    \end{array}
  \end{math}
\end{center}
By assumption, $c(\Pi_1),c(\Pi_2)\leq |B|$. By induction, there is a
proof $\Pi'$ from $\pi_2$ and $\Pi_2$ for sequent
$[[D1; G2; A2; G3; D2 |-l C]]$ s.t. $c(\Pi')\leq |B|$.
Therefore, the proof $\Pi$ can be constructed as follows with
$c(\Pi)\leq |B|$.
\begin{center}
  \scriptsize
  \begin{math}
    $$\mprset{flushleft}
    \inferrule* [right={\tiny impL}] {
      {
        \begin{array}{c}
          \pi_1 \\
          {[[G1 |-l A1 ]]}
        \end{array}
      }
      $$\mprset{flushleft}
      \inferrule* [right={\tiny cut}] {
        {
          \begin{array}{cc}
            \pi_2 & \Pi_2 \\
            {[[G2; A2; G3 |-l B]]} & {[[D1; B; D2 |-l C]]}
          \end{array}
        }
      }{[[D1; G2; A2; G3; D2 |-l C]]}
    }{[[D1; G2; A1 -> A2; G1; G3; D2 |-l C]]}
  \end{math}
\end{center}


\subsubsection{Left introduction of the non-commutative right implication $\rto$}
\begin{center}
  \scriptsize
  $\Pi_1$:
  \begin{math}
    $$\mprset{flushleft}
    \inferrule* [right={\tiny impL}] {
      {
        \begin{array}{cc}
          \pi_1 & \pi_2 \\
          {[[G1 |-l A1]]} & {[[G2; A2; G3 |-l B]]}
        \end{array}
      }
    }{[[G2; G1; A2 <- A1; G3 |-l B]]}
  \end{math}
  \qquad\qquad
  \begin{math}
    \begin{array}{c}
      \Pi_2 \\
      {[[D1; B; D2 |-l C]]}
    \end{array}
  \end{math}
\end{center}
By assumption, $c(\Pi_1),c(\Pi_2)\leq |B|$. By induction, there is a
proof $\Pi'$ from $\pi_2$ and $\Pi_2$ for sequent
$[[D1; G2; A2; G3; D2 |-l C]]$ s.t. $c(\Pi')\leq |B|$. Therefore, the
proof $\Pi$ can be constructed as follows with $c(\Pi)\leq |B|$.
\begin{center}
  \scriptsize
  \begin{math}
    $$\mprset{flushleft}
    \inferrule* [right={\tiny impL}] {
      {
        \begin{array}{c}
          \pi_1 \\
          {[[G1 |-l A1 ]]}
        \end{array}
      }
      $$\mprset{flushleft}
      \inferrule* [right={\tiny cut}] {
        {
          \begin{array}{cc}
            \pi_2 & \Pi_2 \\
            {[[G2; A2; G3 |-l B]]} & {[[D1; B; D2 |-l C]]}
          \end{array}
        }
      }{[[D1; G2; A2; G3; D2 |-l C]]}
    }{[[D1; G2; G1; A2 <- A1; G3; D2 |-l C]]}
  \end{math}
\end{center}

% C-ex Case 1
\subsubsection{$\SCdruleTXXexName$}
\begin{itemize}
\item Case 1:
      \begin{center}
        \scriptsize
        $\Pi_1$:
        \begin{math}
          $$\mprset{flushleft}
          \inferrule* [right={\tiny ex}] {
            {
              \begin{array}{c}
                \pi \\
                {[[I1, X1, X2, I2 |-c Y]]}
              \end{array}
            }
          }{[[I1, X2, X1, I2 |-c Y]]}
        \end{math}
        \qquad\qquad
        \begin{math}
          \begin{array}{c}
            \Pi_2 \\
            {[[P1, Y, P2 |-c Z]]}
          \end{array}
        \end{math}
      \end{center}
      By assumption, $c(\Pi_1),c(\Pi_2)\leq |Y|$. By induction on $\pi$
      and $\Pi_2$, there is a proof $\Pi'$ for sequent
      $[[P1, I1, X1, X2, I2, P2 |-c Z]]$ s.t. $c(\Pi')\leq|Y|$. Therefore,
      the proof $\Pi$ can be constructed as follows, and
      $c(\Pi)=c(\Pi')\leq|Y|$.
      \begin{center}
        \scriptsize
        \begin{math}
          $$\mprset{flushleft}
          \inferrule* [right={\tiny ex}] {
            {
              \begin{array}{c}
                \Pi' \\
                {[[P1, I1, X1, X2, I2, P2 |-c Z]]}
              \end{array}
            }
          }{[[P1, I1, X2, X1, I2, P2 |-c Z]]}
        \end{math}
      \end{center}

% C-ex Case 2
\item Case 2:
      \begin{center}
        \scriptsize
        $\Pi_1$:
        \begin{math}
          $$\mprset{flushleft}
          \inferrule* [right={\tiny beta}] {
            {
              \begin{array}{c}
                \pi \\
                {[[I1, X, Y, I2 |-c Z]]}
              \end{array}
            }
          }{[[I1, Y, X, I2 |-c Z]]}
        \end{math}
        \qquad\qquad
        \begin{math}
          \begin{array}{c}
            \Pi_2 \\
            {[[G1; Z; G2 |-l A]]}
          \end{array}
        \end{math}
      \end{center}
      By assumption, $c(\Pi_1),c(\Pi_2)\leq |Z|$. Similar as above, there
      is a proof $\Pi'$ constructed from $\pi$ and $\Pi_2$ for 
      $[[G1; I1; X; Y; I2; G2 |-l A]]$ s.t. $c(\Pi')\leq|Z|$. Therefore,
      the proof $\Pi$ can be constructed as follows, and
      $c(\Pi)=c(\Pi')\leq|Z|$.
      \begin{center}
        \scriptsize
        \begin{math}
          $$\mprset{flushleft}
          \inferrule* [right={\tiny beta}] {
            {
              \begin{array}{c}
                \Pi' \\
                {[[G1; I1; X; Y; I2; G2 |-l A]]}
              \end{array}
            }
          }{[[G1; I1; Y; X; I2; G2 |-l A]]}
        \end{math}
      \end{center}
\end{itemize}

% LC-ex
\subsubsection{$\SCdruleSXXexName$}
\begin{center}
  \scriptsize
  $\Pi_1$:
  \begin{math}
    $$\mprset{flushleft}
    \inferrule* [right={\tiny beta}] {
      {
        \begin{array}{c}
          \pi \\
          {[[G1; X; Y; G2 |-l A]]}
        \end{array}
      }
    }{[[G1; Y; X; G2 |-l A]]}
  \end{math}
  \qquad\qquad
  \begin{math}
    \begin{array}{c}
      \Pi_2 \\
      {[[D1; A; D2 |-l B]]}
    \end{array}
  \end{math}
\end{center}
By assumption, $c(\Pi_1),c(\Pi_2)\leq |A|$. Similar as above, there
is a proof $\Pi'$ constructed from $\pi$ and $\Pi_2$ for sequent
$[[D1; G1; X; Y; G2; D2 |-l B]]$ s.t. $c(\Pi')\leq|A|$. Therefore,
the proof $\Pi$ can be constructed as follows, and
$c(\Pi)=c(\Pi')\leq|A|$.
\begin{center}
  \scriptsize
  \begin{math}
    $$\mprset{flushleft}
    \inferrule* [right={\tiny beta}] {
      {
        \begin{array}{cc}
          \Pi' \\
          {[[D1; G1; X; Y; G2; D2 |-l B]]}
        \end{array}
      }
    }{[[D1; G1; Y; X; G2; D2 |-l B]]}
  \end{math}
\end{center}





\subsubsection{Left introduction of the commutative tensor product $\otimes$}
\begin{itemize}
\item Case 1:
      \begin{center}
        \scriptsize
        $\Pi_1$:
        \begin{math}
          $$\mprset{flushleft}
          \inferrule* [right={\tiny tenL}] {
            {
              \begin{array}{c}
                \pi \\
                {[[I1, X1, X2, I2 |-c Y]]}
              \end{array}
            }
          }{[[I1, X1 (*) X2, I2 |-c Y]]}
        \end{math}
        \qquad\qquad
        \begin{math}
          \begin{array}{c}
            \Pi_2 \\
            {[[P1, Y, P2 |-c Z]]}
          \end{array}
        \end{math}
      \end{center}
      By assumption, $c(\Pi_1),c(\Pi_2)\leq |Y|$. By induction, there is a
      proof $\Pi'$ from $\pi$ and $\Pi_2$ for sequent
      $[[P1, I1, X1, X2, I2, P2 |-c Z]]$ s.t. $c(\Pi')\leq |Y|$. Therefore,
      the proof $\Pi$ can be constructed as follows with $c(\Pi)\leq |Y|$.
      \begin{center}
        \scriptsize
        \begin{math}
          $$\mprset{flushleft}
          \inferrule* [right={\tiny tenL}] {
            $$\mprset{flushleft}
            \inferrule* [right={\tiny cut}] {
              {
                \begin{array}{cc}
                  \pi & \Pi_2 \\
                  {[[I1, X1, X2, I2 |-c Y]]} & {[[P1, Y, P2 |-c Z]]}
                \end{array}
              }
            }{[[P1, I1, X1, X2, I2, P2 |-c Z]]}
          }{[[P1, I1, X1 (*) X2, I2, P2 |-c Z]]}
        \end{math}
      \end{center}

\item Case 2:
      \begin{center}
        \scriptsize
        $\Pi_1$:
        \begin{math}
          $$\mprset{flushleft}
          \inferrule* [right={\tiny tenL}] {
            {
              \begin{array}{c}
                \pi \\
                {[[I1, X1, X2, I2 |-c Y]]}
              \end{array}
            }
          }{[[I1, X1 (*) X2, I2 |-c Y]]}
        \end{math}
        \qquad\qquad
        \begin{math}
          \begin{array}{c}
            \Pi_2 \\
            {[[G1; Y; G2 |-l A]]}
          \end{array}
        \end{math}
      \end{center}
      By assumption, $c(\Pi_1),c(\Pi_2)\leq |Y|$. By induction, there is a
      proof $\Pi'$ from $\pi$ and $\Pi_2$ for sequent
      $[[G1; I1; X1; X2; I2; G2 |-l A]]$ s.t. $c(\Pi')\leq |Y|$. Therefore,
      the proof $\Pi$ can be constructed as follows with $c(\Pi)\leq |Y|$.
      \begin{center}
        \scriptsize
        \begin{math}
          $$\mprset{flushleft}
          \inferrule* [right={\tiny tenL1}] {
            $$\mprset{flushleft}
            \inferrule* [right={\tiny cut1}] {
              {
                \begin{array}{cc}
                  \pi & \Pi_2 \\
                  {[[I1, X1, X2, I2 |-c Y]]} & {[[G1; Y; G2 |-l A]]}
                \end{array}
              }
            }{[[G1; I1; X1; X2; I2; G2 |-l A]]}
          }{[[G1; I1; X1 (*) X2; I2; G2 |-l A]]}
        \end{math}
      \end{center}

\item Case 3:
      \begin{center}
        \scriptsize
        $\Pi_1$:
        \begin{math}
          $$\mprset{flushleft}
          \inferrule* [right={\tiny tenL}] {
            {
              \begin{array}{c}
                \pi \\
                {[[G1; X; Y; G2 |-l A]]}
              \end{array}
            }
          }{[[G1; X (*) Y; G2 |-l A]]}
        \end{math}
        \qquad\qquad
        \begin{math}
          \begin{array}{c}
            \Pi_2 \\
            {[[D1; A; D2 |-l B]]}
          \end{array}
        \end{math}
      \end{center}
      By assumption, $c(\Pi_1),c(\Pi_2)\leq |A|$. By induction, there is a
      proof $\Pi'$ from $\pi$ and $\Pi_2$ for sequent
      $[[D1; X; Y; G2; D2 |-l B]]$ s.t. $c(\Pi')\leq |A|$. Therefore, the
      proof $\Pi$ can be constructed as follows with $c(\Pi)\leq |A|$.
      \begin{center}
        \scriptsize
        \begin{math}
          $$\mprset{flushleft}
          \inferrule* [right={\tiny tenL1}] {
            $$\mprset{flushleft}
            \inferrule* [right={\tiny cut2}] {
              {
                \begin{array}{cc}
                  \pi & \Pi_2 \\
                  {[[G1; X; Y; G2 |-l A]]} & {[[D1; A; D2 |-l B]]}
                \end{array}
              }
            }{[[D1; G1; X; Y; G2; D2 |-l B]]}
          }{[[D1; G1; X (*) Y; G2; D2 |-l B]]}
        \end{math}
      \end{center}
\end{itemize}

\subsubsection{Left introduction of the non-commutative tensor products $\tri$}
\begin{center}
  \scriptsize
  $\Pi_1$:
  \begin{math}
    $$\mprset{flushleft}
    \inferrule* [right={\tiny tenL2}] {
      {
        \begin{array}{c}
          \pi \\
          {[[G1; A1; A2; G2 |-l B]]}
        \end{array}
      }
    }{[[G1; A1 (>) A2; G2 |-l B]]}
  \end{math}
  \qquad\qquad
  \begin{math}
    \begin{array}{c}
      \Pi_2 \\
      {[[D1; B; D2 |-l C]]}
    \end{array}
  \end{math}
\end{center}
By assumption, $c(\Pi_1),c(\Pi_2)\leq |B|$. By induction, there is a
proof $\Pi'$ from $\pi$ and $\Pi_2$ for sequent \\
$[[D1; G1; A1; A2; G2; D2 |-l C]]$ s.t. $c(\Pi')\leq |B|$.
Therefore, the proof $\Pi$ can be constructed as follows with
$c(\Pi)\leq |B|$.
\begin{center}
  \scriptsize
  \begin{math}
    $$\mprset{flushleft}
    \inferrule* [right={\tiny tenL2}] {
      $$\mprset{flushleft}
      \inferrule* [right={\tiny cut2}] {
        {
          \begin{array}{cc}
            \pi & \Pi_2 \\
            {[[G1; A1; A2; G2 |-l B]]} & {[[D1; B; D2 |-l C]]}
          \end{array}
        }
      }{[[D1; G1; A1; A2; G2; D2 |-l C]]}
    }{[[D1; G1; A1 (>) A2; G2; D2 |-l C]]}
  \end{math}
\end{center}



\subsubsection{Left introduction of the commutative unit $[[UnitT]]$}
\begin{itemize}
\item Case 1:
      \begin{center}
        \scriptsize
        $\Pi_1$:
        \begin{math}
          $$\mprset{flushleft}
          \inferrule* [right={\tiny unitL}] {
            {
              \begin{array}{c}
                \pi \\
                {[[I1, I2 |-c X]]}
              \end{array}
            }
          }{[[I1, UnitT, I2 |-c X]]}
        \end{math}
        \qquad\qquad
        \begin{math}
          \begin{array}{c}
            \Pi_2 \\
            {[[P1, X, P2 |-c Y]]}
          \end{array}
        \end{math}
      \end{center}
      By assumption, $c(\Pi_1),c(\Pi_2)\leq |X|$. By induction, there is a
      proof $\Pi'$ from $\pi$ and $\Pi_2$ for sequent
      $[[P1, I1, I2, P2 |-c Y]]$
      s.t. $c(\Pi')\leq |X|$. Therefore, the proof $\Pi$ can be constructed
      as follows, and $c(\Pi)=c(\Pi')\leq |X|$.
      \begin{center}
        \scriptsize
        \begin{math}
          $$\mprset{flushleft}
          \inferrule* [right={\tiny unitL}] {
            {
              \begin{array}{c}
                \Pi' \\
                {[[P1, I1, I2, P2 |-c Y]]}
              \end{array}
            }
          }{[[P1, I1, UnitT, I2, P2 |-c Y]]}
        \end{math}
      \end{center}

\item Case 2:
      \begin{center}
        \scriptsize
        $\Pi_1$:
        \begin{math}
          $$\mprset{flushleft}
          \inferrule* [right={\tiny unitL}] {
            {
              \begin{array}{c}
                \pi \\
                {[[I1, I2 |-c X]]}
              \end{array}
            }
          }{[[I1, UnitT, I2 |-c X]]}
        \end{math}
        \qquad\qquad
        \begin{math}
          \begin{array}{c}
            \Pi_2 \\
            {[[G1; X; G2 |-l A]]}
          \end{array}
        \end{math}
      \end{center}
      By assumption, $c(\Pi_1),c(\Pi_2)\leq |X|$. By induction, there is a
      proof $\Pi'$ from $\pi$ and $\Pi_2$ for sequent
      $[[G1; I1; I2; G2 |-l A]]$
      s.t. $c(\Pi')\leq |X|$. Therefore, the proof $\Pi$ can be constructed
      as follows, and $c(\Pi)=c(\Pi')\leq |X|$.
      \begin{center}
        \scriptsize
        \begin{math}
          $$\mprset{flushleft}
          \inferrule* [right={\tiny unitL}] {
            {
              \begin{array}{c}
                \Pi' \\
                {[[G1; I1; I2; G2 |-l A]]}
              \end{array}
            }
          }{[[G1; I1; UnitT; I2; G2 |-l A]]}
        \end{math}
      \end{center}

\item Case 3:
      \begin{center}
        \scriptsize
        $\Pi_1$:
        \begin{math}
          $$\mprset{flushleft}
          \inferrule* [right={\tiny unitL}] {
            {
              \begin{array}{c}
                \pi \\
                {[[D1; D2 |-l A]]}
              \end{array}
            }
          }{[[D1; UnitT; D2 |-l A]]}
        \end{math}
        \qquad\qquad
        \begin{math}
          \begin{array}{c}
            \Pi_2 \\
            {[[G1; A; G2 |-l B]]}
          \end{array}
        \end{math}
      \end{center}
      By assumption, $c(\Pi_1),c(\Pi_2)\leq |X|$. By induction, there is a
      proof $\Pi'$ from $\pi$ and $\Pi_2$ for sequent
      $[[G1; D1; D2; G2 |-l B]]$
      s.t. $c(\Pi')\leq |X|$. Therefore, the proof $\Pi$ can be constructed
      as follows, and $c(\Pi)=c(\Pi')\leq |X|$.
      \begin{center}
        \scriptsize
        \begin{math}
          $$\mprset{flushleft}
          \inferrule* [right={\tiny unitL}] {
            {
              \begin{array}{c}
                \Pi' \\
                {[[G1; D1; D2; G2 |-l B]]}
              \end{array}
            }
          }{[[G1; D1; UnitT; D2; G2 |-l B]]}
        \end{math}
      \end{center}
\end{itemize}



\subsubsection{Left introduction of the non-commutative unit $[[UnitS]]$}
\begin{center}
  \scriptsize
  $\Pi_1$:
  \begin{math}
    $$\mprset{flushleft}
    \inferrule* [right={\tiny unitL}] {
      {
        \begin{array}{c}
          \pi \\
          {[[D1; D2 |-l A]]}
        \end{array}
      }
    }{[[D1; UnitS; D2 |-l A]]}
  \end{math}
  \qquad\qquad
  \begin{math}
    \begin{array}{c}
      \Pi_2 \\
      {[[G1; A; G2 |-l B]]}
    \end{array}
  \end{math}
\end{center}
By assumption, $c(\Pi_1),c(\Pi_2)\leq |X|$. By induction, there is a
proof $\Pi'$ from $\pi$ and $\Pi_2$ for sequent \\
$[[G1; D1; D2; G2 |-l B]]$ s.t. $c(\Pi')\leq |X|$. Therefore, the proof
$\Pi$ can be constructed as follows, and \\
$c(\Pi)=c(\Pi')\leq |X|$.
\begin{center}
  \scriptsize
  \begin{math}
    $$\mprset{flushleft}
    \inferrule* [right={\tiny unitL}] {
      {
        \begin{array}{c}
          \Pi' \\
          {[[G1; D1; D2; G2 |-l B]]}
        \end{array}
      }
    }{[[G1; D1; UnitS; D2; G2 |-l B]]}
  \end{math}
\end{center}



\subsubsection{Left introduction of the functor $F$}
\begin{center}
  \scriptsize
  $\Pi_1$:
  \begin{math}
    $$\mprset{flushleft}
    \inferrule* [right={\tiny FL}] {
      {
        \begin{array}{c}
          \pi_1 \\
          {[[G1; X; G2 |-l A]]}
        \end{array}
      }
    }{[[G1; F X; G2 |-l A]]}
  \end{math}
  \qquad\qquad
  \begin{math}
    \begin{array}{c}
      \Pi_2 \\
      {[[D1; A; D2 |-l B]]}
    \end{array}
  \end{math}
\end{center}
By assumption, $c(\Pi_1),c(\Pi_2)\leq |A|$. By induction, there is a
proof $\Pi'$ from $\pi_2$ and $\Pi_2$ for sequent
$[[D1; G1; X; G2; D2 |-l B]]$ s.t. $c(\Pi')\leq |A|$. Therefore, the
proof $\Pi$ can be constructed as follows with $c(\Pi)\leq |A|$.
\begin{center}
  \scriptsize
  \begin{math}
    $$\mprset{flushleft}
    \inferrule* [right={\tiny FL}] {
      $$\mprset{flushleft}
      \inferrule* [right={\tiny cut2}] {
        {
          \begin{array}{cc}
            \pi_2 & \Pi_2 \\
            {[[G1; X; G2 |-l A]]} & {[[D1; A; D2 |-l B]]}
          \end{array}
        }
      }{[[D1; G1; X; G2; D2 |-l B]]}
    }{[[D1; G1; F X; G2; D2 |-l B]]}
  \end{math}
\end{center}

\subsubsection{Left introduction of the functor $G$}
\begin{center}
  \scriptsize
  $\Pi_1$:
  \begin{math}
    $$\mprset{flushleft}
    \inferrule* [right={\tiny GL}] {
      {
        \begin{array}{c}
          \pi_1 \\
          {[[G1; A; G2 |-l B]]}
        \end{array}
      }
    }{[[G1; Gf A; G2 |-l B]]}
  \end{math}
  \qquad\qquad
  \begin{math}
    \begin{array}{c}
      \Pi_2 \\
      {[[D1; B; D2 |-l C]]}
    \end{array}
  \end{math}
\end{center}
By assumption, $c(\Pi_1),c(\Pi_2)\leq |B|$. By induction, there is a
proof $\Pi'$ from $\pi_2$ and $\Pi_2$ for sequent
$[[D1; G1; A; G2; D2 |-l C]]$ s.t. $c(\Pi')\leq |B|$. Therefore, the
proof $\Pi$ can be constructed as follows with $c(\Pi)\leq |B|$.
\begin{center}
  \scriptsize
  \begin{math}
    $$\mprset{flushleft}
    \inferrule* [right={\tiny GL}] {
      $$\mprset{flushleft}
      \inferrule* [right={\tiny cut2}] {
        {
          \begin{array}{cc}
            \pi_2 & \Pi_2 \\
            {[[G1; A; G2 |-l B]]} & {[[D1; B; D2 |-l C]]}
          \end{array}
        }
      }{[[D1; G1; A; G2; D2 |-l C]]}
    }{[[D1; G1; Gf A; G2; D2 |-l C]]}
  \end{math}
\end{center}



\subsection{Secondary Hypothesis}

\subsubsection{Right introduction of the commutative tensor product $\otimes$}
\begin{itemize}
\item Case 1:
      \begin{center}
        \scriptsize
        \begin{math}
          \begin{array}{c}
            \Pi_1 \\
            {[[I2 |-c X]]}
          \end{array}
        \end{math}
        \qquad\qquad
        $\Pi_2$:
        \begin{math}
          $$\mprset{flushleft}
          \inferrule* [right={\tiny tenR}] {
            {
              \begin{array}{cc}
                \pi_1 & \pi_2 \\
                {[[P1, X, P2 |-c Y1]]} & {[[I1 |-c Y2]]}
              \end{array}
            }
          }{[[P1, X, P2, I1 |-c Y1 (*) Y2]]}
        \end{math}
      \end{center}
      By assumption, $c(\Pi_1),c(\Pi_2)\leq |X|$. By induction on $\Pi_1$
      and $\pi_1$, there is a proof $\Pi'$ for sequent
      $[[P1, I2, P2 |-c Y1]]$ s.t. $c(\Pi') \leq |X|$. Therefore, the proof
      $\Pi$ can be constructed as follows with $c(\Pi) = c(\Pi') \leq |X|$.
      \begin{center}
        \scriptsize
        \begin{math}
          $$\mprset{flushleft}
          \inferrule* [right={\tiny tenR}] {
            {
              \begin{array}{cc}
                \Pi' & \pi_1 \\
                {[[P1, I2, P2 |-c Y1]]} & {[[I1 |-c Y2]]}
              \end{array}
            }
          }{[[P1, I2, P2, I1 |-c Y1 (*) Y2]]}
        \end{math}
      \end{center}

\item Case 2:
      \begin{center}
        \scriptsize
        \begin{math}
          \begin{array}{c}
            \Pi_1 \\
            {[[I2 |-c X]]}
          \end{array}
        \end{math}
        \qquad\qquad
        $\Pi_2$:
        \begin{math}
          $$\mprset{flushleft}
          \inferrule* [right={\tiny tenR}] {
            {
              \begin{array}{cc}
                \pi_1 & \pi_2 \\
                {[[I1 |-c Y1]]} & {[[P1, X, P2 |-c Y2]]}
              \end{array}
            }
          }{[[I1, P1, X, P2 |-c Y1 (*) Y2]]}
        \end{math}
      \end{center}
      By assumption, $c(\Pi_1),c(\Pi_2)\leq |X|$. By induction on $\Pi_1$
      and $\pi_2$, there is a proof $\Pi'$ for sequent
      $[[P1, I2, P2 |-c Y2]]$ s.t. $c(\Pi') \leq |X|$. Therefore, the proof
      $\Pi$ can be constructed as follows with $c(\Pi) = c(\Pi') \leq |X|$.
      \begin{center}
        \scriptsize
        \begin{math}
          $$\mprset{flushleft}
          \inferrule* [right={\tiny tenR}] {
            {
              \begin{array}{cc}
                \pi_1 & \Pi' \\
                {[[I1 |-c Y1]]} & {[[P1, I2, P2 |-c Y2]]}
              \end{array}
            }
          }{[[I1, P1, I2, P2 |-c Y1 (*) Y2]]}
        \end{math}
      \end{center}
\end{itemize}



\subsubsection{Right introduction of the non-commutative tensor product $\tri$}
\begin{itemize}
\item Case 1:
      \begin{center}
        \scriptsize
        \begin{math}
          \begin{array}{c}
            \Pi_1 \\
            {[[I |-c X]]}
          \end{array}
        \end{math}
        \qquad\qquad
        $\Pi_2$:
        \begin{math}
          $$\mprset{flushleft}
          \inferrule* [right={\tiny tenR}] {
            {
              \begin{array}{cc}
                \pi_1 & \pi_2 \\
                {[[G1; X; G2 |-l A]]} & {[[G3 |-l B]]}
              \end{array}
            }
          }{[[G1; X; G2; G3 |-l A (>) B]]}
        \end{math}
      \end{center}
      By assumption, $c(\Pi_1),c(\Pi_2)\leq |X|$. By induction on $\Pi_1$
      and $\pi_1$, there is a proof $\Pi'$ for sequent
      $[[G1; I; G2 |-l A]]$ s.t. $c(\Pi') \leq |X|$. Therefore, the proof
      $\Pi$ can be constructed as follows with $c(\Pi) = c(\Pi') \leq |X|$.
      \begin{center}
        \scriptsize
        \begin{math}
          $$\mprset{flushleft}
          \inferrule* [right={\tiny tenR}] {
            {
              \begin{array}{cc}
                \Pi' & \pi_1 \\
                {[[G1; I; G2 |-l A]]} & {[[G3 |-l B]]}
              \end{array}
            }
          }{[[G1; I; G2; G3 |-l A (>) B]]}
        \end{math}
      \end{center}

\item Case 2:
      \begin{center}
        \scriptsize
        \begin{math}
          \begin{array}{c}
            \Pi_1 \\
            {[[D |-l C]]}
          \end{array}
        \end{math}
        \qquad\qquad
        $\Pi_2$:
        \begin{math}
          $$\mprset{flushleft}
          \inferrule* [right={\tiny tenR}] {
            {
              \begin{array}{cc}
                \pi_1 & \pi_2 \\
                {[[G1; C; G2 |-l A]]} & {[[G3 |-l B]]}
              \end{array}
            }
          }{[[G1; C; G2; G3 |-l A (>) B]]}
        \end{math}
      \end{center}
      By assumption, $c(\Pi_1),c(\Pi_2)\leq |C|$. By induction on $\Pi_1$
      and $\pi_1$, there is a proof $\Pi'$ for sequent
      $[[G1; D; G2 |-l A]]$ s.t. $c(\Pi') \leq |C|$. Therefore, the proof
      $\Pi$ can be constructed as follows with $c(\Pi) = c(\Pi') \leq |C|$.
      \begin{center}
        \scriptsize
        \begin{math}
          $$\mprset{flushleft}
          \inferrule* [right={\tiny tenR}] {
            {
              \begin{array}{cc}
                \Pi' & \pi_1 \\
                {[[G1; D; G2 |-l A]]} & {[[G3 |-l B]]}
              \end{array}
            }
          }{[[G1; D; G2; G3 |-l A (>) B]]}
        \end{math}
      \end{center}

\item Case 3:
      \begin{center}
        \scriptsize
        \begin{math}
          \begin{array}{c}
            \Pi_1 \\
            {[[I |-c X]]}
          \end{array}
        \end{math}
        \qquad\qquad
        $\Pi_2$:
        \begin{math}
          $$\mprset{flushleft}
          \inferrule* [right={\tiny tenR}] {
            {
              \begin{array}{cc}
                \pi_1 & \pi_2 \\
                {[[G1 |-l A]]} & {[[G2; X; G3 |-l B]]}
              \end{array}
            }
          }{[[G1; G2; X; G3 |-l A (>) B]]}
        \end{math}
      \end{center}
      By assumption, $c(\Pi_1),c(\Pi_2)\leq |X|$. By induction on $\Pi_1$
      and $\pi_2$, there is a proof $\Pi'$ for sequent
      $[[G2; I; G3 |-l B]]$ s.t. $c(\Pi') \leq |X|$. Therefore, the proof
      $\Pi$ can be constructed as follows with $c(\Pi) = c(\Pi') \leq |X|$.
      \begin{center}
        \scriptsize
        \begin{math}
          $$\mprset{flushleft}
          \inferrule* [right={\tiny tenR}] {
            {
              \begin{array}{cc}
                \pi_1 & \Pi' \\
                {[[G1 |-l A]]} & {[[G2; I; G3 |-l B]]}
              \end{array}
            }
          }{[[G1; G2; I; G3 |-l A (>) B]]}
        \end{math}
      \end{center}

\item Case 4:
      \begin{center}
        \scriptsize
        \begin{math}
          \begin{array}{c}
            \Pi_1 \\
            {[[D |-l C]]}
          \end{array}
        \end{math}
        \qquad\qquad
        $\Pi_2$:
        \begin{math}
          $$\mprset{flushleft}
          \inferrule* [right={\tiny tenR}] {
            {
              \begin{array}{cc}
                \pi_1 & \pi_2 \\
                {[[G1 |-l A]]} & {[[G2; C; G3 |-l B]]}
              \end{array}
            }
          }{[[G1; G2; C; G3 |-l A (>) B]]}
        \end{math}
      \end{center}
      By assumption, $c(\Pi_1),c(\Pi_2)\leq |C|$. By induction on $\Pi_1$
      and $\pi_2$, there is a proof $\Pi'$ for sequent
      $[[G2; D; G3 |-l B]]$ s.t. $c(\Pi') \leq |C|$. Therefore, the proof
      $\Pi$ can be constructed as follows with $c(\Pi) = c(\Pi') \leq |C|$.
      \begin{center}
        \scriptsize
        \begin{math}
          $$\mprset{flushleft}
          \inferrule* [right={\tiny tenR}] {
            {
              \begin{array}{cc}
                \pi_1 & \Pi' \\
                {[[G1 |-l A]]} & {[[G2; D; G3 |-l B]]}
              \end{array}
            }
          }{[[G1; G2; D; G3 |-l A (>) B]]}
        \end{math}
      \end{center}
\end{itemize}



\subsubsection{Left introduction of the commutative implication $\multimap$}
\begin{itemize}
\item Case 1:
      \begin{center}
        \scriptsize
        \begin{math}
          \begin{array}{c}
            \Pi_1 \\
            {[[I |-c X]]}
          \end{array}
        \end{math}
        \qquad\qquad
        $\Pi_2$:
        \begin{math}
          $$\mprset{flushleft}
          \inferrule* [right={\tiny impL}] {
            {
              \begin{array}{cc}
                \pi_1 & \pi_2 \\
                {[[P2, X, P3 |-c Y1]]} & {[[P1, Y2, P4 |-c Z]]}
              \end{array}
            }
          }{[[P1, Y1 -o Y2, P2, X, P3, P4 |-c Z]]}
        \end{math}
      \end{center}
      By assumption, $c(\Pi_1),c(\Pi_2)\leq |X|$. By induction on $\Pi_1$ and $\pi_1$, there is
      a proof $\Pi'$ for sequent $[[P2, I, P3 |-c Y1]]$ s.t. $c(\Pi') \leq |X|$. Therefore, the
      proof $\Pi$ can be constructed as follows with $c(\Pi) = c(\Pi') \leq |X|$.
      \begin{center}
        \scriptsize
        \begin{math}
          $$\mprset{flushleft}
          \inferrule* [right={\tiny impL}] {
            {
              \begin{array}{cc}
                \Pi' & \pi_2 \\
                {[[P2, I, P3 |-c Y1]]} & {[[P1, Y2, P4 |-c Z]]}
              \end{array}
            }
          }{[[P1, Y1 -o Y2, P2, I, P3, P4 |-c Z]]}
        \end{math}
      \end{center}

\item Case 2:
      \begin{center}
        \scriptsize
        \begin{math}
          \begin{array}{c}
            \Pi_1 \\
            {[[I |-c X]]}
          \end{array}
        \end{math}
        \qquad\qquad
        $\Pi_2$:
        \begin{math}
          $$\mprset{flushleft}
          \inferrule* [right={\tiny impL}] {
            {
              \begin{array}{cc}
                \pi_1 & \pi_2 \\
                {[[P3 |-c Y1]]} & {[[P1, X, P2, Y2, P4 |-c Z]]}
              \end{array}
            }
          }{[[P1, X, P2, Y1 -o Y2, P3, P4 |-c Z]]}
        \end{math}
      \end{center}
      By assumption, $c(\Pi_1),c(\Pi_2)\leq |X|$. By induction on $\Pi_1$ and $\pi_2$, there is
      a proof $\Pi'$ for sequent $[[P1, I, P2, Y2, P4 |-c Z]]$ s.t. $c(\Pi') \leq |X|$.
      Therefore, the proof $\Pi$ can be constructed as follows with
      $c(\Pi) = c(\Pi') \leq |X|$.
      \begin{center}
        \scriptsize
        \begin{math}
          $$\mprset{flushleft}
          \inferrule* [right={\tiny impL}] {
            {
              \begin{array}{cc}
                \pi_1 & \Pi' \\
                {[[P3 |-c Y1]]} & {[[P1, I, P2, Y2, P4 |-c Z]]}
              \end{array}
            }
          }{[[P1, I1, P2, Y1 -o Y2, P3, P4 |-c Z]]}
        \end{math}
      \end{center}

\item Case 3:
      \begin{center}
        \scriptsize
        \begin{math}
          \begin{array}{c}
            \Pi_1 \\
            {[[I |-c X]]}
          \end{array}
        \end{math}
        \qquad\qquad
        $\Pi_2$:
        \begin{math}
          $$\mprset{flushleft}
          \inferrule* [right={\tiny impL}] {
            {
              \begin{array}{cc}
                \pi_1 & \pi_2 \\
                {[[P2 |-c Y1]]} & {[[P1, Y2, P3, X, P4 |-c Z]]}
              \end{array}
            }
          }{[[P1, Y1 -o Y2, P2, P3, X, P4 |-c Z]]}
        \end{math}
      \end{center}
      By assumption, $c(\Pi_1),c(\Pi_2)\leq |X|$. By induction on $\Pi_1$
      and $\pi_2$, there is a proof $\Pi'$ for sequent
      $[[P1, I, P2, Y2, P4 |-c Z]]$ s.t. $c(\Pi') \leq |X|$. Therefore,
      the proof $\Pi$ can be constructed as follows with
      $c(\Pi) = c(\Pi') \leq |X|$.
      \begin{center}
        \scriptsize
        \begin{math}
          $$\mprset{flushleft}
          \inferrule* [right={\tiny impL}] {
            {
              \begin{array}{cc}
                \pi_1 & \Pi' \\
                {[[P2 |-c Y1]]} & {[[P1, Y2, P3, I, P4 |-c Z]]}
              \end{array}
            }
          }{[[P1, Y1 -o Y2, P2, P3, I, P4 |-c Z]]}
        \end{math}
      \end{center}

\item Case 4:
      \begin{center}
        \scriptsize
        \begin{math}
          \begin{array}{c}
            \Pi_1 \\
            {[[I |-c X]]}
          \end{array}
        \end{math}
        \qquad\qquad
        $\Pi_2$:
        \begin{math}
          $$\mprset{flushleft}
          \inferrule* [right={\tiny impL}] {
            {
              \begin{array}{cc}
                \pi_1 & \pi_2 \\
                {[[P1, X, P2 |-c Y1]]} & {[[G1; Y2; G2 |-l A]]}
              \end{array}
            }
          }{[[G1; Y1 -o Y2; P1; X; P2; G2 |-l A]]}
        \end{math}
      \end{center}
      By assumption, $c(\Pi_1),c(\Pi_2)\leq |X|$. By induction on $\Pi_1$
      and $\pi_1$, there is a proof $\Pi'$ for sequent
      $[[P1, I, P2 |-c Y1]]$ s.t. $c(\Pi') \leq |X|$. Therefore, the proof
      $\Pi$ can be constructed as follows with $c(\Pi) = c(\Pi') \leq |X|$.
      \begin{center}
        \scriptsize
        \begin{math}
          $$\mprset{flushleft}
          \inferrule* [right={\tiny impL}] {
            {
              \begin{array}{cc}
                \Pi' & \pi_2 \\
                {[[P1, I, P2 |-c Y1]]} & {[[G1; Y2; G2 |-l A]]}
              \end{array}
            }
          }{[[G1; Y1 -o Y2; P1; I; P2; G2 |-l A]]}
        \end{math}
      \end{center}

\item Case 5:
      \begin{center}
        \scriptsize
        \begin{math}
          \begin{array}{c}
            \Pi_1 \\
            {[[I |-c X]]}
          \end{array}
        \end{math}
        \qquad\qquad
        $\Pi_2$:
        \begin{math}
          $$\mprset{flushleft}
          \inferrule* [right={\tiny impL}] {
            {
              \begin{array}{cc}
                \pi_1 & \pi_2 \\
                {[[P |-c Y1]]} & {[[G1; X; G2; Y2; G3 |-l A]]}
              \end{array}
            }
          }{[[G1; X; G2; Y1 -o Y2; P; G3 |-l A]]}
        \end{math}
      \end{center}
      By assumption, $c(\Pi_1),c(\Pi_2)\leq |X|$. By induction on $\Pi_1$
      and $\pi_2$, there is a proof $\Pi'$ for sequent
      $[[G1; I; G2; Y2; G3 |-l A]]$ s.t. $c(\Pi') \leq |X|$. Therefore, the
      proof $\Pi$ can be constructed as follows with
      $c(\Pi) = c(\Pi') \leq |X|$.
      \begin{center}
        \scriptsize
        \begin{math}
          $$\mprset{flushleft}
          \inferrule* [right={\tiny impL}] {
            {
              \begin{array}{cc}
                \pi_1 & \Pi' \\
                {[[P |-c Y1]]} & {[[G1; I; G2; Y2; G3 |-l A]]}
              \end{array}
            }
          }{[[G1; I; G2; Y1 -o Y2; P; G3 |-l A]]}
        \end{math}
      \end{center}

\item Case 6:
      \begin{center}
        \scriptsize
        \begin{math}
          \begin{array}{c}
            \Pi_1 \\
            {[[D |-l B]]}
          \end{array}
        \end{math}
        \qquad\qquad
        $\Pi_2$:
        \begin{math}
          $$\mprset{flushleft}
          \inferrule* [right={\tiny impL}] {
            {
              \begin{array}{cc}
                \pi_1 & \pi_2 \\
                {[[P |-c Y1]]} & {[[G1; B; G2; Y2; G3 |-l A]]}
              \end{array}
            }
          }{[[G1; B; G2; Y1 -o Y2; P; G3 |-l A]]}
        \end{math}
      \end{center}
      By assumption, $c(\Pi_1),c(\Pi_2)\leq |B|$. By induction on $\Pi_1$
      and $\pi_2$, there is a proof $\Pi'$ for sequent
      $[[G1; D; G2; Y2; G3 |-l A]]$ s.t. $c(\Pi') \leq |B|$. Therefore, the
      proof $\Pi$ can be constructed as follows with
      $c(\Pi) = c(\Pi') \leq |B|$.
      \begin{center}
        \scriptsize
        \begin{math}
          $$\mprset{flushleft}
          \inferrule* [right={\tiny impL}] {
            {
              \begin{array}{cc}
                \pi_1 & \Pi' \\
                {[[P |-c Y1]]} & {[[G1; D; G2; Y2; G3 |-l A]]}
              \end{array}
            }
          }{[[G1; D; G2; Y1 -o Y2; P; G3 |-l A]]}
        \end{math}
      \end{center}

\item Case 7:
      \begin{center}
        \scriptsize
        \begin{math}
          \begin{array}{c}
            \Pi_1 \\
            {[[I |-c X]]}
          \end{array}
        \end{math}
        \qquad\qquad
        $\Pi_2$:
        \begin{math}
          $$\mprset{flushleft}
          \inferrule* [right={\tiny impL}] {
            {
              \begin{array}{cc}
                \pi_1 & \pi_2 \\
                {[[P |-c Y1]]} & {[[G1; Y2; G2; X; G3 |-l A]]}
              \end{array}
            }
          }{[[G1; Y1 -o Y2; P; G2; X; G3 |-l A]]}
        \end{math}
      \end{center}
      By assumption, $c(\Pi_1),c(\Pi_2)\leq |X|$. By induction on $\Pi_1$
      and $\pi_2$, there is a proof $\Pi'$ for sequent
      $[[G1; Y2; G2; I; G3 |-l A]]$ s.t. $c(\Pi') \leq |X|$. Therefore, the
      proof $\Pi$ can be constructed as follows with
      $c(\Pi) = c(\Pi') \leq |X|$.
      \begin{center}
        \scriptsize
        \begin{math}
          $$\mprset{flushleft}
          \inferrule* [right={\tiny impL}] {
            {
              \begin{array}{cc}
                \pi_1 & \Pi' \\
                {[[P |-c Y1]]} & {[[G1; Y2; G2; I; G3 |-l A]]}
              \end{array}
            }
          }{[[G1; Y1 -o Y2; P; G2; I; G3 |-l A]]}
        \end{math}
      \end{center}

\item Case 8:
      \begin{center}
        \scriptsize
        \begin{math}
          \begin{array}{c}
            \Pi_1 \\
            {[[D |-l B]]}
          \end{array}
        \end{math}
        \qquad\qquad
        $\Pi_2$:
        \begin{math}
          $$\mprset{flushleft}
          \inferrule* [right={\tiny impL}] {
            {
              \begin{array}{cc}
                \pi_1 & \pi_2 \\
                {[[P |-c Y1]]} & {[[G1; Y2; G2; B; G3 |-l A]]}
              \end{array}
            }
          }{[[G1; Y1 -o Y2; P; G2; B; G3 |-l A]]}
        \end{math}
      \end{center}
      By assumption, $c(\Pi_1),c(\Pi_2)\leq |B|$. By induction on $\Pi_1$
      and $\pi_2$, there is a proof $\Pi'$ for sequent
      $[[G1; Y2; G2; D; G3 |-l A]]$ s.t. $c(\Pi') \leq |B|$. Therefore,
      the proof $\Pi$ can be constructed as follows with
      $c(\Pi) = c(\Pi') \leq |B|$.
      \begin{center}
        \scriptsize
        \begin{math}
          $$\mprset{flushleft}
          \inferrule* [right={\tiny impL}] {
            {
              \begin{array}{cc}
                \pi_1 & \Pi' \\
                {[[P |-c Y1]]} & {[[G1; Y2; G2; D; G3 |-l A]]}
              \end{array}
            }
          }{[[G1; Y1 -o Y2; P; G2; D; G3 |-l A]]}
        \end{math}
      \end{center}
\end{itemize}



\subsubsection{Left introduction of the non-commutative left implication $\lto$}
\begin{itemize}
\item Case 1:
      \begin{center}
        \scriptsize
        \begin{math}
          \begin{array}{c}
            \Pi_1 \\
            {[[I |-c X]]}
          \end{array}
        \end{math}
        \qquad\qquad
        $\Pi_2$:
        \begin{math}
          $$\mprset{flushleft}
          \inferrule* [right={\tiny imprL}] {
            {
              \begin{array}{cc}
                \pi_1 & \pi_2 \\
                {[[D1; X; D2 |-l A1]]} & {[[G1; A2; G2 |-l B]]}
              \end{array}
            }
          }{[[G1; A1 -> A2; D1; X; D2; G2 |-l B]]}
        \end{math}
      \end{center}
      By assumption, $c(\Pi_1),c(\Pi_2)\leq |X|$. By induction on $\Pi_1$
      and $\pi_1$, there is a proof $\Pi'$ for sequent
      $[[D1; I; D2 |-l A1]]$ s.t. $c(\Pi') \leq |X|$. Therefore, the proof
      $\Pi$ can be constructed as follows with $c(\Pi) = c(\Pi') \leq |X|$.
      \begin{center}
        \scriptsize
        \begin{math}
          $$\mprset{flushleft}
          \inferrule* [right={\tiny impL}] {
            {
              \begin{array}{cc}
                \Pi' & \pi_2 \\
                {[[D1; I; D2 |-l A1]]} & {[[G1; A2; G2 |-l B]]}
              \end{array}
            }
          }{[[G1; A1 -> A2; D1; I; D2; G2 |-l B]]}
        \end{math}
      \end{center}

\item Case 2:
      \begin{center}
        \scriptsize
        \begin{math}
          \begin{array}{c}
            \Pi_1 \\
            {[[G |-l C]]}
          \end{array}
        \end{math}
        \qquad\qquad
        $\Pi_2$:
        \begin{math}
          $$\mprset{flushleft}
          \inferrule* [right={\tiny imprL}] {
            {
              \begin{array}{cc}
                \pi_1 & \pi_2 \\
                {[[D1; C; D2 |-l A1]]} & {[[G1; A2; G2 |-l B]]}
              \end{array}
            }
          }{[[G1; A1 -> A2; D1; C; D2; G2 |-l B]]}
        \end{math}
      \end{center}
      By assumption, $c(\Pi_1),c(\Pi_2)\leq |C|$. By induction on $\Pi_1$
      and $\pi_1$, there is a proof $\Pi'$ for sequent
      $[[D1; G; D2 |-l A1]]$ s.t. $c(\Pi') \leq |C|$. Therefore, the proof
      $\Pi$ can be constructed as follows with $c(\Pi) = c(\Pi') \leq |C|$.
      \begin{center}
        \scriptsize
        \begin{math}
          $$\mprset{flushleft}
          \inferrule* [right={\tiny imprL}] {
            {
              \begin{array}{cc}
                \Pi' & \pi_2 \\
                {[[D1; G; D2 |-l A1]]} & {[[G1; A2; G2 |-l B]]}
              \end{array}
            }
          }{[[G1; A1 -> A2; D1; G; D2; G2 |-l B]]}
        \end{math}
      \end{center}

\item Case 3:
      \begin{center}
        \scriptsize
        \begin{math}
          \begin{array}{c}
            \Pi_1 \\
            {[[I |-c X]]}
          \end{array}
        \end{math}
        \qquad\qquad
        $\Pi_2$:
        \begin{math}
          $$\mprset{flushleft}
          \inferrule* [right={\tiny imprL}] {
            {
              \begin{array}{cc}
                \pi_1 & \pi_2 \\
                {[[D |-l A1]]} & {[[G1; X; G2; A2; G3 |-l B]]}
              \end{array}
            }
          }{[[G1; X; G2; A1 -> A2; D; G3 |-l B]]}
        \end{math}
      \end{center}
      By assumption, $c(\Pi_1),c(\Pi_2)\leq |X|$. By induction on $\Pi_1$
      and $\pi_2$, there is a proof $\Pi'$ for sequent
      $[[G1; I; G2; A2; G3 |-l B]]$ s.t. $c(\Pi') \leq |X|$. Therefore,
      the proof $\Pi$ can be constructed as follows with
      $c(\Pi) = c(\Pi') \leq |X|$.
      \begin{center}
        \scriptsize
        \begin{math}
          $$\mprset{flushleft}
          \inferrule* [right={\tiny imprL}] {
            {
              \begin{array}{cc}
                \pi_1 & \Pi' \\
                {[[D |-l A1]]} & {[[G1; I; G2; A2; G3 |-l B]]}
              \end{array}
            }
          }{[[G1; I; G2; A1 -> A2; D; G3 |-l B]]}
        \end{math}
      \end{center}

\item Case 4:
      \begin{center}
        \scriptsize
        \begin{math}
          \begin{array}{c}
            \Pi_1 \\
            {[[D1 |-l B]]}
          \end{array}
        \end{math}
        \qquad\qquad
        $\Pi_2$:
        \begin{math}
          $$\mprset{flushleft}
          \inferrule* [right={\tiny imprL}] {
            {
              \begin{array}{cc}
                \pi_1 & \pi_2 \\
                {[[D2 |-l A1]]} & {[[G1; B; G2; A2; G3 |-l C]]}
              \end{array}
            }
          }{[[G1; B; G2; A1 -> A2; D2; G3 |-l C]]}
        \end{math}
      \end{center}
      By assumption, $c(\Pi_1),c(\Pi_2)\leq |B|$. By induction on $\Pi_1$
      and $\pi_2$, there is a proof $\Pi'$ for sequent
      $[[G1; D1; G2; A2; G3 |-l C]]$ s.t. $c(\Pi') \leq |B|$. Therefore,
      the proof $\Pi$ can be constructed as follows with
      $c(\Pi) = c(\Pi') \leq |B|$.
      \begin{center}
        \scriptsize
        \begin{math}
          $$\mprset{flushleft}
          \inferrule* [right={\tiny imprL}] {
            {
              \begin{array}{cc}
                \pi_1 & \Pi' \\
                {[[D2 |-l A1]]} & {[[G1; D1; G2; A2; G3 |-l C]]}
              \end{array}
            }
          }{[[G1; D1; G2; A1 -> A2; D2; G3 |-l C]]}
        \end{math}
      \end{center}

\item Case 5:
      \begin{center}
        \scriptsize
        \begin{math}
          \begin{array}{c}
            \Pi_1 \\
            {[[I |-c X]]}
          \end{array}
        \end{math}
        \qquad\qquad
        $\Pi_2$:
        \begin{math}
          $$\mprset{flushleft}
          \inferrule* [right={\tiny imprL}] {
            {
              \begin{array}{cc}
                \pi_1 & \pi_2 \\
                {[[D |-l A1]]} & {[[G1; A2; G2; X; G3 |-l B]]}
              \end{array}
            }
          }{[[G1; A1 -> A2; D; G2; X; G3 |-l B]]}
        \end{math}
      \end{center}
      By assumption, $c(\Pi_1),c(\Pi_2)\leq |X|$. By induction on $\Pi_1$
      and $\pi_2$, there is a proof $\Pi'$ for sequent
      $[[G1; A2; G2; I; G3 |-l B]]$ s.t. $c(\Pi') \leq |X|$. Therefore, the
      proof $\Pi$ can be constructed as follows with
      $c(\Pi) = c(\Pi') \leq |X|$.
      \begin{center}
        \scriptsize
        \begin{math}
          $$\mprset{flushleft}
          \inferrule* [right={\tiny imprL}] {
            {
              \begin{array}{cc}
                \pi_1 & \Pi' \\
                {[[D |-l A1]]} & {[[G1; A2; G2; I; G3 |-l B]]}
              \end{array}
            }
          }{[[G1; A1 -> A2; D; G2; I; G3 |-l B]]}
        \end{math}
      \end{center}

\item Case 6:
      \begin{center}
        \scriptsize
        \begin{math}
          \begin{array}{c}
            \Pi_1 \\
            {[[D1 |-l B]]}
          \end{array}
        \end{math}
        \qquad\qquad
        $\Pi_2$:
        \begin{math}
          $$\mprset{flushleft}
          \inferrule* [right={\tiny imprL}] {
            {
              \begin{array}{cc}
                \pi_1 & \pi_2 \\
                {[[D2 |-l A1]]} & {[[G1; A2; G2; B; G3 |-l C]]}
              \end{array}
            }
          }{[[G1; A1 -> A2; D2; G2; B; G3 |-l C]]}
        \end{math}
      \end{center}
      By assumption, $c(\Pi_1),c(\Pi_2)\leq |B|$. By induction on $\Pi_1$
      and $\pi_2$, there is a proof $\Pi'$ for sequent
      $[[G1; A2; G2; D1; G3 |-l C]]$ s.t. $c(\Pi') \leq |B|$. Therefore,
      the proof $\Pi$ can be constructed as follows with
      $c(\Pi) = c(\Pi') \leq |B|$.
      \begin{center}
        \scriptsize
        \begin{math}
          $$\mprset{flushleft}
          \inferrule* [right={\tiny imprL}] {
            {
              \begin{array}{cc}
                \pi_1 & \Pi' \\
                {[[D2 |-l A1]]} & {[[G1; A2; G2; D1; G3 |-l C]]}
              \end{array}
            }
          }{[[G1; A1 -> A2; D2; G2; D1; G3 |-l C]]}
        \end{math}
      \end{center}
\end{itemize}


\subsubsection{Left introduction of the non-commutative right implication $\rto$}
\begin{itemize}
\item Case 1:
      \begin{center}
        \scriptsize
        \begin{math}
          \begin{array}{c}
            \Pi_1 \\
            {[[I |-c X]]}
          \end{array}
        \end{math}
        \qquad\qquad
        $\Pi_2$:
        \begin{math}
          $$\mprset{flushleft}
          \inferrule* [right={\tiny implL}] {
            {
              \begin{array}{cc}
                \pi_1 & \pi_2 \\
                {[[D1; X; D2 |-l A1]]} & {[[G1; A2; G2 |-l B]]}
              \end{array}
            }
          }{[[G1; D1; A2 <- A1; X; D2; G2 |-l B]]}
        \end{math}
      \end{center}
      By assumption, $c(\Pi_1),c(\Pi_2)\leq |X|$. By induction on $\Pi_1$
      and $\pi_1$, there is a proof $\Pi'$ for sequent
      $[[D1; I; D2 |-l A1]]$ s.t. $c(\Pi') \leq |X|$. Therefore, the proof
      $\Pi$ can be constructed as follows with $c(\Pi) = c(\Pi') \leq |X|$.
      \begin{center}
        \scriptsize
        \begin{math}
          $$\mprset{flushleft}
          \inferrule* [right={\tiny implL}] {
            {
              \begin{array}{cc}
                \Pi' & \pi_2 \\
                {[[D1; I; D2 |-l A1]]} & {[[G1; A2; G2 |-l B]]}
              \end{array}
            }
          }{[[G1; D1; A2 <- A1; I; D2; G2 |-l B]]}
        \end{math}
      \end{center}

\item Case 2:
      \begin{center}
        \scriptsize
        \begin{math}
          \begin{array}{c}
            \Pi_1 \\
            {[[G |-l C]]}
          \end{array}
        \end{math}
        \qquad\qquad
        $\Pi_2$:
        \begin{math}
          $$\mprset{flushleft}
          \inferrule* [right={\tiny implL}] {
            {
              \begin{array}{cc}
                \pi_1 & \pi_2 \\
                {[[D1; C; D2 |-l A1]]} & {[[G1; A2; G2 |-l B]]}
              \end{array}
            }
          }{[[G1; D1; C; D2; A2 <- A1; G2 |-l B]]}
        \end{math}
      \end{center}
      By assumption, $c(\Pi_1),c(\Pi_2)\leq |C|$. By induction on $\Pi_1$
      and $\pi_1$, there is a proof $\Pi'$ for sequent
      $[[D1; G; D2 |-l A1]]$ s.t. $c(\Pi') \leq |C|$. Therefore, the proof
      $\Pi$ can be constructed as follows with $c(\Pi) = c(\Pi') \leq |C|$.
      \begin{center}
        \scriptsize
        \begin{math}
          $$\mprset{flushleft}
          \inferrule* [right={\tiny implL}] {
            {
              \begin{array}{cc}
                \Pi' & \pi_2 \\
                {[[D1; G; D2 |-l A1]]} & {[[G1; A2; G2 |-l B]]}
              \end{array}
            }
          }{[[G1; D1; G; D2; A2 <- A1; G2 |-l B]]}
        \end{math}
      \end{center}

\item Case 3:
      \begin{center}
        \scriptsize
        \begin{math}
          \begin{array}{c}
            \Pi_1 \\
            {[[I |-c X]]}
          \end{array}
        \end{math}
        \qquad\qquad
        $\Pi_2$:
        \begin{math}
          $$\mprset{flushleft}
          \inferrule* [right={\tiny implL}] {
            {
              \begin{array}{cc}
                \pi_1 & \pi_2 \\
                {[[D |-l A1]]} & {[[G1; X; G2; A2; G3 |-l B]]}
              \end{array}
            }
          }{[[G1; X; G2; D; A2 <- A1; G3 |-l B]]}
        \end{math}
      \end{center}
      By assumption, $c(\Pi_1),c(\Pi_2)\leq |X|$. By induction on $\Pi_1$
      and $\pi_2$, there is a proof $\Pi'$ for sequent
      $[[G1; I; G2; A2; G3 |-l B]]$ s.t. $c(\Pi') \leq |X|$. Therefore, the
      proof $\Pi$ can be constructed as follows with
      $c(\Pi) = c(\Pi') \leq |X|$.
      \begin{center}
        \scriptsize
        \begin{math}
          $$\mprset{flushleft}
          \inferrule* [right={\tiny implL}] {
            {
              \begin{array}{cc}
                \pi_1 & \Pi' \\
                {[[D |-l A1]]} & {[[G1; I; G2; A2; G3 |-l B]]}
              \end{array}
            }
          }{[[G1; I; G2; D; A2 <- A1; G3 |-l B]]}
        \end{math}
      \end{center}

\item Case 4:
      \begin{center}
        \scriptsize
        \begin{math}
          \begin{array}{c}
            \Pi_1 \\
            {[[D1 |-l B]]}
          \end{array}
        \end{math}
        \qquad\qquad
        $\Pi_2$:
        \begin{math}
          $$\mprset{flushleft}
          \inferrule* [right={\tiny implL}] {
            {
              \begin{array}{cc}
                \pi_1 & \pi_2 \\
                {[[D2 |-l A1]]} & {[[G1; B; G2; A2; G3 |-l C]]}
              \end{array}
            }
          }{[[G1; B; G2; D2; A2 <- A1; G3 |-l C]]}
        \end{math}
      \end{center}
      By assumption, $c(\Pi_1),c(\Pi_2)\leq |B|$. By induction on $\Pi_1$
      and $\pi_2$, there is a proof $\Pi'$ for sequent
      $[[G1; D1; G2; A2; G3 |-l C]]$ s.t. $c(\Pi') \leq |B|$. Therefore,
      the proof $\Pi$ can be constructed as follows with
      $c(\Pi) = c(\Pi') \leq |B|$.
      \begin{center}
        \scriptsize
        \begin{math}
          $$\mprset{flushleft}
          \inferrule* [right={\tiny implL}] {
            {
              \begin{array}{cc}
                \pi_1 & \Pi' \\
                {[[D2 |-l A1]]} & {[[G1; D1; G2; A2; G3 |-l C]]}
              \end{array}
            }
          }{[[G1; D1; G2; D2; A2 <- A1; G3 |-l C]]}
        \end{math}
      \end{center}

\item Case 5:
      \begin{center}
        \scriptsize
        \begin{math}
          \begin{array}{c}
            \Pi_1 \\
            {[[I |-c X]]}
          \end{array}
        \end{math}
        \qquad\qquad
        $\Pi_2$:
        \begin{math}
          $$\mprset{flushleft}
          \inferrule* [right={\tiny implL}] {
            {
              \begin{array}{cc}
                \pi_1 & \pi_2 \\
                {[[D |-l A1]]} & {[[G1; A2; G2; X; G3 |-l B]]}
              \end{array}
            }
          }{[[G1; D; A2 <- A1; D; G2; X; G3 |-l B]]}
        \end{math}
      \end{center}
      By assumption, $c(\Pi_1),c(\Pi_2)\leq |X|$. By induction on $\Pi_1$
      and $\pi_2$, there is a proof $\Pi'$ for sequent
      $[[G1; A2; G2; I; G3 |-l B]]$ s.t. $c(\Pi') \leq |X|$. Therefore, the
      proof $\Pi$ can be constructed as follows with
      $c(\Pi) = c(\Pi') \leq |X|$.
      \begin{center}
        \scriptsize
        \begin{math}
          $$\mprset{flushleft}
          \inferrule* [right={\tiny implL}] {
            {
              \begin{array}{cc}
                \pi_1 & \Pi' \\
                {[[D |-l A1]]} & {[[G1; A2; G2; I; G3 |-l B]]}
              \end{array}
            }
          }{[[G1; D; A2 <- A1; G2; I; G3 |-l B]]}
        \end{math}
      \end{center}

\item Case 6:
    \begin{center}
      \scriptsize
      \begin{math}
        \begin{array}{c}
          \Pi_1 \\
          {[[D1 |-l B]]}
        \end{array}
      \end{math}
      \qquad\qquad
      $\Pi_2$:
      \begin{math}
        $$\mprset{flushleft}
        \inferrule* [right={\tiny implL}] {
          {
            \begin{array}{cc}
              \pi_1 & \pi_2 \\
              {[[D2 |-l A1]]} & {[[G1; A2; G2; B; G3 |-l C]]}
            \end{array}
          }
        }{[[G1; D2; A2 <- A1; G2; B; G3 |-l C]]}
      \end{math}
    \end{center}
    By assumption, $c(\Pi_1),c(\Pi_2)\leq |B|$. By induction on $\Pi_1$ and
    $\pi_2$, there is a proof $\Pi'$ for sequent
    $[[G1; A2; G2; D1; G3 |-l C]]$ s.t. $c(\Pi') \leq |B|$. Therefore, the
    proof $\Pi$ can be constructed as follows with
    $c(\Pi) = c(\Pi') \leq |B|$.
    \begin{center}
      \scriptsize
      \begin{math}
        $$\mprset{flushleft}
        \inferrule* [right={\tiny implL}] {
          {
            \begin{array}{cc}
              \pi_1 & \Pi' \\
              {[[D2 |-l A1]]} & {[[G1; A2; G2; D1; G3 |-l C]]}
            \end{array}
          }
        }{[[G1; D2; A2 <- A1; G2; D1; G3 |-l C]]}
      \end{math}
    \end{center}
\end{itemize}




\subsubsection{Left introduction of the commutative tensor $\otimes$ (with low priority)}
\begin{itemize}
\item Case 1:
      \begin{center}
        \scriptsize
        \begin{math}
          \begin{array}{c}
            \Pi_1 \\
            {[[I |-c X]]}
          \end{array}
        \end{math}
        \qquad\qquad
        $\Pi_2$:
        \begin{math}
          $$\mprset{flushleft}
          \inferrule* [right={\tiny tenL}] {
            {
              \begin{array}{c}
                \pi \\
                {[[P1, X, P2, Y1, Y2, P3 |-c Z]]}
              \end{array}
            }
          }{[[P1, X, P2, Y1 (*) Y2, P3 |-c Z]]}
        \end{math}
      \end{center}
      By assumption, $c(\Pi_1),c(\Pi_2)\leq |X|$. By induction on $\Pi_1$
      and $\pi$, there is a proof $\Pi'$ for sequent
      $[[P1, I, P2, Y1, Y2, P3 |-c Z]]$ s.t. $c(\Pi') \leq |X|$. Therefore,
      the proof $\Pi$ can be constructed as follows with
      $c(\Pi) = c(\Pi') \leq |X|$.
      \begin{center}
        \scriptsize
        \begin{math}
          $$\mprset{flushleft}
          \inferrule* [right={\tiny tenL}] {
            {
              \begin{array}{c}
                \Pi' \\
                {[[P1, I, P2, Y1, Y2, P3 |-c Z]]}
              \end{array}
            }
          }{[[P1, I, P2, Y1 (*) Y2, P3 |-c Z]]}
        \end{math}
      \end{center}

\item Case 2:
      \begin{center}
        \scriptsize
        \begin{math}
          \begin{array}{c}
            \Pi_1 \\
            {[[I |-c X]]}
          \end{array}
        \end{math}
        \qquad\qquad
        $\Pi_2$:
        \begin{math}
          $$\mprset{flushleft}
          \inferrule* [right={\tiny tenL}] {
            {
              \begin{array}{c}
                \pi \\
                {[[P1, Y1, Y2, P2, X, P3 |-c Z]]}
              \end{array}
            }
          }{[[P1, Y1 (*) Y2, P2, X, P3 |-c Z]]}
        \end{math}
      \end{center}
      By assumption, $c(\Pi_1),c(\Pi_2)\leq |X|$. By induction on $\Pi_1$
      and $\pi$, there is a proof $\Pi'$ for sequent
      $[[P1, Y1, Y2, P2, I, P3 |-c Z]]$ s.t. $c(\Pi') \leq |X|$. Therefore,
      the proof $\Pi$ can be constructed as follows with
      $c(\Pi) = c(\Pi') \leq |X|$.
      \begin{center}
        \scriptsize
        \begin{math}
          $$\mprset{flushleft}
          \inferrule* [right={\tiny tenL}] {
            {
              \begin{array}{c}
                \Pi' \\
                {[[P1, Y1, Y2, P2, I, P3 |-c Z]]}
              \end{array}
            }
          }{[[P1, Y1 (*) Y2, P2, I, P3 |-c Z]]}
        \end{math}
      \end{center}

\item Case 3:
      \begin{center}
        \scriptsize
        \begin{math}
          \begin{array}{c}
            \Pi_1 \\
            {[[I |-c X]]}
          \end{array}
        \end{math}
        \qquad\qquad
        $\Pi_2$:
        \begin{math}
          $$\mprset{flushleft}
          \inferrule* [right={\tiny tenL}] {
            {
              \begin{array}{c}
                \pi \\
                {[[G1; X; G2; Y1; Y2; G3 |-l A]]}
              \end{array}
            }
          }{[[G1; X; G2; Y1 (*) Y2; G3 |-l A]]}
        \end{math}
      \end{center}
      By assumption, $c(\Pi_1),c(\Pi_2)\leq |X|$. By induction on $\Pi_1$
      and $\pi$, there is a proof $\Pi'$ for sequent
      $[[G1; I; G2; Y1; Y2; G3 |-l A]]$ s.t. $c(\Pi') \leq |X|$. Therefore,
      the proof $\Pi$ can be constructed as follows with
      $c(\Pi) = c(\Pi') \leq |X|$.
      \begin{center}
        \scriptsize
        \begin{math}
          $$\mprset{flushleft}
          \inferrule* [right={\tiny tenL}] {
            {
              \begin{array}{c}
                \Pi' \\
                {[[G1; I; G2; Y1; Y2; G3 |-l A]]}
              \end{array}
            }
          }{[[G1; I; G2; Y1 (*) Y2; G3 |-l A]]}
        \end{math}
      \end{center}

\item Case 4:
      \begin{center}
        \scriptsize
        \begin{math}
          \begin{array}{c}
            \Pi_1 \\
            {[[D |-l B]]}
          \end{array}
        \end{math}
        \qquad\qquad
        $\Pi_2$:
        \begin{math}
          $$\mprset{flushleft}
          \inferrule* [right={\tiny tenL}] {
            {
              \begin{array}{c}
                \pi \\
                {[[G1; B; G2; Y1; Y2; G3 |-l A]]}
              \end{array}
            }
          }{[[G1; B; G2; Y1 (*) Y2; G3 |-l A]]}
        \end{math}
      \end{center}
      By assumption, $c(\Pi_1),c(\Pi_2)\leq |B|$. By induction on $\Pi_1$
      and $\pi$, there is a proof $\Pi'$ for sequent
      $[[G1; B; G2; Y1; Y2; G3 |-l A]]$ s.t. $c(\Pi') \leq |B|$. Therefore,
      the proof $\Pi$ can be constructed as follows with
      $c(\Pi) = c(\Pi') \leq |B|$.
      \begin{center}
        \scriptsize
        \begin{math}
          $$\mprset{flushleft}
          \inferrule* [right={\tiny tenL}] {
            {
              \begin{array}{c}
                \Pi' \\
                {[[G1; D; G2; Y1; Y2; G3 |-l A]]}
              \end{array}
            }
          }{[[G1; D; G2; Y1 (*) Y2; G3 |-l A]]}
        \end{math}
      \end{center}

\item Case 5:
      \begin{center}
        \scriptsize
        \begin{math}
          \begin{array}{c}
            \Pi_1 \\
            {[[I |-c X]]}
          \end{array}
        \end{math}
        \qquad\qquad
        $\Pi_2$:
        \begin{math}
          $$\mprset{flushleft}
          \inferrule* [right={\tiny tenL}] {
            {
              \begin{array}{c}
                \pi \\
                {[[G1; Y1; Y2; G2; X; G3 |-l A]]}
              \end{array}
            }
          }{[[G1; Y1 (*) Y2; G2; X; G3 |-l A]]}
        \end{math}
      \end{center}
      By assumption, $c(\Pi_1),c(\Pi_2)\leq |X|$. By induction on $\Pi_1$
      and $\pi$, there is a proof $\Pi'$ for sequent
      $[[G1; Y1; Y2; G2; I; G3 |-l A]]$ s.t. $c(\Pi') \leq |X|$. Therefore,
      the proof $\Pi$ can be constructed as follows with
      $c(\Pi) = c(\Pi') \leq |X|$.
      \begin{center}
        \scriptsize
        \begin{math}
          $$\mprset{flushleft}
          \inferrule* [right={\tiny tenL}] {
            {
              \begin{array}{c}
                \Pi' \\
                {[[G1; Y1; Y2; G2; I; G3 |-l A]]}
              \end{array}
            }
          }{[[G1; Y1 (*) Y2; G2; I; G3 |-l A]]}
        \end{math}
      \end{center}

\item Case 6:
      \begin{center}
        \scriptsize
        \begin{math}
          \begin{array}{c}
            \Pi_1 \\
            {[[D |-l B]]}
          \end{array}
        \end{math}
        \qquad\qquad
        $\Pi_2$:
        \begin{math}
          $$\mprset{flushleft}
          \inferrule* [right={\tiny tenL}] {
            {
              \begin{array}{c}
                \pi \\
                {[[G1; Y1; Y2; G2; B; G3 |-l A]]}
              \end{array}
            }
          }{[[G1; Y1 (*) Y2; G2; B; G3 |-l A]]}
        \end{math}
      \end{center}
      By assumption, $c(\Pi_1),c(\Pi_2)\leq |B|$. By induction on $\Pi_1$
      and $\pi$, there is a proof $\Pi'$ for sequent
      $[[G1; Y1; Y2; G2; D; G3 |-l A]]$ s.t. $c(\Pi') \leq |B|$. Therefore,
      the proof $\Pi$ can be constructed as follows with
      $c(\Pi) = c(\Pi') \leq |B|$.
      \begin{center}
        \scriptsize
        \begin{math}
          $$\mprset{flushleft}
          \inferrule* [right={\tiny tenL}] {
            {
              \begin{array}{c}
                \Pi' \\
                {[[G1; Y1; Y2; G2; D; G3 |-l A]]}
              \end{array}
            }
          }{[[G1; Y1 (*) Y2; G2; D; G3 |-l A]]}
        \end{math}
      \end{center}
\end{itemize}


\subsubsection{Left introduction of the non-commutative tensor $\tri$ (with low priority)}:
\begin{itemize}
\item Case 1:
      \begin{center}
        \scriptsize
        \begin{math}
          \begin{array}{c}
            \Pi_1 \\
            {[[I |-c X]]}
          \end{array}
        \end{math}
        \qquad\qquad
        $\Pi_2$:
        \begin{math}
          $$\mprset{flushleft}
          \inferrule* [right={\tiny tenL}] {
            {
              \begin{array}{c}
                \pi \\
                {[[G1; X; G2; A1; A2; G3 |-l B]]}
              \end{array}
            }
          }{[[G1; X; G2; A1 (>) A2; G3 |-l B]]}
        \end{math}
      \end{center}
      By assumption, $c(\Pi_1),c(\Pi_2)\leq |X|$. By induction on $\Pi_1$
      and $\pi$, there is a proof $\Pi'$ for sequent
      $[[G1; I; G2; A1; A2; G3 |-l B]]$ s.t. $c(\Pi') \leq |X|$. Therefore,
      the proof $\Pi$ can be constructed as follows with
      $c(\Pi) = c(\Pi') \leq |X|$.
      \begin{center}
        \scriptsize
        \begin{math}
          $$\mprset{flushleft}
          \inferrule* [right={\tiny tenL}] {
            {
              \begin{array}{c}
                \Pi' \\
                {[[G1; I; G2; A1; A2; G3 |-l B]]}
              \end{array}
            }
          }{[[G1; I; G2; A1 (>) A2; G3 |-l B]]}
        \end{math}
      \end{center}

\item Case 2:
      \begin{center}
        \scriptsize
        \begin{math}
          \begin{array}{c}
            \Pi_1 \\
            {[[D |-l B]]}
          \end{array}
        \end{math}
        \qquad\qquad
        $\Pi_2$:
        \begin{math}
          $$\mprset{flushleft}
          \inferrule* [right={\tiny tenL}] {
            {
              \begin{array}{c}
                \pi \\
                {[[G1; B; G2; A1; A2; G3 |-l C]]}
              \end{array}
            }
          }{[[G1; B; G2; A1 (>) A2; G3 |-l C]]}
        \end{math}
      \end{center}
      By assumption, $c(\Pi_1),c(\Pi_2)\leq |B|$. By induction on $\Pi_1$
      and $\pi$, there is a proof $\Pi'$ for sequent
      $[[G1; D; G2; A1; A2; G3 |-l C]]$ s.t. $c(\Pi') \leq |B|$. Therefore,
      the proof $\Pi$ can be constructed as follows with
      $c(\Pi) = c(\Pi') \leq |B|$.
      \begin{center}
        \scriptsize
        \begin{math}
          $$\mprset{flushleft}
          \inferrule* [right={\tiny tenL}] {
            {
              \begin{array}{c}
                \Pi' \\
                {[[G1; D; G2; A1; A2; G3 |-l C]]}
              \end{array}
            }
          }{[[G1; D; G2; A1 (>) A2; G3 |-l C]]}
        \end{math}
      \end{center}

\item Case 3:
      \begin{center}
        \scriptsize
        \begin{math}
          \begin{array}{c}
            \Pi_1 \\
            {[[I |-c X]]}
          \end{array}
        \end{math}
        \qquad\qquad
        $\Pi_2$:
        \begin{math}
          $$\mprset{flushleft}
          \inferrule* [right={\tiny tenL}] {
            {
              \begin{array}{c}
                \pi \\
                {[[G1; A1; A2; G2; X; G3 |-l B]]}
              \end{array}
            }
          }{[[G1; A1 (>) A2; G2; X; G3 |-l B]]}
        \end{math}
      \end{center}
      By assumption, $c(\Pi_1),c(\Pi_2)\leq |X|$. By induction on $\Pi_1$
      and $\pi$, there is a proof $\Pi'$ for sequent
      $[[G1; A1; A2; G2; I; G3 |-l A]]$ s.t. $c(\Pi') \leq |X|$. Therefore,
      the proof $\Pi$ can be constructed as follows with
      $c(\Pi) = c(\Pi') \leq |X|$.
      \begin{center}
        \scriptsize
        \begin{math}
          $$\mprset{flushleft}
          \inferrule* [right={\tiny tenL}] {
            {
              \begin{array}{c}
                \Pi' \\
                {[[G1; A1; A2; G2; I; G3 |-l B]]}
              \end{array}
            }
          }{[[G1; A1 (>) A2; G2; I; G3 |-l B]]}
        \end{math}
      \end{center}

\item Case 4:
      \begin{center}
        \scriptsize
        \begin{math}
          \begin{array}{c}
            \Pi_1 \\
            {[[D |-l B]]}
          \end{array}
        \end{math}
        \qquad\qquad
        $\Pi_2$:
        \begin{math}
          $$\mprset{flushleft}
          \inferrule* [right={\tiny tenL}] {
            {
              \begin{array}{c}
                \pi \\
                {[[G1; A1; A2; G2; B; G3 |-l C]]}
              \end{array}
            }
          }{[[G1; A1 (>) A2; G2; B; G3 |-l C]]}
        \end{math}
      \end{center}
      By assumption, $c(\Pi_1),c(\Pi_2)\leq |B|$. By induction on $\Pi_1$
      and $\pi$, there is a proof $\Pi'$ for sequent
      $[[G1; A1; A2; G2; D; G3 |-l C]]$ s.t. $c(\Pi') \leq |B|$. Therefore,
      the proof $\Pi$ can be constructed as follows with
      $c(\Pi) = c(\Pi') \leq |B|$.
      \begin{center}
        \scriptsize
        \begin{math}
          $$\mprset{flushleft}
          \inferrule* [right={\tiny tenL}] {
            {
              \begin{array}{c}
                \Pi' \\
                {[[G1; A1; A2; G2; D; G3 |-l C]]}
              \end{array}
            }
          }{[[G1; A1 (>) A2; G2; D; G3 |-l C]]}
        \end{math}
      \end{center}
\end{itemize}



\subsubsection{$\SCdruleTXXexName$}
\begin{itemize}
\item Case 1:
      \begin{center}
        \scriptsize
        \begin{math}
          \begin{array}{c}
            \Pi_1 \\
            {[[I |-c X]]}
          \end{array}
        \end{math}
        \qquad\qquad
        $\Pi_2$:
        \begin{math}
          $$\mprset{flushleft}
          \inferrule* [right={\tiny beta}] {
            {
              \begin{array}{c}
                \pi \\
                {[[P1, X, P2, Y1, Y2, P3 |-c Z]]}
              \end{array}
            }
          }{[[P1, X, P2, Y2, Y1, P3 |-c Z]]}
        \end{math}
      \end{center}
      By assumption, $c(\Pi_1),c(\Pi_2)\leq |X|$. By induction on $\Pi_1$
      and $\pi$, there is a proof $\Pi'$ for sequent
      $[[P1, I, P2, Y1, Y2, P3 |-c Z]]$ s.t. $c(\Pi') \leq |X|$. Therefore,
      the proof $\Pi$ can be constructed as follows with
      $c(\Pi) = c(\Pi') \leq |X|$.
      \begin{center}
        \scriptsize
        \begin{math}
          $$\mprset{flushleft}
          \inferrule* [right={\tiny cut}] {
            {
              \begin{array}{cc}
                \Pi' \\
                {[[P1, I, P2, Y1, Y2, P3 |-c Z]]}
              \end{array}
            }
          }{[[P1, I, P2, Y2, Y1, P3 |-c Z]]}
        \end{math}
      \end{center}

\item Case 2:
      \begin{center}
        \scriptsize
        \begin{math}
          \begin{array}{c}
            \Pi_1 \\
            {[[I |-c X]]}
          \end{array}
        \end{math}
        \qquad\qquad
        $\Pi_2$:
        \begin{math}
          $$\mprset{flushleft}
          \inferrule* [right={\tiny beta}] {
            {
              \begin{array}{c}
                \pi \\
                {[[P1, Y1, Y2, P2, X, P3 |-c Z]]}
              \end{array}
            }
          }{[[P1, X, P2, Y2, Y1, P3 |-c Z]]}
        \end{math}
      \end{center}
      By assumption, $c(\Pi_1),c(\Pi_2)\leq |X|$. By induction on $\Pi_1$
      and $\pi$, there is a proof $\Pi'$ for sequent
      $[[P1, Y1, Y2, P2, I, P3 |-c Z]]$ s.t. $c(\Pi') \leq |X|$. Therefore,
      the proof $\Pi$ can be constructed as follows with
      $c(\Pi) = c(\Pi') \leq |X|$.
      \begin{center}
        \scriptsize
        \begin{math}
          $$\mprset{flushleft}
          \inferrule* [right={\tiny cut}] {
            {
              \begin{array}{cc}
                \Pi' \\
                {[[P1, Y1, Y2, P2, I, P3 |-c Z]]}
              \end{array}
            }
          }{[[P1, Y2, Y1, P2, I, P3 |-c Z]]}
        \end{math}
      \end{center}
\end{itemize}


\subsubsection{$\SCdruleSXXexName$}
\begin{itemize}
\item Case 1:
      \begin{center}
        \scriptsize
        \begin{math}
          \begin{array}{c}
            \Pi_1 \\
            {[[I |-c X]]}
          \end{array}
        \end{math}
        \qquad\qquad
        $\Pi_2$:
        \begin{math}
          $$\mprset{flushleft}
          \inferrule* [right={\tiny beta}] {
            {
              \begin{array}{c}
                \pi \\
                {[[G1; X; G2; Y1; Y2; G3 |-l A]]}
              \end{array}
            }
          }{[[G1; X; G2; Y2; Y1; G3 |-l A]]}
        \end{math}
      \end{center}
      By assumption, $c(\Pi_1),c(\Pi_2)\leq |X|$. By induction on $\Pi_1$
      and $\pi$, there is a proof $\Pi'$ for sequent
      $[[G1; I; G2; Y1; Y2; G3 |-l A]]$ s.t. $c(\Pi') \leq |X|$. Therefore,
      the proof $\Pi$ can be constructed as follows with
      $c(\Pi) = c(\Pi') \leq |X|$.
      \begin{center}
        \scriptsize
        \begin{math}
          $$\mprset{flushleft}
          \inferrule* [right={\tiny cut}] {
            {
              \begin{array}{cc}
                \Pi' \\
                {[[G1; I; G2; Y1; Y2; G3 |-l A]]}
              \end{array}
            }
          }{[[G1; I; G2; Y2; Y1; G3 |-l A]]}
        \end{math}
      \end{center}

\item Case 2:
      \begin{center}
        \scriptsize
        \begin{math}
          \begin{array}{c}
            \Pi_1 \\
            {[[D |-l B]]}
          \end{array}
        \end{math}
        \qquad\qquad
        $\Pi_2$:
        \begin{math}
          $$\mprset{flushleft}
          \inferrule* [right={\tiny beta}] {
            {
              \begin{array}{c}
                \pi \\
                {[[G1; B; G2; Y1; Y2; G3 |-l A]]}
              \end{array}
            }
          }{[[G1; B; G2; Y2; Y1; G3 |-l A]]}
        \end{math}
      \end{center}
      By assumption, $c(\Pi_1),c(\Pi_2)\leq |X|$. By induction on $\Pi_1$
      and $\pi$, there is a proof $\Pi'$ for sequent
      $[[G1; D; G2; Y1; Y2; G3 |-l A]]$ s.t. $c(\Pi') \leq |X|$. Therefore,
      the proof $\Pi$ can be constructed as follows with
      $c(\Pi) = c(\Pi') \leq |X|$.
      \begin{center}
        \scriptsize
        \begin{math}
          $$\mprset{flushleft}
          \inferrule* [right={\tiny cut}] {
            {
              \begin{array}{cc}
                \Pi' \\
                {[[G1; D; G2; Y1; Y2; G3 |-l A]]}
              \end{array}
            }
          }{[[G1; D; G2; Y2; Y1; G3 |-l A]]}
        \end{math}
      \end{center}

\item Case 3:
      \begin{center}
        \scriptsize
        \begin{math}
          \begin{array}{c}
            \Pi_1 \\
            {[[I |-c X]]}
          \end{array}
        \end{math}
        \qquad\qquad
        $\Pi_2$:
        \begin{math}
          $$\mprset{flushleft}
          \inferrule* [right={\tiny beta}] {
            {
              \begin{array}{c}
                \pi \\
                {[[G1; Y1; Y2; G2; X; G3 |-l A]]}
              \end{array}
            }
          }{[[G1; X; G2; Y2; Y1; G3 |-l A]]}
        \end{math}
      \end{center}
      By assumption, $c(\Pi_1),c(\Pi_2)\leq |X|$. By induction on $\Pi_1$
      and $\pi$, there is a proof $\Pi'$ for sequent
      $[[G1; Y1; Y2; G2; I; G3 |-l A]]$ s.t. $c(\Pi') \leq |X|$. Therefore,
      the proof $\Pi$ can be constructed as follows with
      $c(\Pi) = c(\Pi') \leq |X|$.
      \begin{center}
        \scriptsize
        \begin{math}
          $$\mprset{flushleft}
          \inferrule* [right={\tiny cut}] {
            {
              \begin{array}{cc}
                \Pi' \\
                {[[G1; Y1; Y2; G2; I; G3 |-l A]]}
              \end{array}
            }
          }{[[G1; Y2; Y1; G2; I; G3 |-l A]]}
        \end{math}
      \end{center}

\item Case 4:
      \begin{center}
        \scriptsize
        \begin{math}
          \begin{array}{c}
            \Pi_1 \\
            {[[D |-l B]]}
          \end{array}
        \end{math}
        \qquad\qquad
        $\Pi_2$:
        \begin{math}
          $$\mprset{flushleft}
          \inferrule* [right={\tiny beta}] {
            {
              \begin{array}{c}
                \pi \\
                {[[G1; Y1; Y2; G2; B; G3 |-l A]]}
              \end{array}
            }
          }{[[G1; Y2; Y1; G2; B; G3 |-l A]]}
        \end{math}
      \end{center}
      By assumption, $c(\Pi_1),c(\Pi_2)\leq |X|$. By induction on $\Pi_1$
      and $\pi$, there is a proof $\Pi'$ for sequent
      $[[G1; Y1; Y2; G2; D; G3 |-l A]]$ s.t. $c(\Pi') \leq |X|$. Therefore,
      the proof $\Pi$ can be constructed as follows with
      $c(\Pi) = c(\Pi') \leq |X|$.
      \begin{center}
        \scriptsize
        \begin{math}
          $$\mprset{flushleft}
          \inferrule* [right={\tiny cut}] {
            {
              \begin{array}{cc}
                \Pi' \\
                {[[G1; Y1; Y2; G2; D; G3 |-l A]]}
              \end{array}
            }
          }{[[G1; Y2; Y1; G2; D; G3 |-l A]]}
        \end{math}
      \end{center}
\end{itemize}



\subsubsection{Left introduction of the commutative unit $[[UnitT]]$ (with low priority)}
\begin{itemize}
\item Case 1:
      \begin{center}
        \scriptsize
        \begin{math}
          \begin{array}{c}
            \Pi_1 \\
            {[[P |-c X]]}
          \end{array}
        \end{math}
        \qquad\qquad
        $\Pi_2$:
        \begin{math}
          $$\mprset{flushleft}
          \inferrule* [right={\tiny unitL}] {
            {
              \begin{array}{c}
                \pi \\
                {[[I1, I2, X, I3 |-c Y]]}
              \end{array}
            }
          }{[[I1, UnitT, I2, X, I3 |-c Y]]}
        \end{math}
      \end{center}
      By assumption, $c(\Pi_1),c(\Pi_2)\leq |X|$. By induction on $\Pi_1$
      and $\pi$, there is a proof $\Pi'$ for sequent
      $[[I1, I2, P, I3 |-c Y]]$
      s.t. $c(\Pi') \leq |X|$. Therefore, the proof $\Pi$ can be
      constructed as follows with $c(\Pi) = c(\Pi') \leq |X|$.
      \begin{center}
        \scriptsize
        \begin{math}
          $$\mprset{flushleft}
          \inferrule* [right={\tiny unitL}] {
            {
              \begin{array}{c}
                \Pi' \\
                {[[I1, I2, P, I3 |-c Y]]}
              \end{array}
            }
          }{[[I1, UnitT, I2, P, I3 |-c Y]]}
        \end{math}
      \end{center}

\item Case 2:
      \begin{center}
        \scriptsize
        \begin{math}
          \begin{array}{c}
            \Pi_1 \\
            {[[I |-c X]]}
          \end{array}
        \end{math}
        \qquad\qquad
        $\Pi_2$:
        \begin{math}
          $$\mprset{flushleft}
          \inferrule* [right={\tiny unitL1}] {
            {
              \begin{array}{c}
                \pi \\
                {[[G1; G2; X; G3 |-l A]]}
              \end{array}
            }
          }{[[G1; UnitT; G2; X; G3 |-l A]]}
        \end{math}
      \end{center}
      By assumption, $c(\Pi_1),c(\Pi_2)\leq |X|$. By induction on $\Pi_1$
      and $\pi$, there is a proof $\Pi'$ for sequent
      $[[G1; G2; I; G2 |-l A]]$
      s.t. $c(\Pi') \leq |X|$. Therefore, the proof $\Pi$ can be
      constructed as follows with $c(\Pi) = c(\Pi') \leq |X|$.
      \begin{center}
        \scriptsize
        \begin{math}
          $$\mprset{flushleft}
          \inferrule* [right={\tiny unitL1}] {
            {
              \begin{array}{c}
                \Pi' \\
                {[[G1; G2; I; G3 |-l A]]}
              \end{array}
            }
          }{[[G1; UnitT; G2; I; G3 |-l A]]}
        \end{math}
      \end{center}

\item Case 3:
      \begin{center}
        \scriptsize
        \begin{math}
          \begin{array}{c}
            \Pi_1 \\
            {[[D |-l B]]}
          \end{array}
        \end{math}
        \qquad\qquad
        $\Pi_2$:
        \begin{math}
          $$\mprset{flushleft}
          \inferrule* [right={\tiny unitL1}] {
            {
              \begin{array}{c}
                \pi \\
                {[[G1; G2; B; G3 |-l A]]}
              \end{array}
            }
          }{[[G1; UnitT; G2; B; G3 |-l A]]}
        \end{math}
      \end{center}
      By assumption, $c(\Pi_1),c(\Pi_2)\leq |B|$. By induction on $\Pi_1$
      and $\pi$, there is a proof $\Pi'$ for sequent
      $[[G1; G2; D; G3 |-l A]]$
      s.t. $c(\Pi') \leq |B|$. Therefore, the proof $\Pi$ can be
      constructed as follows with $c(\Pi) = c(\Pi') \leq |B|$.
      \begin{center}
        \scriptsize
        \begin{math}
          $$\mprset{flushleft}
          \inferrule* [right={\tiny unitL1}] {
            {
              \begin{array}{c}
                \Pi' \\
                {[[G1; G2; D; G3 |-l A]]}
              \end{array}
            }
          }{[[G1; UnitT; G2; D; G3 |-l A]]}
        \end{math}
      \end{center}
\end{itemize}

\subsubsection{Left introduction of the non-commutative unit $[[UnitS]]$ (with low priority)}
\begin{itemize}
\item Case 1:
      \begin{center}
        \scriptsize
        \begin{math}
          \begin{array}{c}
            \Pi_1 \\
            {[[I |-c X]]}
          \end{array}
        \end{math}
        \qquad\qquad
        $\Pi_2$:
        \begin{math}
          $$\mprset{flushleft}
          \inferrule* [right={\tiny unitL2}] {
            {
              \begin{array}{c}
                \pi \\
                {[[G1; G2; X; G3 |-l A]]}
              \end{array}
            }
          }{[[G1; UnitS; G2; X; G3 |-l A]]}
        \end{math}
      \end{center}
      By assumption, $c(\Pi_1),c(\Pi_2)\leq |X|$. By induction on $\Pi_1$
      and $\pi$, there is a proof $\Pi'$ for sequent
      $[[G1; G2; I; G3 |-l A]]$
      s.t. $c(\Pi') \leq |X|$. Therefore, the proof $\Pi$ can be
      constructed as follows with $c(\Pi) = c(\Pi') \leq |X|$.
      \begin{center}
        \scriptsize
        \begin{math}
          $$\mprset{flushleft}
          \inferrule* [right={\tiny unitL2}] {
            {
              \begin{array}{c}
                \Pi' \\
                {[[G1; G2; I; G3 |-l A]]}
              \end{array}
            }
          }{[[G1; UnitS; G2; I; G3 |-l A]]}
        \end{math}
      \end{center}

\item Case 2:
      \begin{center}
        \scriptsize
        \begin{math}
          \begin{array}{c}
            \Pi_1 \\
            {[[D |-l B]]}
          \end{array}
        \end{math}
        \qquad\qquad
        $\Pi_2$:
        \begin{math}
          $$\mprset{flushleft}
          \inferrule* [right={\tiny unitL2}] {
            {
              \begin{array}{c}
                \pi \\
                {[[G1; G2; B; G3 |-l A]]}
              \end{array}
            }
          }{[[G1; UnitS; G2; B; G3 |-l A]]}
        \end{math}
      \end{center}
      By assumption, $c(\Pi_1),c(\Pi_2)\leq |B|$. By induction on $\Pi_1$
      and $\pi$, there is a proof $\Pi'$ for sequent
      $[[G1; G2; D; G3 |-l A]]$
      s.t. $c(\Pi') \leq |B|$. Therefore, the proof $\Pi$ can be
      constructed as follows with $c(\Pi) = c(\Pi') \leq |B|$.
      \begin{center}
        \scriptsize
        \begin{math}
          $$\mprset{flushleft}
          \inferrule* [right={\tiny unitL2}] {
            {
              \begin{array}{c}
                \Pi' \\
                {[[G1; G2; D; G3 |-l A]]}
              \end{array}
            }
          }{[[G1; UnitS; G2; D; G3 |-l A]]}
        \end{math}
      \end{center}
\end{itemize}



\subsubsection{Right introduction of the commutative implication $\multimap$ (with low priority)}
\begin{center}
  \scriptsize
  \begin{math}
    \begin{array}{c}
      \Pi_1 \\
      {[[I |-c X]]}
    \end{array}
  \end{math}
  \qquad\qquad
  $\Pi_2$:
  \begin{math}
    $$\mprset{flushleft}
    \inferrule* [right={\tiny impR}] {
      {
        \begin{array}{c}
          \pi \\
          {[[P1, X, P2, Y1 |-c Y2]]}
        \end{array}
      }
    }{[[P1, X, P2 |-c Y1 -o Y2]]}
  \end{math}
\end{center}
By assumption, $c(\Pi_1),c(\Pi_2)\leq |X|$. By induction on $\Pi_1$
and $\pi$, there is a proof $\Pi'$ for sequent \\
$[[P1, I, P2, Y1 |-c Y2]]$ s.t. $c(\Pi') \leq |X|$. Therefore, the
proof $\Pi$ can be constructed as follows with \\
$c(\Pi) = c(\Pi') \leq |X|$.
\begin{center}
  \scriptsize
  \begin{math}
    $$\mprset{flushleft}
    \inferrule* [right={\tiny impR}] {
      {
        \begin{array}{c}
          \Pi' \\
          {[[P1, I, P2, Y1 |-c Y2]]}
        \end{array}
      }
    }{[[P1, I, P2 |-c Y1 -o Y2]]}
  \end{math}
\end{center}



\subsubsection{Right introduction of the non-commutative left implication $\lto$ (with low priority)}
\begin{itemize}
\item Case 1:
      \begin{center}
        \scriptsize
        \begin{math}
          \begin{array}{c}
            \Pi_1 \\
            {[[I |-c X]]}
          \end{array}
        \end{math}
        \qquad\qquad
        $\Pi_2$:
        \begin{math}
          $$\mprset{flushleft}
          \inferrule* [right={\tiny impR}] {
            {
              \begin{array}{c}
                \pi \\
                {[[G1; X; G2; A |-l B]]}
              \end{array}
            }
          }{[[G1; X; G2 |-l A -> B]]}
        \end{math}
      \end{center}
      By assumption, $c(\Pi_1),c(\Pi_2)\leq |X|$. By induction on $\Pi_1$
      and $\pi$, there is a proof $\Pi'$ for sequent \\
      $[[G1; I; G2; A |-l B]]$ s.t. $c(\Pi') \leq |X|$. Therefore, the
      proof $\Pi$ can be constructed as follows with \\
      $c(\Pi) = c(\Pi') \leq |X|$.
      \begin{center}
        \scriptsize
        \begin{math}
          $$\mprset{flushleft}
          \inferrule* [right={\tiny implR}] {
            {
              \begin{array}{c}
                \Pi' \\
                {[[G1; I; G2; A |-l B]]}
              \end{array}
            }
          }{[[G1; I; G2 |-l A -> B]]}
        \end{math}
      \end{center}

\item Case 2:
      \begin{center}
        \scriptsize
        \begin{math}
          \begin{array}{c}
            \Pi_1 \\
            {[[D |-l C]]}
          \end{array}
        \end{math}
        \qquad\qquad
        $\Pi_2$:
        \begin{math}
          $$\mprset{flushleft}
          \inferrule* [right={\tiny impR}] {
            {
              \begin{array}{c}
                \pi \\
                {[[G1; C; G2; A |-l B]]}
              \end{array}
            }
          }{[[G1; C; G2 |-l A -> B]]}
        \end{math}
      \end{center}
      By assumption, $c(\Pi_1),c(\Pi_2)\leq |C|$. By induction on $\Pi_1$
      and $\pi$, there is a proof $\Pi'$ for sequent
      $[[G1; D; G2; A |-l B]]$ s.t. $c(\Pi') \leq |C|$. Therefore, the
      proof $\Pi$ can be constructed as follows with \\
      $c(\Pi) = c(\Pi') \leq |C|$.
      \begin{center}
        \scriptsize
        \begin{math}
          $$\mprset{flushleft}
          \inferrule* [right={\tiny implR}] {
            {
              \begin{array}{c}
                \Pi' \\
                {[[G1; D; G2; A |-l B]]}
              \end{array}
            }
          }{[[G1; D; G2 |-l A -> B]]}
        \end{math}
      \end{center}
\end{itemize}




\subsubsection{Right introduction of the non-commutative right implication $\rto$ (with low priority)}
\begin{itemize}
\item Case 1:
      \begin{center}
        \scriptsize
        \begin{math}
          \begin{array}{c}
            \Pi_1 \\
            {[[I |-c X]]}
          \end{array}
        \end{math}
        \qquad\qquad
        $\Pi_2$:
        \begin{math}
          $$\mprset{flushleft}
          \inferrule* [right={\tiny impL}] {
            {
              \begin{array}{c}
                \pi \\
                {[[A; G1; X; G2|-l B]]}
              \end{array}
            }
          }{[[G1; X; G2 |-l B <- A]]}
        \end{math}
      \end{center}
      By assumption, $c(\Pi_1),c(\Pi_2)\leq |X|$. By induction on $\Pi_1$
      and $\pi$, there is a proof $\Pi'$ for sequent
      $[[A; G1; I; G2 |-l B]]$ s.t. $c(\Pi') \leq |X|$. Therefore, the
      proof $\Pi$ can be constructed as follows with \\
      $c(\Pi) = c(\Pi') \leq |X|$.
      \begin{center}
        \scriptsize
        \begin{math}
          $$\mprset{flushleft}
          \inferrule* [right={\tiny impR}] {
            {
              \begin{array}{c}
                \Pi' \\
                {[[A; G1; I; G2 |-l B]]}
              \end{array}
            }
          }{[[G1; I; G2 |-l B <- A]]}
        \end{math}
      \end{center}

\item Case 2:
      \begin{center}
        \scriptsize
        \begin{math}
          \begin{array}{c}
            \Pi_1 \\
            {[[D |-l C]]}
          \end{array}
        \end{math}
        \qquad\qquad
        $\Pi_2$:
        \begin{math}
          $$\mprset{flushleft}
          \inferrule* [right={\tiny impR}] {
            {
              \begin{array}{c}
                \pi \\
                {[[A; G1; C; G2 |-l B]]}
              \end{array}
            }
          }{[[G1; C; G2 |-l B <- A]]}
        \end{math}
      \end{center}
      By assumption, $c(\Pi_1),c(\Pi_2)\leq |C|$. By induction on $\Pi_1$
      and $\pi$, there is a proof $\Pi'$ for sequent
      $[[G1; D; G2; A |-l B]]$ s.t. $c(\Pi') \leq |C|$. Therefore, the
      proof $\Pi$ can be constructed as follows with \\
      $c(\Pi) = c(\Pi') \leq |C|$.
      \begin{center}
        \scriptsize
        \begin{math}
          $$\mprset{flushleft}
          \inferrule* [right={\tiny impR}] {
            {
              \begin{array}{c}
                \Pi' \\
                {[[A; G1; D; G2 |-l B]]}
              \end{array}
            }
          }{[[G1; D; G2 |-l B <- A]]}
        \end{math}
      \end{center}
\end{itemize}



\subsubsection{Right introduction of the functor $F$}
\begin{center}
  \scriptsize
  \begin{math}
    \begin{array}{c}
      \Pi_1 \\
      {[[I |-c X]]}
    \end{array}
  \end{math}
  \qquad\qquad
  $\Pi_2$:
  \begin{math}
    $$\mprset{flushleft}
    \inferrule* [right={\tiny Fr}] {
      {
        \begin{array}{c}
          \pi \\
          {[[P1, X, P2 |-c Y]]}
        \end{array}
      }
    }{[[P1, X, P2 |-l F Y]]}
  \end{math}
\end{center}
By assumption, $c(\Pi_1),c(\Pi_2)\leq |X|$. By induction on $\Pi_1$
and $\pi$, there is a proof $\Pi'$ for sequent \\
$[[P1, I, P2 |-c Y]]$ s.t. $c(\Pi') \leq |X|$. Therefore, the proof $\Pi$
can be constructed as follows with \\
$c(\Pi) = c(\Pi') \leq |X|$.
\begin{center}
  \scriptsize
  \begin{math}
    $$\mprset{flushleft}
    \inferrule* [right={\tiny Fr}] {
      {
        \begin{array}{c}
          \Pi' \\
          {[[P1, I, P2 |-c Y]]}
        \end{array}
      }
    }{[[P1, I, P2 |-l F Y]]}
  \end{math}
\end{center}



\subsubsection{Left introduction of the functor $F$ (with low priority)}
\begin{itemize}
\item Case 1:
      \begin{center}
        \scriptsize
        \begin{math}
          \begin{array}{c}
            \Pi_1 \\
            {[[I |-c X]]}
          \end{array}
        \end{math}
        \qquad\qquad
        $\Pi_2$:
        \begin{math}
          $$\mprset{flushleft}
          \inferrule* [right={\tiny Fl}] {
            {
              \begin{array}{c}
                \pi \\
                {[[G1; X; G2; Y; G3 |-l A]]}
              \end{array}
            }
          }{[[G1; X; G2; F Y; G3 |-l A]]}
        \end{math}
      \end{center}
      By assumption, $c(\Pi_1),c(\Pi_2)\leq |X|$. By induction on $\Pi_1$
      and $\pi$, there is a proof $\Pi'$ for sequent
      $[[G1; I; G2; Y; G3 |-l A]]$ s.t. $c(\Pi') \leq |X|$. Therefore, the
      proof $\Pi$ can be constructed as follows with
      $c(\Pi) = c(\Pi') \leq |X|$.
      \begin{center}
        \scriptsize
        \begin{math}
          $$\mprset{flushleft}
          \inferrule* [right={\tiny Fl}] {
            {
              \begin{array}{c}
                \Pi' \\
                {[[G1; I; G2; Y; G3 |-l A]]}
              \end{array}
            }
          }{[[G1; I; G2; F Y; G3 |-l A]]}
        \end{math}
      \end{center}

\item Case 2:
      \begin{center}
        \scriptsize
        \begin{math}
          \begin{array}{c}
            \Pi_1 \\
            {[[D |-l B]]}
          \end{array}
        \end{math}
        \qquad\qquad
        $\Pi_2$:
        \begin{math}
          $$\mprset{flushleft}
          \inferrule* [right={\tiny Fl}] {
            {
              \begin{array}{c}
                \pi \\
                {[[G1; B; G2; Y; G3 |-l A]]}
              \end{array}
            }
          }{[[G1; B; G2; F Y; G3 |-l A]]}
        \end{math}
      \end{center}
      By assumption, $c(\Pi_1),c(\Pi_2)\leq |B|$. By induction on $\Pi_1$
      and $\pi$, there is a proof $\Pi'$ for sequent
      $[[G1; D; G2; Y; G3 |-l A]]$ s.t. $c(\Pi') \leq |B|$. Therefore, the
      proof $\Pi$ can be constructed as follows with
      $c(\Pi) = c(\Pi') \leq |B|$.
      \begin{center}
        \scriptsize
        \begin{math}
          $$\mprset{flushleft}
          \inferrule* [right={\tiny Fl}] {
            {
              \begin{array}{c}
                \Pi' \\
                {[[G1; D; G2; Y; G3 |-l A]]}
              \end{array}
            }
          }{[[G1; D; G2; F Y; G3 |-l A]]}
        \end{math}
      \end{center}

\item Case 3:
      \begin{center}
        \scriptsize
        \begin{math}
          \begin{array}{c}
            \Pi_1 \\
            {[[I |-c X]]}
          \end{array}
        \end{math}
        \qquad\qquad
        $\Pi_2$:
        \begin{math}
          $$\mprset{flushleft}
          \inferrule* [right={\tiny Fl}] {
            {
              \begin{array}{c}
                \pi \\
                {[[G1; Y; G2; X; G3 |-l A]]}
              \end{array}
            }
          }{[[G1; F Y; G2; X; G3 |-l A]]}
        \end{math}
      \end{center}
      By assumption, $c(\Pi_1),c(\Pi_2)\leq |X|$. By induction on $\Pi_1$
      and $\pi$, there is a proof $\Pi'$ for sequent
      $[[G1; Y; G2; I; G3 |-l A]]$ s.t. $c(\Pi') \leq |X|$. Therefore, the
      proof $\Pi$ can be constructed as follows with
      $c(\Pi) = c(\Pi') \leq |X|$.
      \begin{center}
        \scriptsize
        \begin{math}
          $$\mprset{flushleft}
          \inferrule* [right={\tiny Fl}] {
            {
              \begin{array}{c}
                \Pi' \\
                {[[G1; Y; G2; I; G3 |-l A]]}
              \end{array}
            }
          }{[[G1; F Y; G2; I; G3 |-l A]]}
        \end{math}
      \end{center}

\item Case 4:
      \begin{center}
        \scriptsize
        \begin{math}
          \begin{array}{c}
            \Pi_1 \\
            {[[D |-l B]]}
          \end{array}
        \end{math}
        \qquad\qquad
        $\Pi_2$:
        \begin{math}
          $$\mprset{flushleft}
          \inferrule* [right={\tiny Fl}] {
            {
              \begin{array}{c}
                \pi \\
                {[[G1; Y; G2; B; G3 |-l A]]}
              \end{array}
            }
          }{[[G1; F Y; G2; D; G3 |-l A]]}
        \end{math}
      \end{center}
      By assumption, $c(\Pi_1),c(\Pi_2)\leq |B|$. By induction on $\Pi_1$
      and $\pi$, there is a proof $\Pi'$ for sequent
      $[[G1; Y; G2; D; G3 |-l A]]$ s.t. $c(\Pi') \leq |B|$. Therefore, the
      proof $\Pi$ can be constructed as follows with
      $c(\Pi) = c(\Pi') \leq |B|$.
      \begin{center}
        \scriptsize
        \begin{math}
          $$\mprset{flushleft}
          \inferrule* [right={\tiny Fl}] {
            {
              \begin{array}{c}
                \Pi' \\
                {[[G1; Y; G2; D; G3 |-l A]]}
              \end{array}
            }
          }{[[G1; F Y; G2; D; G3 |-l A]]}
        \end{math}
      \end{center}
\end{itemize}




\subsubsection{Right introduction of the functor $G$ (with low priority)}
\begin{center}
  \scriptsize
  \begin{math}
    \begin{array}{c}
      \Pi_1 \\
      {[[I |-c X]]}
    \end{array}
  \end{math}
  \qquad\qquad
  $\Pi_2$:
  \begin{math}
    $$\mprset{flushleft}
    \inferrule* [right={\tiny Gr}] {
      {
        \begin{array}{c}
          \pi \\
          {[[P1; X; P2 |-l A]]}
        \end{array}
      }
    }{[[P1, X, P2 |-c Gf A]]}
  \end{math}
\end{center}
By assumption, $c(\Pi_1),c(\Pi_2)\leq |X|$. By induction on $\Pi_1$
and $\pi$, there is a proof $\Pi'$ for sequent \\
$[[P1, I, P2 |-l A]]$ s.t. $c(\Pi') \leq |X|$. Therefore, the proof $\Pi$
can be constructed as follows with \\
$c(\Pi) = c(\Pi') \leq |X|$.
\begin{center}
  \scriptsize
  \begin{math}
    $$\mprset{flushleft}
    \inferrule* [right={\tiny Gr}] {
      {
        \begin{array}{c}
          \Pi' \\
          {[[P1; I; P2 |-l A]]}
        \end{array}
      }
    }{[[P1, I, P2 |-c Gf A]]}
  \end{math}
\end{center}




\subsubsection{Left introduction of the functor $G$ (with low priority)}
\begin{itemize}
\item Case 1:
      \begin{center}
        \scriptsize
        \begin{math}
          \begin{array}{c}
            \Pi_1 \\
            {[[I |-c X]]}
          \end{array}
        \end{math}
        \qquad\qquad
        $\Pi_2$:
        \begin{math}
          $$\mprset{flushleft}
          \inferrule* [right={\tiny Gl}] {
            {
              \begin{array}{c}
                \pi \\
                {[[G1; X; G2; B; G3 |-l A]]}
              \end{array}
            }
          }{[[G1; X; G2; Gf B; G3 |-l A]]}
        \end{math}
      \end{center}
      By assumption, $c(\Pi_1),c(\Pi_2)\leq |X|$. By induction on $\Pi_1$
      and $\pi$, there is a proof $\Pi'$ for sequent
      $[[G1; I; G2; B; G3 |-l A]]$ s.t. $c(\Pi') \leq |X|$. Therefore, the
      proof $\Pi$ can be constructed as follows with
      $c(\Pi) = c(\Pi') \leq |X|$.
      \begin{center}
        \scriptsize
        \begin{math}
          $$\mprset{flushleft}
          \inferrule* [right={\tiny Gl}] {
            {
              \begin{array}{c}
                \Pi' \\
                {[[G1; I; G2; B; G3 |-l A]]}
              \end{array}
            }
          }{[[G1; I; G2; Gf B; G3 |-l A]]}
        \end{math}
      \end{center}

\item Case 2:
      \begin{center}
        \scriptsize
        \begin{math}
          \begin{array}{c}
            \Pi_1 \\
            {[[D |-l B]]}
          \end{array}
        \end{math}
        \qquad\qquad
        $\Pi_2$:
        \begin{math}
          $$\mprset{flushleft}
          \inferrule* [right={\tiny Gl}] {
            {
              \begin{array}{c}
                \pi \\
                {[[G1; B; G2; C; G3 |-l A]]}
              \end{array}
            }
          }{[[G1; B; G2; Gf C; G3 |-l A]]}
        \end{math}
      \end{center}
      By assumption, $c(\Pi_1),c(\Pi_2)\leq |B|$. By induction on $\Pi_1$
      and $\pi$, there is a proof $\Pi'$ for sequent
      $[[G1; D; G2; C; G3 |-l A]]$ s.t. $c(\Pi') \leq |B|$. Therefore, the
      proof $\Pi$ can be constructed as follows with
      $c(\Pi) = c(\Pi') \leq |B|$.
      \begin{center}
        \scriptsize
        \begin{math}
          $$\mprset{flushleft}
          \inferrule* [right={\tiny Gl}] {
            {
              \begin{array}{c}
                \Pi' \\
                {[[G1; D; G2; C; G3 |-l A]]}
              \end{array}
            }
          }{[[G1; D; G2; Gf C; G3 |-l A]]}
        \end{math}
      \end{center}

\item Case 3:
      \begin{center}
        \scriptsize
        \begin{math}
          \begin{array}{c}
            \Pi_1 \\
            {[[I |-c X]]}
          \end{array}
        \end{math}
        \qquad\qquad
        $\Pi_2$:
        \begin{math}
          $$\mprset{flushleft}
          \inferrule* [right={\tiny Gl}] {
            {
              \begin{array}{c}
                \pi \\
                {[[G1; B; G2; X; G3 |-l A]]}
              \end{array}
            }
          }{[[G1; Gf B; G2; X; G3 |-l A]]}
        \end{math}
      \end{center}
      By assumption, $c(\Pi_1),c(\Pi_2)\leq |X|$. By induction on $\Pi_1$
      and $\pi$, there is a proof $\Pi'$ for sequent
      $[[G1; B; G2; I; G3 |-l A]]$ s.t. $c(\Pi') \leq |X|$. Therefore, the
      proof $\Pi$ can be constructed as follows with
      $c(\Pi) = c(\Pi') \leq |X|$.
      \begin{center}
        \scriptsize
        \begin{math}
          $$\mprset{flushleft}
          \inferrule* [right={\tiny Gl}] {
            {
              \begin{array}{c}
                \Pi' \\
                {[[G1; B; G2; I; G3 |-l A]]}
              \end{array}
            }
          }{[[G1; Gf B; G2; I; G3 |-l A]]}
        \end{math}
      \end{center}

\item Case 4:
      \begin{center}
        \scriptsize
        \begin{math}
          \begin{array}{c}
            \Pi_1 \\
            {[[D |-l B]]}
          \end{array}
        \end{math}
        \qquad\qquad
        $\Pi_2$:
        \begin{math}
          $$\mprset{flushleft}
          \inferrule* [right={\tiny Gl}] {
            {
              \begin{array}{c}
                \pi \\
                {[[G1; C; G2; B; G3 |-l A]]}
              \end{array}
            }
          }{[[G1; Gf C; G2; B; G3 |-l A]]}
        \end{math}
      \end{center}
      By assumption, $c(\Pi_1),c(\Pi_2)\leq |B|$. By induction on $\Pi_1$
      and $\pi$, there is a proof $\Pi'$ for sequent
      $[[G1; C; G2; D; G3 |-l A]]$ s.t. $c(\Pi') \leq |B|$. Therefore, the
      proof $\Pi$ can be constructed as follows with
      $c(\Pi) = c(\Pi') \leq |B|$.
      \begin{center}
        \scriptsize
        \begin{math}
          $$\mprset{flushleft}
          \inferrule* [right={\tiny Gl}] {
            {
              \begin{array}{c}
                \Pi' \\
                {[[G1; C; G2; D; G3 |-l A]]}
              \end{array}
            }
          }{[[G1; Gf C; G2; D; G3 |-l A]]}
        \end{math}
      \end{center}

\end{itemize}



%--------------------------------------------------
%--------------------------------------------------
\section{Proof For Lemma~\ref{lem:monoidal-monad}}
\label{app:monoidal-monad}

Let $(\cat{C},\cat{L},F,G,\eta,\varepsilon)$ be a LAM. We define the monad
$(T,\eta:id_\cat{C}\rightarrow T,\mu:T^2\rightarrow T)$ on $\cat{C}$ as
$T=GF$, $\eta_X:X\rightarrow GFX$, and
$\mu_X=G\varepsilon_{FX}:GFGFX\rightarrow GFX$. Since $(F,\m{})$ and
$(G,\n{})$ are monoidal functors, we have
$$\t{X,Y}=G\m{X,Y}\circ\n{FX,FY}:TX\otimes TY\rightarrow T(X\otimes Y) \qquad\mbox{and}\qquad\t{I}=G\m{I}\circ\n{I'}:I\rightarrow TI.$$
The monad $T$ being monoidal means:
\begin{enumerate}
\item $T$ is a monoidal functor, i.e. the following diagrams commute:
      \begin{mathpar}
      \bfig
        \hSquares/->`->`->``->`->`->/<400>[
          (TX\otimes TY)\otimes TZ`TX\otimes(TY\otimes TZ)`TX\otimes T(Y\otimes Z)`
          T(X\otimes Y)\otimes TZ`T((X\otimes Y)\otimes Z)`T(X\otimes(Y\otimes Z));
          \alpha_{TX,TY,TZ}`id_{TX}\otimes\t{Y,Z}`\t{X,Y}\otimes id_{TZ}``
          \t{X,Y\otimes Z}`\t{X\otimes Y,Z}`T\alpha_{X,Y,Z}]
        \morphism(1300,200)//<0,0>[`;(1)]
      \efig
      \and
      \bfig
        \square/->`->`<-`->/<600,400>[
          I\otimes TX`TX`TI\otimes TX`T(I\otimes X);
          \lambda_{TX}`\t{I}\otimes id_{TX}`T\lambda_X`\t{I,X}]
        \morphism(350,200)//<0,0>[`;(2)]
      \efig
      \and
      \bfig
        \square/->`->`<-`->/<600,400>[
          TX\otimes I`TX`TX\otimes TI`T(X\otimes I);
          \rho_{TX}`id_{TX}\otimes\t{I}`T\rho_X`\t{X,I}]
        \morphism(350,200)//<0,0>[`;(3)]
      \efig
      \end{mathpar}
      We write $GF$ instead of $T$ in the proof for clarity. \\
      By replacing $\t{X,Y}$ with its definition, diagram (1) above
      commutes by the following commutative diagram, in which the two
      hexagons commute because $G$ and $F$ are monoidal functors, and the
      two quadrilaterals commute by the naturality of $\n{}$.
      \begin{mathpar}
      \bfig
        \iiixiii/->`->`->``->```->`<-`->``/<1400,400>[
          (GFX\otimes GFY)\otimes GFZ`GFX\otimes(GFY\otimes GFZ)`GFX\otimes G(FY\tri FZ)`
          G(FX\tri FY)\otimes GFZ`G(FX\tri(FY\tri FZ))`GFX\otimes GF(Y\otimes Z)`
          GF(X\otimes Y)\otimes GFZ`G((FX\tri FY)\tri FZ)`G(FX\tri F(Y\otimes Z));
          \alpha_{GFX,GFY,GFZ}`id_{GFX}\otimes\n{FY,FZ}`\n{FX,FY}\otimes id_{GFZ}``
          id_{GFX}\otimes G\m{Y,Z}```G\m{X,Y}\otimes id_{GFZ}`G\alpha'_{FX,FY,FZ}`
          \n{FX,F(Y\otimes Z)}``]
        \morphism(2800,800)|m|<-1400,-400>[
          GFX\otimes G(FY\tri FZ)`G(FX\tri(FY\tri FZ));\n{FX,FY\tri FZ}]
        \morphism(0,400)|m|<1400,-400>[
          G(FX\tri FY)\otimes GFZ`G((FX\tri FY)\tri FZ);\n{FX\tri FY,FZ}]
        \morphism(1400,400)|m|<1400,-400>[
          G(FX\tri(FY\tri FZ))`G(FX\tri F(Y\otimes Z));G(id_{FX}\tri\m{Y,Z})]
        \ptriangle(0,-400)|mlm|/`->`->/<1400,400>[
          GF(X\otimes Y)\otimes GFZ`G((FX\tri FY)\tri FZ)`G(F(X\otimes Y)\tri FZ);
          `\n{F(X\otimes Y),FZ}`G(\m{X,Y}\otimes id_{FZ})]
        \morphism(0,-400)|b|<1400,0>[
          G(F(X\otimes Y)\tri FZ)`GF((X\otimes Y)\otimes Z);G\m{X\otimes Y,Z}]
        \dtriangle(1400,-400)|mrb|/`->`->/<1400,400>[
          G(FX\tri F(Y\otimes Z))`GF((X\otimes Y)\otimes Z)`GF(X\otimes(Y\otimes Z));
          `G\m{X,Y\otimes Z}`GF\alpha_{X,Y,Z}]
      \efig
      \end{mathpar}
      Diagram (2) commutes by the following commutative diagrams, in which
      the top quadrilateral commutes because $G$ is monoidal, the right
      quadrilateral commutes because $F$ is monoidal, and the left square
      commutes by the naturality of $\n{}$.
      \begin{mathpar}
      \bfig
        \ptriangle/->`->`/<1600,400>[
          I\otimes GFX`GFX`GI'\otimes GFX;\lambda_{GFX}`\n{I'}\otimes id_{GFX}`]
        \square(0,-400)|lmmb|<800,400>[
          GI'\otimes GFX`G(I'\tri FX)`GFI\otimes GFX`G(FI\tri FX);
          \n{I',FX}`G\m{I}\otimes id_{GFX}`G(\m{I}\tri id_{FX})`\n{FI,FX}]
        \morphism(800,0)|m|<800,400>[G(I'\tri FX)`GFX;G\lambda'_{FX}]
        \dtriangle(800,-400)/`<-`->/<800,800>[
          GFX`G(FI\tri FX)`GF(I\otimes X);
          `GF\lambda_X`G\m{I,X}]
      \efig
      \end{mathpar}
      Similarly, diagram (3) commutes as follows:
      \begin{mathpar}
      \bfig
        \ptriangle/->`->`/<1600,400>[
          GFX\otimes I`GFX`GFX\otimes GI';\rho_{GFX}`id_{GFX}\otimes\n{I'}`]
        \square(0,-400)|lmmb|<800,400>[
          GFX\otimes GI'`G(FX\tri I')`GFX\otimes GFI`G(FX\tri FI);
          \n{FX,I'}`id_{GFX}\otimes G\m{I}`G(id_{FX}\otimes\m{I})`\n{FX,FI}]
        \morphism(800,0)|m|<800,400>[G(FX\tri I')`GFX;G\rho'_{FX}]
        \dtriangle(800,-400)/`<-`->/<800,800>[
          GFX`G(FX\tri FI)`GF(X\otimes I);
          `GF\rho_X`G\m{X,I}]
      \efig
      \end{mathpar}
\item $\eta$ is a monoidal natural transformation. In fact, since $\eta$
      is the unit of the monoidal adjunction, $\eta$ is monoidal by
      definition and thus the following two diagrams commute.
      \begin{mathpar}
      \bfig
        \square/=`->`->`->/<600,400>[
          X\otimes Y`X\otimes Y`TX\otimes TY`T(X\otimes Y);
          `\eta_X\otimes\eta_Y`\eta_{X\otimes Y}`\t{X,Y}]
      \efig
      \and
      \bfig
        \Vtriangle/->`=`<-/<400,400>[I`TI`I;\eta_I``\t{I}]
      \efig
      \end{mathpar}
\item $\mu$ is a monoidal natural transformation. It is obvious that since
      $\mu=G\varepsilon_{FA}$ and $\varepsilon$ is monoidal, so is $\mu$.
      Thus the following diagrams commute.
      \begin{mathpar}
      \bfig
        \square/`->`->`->/<1500,400>[
          T^2X\otimes T^2Y`T^2(X\otimes Y)`TX\otimes TY`T(X\otimes Y);
          `\mu_X\otimes\mu_Y`\mu_{X\otimes Y}`\t{X,Y}]
        \morphism(0,400)<800,0>[T^2X\otimes T^2Y`T(TX\otimes TY);\t{TX,TY}]
        \morphism(800,400)<700,0>[T(TX\otimes TY)`T^2(X\otimes Y);T\t{X,Y}]
      \efig
      \and
      \bfig
        \square/->`<-`<-`<-/<400,400>[T^2I`TI`TI`I;\mu_I`T\t{I}`\t{I}`\t{I}]
      \efig
      \end{mathpar}
\end{enumerate}



%--------------------------------------------------
%--------------------------------------------------
\section{Proof For Lemma~\ref{lem:strong-monad}}
\label{app:strong-monad}

\begin{definition}
\label{def:strong-monad}
Let $(\cat{M},\tri,I,\alpha,\lambda,\rho)$ be a monoidal category and
$(T,\eta,\mu)$ be a monad on $\cat{M}$. $T$ is a \textbf{strong monad} if
there is natural transformation $\tau$, called the \textbf{tensorial
strength}, with components \\
$\tau_{A,B}:A\tri TB\rightarrow T(A\tri B)$ such that the following
diagrams commute:
\begin{mathpar}
\bfig
  \Vtriangle<400,400>[I\tri TA`T(I\tri A)`TA;\tau_{I,A}`\lambda_{TA}`T\lambda_A]
\efig
\and
\bfig
  \Vtriangle<400,400>[
    A\tri B`A\tri TB`T(A\tri B);id_A\tri\eta_B`\eta_{A\tri B}`\tau_{A,B}]
\efig
\and
\bfig
  \square/->`->`->`/<1800,400>[
    (A\tri B)\tri TC`T((A\tri B)\tri C)`
    A\tri(B\tri TC)`T(A\tri(B\tri C));
    \tau_{A\tri B,C}`\alpha_{A,B,TC}`T\alpha_{A,B,C}`]
  \morphism<900,0>[A\tri(B\tri TC)`A\tri T(B\tri C);id_A\tri\tau_{B,C}]
  \morphism(900,0)<900,0>[A\tri T(B\tri C)`T(A\tri(B\tri C));\tau_{A,B\tri C}]  \efig
\and
\bfig
  \square/`->`->`->/<1400,400>[
    A\tri T^2B`T^2(A\tri B)`A\tri TB`T(A\tri B);
    `id_A\tri\mu_B`\mu_{A\tri B}`\tau_{A,B}]
  \morphism(0,400)<700,0>[A\tri T^2B`T(A\tri TB);\tau_{A,TB}]
  \morphism(700,400)<700,0>[T(A\tri TB)`T^2(A\tri B);T\tau_{A,B}]
\efig
\end{mathpar}
\end{definition}
\noindent
The proof for Lemma~\ref{lem:strong-monad} goes as follows.
\noindent
Let $(\cat{C},\cat{L},F,G,\eta,\varepsilon)$ be a LAM, where
$(\cat{C},\otimes,I,\alpha,\lambda,\rho)$ is symmetric monoidal closed,
and \\ $(\cat{L},\tri,I',\alpha',\lambda',\rho')$ is Lambek. In
Lemma~\ref{lem:monoidal-monad}, we have proved that the monad
$(T=GF,\eta,\mu)$ is monoidal with the natural transformation
$\t{X,Y}:TX\otimes TY\rightarrow T(X\otimes Y)$ and the morphism
$\t{I}:I\rightarrow TI$.
\noindent
We define the tensorial strength
$\tau_{X,Y}:X\otimes TY\rightarrow T(X\otimes Y)$ as
$$\tau_{X,Y}=\t{X,Y}\circ(\eta_X\otimes id_{TY}).$$
Since $\eta$ is a monoidal natural transformation, we have
$\eta_I=G\m{I}\circ\n{I'}$, and thus $\eta_I=\t{I}$. The following diagram
commutes because $T$ is monoidal, where the composition
$\t{I,X}\circ(\t{I}\otimes id_{TX})$ is the definition of $\tau_{I,X}$. So
the first triangle in Definition~\ref{def:strong-monad} commutes.
\begin{mathpar}
\bfig
  \square/->`->`->`<-/<600,400>[
    I\otimes TX`TI\otimes TX`TX`T(I\otimes X);
    \t{I}\otimes id_{TX}`\lambda_{TX}`\t{I,X}`T\lambda_X]
\efig
\end{mathpar}
Similarly, by using the definition of $\tau$, the the second triangle in the definition is
equivalent to the following diagram, which commutes because $\eta$ is a monoidal natural
transformation:
\begin{mathpar}
\bfig
  \square/->`->`->`<-/<600,400>[
    X\otimes Y`X\otimes TY`T(X\otimes Y)`TX\otimes TY;
    id_X\otimes\eta_Y`\eta_{X\otimes Y}`\eta_X\otimes id_{TY}`\t{X,Y}]
  \morphism(0,400)|m|<600,-400>[X\otimes Y`TX\otimes TY;\eta_X\otimes\eta_Y]
\efig
\end{mathpar}
The first pentagon in the definition commutes by the following commutative diagrams, because
$\eta$ and $\alpha$ are natural transformations and $T$ is monoidal:
\begin{mathpar}
\bfig
  \qtriangle|amm|/->`->`<-/<1000,400>[
    (X\otimes Y)\otimes TZ`T(X\otimes Y)\otimes TZ`(TX\otimes TY)\otimes TZ;
    \eta_{X\otimes Y}\otimes id_{TZ}`
    (\eta_X\otimes\eta_Y)\otimes id_{TZ}`
    \t{X,Y}\otimes id_{TZ}]
  \morphism(0,400)<0,-400>[(X\otimes Y)\otimes TZ`X\otimes(Y\otimes TZ);\alpha_{X,Y,TZ}]
  \morphism(1000,0)|m|<0,-400>[
    (TX\otimes TY)\otimes TZ`TX\otimes(TY\otimes TZ);\alpha_{TX,TY,TZ}]
  \Dtriangle(0,-800)|lmm|/->`->`<-/<1000,400>[
    X\otimes(Y\otimes TZ)`TX\otimes(TY\otimes TZ)`X\otimes(TY\otimes TZ);
    id_X\otimes(\eta_Y\otimes id_{TZ})`
    \eta_X\otimes(\eta_Y\otimes id_{TZ})`
    \eta_X\otimes id_{TY\otimes TZ}]
  \morphism(0,-800)|b|<1000,0>[
    X\otimes(TY\otimes TZ)`X\otimes T(Y\otimes Z);id_X\otimes\t{Y,Z}]
  \qtriangle(1000,0)|amr|/->``->/<1000,400>[
    T(X\otimes Y)\otimes TZ`T((X\otimes Y)\otimes Z)`T(X\otimes(Y\otimes Z));
    \t{X\otimes Y,Z}``T\alpha_{X,Y,Z}]
  \morphism(2000,-800)<0,800>[
    TX\otimes T(Y\otimes Z)`T(X\otimes(Y\otimes Z));\t{X,Y\otimes Z}]
  \btriangle(1000,-800)|mmb|/`->`->/<1000,400>[
    TX\otimes(TY\otimes TZ)`X\otimes T(Y\otimes Z)`TX\otimes T(Y\otimes Z);
    `id_{TX}\otimes\t{Y,Z}`\eta_X\otimes id_{T(Y\otimes Z)}]
\efig
\end{mathpar}
The last diagram in the definition commutes by the following commutative diagram, because
$T$ is a monad, $\t{}$ is a natural transformation, and $\mu$ is a monoidal natural
transformation:
\begin{mathpar}
\bfig
  \ptriangle/->`->`/<700,400>[
    X\otimes T^2Y`TX\otimes T^2Y`X\otimes TY;\eta_X\otimes id_{T^2Y}`id_X\otimes\mu_Y`]
  \btriangle(0,-400)/->``->/<700,400>[
    X\otimes TY`TX\otimes TY`T(X\otimes Y);\eta_X\otimes id_{TY}``\t{X,Y}]
  \morphism(700,400)|m|<-700,-800>[TX\otimes T^2Y`TX\otimes TY;id_{TX}\otimes\mu_Y]
  \morphism(700,0)|m|<-700,-400>[TX\otimes T^2Y`TX\otimes TY;id_{TX}\otimes\mu_Y]
  \qtriangle(700,0)/->``->/<1800,400>[
    TX\otimes T^2Y`T(X\otimes TY)`T(TX\otimes TY);\t{X,TY}``T(\eta_X\otimes id_{TY})]
  \btriangle(700,0)|mmm|/=`->`<-/<900,400>[
    TX\otimes T^2Y`TX\otimes T^2Y`T^2X\otimes T^2Y;
    `T\eta_X\otimes id_{T^2Y}`\mu_X\otimes id_{T^2Y}]
  \morphism(1600,0)|m|<900,0>[T^2X\otimes T^2Y`T(TX\otimes TY);\t{TX,TY}]
  \morphism(1600,0)|m|<-1600,-400>[T^2X\otimes T^2Y`TX\otimes TY;\mu_X\otimes\mu_Y]
  \dtriangle(700,-400)/`->`<-/<1800,400>[
    T(TX\otimes TY)`T(X\otimes Y)`T^2(X\otimes Y);`T\t{X,Y}`\mu_{X\otimes Y}]
\efig
\end{mathpar}



