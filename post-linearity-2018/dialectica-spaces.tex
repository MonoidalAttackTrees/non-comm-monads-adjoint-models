\newcommand{\Set}{\mathsf{Set}}
\newcommand{\Dial}[2]{\mathsf{Dial}_{#1}(#2)}

In this section we give a different categorical model in terms of
dialectica categories; which are a sound and complete categorical
model of the Lambek Calculus as was shown by de Paiva and Eades
\cite{dePaiva2018}. This section is largely the same as the
corresponding section de Paiva and Eades give, but with some
modifications to their definition of biclosed posets with exchange
(see Definition~\ref{def:biclosed-exchange}).  However, we try to make
this section as self contained as possible.

Dialectica categories were first introduced by de Paiva as a
categorification of G\"odel's Dialectica interpretation
\cite{depaiva1990}.  Dialectica categories were one of the first sound
categorical models of intuitionistic linear logic with linear
modalities.  We show in this section that they can be adapted to
become a sound and complete model for CNC logic, with both the
exchange and of-course modalities.  Due to the complexities of working
with dialectica categories we have formally verified\footnote{The
  complete formalization can be found online at
  \url{https://github.com/MonoidalAttackTrees/non-comm-monads-adjoint-models/tree/master/dialectica-formalization}.}
this section in the proof assistant Agda~\cite{bove2009}.

First, we define the notion of a biclosed poset.  These are used to
control the definition of morphisms in the dialectica model.
\begin{definition}
  \label{def:biclosed-poset}
  Suppose $(M, \leq, \circ, e)$ is an ordered non-commutative monoid.
  If there exists a largest $x \in M$ such that $a \circ x \leq b$ for
  any $a, b \in M$, then we denote $x$ by $a \lto b$ and called it
  the \textbf{left-pseudocomplement} of $a$ w.r.t $b$.  Additionally,
  if there exists a largest $x \in M$ such that $x \circ a \leq b$ for
  any $a, b \in M$, then we denote $x$ by $b \rto a$ and called it
  the \textbf{right-pseudocomplement} of $a$ w.r.t $b$.

  A \textbf{biclosed poset}, $(M, \leq, \circ, e, \lto, \rto)$, is an
  ordered non-commutative monoid, $(M, \leq, \circ, e)$, such that $a
  \lto b$ and $b \rto a$ exist for any $a,b \in M$.
\end{definition}
Now using the previous definition we define dialectica Lambek spaces.
\begin{definition}
  \label{def:dialectica-lambek-spaces}
  Suppose $(M, \leq, \circ, e, \lto, \rto)$ is a biclosed poset. Then
  we define the category of \textbf{dialectica Lambek spaces},
  $\mathsf{Dial}_M(\Set)$, as follows:
  \begin{itemize}
  \item[-] objects, or dialectica Lambek spaces, are triples $(U, X,
    \alpha)$ where $U$ and $X$ are sets, and $\alpha : U \times X \mto
    M$ is a generalized relation over $M$, and

  \item[-] maps that are pairs $(f, F) : (U , X, \alpha) \mto (V , Y ,
    \beta)$ where $f : U \mto V$, and $F : Y \mto X$ are functions
    such that the weak adjointness condition
    $\forall u \in U.\forall y \in Y. \alpha(u , F(y)) \leq \beta(f(u), y)$
    holds.
  \end{itemize}
\end{definition}
Notice that the biclosed poset is used here as the target of the
relations in objects, but also as providing the order  relation in the weak adjoint condition on morphisms.  This will allow the structure of the biclosed
poset to lift up into $\Dial{M}{\Set}$.

We will show that $\Dial{M}{\Set}$ is a model of the Lambek Calculus
with modalities.  First, we must show that $\Dial{M}{\Set}$ is
monoidal biclosed.
\begin{definition}
  \label{def:dial-monoidal-structure}
  Suppose $(U, X, \alpha)$ and $(V, Y, \beta)$ are two objects of
  $\Dial{M}{\Set}$. Then their tensor product is defined as follows:
  \[ \small
  (U, X, \alpha) \rhd (V, Y, \beta) = (U \times V, (V \to X) \times (U \to Y), \alpha \rhd \beta)
  \]
  where $- \to -$ is the function space from $\Set$, and $(\alpha
  \rhd \beta)((u, v), (f, g)) = \alpha(u, f(v)) \circ \beta(g(u), v)$.

  \ \\
  \noindent
  The unit of the above tensor product is defined as follows:
  \[ \small
  I = (\top , \top , \iota)
  \]
  where $\top$ is the initial object in $\Set$, and $\iota(*,*) = e$.
\end{definition}

\noindent
It follows from de Paiva and Eades \cite{dePaiva2018} that this does
indeed define a monoidal tensor product, but take note of the fact
that this tensor product is indeed non-commutative, because the
non-commutative multiplication of the biclosed poset is used to define
the relation of the tensor product.

The tensor product has two right adjoints making $\Dial{M}{\Set}$
biclosed.
\begin{definition}
  \label{def:dial-is-biclosed}
  Suppose $(U, X, \alpha)$ and $(V, Y, \beta)$ are two objects of
  $\Dial{M}{\Set}$. Then two internal-homs can be defined as follows:
  \[ \small
  \begin{array}{lll}
    (U, X, \alpha) \lto (V, Y, \beta) = ((U \to V) \times (Y \to X), U \times Y, \alpha \lto \beta)\\
    (V, Y, \beta) \rto (U, X, \alpha) = ((U \to V) \times (Y \to X), U \times Y, \alpha \rto \beta)\\
  \end{array}
  \]
\end{definition}
\noindent
It is straightforward to show that the typical bijections defining the
corresponding adjunctions hold; see de Paiva and Eades for the details
\cite{dePaiva2018}.

We now extend $\Dial{M}{\Set}$ with two modalities: the usual
modality, of-course, denoted $!A$, and the exchange modality denoted
$\xi A$.  However, we must first extended biclosed posets to
include an exchange operation.
\begin{definition}
  \label{def:biclosed-exchange}
  A \textbf{biclosed poset with exchange} is a biclosed poset $(M,
  \leq, \circ, e, \lto, \rto)$ equipped with an unary operation
  $\xi : M \to M$ satisfying the following:
  \[ \small
  \setlength{\arraycolsep}{4px}
  \begin{array}{lll}
    \begin{array}{lll}
    \text{(Compatibility)} & a \leq b \text{ implies } \xi a \leq \xi b \text{ for all } a,b,c \in M\\
    \text{(Minimality)} & \xi a \leq a \text{ for all } a \in M\\    
  \end{array}
  &
  \begin{array}{lll}
    \text{(Duplication)} & \xi a \leq \xi\xi a \text{ for all } a \in M\\
    \text{(Exchange)} & (\xi a \circ \xi b) \leq (\xi b \circ \xi a) \text{ for all } a, b \in M\\
  \end{array}
  \end{array}
  \]
\end{definition}
\noindent
This definition is where the construction given here departs from the
definition of biclosed posets with exchange given by de Paiva and
Eades \cite{dePaiva2018}.

We can now define the two modalities in $\Dial{M}{\Set}$ where $M$ is
a biclosed poset with exchange.
\begin{definition}
  \label{def:modalities-dial}
  Suppose $(U, X, \alpha)$ is an object of $\Dial{M}{\Set}$ where $M$
  is a biclosed poset with exchange. Then the \textbf{of-course} and
  \textbf{exchange} modalities can be defined as 
  $! (U, X, \alpha) = (U, U \to X^*, !\alpha)$ and
  $\xi (U, X, \alpha) = (U, X, \xi \alpha)$
  where $X^*$ is the free commutative monoid on $X$, $(!\alpha)(u, f)
  = \alpha(u, x_1) \circ \cdots \circ \alpha(u, x_i)$ for $f(u) =
  (x_1, \ldots, x_i)$, and $(\xi \alpha)(u, x) = \xi (\alpha(u,
  x))$.
\end{definition}
This definition highlights a fundamental difference between the two
modalities.  The definition of the exchange modality relies on an
extension of biclosed posets with essentially the exchange modality in
the category of posets.  However, the of-course modality is defined by
the structure already present in $\Dial{M}{\Set}$, specifically, the
structure of $\Set$.

Both of the modalities have the structure of a comonad.  That is,
there are monoidal natural transformations $\varepsilon_! : !A \mto
A$, $\varepsilon_\xi : \xi A \mto A$, $\delta_! : !A \mto !!A$,
and $\delta_\xi : \xi A \mto \xi\xi A$ which satisfy the
appropriate diagrams; see the formalization for the full
proofs. Furthermore, these comonads come equipped with arrows $w : !A
\mto I$, $d : !A \mto !A \otimes !A$, $\e{A,B} : \xi A \otimes \xi B \mto \xi B
\otimes \xi A$. 

Finally, using the fact that $\Dial{M}{\Set}$ for any biclosed poset
is essentially a non-commutative formalization of Bierman's linear
categories \cite{Bierman:1994} we can use Benton's construction of an
LNL model from a linear category to obtain a LAM model, and hence,
obtain the following.
\begin{theorem}
  \label{theorem:sound-dial-exchange-!}
  Suppose $M$ is a biclosed poset with exchange.  Then
  $\Dial{M}{\Set}$ is a sound model for CNC logic.
\end{theorem}


