\documentclass[submission,copyright,creativecommons]{eptcs}

\providecommand{\event}{LINEARITY 2018} % Name of the event you are submitting to
\usepackage{underscore}           % Only needed if you use pdflatex.

\usepackage{amssymb,amsmath,amsthm}
\usepackage{cmll}
\usepackage{txfonts}
\usepackage{graphicx}
\usepackage{stmaryrd}
\usepackage{todonotes}
\usepackage{mathpartir}
\usepackage{hyperref}
\usepackage{mdframed}
\usepackage[barr]{xy}
\usepackage{comment}
\usepackage{graphicx}
\usepackage[inline]{enumitem}
\usepackage{bbm}
\usepackage{array}

\usepackage{caption}
\captionsetup[figure]{name=Figure}

% generated by Ott 0.25 from: Elle-ND/Elle-ND.ott
\newcommand{\NDdrule}[4][]{{\displaystyle\frac{\begin{array}{l}#2\end{array}}{#3}\quad\NDdrulename{#4}}}
\newcommand{\NDusedrule}[1]{\[#1\]}
\newcommand{\NDpremise}[1]{ #1 \\}
\newenvironment{NDdefnblock}[3][]{ \framebox{\mbox{#2}} \quad #3 \\[0pt]}{}
\newenvironment{NDfundefnblock}[3][]{ \framebox{\mbox{#2}} \quad #3 \\[0pt]\begin{displaymath}\begin{array}{l}}{\end{array}\end{displaymath}}
\newcommand{\NDfunclause}[2]{ #1 \equiv #2 \\}
\newcommand{\NDnt}[1]{\mathit{#1}}
\newcommand{\NDmv}[1]{\mathit{#1}}
\newcommand{\NDkw}[1]{\mathbf{#1}}
\newcommand{\NDsym}[1]{#1}
\newcommand{\NDcom}[1]{\text{#1}}
\newcommand{\NDdrulename}[1]{\textsc{#1}}
\newcommand{\NDcomplu}[5]{\overline{#1}^{\,#2\in #3 #4 #5}}
\newcommand{\NDcompu}[3]{\overline{#1}^{\,#2<#3}}
\newcommand{\NDcomp}[2]{\overline{#1}^{\,#2}}
\newcommand{\NDgrammartabular}[1]{\begin{supertabular}{llcllllll}#1\end{supertabular}}
\newcommand{\NDmetavartabular}[1]{\begin{supertabular}{ll}#1\end{supertabular}}
\newcommand{\NDrulehead}[3]{$#1$ & & $#2$ & & & \multicolumn{2}{l}{#3}}
\newcommand{\NDprodline}[6]{& & $#1$ & $#2$ & $#3 #4$ & $#5$ & $#6$}
\newcommand{\NDfirstprodline}[6]{\NDprodline{#1}{#2}{#3}{#4}{#5}{#6}}
\newcommand{\NDlongprodline}[2]{& & $#1$ & \multicolumn{4}{l}{$#2$}}
\newcommand{\NDfirstlongprodline}[2]{\NDlongprodline{#1}{#2}}
\newcommand{\NDbindspecprodline}[6]{\NDprodline{#1}{#2}{#3}{#4}{#5}{#6}}
\newcommand{\NDprodnewline}{\\}
\newcommand{\NDinterrule}{\\[5.0mm]}
\newcommand{\NDafterlastrule}{\\}
\newcommand{\NDmetavars}{
\NDmetavartabular{
 $ \NDmv{vars} ,\, \NDmv{n} ,\, \NDmv{a} ,\, \NDmv{x} ,\, \NDmv{y} ,\, \NDmv{z} ,\, \NDmv{w} ,\, \NDmv{m} ,\, \NDmv{o} $ &  \\
 $ \NDmv{ivar} ,\, \NDmv{i} ,\, \NDmv{k} ,\, \NDmv{j} ,\, \NDmv{l} $ &  \\
 $ \NDmv{const} ,\, \NDmv{b} $ &  \\
}}

\newcommand{\NDA}{
\NDrulehead{\NDnt{A}  ,\ \NDnt{B}  ,\ \NDnt{C}  ,\ D}{::=}{}\NDprodnewline
\NDfirstprodline{|}{ \mathsf{B} }{}{}{}{}\NDprodnewline
\NDprodline{|}{ \mathsf{Unit} }{}{}{}{}\NDprodnewline
\NDprodline{|}{\NDnt{A}  \triangleright  \NDnt{B}}{}{}{}{}\NDprodnewline
\NDprodline{|}{\NDnt{A}  \rightharpoonup  \NDnt{B}}{}{}{}{}\NDprodnewline
\NDprodline{|}{\NDnt{A}  \leftharpoonup  \NDnt{B}}{}{}{}{}\NDprodnewline
\NDprodline{|}{\NDsym{(}  \NDnt{A}  \NDsym{)}} {\textsf{M}}{}{}{}\NDprodnewline
\NDprodline{|}{ \NDnt{A} } {\textsf{M}}{}{}{}\NDprodnewline
\NDprodline{|}{ \mathsf{F} \NDnt{X} }{}{}{}{}}

\newcommand{\NDW}{
\NDrulehead{\NDnt{W}  ,\ \NDnt{X}  ,\ \NDnt{Y}  ,\ \NDnt{Z}}{::=}{}\NDprodnewline
\NDfirstprodline{|}{ \mathsf{B} }{}{}{}{}\NDprodnewline
\NDprodline{|}{ \mathsf{Unit} }{}{}{}{}\NDprodnewline
\NDprodline{|}{\NDnt{X}  \otimes  \NDnt{Y}}{}{}{}{}\NDprodnewline
\NDprodline{|}{\NDnt{X}  \multimap  \NDnt{Y}}{}{}{}{}\NDprodnewline
\NDprodline{|}{\NDsym{(}  \NDnt{X}  \NDsym{)}} {\textsf{M}}{}{}{}\NDprodnewline
\NDprodline{|}{ \NDnt{X} } {\textsf{M}}{}{}{}\NDprodnewline
\NDprodline{|}{ \mathsf{G} \NDnt{A} }{}{}{}{}}

\newcommand{\NDT}{
\NDrulehead{\NDnt{T}}{::=}{}\NDprodnewline
\NDfirstprodline{|}{\NDnt{A}}{}{}{}{}\NDprodnewline
\NDprodline{|}{\NDnt{X}}{}{}{}{}}

\newcommand{\NDp}{
\NDrulehead{\NDnt{p}  ,\ \NDnt{q}}{::=}{}\NDprodnewline
\NDfirstprodline{|}{ \star }{}{}{}{}\NDprodnewline
\NDprodline{|}{\NDmv{x}}{}{}{}{}\NDprodnewline
\NDprodline{|}{ \mathsf{triv} }{}{}{}{}\NDprodnewline
\NDprodline{|}{ \mathsf{triv} }{}{}{}{}\NDprodnewline
\NDprodline{|}{\NDnt{p}  \otimes  \NDnt{p'}}{}{}{}{}\NDprodnewline
\NDprodline{|}{\NDnt{p}  \triangleright  \NDnt{p'}}{}{}{}{}\NDprodnewline
\NDprodline{|}{ \mathsf{F}\, \NDnt{p} }{}{}{}{}\NDprodnewline
\NDprodline{|}{ \mathsf{G}\, \NDnt{p} }{}{}{}{}}

\newcommand{\NDs}{
\NDrulehead{\NDnt{s}}{::=}{}\NDprodnewline
\NDfirstprodline{|}{\NDmv{x}}{}{}{}{}\NDprodnewline
\NDprodline{|}{\NDmv{b}}{}{}{}{}\NDprodnewline
\NDprodline{|}{ \mathsf{triv} }{}{}{}{}\NDprodnewline
\NDprodline{|}{ \mathsf{let}\, \NDnt{s_{{\mathrm{1}}}}  :  \NDnt{A} \,\mathsf{be}\, \NDnt{p} \,\mathsf{in}\, \NDnt{s_{{\mathrm{2}}}} }{}{}{}{}\NDprodnewline
\NDprodline{|}{ \mathsf{let}\, \NDnt{t}  :  \NDnt{X} \,\mathsf{be}\, \NDnt{p} \,\mathsf{in}\, \NDnt{s} }{}{}{}{}\NDprodnewline
\NDprodline{|}{\NDnt{s_{{\mathrm{1}}}}  \triangleright  \NDnt{s_{{\mathrm{2}}}}}{}{}{}{}\NDprodnewline
\NDprodline{|}{ \lambda_l  \NDmv{x}  :  \NDnt{A} . \NDnt{s} }{}{}{}{}\NDprodnewline
\NDprodline{|}{ \lambda_r  \NDmv{x}  :  \NDnt{A} . \NDnt{s} }{}{}{}{}\NDprodnewline
\NDprodline{|}{ \mathsf{app}_l\, \NDnt{s_{{\mathrm{1}}}} \, \NDnt{s_{{\mathrm{2}}}} }{}{}{}{}\NDprodnewline
\NDprodline{|}{ \mathsf{app}_r\, \NDnt{s_{{\mathrm{1}}}} \, \NDnt{s_{{\mathrm{2}}}} }{}{}{}{}\NDprodnewline
\NDprodline{|}{ \mathsf{derelict}\, \NDnt{t} }{}{}{}{}\NDprodnewline
\NDprodline{|}{ \mathsf{ex}\, \NDnt{s_{{\mathrm{1}}}} , \NDnt{s_{{\mathrm{2}}}} \,\mathsf{with}\, \NDmv{x_{{\mathrm{1}}}} , \NDmv{x_{{\mathrm{2}}}} \,\mathsf{in}\, \NDnt{s_{{\mathrm{3}}}} }{}{}{}{}\NDprodnewline
\NDprodline{|}{\NDsym{[}  \NDnt{s_{{\mathrm{1}}}}  \NDsym{/}  \NDmv{x}  \NDsym{]}  \NDnt{s_{{\mathrm{2}}}}} {\textsf{M}}{}{}{}\NDprodnewline
\NDprodline{|}{\NDsym{[}  \NDnt{t}  \NDsym{/}  \NDmv{x}  \NDsym{]}  \NDnt{s}} {\textsf{M}}{}{}{}\NDprodnewline
\NDprodline{|}{\NDsym{(}  \NDnt{s}  \NDsym{)}} {\textsf{S}}{}{}{}\NDprodnewline
\NDprodline{|}{ \NDnt{s} } {\textsf{M}}{}{}{}\NDprodnewline
\NDprodline{|}{ \mathsf{F} \NDnt{t} }{}{}{}{}}

\newcommand{\NDt}{
\NDrulehead{\NDnt{t}}{::=}{}\NDprodnewline
\NDfirstprodline{|}{\NDmv{x}}{}{}{}{}\NDprodnewline
\NDprodline{|}{\NDmv{b}}{}{}{}{}\NDprodnewline
\NDprodline{|}{ \mathsf{triv} }{}{}{}{}\NDprodnewline
\NDprodline{|}{ \mathsf{let}\, \NDnt{t_{{\mathrm{1}}}}  :  \NDnt{X} \,\mathsf{be}\, \NDnt{p} \,\mathsf{in}\, \NDnt{t_{{\mathrm{2}}}} }{}{}{}{}\NDprodnewline
\NDprodline{|}{\NDnt{t_{{\mathrm{1}}}}  \otimes  \NDnt{t_{{\mathrm{2}}}}}{}{}{}{}\NDprodnewline
\NDprodline{|}{ \lambda  \NDmv{x}  :  \NDnt{X} . \NDnt{t} }{}{}{}{}\NDprodnewline
\NDprodline{|}{ \NDnt{t_{{\mathrm{1}}}}   \NDnt{t_{{\mathrm{2}}}} }{}{}{}{}\NDprodnewline
\NDprodline{|}{ \mathsf{ex}\, \NDnt{t_{{\mathrm{1}}}} , \NDnt{t_{{\mathrm{2}}}} \,\mathsf{with}\, \NDmv{x_{{\mathrm{1}}}} , \NDmv{x_{{\mathrm{2}}}} \,\mathsf{in}\, \NDnt{t_{{\mathrm{3}}}} }{}{}{}{}\NDprodnewline
\NDprodline{|}{\NDsym{[}  \NDnt{t_{{\mathrm{1}}}}  \NDsym{/}  \NDmv{x}  \NDsym{]}  \NDnt{t_{{\mathrm{2}}}}} {\textsf{M}}{}{}{}\NDprodnewline
\NDprodline{|}{\NDsym{(}  \NDnt{t}  \NDsym{)}} {\textsf{S}}{}{}{}\NDprodnewline
\NDprodline{|}{\NDsym{h(}  \NDnt{t}  \NDsym{)}} {\textsf{M}}{}{}{}\NDprodnewline
\NDprodline{|}{ \mathsf{G}\, \NDnt{s} }{}{}{}{}}

\newcommand{\NDI}{
\NDrulehead{\Phi  ,\ \Psi}{::=}{}\NDprodnewline
\NDfirstprodline{|}{ \cdot }{}{}{}{}\NDprodnewline
\NDprodline{|}{\Phi_{{\mathrm{1}}}  \NDsym{,}  \Phi_{{\mathrm{2}}}}{}{}{}{}\NDprodnewline
\NDprodline{|}{\NDmv{x}  \NDsym{:}  \NDnt{X}}{}{}{}{}\NDprodnewline
\NDprodline{|}{\NDsym{(}  \Phi  \NDsym{)}} {\textsf{S}}{}{}{}}

\newcommand{\NDG}{
\NDrulehead{\Gamma  ,\ \Delta}{::=}{}\NDprodnewline
\NDfirstprodline{|}{ \cdot }{}{}{}{}\NDprodnewline
\NDprodline{|}{\NDmv{x}  \NDsym{:}  \NDnt{A}}{}{}{}{}\NDprodnewline
\NDprodline{|}{\Phi}{}{}{}{}\NDprodnewline
\NDprodline{|}{\Gamma_{{\mathrm{1}}}  \NDsym{;}  \Gamma_{{\mathrm{2}}}}{}{}{}{}\NDprodnewline
\NDprodline{|}{\NDsym{(}  \Gamma  \NDsym{)}} {\textsf{S}}{}{}{}}

\newcommand{\NDformula}{
\NDrulehead{\NDnt{formula}}{::=}{}\NDprodnewline
\NDfirstprodline{|}{\NDnt{judgement}}{}{}{}{}\NDprodnewline
\NDprodline{|}{ \NDnt{formula_{{\mathrm{1}}}}  \quad  \NDnt{formula_{{\mathrm{2}}}} } {\textsf{M}}{}{}{}\NDprodnewline
\NDprodline{|}{\NDnt{formula_{{\mathrm{1}}}} \, ... \, \NDnt{formula_{\NDmv{i}}}} {\textsf{M}}{}{}{}\NDprodnewline
\NDprodline{|}{ \NDnt{formula} } {\textsf{S}}{}{}{}\NDprodnewline
\NDprodline{|}{ \NDmv{x}  \not\in \mathsf{FV}( \NDnt{s} ) }{}{}{}{}\NDprodnewline
\NDprodline{|}{ \NDmv{x}  \not\in |  \Gamma ,  \Delta ,  \Psi  | }{}{}{}{}\NDprodnewline
\NDprodline{|}{ \NDmv{x}  \not\in |  \Gamma ,  \Delta  | }{}{}{}{}}

\newcommand{\NDterminals}{
\NDrulehead{\NDnt{terminals}}{::=}{}\NDprodnewline
\NDfirstprodline{|}{ \otimes }{}{}{}{}\NDprodnewline
\NDprodline{|}{ \triangleright }{}{}{}{}\NDprodnewline
\NDprodline{|}{ \circop{e} }{}{}{}{}\NDprodnewline
\NDprodline{|}{ \circop{w} }{}{}{}{}\NDprodnewline
\NDprodline{|}{ \circop{c} }{}{}{}{}\NDprodnewline
\NDprodline{|}{ \rightharpoonup }{}{}{}{}\NDprodnewline
\NDprodline{|}{ \leftharpoonup }{}{}{}{}\NDprodnewline
\NDprodline{|}{ \multimap }{}{}{}{}\NDprodnewline
\NDprodline{|}{ \vdash_\mathcal{C} }{}{}{}{}\NDprodnewline
\NDprodline{|}{ \vdash_\mathcal{L} }{}{}{}{}\NDprodnewline
\NDprodline{|}{ \leadsto_\beta }{}{}{}{}\NDprodnewline
\NDprodline{|}{ \leadsto_\mathsf{c} }{}{}{}{}}

\newcommand{\NDJtype}{
\NDrulehead{\NDnt{Jtype}}{::=}{}\NDprodnewline
\NDfirstprodline{|}{\Phi  \vdash_\mathcal{C}  \NDnt{t}  \NDsym{:}  \NDnt{X}}{}{}{}{}\NDprodnewline
\NDprodline{|}{\Gamma  \vdash_\mathcal{L}  \NDnt{s}  \NDsym{:}  \NDnt{A}}{}{}{}{}}

\newcommand{\NDReduction}{
\NDrulehead{\NDnt{Reduction}}{::=}{}\NDprodnewline
\NDfirstprodline{|}{\NDnt{t_{{\mathrm{1}}}}  \leadsto_\beta  \NDnt{t_{{\mathrm{2}}}}}{}{}{}{}\NDprodnewline
\NDprodline{|}{\NDnt{s_{{\mathrm{1}}}}  \leadsto_\beta  \NDnt{s_{{\mathrm{2}}}}}{}{}{}{}}

\newcommand{\NDCommuting}{
\NDrulehead{\NDnt{Commuting}}{::=}{}\NDprodnewline
\NDfirstprodline{|}{\NDnt{t_{{\mathrm{1}}}}  \leadsto_\mathsf{c}  \NDnt{t_{{\mathrm{2}}}}}{}{}{}{}\NDprodnewline
\NDprodline{|}{\NDnt{s_{{\mathrm{1}}}}  \leadsto_\mathsf{c}  \NDnt{s_{{\mathrm{2}}}}}{}{}{}{}}

\newcommand{\NDjudgement}{
\NDrulehead{\NDnt{judgement}}{::=}{}\NDprodnewline
\NDfirstprodline{|}{\NDnt{Jtype}}{}{}{}{}\NDprodnewline
\NDprodline{|}{\NDnt{Reduction}}{}{}{}{}\NDprodnewline
\NDprodline{|}{\NDnt{Commuting}}{}{}{}{}}

\newcommand{\NDuserXXsyntax}{
\NDrulehead{\NDnt{user\_syntax}}{::=}{}\NDprodnewline
\NDfirstprodline{|}{\NDmv{vars}}{}{}{}{}\NDprodnewline
\NDprodline{|}{\NDmv{ivar}}{}{}{}{}\NDprodnewline
\NDprodline{|}{\NDmv{const}}{}{}{}{}\NDprodnewline
\NDprodline{|}{\NDnt{A}}{}{}{}{}\NDprodnewline
\NDprodline{|}{\NDnt{W}}{}{}{}{}\NDprodnewline
\NDprodline{|}{\NDnt{T}}{}{}{}{}\NDprodnewline
\NDprodline{|}{\NDnt{p}}{}{}{}{}\NDprodnewline
\NDprodline{|}{\NDnt{s}}{}{}{}{}\NDprodnewline
\NDprodline{|}{\NDnt{t}}{}{}{}{}\NDprodnewline
\NDprodline{|}{\Phi}{}{}{}{}\NDprodnewline
\NDprodline{|}{\Gamma}{}{}{}{}\NDprodnewline
\NDprodline{|}{\NDnt{formula}}{}{}{}{}\NDprodnewline
\NDprodline{|}{\NDnt{terminals}}{}{}{}{}}

\newcommand{\NDgrammar}{\NDgrammartabular{
\NDA\NDinterrule
\NDW\NDinterrule
\NDT\NDinterrule
\NDp\NDinterrule
\NDs\NDinterrule
\NDt\NDinterrule
\NDI\NDinterrule
\NDG\NDinterrule
\NDformula\NDinterrule
\NDterminals\NDinterrule
\NDJtype\NDinterrule
\NDReduction\NDinterrule
\NDCommuting\NDinterrule
\NDjudgement\NDinterrule
\NDuserXXsyntax\NDafterlastrule
}}

% defnss
% defns Jtype
%% defn tty
\newcommand{\NDdruleTXXidName}[0]{\NDdrulename{T\_id}}
\newcommand{\NDdruleTXXid}[1]{\NDdrule[#1]{%
}{
\NDmv{x}  \NDsym{:}  \NDnt{X}  \vdash_\mathcal{C}  \NDmv{x}  \NDsym{:}  \NDnt{X}}{%
{\NDdruleTXXidName}{}%
}}


\newcommand{\NDdruleTXXunitIName}[0]{\NDdrulename{T\_unitI}}
\newcommand{\NDdruleTXXunitI}[1]{\NDdrule[#1]{%
}{
 \cdot   \vdash_\mathcal{C}   \mathsf{triv}   \NDsym{:}   \mathsf{Unit} }{%
{\NDdruleTXXunitIName}{}%
}}


\newcommand{\NDdruleTXXunitEName}[0]{\NDdrulename{T\_unitE}}
\newcommand{\NDdruleTXXunitE}[1]{\NDdrule[#1]{%
\NDpremise{ \Phi  \vdash_\mathcal{C}  \NDnt{t_{{\mathrm{1}}}}  \NDsym{:}   \mathsf{Unit}   \quad  \Psi  \vdash_\mathcal{C}  \NDnt{t_{{\mathrm{2}}}}  \NDsym{:}  \NDnt{Y} }%
}{
\Phi  \NDsym{,}  \Psi  \vdash_\mathcal{C}   \mathsf{let}\, \NDnt{t_{{\mathrm{1}}}}  :   \mathsf{Unit}  \,\mathsf{be}\,  \mathsf{triv}  \,\mathsf{in}\, \NDnt{t_{{\mathrm{2}}}}   \NDsym{:}  \NDnt{Y}}{%
{\NDdruleTXXunitEName}{}%
}}


\newcommand{\NDdruleTXXtenIName}[0]{\NDdrulename{T\_tenI}}
\newcommand{\NDdruleTXXtenI}[1]{\NDdrule[#1]{%
\NDpremise{ \Phi  \vdash_\mathcal{C}  \NDnt{t_{{\mathrm{1}}}}  \NDsym{:}  \NDnt{X}  \quad  \Psi  \vdash_\mathcal{C}  \NDnt{t_{{\mathrm{2}}}}  \NDsym{:}  \NDnt{Y} }%
}{
\Phi  \NDsym{,}  \Psi  \vdash_\mathcal{C}  \NDnt{t_{{\mathrm{1}}}}  \otimes  \NDnt{t_{{\mathrm{2}}}}  \NDsym{:}  \NDnt{X}  \otimes  \NDnt{Y}}{%
{\NDdruleTXXtenIName}{}%
}}


\newcommand{\NDdruleTXXtenEName}[0]{\NDdrulename{T\_tenE}}
\newcommand{\NDdruleTXXtenE}[1]{\NDdrule[#1]{%
\NDpremise{ \Phi  \vdash_\mathcal{C}  \NDnt{t_{{\mathrm{1}}}}  \NDsym{:}  \NDnt{X}  \otimes  \NDnt{Y}  \quad  \Psi_{{\mathrm{1}}}  \NDsym{,}  \NDmv{x}  \NDsym{:}  \NDnt{X}  \NDsym{,}  \NDmv{y}  \NDsym{:}  \NDnt{Y}  \NDsym{,}  \Psi_{{\mathrm{2}}}  \vdash_\mathcal{C}  \NDnt{t_{{\mathrm{2}}}}  \NDsym{:}  \NDnt{Z} }%
}{
\Psi_{{\mathrm{1}}}  \NDsym{,}  \Phi  \NDsym{,}  \Psi_{{\mathrm{2}}}  \vdash_\mathcal{C}   \mathsf{let}\, \NDnt{t_{{\mathrm{1}}}}  :  \NDnt{X}  \otimes  \NDnt{Y} \,\mathsf{be}\, \NDmv{x}  \otimes  \NDmv{y} \,\mathsf{in}\, \NDnt{t_{{\mathrm{2}}}}   \NDsym{:}  \NDnt{Z}}{%
{\NDdruleTXXtenEName}{}%
}}


\newcommand{\NDdruleTXXimpIName}[0]{\NDdrulename{T\_impI}}
\newcommand{\NDdruleTXXimpI}[1]{\NDdrule[#1]{%
\NDpremise{\Phi  \NDsym{,}  \NDmv{x}  \NDsym{:}  \NDnt{X}  \vdash_\mathcal{C}  \NDnt{t}  \NDsym{:}  \NDnt{Y}}%
}{
\Phi  \vdash_\mathcal{C}   \lambda  \NDmv{x}  :  \NDnt{X} . \NDnt{t}   \NDsym{:}  \NDnt{X}  \multimap  \NDnt{Y}}{%
{\NDdruleTXXimpIName}{}%
}}


\newcommand{\NDdruleTXXimpEName}[0]{\NDdrulename{T\_impE}}
\newcommand{\NDdruleTXXimpE}[1]{\NDdrule[#1]{%
\NDpremise{ \Phi  \vdash_\mathcal{C}  \NDnt{t_{{\mathrm{1}}}}  \NDsym{:}  \NDnt{X}  \multimap  \NDnt{Y}  \quad  \Psi  \vdash_\mathcal{C}  \NDnt{t_{{\mathrm{2}}}}  \NDsym{:}  \NDnt{X} }%
}{
\Phi  \NDsym{,}  \Psi  \vdash_\mathcal{C}   \NDnt{t_{{\mathrm{1}}}}   \NDnt{t_{{\mathrm{2}}}}   \NDsym{:}  \NDnt{Y}}{%
{\NDdruleTXXimpEName}{}%
}}


\newcommand{\NDdruleTXXGIName}[0]{\NDdrulename{T\_GI}}
\newcommand{\NDdruleTXXGI}[1]{\NDdrule[#1]{%
\NDpremise{\Phi  \vdash_\mathcal{L}  \NDnt{s}  \NDsym{:}  \NDnt{A}}%
}{
\Phi  \vdash_\mathcal{C}   \mathsf{G}\, \NDnt{s}   \NDsym{:}   \mathsf{G} \NDnt{A} }{%
{\NDdruleTXXGIName}{}%
}}


\newcommand{\NDdruleTXXbetaName}[0]{\NDdrulename{T\_beta}}
\newcommand{\NDdruleTXXbeta}[1]{\NDdrule[#1]{%
\NDpremise{\Phi  \NDsym{,}  \NDmv{x}  \NDsym{:}  \NDnt{X}  \NDsym{,}  \NDmv{y}  \NDsym{:}  \NDnt{Y}  \NDsym{,}  \Psi  \vdash_\mathcal{C}  \NDnt{t}  \NDsym{:}  \NDnt{Z}}%
}{
\Phi  \NDsym{,}  \NDmv{z}  \NDsym{:}  \NDnt{Y}  \NDsym{,}  \NDmv{w}  \NDsym{:}  \NDnt{X}  \NDsym{,}  \Psi  \vdash_\mathcal{C}   \mathsf{ex}\, \NDmv{w} , \NDmv{z} \,\mathsf{with}\, \NDmv{x} , \NDmv{y} \,\mathsf{in}\, \NDnt{t}   \NDsym{:}  \NDnt{Z}}{%
{\NDdruleTXXbetaName}{}%
}}


\newcommand{\NDdruleTXXcutName}[0]{\NDdrulename{T\_cut}}
\newcommand{\NDdruleTXXcut}[1]{\NDdrule[#1]{%
\NDpremise{ \Phi  \vdash_\mathcal{C}  \NDnt{t_{{\mathrm{1}}}}  \NDsym{:}  \NDnt{X}  \quad  \Psi_{{\mathrm{1}}}  \NDsym{,}  \NDmv{x}  \NDsym{:}  \NDnt{X}  \NDsym{,}  \Psi_{{\mathrm{2}}}  \vdash_\mathcal{C}  \NDnt{t_{{\mathrm{2}}}}  \NDsym{:}  \NDnt{Y} }%
}{
\Psi_{{\mathrm{1}}}  \NDsym{,}  \Phi  \NDsym{,}  \Psi_{{\mathrm{2}}}  \vdash_\mathcal{C}  \NDsym{[}  \NDnt{t_{{\mathrm{1}}}}  \NDsym{/}  \NDmv{x}  \NDsym{]}  \NDnt{t_{{\mathrm{2}}}}  \NDsym{:}  \NDnt{Y}}{%
{\NDdruleTXXcutName}{}%
}}

\newcommand{\NDdefntty}[1]{\begin{NDdefnblock}[#1]{$\Phi  \vdash_\mathcal{C}  \NDnt{t}  \NDsym{:}  \NDnt{X}$}{}
\NDusedrule{\NDdruleTXXid{}}
\NDusedrule{\NDdruleTXXunitI{}}
\NDusedrule{\NDdruleTXXunitE{}}
\NDusedrule{\NDdruleTXXtenI{}}
\NDusedrule{\NDdruleTXXtenE{}}
\NDusedrule{\NDdruleTXXimpI{}}
\NDusedrule{\NDdruleTXXimpE{}}
\NDusedrule{\NDdruleTXXGI{}}
\NDusedrule{\NDdruleTXXbeta{}}
\NDusedrule{\NDdruleTXXcut{}}
\end{NDdefnblock}}

%% defn sty
\newcommand{\NDdruleSXXidName}[0]{\NDdrulename{S\_id}}
\newcommand{\NDdruleSXXid}[1]{\NDdrule[#1]{%
}{
\NDmv{x}  \NDsym{:}  \NDnt{A}  \vdash_\mathcal{L}  \NDmv{x}  \NDsym{:}  \NDnt{A}}{%
{\NDdruleSXXidName}{}%
}}


\newcommand{\NDdruleSXXunitIName}[0]{\NDdrulename{S\_unitI}}
\newcommand{\NDdruleSXXunitI}[1]{\NDdrule[#1]{%
}{
 \cdot   \vdash_\mathcal{L}   \mathsf{triv}   \NDsym{:}   \mathsf{Unit} }{%
{\NDdruleSXXunitIName}{}%
}}


\newcommand{\NDdruleSXXunitEOneName}[0]{\NDdrulename{S\_unitE1}}
\newcommand{\NDdruleSXXunitEOne}[1]{\NDdrule[#1]{%
\NDpremise{ \Phi  \vdash_\mathcal{C}  \NDnt{t}  \NDsym{:}   \mathsf{Unit}   \quad  \Gamma  \vdash_\mathcal{L}  \NDnt{s}  \NDsym{:}  \NDnt{A} }%
}{
\Phi  \NDsym{;}  \Gamma  \vdash_\mathcal{L}   \mathsf{let}\, \NDnt{t}  :   \mathsf{Unit}  \,\mathsf{be}\,  \mathsf{triv}  \,\mathsf{in}\, \NDnt{s}   \NDsym{:}  \NDnt{A}}{%
{\NDdruleSXXunitEOneName}{}%
}}


\newcommand{\NDdruleSXXunitETwoName}[0]{\NDdrulename{S\_unitE2}}
\newcommand{\NDdruleSXXunitETwo}[1]{\NDdrule[#1]{%
\NDpremise{ \Gamma  \vdash_\mathcal{L}  \NDnt{s_{{\mathrm{1}}}}  \NDsym{:}   \mathsf{Unit}   \quad  \Delta  \vdash_\mathcal{L}  \NDnt{s_{{\mathrm{2}}}}  \NDsym{:}  \NDnt{A} }%
}{
\Gamma  \NDsym{;}  \Delta  \vdash_\mathcal{L}   \mathsf{let}\, \NDnt{s_{{\mathrm{1}}}}  :   \mathsf{Unit}  \,\mathsf{be}\,  \mathsf{triv}  \,\mathsf{in}\, \NDnt{s_{{\mathrm{2}}}}   \NDsym{:}  \NDnt{A}}{%
{\NDdruleSXXunitETwoName}{}%
}}


\newcommand{\NDdruleSXXtenIName}[0]{\NDdrulename{S\_tenI}}
\newcommand{\NDdruleSXXtenI}[1]{\NDdrule[#1]{%
\NDpremise{ \Gamma  \vdash_\mathcal{L}  \NDnt{s_{{\mathrm{1}}}}  \NDsym{:}  \NDnt{A}  \quad  \Delta  \vdash_\mathcal{L}  \NDnt{s_{{\mathrm{2}}}}  \NDsym{:}  \NDnt{B} }%
}{
\Gamma  \NDsym{;}  \Delta  \vdash_\mathcal{L}  \NDnt{s_{{\mathrm{1}}}}  \triangleright  \NDnt{s_{{\mathrm{2}}}}  \NDsym{:}  \NDnt{A}  \triangleright  \NDnt{B}}{%
{\NDdruleSXXtenIName}{}%
}}


\newcommand{\NDdruleSXXtenEOneName}[0]{\NDdrulename{S\_tenE1}}
\newcommand{\NDdruleSXXtenEOne}[1]{\NDdrule[#1]{%
\NDpremise{ \Phi  \vdash_\mathcal{C}  \NDnt{t}  \NDsym{:}  \NDnt{X}  \otimes  \NDnt{Y}  \quad  \Gamma_{{\mathrm{1}}}  \NDsym{;}  \NDmv{x}  \NDsym{:}  \NDnt{X}  \NDsym{;}  \NDmv{y}  \NDsym{:}  \NDnt{Y}  \NDsym{;}  \Gamma_{{\mathrm{2}}}  \vdash_\mathcal{L}  \NDnt{s}  \NDsym{:}  \NDnt{A} }%
}{
\Gamma_{{\mathrm{1}}}  \NDsym{;}  \Phi  \NDsym{;}  \Gamma_{{\mathrm{2}}}  \vdash_\mathcal{L}   \mathsf{let}\, \NDnt{t}  :  \NDnt{X}  \otimes  \NDnt{Y} \,\mathsf{be}\, \NDmv{x}  \otimes  \NDmv{y} \,\mathsf{in}\, \NDnt{s}   \NDsym{:}  \NDnt{A}}{%
{\NDdruleSXXtenEOneName}{}%
}}


\newcommand{\NDdruleSXXtenETwoName}[0]{\NDdrulename{S\_tenE2}}
\newcommand{\NDdruleSXXtenETwo}[1]{\NDdrule[#1]{%
\NDpremise{ \Gamma  \vdash_\mathcal{L}  \NDnt{s_{{\mathrm{1}}}}  \NDsym{:}  \NDnt{A}  \triangleright  \NDnt{B}  \quad  \Delta_{{\mathrm{1}}}  \NDsym{;}  \NDmv{x}  \NDsym{:}  \NDnt{A}  \NDsym{;}  \NDmv{y}  \NDsym{:}  \NDnt{B}  \NDsym{;}  \Delta_{{\mathrm{2}}}  \vdash_\mathcal{L}  \NDnt{s_{{\mathrm{2}}}}  \NDsym{:}  \NDnt{C} }%
}{
\Delta_{{\mathrm{1}}}  \NDsym{;}  \Gamma  \NDsym{;}  \Delta_{{\mathrm{2}}}  \vdash_\mathcal{L}   \mathsf{let}\, \NDnt{s_{{\mathrm{1}}}}  :  \NDnt{A}  \triangleright  \NDnt{B} \,\mathsf{be}\, \NDmv{x}  \triangleright  \NDmv{y} \,\mathsf{in}\, \NDnt{s_{{\mathrm{2}}}}   \NDsym{:}  \NDnt{C}}{%
{\NDdruleSXXtenETwoName}{}%
}}


\newcommand{\NDdruleSXXimprIName}[0]{\NDdrulename{S\_imprI}}
\newcommand{\NDdruleSXXimprI}[1]{\NDdrule[#1]{%
\NDpremise{\Gamma  \NDsym{;}  \NDmv{x}  \NDsym{:}  \NDnt{A}  \vdash_\mathcal{L}  \NDnt{s}  \NDsym{:}  \NDnt{B}}%
}{
\Gamma  \vdash_\mathcal{L}   \lambda_r  \NDmv{x}  :  \NDnt{A} . \NDnt{s}   \NDsym{:}  \NDnt{A}  \rightharpoonup  \NDnt{B}}{%
{\NDdruleSXXimprIName}{}%
}}


\newcommand{\NDdruleSXXimprEName}[0]{\NDdrulename{S\_imprE}}
\newcommand{\NDdruleSXXimprE}[1]{\NDdrule[#1]{%
\NDpremise{ \Gamma  \vdash_\mathcal{L}  \NDnt{s_{{\mathrm{1}}}}  \NDsym{:}  \NDnt{A}  \rightharpoonup  \NDnt{B}  \quad  \Delta  \vdash_\mathcal{L}  \NDnt{s_{{\mathrm{2}}}}  \NDsym{:}  \NDnt{A} }%
}{
\Gamma  \NDsym{;}  \Delta  \vdash_\mathcal{L}   \mathsf{app}_r\, \NDnt{s_{{\mathrm{1}}}} \, \NDnt{s_{{\mathrm{2}}}}   \NDsym{:}  \NDnt{B}}{%
{\NDdruleSXXimprEName}{}%
}}


\newcommand{\NDdruleSXXimplIName}[0]{\NDdrulename{S\_implI}}
\newcommand{\NDdruleSXXimplI}[1]{\NDdrule[#1]{%
\NDpremise{\NDmv{x}  \NDsym{:}  \NDnt{A}  \NDsym{;}  \Gamma  \vdash_\mathcal{L}  \NDnt{s}  \NDsym{:}  \NDnt{B}}%
}{
\Gamma  \vdash_\mathcal{L}   \lambda_l  \NDmv{x}  :  \NDnt{A} . \NDnt{s}   \NDsym{:}  \NDnt{B}  \leftharpoonup  \NDnt{A}}{%
{\NDdruleSXXimplIName}{}%
}}


\newcommand{\NDdruleSXXimplEName}[0]{\NDdrulename{S\_implE}}
\newcommand{\NDdruleSXXimplE}[1]{\NDdrule[#1]{%
\NDpremise{ \Gamma  \vdash_\mathcal{L}  \NDnt{s_{{\mathrm{1}}}}  \NDsym{:}  \NDnt{B}  \leftharpoonup  \NDnt{A}  \quad  \Delta  \vdash_\mathcal{L}  \NDnt{s_{{\mathrm{2}}}}  \NDsym{:}  \NDnt{A} }%
}{
\Delta  \NDsym{;}  \Gamma  \vdash_\mathcal{L}   \mathsf{app}_l\, \NDnt{s_{{\mathrm{1}}}} \, \NDnt{s_{{\mathrm{2}}}}   \NDsym{:}  \NDnt{B}}{%
{\NDdruleSXXimplEName}{}%
}}


\newcommand{\NDdruleSXXFIName}[0]{\NDdrulename{S\_FI}}
\newcommand{\NDdruleSXXFI}[1]{\NDdrule[#1]{%
\NDpremise{\Phi  \vdash_\mathcal{C}  \NDnt{t}  \NDsym{:}  \NDnt{X}}%
}{
\Phi  \vdash_\mathcal{L}   \mathsf{F} \NDnt{t}   \NDsym{:}   \mathsf{F} \NDnt{X} }{%
{\NDdruleSXXFIName}{}%
}}


\newcommand{\NDdruleSXXFEName}[0]{\NDdrulename{S\_FE}}
\newcommand{\NDdruleSXXFE}[1]{\NDdrule[#1]{%
\NDpremise{ \Gamma  \vdash_\mathcal{L}  \NDnt{s_{{\mathrm{1}}}}  \NDsym{:}   \mathsf{F} \NDnt{X}   \quad  \Delta_{{\mathrm{1}}}  \NDsym{;}  \NDmv{x}  \NDsym{:}  \NDnt{X}  \NDsym{;}  \Delta_{{\mathrm{2}}}  \vdash_\mathcal{L}  \NDnt{s_{{\mathrm{2}}}}  \NDsym{:}  \NDnt{A} }%
}{
\Delta_{{\mathrm{1}}}  \NDsym{;}  \Gamma  \NDsym{;}  \Delta_{{\mathrm{2}}}  \vdash_\mathcal{L}   \mathsf{let}\, \NDnt{s_{{\mathrm{1}}}}  :   \mathsf{F} \NDnt{X}  \,\mathsf{be}\,  \mathsf{F}\, \NDmv{x}  \,\mathsf{in}\, \NDnt{s_{{\mathrm{2}}}}   \NDsym{:}  \NDnt{A}}{%
{\NDdruleSXXFEName}{}%
}}


\newcommand{\NDdruleSXXGEName}[0]{\NDdrulename{S\_GE}}
\newcommand{\NDdruleSXXGE}[1]{\NDdrule[#1]{%
\NDpremise{\Phi  \vdash_\mathcal{C}  \NDnt{t}  \NDsym{:}   \mathsf{G} \NDnt{A} }%
}{
\Phi  \vdash_\mathcal{L}   \mathsf{derelict}\, \NDnt{t}   \NDsym{:}  \NDnt{A}}{%
{\NDdruleSXXGEName}{}%
}}


\newcommand{\NDdruleSXXbetaName}[0]{\NDdrulename{S\_beta}}
\newcommand{\NDdruleSXXbeta}[1]{\NDdrule[#1]{%
\NDpremise{\Gamma  \NDsym{;}  \NDmv{x}  \NDsym{:}  \NDnt{X}  \NDsym{;}  \NDmv{y}  \NDsym{:}  \NDnt{Y}  \NDsym{;}  \Delta  \vdash_\mathcal{L}  \NDnt{s}  \NDsym{:}  \NDnt{A}}%
}{
\Gamma  \NDsym{;}  \NDmv{z}  \NDsym{:}  \NDnt{Y}  \NDsym{;}  \NDmv{w}  \NDsym{:}  \NDnt{X}  \NDsym{;}  \Delta  \vdash_\mathcal{L}   \mathsf{ex}\, \NDmv{w} , \NDmv{z} \,\mathsf{with}\, \NDmv{x} , \NDmv{y} \,\mathsf{in}\, \NDnt{s}   \NDsym{:}  \NDnt{A}}{%
{\NDdruleSXXbetaName}{}%
}}


\newcommand{\NDdruleSXXcutOneName}[0]{\NDdrulename{S\_cut1}}
\newcommand{\NDdruleSXXcutOne}[1]{\NDdrule[#1]{%
\NDpremise{ \Phi  \vdash_\mathcal{C}  \NDnt{t}  \NDsym{:}  \NDnt{X}  \quad  \Gamma_{{\mathrm{1}}}  \NDsym{;}  \NDmv{x}  \NDsym{:}  \NDnt{X}  \NDsym{;}  \Gamma_{{\mathrm{2}}}  \vdash_\mathcal{L}  \NDnt{s}  \NDsym{:}  \NDnt{A} }%
}{
\Gamma_{{\mathrm{1}}}  \NDsym{;}  \Phi  \NDsym{;}  \Gamma_{{\mathrm{1}}}  \vdash_\mathcal{L}  \NDsym{[}  \NDnt{t}  \NDsym{/}  \NDmv{x}  \NDsym{]}  \NDnt{s}  \NDsym{:}  \NDnt{A}}{%
{\NDdruleSXXcutOneName}{}%
}}


\newcommand{\NDdruleSXXcutTwoName}[0]{\NDdrulename{S\_cut2}}
\newcommand{\NDdruleSXXcutTwo}[1]{\NDdrule[#1]{%
\NDpremise{ \Gamma  \vdash_\mathcal{L}  \NDnt{s_{{\mathrm{1}}}}  \NDsym{:}  \NDnt{A}  \quad  \Delta_{{\mathrm{1}}}  \NDsym{;}  \NDmv{x}  \NDsym{:}  \NDnt{A}  \NDsym{;}  \Delta_{{\mathrm{2}}}  \vdash_\mathcal{L}  \NDnt{s_{{\mathrm{2}}}}  \NDsym{:}  \NDnt{B} }%
}{
\Delta_{{\mathrm{1}}}  \NDsym{;}  \Gamma  \NDsym{;}  \Delta_{{\mathrm{2}}}  \vdash_\mathcal{L}  \NDsym{[}  \NDnt{s_{{\mathrm{1}}}}  \NDsym{/}  \NDmv{x}  \NDsym{]}  \NDnt{s_{{\mathrm{2}}}}  \NDsym{:}  \NDnt{B}}{%
{\NDdruleSXXcutTwoName}{}%
}}

\newcommand{\NDdefnsty}[1]{\begin{NDdefnblock}[#1]{$\Gamma  \vdash_\mathcal{L}  \NDnt{s}  \NDsym{:}  \NDnt{A}$}{}
\NDusedrule{\NDdruleSXXid{}}
\NDusedrule{\NDdruleSXXunitI{}}
\NDusedrule{\NDdruleSXXunitEOne{}}
\NDusedrule{\NDdruleSXXunitETwo{}}
\NDusedrule{\NDdruleSXXtenI{}}
\NDusedrule{\NDdruleSXXtenEOne{}}
\NDusedrule{\NDdruleSXXtenETwo{}}
\NDusedrule{\NDdruleSXXimprI{}}
\NDusedrule{\NDdruleSXXimprE{}}
\NDusedrule{\NDdruleSXXimplI{}}
\NDusedrule{\NDdruleSXXimplE{}}
\NDusedrule{\NDdruleSXXFI{}}
\NDusedrule{\NDdruleSXXFE{}}
\NDusedrule{\NDdruleSXXGE{}}
\NDusedrule{\NDdruleSXXbeta{}}
\NDusedrule{\NDdruleSXXcutOne{}}
\NDusedrule{\NDdruleSXXcutTwo{}}
\end{NDdefnblock}}


\newcommand{\NDdefnsJtype}{
\NDdefntty{}\NDdefnsty{}}

% defns Reduction
%% defn tred
\newcommand{\NDdruleTbetaXXletUName}[0]{\NDdrulename{Tbeta\_letU}}
\newcommand{\NDdruleTbetaXXletU}[1]{\NDdrule[#1]{%
}{
 \mathsf{let}\,  \mathsf{triv}   :   \mathsf{Unit}  \,\mathsf{be}\,  \mathsf{triv}  \,\mathsf{in}\, \NDnt{t}   \leadsto_\beta  \NDnt{t}}{%
{\NDdruleTbetaXXletUName}{}%
}}


\newcommand{\NDdruleTbetaXXletTName}[0]{\NDdrulename{Tbeta\_letT}}
\newcommand{\NDdruleTbetaXXletT}[1]{\NDdrule[#1]{%
}{
 \mathsf{let}\, \NDnt{t_{{\mathrm{1}}}}  \otimes  \NDnt{t_{{\mathrm{2}}}}  :  \NDnt{X}  \otimes  \NDnt{Y} \,\mathsf{be}\, \NDmv{x}  \otimes  \NDmv{y} \,\mathsf{in}\, \NDnt{t_{{\mathrm{3}}}}   \leadsto_\beta  \NDsym{[}  \NDnt{t_{{\mathrm{1}}}}  \NDsym{/}  \NDmv{x}  \NDsym{]}  \NDsym{[}  \NDnt{t_{{\mathrm{2}}}}  \NDsym{/}  \NDmv{y}  \NDsym{]}  \NDnt{t_{{\mathrm{3}}}}}{%
{\NDdruleTbetaXXletTName}{}%
}}


\newcommand{\NDdruleTbetaXXlamName}[0]{\NDdrulename{Tbeta\_lam}}
\newcommand{\NDdruleTbetaXXlam}[1]{\NDdrule[#1]{%
}{
 \NDsym{(}   \lambda  \NDmv{x}  :  \NDnt{X} . \NDnt{t_{{\mathrm{1}}}}   \NDsym{)}   \NDnt{t_{{\mathrm{2}}}}   \leadsto_\beta  \NDsym{[}  \NDnt{t_{{\mathrm{2}}}}  \NDsym{/}  \NDmv{x}  \NDsym{]}  \NDnt{t_{{\mathrm{1}}}}}{%
{\NDdruleTbetaXXlamName}{}%
}}


\newcommand{\NDdruleTbetaXXappOneName}[0]{\NDdrulename{Tbeta\_app1}}
\newcommand{\NDdruleTbetaXXappOne}[1]{\NDdrule[#1]{%
\NDpremise{\NDnt{t_{{\mathrm{1}}}}  \leadsto_\beta  \NDnt{t'_{{\mathrm{1}}}}}%
}{
 \NDnt{t_{{\mathrm{1}}}}   \NDnt{t_{{\mathrm{2}}}}   \leadsto_\beta   \NDnt{t'_{{\mathrm{1}}}}   \NDnt{t_{{\mathrm{2}}}} }{%
{\NDdruleTbetaXXappOneName}{}%
}}


\newcommand{\NDdruleTbetaXXappTwoName}[0]{\NDdrulename{Tbeta\_app2}}
\newcommand{\NDdruleTbetaXXappTwo}[1]{\NDdrule[#1]{%
\NDpremise{\NDnt{t_{{\mathrm{2}}}}  \leadsto_\beta  \NDnt{t'_{{\mathrm{2}}}}}%
}{
 \NDnt{t_{{\mathrm{1}}}}   \NDnt{t_{{\mathrm{2}}}}   \leadsto_\beta   \NDnt{t_{{\mathrm{1}}}}   \NDnt{t'_{{\mathrm{2}}}} }{%
{\NDdruleTbetaXXappTwoName}{}%
}}


\newcommand{\NDdruleTbetaXXappLetName}[0]{\NDdrulename{Tbeta\_appLet}}
\newcommand{\NDdruleTbetaXXappLet}[1]{\NDdrule[#1]{%
}{
 \NDsym{(}   \mathsf{let}\, \NDnt{t}  :  \NDnt{X} \,\mathsf{be}\, \NDnt{p} \,\mathsf{in}\, \NDnt{t_{{\mathrm{1}}}}   \NDsym{)}   \NDnt{t_{{\mathrm{2}}}}   \leadsto_\beta   \mathsf{let}\, \NDnt{t}  :  \NDnt{X} \,\mathsf{be}\, \NDnt{p} \,\mathsf{in}\, \NDsym{(}   \NDnt{t_{{\mathrm{1}}}}   \NDnt{t_{{\mathrm{2}}}}   \NDsym{)} }{%
{\NDdruleTbetaXXappLetName}{}%
}}


\newcommand{\NDdruleTbetaXXletLetName}[0]{\NDdrulename{Tbeta\_letLet}}
\newcommand{\NDdruleTbetaXXletLet}[1]{\NDdrule[#1]{%
}{
 \mathsf{let}\, \NDsym{(}   \mathsf{let}\, \NDnt{t_{{\mathrm{2}}}}  :  \NDnt{X} \,\mathsf{be}\, \NDnt{p_{{\mathrm{1}}}} \,\mathsf{in}\, \NDnt{t_{{\mathrm{1}}}}   \NDsym{)}  :  \NDnt{Y} \,\mathsf{be}\, \NDnt{p_{{\mathrm{2}}}} \,\mathsf{in}\, \NDnt{t_{{\mathrm{3}}}}   \leadsto_\beta   \mathsf{let}\, \NDnt{t_{{\mathrm{2}}}}  :  \NDnt{X} \,\mathsf{be}\, \NDnt{p_{{\mathrm{1}}}} \,\mathsf{in}\,  \mathsf{let}\, \NDnt{t_{{\mathrm{1}}}}  :  \NDnt{Y} \,\mathsf{be}\, \NDnt{p_{{\mathrm{2}}}} \,\mathsf{in}\, \NDnt{t_{{\mathrm{3}}}}  }{%
{\NDdruleTbetaXXletLetName}{}%
}}


\newcommand{\NDdruleTbetaXXletAppName}[0]{\NDdrulename{Tbeta\_letApp}}
\newcommand{\NDdruleTbetaXXletApp}[1]{\NDdrule[#1]{%
}{
 \mathsf{let}\, \NDnt{t_{{\mathrm{1}}}}  :  \NDnt{X} \,\mathsf{be}\, \NDnt{p} \,\mathsf{in}\, \NDsym{(}   \NDnt{t_{{\mathrm{1}}}}   \NDnt{t_{{\mathrm{2}}}}   \NDsym{)}   \leadsto_\beta   \NDsym{(}   \mathsf{let}\, \NDnt{t_{{\mathrm{1}}}}  :  \NDnt{X} \,\mathsf{be}\, \NDnt{p} \,\mathsf{in}\, \NDnt{t_{{\mathrm{1}}}}   \NDsym{)}   \NDsym{(}   \mathsf{let}\, \NDnt{t_{{\mathrm{1}}}}  :  \NDnt{X} \,\mathsf{be}\, \NDnt{p} \,\mathsf{in}\, \NDnt{t_{{\mathrm{2}}}}   \NDsym{)} }{%
{\NDdruleTbetaXXletAppName}{}%
}}

\newcommand{\NDdefntred}[1]{\begin{NDdefnblock}[#1]{$\NDnt{t_{{\mathrm{1}}}}  \leadsto_\beta  \NDnt{t_{{\mathrm{2}}}}$}{}
\NDusedrule{\NDdruleTbetaXXletU{}}
\NDusedrule{\NDdruleTbetaXXletT{}}
\NDusedrule{\NDdruleTbetaXXlam{}}
\NDusedrule{\NDdruleTbetaXXappOne{}}
\NDusedrule{\NDdruleTbetaXXappTwo{}}
\NDusedrule{\NDdruleTbetaXXappLet{}}
\NDusedrule{\NDdruleTbetaXXletLet{}}
\NDusedrule{\NDdruleTbetaXXletApp{}}
\end{NDdefnblock}}

%% defn sred
\newcommand{\NDdruleSbetaXXletUOneName}[0]{\NDdrulename{Sbeta\_letU1}}
\newcommand{\NDdruleSbetaXXletUOne}[1]{\NDdrule[#1]{%
}{
 \mathsf{let}\,  \mathsf{triv}   :   \mathsf{Unit}  \,\mathsf{be}\,  \mathsf{triv}  \,\mathsf{in}\, \NDnt{s}   \leadsto_\beta  \NDnt{s}}{%
{\NDdruleSbetaXXletUOneName}{}%
}}


\newcommand{\NDdruleSbetaXXletTOneName}[0]{\NDdrulename{Sbeta\_letT1}}
\newcommand{\NDdruleSbetaXXletTOne}[1]{\NDdrule[#1]{%
}{
 \mathsf{let}\, \NDnt{t_{{\mathrm{1}}}}  \otimes  \NDnt{t_{{\mathrm{2}}}}  :  \NDnt{X}  \otimes  \NDnt{Y} \,\mathsf{be}\, \NDmv{x}  \triangleright  \NDmv{y} \,\mathsf{in}\, \NDnt{s}   \leadsto_\beta  \NDsym{[}  \NDnt{t_{{\mathrm{1}}}}  \NDsym{/}  \NDmv{x}  \NDsym{]}  \NDsym{[}  \NDnt{t_{{\mathrm{2}}}}  \NDsym{/}  \NDmv{y}  \NDsym{]}  \NDnt{s}}{%
{\NDdruleSbetaXXletTOneName}{}%
}}


\newcommand{\NDdruleSbetaXXletTTwoName}[0]{\NDdrulename{Sbeta\_letT2}}
\newcommand{\NDdruleSbetaXXletTTwo}[1]{\NDdrule[#1]{%
}{
 \mathsf{let}\, \NDnt{s_{{\mathrm{1}}}}  \triangleright  \NDnt{s_{{\mathrm{2}}}}  :  \NDnt{A}  \triangleright  \NDnt{B} \,\mathsf{be}\, \NDmv{x}  \triangleright  \NDmv{y} \,\mathsf{in}\, \NDnt{s_{{\mathrm{3}}}}   \leadsto_\beta  \NDsym{[}  \NDnt{s_{{\mathrm{1}}}}  \NDsym{/}  \NDmv{x}  \NDsym{]}  \NDsym{[}  \NDnt{s_{{\mathrm{2}}}}  \NDsym{/}  \NDmv{y}  \NDsym{]}  \NDnt{s_{{\mathrm{3}}}}}{%
{\NDdruleSbetaXXletTTwoName}{}%
}}


\newcommand{\NDdruleSbetaXXletFName}[0]{\NDdrulename{Sbeta\_letF}}
\newcommand{\NDdruleSbetaXXletF}[1]{\NDdrule[#1]{%
}{
 \mathsf{let}\,  \mathsf{F} \NDnt{t}   :   \mathsf{F} \NDnt{X}  \,\mathsf{be}\,  \mathsf{F}\, \NDmv{x}  \,\mathsf{in}\, \NDnt{s}   \leadsto_\beta  \NDsym{[}  \NDnt{t}  \NDsym{/}  \NDmv{x}  \NDsym{]}  \NDnt{s}}{%
{\NDdruleSbetaXXletFName}{}%
}}


\newcommand{\NDdruleSbetaXXlamLName}[0]{\NDdrulename{Sbeta\_lamL}}
\newcommand{\NDdruleSbetaXXlamL}[1]{\NDdrule[#1]{%
}{
 \mathsf{app}_l\, \NDsym{(}   \lambda_l  \NDmv{x}  :  \NDnt{A} . \NDnt{s_{{\mathrm{1}}}}   \NDsym{)} \, \NDnt{s_{{\mathrm{2}}}}   \leadsto_\beta  \NDsym{[}  \NDnt{s_{{\mathrm{2}}}}  \NDsym{/}  \NDmv{x}  \NDsym{]}  \NDnt{s_{{\mathrm{1}}}}}{%
{\NDdruleSbetaXXlamLName}{}%
}}


\newcommand{\NDdruleSbetaXXlamRName}[0]{\NDdrulename{Sbeta\_lamR}}
\newcommand{\NDdruleSbetaXXlamR}[1]{\NDdrule[#1]{%
}{
 \mathsf{app}_r\, \NDsym{(}   \lambda_r  \NDmv{x}  :  \NDnt{A} . \NDnt{s_{{\mathrm{1}}}}   \NDsym{)} \, \NDnt{s_{{\mathrm{2}}}}   \leadsto_\beta  \NDsym{[}  \NDnt{s_{{\mathrm{2}}}}  \NDsym{/}  \NDmv{x}  \NDsym{]}  \NDnt{s_{{\mathrm{1}}}}}{%
{\NDdruleSbetaXXlamRName}{}%
}}


\newcommand{\NDdruleSbetaXXapplOneName}[0]{\NDdrulename{Sbeta\_appl1}}
\newcommand{\NDdruleSbetaXXapplOne}[1]{\NDdrule[#1]{%
\NDpremise{\NDnt{s_{{\mathrm{1}}}}  \leadsto_\beta  \NDnt{s'_{{\mathrm{1}}}}}%
}{
 \mathsf{app}_l\, \NDnt{s_{{\mathrm{1}}}} \, \NDnt{s_{{\mathrm{2}}}}   \leadsto_\beta   \mathsf{app}_l\, \NDnt{s'_{{\mathrm{1}}}} \, \NDnt{s_{{\mathrm{2}}}} }{%
{\NDdruleSbetaXXapplOneName}{}%
}}


\newcommand{\NDdruleSbetaXXapplTwoName}[0]{\NDdrulename{Sbeta\_appl2}}
\newcommand{\NDdruleSbetaXXapplTwo}[1]{\NDdrule[#1]{%
\NDpremise{\NDnt{s_{{\mathrm{2}}}}  \leadsto_\beta  \NDnt{s'_{{\mathrm{2}}}}}%
}{
 \mathsf{app}_l\, \NDnt{s_{{\mathrm{1}}}} \, \NDnt{s_{{\mathrm{2}}}}   \leadsto_\beta   \mathsf{app}_l\, \NDnt{s_{{\mathrm{1}}}} \, \NDnt{s'_{{\mathrm{2}}}} }{%
{\NDdruleSbetaXXapplTwoName}{}%
}}


\newcommand{\NDdruleSbetaXXapprOneName}[0]{\NDdrulename{Sbeta\_appr1}}
\newcommand{\NDdruleSbetaXXapprOne}[1]{\NDdrule[#1]{%
\NDpremise{\NDnt{s_{{\mathrm{1}}}}  \leadsto_\beta  \NDnt{s'_{{\mathrm{1}}}}}%
}{
 \mathsf{app}_r\, \NDnt{s_{{\mathrm{1}}}} \, \NDnt{s_{{\mathrm{2}}}}   \leadsto_\beta   \mathsf{app}_r\, \NDnt{s'_{{\mathrm{1}}}} \, \NDnt{s_{{\mathrm{2}}}} }{%
{\NDdruleSbetaXXapprOneName}{}%
}}


\newcommand{\NDdruleSbetaXXapprTwoName}[0]{\NDdrulename{Sbeta\_appr2}}
\newcommand{\NDdruleSbetaXXapprTwo}[1]{\NDdrule[#1]{%
\NDpremise{\NDnt{s_{{\mathrm{2}}}}  \leadsto_\beta  \NDnt{s'_{{\mathrm{2}}}}}%
}{
 \mathsf{app}_r\, \NDnt{s_{{\mathrm{1}}}} \, \NDnt{s_{{\mathrm{2}}}}   \leadsto_\beta   \mathsf{app}_r\, \NDnt{s_{{\mathrm{1}}}} \, \NDnt{s'_{{\mathrm{2}}}} }{%
{\NDdruleSbetaXXapprTwoName}{}%
}}


\newcommand{\NDdruleSbetaXXderelictName}[0]{\NDdrulename{Sbeta\_derelict}}
\newcommand{\NDdruleSbetaXXderelict}[1]{\NDdrule[#1]{%
}{
 \mathsf{derelict}\, \NDsym{(}   \mathsf{G}\, \NDnt{s}   \NDsym{)}   \leadsto_\beta  \NDnt{s}}{%
{\NDdruleSbetaXXderelictName}{}%
}}


\newcommand{\NDdruleSbetaXXapplLetName}[0]{\NDdrulename{Sbeta\_applLet}}
\newcommand{\NDdruleSbetaXXapplLet}[1]{\NDdrule[#1]{%
}{
 \mathsf{app}_l\, \NDsym{(}   \mathsf{let}\, \NDnt{s}  :  \NDnt{A} \,\mathsf{be}\, \NDnt{p} \,\mathsf{in}\, \NDnt{s_{{\mathrm{1}}}}   \NDsym{)} \, \NDnt{s_{{\mathrm{2}}}}   \leadsto_\beta   \mathsf{let}\, \NDnt{s}  :  \NDnt{A} \,\mathsf{be}\, \NDnt{p} \,\mathsf{in}\, \NDsym{(}   \mathsf{app}_l\, \NDnt{s_{{\mathrm{1}}}} \, \NDnt{s_{{\mathrm{2}}}}   \NDsym{)} }{%
{\NDdruleSbetaXXapplLetName}{}%
}}


\newcommand{\NDdruleSbetaXXapprLetName}[0]{\NDdrulename{Sbeta\_apprLet}}
\newcommand{\NDdruleSbetaXXapprLet}[1]{\NDdrule[#1]{%
}{
 \mathsf{app}_r\, \NDsym{(}   \mathsf{let}\, \NDnt{s}  :  \NDnt{A} \,\mathsf{be}\, \NDnt{p} \,\mathsf{in}\, \NDnt{s_{{\mathrm{1}}}}   \NDsym{)} \, \NDnt{s_{{\mathrm{2}}}}   \leadsto_\beta   \mathsf{let}\, \NDnt{s}  :  \NDnt{A} \,\mathsf{be}\, \NDnt{p} \,\mathsf{in}\, \NDsym{(}   \mathsf{app}_r\, \NDnt{s_{{\mathrm{1}}}} \, \NDnt{s_{{\mathrm{2}}}}   \NDsym{)} }{%
{\NDdruleSbetaXXapprLetName}{}%
}}


\newcommand{\NDdruleSbetaXXletLetName}[0]{\NDdrulename{Sbeta\_letLet}}
\newcommand{\NDdruleSbetaXXletLet}[1]{\NDdrule[#1]{%
}{
 \mathsf{let}\, \NDsym{(}   \mathsf{let}\, \NDnt{s_{{\mathrm{2}}}}  :  \NDnt{A} \,\mathsf{be}\, \NDnt{p_{{\mathrm{1}}}} \,\mathsf{in}\, \NDnt{s_{{\mathrm{1}}}}   \NDsym{)}  :  \NDnt{B} \,\mathsf{be}\, \NDnt{p_{{\mathrm{2}}}} \,\mathsf{in}\, \NDnt{s_{{\mathrm{3}}}}   \leadsto_\beta   \mathsf{let}\, \NDnt{s_{{\mathrm{2}}}}  :  \NDnt{A} \,\mathsf{be}\, \NDnt{p_{{\mathrm{1}}}} \,\mathsf{in}\,  \mathsf{let}\, \NDnt{s_{{\mathrm{1}}}}  :  \NDnt{B} \,\mathsf{be}\, \NDnt{p_{{\mathrm{2}}}} \,\mathsf{in}\, \NDnt{s_{{\mathrm{3}}}}  }{%
{\NDdruleSbetaXXletLetName}{}%
}}


\newcommand{\NDdruleSbetaXXletApplName}[0]{\NDdrulename{Sbeta\_letAppl}}
\newcommand{\NDdruleSbetaXXletAppl}[1]{\NDdrule[#1]{%
}{
 \mathsf{let}\, \NDnt{s_{{\mathrm{1}}}}  :  \NDnt{A} \,\mathsf{be}\, \NDnt{p} \,\mathsf{in}\, \NDsym{(}   \mathsf{app}_l\, \NDnt{s_{{\mathrm{1}}}} \, \NDnt{s_{{\mathrm{2}}}}   \NDsym{)}   \leadsto_\beta   \mathsf{app}_l\, \NDsym{(}   \mathsf{let}\, \NDnt{s_{{\mathrm{1}}}}  :  \NDnt{A} \,\mathsf{be}\, \NDnt{p} \,\mathsf{in}\, \NDnt{s_{{\mathrm{1}}}}   \NDsym{)} \, \NDsym{(}   \mathsf{let}\, \NDnt{s_{{\mathrm{1}}}}  :  \NDnt{A} \,\mathsf{be}\, \NDnt{p} \,\mathsf{in}\, \NDnt{s_{{\mathrm{2}}}}   \NDsym{)} }{%
{\NDdruleSbetaXXletApplName}{}%
}}


\newcommand{\NDdruleSbetaXXletApprName}[0]{\NDdrulename{Sbeta\_letAppr}}
\newcommand{\NDdruleSbetaXXletAppr}[1]{\NDdrule[#1]{%
}{
 \mathsf{let}\, \NDnt{s_{{\mathrm{1}}}}  :  \NDnt{A} \,\mathsf{be}\, \NDnt{p} \,\mathsf{in}\, \NDsym{(}   \mathsf{app}_r\, \NDnt{s_{{\mathrm{1}}}} \, \NDnt{s_{{\mathrm{2}}}}   \NDsym{)}   \leadsto_\beta   \mathsf{app}_r\, \NDsym{(}   \mathsf{let}\, \NDnt{s_{{\mathrm{1}}}}  :  \NDnt{A} \,\mathsf{be}\, \NDnt{p} \,\mathsf{in}\, \NDnt{s_{{\mathrm{1}}}}   \NDsym{)} \, \NDsym{(}   \mathsf{let}\, \NDnt{s_{{\mathrm{1}}}}  :  \NDnt{A} \,\mathsf{be}\, \NDnt{p} \,\mathsf{in}\, \NDnt{s_{{\mathrm{2}}}}   \NDsym{)} }{%
{\NDdruleSbetaXXletApprName}{}%
}}

\newcommand{\NDdefnsred}[1]{\begin{NDdefnblock}[#1]{$\NDnt{s_{{\mathrm{1}}}}  \leadsto_\beta  \NDnt{s_{{\mathrm{2}}}}$}{}
\NDusedrule{\NDdruleSbetaXXletUOne{}}
\NDusedrule{\NDdruleSbetaXXletTOne{}}
\NDusedrule{\NDdruleSbetaXXletTTwo{}}
\NDusedrule{\NDdruleSbetaXXletF{}}
\NDusedrule{\NDdruleSbetaXXlamL{}}
\NDusedrule{\NDdruleSbetaXXlamR{}}
\NDusedrule{\NDdruleSbetaXXapplOne{}}
\NDusedrule{\NDdruleSbetaXXapplTwo{}}
\NDusedrule{\NDdruleSbetaXXapprOne{}}
\NDusedrule{\NDdruleSbetaXXapprTwo{}}
\NDusedrule{\NDdruleSbetaXXderelict{}}
\NDusedrule{\NDdruleSbetaXXapplLet{}}
\NDusedrule{\NDdruleSbetaXXapprLet{}}
\NDusedrule{\NDdruleSbetaXXletLet{}}
\NDusedrule{\NDdruleSbetaXXletAppl{}}
\NDusedrule{\NDdruleSbetaXXletAppr{}}
\end{NDdefnblock}}


\newcommand{\NDdefnsReduction}{
\NDdefntred{}\NDdefnsred{}}

% defns Commuting
%% defn tcom
\newcommand{\NDdruleTcomXXunitEXXunitEName}[0]{\NDdrulename{Tcom\_unitE\_unitE}}
\newcommand{\NDdruleTcomXXunitEXXunitE}[1]{\NDdrule[#1]{%
}{
 \mathsf{let}\, \NDsym{(}   \mathsf{let}\, \NDnt{t_{{\mathrm{2}}}}  :   \mathsf{Unit}  \,\mathsf{be}\,  \mathsf{triv}  \,\mathsf{in}\, \NDnt{t_{{\mathrm{1}}}}   \NDsym{)}  :   \mathsf{Unit}  \,\mathsf{be}\,  \mathsf{triv}  \,\mathsf{in}\, \NDnt{t_{{\mathrm{3}}}}   \leadsto_\mathsf{c}   \mathsf{let}\, \NDnt{t_{{\mathrm{2}}}}  :   \mathsf{Unit}  \,\mathsf{be}\,  \mathsf{triv}  \,\mathsf{in}\, \NDsym{(}   \mathsf{let}\, \NDnt{t_{{\mathrm{1}}}}  :   \mathsf{Unit}  \,\mathsf{be}\,  \mathsf{triv}  \,\mathsf{in}\, \NDnt{t_{{\mathrm{3}}}}   \NDsym{)} }{%
{\NDdruleTcomXXunitEXXunitEName}{}%
}}


\newcommand{\NDdruleTcomXXunitEXXtenEName}[0]{\NDdrulename{Tcom\_unitE\_tenE}}
\newcommand{\NDdruleTcomXXunitEXXtenE}[1]{\NDdrule[#1]{%
}{
 \mathsf{let}\, \NDsym{(}   \mathsf{let}\, \NDnt{t_{{\mathrm{2}}}}  :   \mathsf{Unit}  \,\mathsf{be}\,  \mathsf{triv}  \,\mathsf{in}\, \NDnt{t_{{\mathrm{1}}}}   \NDsym{)}  :  \NDnt{X}  \otimes  \NDnt{Y} \,\mathsf{be}\, \NDmv{x}  \otimes  \NDmv{y} \,\mathsf{in}\, \NDnt{t_{{\mathrm{3}}}}   \leadsto_\mathsf{c}   \mathsf{let}\, \NDnt{t_{{\mathrm{2}}}}  :   \mathsf{Unit}  \,\mathsf{be}\,  \mathsf{triv}  \,\mathsf{in}\, \NDsym{(}   \mathsf{let}\, \NDnt{t_{{\mathrm{1}}}}  :  \NDnt{X}  \otimes  \NDnt{Y} \,\mathsf{be}\, \NDmv{x}  \otimes  \NDmv{y} \,\mathsf{in}\, \NDnt{t_{{\mathrm{3}}}}   \NDsym{)} }{%
{\NDdruleTcomXXunitEXXtenEName}{}%
}}


\newcommand{\NDdruleTcomXXunitEXXimpEName}[0]{\NDdrulename{Tcom\_unitE\_impE}}
\newcommand{\NDdruleTcomXXunitEXXimpE}[1]{\NDdrule[#1]{%
}{
 \NDsym{(}   \mathsf{let}\, \NDnt{t_{{\mathrm{2}}}}  :   \mathsf{Unit}  \,\mathsf{be}\,  \mathsf{triv}  \,\mathsf{in}\, \NDnt{t_{{\mathrm{1}}}}   \NDsym{)}   \NDnt{t_{{\mathrm{3}}}}   \leadsto_\mathsf{c}   \mathsf{let}\, \NDnt{t_{{\mathrm{2}}}}  :   \mathsf{Unit}  \,\mathsf{be}\,  \mathsf{triv}  \,\mathsf{in}\, \NDsym{(}   \NDnt{t_{{\mathrm{1}}}}   \NDnt{t_{{\mathrm{3}}}}   \NDsym{)} }{%
{\NDdruleTcomXXunitEXXimpEName}{}%
}}


\newcommand{\NDdruleTcomXXtenEXXunitEName}[0]{\NDdrulename{Tcom\_tenE\_unitE}}
\newcommand{\NDdruleTcomXXtenEXXunitE}[1]{\NDdrule[#1]{%
}{
 \mathsf{let}\, \NDsym{(}   \mathsf{let}\, \NDnt{t_{{\mathrm{2}}}}  :  \NDnt{X}  \otimes  \NDnt{Y} \,\mathsf{be}\, \NDmv{x}  \otimes  \NDmv{y} \,\mathsf{in}\, \NDnt{t_{{\mathrm{1}}}}   \NDsym{)}  :   \mathsf{Unit}  \,\mathsf{be}\,  \mathsf{triv}  \,\mathsf{in}\, \NDnt{t_{{\mathrm{3}}}}   \leadsto_\mathsf{c}   \mathsf{let}\, \NDnt{t_{{\mathrm{2}}}}  :  \NDnt{X}  \otimes  \NDnt{Y} \,\mathsf{be}\, \NDmv{x}  \otimes  \NDmv{y} \,\mathsf{in}\, \NDsym{(}   \mathsf{let}\, \NDnt{t_{{\mathrm{1}}}}  :   \mathsf{Unit}  \,\mathsf{be}\,  \mathsf{triv}  \,\mathsf{in}\, \NDnt{t_{{\mathrm{3}}}}   \NDsym{)} }{%
{\NDdruleTcomXXtenEXXunitEName}{}%
}}


\newcommand{\NDdruleTcomXXtenEXXtenEName}[0]{\NDdrulename{Tcom\_tenE\_tenE}}
\newcommand{\NDdruleTcomXXtenEXXtenE}[1]{\NDdrule[#1]{%
}{
 \mathsf{let}\, \NDsym{(}   \mathsf{let}\, \NDnt{t_{{\mathrm{2}}}}  :  \NDnt{X_{{\mathrm{2}}}}  \otimes  \NDnt{Y_{{\mathrm{2}}}} \,\mathsf{be}\, \NDmv{x}  \otimes  \NDmv{y} \,\mathsf{in}\, \NDnt{t_{{\mathrm{1}}}}   \NDsym{)}  :  \NDnt{X_{{\mathrm{1}}}}  \otimes  \NDnt{Y_{{\mathrm{1}}}} \,\mathsf{be}\, \NDmv{w}  \otimes  \NDmv{z} \,\mathsf{in}\, \NDnt{t_{{\mathrm{3}}}}   \leadsto_\mathsf{c}   \mathsf{let}\, \NDnt{t_{{\mathrm{2}}}}  :  \NDnt{X_{{\mathrm{2}}}}  \otimes  \NDnt{Y_{{\mathrm{2}}}} \,\mathsf{be}\, \NDmv{x}  \otimes  \NDmv{y} \,\mathsf{in}\, \NDsym{(}   \mathsf{let}\, \NDnt{t_{{\mathrm{1}}}}  :  \NDnt{X_{{\mathrm{1}}}}  \otimes  \NDnt{Y_{{\mathrm{1}}}} \,\mathsf{be}\, \NDmv{w}  \otimes  \NDmv{z} \,\mathsf{in}\, \NDnt{t_{{\mathrm{3}}}}   \NDsym{)} }{%
{\NDdruleTcomXXtenEXXtenEName}{}%
}}


\newcommand{\NDdruleTcomXXtenEXXimpEName}[0]{\NDdrulename{Tcom\_tenE\_impE}}
\newcommand{\NDdruleTcomXXtenEXXimpE}[1]{\NDdrule[#1]{%
}{
 \NDsym{(}   \mathsf{let}\, \NDnt{t_{{\mathrm{2}}}}  :  \NDnt{X_{{\mathrm{2}}}}  \otimes  \NDnt{Y_{{\mathrm{2}}}} \,\mathsf{be}\, \NDmv{x}  \otimes  \NDmv{y} \,\mathsf{in}\, \NDnt{t_{{\mathrm{1}}}}   \NDsym{)}   \NDnt{t_{{\mathrm{3}}}}   \leadsto_\mathsf{c}   \mathsf{let}\, \NDnt{t_{{\mathrm{2}}}}  :  \NDnt{X_{{\mathrm{2}}}}  \otimes  \NDnt{Y_{{\mathrm{2}}}} \,\mathsf{be}\, \NDmv{x}  \otimes  \NDmv{y} \,\mathsf{in}\, \NDsym{(}   \NDnt{t_{{\mathrm{1}}}}   \NDnt{t_{{\mathrm{3}}}}   \NDsym{)} }{%
{\NDdruleTcomXXtenEXXimpEName}{}%
}}


\newcommand{\NDdruleTcomXXimpEXXunitEName}[0]{\NDdrulename{Tcom\_impE\_unitE}}
\newcommand{\NDdruleTcomXXimpEXXunitE}[1]{\NDdrule[#1]{%
}{
 \mathsf{let}\, \NDsym{(}   \NDnt{t_{{\mathrm{1}}}}   \NDnt{t_{{\mathrm{2}}}}   \NDsym{)}  :   \mathsf{Unit}  \,\mathsf{be}\,  \mathsf{triv}  \,\mathsf{in}\, \NDnt{t_{{\mathrm{3}}}}   \leadsto_\mathsf{c}   \NDnt{t_{{\mathrm{1}}}}   \NDsym{(}   \mathsf{let}\, \NDnt{t_{{\mathrm{2}}}}  :   \mathsf{Unit}  \,\mathsf{be}\,  \mathsf{triv}  \,\mathsf{in}\, \NDnt{t_{{\mathrm{3}}}}   \NDsym{)} }{%
{\NDdruleTcomXXimpEXXunitEName}{}%
}}

\newcommand{\NDdefntcom}[1]{\begin{NDdefnblock}[#1]{$\NDnt{t_{{\mathrm{1}}}}  \leadsto_\mathsf{c}  \NDnt{t_{{\mathrm{2}}}}$}{}
\NDusedrule{\NDdruleTcomXXunitEXXunitE{}}
\NDusedrule{\NDdruleTcomXXunitEXXtenE{}}
\NDusedrule{\NDdruleTcomXXunitEXXimpE{}}
\NDusedrule{\NDdruleTcomXXtenEXXunitE{}}
\NDusedrule{\NDdruleTcomXXtenEXXtenE{}}
\NDusedrule{\NDdruleTcomXXtenEXXimpE{}}
\NDusedrule{\NDdruleTcomXXimpEXXunitE{}}
\end{NDdefnblock}}

%% defn scom
\newcommand{\NDdruleScomXXunitEXXunitEName}[0]{\NDdrulename{Scom\_unitE\_unitE}}
\newcommand{\NDdruleScomXXunitEXXunitE}[1]{\NDdrule[#1]{%
}{
 \mathsf{let}\, \NDsym{(}   \mathsf{let}\, \NDnt{s_{{\mathrm{2}}}}  :   \mathsf{Unit}  \,\mathsf{be}\,  \mathsf{triv}  \,\mathsf{in}\, \NDnt{s_{{\mathrm{1}}}}   \NDsym{)}  :   \mathsf{Unit}  \,\mathsf{be}\,  \mathsf{triv}  \,\mathsf{in}\, \NDnt{s_{{\mathrm{3}}}}   \leadsto_\mathsf{c}   \mathsf{let}\, \NDnt{s_{{\mathrm{2}}}}  :   \mathsf{Unit}  \,\mathsf{be}\,  \mathsf{triv}  \,\mathsf{in}\, \NDsym{(}   \mathsf{let}\, \NDnt{s_{{\mathrm{1}}}}  :   \mathsf{Unit}  \,\mathsf{be}\,  \mathsf{triv}  \,\mathsf{in}\, \NDnt{s_{{\mathrm{3}}}}   \NDsym{)} }{%
{\NDdruleScomXXunitEXXunitEName}{}%
}}


\newcommand{\NDdruleScomXXunitETwoXXunitEName}[0]{\NDdrulename{Scom\_unitE2\_unitE}}
\newcommand{\NDdruleScomXXunitETwoXXunitE}[1]{\NDdrule[#1]{%
}{
 \mathsf{let}\, \NDsym{(}   \mathsf{let}\, \NDnt{t}  :   \mathsf{Unit}  \,\mathsf{be}\,  \mathsf{triv}  \,\mathsf{in}\, \NDnt{s_{{\mathrm{1}}}}   \NDsym{)}  :   \mathsf{Unit}  \,\mathsf{be}\,  \mathsf{triv}  \,\mathsf{in}\, \NDnt{s_{{\mathrm{2}}}}   \leadsto_\mathsf{c}   \mathsf{let}\, \NDnt{t}  :   \mathsf{Unit}  \,\mathsf{be}\,  \mathsf{triv}  \,\mathsf{in}\, \NDsym{(}   \mathsf{let}\, \NDnt{s_{{\mathrm{1}}}}  :   \mathsf{Unit}  \,\mathsf{be}\,  \mathsf{triv}  \,\mathsf{in}\, \NDnt{s_{{\mathrm{2}}}}   \NDsym{)} }{%
{\NDdruleScomXXunitETwoXXunitEName}{}%
}}


\newcommand{\NDdruleScomXXunitEXXimprEName}[0]{\NDdrulename{Scom\_unitE\_imprE}}
\newcommand{\NDdruleScomXXunitEXXimprE}[1]{\NDdrule[#1]{%
}{
 \mathsf{app}_r\, \NDsym{(}   \mathsf{let}\, \NDnt{s_{{\mathrm{2}}}}  :   \mathsf{Unit}  \,\mathsf{be}\,  \mathsf{triv}  \,\mathsf{in}\, \NDnt{s_{{\mathrm{1}}}}   \NDsym{)} \, \NDnt{s_{{\mathrm{3}}}}   \leadsto_\mathsf{c}   \mathsf{let}\, \NDnt{s_{{\mathrm{2}}}}  :   \mathsf{Unit}  \,\mathsf{be}\,  \mathsf{triv}  \,\mathsf{in}\, \NDsym{(}   \mathsf{app}_r\, \NDnt{s_{{\mathrm{1}}}} \, \NDnt{s_{{\mathrm{3}}}}   \NDsym{)} }{%
{\NDdruleScomXXunitEXXimprEName}{}%
}}


\newcommand{\NDdruleScomXXunitETwoXXimprEName}[0]{\NDdrulename{Scom\_unitE2\_imprE}}
\newcommand{\NDdruleScomXXunitETwoXXimprE}[1]{\NDdrule[#1]{%
}{
 \mathsf{app}_r\, \NDsym{(}   \mathsf{let}\, \NDnt{t}  :   \mathsf{Unit}  \,\mathsf{be}\,  \mathsf{triv}  \,\mathsf{in}\, \NDnt{s_{{\mathrm{1}}}}   \NDsym{)} \, \NDnt{s_{{\mathrm{2}}}}   \leadsto_\mathsf{c}   \mathsf{let}\, \NDnt{t}  :   \mathsf{Unit}  \,\mathsf{be}\,  \mathsf{triv}  \,\mathsf{in}\, \NDsym{(}   \mathsf{app}_r\, \NDnt{s_{{\mathrm{1}}}} \, \NDnt{s_{{\mathrm{2}}}}   \NDsym{)} }{%
{\NDdruleScomXXunitETwoXXimprEName}{}%
}}


\newcommand{\NDdruleScomXXunitEXXFEName}[0]{\NDdrulename{Scom\_unitE\_FE}}
\newcommand{\NDdruleScomXXunitEXXFE}[1]{\NDdrule[#1]{%
}{
 \mathsf{let}\, \NDsym{(}   \mathsf{let}\, \NDnt{s_{{\mathrm{2}}}}  :   \mathsf{Unit}  \,\mathsf{be}\,  \mathsf{triv}  \,\mathsf{in}\, \NDnt{s_{{\mathrm{1}}}}   \NDsym{)}  :   \mathsf{F} \NDnt{X}  \,\mathsf{be}\,  \mathsf{F}\, \NDmv{x}  \,\mathsf{in}\, \NDnt{s_{{\mathrm{3}}}}   \leadsto_\mathsf{c}   \mathsf{let}\, \NDnt{s_{{\mathrm{2}}}}  :   \mathsf{Unit}  \,\mathsf{be}\,  \mathsf{triv}  \,\mathsf{in}\, \NDsym{(}   \mathsf{let}\, \NDnt{s_{{\mathrm{1}}}}  :   \mathsf{F} \NDnt{X}  \,\mathsf{be}\,  \mathsf{F}\, \NDmv{x}  \,\mathsf{in}\, \NDnt{s_{{\mathrm{3}}}}   \NDsym{)} }{%
{\NDdruleScomXXunitEXXFEName}{}%
}}


\newcommand{\NDdruleScomXXunitETwoXXFEName}[0]{\NDdrulename{Scom\_unitE2\_FE}}
\newcommand{\NDdruleScomXXunitETwoXXFE}[1]{\NDdrule[#1]{%
}{
 \mathsf{let}\, \NDsym{(}   \mathsf{let}\, \NDnt{t}  :   \mathsf{Unit}  \,\mathsf{be}\,  \mathsf{triv}  \,\mathsf{in}\, \NDnt{s_{{\mathrm{1}}}}   \NDsym{)}  :   \mathsf{F} \NDnt{X}  \,\mathsf{be}\,  \mathsf{F}\, \NDmv{x}  \,\mathsf{in}\, \NDnt{s_{{\mathrm{2}}}}   \leadsto_\mathsf{c}   \mathsf{let}\, \NDnt{t}  :   \mathsf{Unit}  \,\mathsf{be}\,  \mathsf{triv}  \,\mathsf{in}\, \NDsym{(}   \mathsf{let}\, \NDnt{s_{{\mathrm{1}}}}  :   \mathsf{F} \NDnt{X}  \,\mathsf{be}\,  \mathsf{F}\, \NDmv{x}  \,\mathsf{in}\, \NDnt{s_{{\mathrm{2}}}}   \NDsym{)} }{%
{\NDdruleScomXXunitETwoXXFEName}{}%
}}


\newcommand{\NDdruleScomXXtenEXXunitEName}[0]{\NDdrulename{Scom\_tenE\_unitE}}
\newcommand{\NDdruleScomXXtenEXXunitE}[1]{\NDdrule[#1]{%
}{
 \mathsf{let}\, \NDsym{(}   \mathsf{let}\, \NDnt{s_{{\mathrm{2}}}}  :  \NDnt{A}  \triangleright  \NDnt{B} \,\mathsf{be}\, \NDmv{x}  \triangleright  \NDmv{y} \,\mathsf{in}\, \NDnt{s_{{\mathrm{1}}}}   \NDsym{)}  :   \mathsf{Unit}  \,\mathsf{be}\,  \mathsf{triv}  \,\mathsf{in}\, \NDnt{s_{{\mathrm{3}}}}   \leadsto_\mathsf{c}   \mathsf{let}\, \NDnt{s_{{\mathrm{2}}}}  :  \NDnt{A}  \triangleright  \NDnt{B} \,\mathsf{be}\, \NDmv{x}  \triangleright  \NDmv{y} \,\mathsf{in}\, \NDsym{(}   \mathsf{let}\, \NDnt{s_{{\mathrm{1}}}}  :   \mathsf{Unit}  \,\mathsf{be}\,  \mathsf{triv}  \,\mathsf{in}\, \NDnt{s_{{\mathrm{3}}}}   \NDsym{)} }{%
{\NDdruleScomXXtenEXXunitEName}{}%
}}


\newcommand{\NDdruleScomXXtenETwoXXunitEName}[0]{\NDdrulename{Scom\_tenE2\_unitE}}
\newcommand{\NDdruleScomXXtenETwoXXunitE}[1]{\NDdrule[#1]{%
}{
 \mathsf{let}\, \NDsym{(}   \mathsf{let}\, \NDnt{t}  :  \NDnt{X}  \otimes  \NDnt{Y} \,\mathsf{be}\, \NDmv{x}  \otimes  \NDmv{y} \,\mathsf{in}\, \NDnt{s_{{\mathrm{1}}}}   \NDsym{)}  :   \mathsf{Unit}  \,\mathsf{be}\,  \mathsf{triv}  \,\mathsf{in}\, \NDnt{s_{{\mathrm{2}}}}   \leadsto_\mathsf{c}   \mathsf{let}\, \NDnt{t}  :  \NDnt{X}  \otimes  \NDnt{Y} \,\mathsf{be}\, \NDmv{x}  \otimes  \NDmv{y} \,\mathsf{in}\, \NDsym{(}   \mathsf{let}\, \NDnt{s_{{\mathrm{1}}}}  :   \mathsf{Unit}  \,\mathsf{be}\,  \mathsf{triv}  \,\mathsf{in}\, \NDnt{s_{{\mathrm{2}}}}   \NDsym{)} }{%
{\NDdruleScomXXtenETwoXXunitEName}{}%
}}


\newcommand{\NDdruleScomXXtenEXXtenEName}[0]{\NDdrulename{Scom\_tenE\_tenE}}
\newcommand{\NDdruleScomXXtenEXXtenE}[1]{\NDdrule[#1]{%
}{
 \mathsf{let}\, \NDsym{(}   \mathsf{let}\, \NDnt{s_{{\mathrm{2}}}}  :  \NDnt{A_{{\mathrm{2}}}}  \triangleright  \NDnt{B_{{\mathrm{2}}}} \,\mathsf{be}\, \NDmv{x}  \triangleright  \NDmv{y} \,\mathsf{in}\, \NDnt{s_{{\mathrm{1}}}}   \NDsym{)}  :  \NDnt{A_{{\mathrm{1}}}}  \triangleright  \NDnt{B_{{\mathrm{1}}}} \,\mathsf{be}\, \NDmv{w}  \triangleright  \NDmv{z} \,\mathsf{in}\, \NDnt{s_{{\mathrm{3}}}}   \leadsto_\mathsf{c}   \mathsf{let}\, \NDnt{s_{{\mathrm{2}}}}  :  \NDnt{A_{{\mathrm{2}}}}  \triangleright  \NDnt{B_{{\mathrm{2}}}} \,\mathsf{be}\, \NDmv{x}  \triangleright  \NDmv{y} \,\mathsf{in}\, \NDsym{(}   \mathsf{let}\, \NDnt{s_{{\mathrm{1}}}}  :  \NDnt{A_{{\mathrm{1}}}}  \triangleright  \NDnt{B_{{\mathrm{1}}}} \,\mathsf{be}\, \NDmv{w}  \triangleright  \NDmv{z} \,\mathsf{in}\, \NDnt{s_{{\mathrm{3}}}}   \NDsym{)} }{%
{\NDdruleScomXXtenEXXtenEName}{}%
}}


\newcommand{\NDdruleScomXXtenETwoXXtenEName}[0]{\NDdrulename{Scom\_tenE2\_tenE}}
\newcommand{\NDdruleScomXXtenETwoXXtenE}[1]{\NDdrule[#1]{%
}{
 \mathsf{let}\, \NDsym{(}   \mathsf{let}\, \NDnt{t}  :  \NDnt{X}  \otimes  \NDnt{Y} \,\mathsf{be}\, \NDmv{x}  \otimes  \NDmv{y} \,\mathsf{in}\, \NDnt{s_{{\mathrm{1}}}}   \NDsym{)}  :  \NDnt{A_{{\mathrm{1}}}}  \triangleright  \NDnt{B_{{\mathrm{1}}}} \,\mathsf{be}\, \NDmv{w}  \triangleright  \NDmv{z} \,\mathsf{in}\, \NDnt{s_{{\mathrm{2}}}}   \leadsto_\mathsf{c}   \mathsf{let}\, \NDnt{t}  :  \NDnt{X}  \otimes  \NDnt{Y} \,\mathsf{be}\, \NDmv{x}  \otimes  \NDmv{y} \,\mathsf{in}\, \NDsym{(}   \mathsf{let}\, \NDnt{s_{{\mathrm{1}}}}  :  \NDnt{A_{{\mathrm{1}}}}  \triangleright  \NDnt{B_{{\mathrm{1}}}} \,\mathsf{be}\, \NDmv{w}  \triangleright  \NDmv{z} \,\mathsf{in}\, \NDnt{s_{{\mathrm{2}}}}   \NDsym{)} }{%
{\NDdruleScomXXtenETwoXXtenEName}{}%
}}


\newcommand{\NDdruleScomXXtenEXXimprEName}[0]{\NDdrulename{Scom\_tenE\_imprE}}
\newcommand{\NDdruleScomXXtenEXXimprE}[1]{\NDdrule[#1]{%
}{
 \mathsf{app}_r\, \NDsym{(}   \mathsf{let}\, \NDnt{s_{{\mathrm{2}}}}  :  \NDnt{A_{{\mathrm{2}}}}  \triangleright  \NDnt{B_{{\mathrm{2}}}} \,\mathsf{be}\, \NDmv{x}  \triangleright  \NDmv{y} \,\mathsf{in}\, \NDnt{s_{{\mathrm{1}}}}   \NDsym{)} \, \NDnt{s_{{\mathrm{3}}}}   \leadsto_\mathsf{c}   \mathsf{let}\, \NDnt{s_{{\mathrm{2}}}}  :  \NDnt{A_{{\mathrm{2}}}}  \triangleright  \NDnt{B_{{\mathrm{2}}}} \,\mathsf{be}\, \NDmv{x}  \triangleright  \NDmv{y} \,\mathsf{in}\, \NDsym{(}   \mathsf{app}_r\, \NDnt{s_{{\mathrm{1}}}} \, \NDnt{s_{{\mathrm{3}}}}   \NDsym{)} }{%
{\NDdruleScomXXtenEXXimprEName}{}%
}}


\newcommand{\NDdruleScomXXtenETwoXXimprEName}[0]{\NDdrulename{Scom\_tenE2\_imprE}}
\newcommand{\NDdruleScomXXtenETwoXXimprE}[1]{\NDdrule[#1]{%
}{
 \mathsf{app}_r\, \NDsym{(}   \mathsf{let}\, \NDnt{t}  :  \NDnt{X}  \otimes  \NDnt{Y} \,\mathsf{be}\, \NDmv{x}  \otimes  \NDmv{y} \,\mathsf{in}\, \NDnt{s_{{\mathrm{1}}}}   \NDsym{)} \, \NDnt{s_{{\mathrm{2}}}}   \leadsto_\mathsf{c}   \mathsf{let}\, \NDnt{t}  :  \NDnt{X}  \otimes  \NDnt{Y} \,\mathsf{be}\, \NDmv{x}  \otimes  \NDmv{y} \,\mathsf{in}\, \NDsym{(}   \mathsf{app}_r\, \NDnt{s_{{\mathrm{1}}}} \, \NDnt{s_{{\mathrm{2}}}}   \NDsym{)} }{%
{\NDdruleScomXXtenETwoXXimprEName}{}%
}}


\newcommand{\NDdruleScomXXtenEXXimplEName}[0]{\NDdrulename{Scom\_tenE\_implE}}
\newcommand{\NDdruleScomXXtenEXXimplE}[1]{\NDdrule[#1]{%
}{
 \mathsf{app}_l\, \NDsym{(}   \mathsf{let}\, \NDnt{s_{{\mathrm{2}}}}  :  \NDnt{A_{{\mathrm{2}}}}  \triangleright  \NDnt{B_{{\mathrm{2}}}} \,\mathsf{be}\, \NDmv{x}  \triangleright  \NDmv{y} \,\mathsf{in}\, \NDnt{s_{{\mathrm{1}}}}   \NDsym{)} \, \NDnt{s_{{\mathrm{3}}}}   \leadsto_\mathsf{c}   \mathsf{let}\, \NDnt{s_{{\mathrm{2}}}}  :  \NDnt{A_{{\mathrm{2}}}}  \triangleright  \NDnt{B_{{\mathrm{2}}}} \,\mathsf{be}\, \NDmv{x}  \triangleright  \NDmv{y} \,\mathsf{in}\, \NDsym{(}   \mathsf{app}_l\, \NDnt{s_{{\mathrm{1}}}} \, \NDnt{s_{{\mathrm{3}}}}   \NDsym{)} }{%
{\NDdruleScomXXtenEXXimplEName}{}%
}}


\newcommand{\NDdruleScomXXtenETwoXXimplEName}[0]{\NDdrulename{Scom\_tenE2\_implE}}
\newcommand{\NDdruleScomXXtenETwoXXimplE}[1]{\NDdrule[#1]{%
}{
 \mathsf{app}_l\, \NDsym{(}   \mathsf{let}\, \NDnt{t}  :  \NDnt{X}  \otimes  \NDnt{Y} \,\mathsf{be}\, \NDmv{x}  \otimes  \NDmv{y} \,\mathsf{in}\, \NDnt{s_{{\mathrm{1}}}}   \NDsym{)} \, \NDnt{s_{{\mathrm{2}}}}   \leadsto_\mathsf{c}   \mathsf{let}\, \NDnt{t}  :  \NDnt{X}  \otimes  \NDnt{Y} \,\mathsf{be}\, \NDmv{x}  \otimes  \NDmv{y} \,\mathsf{in}\, \NDsym{(}   \mathsf{app}_l\, \NDnt{s_{{\mathrm{1}}}} \, \NDnt{s_{{\mathrm{2}}}}   \NDsym{)} }{%
{\NDdruleScomXXtenETwoXXimplEName}{}%
}}


\newcommand{\NDdruleScomXXtenEXXFEName}[0]{\NDdrulename{Scom\_tenE\_FE}}
\newcommand{\NDdruleScomXXtenEXXFE}[1]{\NDdrule[#1]{%
}{
 \mathsf{let}\, \NDsym{(}   \mathsf{let}\, \NDnt{s_{{\mathrm{2}}}}  :  \NDnt{A}  \triangleright  \NDnt{B} \,\mathsf{be}\, \NDmv{x}  \triangleright  \NDmv{y} \,\mathsf{in}\, \NDnt{s_{{\mathrm{1}}}}   \NDsym{)}  :   \mathsf{F} \NDnt{X}  \,\mathsf{be}\,  \mathsf{F}\, \NDmv{z}  \,\mathsf{in}\, \NDnt{s_{{\mathrm{3}}}}   \leadsto_\mathsf{c}   \mathsf{let}\, \NDnt{s_{{\mathrm{2}}}}  :  \NDnt{A}  \triangleright  \NDnt{B} \,\mathsf{be}\, \NDmv{x}  \triangleright  \NDmv{y} \,\mathsf{in}\, \NDsym{(}   \mathsf{let}\, \NDnt{s_{{\mathrm{1}}}}  :   \mathsf{F} \NDnt{X}  \,\mathsf{be}\,  \mathsf{F}\, \NDmv{z}  \,\mathsf{in}\, \NDnt{s_{{\mathrm{3}}}}   \NDsym{)} }{%
{\NDdruleScomXXtenEXXFEName}{}%
}}


\newcommand{\NDdruleScomXXtenETwoXXFEName}[0]{\NDdrulename{Scom\_tenE2\_FE}}
\newcommand{\NDdruleScomXXtenETwoXXFE}[1]{\NDdrule[#1]{%
}{
 \mathsf{let}\, \NDsym{(}   \mathsf{let}\, \NDnt{t}  :  \NDnt{X}  \otimes  \NDnt{Y} \,\mathsf{be}\, \NDmv{x}  \otimes  \NDmv{y} \,\mathsf{in}\, \NDnt{s_{{\mathrm{1}}}}   \NDsym{)}  :   \mathsf{F} \NDnt{Z}  \,\mathsf{be}\,  \mathsf{F}\, \NDmv{z}  \,\mathsf{in}\, \NDnt{s_{{\mathrm{3}}}}   \leadsto_\mathsf{c}   \mathsf{let}\, \NDnt{t}  :  \NDnt{X}  \otimes  \NDnt{Y} \,\mathsf{be}\, \NDmv{x}  \otimes  \NDmv{y} \,\mathsf{in}\, \NDsym{(}   \mathsf{let}\, \NDnt{s_{{\mathrm{1}}}}  :   \mathsf{F} \NDnt{Z}  \,\mathsf{be}\,  \mathsf{F}\, \NDmv{z}  \,\mathsf{in}\, \NDnt{s_{{\mathrm{3}}}}   \NDsym{)} }{%
{\NDdruleScomXXtenETwoXXFEName}{}%
}}


\newcommand{\NDdruleScomXXFEXXunitEName}[0]{\NDdrulename{Scom\_FE\_unitE}}
\newcommand{\NDdruleScomXXFEXXunitE}[1]{\NDdrule[#1]{%
}{
 \mathsf{let}\, \NDsym{(}   \mathsf{let}\, \NDnt{s_{{\mathrm{2}}}}  :   \mathsf{F} \NDnt{X}  \,\mathsf{be}\,  \mathsf{F}\, \NDmv{x}  \,\mathsf{in}\, \NDnt{s_{{\mathrm{1}}}}   \NDsym{)}  :   \mathsf{Unit}  \,\mathsf{be}\,  \mathsf{triv}  \,\mathsf{in}\, \NDnt{s_{{\mathrm{3}}}}   \leadsto_\mathsf{c}   \mathsf{let}\, \NDnt{s_{{\mathrm{2}}}}  :   \mathsf{F} \NDnt{X}  \,\mathsf{be}\,  \mathsf{F}\, \NDmv{x}  \,\mathsf{in}\, \NDsym{(}   \mathsf{let}\, \NDnt{s_{{\mathrm{1}}}}  :   \mathsf{Unit}  \,\mathsf{be}\,  \mathsf{triv}  \,\mathsf{in}\, \NDnt{s_{{\mathrm{2}}}}   \NDsym{)} }{%
{\NDdruleScomXXFEXXunitEName}{}%
}}


\newcommand{\NDdruleScomXXFEXXtenEName}[0]{\NDdrulename{Scom\_FE\_tenE}}
\newcommand{\NDdruleScomXXFEXXtenE}[1]{\NDdrule[#1]{%
}{
 \mathsf{let}\, \NDsym{(}   \mathsf{let}\, \NDnt{s_{{\mathrm{2}}}}  :   \mathsf{F} \NDnt{X}  \,\mathsf{be}\,  \mathsf{F}\, \NDmv{x}  \,\mathsf{in}\, \NDnt{s_{{\mathrm{1}}}}   \NDsym{)}  :  \NDnt{A}  \triangleright  \NDnt{B} \,\mathsf{be}\, \NDmv{x}  \triangleright  \NDmv{y} \,\mathsf{in}\, \NDnt{s_{{\mathrm{3}}}}   \leadsto_\mathsf{c}   \mathsf{let}\, \NDnt{s_{{\mathrm{2}}}}  :   \mathsf{F} \NDnt{X}  \,\mathsf{be}\,  \mathsf{F}\, \NDmv{x}  \,\mathsf{in}\, \NDsym{(}   \mathsf{let}\, \NDnt{s_{{\mathrm{1}}}}  :  \NDnt{A}  \triangleright  \NDnt{B} \,\mathsf{be}\, \NDmv{x}  \triangleright  \NDmv{y} \,\mathsf{in}\, \NDnt{s_{{\mathrm{3}}}}   \NDsym{)} }{%
{\NDdruleScomXXFEXXtenEName}{}%
}}


\newcommand{\NDdruleScomXXFEXXimprEName}[0]{\NDdrulename{Scom\_FE\_imprE}}
\newcommand{\NDdruleScomXXFEXXimprE}[1]{\NDdrule[#1]{%
}{
 \mathsf{app}_r\, \NDsym{(}   \mathsf{let}\, \NDnt{s_{{\mathrm{2}}}}  :   \mathsf{F} \NDnt{X}  \,\mathsf{be}\,  \mathsf{F}\, \NDmv{x}  \,\mathsf{in}\, \NDnt{s_{{\mathrm{1}}}}   \NDsym{)} \, \NDnt{s_{{\mathrm{3}}}}   \leadsto_\mathsf{c}   \mathsf{let}\, \NDnt{s_{{\mathrm{2}}}}  :   \mathsf{F} \NDnt{X}  \,\mathsf{be}\,  \mathsf{F}\, \NDmv{x}  \,\mathsf{in}\, \NDsym{(}   \mathsf{app}_r\, \NDnt{s_{{\mathrm{1}}}} \, \NDnt{s_{{\mathrm{3}}}}   \NDsym{)} }{%
{\NDdruleScomXXFEXXimprEName}{}%
}}


\newcommand{\NDdruleScomXXFEXXimplEName}[0]{\NDdrulename{Scom\_FE\_implE}}
\newcommand{\NDdruleScomXXFEXXimplE}[1]{\NDdrule[#1]{%
}{
 \mathsf{app}_l\, \NDsym{(}   \mathsf{let}\, \NDnt{s_{{\mathrm{2}}}}  :   \mathsf{F} \NDnt{X}  \,\mathsf{be}\,  \mathsf{F}\, \NDmv{x}  \,\mathsf{in}\, \NDnt{s_{{\mathrm{1}}}}   \NDsym{)} \, \NDnt{s_{{\mathrm{3}}}}   \leadsto_\mathsf{c}   \mathsf{let}\, \NDnt{s_{{\mathrm{2}}}}  :   \mathsf{F} \NDnt{X}  \,\mathsf{be}\,  \mathsf{F}\, \NDmv{x}  \,\mathsf{in}\, \NDsym{(}   \mathsf{app}_l\, \NDnt{s_{{\mathrm{1}}}} \, \NDnt{s_{{\mathrm{3}}}}   \NDsym{)} }{%
{\NDdruleScomXXFEXXimplEName}{}%
}}


\newcommand{\NDdruleScomXXFEXXFEName}[0]{\NDdrulename{Scom\_FE\_FE}}
\newcommand{\NDdruleScomXXFEXXFE}[1]{\NDdrule[#1]{%
}{
 \mathsf{let}\, \NDsym{(}   \mathsf{let}\, \NDnt{s_{{\mathrm{2}}}}  :   \mathsf{F} \NDnt{X}  \,\mathsf{be}\,  \mathsf{F}\, \NDmv{x}  \,\mathsf{in}\, \NDnt{s_{{\mathrm{1}}}}   \NDsym{)}  :   \mathsf{F} \NDnt{Y}  \,\mathsf{be}\,  \mathsf{F}\, \NDmv{y}  \,\mathsf{in}\, \NDnt{s_{{\mathrm{3}}}}   \leadsto_\mathsf{c}   \mathsf{let}\, \NDnt{s_{{\mathrm{2}}}}  :   \mathsf{F} \NDnt{X}  \,\mathsf{be}\,  \mathsf{F}\, \NDmv{x}  \,\mathsf{in}\, \NDsym{(}   \mathsf{let}\, \NDnt{s_{{\mathrm{1}}}}  :   \mathsf{F} \NDnt{Y}  \,\mathsf{be}\,  \mathsf{F}\, \NDmv{y}  \,\mathsf{in}\, \NDnt{s_{{\mathrm{3}}}}   \NDsym{)} }{%
{\NDdruleScomXXFEXXFEName}{}%
}}

\newcommand{\NDdefnscom}[1]{\begin{NDdefnblock}[#1]{$\NDnt{s_{{\mathrm{1}}}}  \leadsto_\mathsf{c}  \NDnt{s_{{\mathrm{2}}}}$}{}
\NDusedrule{\NDdruleScomXXunitEXXunitE{}}
\NDusedrule{\NDdruleScomXXunitETwoXXunitE{}}
\NDusedrule{\NDdruleScomXXunitEXXimprE{}}
\NDusedrule{\NDdruleScomXXunitETwoXXimprE{}}
\NDusedrule{\NDdruleScomXXunitEXXFE{}}
\NDusedrule{\NDdruleScomXXunitETwoXXFE{}}
\NDusedrule{\NDdruleScomXXtenEXXunitE{}}
\NDusedrule{\NDdruleScomXXtenETwoXXunitE{}}
\NDusedrule{\NDdruleScomXXtenEXXtenE{}}
\NDusedrule{\NDdruleScomXXtenETwoXXtenE{}}
\NDusedrule{\NDdruleScomXXtenEXXimprE{}}
\NDusedrule{\NDdruleScomXXtenETwoXXimprE{}}
\NDusedrule{\NDdruleScomXXtenEXXimplE{}}
\NDusedrule{\NDdruleScomXXtenETwoXXimplE{}}
\NDusedrule{\NDdruleScomXXtenEXXFE{}}
\NDusedrule{\NDdruleScomXXtenETwoXXFE{}}
\NDusedrule{\NDdruleScomXXFEXXunitE{}}
\NDusedrule{\NDdruleScomXXFEXXtenE{}}
\NDusedrule{\NDdruleScomXXFEXXimprE{}}
\NDusedrule{\NDdruleScomXXFEXXimplE{}}
\NDusedrule{\NDdruleScomXXFEXXFE{}}
\end{NDdefnblock}}


\newcommand{\NDdefnsCommuting}{
\NDdefntcom{}\NDdefnscom{}}

\newcommand{\NDdefnss}{
\NDdefnsJtype
\NDdefnsReduction
\NDdefnsCommuting
}

\newcommand{\NDall}{\NDmetavars\\[0pt]
\NDgrammar\\[5.0mm]
\NDdefnss}


\renewcommand{\NDdrule}[4][]{{\displaystyle\frac{\begin{array}{l}#2\end{array}}{#3}\,#4}}
\renewcommand{\NDdruleTXXidName}{\mathcal{C}\text{-ax}}
\renewcommand{\NDdruleTXXunitIName}{\mathcal{C}\text{-}\mathsf{Unit}_I}
\renewcommand{\NDdruleTXXunitEName}{\mathcal{C}\text{-}\mathsf{Unit}_E}
\renewcommand{\NDdruleTXXtenIName}{\mathcal{C}\text{-}\otimes_I}
\renewcommand{\NDdruleTXXtenEName}{\mathcal{C}\text{-}\otimes_E}
\renewcommand{\NDdruleTXXimpIName}{\mathcal{C}\text{-}\multimap_I}
\renewcommand{\NDdruleTXXimpEName}{\mathcal{C}\text{-}\multimap_E}
\renewcommand{\NDdruleTXXGIName}{\mathcal{C}\text{-}\mathsf{G}_I}
\renewcommand{\NDdruleTXXbetaName}{\mathcal{C}\text{-}\mathsf{ex}}
\renewcommand{\NDdruleTXXcutName}{\mathcal{C}\text{-}\mathsf{Cut}}
\renewcommand{\NDdruleSXXidName}{\mathcal{L}\text{-ax}}
\renewcommand{\NDdruleSXXunitIName}{\mathcal{L}\text{-}\mathsf{Unit}_I}
\renewcommand{\NDdruleSXXunitEOneName}{\mathcal{LC}\text{-}\mathsf{Unit}_E}
\renewcommand{\NDdruleSXXunitETwoName}{\mathcal{L}\text{-}\mathsf{Unit}_E}
\renewcommand{\NDdruleSXXtenIName}{\mathcal{L}\text{-}\otimes_I}
\renewcommand{\NDdruleSXXtenEOneName}{\mathcal{LC}\text{-}\otimes_E}
\renewcommand{\NDdruleSXXtenETwoName}{\mathcal{L}\text{-}\otimes_E}
\renewcommand{\NDdruleSXXimprIName}{\mathcal{L}\text{-}\rightharpoonup_I}
\renewcommand{\NDdruleSXXimprEName}{\mathcal{L}\text{-}\rightharpoonup_E}
\renewcommand{\NDdruleSXXimplIName}{\mathcal{L}\text{-}\leftharpoonup_I}
\renewcommand{\NDdruleSXXimplEName}{\mathcal{L}\text{-}\leftharpoonup_E}
\renewcommand{\NDdruleSXXFIName}{\mathcal{L}\text{-}\mathsf{F}_I}
\renewcommand{\NDdruleSXXFEName}{\mathcal{L}\text{-}\mathsf{F}_E}
\renewcommand{\NDdruleSXXGEName}{\mathcal{L}\text{-}\mathsf{G}_E}
\renewcommand{\NDdruleSXXbetaName}{\mathcal{L}\text{-}\mathsf{ex}}
\renewcommand{\NDdruleSXXcutOneName}{\mathcal{LC}\text{-}\mathsf{Cut}}
\renewcommand{\NDdruleSXXcutTwoName}{\mathcal{L}\text{-}\mathsf{Cut}}

% generated by Ott 0.25 from: ElleType/ElleType.ott
\newcommand{\SCdrule}[4][]{{\displaystyle\frac{\begin{array}{l}#2\end{array}}{#3}\quad\SCdrulename{#4}}}
\newcommand{\SCusedrule}[1]{\[#1\]}
\newcommand{\SCpremise}[1]{ #1 \\}
\newenvironment{SCdefnblock}[3][]{ \framebox{\mbox{#2}} \quad #3 \\[0pt]}{}
\newenvironment{SCfundefnblock}[3][]{ \framebox{\mbox{#2}} \quad #3 \\[0pt]\begin{displaymath}\begin{array}{l}}{\end{array}\end{displaymath}}
\newcommand{\SCfunclause}[2]{ #1 \equiv #2 \\}
\newcommand{\SCnt}[1]{\mathit{#1}}
\newcommand{\SCmv}[1]{\mathit{#1}}
\newcommand{\SCkw}[1]{\mathbf{#1}}
\newcommand{\SCsym}[1]{#1}
\newcommand{\SCcom}[1]{\text{#1}}
\newcommand{\SCdrulename}[1]{\textsc{#1}}
\newcommand{\SCcomplu}[5]{\overline{#1}^{\,#2\in #3 #4 #5}}
\newcommand{\SCcompu}[3]{\overline{#1}^{\,#2<#3}}
\newcommand{\SCcomp}[2]{\overline{#1}^{\,#2}}
\newcommand{\SCgrammartabular}[1]{\begin{supertabular}{llcllllll}#1\end{supertabular}}
\newcommand{\SCmetavartabular}[1]{\begin{supertabular}{ll}#1\end{supertabular}}
\newcommand{\SCrulehead}[3]{$#1$ & & $#2$ & & & \multicolumn{2}{l}{#3}}
\newcommand{\SCprodline}[6]{& & $#1$ & $#2$ & $#3 #4$ & $#5$ & $#6$}
\newcommand{\SCfirstprodline}[6]{\SCprodline{#1}{#2}{#3}{#4}{#5}{#6}}
\newcommand{\SClongprodline}[2]{& & $#1$ & \multicolumn{4}{l}{$#2$}}
\newcommand{\SCfirstlongprodline}[2]{\SClongprodline{#1}{#2}}
\newcommand{\SCbindspecprodline}[6]{\SCprodline{#1}{#2}{#3}{#4}{#5}{#6}}
\newcommand{\SCprodnewline}{\\}
\newcommand{\SCinterrule}{\\[5.0mm]}
\newcommand{\SCafterlastrule}{\\}
\newcommand{\SCmetavars}{
\SCmetavartabular{
 $ \SCmv{vars} ,\, \SCmv{n} ,\, \SCmv{a} ,\, \SCmv{x} ,\, \SCmv{y} ,\, \SCmv{z} ,\, \SCmv{w} ,\, \SCmv{m} ,\, \SCmv{o} $ &  \\
 $ \SCmv{ivar} ,\, \SCmv{i} ,\, \SCmv{k} ,\, \SCmv{j} ,\, \SCmv{l} $ &  \\
 $ \SCmv{const} ,\, \SCmv{b} $ &  \\
}}

\newcommand{\SCA}{
\SCrulehead{\SCnt{A}  ,\ \SCnt{B}  ,\ \SCnt{C}  ,\ D}{::=}{}\SCprodnewline
\SCfirstprodline{|}{ \mathsf{B} }{}{}{}{}\SCprodnewline
\SCprodline{|}{ \mathsf{Unit} }{}{}{}{}\SCprodnewline
\SCprodline{|}{\SCnt{A}  \triangleright  \SCnt{B}}{}{}{}{}\SCprodnewline
\SCprodline{|}{\SCnt{A}  \rightharpoonup  \SCnt{B}}{}{}{}{}\SCprodnewline
\SCprodline{|}{\SCnt{A}  \leftharpoonup  \SCnt{B}}{}{}{}{}\SCprodnewline
\SCprodline{|}{\SCsym{(}  \SCnt{A}  \SCsym{)}} {\textsf{M}}{}{}{}\SCprodnewline
\SCprodline{|}{ \SCnt{A} } {\textsf{M}}{}{}{}\SCprodnewline
\SCprodline{|}{ \mathsf{F} \SCnt{X} }{}{}{}{}}

\newcommand{\SCW}{
\SCrulehead{\SCnt{W}  ,\ \SCnt{X}  ,\ \SCnt{Y}  ,\ \SCnt{Z}}{::=}{}\SCprodnewline
\SCfirstprodline{|}{ \mathsf{B} }{}{}{}{}\SCprodnewline
\SCprodline{|}{ \mathsf{Unit} }{}{}{}{}\SCprodnewline
\SCprodline{|}{\SCnt{X}  \otimes  \SCnt{Y}}{}{}{}{}\SCprodnewline
\SCprodline{|}{\SCnt{X}  \multimap  \SCnt{Y}}{}{}{}{}\SCprodnewline
\SCprodline{|}{\SCsym{(}  \SCnt{X}  \SCsym{)}} {\textsf{M}}{}{}{}\SCprodnewline
\SCprodline{|}{ \SCnt{X} } {\textsf{M}}{}{}{}\SCprodnewline
\SCprodline{|}{ \mathsf{G} \SCnt{A} }{}{}{}{}}

\newcommand{\SCT}{
\SCrulehead{\SCnt{T}}{::=}{}\SCprodnewline
\SCfirstprodline{|}{\SCnt{A}}{}{}{}{}\SCprodnewline
\SCprodline{|}{\SCnt{X}}{}{}{}{}}

\newcommand{\SCI}{
\SCrulehead{\Phi  ,\ \Psi}{::=}{}\SCprodnewline
\SCfirstprodline{|}{ \cdot }{}{}{}{}\SCprodnewline
\SCprodline{|}{\Phi_{{\mathrm{1}}}  \SCsym{,}  \Phi_{{\mathrm{2}}}}{}{}{}{}\SCprodnewline
\SCprodline{|}{\SCnt{X}}{}{}{}{}\SCprodnewline
\SCprodline{|}{\SCsym{(}  \Phi  \SCsym{)}} {\textsf{S}}{}{}{}}

\newcommand{\SCG}{
\SCrulehead{\Gamma  ,\ \Delta}{::=}{}\SCprodnewline
\SCfirstprodline{|}{ \cdot }{}{}{}{}\SCprodnewline
\SCprodline{|}{\SCnt{A}}{}{}{}{}\SCprodnewline
\SCprodline{|}{\Phi}{}{}{}{}\SCprodnewline
\SCprodline{|}{\Gamma_{{\mathrm{1}}}  \SCsym{;}  \Gamma_{{\mathrm{2}}}}{}{}{}{}\SCprodnewline
\SCprodline{|}{\SCsym{(}  \Gamma  \SCsym{)}} {\textsf{S}}{}{}{}}

\newcommand{\SCformula}{
\SCrulehead{\SCnt{formula}}{::=}{}\SCprodnewline
\SCfirstprodline{|}{\SCnt{judgement}}{}{}{}{}\SCprodnewline
\SCprodline{|}{ \SCnt{formula_{{\mathrm{1}}}}  \quad  \SCnt{formula_{{\mathrm{2}}}} } {\textsf{M}}{}{}{}\SCprodnewline
\SCprodline{|}{\SCnt{formula_{{\mathrm{1}}}} \, ... \, \SCnt{formula_{\SCmv{i}}}} {\textsf{M}}{}{}{}\SCprodnewline
\SCprodline{|}{ \SCnt{formula} } {\textsf{S}}{}{}{}}

\newcommand{\SCterminals}{
\SCrulehead{\SCnt{terminals}}{::=}{}\SCprodnewline
\SCfirstprodline{|}{ \otimes }{}{}{}{}\SCprodnewline
\SCprodline{|}{ \triangleright }{}{}{}{}\SCprodnewline
\SCprodline{|}{ \circop{e} }{}{}{}{}\SCprodnewline
\SCprodline{|}{ \circop{w} }{}{}{}{}\SCprodnewline
\SCprodline{|}{ \circop{c} }{}{}{}{}\SCprodnewline
\SCprodline{|}{ \rightharpoonup }{}{}{}{}\SCprodnewline
\SCprodline{|}{ \leftharpoonup }{}{}{}{}\SCprodnewline
\SCprodline{|}{ \multimap }{}{}{}{}\SCprodnewline
\SCprodline{|}{ \vdash_\mathcal{C} }{}{}{}{}\SCprodnewline
\SCprodline{|}{ \vdash_\mathcal{L} }{}{}{}{}\SCprodnewline
\SCprodline{|}{ \leadsto }{}{}{}{}}

\newcommand{\SCJtype}{
\SCrulehead{\SCnt{Jtype}}{::=}{}\SCprodnewline
\SCfirstprodline{|}{\Phi  \vdash_\mathcal{C}  \SCnt{X}}{}{}{}{}\SCprodnewline
\SCprodline{|}{\Gamma  \vdash_\mathcal{L}  \SCnt{A}}{}{}{}{}}

\newcommand{\SCjudgement}{
\SCrulehead{\SCnt{judgement}}{::=}{}\SCprodnewline
\SCfirstprodline{|}{\SCnt{Jtype}}{}{}{}{}}

\newcommand{\SCuserXXsyntax}{
\SCrulehead{\SCnt{user\_syntax}}{::=}{}\SCprodnewline
\SCfirstprodline{|}{\SCmv{vars}}{}{}{}{}\SCprodnewline
\SCprodline{|}{\SCmv{ivar}}{}{}{}{}\SCprodnewline
\SCprodline{|}{\SCmv{const}}{}{}{}{}\SCprodnewline
\SCprodline{|}{\SCnt{A}}{}{}{}{}\SCprodnewline
\SCprodline{|}{\SCnt{W}}{}{}{}{}\SCprodnewline
\SCprodline{|}{\SCnt{T}}{}{}{}{}\SCprodnewline
\SCprodline{|}{\Phi}{}{}{}{}\SCprodnewline
\SCprodline{|}{\Gamma}{}{}{}{}\SCprodnewline
\SCprodline{|}{\SCnt{formula}}{}{}{}{}\SCprodnewline
\SCprodline{|}{\SCnt{terminals}}{}{}{}{}}

\newcommand{\SCgrammar}{\SCgrammartabular{
\SCA\SCinterrule
\SCW\SCinterrule
\SCT\SCinterrule
\SCI\SCinterrule
\SCG\SCinterrule
\SCformula\SCinterrule
\SCterminals\SCinterrule
\SCJtype\SCinterrule
\SCjudgement\SCinterrule
\SCuserXXsyntax\SCafterlastrule
}}

% defnss
% defns Jtype
%% defn tty
\newcommand{\SCdruleTXXaxName}[0]{\SCdrulename{T\_ax}}
\newcommand{\SCdruleTXXax}[1]{\SCdrule[#1]{%
}{
\SCnt{X}  \vdash_\mathcal{C}  \SCnt{X}}{%
{\SCdruleTXXaxName}{}%
}}


\newcommand{\SCdruleTXXunitLName}[0]{\SCdrulename{T\_unitL}}
\newcommand{\SCdruleTXXunitL}[1]{\SCdrule[#1]{%
\SCpremise{\Phi  \SCsym{,}  \Psi  \vdash_\mathcal{C}  \SCnt{X}}%
}{
\Phi  \SCsym{,}   \mathsf{Unit}   \SCsym{,}  \Psi  \vdash_\mathcal{C}  \SCnt{X}}{%
{\SCdruleTXXunitLName}{}%
}}


\newcommand{\SCdruleTXXunitRName}[0]{\SCdrulename{T\_unitR}}
\newcommand{\SCdruleTXXunitR}[1]{\SCdrule[#1]{%
}{
 \cdot   \vdash_\mathcal{C}   \mathsf{Unit} }{%
{\SCdruleTXXunitRName}{}%
}}


\newcommand{\SCdruleTXXtenLName}[0]{\SCdrulename{T\_tenL}}
\newcommand{\SCdruleTXXtenL}[1]{\SCdrule[#1]{%
\SCpremise{\Phi  \SCsym{,}  \SCnt{X}  \SCsym{,}  \SCnt{Y}  \SCsym{,}  \Psi  \vdash_\mathcal{C}  \SCnt{Z}}%
}{
\Phi  \SCsym{,}  \SCnt{X}  \otimes  \SCnt{Y}  \SCsym{,}  \Psi  \vdash_\mathcal{C}  \SCnt{Z}}{%
{\SCdruleTXXtenLName}{}%
}}


\newcommand{\SCdruleTXXtenRName}[0]{\SCdrulename{T\_tenR}}
\newcommand{\SCdruleTXXtenR}[1]{\SCdrule[#1]{%
\SCpremise{ \Phi  \vdash_\mathcal{C}  \SCnt{X}  \quad  \Psi  \vdash_\mathcal{C}  \SCnt{Y} }%
}{
\Phi  \SCsym{,}  \Psi  \vdash_\mathcal{C}  \SCnt{X}  \otimes  \SCnt{Y}}{%
{\SCdruleTXXtenRName}{}%
}}


\newcommand{\SCdruleTXXimpLName}[0]{\SCdrulename{T\_impL}}
\newcommand{\SCdruleTXXimpL}[1]{\SCdrule[#1]{%
\SCpremise{ \Phi  \vdash_\mathcal{C}  \SCnt{X}  \quad  \Psi_{{\mathrm{1}}}  \SCsym{,}  \SCnt{Y}  \SCsym{,}  \Psi_{{\mathrm{2}}}  \vdash_\mathcal{C}  \SCnt{Z} }%
}{
\Psi_{{\mathrm{1}}}  \SCsym{,}  \SCnt{X}  \multimap  \SCnt{Y}  \SCsym{,}  \Phi  \SCsym{,}  \Psi_{{\mathrm{2}}}  \vdash_\mathcal{C}  \SCnt{Z}}{%
{\SCdruleTXXimpLName}{}%
}}


\newcommand{\SCdruleTXXimpRName}[0]{\SCdrulename{T\_impR}}
\newcommand{\SCdruleTXXimpR}[1]{\SCdrule[#1]{%
\SCpremise{\Phi  \SCsym{,}  \SCnt{X}  \SCsym{,}  \Psi  \vdash_\mathcal{C}  \SCnt{Y}}%
}{
\Phi  \SCsym{,}  \Psi  \vdash_\mathcal{C}  \SCnt{X}  \multimap  \SCnt{Y}}{%
{\SCdruleTXXimpRName}{}%
}}


\newcommand{\SCdruleTXXGrName}[0]{\SCdrulename{T\_Gr}}
\newcommand{\SCdruleTXXGr}[1]{\SCdrule[#1]{%
\SCpremise{\Phi  \vdash_\mathcal{L}  \SCnt{A}}%
}{
\Phi  \vdash_\mathcal{C}   \mathsf{G} \SCnt{A} }{%
{\SCdruleTXXGrName}{}%
}}


\newcommand{\SCdruleTXXcutName}[0]{\SCdrulename{T\_cut}}
\newcommand{\SCdruleTXXcut}[1]{\SCdrule[#1]{%
\SCpremise{ \Phi  \vdash_\mathcal{C}  \SCnt{X}  \quad  \Psi_{{\mathrm{1}}}  \SCsym{,}  \SCnt{X}  \SCsym{,}  \Psi_{{\mathrm{2}}}  \vdash_\mathcal{C}  \SCnt{Y} }%
}{
\Psi_{{\mathrm{1}}}  \SCsym{,}  \Phi  \SCsym{,}  \Psi_{{\mathrm{2}}}  \vdash_\mathcal{C}  \SCnt{Y}}{%
{\SCdruleTXXcutName}{}%
}}


\newcommand{\SCdruleTXXexName}[0]{\SCdrulename{T\_ex}}
\newcommand{\SCdruleTXXex}[1]{\SCdrule[#1]{%
\SCpremise{\Phi  \SCsym{,}  \SCnt{X}  \SCsym{,}  \SCnt{Y}  \SCsym{,}  \Psi  \vdash_\mathcal{C}  \SCnt{Z}}%
}{
\Phi  \SCsym{,}  \SCnt{Y}  \SCsym{,}  \SCnt{X}  \SCsym{,}  \Psi  \vdash_\mathcal{C}  \SCnt{Z}}{%
{\SCdruleTXXexName}{}%
}}

\newcommand{\SCdefntty}[1]{\begin{SCdefnblock}[#1]{$\Phi  \vdash_\mathcal{C}  \SCnt{X}$}{}
\SCusedrule{\SCdruleTXXax{}}
\SCusedrule{\SCdruleTXXunitL{}}
\SCusedrule{\SCdruleTXXunitR{}}
\SCusedrule{\SCdruleTXXtenL{}}
\SCusedrule{\SCdruleTXXtenR{}}
\SCusedrule{\SCdruleTXXimpL{}}
\SCusedrule{\SCdruleTXXimpR{}}
\SCusedrule{\SCdruleTXXGr{}}
\SCusedrule{\SCdruleTXXcut{}}
\SCusedrule{\SCdruleTXXex{}}
\end{SCdefnblock}}

%% defn sty
\newcommand{\SCdruleSXXaxName}[0]{\SCdrulename{S\_ax}}
\newcommand{\SCdruleSXXax}[1]{\SCdrule[#1]{%
}{
\SCnt{A}  \vdash_\mathcal{L}  \SCnt{A}}{%
{\SCdruleSXXaxName}{}%
}}


\newcommand{\SCdruleSXXunitLOneName}[0]{\SCdrulename{S\_unitL1}}
\newcommand{\SCdruleSXXunitLOne}[1]{\SCdrule[#1]{%
\SCpremise{\Gamma  \SCsym{;}  \Delta  \vdash_\mathcal{L}  \SCnt{A}}%
}{
\Gamma  \SCsym{;}   \mathsf{Unit}   \SCsym{;}  \Delta  \vdash_\mathcal{L}  \SCnt{A}}{%
{\SCdruleSXXunitLOneName}{}%
}}


\newcommand{\SCdruleSXXunitLTwoName}[0]{\SCdrulename{S\_unitL2}}
\newcommand{\SCdruleSXXunitLTwo}[1]{\SCdrule[#1]{%
\SCpremise{\Gamma  \SCsym{;}  \Delta  \vdash_\mathcal{L}  \SCnt{A}}%
}{
\Gamma  \SCsym{;}   \mathsf{Unit}   \SCsym{;}  \Delta  \vdash_\mathcal{L}  \SCnt{A}}{%
{\SCdruleSXXunitLTwoName}{}%
}}


\newcommand{\SCdruleSXXunitRName}[0]{\SCdrulename{S\_unitR}}
\newcommand{\SCdruleSXXunitR}[1]{\SCdrule[#1]{%
}{
 \cdot   \vdash_\mathcal{L}   \mathsf{Unit} }{%
{\SCdruleSXXunitRName}{}%
}}


\newcommand{\SCdruleSXXexName}[0]{\SCdrulename{S\_ex}}
\newcommand{\SCdruleSXXex}[1]{\SCdrule[#1]{%
\SCpremise{\Gamma  \SCsym{;}  \SCnt{X}  \SCsym{;}  \SCnt{Y}  \SCsym{;}  \Delta  \vdash_\mathcal{L}  \SCnt{A}}%
}{
\Gamma  \SCsym{;}  \SCnt{Y}  \SCsym{;}  \SCnt{X}  \SCsym{;}  \Delta  \vdash_\mathcal{L}  \SCnt{A}}{%
{\SCdruleSXXexName}{}%
}}


\newcommand{\SCdruleSXXtenLOneName}[0]{\SCdrulename{S\_tenL1}}
\newcommand{\SCdruleSXXtenLOne}[1]{\SCdrule[#1]{%
\SCpremise{\Gamma  \SCsym{;}  \SCnt{X}  \SCsym{;}  \SCnt{Y}  \SCsym{;}  \Delta  \vdash_\mathcal{L}  \SCnt{A}}%
}{
\Gamma  \SCsym{;}  \SCnt{X}  \otimes  \SCnt{Y}  \SCsym{;}  \Delta  \vdash_\mathcal{L}  \SCnt{A}}{%
{\SCdruleSXXtenLOneName}{}%
}}


\newcommand{\SCdruleSXXtenLTwoName}[0]{\SCdrulename{S\_tenL2}}
\newcommand{\SCdruleSXXtenLTwo}[1]{\SCdrule[#1]{%
\SCpremise{\Gamma  \SCsym{;}  \SCnt{A}  \SCsym{;}  \SCnt{B}  \SCsym{;}  \Delta  \vdash_\mathcal{L}  \SCnt{C}}%
}{
\Gamma  \SCsym{;}  \SCnt{A}  \triangleright  \SCnt{B}  \SCsym{;}  \Delta  \vdash_\mathcal{L}  \SCnt{C}}{%
{\SCdruleSXXtenLTwoName}{}%
}}


\newcommand{\SCdruleSXXtenRName}[0]{\SCdrulename{S\_tenR}}
\newcommand{\SCdruleSXXtenR}[1]{\SCdrule[#1]{%
\SCpremise{ \Gamma  \vdash_\mathcal{L}  \SCnt{A}  \quad  \Delta  \vdash_\mathcal{L}  \SCnt{B} }%
}{
\Gamma  \SCsym{;}  \Delta  \vdash_\mathcal{L}  \SCnt{A}  \triangleright  \SCnt{B}}{%
{\SCdruleSXXtenRName}{}%
}}


\newcommand{\SCdruleSXXimpLName}[0]{\SCdrulename{S\_impL}}
\newcommand{\SCdruleSXXimpL}[1]{\SCdrule[#1]{%
\SCpremise{ \Phi  \vdash_\mathcal{C}  \SCnt{X}  \quad  \Gamma  \SCsym{;}  \SCnt{Y}  \SCsym{;}  \Delta  \vdash_\mathcal{L}  \SCnt{A} }%
}{
\Gamma  \SCsym{;}  \SCnt{X}  \multimap  \SCnt{Y}  \SCsym{;}  \Phi  \SCsym{;}  \Delta  \vdash_\mathcal{L}  \SCnt{A}}{%
{\SCdruleSXXimpLName}{}%
}}


\newcommand{\SCdruleSXXimprLName}[0]{\SCdrulename{S\_imprL}}
\newcommand{\SCdruleSXXimprL}[1]{\SCdrule[#1]{%
\SCpremise{ \Gamma  \vdash_\mathcal{L}  \SCnt{A}  \quad  \Delta_{{\mathrm{1}}}  \SCsym{;}  \SCnt{B}  \SCsym{;}  \Delta_{{\mathrm{2}}}  \vdash_\mathcal{L}  \SCnt{C} }%
}{
\Delta_{{\mathrm{1}}}  \SCsym{;}  \SCnt{A}  \rightharpoonup  \SCnt{B}  \SCsym{;}  \Gamma  \SCsym{;}  \Delta_{{\mathrm{2}}}  \vdash_\mathcal{L}  \SCnt{C}}{%
{\SCdruleSXXimprLName}{}%
}}


\newcommand{\SCdruleSXXimprRName}[0]{\SCdrulename{S\_imprR}}
\newcommand{\SCdruleSXXimprR}[1]{\SCdrule[#1]{%
\SCpremise{\Gamma  \SCsym{;}  \SCnt{A}  \vdash_\mathcal{L}  \SCnt{B}}%
}{
\Gamma  \vdash_\mathcal{L}  \SCnt{A}  \rightharpoonup  \SCnt{B}}{%
{\SCdruleSXXimprRName}{}%
}}


\newcommand{\SCdruleSXXimplLName}[0]{\SCdrulename{S\_implL}}
\newcommand{\SCdruleSXXimplL}[1]{\SCdrule[#1]{%
\SCpremise{ \Gamma  \vdash_\mathcal{L}  \SCnt{A}  \quad  \Delta_{{\mathrm{1}}}  \SCsym{;}  \SCnt{B}  \SCsym{;}  \Delta_{{\mathrm{2}}}  \vdash_\mathcal{L}  \SCnt{C} }%
}{
\Delta_{{\mathrm{1}}}  \SCsym{;}  \Gamma  \SCsym{;}  \SCnt{B}  \leftharpoonup  \SCnt{A}  \SCsym{;}  \Delta_{{\mathrm{2}}}  \vdash_\mathcal{L}  \SCnt{C}}{%
{\SCdruleSXXimplLName}{}%
}}


\newcommand{\SCdruleSXXimplRName}[0]{\SCdrulename{S\_implR}}
\newcommand{\SCdruleSXXimplR}[1]{\SCdrule[#1]{%
\SCpremise{\SCnt{A}  \SCsym{;}  \Gamma  \vdash_\mathcal{L}  \SCnt{B}}%
}{
\Gamma  \vdash_\mathcal{L}  \SCnt{B}  \leftharpoonup  \SCnt{A}}{%
{\SCdruleSXXimplRName}{}%
}}


\newcommand{\SCdruleSXXFlName}[0]{\SCdrulename{S\_Fl}}
\newcommand{\SCdruleSXXFl}[1]{\SCdrule[#1]{%
\SCpremise{\Gamma  \SCsym{;}  \SCnt{X}  \SCsym{;}  \Delta  \vdash_\mathcal{L}  \SCnt{A}}%
}{
\Gamma  \SCsym{;}   \mathsf{F} \SCnt{X}   \SCsym{;}  \Delta  \vdash_\mathcal{L}  \SCnt{A}}{%
{\SCdruleSXXFlName}{}%
}}


\newcommand{\SCdruleSXXFrName}[0]{\SCdrulename{S\_Fr}}
\newcommand{\SCdruleSXXFr}[1]{\SCdrule[#1]{%
\SCpremise{\Phi  \vdash_\mathcal{C}  \SCnt{X}}%
}{
\Phi  \vdash_\mathcal{L}   \mathsf{F} \SCnt{X} }{%
{\SCdruleSXXFrName}{}%
}}


\newcommand{\SCdruleSXXGlName}[0]{\SCdrulename{S\_Gl}}
\newcommand{\SCdruleSXXGl}[1]{\SCdrule[#1]{%
\SCpremise{\Gamma  \SCsym{;}  \SCnt{A}  \SCsym{;}  \Delta  \vdash_\mathcal{L}  \SCnt{B}}%
}{
\Gamma  \SCsym{;}   \mathsf{G} \SCnt{A}   \SCsym{;}  \Delta  \vdash_\mathcal{L}  \SCnt{B}}{%
{\SCdruleSXXGlName}{}%
}}


\newcommand{\SCdruleSXXcutOneName}[0]{\SCdrulename{S\_cut1}}
\newcommand{\SCdruleSXXcutOne}[1]{\SCdrule[#1]{%
\SCpremise{ \Phi  \vdash_\mathcal{C}  \SCnt{X}  \quad  \Delta_{{\mathrm{1}}}  \SCsym{;}  \SCnt{X}  \SCsym{;}  \Delta_{{\mathrm{2}}}  \vdash_\mathcal{L}  \SCnt{A} }%
}{
\Delta_{{\mathrm{1}}}  \SCsym{;}  \Phi  \SCsym{;}  \Delta_{{\mathrm{1}}}  \vdash_\mathcal{L}  \SCnt{A}}{%
{\SCdruleSXXcutOneName}{}%
}}


\newcommand{\SCdruleSXXcutTwoName}[0]{\SCdrulename{S\_cut2}}
\newcommand{\SCdruleSXXcutTwo}[1]{\SCdrule[#1]{%
\SCpremise{ \Gamma  \vdash_\mathcal{L}  \SCnt{A}  \quad  \Delta_{{\mathrm{1}}}  \SCsym{;}  \SCnt{A}  \SCsym{;}  \Delta_{{\mathrm{2}}}  \vdash_\mathcal{L}  \SCnt{B} }%
}{
\Delta_{{\mathrm{1}}}  \SCsym{;}  \Gamma  \SCsym{;}  \Delta_{{\mathrm{2}}}  \vdash_\mathcal{L}  \SCnt{B}}{%
{\SCdruleSXXcutTwoName}{}%
}}

\newcommand{\SCdefnsty}[1]{\begin{SCdefnblock}[#1]{$\Gamma  \vdash_\mathcal{L}  \SCnt{A}$}{}
\SCusedrule{\SCdruleSXXax{}}
\SCusedrule{\SCdruleSXXunitLOne{}}
\SCusedrule{\SCdruleSXXunitLTwo{}}
\SCusedrule{\SCdruleSXXunitR{}}
\SCusedrule{\SCdruleSXXex{}}
\SCusedrule{\SCdruleSXXtenLOne{}}
\SCusedrule{\SCdruleSXXtenLTwo{}}
\SCusedrule{\SCdruleSXXtenR{}}
\SCusedrule{\SCdruleSXXimpL{}}
\SCusedrule{\SCdruleSXXimprL{}}
\SCusedrule{\SCdruleSXXimprR{}}
\SCusedrule{\SCdruleSXXimplL{}}
\SCusedrule{\SCdruleSXXimplR{}}
\SCusedrule{\SCdruleSXXFl{}}
\SCusedrule{\SCdruleSXXFr{}}
\SCusedrule{\SCdruleSXXGl{}}
\SCusedrule{\SCdruleSXXcutOne{}}
\SCusedrule{\SCdruleSXXcutTwo{}}
\end{SCdefnblock}}


\newcommand{\SCdefnsJtype}{
\SCdefntty{}\SCdefnsty{}}

\newcommand{\SCdefnss}{
\SCdefnsJtype
}

\newcommand{\SCall}{\SCmetavars\\[0pt]
\SCgrammar\\[5.0mm]
\SCdefnss}


\renewcommand{\SCdrule}[4][]{{\displaystyle\frac{\begin{array}{l}#2\end{array}}{#3}\,#4}}
\renewcommand{\SCdruleTXXaxName}{\mathcal{C}\text{-ax}}
\renewcommand{\SCdruleTXXunitLName}{\mathcal{C}\text{-}\mathsf{Unit}_L}
\renewcommand{\SCdruleTXXunitRName}{\mathcal{C}\text{-}\mathsf{Unit}_L}
\renewcommand{\SCdruleTXXtenLName}{\mathcal{C}\text{-}\otimes_L}
\renewcommand{\SCdruleTXXtenRName}{\mathcal{C}\text{-}\otimes_R}
\renewcommand{\SCdruleTXXimpLName}{\mathcal{C}\text{-}\multimap_L}
\renewcommand{\SCdruleTXXimpRName}{\mathcal{C}\text{-}\multimap_R}
\renewcommand{\SCdruleTXXGrName}{\mathcal{C}\text{-}\mathsf{G}_R}
\renewcommand{\SCdruleTXXcutName}{\mathcal{C}\text{-}\mathsf{Cut}}
\renewcommand{\SCdruleTXXexName}{\mathcal{C}\text{-}\mathsf{ex}}
\renewcommand{\SCdruleSXXaxName}{\mathcal{L}\text{-ax}}
\renewcommand{\SCdruleSXXunitLOneName}{\mathcal{LC}\text{-}\mathsf{Unit}_L}
\renewcommand{\SCdruleSXXunitLTwoName}{\mathcal{L}\text{-}\mathsf{Unit}_L}
\renewcommand{\SCdruleSXXunitRName}{\mathcal{L}\text{-}\mathsf{Unit}_R}
\renewcommand{\SCdruleSXXexName}{\mathcal{L}\text{-}\mathsf{ex}}
\renewcommand{\SCdruleSXXtenLOneName}{\mathcal{LC}\text{-}\otimes_L}
\renewcommand{\SCdruleSXXtenLTwoName}{\mathcal{L}\text{-}\otimes_L}
\renewcommand{\SCdruleSXXtenRName}{\mathcal{L}\text{-}\otimes_R}
\renewcommand{\SCdruleSXXimpLName}{\mathcal{L}\text{-}\multimap_L}
\renewcommand{\SCdruleSXXimprLName}{\mathcal{L}\text{-}\rightharpoonup_L}
\renewcommand{\SCdruleSXXimprRName}{\mathcal{L}\text{-}\rightharpoonup_R}
\renewcommand{\SCdruleSXXimplLName}{\mathcal{L}\text{-}\leftharpoonup_L}
\renewcommand{\SCdruleSXXimplRName}{\mathcal{L}\text{-}\leftharpoonup_R}
\renewcommand{\SCdruleSXXFlName}{\mathcal{L}\text{-}\mathsf{F}_L}
\renewcommand{\SCdruleSXXFrName}{\mathcal{L}\text{-}\mathsf{F}_R}
\renewcommand{\SCdruleSXXGlName}{\mathcal{L}\text{-}\mathsf{G}_L}
\renewcommand{\SCdruleSXXcutOneName}{\mathcal{LC}\text{-}\mathsf{Cut}}
\renewcommand{\SCdruleSXXcutTwoName}{\mathcal{L}\text{-}\mathsf{Cut}}

% generated by Ott 0.25 from: Elle/Elle.ott
\newcommand{\Elledrule}[4][]{{\displaystyle\frac{\begin{array}{l}#2\end{array}}{#3}\quad\Elledrulename{#4}}}
\newcommand{\Elleusedrule}[1]{\[#1\]}
\newcommand{\Ellepremise}[1]{ #1 \\}
\newenvironment{Elledefnblock}[3][]{ \framebox{\mbox{#2}} \quad #3 \\[0pt]}{}
\newenvironment{Ellefundefnblock}[3][]{ \framebox{\mbox{#2}} \quad #3 \\[0pt]\begin{displaymath}\begin{array}{l}}{\end{array}\end{displaymath}}
\newcommand{\Ellefunclause}[2]{ #1 \equiv #2 \\}
\newcommand{\Ellent}[1]{\mathit{#1}}
\newcommand{\Ellemv}[1]{\mathit{#1}}
\newcommand{\Ellekw}[1]{\mathbf{#1}}
\newcommand{\Ellesym}[1]{#1}
\newcommand{\Ellecom}[1]{\text{#1}}
\newcommand{\Elledrulename}[1]{\textsc{#1}}
\newcommand{\Ellecomplu}[5]{\overline{#1}^{\,#2\in #3 #4 #5}}
\newcommand{\Ellecompu}[3]{\overline{#1}^{\,#2<#3}}
\newcommand{\Ellecomp}[2]{\overline{#1}^{\,#2}}
\newcommand{\Ellegrammartabular}[1]{\begin{supertabular}{llcllllll}#1\end{supertabular}}
\newcommand{\Ellemetavartabular}[1]{\begin{supertabular}{ll}#1\end{supertabular}}
\newcommand{\Ellerulehead}[3]{$#1$ & & $#2$ & & & \multicolumn{2}{l}{#3}}
\newcommand{\Elleprodline}[6]{& & $#1$ & $#2$ & $#3 #4$ & $#5$ & $#6$}
\newcommand{\Ellefirstprodline}[6]{\Elleprodline{#1}{#2}{#3}{#4}{#5}{#6}}
\newcommand{\Ellelongprodline}[2]{& & $#1$ & \multicolumn{4}{l}{$#2$}}
\newcommand{\Ellefirstlongprodline}[2]{\Ellelongprodline{#1}{#2}}
\newcommand{\Ellebindspecprodline}[6]{\Elleprodline{#1}{#2}{#3}{#4}{#5}{#6}}
\newcommand{\Elleprodnewline}{\\}
\newcommand{\Elleinterrule}{\\[5.0mm]}
\newcommand{\Elleafterlastrule}{\\}
\newcommand{\Ellemetavars}{
\Ellemetavartabular{
 $ \Ellemv{vars} ,\, \Ellemv{n} ,\, \Ellemv{a} ,\, \Ellemv{x} ,\, \Ellemv{y} ,\, \Ellemv{z} ,\, \Ellemv{w} ,\, \Ellemv{m} ,\, \Ellemv{o} $ &  \\
 $ \Ellemv{ivar} ,\, \Ellemv{i} ,\, \Ellemv{k} ,\, \Ellemv{j} ,\, \Ellemv{l} $ &  \\
 $ \Ellemv{const} ,\, \Ellemv{b} $ &  \\
}}

\newcommand{\ElleA}{
\Ellerulehead{\Ellent{A}  ,\ \Ellent{B}  ,\ \Ellent{C}  ,\ D}{::=}{}\Elleprodnewline
\Ellefirstprodline{|}{ \mathsf{B} }{}{}{}{}\Elleprodnewline
\Elleprodline{|}{ \mathsf{Unit} }{}{}{}{}\Elleprodnewline
\Elleprodline{|}{\Ellent{A}  \triangleright  \Ellent{B}}{}{}{}{}\Elleprodnewline
\Elleprodline{|}{\Ellent{A}  \rightharpoonup  \Ellent{B}}{}{}{}{}\Elleprodnewline
\Elleprodline{|}{\Ellent{A}  \leftharpoonup  \Ellent{B}}{}{}{}{}\Elleprodnewline
\Elleprodline{|}{\Ellesym{(}  \Ellent{A}  \Ellesym{)}} {\textsf{M}}{}{}{}\Elleprodnewline
\Elleprodline{|}{ \Ellent{A} } {\textsf{M}}{}{}{}\Elleprodnewline
\Elleprodline{|}{ \mathsf{F} \Ellent{X} }{}{}{}{}}

\newcommand{\ElleW}{
\Ellerulehead{\Ellent{W}  ,\ \Ellent{X}  ,\ \Ellent{Y}  ,\ \Ellent{Z}}{::=}{}\Elleprodnewline
\Ellefirstprodline{|}{ \mathsf{B} }{}{}{}{}\Elleprodnewline
\Elleprodline{|}{ \mathsf{Unit} }{}{}{}{}\Elleprodnewline
\Elleprodline{|}{\Ellent{X}  \otimes  \Ellent{Y}}{}{}{}{}\Elleprodnewline
\Elleprodline{|}{\Ellent{X}  \multimap  \Ellent{Y}}{}{}{}{}\Elleprodnewline
\Elleprodline{|}{\Ellesym{(}  \Ellent{X}  \Ellesym{)}} {\textsf{M}}{}{}{}\Elleprodnewline
\Elleprodline{|}{ \Ellent{X} } {\textsf{M}}{}{}{}\Elleprodnewline
\Elleprodline{|}{ \mathsf{G} \Ellent{A} }{}{}{}{}}

\newcommand{\ElleT}{
\Ellerulehead{\Ellent{T}}{::=}{}\Elleprodnewline
\Ellefirstprodline{|}{\Ellent{A}}{}{}{}{}\Elleprodnewline
\Elleprodline{|}{\Ellent{X}}{}{}{}{}}

\newcommand{\Ellep}{
\Ellerulehead{\Ellent{p}  ,\ \Ellent{q}}{::=}{}\Elleprodnewline
\Ellefirstprodline{|}{ \star }{}{}{}{}\Elleprodnewline
\Elleprodline{|}{\Ellemv{x}}{}{}{}{}\Elleprodnewline
\Elleprodline{|}{ \mathsf{triv} }{}{}{}{}\Elleprodnewline
\Elleprodline{|}{ \mathsf{triv} }{}{}{}{}\Elleprodnewline
\Elleprodline{|}{\Ellent{p}  \otimes  \Ellent{p'}}{}{}{}{}\Elleprodnewline
\Elleprodline{|}{\Ellent{p}  \triangleright  \Ellent{p'}}{}{}{}{}\Elleprodnewline
\Elleprodline{|}{ \mathsf{F}\, \Ellent{p} }{}{}{}{}\Elleprodnewline
\Elleprodline{|}{ \mathsf{G}\, \Ellent{p} }{}{}{}{}}

\newcommand{\Elles}{
\Ellerulehead{\Ellent{s}}{::=}{}\Elleprodnewline
\Ellefirstprodline{|}{\Ellemv{x}}{}{}{}{}\Elleprodnewline
\Elleprodline{|}{\Ellemv{b}}{}{}{}{}\Elleprodnewline
\Elleprodline{|}{ \mathsf{triv} }{}{}{}{}\Elleprodnewline
\Elleprodline{|}{ \mathsf{let}\, \Ellent{s_{{\mathrm{1}}}}  :  \Ellent{A} \,\mathsf{be}\, \Ellent{p} \,\mathsf{in}\, \Ellent{s_{{\mathrm{2}}}} }{}{}{}{}\Elleprodnewline
\Elleprodline{|}{ \mathsf{let}\, \Ellent{t}  :  \Ellent{X} \,\mathsf{be}\, \Ellent{p} \,\mathsf{in}\, \Ellent{s} }{}{}{}{}\Elleprodnewline
\Elleprodline{|}{\Ellent{s_{{\mathrm{1}}}}  \triangleright  \Ellent{s_{{\mathrm{2}}}}}{}{}{}{}\Elleprodnewline
\Elleprodline{|}{ \lambda_l  \Ellemv{x}  :  \Ellent{A} . \Ellent{s} }{}{}{}{}\Elleprodnewline
\Elleprodline{|}{ \lambda_r  \Ellemv{x}  :  \Ellent{A} . \Ellent{s} }{}{}{}{}\Elleprodnewline
\Elleprodline{|}{ \mathsf{app}_l\, \Ellent{s_{{\mathrm{1}}}} \, \Ellent{s_{{\mathrm{2}}}} }{}{}{}{}\Elleprodnewline
\Elleprodline{|}{ \mathsf{app}_r\, \Ellent{s_{{\mathrm{1}}}} \, \Ellent{s_{{\mathrm{2}}}} }{}{}{}{}\Elleprodnewline
\Elleprodline{|}{ \mathsf{derelict}\, \Ellent{t} }{}{}{}{}\Elleprodnewline
\Elleprodline{|}{ \mathsf{ex}\, \Ellent{s_{{\mathrm{1}}}} , \Ellent{s_{{\mathrm{2}}}} \,\mathsf{with}\, \Ellemv{x_{{\mathrm{1}}}} , \Ellemv{x_{{\mathrm{2}}}} \,\mathsf{in}\, \Ellent{s_{{\mathrm{3}}}} }{}{}{}{}\Elleprodnewline
\Elleprodline{|}{\Ellesym{[}  \Ellent{s_{{\mathrm{1}}}}  \Ellesym{/}  \Ellemv{x}  \Ellesym{]}  \Ellent{s_{{\mathrm{2}}}}} {\textsf{M}}{}{}{}\Elleprodnewline
\Elleprodline{|}{\Ellesym{[}  \Ellent{t}  \Ellesym{/}  \Ellemv{x}  \Ellesym{]}  \Ellent{s}} {\textsf{M}}{}{}{}\Elleprodnewline
\Elleprodline{|}{\Ellesym{(}  \Ellent{s}  \Ellesym{)}} {\textsf{S}}{}{}{}\Elleprodnewline
\Elleprodline{|}{ \Ellent{s} } {\textsf{M}}{}{}{}\Elleprodnewline
\Elleprodline{|}{ \mathsf{F} \Ellent{t} }{}{}{}{}}

\newcommand{\Ellet}{
\Ellerulehead{\Ellent{t}}{::=}{}\Elleprodnewline
\Ellefirstprodline{|}{\Ellemv{x}}{}{}{}{}\Elleprodnewline
\Elleprodline{|}{\Ellemv{b}}{}{}{}{}\Elleprodnewline
\Elleprodline{|}{ \mathsf{triv} }{}{}{}{}\Elleprodnewline
\Elleprodline{|}{ \mathsf{let}\, \Ellent{t_{{\mathrm{1}}}}  :  \Ellent{X} \,\mathsf{be}\, \Ellent{p} \,\mathsf{in}\, \Ellent{t_{{\mathrm{2}}}} }{}{}{}{}\Elleprodnewline
\Elleprodline{|}{\Ellent{t_{{\mathrm{1}}}}  \otimes  \Ellent{t_{{\mathrm{2}}}}}{}{}{}{}\Elleprodnewline
\Elleprodline{|}{ \lambda  \Ellemv{x}  :  \Ellent{X} . \Ellent{t} }{}{}{}{}\Elleprodnewline
\Elleprodline{|}{ \Ellent{t_{{\mathrm{1}}}}   \Ellent{t_{{\mathrm{2}}}} }{}{}{}{}\Elleprodnewline
\Elleprodline{|}{ \mathsf{ex}\, \Ellent{t_{{\mathrm{1}}}} , \Ellent{t_{{\mathrm{2}}}} \,\mathsf{with}\, \Ellemv{x_{{\mathrm{1}}}} , \Ellemv{x_{{\mathrm{2}}}} \,\mathsf{in}\, \Ellent{t_{{\mathrm{3}}}} }{}{}{}{}\Elleprodnewline
\Elleprodline{|}{\Ellesym{[}  \Ellent{t_{{\mathrm{1}}}}  \Ellesym{/}  \Ellemv{x}  \Ellesym{]}  \Ellent{t_{{\mathrm{2}}}}} {\textsf{M}}{}{}{}\Elleprodnewline
\Elleprodline{|}{\Ellesym{(}  \Ellent{t}  \Ellesym{)}} {\textsf{S}}{}{}{}\Elleprodnewline
\Elleprodline{|}{\Ellesym{h(}  \Ellent{t}  \Ellesym{)}} {\textsf{M}}{}{}{}\Elleprodnewline
\Elleprodline{|}{ \mathsf{G} \Ellent{s} }{}{}{}{}}

\newcommand{\ElleI}{
\Ellerulehead{\Phi  ,\ \Psi}{::=}{}\Elleprodnewline
\Ellefirstprodline{|}{ \cdot }{}{}{}{}\Elleprodnewline
\Elleprodline{|}{\Phi_{{\mathrm{1}}}  \Ellesym{,}  \Phi_{{\mathrm{2}}}}{}{}{}{}\Elleprodnewline
\Elleprodline{|}{\Ellemv{x}  \Ellesym{:}  \Ellent{X}}{}{}{}{}\Elleprodnewline
\Elleprodline{|}{\Ellesym{(}  \Phi  \Ellesym{)}} {\textsf{S}}{}{}{}}

\newcommand{\ElleG}{
\Ellerulehead{\Gamma  ,\ \Delta}{::=}{}\Elleprodnewline
\Ellefirstprodline{|}{ \cdot }{}{}{}{}\Elleprodnewline
\Elleprodline{|}{\Ellemv{x}  \Ellesym{:}  \Ellent{A}}{}{}{}{}\Elleprodnewline
\Elleprodline{|}{\Phi}{}{}{}{}\Elleprodnewline
\Elleprodline{|}{\Gamma_{{\mathrm{1}}}  \Ellesym{;}  \Gamma_{{\mathrm{2}}}}{}{}{}{}\Elleprodnewline
\Elleprodline{|}{\Ellesym{(}  \Gamma  \Ellesym{)}} {\textsf{S}}{}{}{}}

\newcommand{\Elleformula}{
\Ellerulehead{\Ellent{formula}}{::=}{}\Elleprodnewline
\Ellefirstprodline{|}{\Ellent{judgement}}{}{}{}{}\Elleprodnewline
\Elleprodline{|}{ \Ellent{formula_{{\mathrm{1}}}}  \quad  \Ellent{formula_{{\mathrm{2}}}} } {\textsf{M}}{}{}{}\Elleprodnewline
\Elleprodline{|}{\Ellent{formula_{{\mathrm{1}}}} \, ... \, \Ellent{formula_{\Ellemv{i}}}} {\textsf{M}}{}{}{}\Elleprodnewline
\Elleprodline{|}{ \Ellent{formula} } {\textsf{S}}{}{}{}\Elleprodnewline
\Elleprodline{|}{ \Ellemv{x}  \not\in \mathsf{FV}( \Ellent{s} ) }{}{}{}{}\Elleprodnewline
\Elleprodline{|}{ \Ellemv{x}  \not\in |  \Gamma ,  \Delta ,  \Psi  | }{}{}{}{}\Elleprodnewline
\Elleprodline{|}{ \Ellemv{x}  \not\in |  \Gamma ,  \Delta  | }{}{}{}{}}

\newcommand{\Elleterminals}{
\Ellerulehead{\Ellent{terminals}}{::=}{}\Elleprodnewline
\Ellefirstprodline{|}{ \otimes }{}{}{}{}\Elleprodnewline
\Elleprodline{|}{ \triangleright }{}{}{}{}\Elleprodnewline
\Elleprodline{|}{ \circop{e} }{}{}{}{}\Elleprodnewline
\Elleprodline{|}{ \circop{w} }{}{}{}{}\Elleprodnewline
\Elleprodline{|}{ \circop{c} }{}{}{}{}\Elleprodnewline
\Elleprodline{|}{ \rightharpoonup }{}{}{}{}\Elleprodnewline
\Elleprodline{|}{ \leftharpoonup }{}{}{}{}\Elleprodnewline
\Elleprodline{|}{ \multimap }{}{}{}{}\Elleprodnewline
\Elleprodline{|}{ \vdash_\mathcal{C} }{}{}{}{}\Elleprodnewline
\Elleprodline{|}{ \vdash_\mathcal{L} }{}{}{}{}\Elleprodnewline
\Elleprodline{|}{ \leadsto }{}{}{}{}\Elleprodnewline
\Elleprodline{|}{ \leadsto_\mathsf{c} }{}{}{}{}}

\newcommand{\ElleJtype}{
\Ellerulehead{\Ellent{Jtype}}{::=}{}\Elleprodnewline
\Ellefirstprodline{|}{\Phi  \vdash_\mathcal{C}  \Ellent{t}  \Ellesym{:}  \Ellent{X}}{}{}{}{}\Elleprodnewline
\Elleprodline{|}{\Gamma  \vdash_\mathcal{L}  \Ellent{s}  \Ellesym{:}  \Ellent{A}}{}{}{}{}}

\newcommand{\Ellejudgement}{
\Ellerulehead{\Ellent{judgement}}{::=}{}\Elleprodnewline
\Ellefirstprodline{|}{\Ellent{Jtype}}{}{}{}{}}

\newcommand{\ElleuserXXsyntax}{
\Ellerulehead{\Ellent{user\_syntax}}{::=}{}\Elleprodnewline
\Ellefirstprodline{|}{\Ellemv{vars}}{}{}{}{}\Elleprodnewline
\Elleprodline{|}{\Ellemv{ivar}}{}{}{}{}\Elleprodnewline
\Elleprodline{|}{\Ellemv{const}}{}{}{}{}\Elleprodnewline
\Elleprodline{|}{\Ellent{A}}{}{}{}{}\Elleprodnewline
\Elleprodline{|}{\Ellent{W}}{}{}{}{}\Elleprodnewline
\Elleprodline{|}{\Ellent{T}}{}{}{}{}\Elleprodnewline
\Elleprodline{|}{\Ellent{p}}{}{}{}{}\Elleprodnewline
\Elleprodline{|}{\Ellent{s}}{}{}{}{}\Elleprodnewline
\Elleprodline{|}{\Ellent{t}}{}{}{}{}\Elleprodnewline
\Elleprodline{|}{\Phi}{}{}{}{}\Elleprodnewline
\Elleprodline{|}{\Gamma}{}{}{}{}\Elleprodnewline
\Elleprodline{|}{\Ellent{formula}}{}{}{}{}\Elleprodnewline
\Elleprodline{|}{\Ellent{terminals}}{}{}{}{}}

\newcommand{\Ellegrammar}{\Ellegrammartabular{
\ElleA\Elleinterrule
\ElleW\Elleinterrule
\ElleT\Elleinterrule
\Ellep\Elleinterrule
\Elles\Elleinterrule
\Ellet\Elleinterrule
\ElleI\Elleinterrule
\ElleG\Elleinterrule
\Elleformula\Elleinterrule
\Elleterminals\Elleinterrule
\ElleJtype\Elleinterrule
\Ellejudgement\Elleinterrule
\ElleuserXXsyntax\Elleafterlastrule
}}

% defnss
% defns Jtype
%% defn tty
\newcommand{\ElledruleTXXaxName}[0]{\Elledrulename{T\_ax}}
\newcommand{\ElledruleTXXax}[1]{\Elledrule[#1]{%
}{
\Ellemv{x}  \Ellesym{:}  \Ellent{X}  \vdash_\mathcal{C}  \Ellemv{x}  \Ellesym{:}  \Ellent{X}}{%
{\ElledruleTXXaxName}{}%
}}


\newcommand{\ElledruleTXXunitLName}[0]{\Elledrulename{T\_unitL}}
\newcommand{\ElledruleTXXunitL}[1]{\Elledrule[#1]{%
\Ellepremise{\Phi  \Ellesym{,}  \Psi  \vdash_\mathcal{C}  \Ellent{t}  \Ellesym{:}  \Ellent{X}}%
}{
\Phi  \Ellesym{,}  \Ellemv{x}  \Ellesym{:}   \mathsf{Unit}   \Ellesym{,}  \Psi  \vdash_\mathcal{C}   \mathsf{let}\, \Ellemv{x}  :   \mathsf{Unit}  \,\mathsf{be}\,  \mathsf{triv}  \,\mathsf{in}\, \Ellent{t}   \Ellesym{:}  \Ellent{X}}{%
{\ElledruleTXXunitLName}{}%
}}


\newcommand{\ElledruleTXXunitRName}[0]{\Elledrulename{T\_unitR}}
\newcommand{\ElledruleTXXunitR}[1]{\Elledrule[#1]{%
}{
 \cdot   \vdash_\mathcal{C}   \mathsf{triv}   \Ellesym{:}   \mathsf{Unit} }{%
{\ElledruleTXXunitRName}{}%
}}


\newcommand{\ElledruleTXXtenLName}[0]{\Elledrulename{T\_tenL}}
\newcommand{\ElledruleTXXtenL}[1]{\Elledrule[#1]{%
\Ellepremise{\Phi  \Ellesym{,}  \Ellemv{x}  \Ellesym{:}  \Ellent{X}  \Ellesym{,}  \Ellemv{y}  \Ellesym{:}  \Ellent{Y}  \Ellesym{,}  \Psi  \vdash_\mathcal{C}  \Ellent{t}  \Ellesym{:}  \Ellent{Z}}%
}{
\Phi  \Ellesym{,}  \Ellemv{z}  \Ellesym{:}  \Ellent{X}  \otimes  \Ellent{Y}  \Ellesym{,}  \Psi  \vdash_\mathcal{C}   \mathsf{let}\, \Ellemv{z}  :  \Ellent{X}  \otimes  \Ellent{Y} \,\mathsf{be}\, \Ellemv{x}  \otimes  \Ellemv{y} \,\mathsf{in}\, \Ellent{t}   \Ellesym{:}  \Ellent{Z}}{%
{\ElledruleTXXtenLName}{}%
}}


\newcommand{\ElledruleTXXtenRName}[0]{\Elledrulename{T\_tenR}}
\newcommand{\ElledruleTXXtenR}[1]{\Elledrule[#1]{%
\Ellepremise{ \Phi  \vdash_\mathcal{C}  \Ellent{t_{{\mathrm{1}}}}  \Ellesym{:}  \Ellent{X}  \quad  \Psi  \vdash_\mathcal{C}  \Ellent{t_{{\mathrm{2}}}}  \Ellesym{:}  \Ellent{Y} }%
}{
\Phi  \Ellesym{,}  \Psi  \vdash_\mathcal{C}  \Ellent{t_{{\mathrm{1}}}}  \otimes  \Ellent{t_{{\mathrm{2}}}}  \Ellesym{:}  \Ellent{X}  \otimes  \Ellent{Y}}{%
{\ElledruleTXXtenRName}{}%
}}


\newcommand{\ElledruleTXXimpLName}[0]{\Elledrulename{T\_impL}}
\newcommand{\ElledruleTXXimpL}[1]{\Elledrule[#1]{%
\Ellepremise{ \Phi  \vdash_\mathcal{C}  \Ellent{t_{{\mathrm{1}}}}  \Ellesym{:}  \Ellent{X}  \quad  \Psi_{{\mathrm{1}}}  \Ellesym{,}  \Ellemv{x}  \Ellesym{:}  \Ellent{Y}  \Ellesym{,}  \Psi_{{\mathrm{2}}}  \vdash_\mathcal{C}  \Ellent{t_{{\mathrm{2}}}}  \Ellesym{:}  \Ellent{Z} }%
}{
\Psi_{{\mathrm{1}}}  \Ellesym{,}  \Ellemv{y}  \Ellesym{:}  \Ellent{X}  \multimap  \Ellent{Y}  \Ellesym{,}  \Phi  \Ellesym{,}  \Psi_{{\mathrm{2}}}  \vdash_\mathcal{C}  \Ellesym{[}   \Ellemv{y}   \Ellent{t_{{\mathrm{1}}}}   \Ellesym{/}  \Ellemv{x}  \Ellesym{]}  \Ellent{t_{{\mathrm{2}}}}  \Ellesym{:}  \Ellent{Z}}{%
{\ElledruleTXXimpLName}{}%
}}


\newcommand{\ElledruleTXXimpRName}[0]{\Elledrulename{T\_impR}}
\newcommand{\ElledruleTXXimpR}[1]{\Elledrule[#1]{%
\Ellepremise{\Phi  \Ellesym{,}  \Ellemv{x}  \Ellesym{:}  \Ellent{X}  \Ellesym{,}  \Psi  \vdash_\mathcal{C}  \Ellent{t}  \Ellesym{:}  \Ellent{Y}}%
}{
\Phi  \Ellesym{,}  \Psi  \vdash_\mathcal{C}   \lambda  \Ellemv{x}  :  \Ellent{X} . \Ellent{t}   \Ellesym{:}  \Ellent{X}  \multimap  \Ellent{Y}}{%
{\ElledruleTXXimpRName}{}%
}}


\newcommand{\ElledruleTXXGrName}[0]{\Elledrulename{T\_Gr}}
\newcommand{\ElledruleTXXGr}[1]{\Elledrule[#1]{%
\Ellepremise{\Phi  \vdash_\mathcal{L}  \Ellent{s}  \Ellesym{:}  \Ellent{A}}%
}{
\Phi  \vdash_\mathcal{C}   \mathsf{G} \Ellent{s}   \Ellesym{:}   \mathsf{G} \Ellent{A} }{%
{\ElledruleTXXGrName}{}%
}}


\newcommand{\ElledruleTXXcutName}[0]{\Elledrulename{T\_cut}}
\newcommand{\ElledruleTXXcut}[1]{\Elledrule[#1]{%
\Ellepremise{ \Phi  \vdash_\mathcal{C}  \Ellent{t_{{\mathrm{1}}}}  \Ellesym{:}  \Ellent{X}  \quad  \Psi_{{\mathrm{1}}}  \Ellesym{,}  \Ellemv{x}  \Ellesym{:}  \Ellent{X}  \Ellesym{,}  \Psi_{{\mathrm{2}}}  \vdash_\mathcal{C}  \Ellent{t_{{\mathrm{2}}}}  \Ellesym{:}  \Ellent{Y} }%
}{
\Psi_{{\mathrm{1}}}  \Ellesym{,}  \Phi  \Ellesym{,}  \Psi_{{\mathrm{2}}}  \vdash_\mathcal{C}  \Ellesym{[}  \Ellent{t_{{\mathrm{1}}}}  \Ellesym{/}  \Ellemv{x}  \Ellesym{]}  \Ellent{t_{{\mathrm{2}}}}  \Ellesym{:}  \Ellent{Y}}{%
{\ElledruleTXXcutName}{}%
}}


\newcommand{\ElledruleTXXexName}[0]{\Elledrulename{T\_ex}}
\newcommand{\ElledruleTXXex}[1]{\Elledrule[#1]{%
\Ellepremise{\Phi  \Ellesym{,}  \Ellemv{x}  \Ellesym{:}  \Ellent{X}  \Ellesym{,}  \Ellemv{y}  \Ellesym{:}  \Ellent{Y}  \Ellesym{,}  \Psi  \vdash_\mathcal{C}  \Ellent{t}  \Ellesym{:}  \Ellent{Z}}%
}{
\Phi  \Ellesym{,}  \Ellemv{z}  \Ellesym{:}  \Ellent{Y}  \Ellesym{,}  \Ellemv{w}  \Ellesym{:}  \Ellent{X}  \Ellesym{,}  \Psi  \vdash_\mathcal{C}   \mathsf{ex}\, \Ellemv{w} , \Ellemv{z} \,\mathsf{with}\, \Ellemv{x} , \Ellemv{y} \,\mathsf{in}\, \Ellent{t}   \Ellesym{:}  \Ellent{Z}}{%
{\ElledruleTXXexName}{}%
}}

\newcommand{\Elledefntty}[1]{\begin{Elledefnblock}[#1]{$\Phi  \vdash_\mathcal{C}  \Ellent{t}  \Ellesym{:}  \Ellent{X}$}{}
\Elleusedrule{\ElledruleTXXax{}}
\Elleusedrule{\ElledruleTXXunitL{}}
\Elleusedrule{\ElledruleTXXunitR{}}
\Elleusedrule{\ElledruleTXXtenL{}}
\Elleusedrule{\ElledruleTXXtenR{}}
\Elleusedrule{\ElledruleTXXimpL{}}
\Elleusedrule{\ElledruleTXXimpR{}}
\Elleusedrule{\ElledruleTXXGr{}}
\Elleusedrule{\ElledruleTXXcut{}}
\Elleusedrule{\ElledruleTXXex{}}
\end{Elledefnblock}}

%% defn sty
\newcommand{\ElledruleSXXaxName}[0]{\Elledrulename{S\_ax}}
\newcommand{\ElledruleSXXax}[1]{\Elledrule[#1]{%
}{
\Ellemv{x}  \Ellesym{:}  \Ellent{A}  \vdash_\mathcal{L}  \Ellemv{x}  \Ellesym{:}  \Ellent{A}}{%
{\ElledruleSXXaxName}{}%
}}


\newcommand{\ElledruleSXXunitLOneName}[0]{\Elledrulename{S\_unitL1}}
\newcommand{\ElledruleSXXunitLOne}[1]{\Elledrule[#1]{%
\Ellepremise{\Gamma  \Ellesym{;}  \Delta  \vdash_\mathcal{L}  \Ellent{s}  \Ellesym{:}  \Ellent{A}}%
}{
\Gamma  \Ellesym{;}  \Ellemv{x}  \Ellesym{:}   \mathsf{Unit}   \Ellesym{;}  \Delta  \vdash_\mathcal{L}   \mathsf{let}\, \Ellemv{x}  :   \mathsf{Unit}  \,\mathsf{be}\,  \mathsf{triv}  \,\mathsf{in}\, \Ellent{s}   \Ellesym{:}  \Ellent{A}}{%
{\ElledruleSXXunitLOneName}{}%
}}


\newcommand{\ElledruleSXXunitLTwoName}[0]{\Elledrulename{S\_unitL2}}
\newcommand{\ElledruleSXXunitLTwo}[1]{\Elledrule[#1]{%
\Ellepremise{\Gamma  \Ellesym{;}  \Delta  \vdash_\mathcal{L}  \Ellent{s}  \Ellesym{:}  \Ellent{A}}%
}{
\Gamma  \Ellesym{;}  \Ellemv{x}  \Ellesym{:}   \mathsf{Unit}   \Ellesym{;}  \Delta  \vdash_\mathcal{L}   \mathsf{let}\, \Ellemv{x}  :   \mathsf{Unit}  \,\mathsf{be}\,  \mathsf{triv}  \,\mathsf{in}\, \Ellent{s}   \Ellesym{:}  \Ellent{A}}{%
{\ElledruleSXXunitLTwoName}{}%
}}


\newcommand{\ElledruleSXXunitRName}[0]{\Elledrulename{S\_unitR}}
\newcommand{\ElledruleSXXunitR}[1]{\Elledrule[#1]{%
}{
 \cdot   \vdash_\mathcal{L}   \mathsf{triv}   \Ellesym{:}   \mathsf{Unit} }{%
{\ElledruleSXXunitRName}{}%
}}


\newcommand{\ElledruleSXXexName}[0]{\Elledrulename{S\_ex}}
\newcommand{\ElledruleSXXex}[1]{\Elledrule[#1]{%
\Ellepremise{\Gamma  \Ellesym{;}  \Ellemv{x}  \Ellesym{:}  \Ellent{X}  \Ellesym{;}  \Ellemv{y}  \Ellesym{:}  \Ellent{Y}  \Ellesym{;}  \Delta  \vdash_\mathcal{L}  \Ellent{s}  \Ellesym{:}  \Ellent{A}}%
}{
\Gamma  \Ellesym{;}  \Ellemv{z}  \Ellesym{:}  \Ellent{Y}  \Ellesym{;}  \Ellemv{w}  \Ellesym{:}  \Ellent{X}  \Ellesym{;}  \Delta  \vdash_\mathcal{L}   \mathsf{ex}\, \Ellemv{w} , \Ellemv{z} \,\mathsf{with}\, \Ellemv{x} , \Ellemv{y} \,\mathsf{in}\, \Ellent{s}   \Ellesym{:}  \Ellent{A}}{%
{\ElledruleSXXexName}{}%
}}


\newcommand{\ElledruleSXXtenLOneName}[0]{\Elledrulename{S\_tenL1}}
\newcommand{\ElledruleSXXtenLOne}[1]{\Elledrule[#1]{%
\Ellepremise{\Gamma  \Ellesym{;}  \Ellemv{x}  \Ellesym{:}  \Ellent{X}  \Ellesym{;}  \Ellemv{y}  \Ellesym{:}  \Ellent{Y}  \Ellesym{;}  \Delta  \vdash_\mathcal{L}  \Ellent{s}  \Ellesym{:}  \Ellent{A}}%
}{
\Gamma  \Ellesym{;}  \Ellemv{z}  \Ellesym{:}  \Ellent{X}  \otimes  \Ellent{Y}  \Ellesym{;}  \Delta  \vdash_\mathcal{L}   \mathsf{let}\, \Ellemv{z}  :  \Ellent{X}  \otimes  \Ellent{Y} \,\mathsf{be}\, \Ellemv{x}  \otimes  \Ellemv{y} \,\mathsf{in}\, \Ellent{s}   \Ellesym{:}  \Ellent{A}}{%
{\ElledruleSXXtenLOneName}{}%
}}


\newcommand{\ElledruleSXXtenLTwoName}[0]{\Elledrulename{S\_tenL2}}
\newcommand{\ElledruleSXXtenLTwo}[1]{\Elledrule[#1]{%
\Ellepremise{\Gamma  \Ellesym{;}  \Ellemv{x}  \Ellesym{:}  \Ellent{A}  \Ellesym{;}  \Ellemv{y}  \Ellesym{:}  \Ellent{B}  \Ellesym{;}  \Delta  \vdash_\mathcal{L}  \Ellent{s}  \Ellesym{:}  \Ellent{C}}%
}{
\Gamma  \Ellesym{;}  \Ellemv{z}  \Ellesym{:}  \Ellent{A}  \triangleright  \Ellent{B}  \Ellesym{;}  \Delta  \vdash_\mathcal{L}   \mathsf{let}\, \Ellemv{z}  :  \Ellent{A}  \triangleright  \Ellent{B} \,\mathsf{be}\, \Ellemv{x}  \triangleright  \Ellemv{y} \,\mathsf{in}\, \Ellent{s}   \Ellesym{:}  \Ellent{C}}{%
{\ElledruleSXXtenLTwoName}{}%
}}


\newcommand{\ElledruleSXXtenRName}[0]{\Elledrulename{S\_tenR}}
\newcommand{\ElledruleSXXtenR}[1]{\Elledrule[#1]{%
\Ellepremise{ \Gamma  \vdash_\mathcal{L}  \Ellent{s_{{\mathrm{1}}}}  \Ellesym{:}  \Ellent{A}  \quad  \Delta  \vdash_\mathcal{L}  \Ellent{s_{{\mathrm{2}}}}  \Ellesym{:}  \Ellent{B} }%
}{
\Gamma  \Ellesym{;}  \Delta  \vdash_\mathcal{L}  \Ellent{s_{{\mathrm{1}}}}  \triangleright  \Ellent{s_{{\mathrm{2}}}}  \Ellesym{:}  \Ellent{A}  \triangleright  \Ellent{B}}{%
{\ElledruleSXXtenRName}{}%
}}


\newcommand{\ElledruleSXXimpLName}[0]{\Elledrulename{S\_impL}}
\newcommand{\ElledruleSXXimpL}[1]{\Elledrule[#1]{%
\Ellepremise{ \Phi  \vdash_\mathcal{C}  \Ellent{t}  \Ellesym{:}  \Ellent{X}  \quad  \Gamma  \Ellesym{;}  \Ellemv{x}  \Ellesym{:}  \Ellent{Y}  \Ellesym{;}  \Delta  \vdash_\mathcal{L}  \Ellent{s}  \Ellesym{:}  \Ellent{A} }%
}{
\Gamma  \Ellesym{;}  \Ellemv{y}  \Ellesym{:}  \Ellent{X}  \multimap  \Ellent{Y}  \Ellesym{;}  \Phi  \Ellesym{;}  \Delta  \vdash_\mathcal{L}  \Ellesym{[}   \Ellemv{y}   \Ellent{t}   \Ellesym{/}  \Ellemv{x}  \Ellesym{]}  \Ellent{s}  \Ellesym{:}  \Ellent{A}}{%
{\ElledruleSXXimpLName}{}%
}}


\newcommand{\ElledruleSXXimprLName}[0]{\Elledrulename{S\_imprL}}
\newcommand{\ElledruleSXXimprL}[1]{\Elledrule[#1]{%
\Ellepremise{ \Gamma  \vdash_\mathcal{L}  \Ellent{s_{{\mathrm{1}}}}  \Ellesym{:}  \Ellent{A}  \quad  \Delta_{{\mathrm{1}}}  \Ellesym{;}  \Ellemv{x}  \Ellesym{:}  \Ellent{B}  \Ellesym{;}  \Delta_{{\mathrm{2}}}  \vdash_\mathcal{L}  \Ellent{s_{{\mathrm{2}}}}  \Ellesym{:}  \Ellent{C} }%
}{
\Delta_{{\mathrm{1}}}  \Ellesym{;}  \Ellemv{y}  \Ellesym{:}  \Ellent{A}  \rightharpoonup  \Ellent{B}  \Ellesym{;}  \Gamma  \Ellesym{;}  \Delta_{{\mathrm{2}}}  \vdash_\mathcal{L}  \Ellesym{[}   \mathsf{app}_r\, \Ellemv{y} \, \Ellent{s_{{\mathrm{1}}}}   \Ellesym{/}  \Ellemv{x}  \Ellesym{]}  \Ellent{s_{{\mathrm{2}}}}  \Ellesym{:}  \Ellent{C}}{%
{\ElledruleSXXimprLName}{}%
}}


\newcommand{\ElledruleSXXimprRName}[0]{\Elledrulename{S\_imprR}}
\newcommand{\ElledruleSXXimprR}[1]{\Elledrule[#1]{%
\Ellepremise{\Gamma  \Ellesym{;}  \Ellemv{x}  \Ellesym{:}  \Ellent{A}  \vdash_\mathcal{L}  \Ellent{s}  \Ellesym{:}  \Ellent{B}}%
}{
\Gamma  \vdash_\mathcal{L}   \lambda_r  \Ellemv{x}  :  \Ellent{A} . \Ellent{s}   \Ellesym{:}  \Ellent{A}  \rightharpoonup  \Ellent{B}}{%
{\ElledruleSXXimprRName}{}%
}}


\newcommand{\ElledruleSXXimplLName}[0]{\Elledrulename{S\_implL}}
\newcommand{\ElledruleSXXimplL}[1]{\Elledrule[#1]{%
\Ellepremise{ \Gamma  \vdash_\mathcal{L}  \Ellent{s_{{\mathrm{1}}}}  \Ellesym{:}  \Ellent{A}  \quad  \Delta_{{\mathrm{1}}}  \Ellesym{;}  \Ellemv{x}  \Ellesym{:}  \Ellent{B}  \Ellesym{;}  \Delta_{{\mathrm{2}}}  \vdash_\mathcal{L}  \Ellent{s_{{\mathrm{2}}}}  \Ellesym{:}  \Ellent{C} }%
}{
\Delta_{{\mathrm{1}}}  \Ellesym{;}  \Gamma  \Ellesym{;}  \Ellemv{y}  \Ellesym{:}  \Ellent{B}  \leftharpoonup  \Ellent{A}  \Ellesym{;}  \Delta_{{\mathrm{2}}}  \vdash_\mathcal{L}  \Ellesym{[}   \mathsf{app}_l\, \Ellemv{y} \, \Ellent{s_{{\mathrm{1}}}}   \Ellesym{/}  \Ellemv{x}  \Ellesym{]}  \Ellent{s_{{\mathrm{2}}}}  \Ellesym{:}  \Ellent{C}}{%
{\ElledruleSXXimplLName}{}%
}}


\newcommand{\ElledruleSXXimplRName}[0]{\Elledrulename{S\_implR}}
\newcommand{\ElledruleSXXimplR}[1]{\Elledrule[#1]{%
\Ellepremise{\Ellemv{x}  \Ellesym{:}  \Ellent{A}  \Ellesym{;}  \Gamma  \vdash_\mathcal{L}  \Ellent{s}  \Ellesym{:}  \Ellent{B}}%
}{
\Gamma  \vdash_\mathcal{L}   \lambda_l  \Ellemv{x}  :  \Ellent{A} . \Ellent{s}   \Ellesym{:}  \Ellent{B}  \leftharpoonup  \Ellent{A}}{%
{\ElledruleSXXimplRName}{}%
}}


\newcommand{\ElledruleSXXFlName}[0]{\Elledrulename{S\_Fl}}
\newcommand{\ElledruleSXXFl}[1]{\Elledrule[#1]{%
\Ellepremise{\Gamma  \Ellesym{;}  \Ellemv{x}  \Ellesym{:}  \Ellent{X}  \Ellesym{;}  \Delta  \vdash_\mathcal{L}  \Ellent{s}  \Ellesym{:}  \Ellent{A}}%
}{
\Gamma  \Ellesym{;}  \Ellemv{y}  \Ellesym{:}   \mathsf{F} \Ellent{X}   \Ellesym{;}  \Delta  \vdash_\mathcal{L}   \mathsf{let}\, \Ellemv{y}  :   \mathsf{F} \Ellent{X}  \,\mathsf{be}\,  \mathsf{F}\, \Ellemv{x}  \,\mathsf{in}\, \Ellent{s}   \Ellesym{:}  \Ellent{A}}{%
{\ElledruleSXXFlName}{}%
}}


\newcommand{\ElledruleSXXFrName}[0]{\Elledrulename{S\_Fr}}
\newcommand{\ElledruleSXXFr}[1]{\Elledrule[#1]{%
\Ellepremise{\Phi  \vdash_\mathcal{C}  \Ellent{t}  \Ellesym{:}  \Ellent{X}}%
}{
\Phi  \vdash_\mathcal{L}   \mathsf{F} \Ellent{t}   \Ellesym{:}   \mathsf{F} \Ellent{X} }{%
{\ElledruleSXXFrName}{}%
}}


\newcommand{\ElledruleSXXGlName}[0]{\Elledrulename{S\_Gl}}
\newcommand{\ElledruleSXXGl}[1]{\Elledrule[#1]{%
\Ellepremise{\Gamma  \Ellesym{;}  \Ellemv{x}  \Ellesym{:}  \Ellent{A}  \Ellesym{;}  \Delta  \vdash_\mathcal{L}  \Ellent{s}  \Ellesym{:}  \Ellent{B}}%
}{
\Gamma  \Ellesym{;}  \Ellemv{y}  \Ellesym{:}   \mathsf{G} \Ellent{A}   \Ellesym{;}  \Delta  \vdash_\mathcal{L}   \mathsf{let}\, \Ellemv{y}  :   \mathsf{G} \Ellent{A}  \,\mathsf{be}\,  \mathsf{G}\, \Ellemv{x}  \,\mathsf{in}\, \Ellent{s}   \Ellesym{:}  \Ellent{B}}{%
{\ElledruleSXXGlName}{}%
}}


\newcommand{\ElledruleSXXcutOneName}[0]{\Elledrulename{S\_cut1}}
\newcommand{\ElledruleSXXcutOne}[1]{\Elledrule[#1]{%
\Ellepremise{ \Phi  \vdash_\mathcal{C}  \Ellent{t}  \Ellesym{:}  \Ellent{X}  \quad  \Gamma_{{\mathrm{1}}}  \Ellesym{;}  \Ellemv{x}  \Ellesym{:}  \Ellent{X}  \Ellesym{;}  \Gamma_{{\mathrm{2}}}  \vdash_\mathcal{L}  \Ellent{s}  \Ellesym{:}  \Ellent{A} }%
}{
\Gamma_{{\mathrm{1}}}  \Ellesym{;}  \Phi  \Ellesym{;}  \Gamma_{{\mathrm{1}}}  \vdash_\mathcal{L}  \Ellesym{[}  \Ellent{t}  \Ellesym{/}  \Ellemv{x}  \Ellesym{]}  \Ellent{s}  \Ellesym{:}  \Ellent{A}}{%
{\ElledruleSXXcutOneName}{}%
}}


\newcommand{\ElledruleSXXcutTwoName}[0]{\Elledrulename{S\_cut2}}
\newcommand{\ElledruleSXXcutTwo}[1]{\Elledrule[#1]{%
\Ellepremise{ \Gamma  \vdash_\mathcal{L}  \Ellent{s_{{\mathrm{1}}}}  \Ellesym{:}  \Ellent{A}  \quad  \Delta_{{\mathrm{1}}}  \Ellesym{;}  \Ellemv{x}  \Ellesym{:}  \Ellent{A}  \Ellesym{;}  \Delta_{{\mathrm{2}}}  \vdash_\mathcal{L}  \Ellent{s_{{\mathrm{2}}}}  \Ellesym{:}  \Ellent{B} }%
}{
\Delta_{{\mathrm{1}}}  \Ellesym{;}  \Gamma  \Ellesym{;}  \Delta_{{\mathrm{2}}}  \vdash_\mathcal{L}  \Ellesym{[}  \Ellent{s_{{\mathrm{1}}}}  \Ellesym{/}  \Ellemv{x}  \Ellesym{]}  \Ellent{s_{{\mathrm{2}}}}  \Ellesym{:}  \Ellent{B}}{%
{\ElledruleSXXcutTwoName}{}%
}}

\newcommand{\Elledefnsty}[1]{\begin{Elledefnblock}[#1]{$\Gamma  \vdash_\mathcal{L}  \Ellent{s}  \Ellesym{:}  \Ellent{A}$}{}
\Elleusedrule{\ElledruleSXXax{}}
\Elleusedrule{\ElledruleSXXunitLOne{}}
\Elleusedrule{\ElledruleSXXunitLTwo{}}
\Elleusedrule{\ElledruleSXXunitR{}}
\Elleusedrule{\ElledruleSXXex{}}
\Elleusedrule{\ElledruleSXXtenLOne{}}
\Elleusedrule{\ElledruleSXXtenLTwo{}}
\Elleusedrule{\ElledruleSXXtenR{}}
\Elleusedrule{\ElledruleSXXimpL{}}
\Elleusedrule{\ElledruleSXXimprL{}}
\Elleusedrule{\ElledruleSXXimprR{}}
\Elleusedrule{\ElledruleSXXimplL{}}
\Elleusedrule{\ElledruleSXXimplR{}}
\Elleusedrule{\ElledruleSXXFl{}}
\Elleusedrule{\ElledruleSXXFr{}}
\Elleusedrule{\ElledruleSXXGl{}}
\Elleusedrule{\ElledruleSXXcutOne{}}
\Elleusedrule{\ElledruleSXXcutTwo{}}
\end{Elledefnblock}}


\newcommand{\ElledefnsJtype}{
\Elledefntty{}\Elledefnsty{}}

\newcommand{\Elledefnss}{
\ElledefnsJtype
}

\newcommand{\Elleall}{\Ellemetavars\\[0pt]
\Ellegrammar\\[5.0mm]
\Elledefnss}


\renewcommand{\Elledrule}[4][]{{\displaystyle\frac{\begin{array}{l}#2\end{array}}{#3}\,#4}}
\renewcommand{\ElledruleTXXaxName}{\mathcal{C}\text{-ax}}
\renewcommand{\ElledruleTXXunitLName}{\mathcal{C}\text{-}\mathsf{Unit}_L}
\renewcommand{\ElledruleTXXunitRName}{\mathcal{C}\text{-}\mathsf{Unit}_L}
\renewcommand{\ElledruleTXXtenLName}{\mathcal{C}\text{-}\otimes_L}
\renewcommand{\ElledruleTXXtenRName}{\mathcal{C}\text{-}\otimes_R}
\renewcommand{\ElledruleTXXimpLName}{\mathcal{C}\text{-}\multimap_L}
\renewcommand{\ElledruleTXXimpRName}{\mathcal{C}\text{-}\multimap_R}
\renewcommand{\ElledruleTXXGrName}{\mathcal{C}\text{-}\mathsf{G}_R}
\renewcommand{\ElledruleTXXcutName}{\mathcal{C}\text{-}\mathsf{Cut}}
\renewcommand{\ElledruleTXXexName}{\mathcal{C}\text{-}\mathsf{ex}}
\renewcommand{\ElledruleSXXaxName}{\mathcal{L}\text{-ax}}
\renewcommand{\ElledruleSXXunitLOneName}{\mathcal{LC}\text{-}\mathsf{Unit}_L}
\renewcommand{\ElledruleSXXunitLTwoName}{\mathcal{L}\text{-}\mathsf{Unit}_L}
\renewcommand{\ElledruleSXXunitRName}{\mathcal{L}\text{-}\mathsf{Unit}_R}
\renewcommand{\ElledruleSXXexName}{\mathcal{L}\text{-}\mathsf{ex}}
\renewcommand{\ElledruleSXXtenLOneName}{\mathcal{LC}\text{-}\otimes_L}
\renewcommand{\ElledruleSXXtenLTwoName}{\mathcal{L}\text{-}\otimes_L}
\renewcommand{\ElledruleSXXtenRName}{\mathcal{L}\text{-}\otimes_R}
\renewcommand{\ElledruleSXXimpLName}{\mathcal{L}\text{-}\multimap_L}
\renewcommand{\ElledruleSXXimprLName}{\mathcal{L}\text{-}\rightharpoonup_L}
\renewcommand{\ElledruleSXXimprRName}{\mathcal{L}\text{-}\rightharpoonup_R}
\renewcommand{\ElledruleSXXimplLName}{\mathcal{L}\text{-}\leftharpoonup_L}
\renewcommand{\ElledruleSXXimplRName}{\mathcal{L}\text{-}\leftharpoonup_R}
\renewcommand{\ElledruleSXXFlName}{\mathcal{L}\text{-}\mathsf{F}_L}
\renewcommand{\ElledruleSXXFrName}{\mathcal{L}\text{-}\mathsf{F}_R}
\renewcommand{\ElledruleSXXGlName}{\mathcal{L}\text{-}\mathsf{G}_L}
\renewcommand{\ElledruleSXXcutOneName}{\mathcal{LC}\text{-}\mathsf{Cut}}
\renewcommand{\ElledruleSXXcutTwoName}{\mathcal{L}\text{-}\mathsf{Cut}}

\input{dePaiva-Eades-LAM-ott}
\renewcommand{\Ldrule}[4][]{{\displaystyle\frac{\begin{array}{l}#2\end{array}}{#3}\,\Ldrulename{#4}}}
%% End Ott Renaming

%% This renames Barr's \to to \mto.  This allows us to use \to for imp
%% and \mto for a inline morphism.
\let\mto\to
\let\to\relax
\newcommand{\to}{\rightarrow}
\newcommand{\ndto}[1]{\to_{#1}}
\newcommand{\ndwedge}[1]{\wedge_{#1}}
\newcommand{\rto}{\leftharpoonup}
\newcommand{\lto}{\rightharpoonup}
\newcommand{\tri}{\triangleright}
\newcommand{\LNL}{\mathit{LNL}}

% Commands that are useful for writing about type theory and programming language design.
%% \newcommand{\case}[4]{\text{case}\ #1\ \text{of}\ #2\text{.}#3\text{,}#2\text{.}#4}
\newcommand{\interp}[1]{\llbracket #1 \rrbracket}
\newcommand{\normto}[0]{\rightsquigarrow^{!}}
\newcommand{\join}[0]{\downarrow}
\newcommand{\redto}[0]{\rightsquigarrow}
\newcommand{\nat}[0]{\mathbb{N}}
\newcommand{\fun}[2]{\lambda #1.#2}
\newcommand{\CRI}[0]{\text{CR-Norm}}
\newcommand{\CRII}[0]{\text{CR-Pres}}
\newcommand{\CRIII}[0]{\text{CR-Prog}}
\newcommand{\subexp}[0]{\sqsubseteq}

%% Must include \usepackage{mathrsfs} for this to work.

%% \date{}

\let\b\relax
\let\d\relax
\let\t\relax
\let\r\relax
\let\c\relax
\let\j\relax
\let\wn\relax
\let\H\relax

% Cat commands.
\newcommand{\powerset}[1]{\mathcal{P}(#1)}
\newcommand{\cat}[1]{\mathcal{#1}}
\newcommand{\func}[1]{\mathsf{#1}}
\newcommand{\iso}[0]{\mathsf{iso}}
\newcommand{\H}[0]{\func{H}}
\newcommand{\J}[0]{\func{J}}
\newcommand{\catop}[1]{\cat{#1}^{\mathsf{op}}}
\newcommand{\Hom}[3]{\mathsf{Hom}_{\cat{#1}}(#2,#3)}
\newcommand{\limp}[0]{\multimap}
\newcommand{\colimp}[0]{\multimapdotinv}
\newcommand{\dial}[1]{\mathsf{Dial_{#1}}(\mathsf{Sets^{op}})}
\newcommand{\dialSets}[1]{\mathsf{Dial_{#1}}(\mathsf{Sets})}
\newcommand{\dcSets}[1]{\mathsf{DC_{#1}}(\mathsf{Sets})}
\newcommand{\sets}[0]{\mathsf{Sets}}
\newcommand{\obj}[1]{\mathsf{Obj}(#1)}
\newcommand{\mor}[1]{\mathsf{Mor(#1)}}
\newcommand{\id}[0]{\mathsf{id}}
\newcommand{\Id}[0]{\mathsf{Id}}
\newcommand{\lett}[0]{\mathsf{let}\,}
\newcommand{\inn}[0]{\,\mathsf{in}\,}
\newcommand{\cur}[1]{\mathsf{cur}(#1)}
\newcommand{\curi}[1]{\mathsf{cur}^{-1}(#1)}

\newcommand{\w}[1]{\mathsf{weak}_{#1}}
\newcommand{\c}[1]{\mathsf{contra}_{#1}}
\newcommand{\cL}[1]{\mathsf{contraL}_{#1}}
\newcommand{\cR}[1]{\mathsf{contraR}_{#1}}
\newcommand{\e}[1]{\mathsf{ex}_{#1}}

\newcommand{\m}[1]{\mathsf{m}_{#1}}
\newcommand{\n}[1]{\mathsf{n}_{#1}}
\newcommand{\b}[1]{\mathsf{b}_{#1}}
\newcommand{\d}[1]{\mathsf{d}_{#1}}
\newcommand{\h}[1]{\mathsf{h}_{#1}}
\newcommand{\p}[1]{\mathsf{p}_{#1}}
\newcommand{\q}[1]{\mathsf{q}_{#1}}
\newcommand{\t}[1]{\mathsf{t}_{#1}}
\newcommand{\r}[1]{\mathsf{r}_{#1}}
\newcommand{\s}[1]{\mathsf{s}_{#1}}
\newcommand{\j}[1]{\mathsf{j}_{#1}}
\newcommand{\jinv}[1]{\mathsf{j}^{-1}_{#1}}
\newcommand{\wn}[0]{\mathop{?}}
\newcommand{\codiag}[1]{\bigtriangledown_{#1}}

\newcommand{\seq}{\rhd}

\newenvironment{changemargin}[2]{%
  \begin{list}{}{%
    \setlength{\topsep}{0pt}%
    \setlength{\leftmargin}{#1}%
    \setlength{\rightmargin}{#2}%
    \setlength{\listparindent}{\parindent}%
    \setlength{\itemindent}{\parindent}%
    \setlength{\parsep}{\parskip}%
  }%
  \item[]}{\end{list}}

\newenvironment{diagram}{
  \begin{center}
    \begin{math}
      \bfig
}{
      \efig
    \end{math}
  \end{center}
}

\newtheorem{theorem}{Theorem}
\newtheorem{lemma}[theorem]{Lemma}
\newtheorem{corollary}[theorem]{Corollary}
\newtheorem{definition}[theorem]{Definition}
\newtheorem{proposition}[theorem]{Proposition}
\newtheorem{example}[theorem]{Example}

\title{On the Lambek Calculus with an Exchange Modality}

\author{Jiaming Jiang
\institute{Computer Science \\ North Carolina State University \\ Raleigh, North Carolina, USA}
\email{jjiang13@ncsu.edu}
\and
Harley Eades III
\institute{Computer Science \\ Augusta University \\ Augusta, Georgia, USA}
\email{harley.eades@gmail.com}
\and
Valeria de Paiva
\institute{Nuance Communications \\ Sunnyvale, California, USA}
\email{valeria.depaiva@gmail.com}
}
\def\titlerunning{Adjoint Models for the Lambek Calculus}
\def\authorrunning{J. Jiang \& H. Eades III \& V. de Paiva}

\begin{document}
\maketitle 

\begin{abstract}
  In this paper we introduce Commutative/Non-Commutative Logic (CNC
  logic) and two categorical models for CNC logic.  This work
  abstracts Benton's Linear/Non-Linear Logic \cite{Benton:1994} by
  removing the existence of the exchange structural rule. One should
  view this logic as composed of two logics; one sitting to the left
  of the other.  On the left, there is intuitionistic linear logic,
  and on the right is a mixed commutative/non-commutative
  formalization of the Lambek calculus. Then both of these logics are
  connected via a pair of monoidal adjoint functors.  An exchange
  modality is then derivable within the logic using the adjunction
  between both sides.  Thus, the adjoint functors allow one to pull
  the exchange structural rule from the left side to the right side.
  We then give a categorical model in terms of a monoidal adjunction,
  and then a concrete model in terms of dialectica Lambek spaces.
\end{abstract}

\section{Introduction}
\label{sec:introduction}
Perhaps the most elegant model of intuitionistic linear logic is
Benton's linear/non-linear (LNL) models \cite{Benton:1994}. On the
semantic side, LNL models are very compact and intuitive, but on the
syntactic side, they correspond to a logic and type theory, called LNL
logic, that allows one to reason with or without weakening and
contraction without the need to annotate every formula of the logic
with the of-course!  modality.  This is possible because formulas like
$A \to B$ are primitives of the logic, and not encoded using the
of-course! modality.  Therefore, we ask the question, ``can a similar
elegant, intuitive, and flexible model and corresponding logic be
defined for the Lambek Calculus?''

LNL models are symmetric monoidal adjunctions, $\cat{C} : F \dashv G :
\cat{I}$, where $\cat{C}$ is a symmetric monoidal closed category with
weakening and contraction, hence is a cartesian closed category, and
$\cat{I}$ is a symmetric monoidal closed category.  Thus, a LNL model
consists of a model of intuitionistic logic on the left and a model of
intuitionistic linear logic on the right related via a pair of adjoint
functors.  Then Benton shows that the of-course! modality can be
recovered by $!A = F(G\,A)$.  These models suggest a more general
framework for working with structural rules.

Our main contributions in this paper are to the Lambek Calculus, but
we make use of a new more general framework that will put this work in
line with planned future work.  Suppose $\cat{M}_1$ and $\cat{M}_2$
are two categories with a bifunctor $\odot_i : \cat{M}_i \times
\cat{M}_i \mto \cat{M}_i$, a distinguished object $I_i \in
\mathsf{Obj}(\cat{M}_i)$, and the two natural isomorphisms $\lambda_A
: A \odot I_i \mto A$ and $\rho_A : I_i \odot A \mto A$.  We call
these categories \emph{magmoidal categories with a unit}.  Examples of
magmoidal categories with a unit are monoidal categories, symmetric
monoidal categories, and cartesian closed categories.

Benton's LNL models can be generalized to magmodial categories.
Simply, take an adjunction $\cat{M}_1 : F \dashv G : \cat{M}_2$ called
an \emph{adjoint model} where the functors $F$ and $G$ preserve the
magmoidal structure similarly to monoidal functors, but without the
coherence diagram for the associator.  Now if we add to $\cat{M}_1$
and $\cat{M}_2$ some structural rules, for example, by making
$\cat{M}_1$ a symmetric monoidal category and $\cat{M}_2$ a monoidal
category, then we obtain an adjoint model that corresponds to a logic
where intuitionistic linear logic ($\cat{M}_1$) and the Lambek
calculus ($\cat{M}_2$) are mixed similarly to LNL models.  However,
this also leads to a new solution to an existing problem.

Linguists \cite{?} have wondered if a modality for exchange can be
added to the Lambek Calculus similarly to how the of-course! modality
adds weakening and contraction to linear logic.  de Paiva and
Eades~\cite{?} show that this is possible by adding a modality $\kappa
A$ with the property that $(\kappa A \otimes B) \multimap (B \otimes
\kappa A)$.  The adjoint model introduced above induces the modality
$eA = F(G\,A)$ with the property $eA \otimes eB \multimap eB \otimes
eA$.  Thus providing a second solution to the problem.

Linguists as well as computer scientists have found many applications
of substructural logics that contain more than one tensor product
satisfying different structural rules.  For example, a commutative
tensor product and a non-commutative tensor product within the same
logic.  An example of such a logic is the logic of bunched
implications, but there sequents $\Gamma \vdash A$ depend on contexts
$\Gamma$ that are trees with multiple different branching nodes
instead of lists.  The branching nodes are necessary to accommodate
multiple different tensor products with different structural
properties.  We introduce, but leave the details to future work, the
idea of using adjoint models to build similar logics, but without the
need for multiple different branching nodes.  Furthermore, adjoints
compose, and this property can be exploited to compose modalities.

We introduce the idea above of having a modality for exchange, but
what about associativity, weakening and contraction?  Indeed it is
possible to give modalities for these structural rules as well using
adjoint models.  Now that we have each structural rule isolated into
their own modality is it possible to put them together to form new
modalities that combine structural rules?  The answer to this question
has already been show to be positive, at least for weakening and
contraction, but we extend this line of work to include exchange, and
in the future associativity as well.  Jacobs~\cite{JACOBS199473} used
monads and distributive laws to relate modalities, but
Melli{\'e}s~\cite{Mellies:2004} shows how to use adjunctions to
accomplish the same, but in a more intuitive and natural way.  Thus,
this part of our work is a natural extension of Melli{\'e}s'.

This work can be used to define a general core logic that is capable
of encoding several different substructural logics.  Some examples are
intuitionistic linear logic, affine logic, contraction logic, and the
(non-associative) Lambek Calculus.  This does have some similarities
to Licata et al.~\cite{licata2017fibrational}, but there they embrace
the tree based structures with multiple different branching node for
contexts from the logic of bunched implications.  However, our work is
trying to understand if this is truly necessary, and if adjunctions
provide a means of, at the very least, hiding these multiple branching
nodes.  Thus, adjunctions may lead to a simplification.

%%% Local Variables: 
%%% mode: latex
%%% TeX-master: main.tex
%%% End:

% section introduction (end)

\section{Sequent Calculus of CNC Logic}
\label{sec:sequent-calc}
We now introduce Commutative/Non-commutative (CNC) logic in the form of a
sequent calculus. One should view this logic as composed of two logics; one
sitting to the left of the other. On the left, there is intuitionistic
linear logic, denoted by $\cat{C}$ and on the right is the Lambek calculus
denoted by $\cat{L}$. Then we connect these two systems by a pair of
monoidal adjoint functors $\cat{C} : \func{F} \dashv \func{G} : \cat{L}$.
Keeping this intuition in mind we now define the syntax for CNC logic.

\begin{definition}
  \label{def:Lambek-syntax}
  The following grammar describes the syntax of the sequent calculus of
  CNC logic:
  \begin{center}\vspace{-3px}\small
    \begin{math}
      \begin{array}{lll}
        \text{($\cat{C}$-Types)} & \SCnt{W},\SCnt{X},\SCnt{Y},\SCnt{Z} ::=  \mathsf{Unit}  \mid \SCnt{X}  \otimes  \SCnt{Y} \mid \SCnt{X}  \multimap  \SCnt{Y} \mid  \mathsf{G} \SCnt{A} \\
        \text{($\cat{L}$-Types)} & \SCnt{A},\SCnt{B},\SCnt{C},D ::=  \mathsf{Unit}  \mid \SCnt{A}  \triangleright  \SCnt{B} \mid \SCnt{A}  \rightharpoonup  \SCnt{B} \mid \SCnt{B}  \leftharpoonup  \SCnt{A} \mid  \mathsf{F} \SCnt{X} \\
        \text{($\cat{C}$-Contexts)} & \Phi,\Psi ::=  \cdot  \mid \SCnt{X} \mid \Phi  \SCsym{,}  \Psi\\
        \text{($\cat{L}$-Contexts)} & \Gamma,\Delta ::=  \cdot  \mid \SCnt{A} \mid \SCnt{X} \mid \Gamma  \SCsym{;}  \Delta\\
      \end{array}
    \end{math}
  \end{center}
\end{definition}

The syntax for $\cat{C}$-types are the standard types for intuitionistic
linear logic. We have a constant $ \mathsf{Unit} $, tensor product $\SCnt{X}  \otimes  \SCnt{Y}$,
and linear implication $\SCnt{X}  \multimap  \SCnt{Y}$, but just as in LNL logic we also have a
type $ \mathsf{G} \SCnt{A} $ where $\SCnt{A}$ is an $\cat{L}$-type; that is, a type from the
non-commutative side corresponding to the right-adjoint functor between
$\cat{L}$ and $\cat{C}$. This functor can be used to import types from the
non-commutative side into the commutative side. Now a sequent in the the
commutative side is denoted by $\Phi  \vdash_\mathcal{C}  \SCnt{X}$ where $\Phi$ is a
$\cat{C}$-context, which is a sequence of types $\SCnt{X}$.

The non-commutative side is a bit more interesting than the commutative side
just introduced. Sequents in the non-commutative side are denoted by
$\Gamma  \vdash_\mathcal{L}  \SCnt{A}$ where $\Gamma$ is now a $\cat{L}$-context. These contexts are
ordered sequences of types from \emph{both} sides denoted by $\SCnt{B}$ and
$\SCnt{X}$ respectively. Given two contexts $\Gamma$ and $\Delta$ we denote their
concatenation by $\Gamma  \SCsym{;}  \Delta$; we use a semicolon here to emphasize the fact
that the contexts are ordered.

The context consisting of hypotheses from both sides goes back to
Benton~\cite{Benton:1994} and is a property unique to adjoint logics such as
Benton's LNL logic and CNC logic. This is also a very useful property
because it allows one to make use of both sides within the Lambek calculus
without the need to annotate every formula with a modality.

The reader familiar with LNL logic will notice that our sequent,
$\Gamma  \vdash_\mathcal{L}  \SCnt{A}$, differs from Benton's. His is of the form $\Gamma  \SCsym{;}  \Delta  \vdash_\mathcal{L}  \SCnt{A}$,
where $\Gamma$ contains non-linear types, and $\Delta$ contains linear
formulas. Just as Benton remarks, the splitting of his contexts was a
presentational device. One should view his contexts as merged, and hence,
linear formulas were fully mixed with non-linear formulas. Now why did we
not use this presentational device? Because, when contexts from LNL logic
become out of order Benton could use the exchange rule to put them back in
order again, but we no longer have general exchange. Thus, we are not able
to keep the context organized in this way.

The syntax for $\cat{L}$-types are of the typical form for the Lambek
Calculus. We have two unit types $ \mathsf{Unit} $ (one for each side), a
non-commutative tensor product $\SCnt{A}  \triangleright  \SCnt{B}$, right implication $\SCnt{A}  \rightharpoonup  \SCnt{B}$, and left implication $\SCnt{B}  \leftharpoonup  \SCnt{A}$. In standard Lambek
Calculus \cite{Pentus1995}, $\SCnt{A}  \rightharpoonup  \SCnt{B}$ is written as $B / A$ and
$\SCnt{B}  \leftharpoonup  \SCnt{A}$ as $A \backslash B$. We use $\rightharpoonup$ and
$\leftharpoonup$ here instead to indicate they are two directions of
the linear implication $\multimap$.

The sequent calculus for CNC logic can be found in
Figure~\ref{fig:CNC-sequent-calculus}. We split the figure in two: the top
of the figure are the rules of intuitionistic linear logic whose sequents
are the $\mathcal{C}$-sequents denoted by $\Psi  \vdash_\mathcal{C}  \SCnt{X}$, and the bottom of
the figure are the rules for the mixed commutative/non-commutative Lambek
calculus whose sequents are the $\mathcal{L}$-sequents denoted by
$\Gamma  \vdash_\mathcal{L}  \SCnt{A}$, but the two halves are connected via the functor rules
$\SCdruleTXXGrName{}$, $\SCdruleSXXGlName{}$, $\SCdruleSXXFlName{}$, and
$\SCdruleSXXFrName{}$, and the rules $\SCdruleSXXunitLOneName{}$,
$\SCdruleSXXexName{}$, $\SCdruleSXXtenLOneName{}$, $\SCdruleSXXimpLName{}$,
$\NDdruleSXXcutOneName{}$.
\begin{figure}[!h]
  \footnotesize
  \begin{tabular}{|c|}
    \hline\\
    \begin{mathpar}
    \SCdruleTXXax{} \and
    \SCdruleTXXunitL{} \and
    \SCdruleTXXunitR{} \and
    \SCdruleTXXtenL{} \and
    \SCdruleTXXtenR{} \and
    \SCdruleTXXimpL{} \and
    \SCdruleTXXimpR{} \and
    \SCdruleTXXGr{} \and
    \SCdruleTXXex{} \and
    \SCdruleTXXcut{}
    \end{mathpar}\\\\
    \hline
    \\[5px]
    \begin{mathpar}
    \SCdruleSXXax{} \and
    \SCdruleSXXunitLOne{} \and
    \SCdruleSXXunitLTwo{} \and
    \SCdruleSXXunitR{} \and
    \SCdruleSXXex{} \and
    \SCdruleSXXtenLOne{} \and
    \SCdruleSXXtenLTwo{} \and
    \SCdruleSXXtenR{} \and
    \SCdruleSXXimpL{} \and
    \SCdruleSXXimprL{} \and
    \SCdruleSXXimprR{} \and
    \SCdruleSXXimplL{} \and
    \SCdruleSXXimplR{} \and
    \SCdruleSXXFl{} \and
    \SCdruleSXXFr{} \and
    \SCdruleSXXGl{} \and
    \SCdruleSXXcutOne{} \and
    \SCdruleSXXcutTwo{} \and
    \end{mathpar}\\\\
    \hline
  \end{tabular}
  \caption{Sequent Calculus for CNC Logic}
  \label{fig:CNC-sequent-calculus}
\end{figure}

We prove cut elimination for the sequent calculus. We define the
\textit{rank} $|X|$ (resp. $|A|$) of a commutative (resp. non-commutative)
formula to be the number of logical connectives in the proposition. For
instance, $|\SCnt{X}  \otimes  \SCnt{Y}| = |\SCnt{X}| + |\SCnt{Y}| + 1$. The \textit{cut rank}
$c(\Pi)$ of a proof $\Pi$ is one more than the maximum of the ranks of all
the cut formulae in $\Pi$, and $0$ if $\Pi$ is cut-free. Then the
\textit{depth} $d(\Pi)$ of a proof $\Pi$ is the length of the longest path
in the proof tree (so the depth of an axiom is $0$). The key to the proof
of cut elimination is the following lemma, which shows how to transform a
single cut, either by removing it or by replacing it with one or more
simpler cuts.
\begin{lemma}[Cut Reduction]
  \label{lem:cut-reduction}
  The cut-reduction steps are as follows:
  \begin{enumerate}
  \item If $\Pi_1$ is a proof of $\Phi  \vdash_\mathcal{C}  \SCnt{X}$ and $\Pi_2$ is a proof of
  $\Psi_{{\mathrm{1}}}  \SCsym{,}  \SCnt{X}  \SCsym{,}  \Psi_{{\mathrm{2}}}  \vdash_\mathcal{C}  \SCnt{Y}$ with $c(\Pi_1)$, $c(\Pi_2)\leq |X|$, then there exists
  a proof $\Pi$ of $\Psi_{{\mathrm{1}}}  \SCsym{,}  \Phi  \SCsym{,}  \Psi_{{\mathrm{2}}}  \vdash_\mathcal{C}  \SCnt{Y}$ with $c(\Pi)\leq |X|$.
  \item If $\Pi_1$ is a proof of $\Phi  \vdash_\mathcal{C}  \SCnt{X}$ and $\Pi_2$ is a proof of
  $\Gamma_{{\mathrm{1}}}  \SCsym{;}  \SCnt{X}  \SCsym{;}  \Gamma_{{\mathrm{2}}}  \vdash_\mathcal{L}  \SCnt{A}$ with $c(\Pi_1)$, $c(\Pi_2)\leq |X|$, then there
  exists a proof $\Pi$ of $\Gamma_{{\mathrm{1}}}  \SCsym{;}  \Phi  \SCsym{;}  \Gamma_{{\mathrm{2}}}  \vdash_\mathcal{L}  \SCnt{A}$ with $c(\Pi)\leq |X|$.
  \item If $\Pi_1$ is a proof of $\Gamma  \vdash_\mathcal{L}  \SCnt{A}$ and $\Pi_2$ is a proof of
  $\Delta_{{\mathrm{1}}}  \SCsym{;}  \SCnt{A}  \SCsym{;}  \Delta_{{\mathrm{2}}}  \vdash_\mathcal{L}  \SCnt{B}$ with $c(\Pi_1)$, $c(\Pi_2)\leq |A|$, then there
  exists a proof $\Pi$ of $\Delta_{{\mathrm{1}}}  \SCsym{;}  \Gamma  \SCsym{;}  \Delta_{{\mathrm{2}}}  \vdash_\mathcal{L}  \SCnt{B}$ with $c(\Pi)\leq |A|$.
  \end{enumerate}
\end{lemma}
\begin{proof}
  This proof is done case by case on the last step of $\Pi_1$ and
  $\Pi_2$ and by induction on $d(\Pi_1)$ and $d(\Pi_2)$, following
  \cite{Mellies:2009}. For instance, suppose $\Pi_1$ is a proof of
  $\Phi_{{\mathrm{1}}}  \SCsym{,}  \SCnt{X_{{\mathrm{2}}}}  \SCsym{,}  \SCnt{X_{{\mathrm{1}}}}  \SCsym{,}  \Phi_{{\mathrm{2}}}  \vdash_\mathcal{C}  \SCnt{Y}$ and $\Pi_2$ is a proof of $\Psi_{{\mathrm{1}}}  \SCsym{,}  \SCnt{Y}  \SCsym{,}  \Psi_{{\mathrm{2}}}  \vdash_\mathcal{C}  \SCnt{Z}$.  Consider the case where the last step in $\Pi_1$ uses
  the rule $\NDdruleTXXbetaName{}$. $\Pi_1$ can be depicted as
  follows, where the previous steps are denoted by $\pi$:
  \begin{center}
    \scriptsize
    $\Pi_1$:
    \begin{math}
      $$\mprset{flushleft}
      \inferrule* [right={\tiny $\SCdruleTXXexName$}] {
        {
          \begin{array}{c}
            \pi \\
                {\Phi_{{\mathrm{1}}}  \SCsym{,}  \SCnt{X_{{\mathrm{1}}}}  \SCsym{,}  \SCnt{X_{{\mathrm{2}}}}  \SCsym{,}  \Phi_{{\mathrm{2}}}  \vdash_\mathcal{C}  \SCnt{Y}}
          \end{array}
        }
      }{\Phi_{{\mathrm{1}}}  \SCsym{,}  \SCnt{X_{{\mathrm{2}}}}  \SCsym{,}  \SCnt{X_{{\mathrm{1}}}}  \SCsym{,}  \Phi_{{\mathrm{2}}}  \vdash_\mathcal{C}  \SCnt{Y}}
    \end{math}
  \end{center}
  By assumption, $c(\Pi_1),c(\Pi_2)\leq |Y|$. By induction on $\pi$ and
  $\Pi_2$, there is a proof $\Pi'$ for the sequent \\
  $\Psi_{{\mathrm{1}}}  \SCsym{,}  \Phi_{{\mathrm{1}}}  \SCsym{,}  \SCnt{X_{{\mathrm{1}}}}  \SCsym{,}  \SCnt{X_{{\mathrm{2}}}}  \SCsym{,}  \Phi_{{\mathrm{2}}}  \SCsym{,}  \Psi_{{\mathrm{2}}}  \vdash_\mathcal{C}  \SCnt{Z}$ s.t. $c(\Pi')\leq|Y|$. Therefore, the
  proof $\Pi$ can be constructed as follows, and $c(\Pi)=c(\Pi')\leq|Y|$.
  \begin{center}
    \scriptsize
    \begin{math}
      $$\mprset{flushleft}
      \inferrule* [right={\tiny $\SCdruleTXXexName$}] {
        {
          \begin{array}{c}
            \Pi' \\
                 {\Psi_{{\mathrm{1}}}  \SCsym{,}  \Phi_{{\mathrm{1}}}  \SCsym{,}  \SCnt{X_{{\mathrm{1}}}}  \SCsym{,}  \SCnt{X_{{\mathrm{2}}}}  \SCsym{,}  \Phi_{{\mathrm{2}}}  \SCsym{,}  \Psi_{{\mathrm{2}}}  \vdash_\mathcal{C}  \SCnt{Z}}
          \end{array}
        }
      }{\Psi_{{\mathrm{1}}}  \SCsym{,}  \Phi_{{\mathrm{1}}}  \SCsym{,}  \SCnt{X_{{\mathrm{2}}}}  \SCsym{,}  \SCnt{X_{{\mathrm{1}}}}  \SCsym{,}  \Phi_{{\mathrm{2}}}  \SCsym{,}  \Psi_{{\mathrm{2}}}  \vdash_\mathcal{C}  \SCnt{Z}}
    \end{math}
  \end{center}
  The full proof can be found in Appendix~\ref{app:cut-reduction}.
\end{proof}
\noindent
Then we have the following lemma.

\begin{lemma}
  \label{lem:less-cut-rank}
  Let $\Pi$ be a proof of a sequent $\Phi  \vdash_\mathcal{C}  \SCnt{X}$ or $\Gamma  \vdash_\mathcal{L}  \SCnt{A}$ s.t.
  $c(\Pi)>0$. Then there is a proof $\Pi'$ of the same sequent with
  $c(\Pi')<c(\Pi)$.
\end{lemma}
\begin{proof}
  We prove the lemma by induction on $d(\Pi)$. We denote the proof $\Pi$ by 
  $\pi+r$, where $r$ is the last inference of $\Pi$ and $\pi$ denotes the
  rest of the proof. If $r$ is not a cut, then by induction hypothesis on
  $\pi$, there is a proof $\pi'$ s.t. $c(\pi')<c(\pi)$ and $\Pi'=\pi'+r$.
  Otherwise, we assume $r$ is a cut on a formula $Y$. If $c(\Pi)>|X|+1$,
  then there is a cut on $|Y|$ in $\pi$ with $|Y|>|X|$. So we can apply
  the induction hypothesis on $\pi$ to get $\Pi'$ with $c(\Pi')<c(\Pi)$. The
  last case to consider is when $c(\Pi)=|X|+1$ (note that $c(\Pi)$ cannot be
  less than $|X|+1$). In this case, $\Pi$ is in the form of
  \begin{center}
    \scriptsize
    \begin{math}
      $$\mprset{flushleft}
      \inferrule* [right={\tiny $\SCdruleTXXcutName$}] {
        {
          \begin{array}{cc}
            \Pi_1 & \Pi_2 \\
            {\Phi  \vdash_\mathcal{C}  \SCnt{X}} & {\Psi_{{\mathrm{1}}}  \SCsym{,}  \SCnt{X}  \SCsym{,}  \Psi_{{\mathrm{2}}}  \vdash_\mathcal{C}  \SCnt{Y}}
          \end{array}
        }
      }{\Psi_{{\mathrm{1}}}  \SCsym{,}  \Phi  \SCsym{,}  \Psi_{{\mathrm{2}}}  \vdash_\mathcal{C}  \SCnt{Y}}
    \end{math}
    \qquad\qquad
    or,
    \qquad\qquad
    \begin{math}
      $$\mprset{flushleft}
      \inferrule* [right={\tiny $\SCdruleSXXcutOneName$}] {
        {
          \begin{array}{cc}
            \Pi_1 & \Pi_2 \\
            {\Phi  \vdash_\mathcal{C}  \SCnt{X}} & {\Gamma_{{\mathrm{1}}}  \SCsym{;}  \SCnt{X}  \SCsym{;}  \Gamma_{{\mathrm{2}}}  \vdash_\mathcal{L}  \SCnt{A}}
          \end{array}
        }
      }{\Gamma_{{\mathrm{1}}}  \SCsym{;}  \Phi  \SCsym{;}  \Gamma_{{\mathrm{2}}}  \vdash_\mathcal{L}  \SCnt{A}}
    \end{math}
  \end{center}
  By assumption, $c(\Pi_1),c(\Pi_2)\leq |X|+1$. By induction, we can
  construct $c(\Pi_1')$ proving $\Phi  \vdash_\mathcal{C}  \SCnt{X}$ and $c(\Pi_2')$ proving
  $\Psi_{{\mathrm{1}}}  \SCsym{,}  \SCnt{X}  \SCsym{,}  \Psi_{{\mathrm{2}}}  \vdash_\mathcal{C}  \SCnt{Y}$ (or $<\Gamma_{{\mathrm{1}}}  \SCsym{;}  \SCnt{X}  \SCsym{;}  \Gamma_{{\mathrm{2}}}  \vdash_\mathcal{L}  \SCnt{A}$) with
  $c(\Pi_1'), c(\Pi_2')\leq |X|$. Then by Lemma~\ref{lem:cut-reduction}, we
  can construct $\Pi'$ proving $\Psi_{{\mathrm{1}}}  \SCsym{,}  \Phi  \SCsym{,}  \Psi_{{\mathrm{2}}}  \vdash_\mathcal{C}  \SCnt{Y}$ (or
  $\Gamma_{{\mathrm{1}}}  \SCsym{;}  \Phi  \SCsym{;}  \Gamma_{{\mathrm{2}}}  \vdash_\mathcal{L}  \SCnt{A}$) with $c(\Pi')\leq |X|$. \\
  The case where the last inference is a cut on a formula $A$ is similar as
  when it is a cut on $X$.
\end{proof}
\noindent
By induction on $c(\Pi)$ and Lemma~\ref{lem:less-cut-rank}, the cut
elimination theorem follows immediately.
\begin{theorem}[Cut Elimination]
  Let $\Pi$ be a proof of a sequent $\Phi  \vdash_\mathcal{C}  \SCnt{X}$ or $\Gamma  \vdash_\mathcal{L}  \SCnt{A}$ s.t.
  $c(\Pi)>0$. Then there is an algorithm which yields a cut-free proof
  $\Pi'$ of the same sequent.
\end{theorem}

% section sequent_calculus (end)

\section{An Adjoint Formalization of the Lambek Calculus}
\label{sec:the-lambek-calculus}
Similar as the sequent calculus, the term assignment for CNC logic is
also composed of two logics; intuitionistic linear logic on the left,
denoted by $\cat{C}$, and the Lambek calculus on the right, denoted by
$\cat{L}$. The syntax for types and contexts we use in the term
assignment is the same as in the sequent calculus. The rest of the
syntax for the term assignment is defined as follows.
\begin{definition}
  \label{def:Lambek-syntax}
  The following grammar describes the syntax of the term assignment of the
  CNC logic:
  \begin{center}\vspace{-3px}\small
    \begin{math}
      \begin{array}{lll}        
        \text{($\cat{C}$-Terms)} & \NDnt{t} ::= \NDmv{x} \mid  \mathsf{triv}  \mid \NDnt{t_{{\mathrm{1}}}}  \otimes  \NDnt{t_{{\mathrm{2}}}} \mid  \mathsf{let}\, \NDnt{t_{{\mathrm{1}}}}  :  \NDnt{X} \,\mathsf{be}\, \NDnt{q} \,\mathsf{in}\, \NDnt{t_{{\mathrm{2}}}}  \mid  \lambda  \NDmv{x}  :  \NDnt{X} . \NDnt{t}  \mid  \NDnt{t_{{\mathrm{1}}}}   \NDnt{t_{{\mathrm{2}}}}  \mid  \mathsf{ex}\, \NDnt{t_{{\mathrm{1}}}} , \NDnt{t_{{\mathrm{2}}}} \,\mathsf{with}\, \NDmv{x_{{\mathrm{1}}}} , \NDmv{x_{{\mathrm{2}}}} \,\mathsf{in}\, \NDnt{t_{{\mathrm{3}}}}  \mid  \mathsf{G}\, \NDnt{s} \\
        \text{($\cat{L}$-Terms)} & \NDnt{s} ::= \NDmv{x} \mid  \mathsf{triv}  \mid \NDnt{s_{{\mathrm{1}}}}  \triangleright  \NDnt{s_{{\mathrm{2}}}} \mid  \mathsf{let}\, \NDnt{s_{{\mathrm{1}}}}  :  \NDnt{A} \,\mathsf{be}\, \NDnt{p} \,\mathsf{in}\, \NDnt{s_{{\mathrm{2}}}}  \mid  \mathsf{let}\, \NDnt{t}  :  \NDnt{X} \,\mathsf{be}\, \NDnt{q} \,\mathsf{in}\, \NDnt{s}  \mid  \lambda_l  \NDmv{x}  :  \NDnt{A} . \NDnt{s}  \mid  \lambda_r  \NDmv{x}  :  \NDnt{A} . \NDnt{s}  \\
        & \,\,\,\,\,\,\,\,\,\mid  \mathsf{app}_l\, \NDnt{s_{{\mathrm{1}}}} \, \NDnt{s_{{\mathrm{2}}}}  \mid  \mathsf{app}_r\, \NDnt{s_{{\mathrm{1}}}} \, \NDnt{s_{{\mathrm{2}}}}  \mid  \mathsf{F} \NDnt{t} \\        
        \text{($\cat{C}$-Patterns)} & \NDnt{q} ::=  \mathsf{triv}  \mid \NDmv{x} \mid \NDnt{q_{{\mathrm{1}}}}  \otimes  \NDnt{q_{{\mathrm{2}}}} \mid  \mathsf{G}\, \NDnt{p} \\
        \text{($\cat{L}$-Patterns)} & \NDnt{p} ::=  \mathsf{triv}  \mid \NDmv{x} \mid \NDnt{p_{{\mathrm{1}}}}  \triangleright  \NDnt{p_{{\mathrm{2}}}} \mid  \mathsf{F}\, \NDnt{q} \\        
        \text{($\cat{C}$-Typing Judgment)} & \Phi  \vdash_\mathcal{C}  \NDnt{t}  \NDsym{:}  \NDnt{X}\\
        \text{($\cat{L}$-Typing Judgment)} & \Gamma  \vdash_\mathcal{L}  \NDnt{s}  \NDsym{:}  \NDnt{A}\\
      \end{array}
    \end{math}
  \end{center}
\end{definition}

Now $\cat{C}$-typing judgments are denoted by $\Psi  \vdash_\mathcal{C}  \NDnt{t}  \NDsym{:}  \NDnt{X}$ where
$\Psi$ is a sequence of pairs of variables and their types, denoted by
$\NDmv{x}  \NDsym{:}  \NDnt{X}$, $\NDnt{t}$ is a $\cat{C}$-term, and $\NDnt{X}$ is a $\cat{C}$-type.  
The $\cat{C}$-terms are all standard, but $ \mathsf{G}\, \NDnt{s} $ corresponds to the
morphism part of the right-adjoint of the adjunction between both logics,
and $ \mathsf{ex}\, \NDnt{t_{{\mathrm{1}}}} , \NDnt{t_{{\mathrm{2}}}} \,\mathsf{with}\, \NDmv{x_{{\mathrm{1}}}} , \NDmv{x_{{\mathrm{2}}}} \,\mathsf{in}\, \NDnt{t_{{\mathrm{3}}}} $ is the introduction form for the
structural rule exchange.

The $\cat{L}$-typing judgment has the form $\Gamma  \vdash_\mathcal{L}  \NDnt{s}  \NDsym{:}  \NDnt{A}$ where $\Gamma$
is now a $\cat{L}$-context, denoted by $\Gamma$ or $\Delta$. These contexts
are ordered sequences of pairs of free variables with their types from
\emph{both} sides denoted by $\NDmv{x}  \NDsym{:}  \NDnt{B}$ and $\NDmv{x}  \NDsym{:}  \NDnt{X}$ respectively.
Finally, the term $\NDnt{s}$ is a $\cat{L}$-term, and $\NDnt{A}$ is a
$\cat{L}$-type.  Given two typing contexts $\Gamma$ and $\Delta$ we denote
their concatenation by $\Gamma  \NDsym{;}  \Delta$; we use a semicolon here to emphasize the
fact that the contexts are ordered. $\cat{L}$-terms correspond to
introduction and elimination forms for each of the previous types. For
example, $\NDnt{s_{{\mathrm{1}}}}  \triangleright  \NDnt{s_{{\mathrm{2}}}}$ introduces a tensor, and
$ \mathsf{let}\, \NDnt{s_{{\mathrm{1}}}}  :  \NDnt{A}  \triangleright  \NDnt{B} \,\mathsf{be}\, \NDmv{x}  \triangleright  \NDmv{y} \,\mathsf{in}\, \NDnt{s_{{\mathrm{2}}}} $ eliminates a tensor.

The typing rules for CNC logic can be found in
Figure~\ref{fig:CNC-typing-rules}.
\begin{figure}
  \footnotesize
  \begin{tabular}{|c|}
    \hline\\
      \begin{mathpar}
      \NDdruleTXXid{} \and
      \NDdruleTXXunitI{} \and
      \NDdruleTXXunitE{} \and
      \NDdruleTXXtenI{} \and
      \NDdruleTXXtenE{} \and
      \NDdruleTXXimpI{} \and
      \NDdruleTXXimpE{} \and
      \NDdruleTXXGI{} \and
      \NDdruleTXXbeta{} \and
      \NDdruleTXXcut{}      
      \end{mathpar}
      \\
      \\
      \hline
      \\[5px]
    \begin{mathpar}
      \NDdruleSXXid{} \and
      \NDdruleSXXunitI{} \and
      \NDdruleSXXunitETwo{} \and
      \NDdruleSXXunitEOne{} \and
      \NDdruleSXXtenI{} \and
      \NDdruleSXXtenETwo{} \and
      \NDdruleSXXtenEOne{} \and
      \NDdruleSXXimprI{} \and
      \NDdruleSXXimprE{} \and
      \NDdruleSXXimplI{} \and
      \NDdruleSXXimplE{} \and
      \NDdruleSXXFI{} \and
      \NDdruleSXXFE{} \and
      \NDdruleSXXGE{} \and
      \NDdruleSXXbeta{} \and
      \NDdruleSXXcutTwo{} \and
      \NDdruleSXXcutOne{}
    \end{mathpar}\\\\
    \hline
  \end{tabular}  
  \caption{Typing Rules for CNC Logic}
  \label{fig:CNC-typing-rules}
\end{figure}
We split the figure in two: the top of the figure are the rules of
intuitionistic linear logic whose judgment is the $\mathcal{C}$-typing
judgment denoted by $\Psi  \vdash_\mathcal{C}  \NDnt{t}  \NDsym{:}  \NDnt{X}$, and the bottom of the figure
are the rules for the mixed commutative/non-commutative Lambek
calculus whose judgment is the $\mathcal{L}$-judgment denoted by
$\Gamma  \vdash_\mathcal{L}  \NDnt{s}  \NDsym{:}  \NDnt{A}$, and the two halves are connected via the rules 
rules $\NDdruleTXXGIName{}$, $\NDdruleSXXGEName{}$,
$\NDdruleSXXFIName{}$, and $\NDdruleSXXFEName{}$,
$\NDdruleSXXunitEOneName{}$, $\NDdruleSXXtenEOneName{}$, and
$\NDdruleSXXcutOneName{}$.

The one step $\beta$-reduction rules are listed in
Figure~\ref{fig:CNC-beta-reductions}. Similarly to the typing rules,
the figure is split in two: the top lists the rules of the
intuitionistic linear logic, and the bottom are those of the mixed
commutative/non-commutative Lambek calculus. 
\renewcommand{\NDdruleTbetaXXletUName}{}
\renewcommand{\NDdruleTbetaXXletTName}{}
\renewcommand{\NDdruleTbetaXXlamName}{}
\renewcommand{\NDdruleTbetaXXappOneName}{}
\renewcommand{\NDdruleTbetaXXappTwoName}{}
\renewcommand{\NDdruleTbetaXXappLetName}{}
\renewcommand{\NDdruleTbetaXXletLetName}{}
\renewcommand{\NDdruleTbetaXXletAppName}{}
\renewcommand{\NDdruleSbetaXXletUOneName}{}
\renewcommand{\NDdruleSbetaXXletTOneName}{}
\renewcommand{\NDdruleSbetaXXletTTwoName}{}
\renewcommand{\NDdruleSbetaXXletFName}{}
\renewcommand{\NDdruleSbetaXXlamLName}{}
\renewcommand{\NDdruleSbetaXXlamRName}{}
\renewcommand{\NDdruleSbetaXXapplOneName}{}
\renewcommand{\NDdruleSbetaXXapplTwoName}{}
\renewcommand{\NDdruleSbetaXXapprOneName}{}
\renewcommand{\NDdruleSbetaXXapprTwoName}{}
\renewcommand{\NDdruleSbetaXXderelictName}{}
\renewcommand{\NDdruleSbetaXXapplLetName}{}
\renewcommand{\NDdruleSbetaXXapprLetName}{}
\renewcommand{\NDdruleSbetaXXletLetName}{}
\renewcommand{\NDdruleSbetaXXletApplName}{}
\renewcommand{\NDdruleSbetaXXletApprName}{}
\renewcommand{\NDdruleTcomXXunitEXXunitEName}{}
\renewcommand{\NDdruleTcomXXunitEXXtenEName}{}
\renewcommand{\NDdruleTcomXXunitEXXimpEName}{}
\renewcommand{\NDdruleTcomXXtenEXXunitEName}{}
\renewcommand{\NDdruleTcomXXtenEXXtenEName}{}
\renewcommand{\NDdruleTcomXXtenEXXimpEName}{}
\renewcommand{\NDdruleTcomXXimpEXXunitEName}{}
\renewcommand{\NDdruleScomXXunitEXXunitEName}{}
\renewcommand{\NDdruleScomXXunitETwoXXunitEName}{}
\renewcommand{\NDdruleScomXXunitEXXimprEName}{}
\renewcommand{\NDdruleScomXXunitETwoXXimprEName}{}
\renewcommand{\NDdruleScomXXunitEXXFEName}{}
\renewcommand{\NDdruleScomXXunitETwoXXFEName}{}
\renewcommand{\NDdruleScomXXtenEXXunitEName}{}
\renewcommand{\NDdruleScomXXtenETwoXXunitEName}{}
\renewcommand{\NDdruleScomXXtenEXXtenEName}{}
\renewcommand{\NDdruleScomXXtenETwoXXtenEName}{}
\renewcommand{\NDdruleScomXXtenEXXimprEName}{}
\renewcommand{\NDdruleScomXXtenETwoXXimprEName}{}
\renewcommand{\NDdruleScomXXtenEXXimplEName}{}
\renewcommand{\NDdruleScomXXtenETwoXXimplEName}{}
\renewcommand{\NDdruleScomXXtenEXXFEName}{}
\renewcommand{\NDdruleScomXXtenETwoXXFEName}{}
\renewcommand{\NDdruleScomXXFEXXunitEName}{}
\renewcommand{\NDdruleScomXXFEXXtenEName}{}
\renewcommand{\NDdruleScomXXFEXXimprEName}{}
\renewcommand{\NDdruleScomXXFEXXimplEName}{}
\renewcommand{\NDdruleScomXXFEXXFEName}{}
\begin{figure}[!h]
  \footnotesize
  \begin{tabular}{|c|}
    \hline\\
      \begin{mathpar}
      \NDdruleTbetaXXletU{} \and
      \NDdruleTbetaXXletT{} \and
      \NDdruleTbetaXXlam{}
      \end{mathpar}
      \\
      \\
      \hline
      \\
    \begin{mathpar}
      \NDdruleSbetaXXletUOne{} \and
      \NDdruleSbetaXXletTOne{} \and
      \NDdruleSbetaXXletTTwo{} \and
      \NDdruleSbetaXXletF{} \and
      \NDdruleSbetaXXlamL{} \and
      \NDdruleSbetaXXlamR{} \and
      \NDdruleSbetaXXderelict{}
    \end{mathpar}\\\\
    \hline
  \end{tabular}  
  \caption{$\beta$-reductions for CNC Logic}
  \label{fig:CNC-beta-reductions}
\end{figure}


The commuting conversions can be found in
Figures~\ref{fig:CNC-commutating-conversions-intuitionistic}-\ref{fig:CNC-commutating-conversions-both}. We
divide the rules into three parts due to the length. The first part,
Figure~\ref{fig:CNC-commutating-conversions-intuitionistic}, includes
the rules for the intuitionistic linear logic. The second,
Figure~\ref{fig:CNC-commutating-conversions-mixed}, includes the rules
for the commutative/non-commutative Lambek calculus. The third,
Figure~\ref{fig:CNC-commutating-conversions-both}, includes the mixed
rules $\NDdruleSXXunitEOneName{}$ and $\NDdruleSXXtenEOneName{}$.

\begin{figure}[!h]
  \footnotesize
  \begin{tabular}{|c|}
    \hline\\
    \begin{mathpar}
      \NDdruleTcomXXunitEXXunitE{} \and
      \NDdruleTcomXXunitEXXtenE{} \and
      \NDdruleTcomXXunitEXXimpE{} \and
      \NDdruleTcomXXtenEXXunitE{} \and
      \NDdruleTcomXXtenEXXtenE{} \and
      \NDdruleTcomXXtenEXXimpE{} \and
      \NDdruleTcomXXimpEXXunitE{}
    \end{mathpar}
    \\
    \\
    \hline
  \end{tabular}  
  \caption{Commuting Conversions: Intuitionistic Linear Logic}
  \label{fig:CNC-commutating-conversions-intuitionistic}
\end{figure}
\begin{figure}[!h]
  \footnotesize
  \begin{tabular}{|c|}
    \hline\\
    \begin{mathpar}
      \NDdruleScomXXunitEXXunitE{} \and
      \NDdruleScomXXunitEXXimprE{} \and
      \NDdruleScomXXunitEXXFE{} \and
      \NDdruleScomXXtenEXXunitE{} \and
      \NDdruleScomXXtenEXXtenE{} \and
      \NDdruleScomXXtenEXXimprE{} \and
      \NDdruleScomXXtenEXXimplE{} \and
      \NDdruleScomXXtenEXXFE{} \and
      \NDdruleScomXXFEXXunitE{} \and
      \NDdruleScomXXFEXXtenE{} \and
      \NDdruleScomXXFEXXimprE{} \and
      \NDdruleScomXXFEXXimplE{} \and
      \NDdruleScomXXFEXXFE{}
    \end{mathpar}\\\\
    \hline
  \end{tabular}  
  \caption{Commuting Conversions: Commutative/Non-commutative Lambek Calculus}
  \label{fig:CNC-commutating-conversions-mixed}
\end{figure}
\begin{figure}[!h]
  \footnotesize
  \begin{tabular}{|c|}
    \hline\\
    \begin{mathpar}
      \NDdruleScomXXunitETwoXXunitE{} \and
      \NDdruleScomXXunitETwoXXimprE{} \and
      \NDdruleScomXXunitETwoXXFE{} \and
      \NDdruleScomXXtenETwoXXunitE{} \and
      \NDdruleScomXXtenETwoXXtenE{} \and
      \NDdruleScomXXtenETwoXXimprE{} \and
      \NDdruleScomXXtenETwoXXimplE{} \and
      \NDdruleScomXXtenETwoXXFE{}
    \end{mathpar}\\\\
    \hline
  \end{tabular}  
  \caption{Commuting Conversions: Mixed Rules}
  \label{fig:CNC-commutating-conversions-both}
\end{figure}

We also proved that the sequent calculus formalization given in
Figure~\ref{fig:CNC-sequent-calculus} is equivalent to the typing rules (or
else called the natural deduction formalization) given in
Figure~\ref{fig:CNC-typing-rules} are equivalent, as stated in the following
theorem.
\begin{theorem}
  \label{thm:sc-nd-equiv}
  The sequent calculus ($\mathit{SC}$) and natural deduction ($\mathit{ND}$)
  formalizations for CNC logic are equivalent in the sense that there are
  two mappings $N:\mathit{SC}\rightarrow\mathit{ND}$ and
  $S:\mathit{ND}\rightarrow\mathit{SC}$ that map each rule in $\mathit{SC}$
  to a proof in $\mathit{ND}$, and each rule in $\mathit{ND}$ to a proof
  in $\mathit{SC}$, respectively.
\end{theorem}
\begin{proof}
  The proof is done case by case on each rule in the sequence calculus and
  natural deduction formalizations. It is obvious that the axioms in one
  formalization can be mapped to the axioms in the other. The introduction
  rules in $\mathit{ND}$ are mapped to the right rules in $\mathit{SC}$, and
  vice versa. The elimination rules and lefts rules are mapped to each other
  with some fiddling. For instance, the elimination rule for the
  non-commutative tensor is mapped to the following proof in $\mathit{SC}$:
  \begin{center}
    \scriptsize
    \begin{math}
      $$\mprset{flushleft}
      \inferrule* [right={\scriptsize $\ElledruleTXXcutName$}] {
        {\Phi  \vdash_\mathcal{C}  \NDnt{t_{{\mathrm{1}}}}  \NDsym{:}  \NDnt{X}  \otimes  \NDnt{Y}} \\
        $$\mprset{flushleft}
        \inferrule* [right={\scriptsize $\ElledruleTXXtenLName$}] {
          {\Psi_{{\mathrm{1}}}  \NDsym{,}  \NDmv{x}  \NDsym{:}  \NDnt{X}  \NDsym{,}  \NDmv{y}  \NDsym{:}  \NDnt{Y}  \NDsym{,}  \Psi_{{\mathrm{2}}}  \vdash_\mathcal{C}  \NDnt{t_{{\mathrm{2}}}}  \NDsym{:}  \NDnt{Z}}
        }{\Psi_{{\mathrm{1}}}  \NDsym{,}  \NDmv{z}  \NDsym{:}  \NDnt{X}  \otimes  \NDnt{Y}  \NDsym{,}  \Psi_{{\mathrm{2}}}  \vdash_\mathcal{C}   \mathsf{let}\, \NDmv{z}  :  \NDnt{X}  \otimes  \NDnt{Y} \,\mathsf{be}\, \NDmv{x}  \otimes  \NDmv{y} \,\mathsf{in}\, \NDnt{t_{{\mathrm{2}}}}   \NDsym{:}  \NDnt{Z}}
      }{\Psi_{{\mathrm{1}}}  \NDsym{,}  \Phi  \NDsym{,}  \Psi_{{\mathrm{2}}}  \vdash_\mathcal{C}  \NDsym{[}  \NDnt{t_{{\mathrm{1}}}}  \NDsym{/}  \NDmv{z}  \NDsym{]}  \NDsym{(}   \mathsf{let}\, \NDmv{z}  :  \NDnt{X}  \otimes  \NDnt{Y} \,\mathsf{be}\, \NDmv{x}  \otimes  \NDmv{y} \,\mathsf{in}\, \NDnt{t_{{\mathrm{2}}}}   \NDsym{)}  \NDsym{:}  \NDnt{Z}}
    \end{math}
  \end{center}
  The full proof is in Appendix~\ref{app:sc-nd-equiv}.
\end{proof}

%% \noindent
%% Finally, we have strong normalization of CNC logic.
%% 
%% \begin{theorem}[Strong Normalization]
%%   \label{thm:strong-normalization}
%%   CNC logic is strongly normalizing.
%% \end{theorem}
%% \begin{proof}
%%   The proof is via a translation to LNL logic, which has been proved to be
%%   strongly normalizing \cite{Mellies:2009}. The proof is in
%%   Appendix~\ref{app:strong-normalization}. We first define a function
%%   $CL$ that maps our term assignment for natural deduction to Benton's
%%   term assignment, and then shows that if there is reduction $\NDnt{t_{{\mathrm{1}}}}  \leadsto_\beta  \NDnt{t_{{\mathrm{2}}}}$ (resp. $\NDnt{s_{{\mathrm{1}}}}  \leadsto_\beta  \NDnt{s_{{\mathrm{2}}}}$) in CNC, then there is a reduction
%%   $CL(t_1)\leadsto CL(t_2)$ (resp. $CL(s_1)\leadsto CL(s_2)$) in
%%   LNL. Recall that a LNL model consists of an adjunction
%%   $F:\cat{C}\dashv\cat{L}:G$ in which $\cat{C}$ is a Cartesian closed
%%   category and $\cat{L}$ is a SMCC.  The basic idea of mapping a LAM
%%   model to a LNL model is mapping the SMCC in LAM to the Cartesian
%%   closed category in LNL, and the Lambek category in LAM to the SMCC in
%%   LNL. For instance, $s_1\triangleright s_2$ in CNC, where $\NDnt{s_{{\mathrm{1}}}}$ and
%%   $\NDnt{s_{{\mathrm{2}}}}$ are atomic terms, maps to $CL(s_1)\otimes CL(s_2)$ in LNL.
%% \end{proof}

%%% Local Variables: 
%%% mode: latex
%%% TeX-master: main.tex
%%% End:

% section the (end)

\section{Adjoint Model}
\label{sec:adjoint-model}
We first introduce adjoint models, and the two particular adjoint
models that will be used to model the Lambek Calculus with and without
associativity.  However, before we can do this we have to define a
very simple categorical structure known as a magmoidal category with a
unit.
%% \begin{definition}
%%   \label{def:magmoidal-categories}
%%   1
%% \end{definition}


% section adjoint_models (end)

\section{A Model in Dialectica Spaces}
\label{sec:a-model-in-dialectica-spaces}
\newcommand{\Set}{\mathsf{Set}}
\newcommand{\Dial}[2]{\mathsf{Dial}_{#1}(#2)}

In this section we give a different categorical model in terms of
dialectica categories; which are a sound and complete categorical
model of the Lambek Calculus as was shown by de Paiva and Eades
\cite{dePaiva2018}. This section is largely the same as the
corresponding section de Paiva and Eades give, but with some
modifications to their definition of biclosed posets with exchange
(see Definition~\ref{def:biclosed-exchange}).  However, we try to make
this section as self contained as possible.

Dialectica categories were first introduced by de Paiva as a
categorification of G\"odel's Dialectica interpretation
\cite{depaiva1990}.  Dialectica categories were one of the first sound
categorical models of intuitionistic linear logic with linear
modalities.  We show in this section that they can be adapted to
become a sound and complete model for CNC logic, with both the
exchange and of-course modalities.  Due to the complexities of working
with dialectica categories we have formally verified\footnote{The
  complete formalization can be found online at
  \url{https://github.com/MonoidalAttackTrees/non-comm-monads-adjoint-models/tree/master/dialectica-formalization}.}
this section in the proof assistant Agda~\cite{bove2009}.

First, we define the notion of a biclosed poset.  These are used to
control the definition of morphisms in the dialectica model.
\begin{definition}
  \label{def:biclosed-poset}
  Suppose $(M, \leq, \circ, e)$ is an ordered non-commutative monoid.
  If there exists a largest $x \in M$ such that $a \circ x \leq b$ for
  any $a, b \in M$, then we denote $x$ by $a \lto b$ and called it
  the \textbf{left-pseudocomplement} of $a$ w.r.t $b$.  Additionally,
  if there exists a largest $x \in M$ such that $x \circ a \leq b$ for
  any $a, b \in M$, then we denote $x$ by $b \rto a$ and called it
  the \textbf{right-pseudocomplement} of $a$ w.r.t $b$.

  A \textbf{biclosed poset}, $(M, \leq, \circ, e, \lto, \rto)$, is an
  ordered non-commutative monoid, $(M, \leq, \circ, e)$, such that $a
  \lto b$ and $b \rto a$ exist for any $a,b \in M$.
\end{definition}
Now using the previous definition we define dialectica Lambek spaces.
\begin{definition}
  \label{def:dialectica-lambek-spaces}
  Suppose $(M, \leq, \circ, e, \lto, \rto)$ is a biclosed poset. Then
  we define the category of \textbf{dialectica Lambek spaces},
  $\mathsf{Dial}_M(\Set)$, as follows:
  \begin{itemize}
  \item[-] objects, or dialectica Lambek spaces, are triples $(U, X,
    \alpha)$ where $U$ and $X$ are sets, and $\alpha : U \times X \mto
    M$ is a generalized relation over $M$, and

  \item[-] maps that are pairs $(f, F) : (U , X, \alpha) \mto (V , Y ,
    \beta)$ where $f : U \mto V$, and $F : Y \mto X$ are functions
    such that the weak adjointness condition
    $\forall u \in U.\forall y \in Y. \alpha(u , F(y)) \leq \beta(f(u), y)$
    holds.
  \end{itemize}
\end{definition}
Notice that the biclosed poset is used here as the target of the
relations in objects, but also as providing the order  relation in the weak adjoint condition on morphisms.  This will allow the structure of the biclosed
poset to lift up into $\Dial{M}{\Set}$.

We will show that $\Dial{M}{\Set}$ is a model of the Lambek Calculus
with modalities.  First, we must show that $\Dial{M}{\Set}$ is
monoidal biclosed.
\begin{definition}
  \label{def:dial-monoidal-structure}
  Suppose $(U, X, \alpha)$ and $(V, Y, \beta)$ are two objects of
  $\Dial{M}{\Set}$. Then their tensor product is defined as follows:
  \[ \small
  (U, X, \alpha) \rhd (V, Y, \beta) = (U \times V, (V \to X) \times (U \to Y), \alpha \rhd \beta)
  \]
  where $- \to -$ is the function space from $\Set$, and $(\alpha
  \rhd \beta)((u, v), (f, g)) = \alpha(u, f(v)) \circ \beta(g(u), v)$.

  \ \\
  \noindent
  The unit of the above tensor product is defined as follows:
  \[ \small
  I = (\top , \top , \iota)
  \]
  where $\top$ is the initial object in $\Set$, and $\iota(*,*) = e$.
\end{definition}

\noindent
It follows from de Paiva and Eades \cite{dePaiva2018} that this does
indeed define a monoidal tensor product, but take note of the fact
that this tensor product is indeed non-commutative, because the
non-commutative multiplication of the biclosed poset is used to define
the relation of the tensor product.

The tensor product has two right adjoints making $\Dial{M}{\Set}$
biclosed.
\begin{definition}
  \label{def:dial-is-biclosed}
  Suppose $(U, X, \alpha)$ and $(V, Y, \beta)$ are two objects of
  $\Dial{M}{\Set}$. Then two internal-homs can be defined as follows:
  \[ \small
  \begin{array}{lll}
    (U, X, \alpha) \lto (V, Y, \beta) = ((U \to V) \times (Y \to X), U \times Y, \alpha \lto \beta)\\
    (V, Y, \beta) \rto (U, X, \alpha) = ((U \to V) \times (Y \to X), U \times Y, \alpha \rto \beta)\\
  \end{array}
  \]
\end{definition}
\noindent
It is straightforward to show that the typical bijections defining the
corresponding adjunctions hold; see de Paiva and Eades for the details
\cite{dePaiva2018}.

We now extend $\Dial{M}{\Set}$ with two modalities: the usual
modality, of-course, denoted $!A$, and the exchange modality denoted
$\xi A$.  However, we must first extended biclosed posets to
include an exchange operation.
\begin{definition}
  \label{def:biclosed-exchange}
  A \textbf{biclosed poset with exchange} is a biclosed poset $(M,
  \leq, \circ, e, \lto, \rto)$ equipped with an unary operation
  $\xi : M \to M$ satisfying the following:
  \[ \small
  \setlength{\arraycolsep}{4px}
  \begin{array}{lll}
    \begin{array}{lll}
    \text{(Compatibility)} & a \leq b \text{ implies } \xi a \leq \xi b \text{ for all } a,b,c \in M\\
    \text{(Minimality)} & \xi a \leq a \text{ for all } a \in M\\    
  \end{array}
  &
  \begin{array}{lll}
    \text{(Duplication)} & \xi a \leq \xi\xi a \text{ for all } a \in M\\
    \text{(Exchange)} & (\xi a \circ \xi b) \leq (\xi b \circ \xi a) \text{ for all } a, b \in M\\
  \end{array}
  \end{array}
  \]
\end{definition}
\noindent
This definition is where the construction given here departs from the
definition of biclosed posets with exchange given by de Paiva and
Eades \cite{dePaiva2018}.

We can now define the two modalities in $\Dial{M}{\Set}$ where $M$ is
a biclosed poset with exchange.
\begin{definition}
  \label{def:modalities-dial}
  Suppose $(U, X, \alpha)$ is an object of $\Dial{M}{\Set}$ where $M$
  is a biclosed poset with exchange. Then the \textbf{of-course} and
  \textbf{exchange} modalities can be defined as 
  $! (U, X, \alpha) = (U, U \to X^*, !\alpha)$ and
  $\xi (U, X, \alpha) = (U, X, \xi \alpha)$
  where $X^*$ is the free commutative monoid on $X$, $(!\alpha)(u, f)
  = \alpha(u, x_1) \circ \cdots \circ \alpha(u, x_i)$ for $f(u) =
  (x_1, \ldots, x_i)$, and $(\xi \alpha)(u, x) = \xi (\alpha(u,
  x))$.
\end{definition}
This definition highlights a fundamental difference between the two
modalities.  The definition of the exchange modality relies on an
extension of biclosed posets with essentially the exchange modality in
the category of posets.  However, the of-course modality is defined by
the structure already present in $\Dial{M}{\Set}$, specifically, the
structure of $\Set$.

Both of the modalities have the structure of a comonad.  That is,
there are monoidal natural transformations $\varepsilon_! : !A \mto
A$, $\varepsilon_\xi : \xi A \mto A$, $\delta_! : !A \mto !!A$,
and $\delta_\xi : \xi A \mto \xi\xi A$ which satisfy the
appropriate diagrams; see the formalization for the full
proofs. Furthermore, these comonads come equipped with arrows $w : !A
\mto I$, $d : !A \mto !A \otimes !A$, $\e{A,B} : \xi A \otimes \xi B \mto \xi B
\otimes \xi A$. 

Finally, using the fact that $\Dial{M}{\Set}$ for any biclosed poset
is essentially a non-commutative formalization of Bierman's linear
categories \cite{Bierman:1994} we can use Benton's construction of an
LNL model from a linear category to obtain a LAM model, and hence,
obtain the following.
\begin{theorem}
  \label{theorem:sound-dial-exchange-!}
  Suppose $M$ is a biclosed poset with exchange.  Then
  $\Dial{M}{\Set}$ is a sound model for CNC logic.
\end{theorem}



% section a_model_in_dialectica_spaces (end)

\section{Future Work}
\label{sec:future-work}
We introduce the idea above of having a modality for exchange, but
what about individual modalities for weakening and contraction?
Indeed it is possible to give modalities for these structural rules as
well using adjoint models.  Now that we have each structural rule
isolated into their own modality is it possible to put them together
to form new modalities that combine structural rules?  The answer to
this question has already been shown to be positive, at least for
weakening and contraction, by Melli{\'e}s~\cite{Mellies:2004}, but we
plan to extend this line of work to include exchange.

The monads induced by the adjunction in CNC logic is non-commutative,
but Benton and Wadler show that the monads induced by the adjunction
in LNL logic \cite{Benton:1996} are commutative.  Using the extension
of Melli{\'e}s' work we mention above would allow us to combine both
CNC logic with LNL logic, and then be able to support both commutative
monads as well as non-commutative monads.  We plan on exploring this
in the future.

% section future_work (end)



\bibliographystyle{plainurl}
\bibliography{ref}



\appendix
\label{sec:appendix}
\section{Proof For Lemma~\ref{lem:cut-reduction}}
\label{app:cut-reduction}


\subsection{Commuting Conversion Cut vs. Cut}

\subsubsection{$\SCdruleTXXcutName$ vs. $\SCdruleTXXcutName$}
\begin{itemize}
% C-Cut vs. C-Cut Case 1
\item Case 1:
      \begin{center}
        \scriptsize
        \begin{math}
          \begin{array}{c}
            \Pi_1 \\
            {\Phi  \vdash_\mathcal{C}  \SCnt{X}}
          \end{array}
        \end{math}
        \qquad\qquad
        $\Pi_2:$
        \begin{math}
          $$\mprset{flushleft}
          \inferrule* [right={\tiny cut}] {
            {
              \begin{array}{cc}
                \pi_1 & \pi_2 \\
                {\Psi_{{\mathrm{2}}}  \SCsym{,}  \SCnt{X}  \SCsym{,}  \Psi_{{\mathrm{3}}}  \vdash_\mathcal{C}  \SCnt{Y}} & {\Psi_{{\mathrm{1}}}  \SCsym{,}  \SCnt{Y}  \SCsym{,}  \Psi_{{\mathrm{4}}}  \vdash_\mathcal{C}  \SCnt{Z}}
              \end{array}
            }
          }{\Psi_{{\mathrm{1}}}  \SCsym{,}  \Psi_{{\mathrm{2}}}  \SCsym{,}  \SCnt{X}  \SCsym{,}  \Psi_{{\mathrm{3}}}  \SCsym{,}  \Psi_{{\mathrm{4}}}  \vdash_\mathcal{C}  \SCnt{Z}}
        \end{math}
      \end{center}
      By assumption, $c(\Pi_1),c(\Pi_2)\leq |X|$. Therefore, $c(\pi_1)$,
      $c(\pi_2)\leq |X|$. Since $Y$ is the cut formula on $\pi_1$ and
      $\pi_2$, we have $|Y|+1\leq|X|$. By induction on $\Pi_1$ and $\pi_1$
      there exists a proof $\Pi'$ for sequent $\Psi_{{\mathrm{2}}}  \SCsym{,}  \Phi  \SCsym{,}  \Psi_{{\mathrm{3}}}  \vdash_\mathcal{C}  \SCnt{Y}$ s.t.
      $c(\Pi')\leq|X|$. So $\Pi$ can be constructed as follows, with
      $c(\Pi)\leq max\{c(\Pi'),c(\pi_2),|Y|+1\}\leq |X|$.
      \begin{center}
        \scriptsize
        \begin{math}
          $$\mprset{flushleft}
          \inferrule* [right={\tiny cut}] {
            {
              \begin{array}{cc}
                \Pi' & \pi_2 \\
                {\Psi_{{\mathrm{2}}}  \SCsym{,}  \Phi  \SCsym{,}  \Psi_{{\mathrm{3}}}  \vdash_\mathcal{C}  \SCnt{Y}} & {\Psi_{{\mathrm{1}}}  \SCsym{,}  \SCnt{Y}  \SCsym{,}  \Psi_{{\mathrm{4}}}  \vdash_\mathcal{C}  \SCnt{Z}}
              \end{array}
            }
          }{\Psi_{{\mathrm{1}}}  \SCsym{,}  \Psi_{{\mathrm{2}}}  \SCsym{,}  \Phi  \SCsym{,}  \Psi_{{\mathrm{3}}}  \SCsym{,}  \Psi_{{\mathrm{4}}}  \vdash_\mathcal{C}  \SCnt{Z}}
        \end{math}
      \end{center}

% C-Cut vs. C-Cut Case 2
\item Case 2:
      \begin{center}
        \scriptsize
        $\Pi_1$:
        \begin{math}
          $$\mprset{flushleft}
          \inferrule* [right={\tiny cut}] {
            {
              \begin{array}{cc}
                \pi_1 & \pi_2 \\
                {\Phi  \vdash_\mathcal{C}  \SCnt{X}} & {\Psi_{{\mathrm{2}}}  \SCsym{,}  \SCnt{X}  \SCsym{,}  \Psi_{{\mathrm{3}}}  \vdash_\mathcal{C}  \SCnt{Y}}
              \end{array}
            }
          }{\Psi_{{\mathrm{2}}}  \SCsym{,}  \Phi  \SCsym{,}  \Psi_{{\mathrm{3}}}  \vdash_\mathcal{C}  \SCnt{Y}}
        \end{math}
        \qquad\qquad
        \begin{math}
          \begin{array}{c}
            \Pi_2 \\
            {\Psi_{{\mathrm{1}}}  \SCsym{,}  \SCnt{Y}  \SCsym{,}  \Psi_{{\mathrm{4}}}  \vdash_\mathcal{C}  \SCnt{Z}}
          \end{array}
        \end{math}
      \end{center}
      By assumption, $c(\Pi_1),c(\Pi_2)\leq |Y|$. Since the cut rank of the last cut in
      $\Pi_1$ is $|X|+1$, then $|X|+1\leq |Y|$. By induction on $\Pi_1$ and $\Pi_2$, there is
      a proof $\Pi'$ for sequent $\Psi_{{\mathrm{1}}}  \SCsym{,}  \Psi_{{\mathrm{2}}}  \SCsym{,}  \SCnt{X}  \SCsym{,}  \Psi_{{\mathrm{3}}}  \SCsym{,}  \Psi_{{\mathrm{4}}}  \vdash_\mathcal{C}  \SCnt{Z}$ s.t. $c(\Pi')\leq|Y|$.
      Therefore, the proof $\Pi$ can be constructed as follows, and
      $c(\Pi)\leq max\{c(\pi_1),c(\Pi'),|X|+1\}\leq |Y|$.
      \begin{center}
        \scriptsize
        \begin{math}
          $$\mprset{flushleft}
          \inferrule* [right={\tiny cut}] {
            {
              \begin{array}{cc}
                \pi_1 & \Pi' \\
                {\Phi  \vdash_\mathcal{C}  \SCnt{X}} & {\Psi_{{\mathrm{1}}}  \SCsym{,}  \Psi_{{\mathrm{2}}}  \SCsym{,}  \SCnt{X}  \SCsym{,}  \Psi_{{\mathrm{3}}}  \SCsym{,}  \Psi_{{\mathrm{4}}}  \vdash_\mathcal{C}  \SCnt{Z}}
              \end{array}
            }
          }{\Psi_{{\mathrm{1}}}  \SCsym{,}  \Psi_{{\mathrm{2}}}  \SCsym{,}  \Phi  \SCsym{,}  \Psi_{{\mathrm{3}}}  \SCsym{,}  \Psi_{{\mathrm{4}}}  \vdash_\mathcal{C}  \SCnt{Z}}
        \end{math}
      \end{center}
\end{itemize}



% C-Cut vs. LC-Cut Case 1
\subsubsection{$\SCdruleTXXcutName$ vs. $\SCdruleSXXcutOneName$}
\begin{itemize}
\item Case 1:
      \begin{center}
        \scriptsize
        \begin{math}
          \begin{array}{c}
            \Pi_1 \\
            {\Phi  \vdash_\mathcal{C}  \SCnt{X}}
          \end{array}
        \end{math}
        \qquad\qquad
        $\Pi_2:$
        \begin{math}
          $$\mprset{flushleft}
          \inferrule* [right={\tiny cut1}] {
            {
              \begin{array}{cc}
                \pi_2 & \pi_3 \\
                {\Psi_{{\mathrm{1}}}  \SCsym{,}  \SCnt{X}  \SCsym{,}  \Psi_{{\mathrm{2}}}  \vdash_\mathcal{C}  \SCnt{Y}} & {\Gamma_{{\mathrm{1}}}  \SCsym{;}  \SCnt{Y}  \SCsym{;}  \Gamma_{{\mathrm{2}}}  \vdash_\mathcal{L}  \SCnt{A}}
              \end{array}
            }
          }{\Gamma_{{\mathrm{1}}}  \SCsym{;}  \Psi_{{\mathrm{1}}}  \SCsym{;}  \SCnt{X}  \SCsym{;}  \Psi_{{\mathrm{2}}}  \SCsym{;}  \Gamma_{{\mathrm{2}}}  \vdash_\mathcal{L}  \SCnt{A}}
        \end{math}
      \end{center}
      By assumption, $c(\Pi_1),c(\Pi_2)\leq |X|$. Therefore, $c(\pi_1)$,
      $c(\pi_2)\leq |X|$. Since $Y$ is the cut formula on $\pi_1$ and
      $\pi_2$, we have $|Y|+1\leq|X|$. By induction on $\Pi_1$ and $\pi_1$,
      there exists a proof $\Pi'$ for sequent $\Psi_{{\mathrm{1}}}  \SCsym{,}  \Phi  \SCsym{,}  \Psi_{{\mathrm{2}}}  \vdash_\mathcal{C}  \SCnt{Y}$ s.t.
      $c(\Pi')\leq|X|$. So $\Pi$ can be constructed as follows, with
      $c(\Pi)\leq max\{c(\Pi'),c(\pi_2),|Y|+1\}\leq |X|$.
      \begin{center}
        \scriptsize
        \begin{math}
          $$\mprset{flushleft}
          \inferrule* [right={\tiny cut1}] {
            {
              \begin{array}{cc}
                \Pi' & \pi_2 \\
                {\Psi_{{\mathrm{1}}}  \SCsym{,}  \Phi  \SCsym{,}  \Psi_{{\mathrm{2}}}  \vdash_\mathcal{C}  \SCnt{Y}} & {\Gamma_{{\mathrm{1}}}  \SCsym{;}  \SCnt{Y}  \SCsym{;}  \Gamma_{{\mathrm{2}}}  \vdash_\mathcal{L}  \SCnt{A}}
              \end{array}
            }
          }{\Gamma_{{\mathrm{1}}}  \SCsym{;}  \Psi_{{\mathrm{1}}}  \SCsym{;}  \Phi  \SCsym{;}  \Psi_{{\mathrm{2}}}  \SCsym{;}  \Gamma_{{\mathrm{2}}}  \vdash_\mathcal{L}  \SCnt{A}}
        \end{math}
      \end{center}

% C-Cut vs. LC-Cut Case 2
\item Case 2:
      \begin{center}
        \scriptsize
        $\Pi_1$:
        \begin{math}
          $$\mprset{flushleft}
          \inferrule* [right={\tiny cut}] {
            {
              \begin{array}{cc}
                \pi_1 & \pi_2 \\
                {\Phi  \vdash_\mathcal{C}  \SCnt{X}} & {\Psi_{{\mathrm{1}}}  \SCsym{,}  \SCnt{X}  \SCsym{,}  \Psi_{{\mathrm{2}}}  \vdash_\mathcal{C}  \SCnt{Y}}
              \end{array}
            }
          }{\Psi_{{\mathrm{1}}}  \SCsym{,}  \Phi  \SCsym{,}  \Psi_{{\mathrm{2}}}  \vdash_\mathcal{C}  \SCnt{Y}}
        \end{math}
        \qquad\qquad
        \begin{math}
          \begin{array}{c}
            \Pi_2 \\
            {\Gamma_{{\mathrm{1}}}  \SCsym{;}  \SCnt{Y}  \SCsym{;}  \Gamma_{{\mathrm{2}}}  \vdash_\mathcal{L}  \SCnt{A}}
          \end{array}
        \end{math}
      \end{center}
      By assumption, $c(\Pi_1),c(\Pi_2)\leq |Y|$. Similar as above,
      $|X|+1\leq |Y|$ and there is a proof $\Pi'$ constructed from $\pi_2$
      and $\Pi_2$ for sequent $\Gamma_{{\mathrm{1}}}  \SCsym{;}  \Psi_{{\mathrm{1}}}  \SCsym{;}  \SCnt{X}  \SCsym{;}  \Psi_{{\mathrm{2}}}  \SCsym{;}  \Gamma_{{\mathrm{2}}}  \vdash_\mathcal{L}  \SCnt{A}$ s.t.
      $c(\Pi')\leq|Y|$. Therefore, the proof $\Pi$ can be constructed as
      follows, and $c(\Pi)\leq max\{c(\pi_1),c(\Pi'),|X|+1\}\leq |Y|$.
      \begin{center}
        \scriptsize
        \begin{math}
          $$\mprset{flushleft}
          \inferrule* [right={\tiny cut}] {
            {
              \begin{array}{cc}
                \pi_1 & \Pi'\\
                {\Phi  \vdash_\mathcal{C}  \SCnt{X}} & {\Gamma_{{\mathrm{1}}}  \SCsym{;}  \Psi_{{\mathrm{1}}}  \SCsym{;}  \SCnt{X}  \SCsym{;}  \Psi_{{\mathrm{2}}}  \SCsym{;}  \Gamma_{{\mathrm{2}}}  \vdash_\mathcal{L}  \SCnt{A}}
              \end{array}
            }
          }{\Gamma_{{\mathrm{1}}}  \SCsym{;}  \Psi_{{\mathrm{1}}}  \SCsym{;}  \Phi  \SCsym{;}  \Psi_{{\mathrm{2}}}  \SCsym{;}  \Gamma_{{\mathrm{2}}}  \vdash_\mathcal{L}  \SCnt{A}}
        \end{math}
      \end{center}
\end{itemize}

% LC-Cut vs. L-Cut Case 1
\subsubsection{$\SCdruleSXXcutOneName$ vs. $\SCdruleSXXcutTwoName$}
\begin{itemize}
\item Case 1:
      \begin{center}
        \scriptsize
        \begin{math}
          \begin{array}{c}
            \Pi_1 \\
            {\Phi  \vdash_\mathcal{C}  \SCnt{X}}
          \end{array}
        \end{math}
        \qquad\qquad
        $\Pi_2:$
        \begin{math}
          $$\mprset{flushleft}
          \inferrule* [right={\tiny cut2}] {
            {
              \begin{array}{cc}
                \pi_1 & \pi_2 \\
                {\Gamma_{{\mathrm{2}}}  \SCsym{;}  \SCnt{X}  \SCsym{;}  \Gamma_{{\mathrm{3}}}  \vdash_\mathcal{L}  \SCnt{A}} & {\Gamma_{{\mathrm{1}}}  \SCsym{;}  \SCnt{A}  \SCsym{;}  \Gamma_{{\mathrm{4}}}  \vdash_\mathcal{L}  \SCnt{B}}
              \end{array}
            }
          }{\Gamma_{{\mathrm{1}}}  \SCsym{;}  \Gamma_{{\mathrm{2}}}  \SCsym{;}  \SCnt{X}  \SCsym{;}  \Gamma_{{\mathrm{3}}}  \SCsym{;}  \Gamma_{{\mathrm{4}}}  \vdash_\mathcal{L}  \SCnt{B}}
        \end{math}
      \end{center}
      By assumption, $c(\Pi_1),c(\Pi_2)\leq |X|$. Therefore, $c(\pi_1)$,
      $c(\pi_2)\leq |X|$. Since $A$ is the cut formula on $\pi_1$ and
      $\pi_2$, we have $|A|+1\leq|X|$. By induction on $\Pi_1$ and $\pi_1$,
      there exists a proof $\Pi'$ for sequent $\Gamma_{{\mathrm{2}}}  \SCsym{;}  \Phi  \SCsym{;}  \Gamma_{{\mathrm{3}}}  \vdash_\mathcal{L}  \SCnt{A}$ s.t.
      $c(\Pi')\leq|X|$. So $\Pi$ can be constructed as follows, with
      $c(\Pi)\leq max\{c(\Pi'),c(\pi_2),|A|+1\}\leq |X|$.
      \begin{center}
        \scriptsize
        \begin{math}
          $$\mprset{flushleft}
          \inferrule* [right={\tiny cut2}] {
            {
              \begin{array}{cc}
                \Pi' & \pi_2 \\
                {\Gamma_{{\mathrm{2}}}  \SCsym{;}  \Phi  \SCsym{;}  \Gamma_{{\mathrm{3}}}  \vdash_\mathcal{L}  \SCnt{A}} & {\Gamma_{{\mathrm{1}}}  \SCsym{;}  \SCnt{A}  \SCsym{;}  \Gamma_{{\mathrm{4}}}  \vdash_\mathcal{L}  \SCnt{B}}
              \end{array}
            }
          }{\Gamma_{{\mathrm{1}}}  \SCsym{;}  \Gamma_{{\mathrm{2}}}  \SCsym{;}  \Phi  \SCsym{;}  \Gamma_{{\mathrm{3}}}  \SCsym{;}  \Gamma_{{\mathrm{4}}}  \vdash_\mathcal{L}  \SCnt{B}}
        \end{math}
      \end{center}

% LC-Cut vs. L-Cut Case 2
\item Case 2:
      \begin{center}
        \scriptsize
        $\Pi_1$:
        \begin{math}
          $$\mprset{flushleft}
          \inferrule* [right={\tiny cut}] {
            {
              \begin{array}{cc}
                \pi_1 & \pi_2 \\
                {\Phi  \vdash_\mathcal{C}  \SCnt{X}} & {\Gamma_{{\mathrm{2}}}  \SCsym{;}  \SCnt{X}  \SCsym{;}  \Gamma_{{\mathrm{3}}}  \vdash_\mathcal{L}  \SCnt{A}}
              \end{array}
            }
          }{\Gamma_{{\mathrm{2}}}  \SCsym{;}  \Phi  \SCsym{;}  \Gamma_{{\mathrm{3}}}  \vdash_\mathcal{L}  \SCnt{A}}
        \end{math}
        \qquad\qquad
        \begin{math}
          \begin{array}{c}
            \Pi_2 \\
            {\Gamma_{{\mathrm{1}}}  \SCsym{;}  \SCnt{A}  \SCsym{;}  \Gamma_{{\mathrm{4}}}  \vdash_\mathcal{L}  \SCnt{B}}
          \end{array}
        \end{math}
      \end{center}
      By assumption, $c(\Pi_1),c(\Pi_2)\leq |A|$. Similar as above,
      $|X|+1\leq |A|$ and there is a proof $\Pi'$ constructed from'
      $\pi_2$ and $\Pi_2$ for sequent $\Gamma_{{\mathrm{1}}}  \SCsym{;}  \Gamma_{{\mathrm{2}}}  \SCsym{;}  \SCnt{X}  \SCsym{;}  \Gamma_{{\mathrm{3}}}  \SCsym{;}  \Gamma_{{\mathrm{4}}}  \vdash_\mathcal{L}  \SCnt{B}$ s.t.
      $c(\Pi')\leq|A|$. Therefore, the proof $\Pi$ can be constructed as
      follows, and $c(\Pi)\leq max\{c(\pi_1),c(\Pi'),|X|+1\}\leq |A|$.
      \begin{center}
        \scriptsize
        \begin{math}
          $$\mprset{flushleft}
          \inferrule* [right={\tiny cut}] {
            {
              \begin{array}{cc}
                \pi_1  & \Pi' \\
                {\Phi  \vdash_\mathcal{C}  \SCnt{X}} & {\Gamma_{{\mathrm{1}}}  \SCsym{;}  \Gamma_{{\mathrm{2}}}  \SCsym{;}  \SCnt{X}  \SCsym{;}  \Gamma_{{\mathrm{3}}}  \SCsym{;}  \Gamma_{{\mathrm{4}}}  \vdash_\mathcal{L}  \SCnt{B}}
              \end{array}
            }
          }{\Gamma_{{\mathrm{1}}}  \SCsym{;}  \Gamma_{{\mathrm{2}}}  \SCsym{;}  \Phi  \SCsym{;}  \Gamma_{{\mathrm{3}}}  \SCsym{;}  \Gamma_{{\mathrm{4}}}  \vdash_\mathcal{L}  \SCnt{B}}
        \end{math}
      \end{center}
\end{itemize}

% L-Cut vs. L-Cut Case 1
\subsubsection{$\SCdruleSXXcutTwoName$ vs. $\SCdruleSXXcutTwoName$}
\begin{itemize}
\item Case 1:
      \begin{center}
        \scriptsize
        \begin{math}
          \begin{array}{c}
            \Pi_1 \\
            {\Gamma  \vdash_\mathcal{L}  \SCnt{A}}
          \end{array}
        \end{math}
        \qquad\qquad
        $\Pi_2:$
        \begin{math}
          $$\mprset{flushleft}
          \inferrule* [right={\tiny cut2}] {
            {
              \begin{array}{cc}
                \pi_1 & \pi_2 \\
                {\Delta_{{\mathrm{2}}}  \SCsym{;}  \SCnt{A}  \SCsym{;}  \Delta_{{\mathrm{3}}}  \vdash_\mathcal{L}  \SCnt{B}} & {\Delta_{{\mathrm{1}}}  \SCsym{;}  \SCnt{B}  \SCsym{;}  \Delta_{{\mathrm{4}}}  \vdash_\mathcal{L}  \SCnt{C}}
              \end{array}
            }
          }{\Delta_{{\mathrm{1}}}  \SCsym{;}  \Delta_{{\mathrm{2}}}  \SCsym{;}  \SCnt{A}  \SCsym{;}  \Delta_{{\mathrm{3}}}  \SCsym{;}  \Delta_{{\mathrm{4}}}  \vdash_\mathcal{L}  \SCnt{C}}
        \end{math}
      \end{center}
      By assumption, $c(\Pi_1),c(\Pi_2)\leq |A|$. Therefore, $c(\pi_1)$,
      $c(\pi_2)\leq |A|$. Since $B$ is the cut formula on $\pi_1$ and
      $\pi_3$, we have $|B|+1\leq|A|$. By induction on $\Pi_1$ and
      $\pi_1$, there exists a proof $\Pi'$ for sequent
      $\Delta_{{\mathrm{2}}}  \SCsym{;}  \Gamma  \SCsym{;}  \Delta_{{\mathrm{3}}}  \vdash_\mathcal{L}  \SCnt{B}$ s.t. $c(\Pi')\leq|A|$. So $\Pi$ can be
      constructed as follows,  with
      $c(\Pi)\leq max\{c(\Pi'),c(\pi_2),|B|+1\}\leq |A|$.
      \begin{center}
        \scriptsize
        \begin{math}
          $$\mprset{flushleft}
          \inferrule* [right={\tiny cut}] {
            {
              \begin{array}{cc}
                \Pi' & \pi_2 \\
                {\Delta_{{\mathrm{2}}}  \SCsym{;}  \Gamma  \SCsym{;}  \Delta_{{\mathrm{3}}}  \vdash_\mathcal{L}  \SCnt{B}} & {\Delta_{{\mathrm{1}}}  \SCsym{;}  \SCnt{B}  \SCsym{;}  \Delta_{{\mathrm{4}}}  \vdash_\mathcal{L}  \SCnt{C}}
              \end{array}
            }
          }{\Delta_{{\mathrm{1}}}  \SCsym{;}  \Delta_{{\mathrm{2}}}  \SCsym{;}  \Gamma  \SCsym{;}  \Delta_{{\mathrm{3}}}  \SCsym{;}  \Delta_{{\mathrm{4}}}  \vdash_\mathcal{L}  \SCnt{C}}
        \end{math}
      \end{center}

% L-Cut vs. L-Cut Case 2
\item Case 2:
      \begin{center}
        \scriptsize
        $\Pi_1$:
        \begin{math}
          $$\mprset{flushleft}
          \inferrule* [right={\tiny cut}] {
            {
              \begin{array}{cc}
                \pi_1 & \pi_2 \\
                {\Delta  \vdash_\mathcal{L}  \SCnt{A}} & {\Delta_{{\mathrm{2}}}  \SCsym{;}  \SCnt{A}  \SCsym{;}  \Delta_{{\mathrm{3}}}  \vdash_\mathcal{L}  \SCnt{B}}
              \end{array}
            }
          }{\Delta_{{\mathrm{2}}}  \SCsym{;}  \Delta  \SCsym{;}  \Delta_{{\mathrm{3}}}  \vdash_\mathcal{L}  \SCnt{A}}
        \end{math}
        \qquad\qquad
        \begin{math}
          \begin{array}{c}
            \Pi_2 \\
            {\Delta_{{\mathrm{1}}}  \SCsym{;}  \SCnt{B}  \SCsym{;}  \Delta_{{\mathrm{4}}}  \vdash_\mathcal{L}  \SCnt{C}}
          \end{array}
        \end{math}
      \end{center}
      By assumption, $c(\Pi_1),c(\Pi_2)\leq |B|$. Similar as above,
      $|A|+1\leq |B|$ and there is a proof $\Pi'$ constructed from $\pi_2$ 
      and $\Pi_2$ for sequent $\Delta_{{\mathrm{1}}}  \SCsym{;}  \Delta_{{\mathrm{2}}}  \SCsym{;}  \SCnt{A}  \SCsym{;}  \Delta_{{\mathrm{3}}}  \SCsym{;}  \Delta_{{\mathrm{4}}}  \vdash_\mathcal{L}  \SCnt{C}$ s.t.
      $c(\Pi')\leq|A|$. Therefore, the proof $\Pi$ can be constructed as
      follows, and $c(\Pi)\leq max\{c(\pi_1),c(\Pi'),|A|+1\}\leq |B|$.
      \begin{center}
        \scriptsize
        \begin{math}
          $$\mprset{flushleft}
          \inferrule* [right={\tiny cut}] {
            {
              \begin{array}{cc}
                \pi_1 & \Pi' \\
                {\Gamma  \vdash_\mathcal{L}  \SCnt{A}} & {\Delta_{{\mathrm{1}}}  \SCsym{;}  \Delta_{{\mathrm{2}}}  \SCsym{;}  \SCnt{A}  \SCsym{;}  \Delta_{{\mathrm{3}}}  \SCsym{;}  \Delta_{{\mathrm{4}}}  \vdash_\mathcal{L}  \SCnt{C}}
              \end{array}
            }
          }{\Delta_{{\mathrm{1}}}  \SCsym{;}  \Delta_{{\mathrm{2}}}  \SCsym{;}  \Gamma  \SCsym{;}  \Delta_{{\mathrm{3}}}  \SCsym{;}  \Delta_{{\mathrm{4}}}  \vdash_\mathcal{L}  \SCnt{C}}
        \end{math}
      \end{center}

\end{itemize}
% End of subsubsection Commuting conversion cut vs. cut



\subsection{The Axiom Steps}

\subsubsection{$\SCdruleTXXaxName$}
\begin{itemize}
% C-id Case 1
\item Case 1:
      \begin{center}
        \scriptsize
        $\Pi_1$:
        \begin{math}
          $$\mprset{flushleft}
          \inferrule* [right={\tiny ax}] {
            \,
          }{\SCnt{X}  \vdash_\mathcal{C}  \SCnt{X}}
        \end{math}
        \qquad\qquad
        \begin{math}
          \begin{array}{c}
            \Pi_2 \\
            {\Phi_{{\mathrm{1}}}  \SCsym{,}  \SCnt{X}  \SCsym{,}  \Phi_{{\mathrm{2}}}  \vdash_\mathcal{C}  \SCnt{Y}}
          \end{array}
        \end{math}
      \end{center}
      By assumption, $c(\Pi_1),c(\Pi_2)\leq |X|$. The proof $\Pi$ is the
      same as $\Pi_2$.

% C-id Case 2
\item Case 2:
      \begin{center}
        \scriptsize
        $\Pi_1$:
        \begin{math}
          \begin{array}{c}
            \Pi_1 \\
            {\Phi  \vdash_\mathcal{C}  \SCnt{X}}
          \end{array}
        \end{math}
        \qquad\qquad
        $\Pi_2$:
        \begin{math}
          $$\mprset{flushleft}
          \inferrule* [right={\tiny ax}] {
            \,
          }{\SCnt{X}  \vdash_\mathcal{C}  \SCnt{X}}
        \end{math}
      \end{center}
      By assumption, $c(\Pi_1),c(\Pi_2)\leq |X|$. The proof $\Pi$ is the
      same as $\Pi_1$.

% C-id Case 3
\item Case 3:
      \begin{center}
        \scriptsize
        $\Pi_1$:
        \begin{math}
          $$\mprset{flushleft}
          \inferrule* [right={\tiny ax}] {
            \,
          }{\SCnt{X}  \vdash_\mathcal{C}  \SCnt{X}}
        \end{math}
        \qquad\qquad
        \begin{math}
          \begin{array}{c}
            \Pi_2 \\
            {\Gamma_{{\mathrm{1}}}  \SCsym{;}  \SCnt{X}  \SCsym{;}  \Gamma_{{\mathrm{2}}}  \vdash_\mathcal{L}  \SCnt{A}}
          \end{array}
        \end{math}
      \end{center}
      By assumption, $c(\Pi_1),c(\Pi_2)\leq |X|$. The proof $\Pi$ is the
      same as $\Pi_2$.
\end{itemize}

% L-id Case 1
\subsubsection{$\SCdruleTXXaxName$}
\begin{itemize}
\item Case 1:
      \begin{center}
        \scriptsize
        $\Pi_1$:
        \begin{math}
          $$\mprset{flushleft}
          \inferrule* [right={\tiny ax}] {
            \,
          }{\SCnt{A}  \vdash_\mathcal{L}  \SCnt{A}}
        \end{math}
        \qquad\qquad
        \begin{math}
          \begin{array}{c}
            \Pi_2 \\
            {\Gamma_{{\mathrm{1}}}  \SCsym{;}  \SCnt{A}  \SCsym{;}  \Gamma_{{\mathrm{2}}}  \vdash_\mathcal{L}  \SCnt{B}}
          \end{array}
        \end{math}
      \end{center}
      By assumption, $c(\Pi_1),c(\Pi_2)\leq |A|$. The proof $\Pi$ is the
      same as $\Pi_2$.

% L-id Case 2
\item Case 2:
      \begin{center}
        \scriptsize
        $\Pi_1$:
        \begin{math}
          \begin{array}{c}
            \Pi_1 \\
            {\Delta  \vdash_\mathcal{L}  \SCnt{A}}
          \end{array}
        \end{math}
        \qquad\qquad
        $\Pi_2$:
        \begin{math}
          $$\mprset{flushleft}
          \inferrule* [right={\tiny ax}] {
            \,
          }{\SCnt{A}  \vdash_\mathcal{L}  \SCnt{A}}
        \end{math}
      \end{center}
      By assumption, $c(\Pi_1),c(\Pi_2)\leq |X|$. The proof $\Pi$ is the
      same as $\Pi_1$.
\end{itemize}
% End of subsubsection Axiom steps



\subsection{The Exchange Steps}

\subsubsection{$\SCdruleTXXexName$}

\begin{itemize}
% Conclusion vs. C-ex Case 1
\item Case 1:
      \begin{center}
        \scriptsize
        \begin{math}
          \begin{array}{c}
            \Pi_1 \\
            {\Psi  \vdash_\mathcal{C}  \SCnt{X_{{\mathrm{1}}}}}
          \end{array}
        \end{math}
        \qquad\qquad
        $\Pi_2$:
        \begin{math}
          $$\mprset{flushleft}
          \inferrule* [right={\tiny ex}] {
            {
              \begin{array}{c}
                \pi \\
                {\Phi_{{\mathrm{1}}}  \SCsym{,}  \SCnt{X_{{\mathrm{1}}}}  \SCsym{,}  \SCnt{X_{{\mathrm{2}}}}  \SCsym{,}  \Phi_{{\mathrm{2}}}  \vdash_\mathcal{C}  \SCnt{Y}}
              \end{array}
            }
          }{\Phi_{{\mathrm{1}}}  \SCsym{,}  \SCnt{X_{{\mathrm{2}}}}  \SCsym{,}  \SCnt{X_{{\mathrm{1}}}}  \SCsym{,}  \Phi_{{\mathrm{2}}}  \vdash_\mathcal{C}  \SCnt{Y}}
        \end{math}
      \end{center}
      By assumption, $c(\Pi_1),c(\Pi_2)\leq |X_1|$. By induction on $\pi$
      and $\Pi_1$, there is a proof $\Pi'$ for sequent
      $\Phi_{{\mathrm{1}}}  \SCsym{,}  \Psi  \SCsym{,}  \SCnt{X_{{\mathrm{2}}}}  \SCsym{,}  \Phi_{{\mathrm{2}}}  \vdash_\mathcal{C}  \SCnt{Y}$ s.t. $c(\Pi')\leq|X_1|$. Therefore, the
      proof $\Pi$ can be constructed as follows, and
      $c(\Pi)=c(\Pi')\leq|X_1|$.
      \begin{center}
        \scriptsize
        \begin{math}
          $$\mprset{flushleft}
          \inferrule* [right={\tiny series of ex}] {
            {
              \begin{array}{c}
                \Pi' \\
                {\Phi_{{\mathrm{1}}}  \SCsym{,}  \Psi  \SCsym{,}  \SCnt{X_{{\mathrm{2}}}}  \SCsym{,}  \Phi_{{\mathrm{2}}}  \vdash_\mathcal{C}  \SCnt{Y}}
              \end{array}
            }
          }{\Phi_{{\mathrm{1}}}  \SCsym{,}  \SCnt{X_{{\mathrm{2}}}}  \SCsym{,}  \Psi  \SCsym{,}  \Phi_{{\mathrm{2}}}  \vdash_\mathcal{C}  \SCnt{Y}}
        \end{math}
      \end{center}

% Conclusion vs. C-ex Case 2
\item Case 2:
      \begin{center}
        \scriptsize
        \begin{math}
          \begin{array}{c}
            \Pi_1 \\
            {\Psi  \vdash_\mathcal{C}  \SCnt{X_{{\mathrm{2}}}}}
          \end{array}
        \end{math}
        \qquad\qquad
        $\Pi_2$:
        \begin{math}
          $$\mprset{flushleft}
          \inferrule* [right={\tiny ex}] {
            {
              \begin{array}{c}
                \pi \\
                {\Phi_{{\mathrm{1}}}  \SCsym{,}  \SCnt{X_{{\mathrm{1}}}}  \SCsym{,}  \SCnt{X_{{\mathrm{2}}}}  \SCsym{,}  \Phi_{{\mathrm{2}}}  \vdash_\mathcal{C}  \SCnt{Y}}
              \end{array}
            }
          }{\Phi_{{\mathrm{1}}}  \SCsym{,}  \SCnt{X_{{\mathrm{2}}}}  \SCsym{,}  \SCnt{X_{{\mathrm{1}}}}  \SCsym{,}  \Phi_{{\mathrm{2}}}  \vdash_\mathcal{C}  \SCnt{Y}}
        \end{math}
      \end{center}
      By assumption, $c(\Pi_1),c(\Pi_2)\leq |X_2|$. By induction on $\pi$
      and $\Pi_1$, there is a proof $\Pi'$ for sequent
      $\Phi_{{\mathrm{1}}}  \SCsym{,}  \SCnt{X_{{\mathrm{1}}}}  \SCsym{,}  \Psi  \SCsym{,}  \Phi_{{\mathrm{2}}}  \vdash_\mathcal{C}  \SCnt{Y}$ s.t. $c(\Pi')\leq|X_2|$. Therefore, the
      proof $\Pi$ can be constructed as follows, and
      $c(\Pi)=c(\Pi')\leq|X_2|$.
      \begin{center}
        \scriptsize
        \begin{math}
          $$\mprset{flushleft}
          \inferrule* [right={\tiny series of ex}] {
            {
              \begin{array}{c}
                \Pi' \\
                {\Phi_{{\mathrm{1}}}  \SCsym{,}  \SCnt{X_{{\mathrm{1}}}}  \SCsym{,}  \Psi  \SCsym{,}  \Phi_{{\mathrm{2}}}  \vdash_\mathcal{C}  \SCnt{Y}}
              \end{array}
            }
          }{\Phi_{{\mathrm{1}}}  \SCsym{,}  \Psi  \SCsym{,}  \SCnt{X_{{\mathrm{1}}}}  \SCsym{,}  \Phi_{{\mathrm{2}}}  \vdash_\mathcal{C}  \SCnt{Y}}
        \end{math}
      \end{center}
\end{itemize}

% Conclusion vs. LC-ex Case 1
\subsubsection{$\SCdruleSXXexName$}
\begin{itemize}
\item Case 1:
      \begin{center}
        \scriptsize
        \begin{math}
          \begin{array}{c}
            \Pi_1 \\
            {\Psi  \vdash_\mathcal{C}  \SCnt{X_{{\mathrm{1}}}}}
          \end{array}
        \end{math}
        \qquad\qquad
        $\Pi_2$:
        \begin{math}
          $$\mprset{flushleft}
          \inferrule* [right={\tiny ex}] {
            {
              \begin{array}{c}
                \pi \\
                {\Delta_{{\mathrm{1}}}  \SCsym{;}  \SCnt{X_{{\mathrm{1}}}}  \SCsym{;}  \SCnt{X_{{\mathrm{2}}}}  \SCsym{;}  \Delta_{{\mathrm{2}}}  \vdash_\mathcal{L}  \SCnt{A}}
              \end{array}
            }
          }{\Delta_{{\mathrm{1}}}  \SCsym{;}  \SCnt{X_{{\mathrm{2}}}}  \SCsym{;}  \SCnt{X_{{\mathrm{1}}}}  \SCsym{;}  \Delta_{{\mathrm{2}}}  \vdash_\mathcal{L}  \SCnt{A}}
        \end{math}
      \end{center}
      By assumption, $c(\Pi_1),c(\Pi_2)\leq |X_1|$. By induction on $\pi$
      and $\Pi_1$, there is a proof $\Pi'$ for sequent
      $\Delta_{{\mathrm{1}}}  \SCsym{;}  \Psi  \SCsym{;}  \SCnt{X_{{\mathrm{2}}}}  \SCsym{;}  \Delta_{{\mathrm{2}}}  \vdash_\mathcal{L}  \SCnt{A}$ s.t. $c(\Pi')\leq|X_1|$. Therefore, the
      proof $\Pi$ can be constructed as follows, and
      $c(\Pi)=c(\Pi')\leq|X_1|$.
      \begin{center}
        \scriptsize
        \begin{math}
          $$\mprset{flushleft}
          \inferrule* [right={\tiny series of ex}] {
            {
              \begin{array}{c}
                \Pi' \\
                {\Delta_{{\mathrm{1}}}  \SCsym{;}  \Psi  \SCsym{;}  \SCnt{X_{{\mathrm{2}}}}  \SCsym{;}  \Delta_{{\mathrm{2}}}  \vdash_\mathcal{L}  \SCnt{A}}
              \end{array}
            }
          }{\Delta_{{\mathrm{1}}}  \SCsym{;}  \SCnt{X_{{\mathrm{2}}}}  \SCsym{;}  \Psi  \SCsym{;}  \Delta_{{\mathrm{2}}}  \vdash_\mathcal{L}  \SCnt{A}}
        \end{math}
      \end{center}

% Conclusion vs. LC-ex Case 2
\item Case 2:
      \begin{center}
        \scriptsize
        \begin{math}
          \begin{array}{c}
            \Pi_1 \\
            {\Psi  \vdash_\mathcal{C}  \SCnt{X_{{\mathrm{2}}}}}
          \end{array}
        \end{math}
        \qquad\qquad
        $\Pi_2$:
        \begin{math}
          $$\mprset{flushleft}
          \inferrule* [right={\tiny ex}] {
            {
              \begin{array}{c}
                \pi \\
                {\Delta_{{\mathrm{1}}}  \SCsym{;}  \SCnt{X_{{\mathrm{1}}}}  \SCsym{;}  \SCnt{X_{{\mathrm{2}}}}  \SCsym{;}  \Delta_{{\mathrm{2}}}  \vdash_\mathcal{L}  \SCnt{A}}
              \end{array}
            }
          }{\Delta_{{\mathrm{1}}}  \SCsym{;}  \SCnt{X_{{\mathrm{2}}}}  \SCsym{;}  \SCnt{X_{{\mathrm{1}}}}  \SCsym{;}  \Delta_{{\mathrm{2}}}  \vdash_\mathcal{L}  \SCnt{A}}
        \end{math}
      \end{center}
      By assumption, $c(\Pi_1),c(\Pi_2)\leq |X_2|$. By induction on $\pi$
      and $\Pi_1$, there is a proof $\Pi'$ for sequent
      $\Delta_{{\mathrm{1}}}  \SCsym{;}  \SCnt{X_{{\mathrm{1}}}}  \SCsym{;}  \Psi  \SCsym{;}  \Delta_{{\mathrm{2}}}  \vdash_\mathcal{L}  \SCnt{A}$ s.t. $c(\Pi')\leq|X_2|$. Therefore, the
      proof $\Pi$ can be constructed as follows, and
      $c(\Pi)=c(\Pi')\leq|X_2|$.
      \begin{center}
        \scriptsize
        \begin{math}
          $$\mprset{flushleft}
          \inferrule* [right={\tiny series of ex}] {
            {
              \begin{array}{c}
                \Pi' \\
                {\Delta_{{\mathrm{1}}}  \SCsym{;}  \SCnt{X_{{\mathrm{1}}}}  \SCsym{;}  \Psi  \SCsym{;}  \Phi_{{\mathrm{2}}}  \vdash_\mathcal{L}  \SCnt{A}}
              \end{array}
            }
          }{\Phi_{{\mathrm{1}}}  \SCsym{;}  \Psi  \SCsym{;}  \SCnt{X_{{\mathrm{1}}}}  \SCsym{;}  \Phi_{{\mathrm{2}}}  \vdash_\mathcal{L}  \SCnt{A}}
        \end{math}
      \end{center}
\end{itemize}



\subsection{Principal Formula vs. Principal Formula} 

\subsubsection{The Commutative Tensor Product $\otimes$}
\begin{center}
  \scriptsize
  $\Pi_1:$
  \begin{math}
    $$\mprset{flushleft}
    \inferrule* [right={\tiny tenR}] {
      {
        \begin{array}{cc}
          \pi_1 & \pi_2 \\
          {\Phi_{{\mathrm{1}}}  \vdash_\mathcal{C}  \SCnt{X}} & {\Phi_{{\mathrm{2}}}  \vdash_\mathcal{C}  \SCnt{Y}}
        \end{array}
      }
    }{\Phi_{{\mathrm{1}}}  \SCsym{,}  \Phi_{{\mathrm{2}}}  \vdash_\mathcal{C}  \SCnt{X}  \otimes  \SCnt{Y}}
  \end{math}
  \qquad\qquad
  $\Pi_2:$
  \begin{math}
    $$\mprset{flushleft}
    \inferrule* [right={\tiny tenL}] {
      {
        \begin{array}{c}
          \pi_3 \\
          {\Psi_{{\mathrm{1}}}  \SCsym{,}  \SCnt{X}  \SCsym{,}  \SCnt{Y}  \SCsym{,}  \Psi_{{\mathrm{2}}}  \vdash_\mathcal{C}  \SCnt{Z}}
        \end{array}
      }
    }{\Psi_{{\mathrm{1}}}  \SCsym{,}  \SCnt{X}  \otimes  \SCnt{Y}  \SCsym{,}  \Psi_{{\mathrm{2}}}  \vdash_\mathcal{C}  \SCnt{Z}}
  \end{math}
\end{center}
By assumption, $c(\Pi_1),c(\Pi_2)\leq |\SCnt{X}  \otimes  \SCnt{Y}| = |X|+|Y|+1$. The proof
$\Pi$ can be constructed as follows, and
$c(\Pi)\leq max\{c(\pi_1),c(\pi_2),c(\pi_3),|X|+1,|Y|+1\}\leq |X|+|Y|+1 = |\SCnt{X}  \otimes  \SCnt{Y}|$.
\begin{center}
  \scriptsize
  \begin{math}
    $$\mprset{flushleft}
    \inferrule* [right={\tiny cut}] {
      {
        \begin{array}{c}
          \pi_1 \\
          {\Phi_{{\mathrm{1}}}  \vdash_\mathcal{C}  \SCnt{X}}
        \end{array}
      }
      $$\mprset{flushleft}
      \inferrule* [right={\tiny cut}] {
      {
        \begin{array}{cc}
          \pi_2 & \pi_3 \\
          {\Phi_{{\mathrm{2}}}  \vdash_\mathcal{C}  \SCnt{Y}} & {\Psi_{{\mathrm{1}}}  \SCsym{,}  \SCnt{X}  \SCsym{,}  \SCnt{Y}  \SCsym{,}  \Psi_{{\mathrm{2}}}  \vdash_\mathcal{C}  \SCnt{Z}}
        \end{array}
      }
      }{\Psi_{{\mathrm{1}}}  \SCsym{,}  \SCnt{X}  \SCsym{,}  \Phi_{{\mathrm{2}}}  \SCsym{,}  \Psi_{{\mathrm{2}}}  \vdash_\mathcal{C}  \SCnt{Z}}
    }{\Psi_{{\mathrm{1}}}  \SCsym{,}  \Phi_{{\mathrm{1}}}  \SCsym{,}  \Phi_{{\mathrm{2}}}  \SCsym{,}  \Psi_{{\mathrm{2}}}  \vdash_\mathcal{C}  \SCnt{Z}}
  \end{math}
\end{center}

\subsubsection{The Non-commutative Tensor Product $\tri$}
\begin{center}
  \scriptsize
  $\Pi_1:$
  \begin{math}
    $$\mprset{flushleft}
    \inferrule* [right={\tiny tenR}] {
      {
        \begin{array}{cc}
          \pi_1 & \pi_2 \\
          {\Gamma_{{\mathrm{1}}}  \vdash_\mathcal{L}  \SCnt{A}} & {\Gamma_{{\mathrm{2}}}  \vdash_\mathcal{L}  \SCnt{B}}
        \end{array}
      }
    }{\Gamma_{{\mathrm{1}}}  \SCsym{;}  \Gamma_{{\mathrm{2}}}  \vdash_\mathcal{L}  \SCnt{A}  \triangleright  \SCnt{B}}
  \end{math}
  \qquad\qquad
  $\Pi_2:$
  \begin{math}
    $$\mprset{flushleft}
    \inferrule* [right={\tiny tenL1}] {
      {
        \begin{array}{c}
          \pi_3 \\
          {\Delta_{{\mathrm{1}}}  \SCsym{;}  \SCnt{A}  \SCsym{;}  \SCnt{B}  \SCsym{;}  \Delta_{{\mathrm{2}}}  \vdash_\mathcal{L}  \SCnt{C}}
        \end{array}
      }
    }{\Delta_{{\mathrm{1}}}  \SCsym{;}  \SCnt{A}  \triangleright  \SCnt{B}  \SCsym{;}  \Delta_{{\mathrm{2}}}  \vdash_\mathcal{L}  \SCnt{C}}
  \end{math}
\end{center}
By assumption, $c(\Pi_1),c(\Pi_2)\leq |\SCnt{A}  \triangleright  \SCnt{B}| = |X|+|Y|+1$. The proof
$\Pi$ can be constructed as follows, and
$c(\Pi)\leq max\{c(\pi_1),c(\pi_2),c(\pi_3),|A|+1,|B|+1\}\leq |A|+|B|+1 = |\SCnt{A}  \triangleright  \SCnt{B}|$.
\begin{center}
  \scriptsize
  \begin{math}
    $$\mprset{flushleft}
    \inferrule* [right={\tiny cut2}] {
      {
        \begin{array}{c}
          \pi_1 \\
          {\Gamma_{{\mathrm{1}}}  \vdash_\mathcal{L}  \SCnt{A}}
        \end{array}
      }
      $$\mprset{flushleft}
      \inferrule* [right={\tiny cut2}] {
      {
        \begin{array}{cc}
          \pi_2 & \pi_3 \\
          {\Gamma_{{\mathrm{2}}}  \vdash_\mathcal{L}  \SCnt{B}} & {\Delta_{{\mathrm{1}}}  \SCsym{;}  \SCnt{A}  \SCsym{;}  \SCnt{B}  \SCsym{;}  \Delta_{{\mathrm{2}}}  \vdash_\mathcal{L}  \SCnt{C}}
        \end{array}
      }
      }{\Delta_{{\mathrm{1}}}  \SCsym{;}  \SCnt{A}  \SCsym{;}  \Gamma_{{\mathrm{2}}}  \SCsym{;}  \Delta_{{\mathrm{2}}}  \vdash_\mathcal{L}  \SCnt{C}}
    }{\Delta_{{\mathrm{1}}}  \SCsym{;}  \Gamma_{{\mathrm{1}}}  \SCsym{;}  \Gamma_{{\mathrm{2}}}  \SCsym{;}  \Psi_{{\mathrm{2}}}  \vdash_\mathcal{L}  \SCnt{C}}
  \end{math}
\end{center}

\subsubsection{The Commutative Implication $\multimap$}
\begin{center}
  \scriptsize
  $\Pi_1:$
  \begin{math}
    $$\mprset{flushleft}
    \inferrule* [right={\tiny tenR}] {
      {
        \begin{array}{c}
          \pi_1 \\
          {\Phi_{{\mathrm{1}}}  \SCsym{,}  \SCnt{X}  \vdash_\mathcal{C}  \SCnt{Y}}
        \end{array}
      }
    }{\Phi_{{\mathrm{1}}}  \vdash_\mathcal{C}  \SCnt{X}  \multimap  \SCnt{Y}}
  \end{math}
  \qquad\qquad
  $\Pi_2:$
  \begin{math}
    $$\mprset{flushleft}
    \inferrule* [right={\tiny tenL}] {
      {
        \begin{array}{cc}
          \pi_2 & \pi_3 \\
          {\Phi_{{\mathrm{2}}}  \vdash_\mathcal{C}  \SCnt{X}} & {\Psi_{{\mathrm{1}}}  \SCsym{,}  \SCnt{Y}  \SCsym{,}  \Psi_{{\mathrm{2}}}  \vdash_\mathcal{C}  \SCnt{Z}}
        \end{array}
      }
    }{\Psi_{{\mathrm{1}}}  \SCsym{,}  \SCnt{X}  \multimap  \SCnt{Y}  \SCsym{,}  \Phi  \SCsym{,}  \Psi_{{\mathrm{2}}}  \vdash_\mathcal{C}  \SCnt{Z}}
  \end{math}
\end{center}
By assumption, $c(\Pi_1),c(\Pi_2)\leq |\SCnt{X}  \multimap  \SCnt{Y}| = |X|+|Y|+1$. The proof 
$\Pi$ is constructed as follows
$c(\Pi)\leq max\{c(\pi_1),c(\pi_2),c(\pi_3),|X|+1,|Y|+1\}\leq |X|+|Y|+1 = |\SCnt{X}  \multimap  \SCnt{Y}|$.
\begin{center}
  \scriptsize
  \begin{math}
    $$\mprset{flushleft}
    \inferrule* [right={\tiny tenR}] {
      $$\mprset{flushleft}
      \inferrule* [right={\tiny tenR}] {
        {
          \begin{array}{cc}
            \pi_1 & \pi_2 \\
            {\Phi_{{\mathrm{1}}}  \SCsym{,}  \SCnt{X}  \vdash_\mathcal{C}  \SCnt{Y}} & {\Phi_{{\mathrm{2}}}  \vdash_\mathcal{C}  \SCnt{X}}
          \end{array}
        }
      }{\Phi_{{\mathrm{1}}}  \SCsym{,}  \Phi_{{\mathrm{2}}}  \vdash_\mathcal{C}  \SCnt{Y}} \\
       {
         \begin{array}{c}
           \pi_3 \\
           {\Psi_{{\mathrm{1}}}  \SCsym{,}  \SCnt{Y}  \SCsym{,}  \Psi_{{\mathrm{2}}}  \vdash_\mathcal{C}  \SCnt{Z}}
         \end{array}
       }
    }{\Psi_{{\mathrm{1}}}  \SCsym{,}  \Phi_{{\mathrm{1}}}  \SCsym{,}  \Phi_{{\mathrm{2}}}  \SCsym{,}  \Psi_{{\mathrm{2}}}  \vdash_\mathcal{C}  \SCnt{Z}}
  \end{math}
\end{center}

\subsubsection{The Non-commutative Right Implication $\lto$}
\begin{center}
  \scriptsize
  $\Pi_1:$
  \begin{math}
    $$\mprset{flushleft}
    \inferrule* [right={\tiny imprR}] {
      {
        \begin{array}{c}
          \pi_1 \\
          {\Gamma  \SCsym{;}  \SCnt{A}  \vdash_\mathcal{L}  \SCnt{B}}
        \end{array}
      }
    }{\Gamma  \vdash_\mathcal{L}  \SCnt{A}  \rightharpoonup  \SCnt{B}}
  \end{math}
  \qquad\qquad
  $\Pi_2:$
  \begin{math}
    $$\mprset{flushleft}
    \inferrule* [right={\tiny imprL}] {
      {
        \begin{array}{cc}
          \pi_2 & \pi_3 \\
          {\Delta_{{\mathrm{1}}}  \vdash_\mathcal{L}  \SCnt{A}} & {\Delta_{{\mathrm{2}}}  \SCsym{;}  \SCnt{B}  \vdash_\mathcal{L}  \SCnt{C}}
        \end{array}
      }
    }{\Delta_{{\mathrm{2}}}  \SCsym{;}  \SCnt{A}  \rightharpoonup  \SCnt{B}  \SCsym{;}  \Delta_{{\mathrm{1}}}  \vdash_\mathcal{L}  \SCnt{C}}
  \end{math}
\end{center}
By assumption, $c(\Pi_1),c(\Pi_2)\leq |\SCnt{A}  \rightharpoonup  \SCnt{B}| = |A|+|B|+1$. The proof
$\Pi$ is constructed as follows, and
$c(\Pi)\leq max\{c(\pi_1),c(\pi_2),c(\pi_3),|A|+1,|B|+1\}\leq |A|+|B|+1 = |\SCnt{A}  \rightharpoonup  \SCnt{B}|$.
\begin{center}
  \scriptsize
  \begin{math}
    $$\mprset{flushleft}
    \inferrule* [right={\tiny cut2}] {
      $$\mprset{flushleft}
      \inferrule* [right={\tiny cut2}] {
        {
          \begin{array}{cc}
            \pi_1 & \pi_2 \\
            {\Gamma  \SCsym{;}  \SCnt{A}  \vdash_\mathcal{L}  \SCnt{B}} & {\Delta_{{\mathrm{1}}}  \vdash_\mathcal{L}  \SCnt{A}}
          \end{array}
        }
      }{\Gamma  \SCsym{;}  \Delta_{{\mathrm{1}}}  \vdash_\mathcal{L}  \SCnt{B}}
       {
         \begin{array}{c}
           \pi_3 \\
           {\Delta_{{\mathrm{2}}}  \SCsym{;}  \SCnt{B}  \vdash_\mathcal{L}  \SCnt{C}}
         \end{array}
       }
    }{\Delta_{{\mathrm{2}}}  \SCsym{;}  \Gamma  \SCsym{;}  \Delta_{{\mathrm{1}}}  \vdash_\mathcal{L}  \SCnt{C}}
  \end{math}
\end{center}

\subsubsection{The Non-commutative Left Implication $\rto$}
\begin{center}
  \scriptsize
  $\Pi_1:$
  \begin{math}
    $$\mprset{flushleft}
    \inferrule* [right={\tiny implR}] {
      {
        \begin{array}{c}
          \pi_1 \\
          {\SCnt{A}  \SCsym{;}  \Gamma  \vdash_\mathcal{L}  \SCnt{B}}
        \end{array}
      }
    }{\Gamma  \vdash_\mathcal{L}  \SCnt{B}  \leftharpoonup  \SCnt{A}}
  \end{math}
  \qquad\qquad
  $\Pi_2:$
  \begin{math}
    $$\mprset{flushleft}
    \inferrule* [right={\tiny implL}] {
      {
        \begin{array}{cc}
          \pi_2 & \pi_3 \\
          {\Delta_{{\mathrm{1}}}  \vdash_\mathcal{L}  \SCnt{A}} & {\SCnt{B}  \SCsym{;}  \Delta_{{\mathrm{2}}}  \vdash_\mathcal{L}  \SCnt{C}}
        \end{array}
      }
    }{\Delta_{{\mathrm{1}}}  \SCsym{;}  \SCnt{B}  \leftharpoonup  \SCnt{A}  \SCsym{;}  \Delta_{{\mathrm{2}}}  \vdash_\mathcal{L}  \SCnt{C}}
  \end{math}
\end{center}
By assumption, $c(\Pi_1),c(\Pi_2)\leq |\SCnt{B}  \leftharpoonup  \SCnt{A}| = |A|+|B|+1$. The
proof $\Pi$ is constructed as follows, and
$c(\Pi)\leq max\{c(\pi_1),c(\pi_2),c(\pi_3),|A|+1,|B|+1\}\leq |A|+|B|+1 = |\SCnt{B}  \leftharpoonup  \SCnt{A}|$.
\begin{center}
  \scriptsize
  \begin{math}
    $$\mprset{flushleft}
    \inferrule* [right={\tiny cut1}] {
      $$\mprset{flushleft}
      \inferrule* [right={\tiny cut2}] {
        {
          \begin{array}{cc}
            \pi_1 & \pi_2 \\
            {\SCnt{A}  \SCsym{;}  \Gamma  \vdash_\mathcal{L}  \SCnt{B}} & {\Delta_{{\mathrm{1}}}  \vdash_\mathcal{L}  \SCnt{A}}
          \end{array}
        }
      }{\Delta_{{\mathrm{1}}}  \SCsym{;}  \Gamma  \vdash_\mathcal{L}  \SCnt{B}}
       {
         \begin{array}{c}
           \pi_3 \\
           {\SCnt{B}  \SCsym{;}  \Delta_{{\mathrm{2}}}  \vdash_\mathcal{L}  \SCnt{C}}
         \end{array}
       }
    }{\Delta_{{\mathrm{1}}}  \SCsym{;}  \Gamma  \SCsym{;}  \Delta_{{\mathrm{2}}}  \vdash_\mathcal{L}  \SCnt{C}}
  \end{math}
\end{center}



\subsubsection{The Commutative Unit $ \mathsf{Unit} $}
\begin{itemize}
\item Case 1:
      \begin{center}
        \scriptsize
        $\Pi_1:$
        \begin{math}
          $$\mprset{flushleft}
          \inferrule* [right={\tiny unitR}] {
            \,
          }{ \cdot   \vdash_\mathcal{C}   \mathsf{Unit} }
        \end{math}
        \qquad\qquad
        $\Pi_2:$
        \begin{math}
          $$\mprset{flushleft}
          \inferrule* [right={\tiny unitL}] {
            {
              \begin{array}{c}
                \pi \\
                {\Phi  \SCsym{,}  \Psi  \vdash_\mathcal{C}  \SCnt{X}}
              \end{array}
            }
          }{\Phi  \SCsym{,}   \mathsf{Unit}   \SCsym{,}  \Psi  \vdash_\mathcal{C}  \SCnt{X}}
        \end{math}
      \end{center}
      By assumption, $c(\Pi_1),c(\Pi_2)\leq | \mathsf{Unit} |$. The proof $\Pi$
      is the subproof $\pi$ in $\Pi_2$ for sequent $\Phi  \vdash_\mathcal{C}  \SCnt{X}$. So
      $c(\Pi)=c(\Pi_2)\leq | \mathsf{Unit} |$.

\item Case 2:
      \begin{center}
        \scriptsize
        $\Pi_1:$
        \begin{math}
          $$\mprset{flushleft}
          \inferrule* [right={\tiny unitR}] {
            \,
          }{ \cdot   \vdash_\mathcal{C}   \mathsf{Unit} }
        \end{math}
        \qquad\qquad
        $\Pi_2:$
        \begin{math}
          $$\mprset{flushleft}
          \inferrule* [right={\tiny unitL1}] {
            {
              \begin{array}{c}
                \pi \\
                {\Gamma  \SCsym{;}  \Delta  \vdash_\mathcal{L}  \SCnt{A}}
              \end{array}
            }
          }{\Gamma  \SCsym{;}   \mathsf{Unit}   \SCsym{;}  \Delta  \vdash_\mathcal{L}  \SCnt{A}}
        \end{math}
      \end{center}
      Similar as above, $\Pi$ is $\pi$.
\end{itemize}


\subsubsection{The Non-commutative Unit $ \mathsf{Unit} $}
\begin{center}
  \scriptsize
  $\Pi_1:$
  \begin{math}
    $$\mprset{flushleft}
    \inferrule* [right={\tiny unitR}] {
      \,
    }{ \cdot   \vdash_\mathcal{L}   \mathsf{Unit} }
  \end{math}
  \qquad\qquad
  $\Pi_2:$
  \begin{math}
    $$\mprset{flushleft}
    \inferrule* [right={\tiny unitL2}] {
      {
        \begin{array}{c}
          \pi \\
          {\Gamma  \SCsym{;}  \Delta  \vdash_\mathcal{L}  \SCnt{A}}
        \end{array}
      }
    }{\Gamma  \SCsym{;}   \mathsf{Unit}   \SCsym{;}  \Delta  \vdash_\mathcal{L}  \SCnt{A}}
  \end{math}
\end{center}
By assumption, $c(\Pi_1),c(\Pi_2)\leq | \mathsf{Unit} |$. The proof $\Pi$ is the
subproof $\pi$ in $\Pi_2$ for sequent $\Delta  \vdash_\mathcal{L}  \SCnt{A}$. So
$c(\Pi)=c(\Pi_2)\leq | \mathsf{Unit} |$.

\subsubsection{The Functor $F$}
\begin{center}
  \scriptsize
  $\Pi_1:$
  \begin{math}
    $$\mprset{flushleft}
    \inferrule* [right={\tiny FR}] {
      {
        \begin{array}{c}
          \pi_1 \\
          {\Phi  \vdash_\mathcal{C}  \SCnt{X}}
        \end{array}
      }
    }{\Phi  \vdash_\mathcal{L}   \mathsf{F} \SCnt{X} }
  \end{math}
  \qquad\qquad
  $\Pi_2:$
  \begin{math}
    $$\mprset{flushleft}
    \inferrule* [right={\tiny FL}] {
      {
        \begin{array}{c}
          \pi_2 \\
          {\Gamma  \SCsym{;}  \SCnt{X}  \SCsym{;}  \Delta  \vdash_\mathcal{L}  \SCnt{A}}
        \end{array}
      }
    }{\Gamma  \SCsym{;}   \mathsf{F} \SCnt{X}   \SCsym{;}  \Delta  \vdash_\mathcal{L}  \SCnt{A}}
  \end{math}
\end{center}
By assumption, $c(\Pi_1),c(\Pi_2)\leq | \mathsf{F} \SCnt{X} | = |X|+1$. The proof
$\Pi$ is constructed as follows, and \\
$c(\Pi)\leq max\{c(\pi_1),c(\pi_2),|X|+1\}\leq | \mathsf{F} \SCnt{X} |$.
\begin{center}
  \scriptsize
  \begin{math}
    $$\mprset{flushleft}
    \inferrule* [right={\tiny cut2}] {
      {
        \begin{array}{cc}
          \pi_1 & \pi_2 \\
          {\Phi  \vdash_\mathcal{C}  \SCnt{X}} & {\Gamma  \SCsym{;}  \SCnt{X}  \SCsym{;}  \Delta  \vdash_\mathcal{L}  \SCnt{A}}
        \end{array}
      }
    }{\Gamma  \SCsym{;}  \Phi  \SCsym{;}  \Delta  \vdash_\mathcal{L}  \SCnt{A}}
  \end{math}
\end{center}

\subsubsection{The Functor $G$}
\begin{center}
  \scriptsize
  $\Pi_1:$
  \begin{math}
    $$\mprset{flushleft}
    \inferrule* [right={\tiny GR}] {
      {
        \begin{array}{c}
          \pi_1 \\
          {\Phi  \vdash_\mathcal{L}  \SCnt{A}}
        \end{array}
      }
    }{\Phi  \vdash_\mathcal{C}   \mathsf{G} \SCnt{A} }
  \end{math}
  \qquad\qquad
  $\Pi_2:$
  \begin{math}
    $$\mprset{flushleft}
    \inferrule* [right={\tiny GL}] {
      {
        \begin{array}{c}
          \pi_2 \\
          {\Gamma  \SCsym{;}  \SCnt{A}  \SCsym{;}  \Delta  \vdash_\mathcal{L}  \SCnt{B}}
        \end{array}
      }
    }{\Gamma  \SCsym{;}   \mathsf{G} \SCnt{A}   \SCsym{;}  \Delta  \vdash_\mathcal{L}  \SCnt{B}}
  \end{math}
\end{center}
By assumption, $c(\Pi_1),c(\Pi_2)\leq | \mathsf{G} \SCnt{A} | = |A|+1$. The proof $\Pi$ 
is constructed as follows, and \\
$c(\Pi)\leq max\{c(\pi_1),c(\pi_2),|A|+1\}\leq | \mathsf{G} \SCnt{A} |$.
\begin{center}
  \scriptsize
  \begin{math}
    $$\mprset{flushleft}
    \inferrule* [right={\tiny GL}] {
      {
        \begin{array}{cc}
          \pi_1 & \pi_2 \\
          {\Phi  \vdash_\mathcal{L}  \SCnt{A}} & {\Gamma  \SCsym{;}  \SCnt{A}  \SCsym{;}  \Delta  \vdash_\mathcal{L}  \SCnt{B}}
        \end{array}
      }
    }{\Gamma  \SCsym{;}  \Phi  \SCsym{;}  \Delta  \vdash_\mathcal{L}  \SCnt{B}}
  \end{math}
\end{center}



\subsection{Secondary Conclusion}

\subsubsection{Left introduction of the commutative implication $\multimap$}
\begin{itemize}
\item Case 1:
      \begin{center}
        \scriptsize
        $\Pi_1$:
        \begin{math}
          $$\mprset{flushleft}
          \inferrule* [right={\tiny impL}] {
            {
              \begin{array}{cc}
                \pi_1 & \pi_2 \\
                {\Phi_{{\mathrm{1}}}  \vdash_\mathcal{C}  \SCnt{X_{{\mathrm{1}}}}} & {\Phi_{{\mathrm{2}}}  \SCsym{,}  \SCnt{X_{{\mathrm{2}}}}  \SCsym{,}  \Phi_{{\mathrm{3}}}  \vdash_\mathcal{C}  \SCnt{Y}}
              \end{array}
            }
          }{\Phi_{{\mathrm{2}}}  \SCsym{,}  \SCnt{X_{{\mathrm{1}}}}  \multimap  \SCnt{X_{{\mathrm{2}}}}  \SCsym{,}  \Phi_{{\mathrm{1}}}  \SCsym{,}  \Phi_{{\mathrm{3}}}  \vdash_\mathcal{C}  \SCnt{Y}}
        \end{math}
        \qquad\qquad
        \begin{math}
          \begin{array}{c}
            \Pi_2 \\
            {\Psi_{{\mathrm{1}}}  \SCsym{,}  \SCnt{Y}  \SCsym{,}  \Psi_{{\mathrm{2}}}  \vdash_\mathcal{C}  \SCnt{Z}}
          \end{array}
        \end{math}
      \end{center}
      By assumption, $c(\Pi_1),c(\Pi_2)\leq |Y|$. By induction, there is a
      proof $\Pi'$ from $\pi_2$ and $\Pi_2$ for sequent
      $\Psi_{{\mathrm{1}}}  \SCsym{,}  \Phi_{{\mathrm{2}}}  \SCsym{,}  \SCnt{X_{{\mathrm{2}}}}  \SCsym{,}  \Phi_{{\mathrm{3}}}  \SCsym{,}  \Psi_{{\mathrm{2}}}  \vdash_\mathcal{C}  \SCnt{Z}$ s.t. $c(\Pi')\leq |Y|$. Therefore,
      the proof $\Pi$ can be constructed as follows with $c(\Pi)\leq |Y|$.
      \begin{center}
        \scriptsize
        \begin{math}
          $$\mprset{flushleft}
          \inferrule* [right={\tiny impL}] {
            {
              \begin{array}{c}
                \pi_1 \\
                {\Phi_{{\mathrm{1}}}  \vdash_\mathcal{C}  \SCnt{X_{{\mathrm{1}}}}}
              \end{array}
            }
            $$\mprset{flushleft}
            \inferrule* [right={\tiny cut}] {
              {
                \begin{array}{cc}
                  \pi_2 & \Pi_2 \\
                  {\Phi_{{\mathrm{2}}}  \SCsym{,}  \SCnt{X_{{\mathrm{2}}}}  \SCsym{,}  \Phi_{{\mathrm{3}}}  \vdash_\mathcal{C}  \SCnt{Y}} & {\Psi_{{\mathrm{1}}}  \SCsym{,}  \SCnt{Y}  \SCsym{,}  \Psi_{{\mathrm{2}}}  \vdash_\mathcal{C}  \SCnt{Z}}
                \end{array}
              }
            }{\Psi_{{\mathrm{1}}}  \SCsym{,}  \Phi_{{\mathrm{2}}}  \SCsym{,}  \SCnt{X_{{\mathrm{2}}}}  \SCsym{,}  \Phi_{{\mathrm{3}}}  \SCsym{,}  \Psi_{{\mathrm{2}}}  \vdash_\mathcal{C}  \SCnt{Z}}
          }{\Psi_{{\mathrm{1}}}  \SCsym{,}  \Phi_{{\mathrm{2}}}  \SCsym{,}  \SCnt{X_{{\mathrm{1}}}}  \multimap  \SCnt{X_{{\mathrm{2}}}}  \SCsym{,}  \Phi_{{\mathrm{1}}}  \SCsym{,}  \Phi_{{\mathrm{3}}}  \SCsym{,}  \Psi_{{\mathrm{2}}}  \vdash_\mathcal{C}  \SCnt{Z}}
        \end{math}
      \end{center}

\item Case 2:
      \begin{center}
        \scriptsize
        $\Pi_1$:
        \begin{math}
          $$\mprset{flushleft}
          \inferrule* [right={\tiny impL}] {
            {
              \begin{array}{cc}
                \pi_1 & \pi_2 \\
                {\Phi_{{\mathrm{1}}}  \vdash_\mathcal{C}  \SCnt{X_{{\mathrm{1}}}}} & {\Phi_{{\mathrm{2}}}  \SCsym{,}  \SCnt{X_{{\mathrm{2}}}}  \SCsym{,}  \Phi_{{\mathrm{3}}}  \vdash_\mathcal{C}  \SCnt{Y}}
              \end{array}
            }
          }{\Phi_{{\mathrm{2}}}  \SCsym{,}  \SCnt{X_{{\mathrm{1}}}}  \multimap  \SCnt{X_{{\mathrm{2}}}}  \SCsym{,}  \Phi_{{\mathrm{1}}}  \SCsym{,}  \Phi_{{\mathrm{3}}}  \vdash_\mathcal{C}  \SCnt{Y}}
        \end{math}
        \qquad\qquad
        \begin{math}
          \begin{array}{c}
            \Pi_2 \\
            {\Gamma_{{\mathrm{1}}}  \SCsym{;}  \SCnt{Y}  \SCsym{;}  \Gamma_{{\mathrm{2}}}  \vdash_\mathcal{L}  \SCnt{A}}
          \end{array}
        \end{math}
      \end{center}
      By assumption, $c(\Pi_1),c(\Pi_2)\leq |Y|$. By induction, there is a
      proof $\Pi'$ from $\pi_2$ and $\Pi_2$ for sequent
      $\Gamma_{{\mathrm{1}}}  \SCsym{;}  \Phi_{{\mathrm{2}}}  \SCsym{;}  \SCnt{X_{{\mathrm{2}}}}  \SCsym{;}  \Phi_{{\mathrm{3}}}  \SCsym{;}  \Gamma_{{\mathrm{2}}}  \vdash_\mathcal{L}  \SCnt{A}$ s.t. $c(\Pi')\leq |Y|$. Therefore, the
      proof $\Pi$ can be constructed as follows with $c(\Pi)\leq |Y|$.
      \begin{center}
        \scriptsize
        \begin{math}
          $$\mprset{flushleft}
          \inferrule* [right={\tiny impL}] {
            {
              \begin{array}{c}
                \pi_1 \\
                {\Phi_{{\mathrm{1}}}  \vdash_\mathcal{C}  \SCnt{X_{{\mathrm{1}}}}}
              \end{array}
            }
            $$\mprset{flushleft}
            \inferrule* [right={\tiny cut}] {
              {
                \begin{array}{cc}
                  \pi_2 & \Pi_2 \\
                  {\Phi_{{\mathrm{2}}}  \SCsym{,}  \SCnt{X_{{\mathrm{2}}}}  \SCsym{,}  \Phi_{{\mathrm{3}}}  \vdash_\mathcal{C}  \SCnt{Y}} & {\Gamma_{{\mathrm{1}}}  \SCsym{;}  \SCnt{Y}  \SCsym{;}  \Gamma_{{\mathrm{2}}}  \vdash_\mathcal{L}  \SCnt{A}}
                \end{array}
              }
            }{\Gamma_{{\mathrm{1}}}  \SCsym{;}  \Phi_{{\mathrm{2}}}  \SCsym{;}  \SCnt{X_{{\mathrm{2}}}}  \SCsym{;}  \Phi_{{\mathrm{3}}}  \SCsym{;}  \Gamma_{{\mathrm{2}}}  \vdash_\mathcal{L}  \SCnt{A}}
          }{\Gamma_{{\mathrm{1}}}  \SCsym{;}  \Phi_{{\mathrm{2}}}  \SCsym{;}  \SCnt{X_{{\mathrm{1}}}}  \multimap  \SCnt{X_{{\mathrm{2}}}}  \SCsym{;}  \Phi_{{\mathrm{1}}}  \SCsym{;}  \Phi_{{\mathrm{3}}}  \SCsym{;}  \Gamma_{{\mathrm{2}}}  \vdash_\mathcal{L}  \SCnt{A}}
        \end{math}
      \end{center}
\end{itemize}



\subsubsection{Left introduction of the non-commutative left implication $\lto$}
\begin{center}
\scriptsize
  $\Pi_1$:
  \begin{math}
    $$\mprset{flushleft}
    \inferrule* [right={\tiny impL}] {
      {
        \begin{array}{cc}
          \pi_1 & \pi_2 \\
          {\Gamma_{{\mathrm{1}}}  \vdash_\mathcal{L}  \SCnt{A_{{\mathrm{1}}}}} & {\Gamma_{{\mathrm{2}}}  \SCsym{;}  \SCnt{A_{{\mathrm{2}}}}  \SCsym{;}  \Gamma_{{\mathrm{3}}}  \vdash_\mathcal{L}  \SCnt{B}}
        \end{array}
      }
    }{\Gamma_{{\mathrm{2}}}  \SCsym{;}  \SCnt{A_{{\mathrm{1}}}}  \rightharpoonup  \SCnt{A_{{\mathrm{2}}}}  \SCsym{;}  \Gamma_{{\mathrm{1}}}  \SCsym{;}  \Gamma_{{\mathrm{3}}}  \vdash_\mathcal{L}  \SCnt{B}}
  \end{math}
  \qquad\qquad
  \begin{math}
    \begin{array}{c}
      \Pi_2 \\
      {\Delta_{{\mathrm{1}}}  \SCsym{;}  \SCnt{B}  \SCsym{;}  \Delta_{{\mathrm{2}}}  \vdash_\mathcal{L}  \SCnt{C}}
    \end{array}
  \end{math}
\end{center}
By assumption, $c(\Pi_1),c(\Pi_2)\leq |B|$. By induction, there is a
proof $\Pi'$ from $\pi_2$ and $\Pi_2$ for sequent
$\Delta_{{\mathrm{1}}}  \SCsym{;}  \Gamma_{{\mathrm{2}}}  \SCsym{;}  \SCnt{A_{{\mathrm{2}}}}  \SCsym{;}  \Gamma_{{\mathrm{3}}}  \SCsym{;}  \Delta_{{\mathrm{2}}}  \vdash_\mathcal{L}  \SCnt{C}$ s.t. $c(\Pi')\leq |B|$.
Therefore, the proof $\Pi$ can be constructed as follows with
$c(\Pi)\leq |B|$.
\begin{center}
  \scriptsize
  \begin{math}
    $$\mprset{flushleft}
    \inferrule* [right={\tiny impL}] {
      {
        \begin{array}{c}
          \pi_1 \\
          {\Gamma_{{\mathrm{1}}}  \vdash_\mathcal{L}  \SCnt{A_{{\mathrm{1}}}}}
        \end{array}
      }
      $$\mprset{flushleft}
      \inferrule* [right={\tiny cut}] {
        {
          \begin{array}{cc}
            \pi_2 & \Pi_2 \\
            {\Gamma_{{\mathrm{2}}}  \SCsym{;}  \SCnt{A_{{\mathrm{2}}}}  \SCsym{;}  \Gamma_{{\mathrm{3}}}  \vdash_\mathcal{L}  \SCnt{B}} & {\Delta_{{\mathrm{1}}}  \SCsym{;}  \SCnt{B}  \SCsym{;}  \Delta_{{\mathrm{2}}}  \vdash_\mathcal{L}  \SCnt{C}}
          \end{array}
        }
      }{\Delta_{{\mathrm{1}}}  \SCsym{;}  \Gamma_{{\mathrm{2}}}  \SCsym{;}  \SCnt{A_{{\mathrm{2}}}}  \SCsym{;}  \Gamma_{{\mathrm{3}}}  \SCsym{;}  \Delta_{{\mathrm{2}}}  \vdash_\mathcal{L}  \SCnt{C}}
    }{\Delta_{{\mathrm{1}}}  \SCsym{;}  \Gamma_{{\mathrm{2}}}  \SCsym{;}  \SCnt{A_{{\mathrm{1}}}}  \rightharpoonup  \SCnt{A_{{\mathrm{2}}}}  \SCsym{;}  \Gamma_{{\mathrm{1}}}  \SCsym{;}  \Gamma_{{\mathrm{3}}}  \SCsym{;}  \Delta_{{\mathrm{2}}}  \vdash_\mathcal{L}  \SCnt{C}}
  \end{math}
\end{center}


\subsubsection{Left introduction of the non-commutative right implication $\rto$}
\begin{center}
  \scriptsize
  $\Pi_1$:
  \begin{math}
    $$\mprset{flushleft}
    \inferrule* [right={\tiny impL}] {
      {
        \begin{array}{cc}
          \pi_1 & \pi_2 \\
          {\Gamma_{{\mathrm{1}}}  \vdash_\mathcal{L}  \SCnt{A_{{\mathrm{1}}}}} & {\Gamma_{{\mathrm{2}}}  \SCsym{;}  \SCnt{A_{{\mathrm{2}}}}  \SCsym{;}  \Gamma_{{\mathrm{3}}}  \vdash_\mathcal{L}  \SCnt{B}}
        \end{array}
      }
    }{\Gamma_{{\mathrm{2}}}  \SCsym{;}  \Gamma_{{\mathrm{1}}}  \SCsym{;}  \SCnt{A_{{\mathrm{2}}}}  \leftharpoonup  \SCnt{A_{{\mathrm{1}}}}  \SCsym{;}  \Gamma_{{\mathrm{3}}}  \vdash_\mathcal{L}  \SCnt{B}}
  \end{math}
  \qquad\qquad
  \begin{math}
    \begin{array}{c}
      \Pi_2 \\
      {\Delta_{{\mathrm{1}}}  \SCsym{;}  \SCnt{B}  \SCsym{;}  \Delta_{{\mathrm{2}}}  \vdash_\mathcal{L}  \SCnt{C}}
    \end{array}
  \end{math}
\end{center}
By assumption, $c(\Pi_1),c(\Pi_2)\leq |B|$. By induction, there is a
proof $\Pi'$ from $\pi_2$ and $\Pi_2$ for sequent
$\Delta_{{\mathrm{1}}}  \SCsym{;}  \Gamma_{{\mathrm{2}}}  \SCsym{;}  \SCnt{A_{{\mathrm{2}}}}  \SCsym{;}  \Gamma_{{\mathrm{3}}}  \SCsym{;}  \Delta_{{\mathrm{2}}}  \vdash_\mathcal{L}  \SCnt{C}$ s.t. $c(\Pi')\leq |B|$. Therefore, the
proof $\Pi$ can be constructed as follows with $c(\Pi)\leq |B|$.
\begin{center}
  \scriptsize
  \begin{math}
    $$\mprset{flushleft}
    \inferrule* [right={\tiny impL}] {
      {
        \begin{array}{c}
          \pi_1 \\
          {\Gamma_{{\mathrm{1}}}  \vdash_\mathcal{L}  \SCnt{A_{{\mathrm{1}}}}}
        \end{array}
      }
      $$\mprset{flushleft}
      \inferrule* [right={\tiny cut}] {
        {
          \begin{array}{cc}
            \pi_2 & \Pi_2 \\
            {\Gamma_{{\mathrm{2}}}  \SCsym{;}  \SCnt{A_{{\mathrm{2}}}}  \SCsym{;}  \Gamma_{{\mathrm{3}}}  \vdash_\mathcal{L}  \SCnt{B}} & {\Delta_{{\mathrm{1}}}  \SCsym{;}  \SCnt{B}  \SCsym{;}  \Delta_{{\mathrm{2}}}  \vdash_\mathcal{L}  \SCnt{C}}
          \end{array}
        }
      }{\Delta_{{\mathrm{1}}}  \SCsym{;}  \Gamma_{{\mathrm{2}}}  \SCsym{;}  \SCnt{A_{{\mathrm{2}}}}  \SCsym{;}  \Gamma_{{\mathrm{3}}}  \SCsym{;}  \Delta_{{\mathrm{2}}}  \vdash_\mathcal{L}  \SCnt{C}}
    }{\Delta_{{\mathrm{1}}}  \SCsym{;}  \Gamma_{{\mathrm{2}}}  \SCsym{;}  \Gamma_{{\mathrm{1}}}  \SCsym{;}  \SCnt{A_{{\mathrm{2}}}}  \leftharpoonup  \SCnt{A_{{\mathrm{1}}}}  \SCsym{;}  \Gamma_{{\mathrm{3}}}  \SCsym{;}  \Delta_{{\mathrm{2}}}  \vdash_\mathcal{L}  \SCnt{C}}
  \end{math}
\end{center}

% C-ex Case 1
\subsubsection{$\SCdruleTXXexName$}
\begin{itemize}
\item Case 1:
      \begin{center}
        \scriptsize
        $\Pi_1$:
        \begin{math}
          $$\mprset{flushleft}
          \inferrule* [right={\tiny ex}] {
            {
              \begin{array}{c}
                \pi \\
                {\Phi_{{\mathrm{1}}}  \SCsym{,}  \SCnt{X_{{\mathrm{1}}}}  \SCsym{,}  \SCnt{X_{{\mathrm{2}}}}  \SCsym{,}  \Phi_{{\mathrm{2}}}  \vdash_\mathcal{C}  \SCnt{Y}}
              \end{array}
            }
          }{\Phi_{{\mathrm{1}}}  \SCsym{,}  \SCnt{X_{{\mathrm{2}}}}  \SCsym{,}  \SCnt{X_{{\mathrm{1}}}}  \SCsym{,}  \Phi_{{\mathrm{2}}}  \vdash_\mathcal{C}  \SCnt{Y}}
        \end{math}
        \qquad\qquad
        \begin{math}
          \begin{array}{c}
            \Pi_2 \\
            {\Psi_{{\mathrm{1}}}  \SCsym{,}  \SCnt{Y}  \SCsym{,}  \Psi_{{\mathrm{2}}}  \vdash_\mathcal{C}  \SCnt{Z}}
          \end{array}
        \end{math}
      \end{center}
      By assumption, $c(\Pi_1),c(\Pi_2)\leq |Y|$. By induction on $\pi$
      and $\Pi_2$, there is a proof $\Pi'$ for sequent
      $\Psi_{{\mathrm{1}}}  \SCsym{,}  \Phi_{{\mathrm{1}}}  \SCsym{,}  \SCnt{X_{{\mathrm{1}}}}  \SCsym{,}  \SCnt{X_{{\mathrm{2}}}}  \SCsym{,}  \Phi_{{\mathrm{2}}}  \SCsym{,}  \Psi_{{\mathrm{2}}}  \vdash_\mathcal{C}  \SCnt{Z}$ s.t. $c(\Pi')\leq|Y|$. Therefore,
      the proof $\Pi$ can be constructed as follows, and
      $c(\Pi)=c(\Pi')\leq|Y|$.
      \begin{center}
        \scriptsize
        \begin{math}
          $$\mprset{flushleft}
          \inferrule* [right={\tiny ex}] {
            {
              \begin{array}{c}
                \Pi' \\
                {\Psi_{{\mathrm{1}}}  \SCsym{,}  \Phi_{{\mathrm{1}}}  \SCsym{,}  \SCnt{X_{{\mathrm{1}}}}  \SCsym{,}  \SCnt{X_{{\mathrm{2}}}}  \SCsym{,}  \Phi_{{\mathrm{2}}}  \SCsym{,}  \Psi_{{\mathrm{2}}}  \vdash_\mathcal{C}  \SCnt{Z}}
              \end{array}
            }
          }{\Psi_{{\mathrm{1}}}  \SCsym{,}  \Phi_{{\mathrm{1}}}  \SCsym{,}  \SCnt{X_{{\mathrm{2}}}}  \SCsym{,}  \SCnt{X_{{\mathrm{1}}}}  \SCsym{,}  \Phi_{{\mathrm{2}}}  \SCsym{,}  \Psi_{{\mathrm{2}}}  \vdash_\mathcal{C}  \SCnt{Z}}
        \end{math}
      \end{center}

% C-ex Case 2
\item Case 2:
      \begin{center}
        \scriptsize
        $\Pi_1$:
        \begin{math}
          $$\mprset{flushleft}
          \inferrule* [right={\tiny beta}] {
            {
              \begin{array}{c}
                \pi \\
                {\Phi_{{\mathrm{1}}}  \SCsym{,}  \SCnt{X}  \SCsym{,}  \SCnt{Y}  \SCsym{,}  \Phi_{{\mathrm{2}}}  \vdash_\mathcal{C}  \SCnt{Z}}
              \end{array}
            }
          }{\Phi_{{\mathrm{1}}}  \SCsym{,}  \SCnt{Y}  \SCsym{,}  \SCnt{X}  \SCsym{,}  \Phi_{{\mathrm{2}}}  \vdash_\mathcal{C}  \SCnt{Z}}
        \end{math}
        \qquad\qquad
        \begin{math}
          \begin{array}{c}
            \Pi_2 \\
            {\Gamma_{{\mathrm{1}}}  \SCsym{;}  \SCnt{Z}  \SCsym{;}  \Gamma_{{\mathrm{2}}}  \vdash_\mathcal{L}  \SCnt{A}}
          \end{array}
        \end{math}
      \end{center}
      By assumption, $c(\Pi_1),c(\Pi_2)\leq |Z|$. Similar as above, there
      is a proof $\Pi'$ constructed from $\pi$ and $\Pi_2$ for 
      $\Gamma_{{\mathrm{1}}}  \SCsym{;}  \Phi_{{\mathrm{1}}}  \SCsym{;}  \SCnt{X}  \SCsym{;}  \SCnt{Y}  \SCsym{;}  \Phi_{{\mathrm{2}}}  \SCsym{;}  \Gamma_{{\mathrm{2}}}  \vdash_\mathcal{L}  \SCnt{A}$ s.t. $c(\Pi')\leq|Z|$. Therefore,
      the proof $\Pi$ can be constructed as follows, and
      $c(\Pi)=c(\Pi')\leq|Z|$.
      \begin{center}
        \scriptsize
        \begin{math}
          $$\mprset{flushleft}
          \inferrule* [right={\tiny beta}] {
            {
              \begin{array}{c}
                \Pi' \\
                {\Gamma_{{\mathrm{1}}}  \SCsym{;}  \Phi_{{\mathrm{1}}}  \SCsym{;}  \SCnt{X}  \SCsym{;}  \SCnt{Y}  \SCsym{;}  \Phi_{{\mathrm{2}}}  \SCsym{;}  \Gamma_{{\mathrm{2}}}  \vdash_\mathcal{L}  \SCnt{A}}
              \end{array}
            }
          }{\Gamma_{{\mathrm{1}}}  \SCsym{;}  \Phi_{{\mathrm{1}}}  \SCsym{;}  \SCnt{Y}  \SCsym{;}  \SCnt{X}  \SCsym{;}  \Phi_{{\mathrm{2}}}  \SCsym{;}  \Gamma_{{\mathrm{2}}}  \vdash_\mathcal{L}  \SCnt{A}}
        \end{math}
      \end{center}
\end{itemize}

% LC-ex
\subsubsection{$\SCdruleSXXexName$}
\begin{center}
  \scriptsize
  $\Pi_1$:
  \begin{math}
    $$\mprset{flushleft}
    \inferrule* [right={\tiny beta}] {
      {
        \begin{array}{c}
          \pi \\
          {\Gamma_{{\mathrm{1}}}  \SCsym{;}  \SCnt{X}  \SCsym{;}  \SCnt{Y}  \SCsym{;}  \Gamma_{{\mathrm{2}}}  \vdash_\mathcal{L}  \SCnt{A}}
        \end{array}
      }
    }{\Gamma_{{\mathrm{1}}}  \SCsym{;}  \SCnt{Y}  \SCsym{;}  \SCnt{X}  \SCsym{;}  \Gamma_{{\mathrm{2}}}  \vdash_\mathcal{L}  \SCnt{A}}
  \end{math}
  \qquad\qquad
  \begin{math}
    \begin{array}{c}
      \Pi_2 \\
      {\Delta_{{\mathrm{1}}}  \SCsym{;}  \SCnt{A}  \SCsym{;}  \Delta_{{\mathrm{2}}}  \vdash_\mathcal{L}  \SCnt{B}}
    \end{array}
  \end{math}
\end{center}
By assumption, $c(\Pi_1),c(\Pi_2)\leq |A|$. Similar as above, there
is a proof $\Pi'$ constructed from $\pi$ and $\Pi_2$ for sequent
$\Delta_{{\mathrm{1}}}  \SCsym{;}  \Gamma_{{\mathrm{1}}}  \SCsym{;}  \SCnt{X}  \SCsym{;}  \SCnt{Y}  \SCsym{;}  \Gamma_{{\mathrm{2}}}  \SCsym{;}  \Delta_{{\mathrm{2}}}  \vdash_\mathcal{L}  \SCnt{B}$ s.t. $c(\Pi')\leq|A|$. Therefore,
the proof $\Pi$ can be constructed as follows, and
$c(\Pi)=c(\Pi')\leq|A|$.
\begin{center}
  \scriptsize
  \begin{math}
    $$\mprset{flushleft}
    \inferrule* [right={\tiny beta}] {
      {
        \begin{array}{cc}
          \Pi' \\
          {\Delta_{{\mathrm{1}}}  \SCsym{;}  \Gamma_{{\mathrm{1}}}  \SCsym{;}  \SCnt{X}  \SCsym{;}  \SCnt{Y}  \SCsym{;}  \Gamma_{{\mathrm{2}}}  \SCsym{;}  \Delta_{{\mathrm{2}}}  \vdash_\mathcal{L}  \SCnt{B}}
        \end{array}
      }
    }{\Delta_{{\mathrm{1}}}  \SCsym{;}  \Gamma_{{\mathrm{1}}}  \SCsym{;}  \SCnt{Y}  \SCsym{;}  \SCnt{X}  \SCsym{;}  \Gamma_{{\mathrm{2}}}  \SCsym{;}  \Delta_{{\mathrm{2}}}  \vdash_\mathcal{L}  \SCnt{B}}
  \end{math}
\end{center}





\subsubsection{Left introduction of the commutative tensor product $\otimes$}
\begin{itemize}
\item Case 1:
      \begin{center}
        \scriptsize
        $\Pi_1$:
        \begin{math}
          $$\mprset{flushleft}
          \inferrule* [right={\tiny tenL}] {
            {
              \begin{array}{c}
                \pi \\
                {\Phi_{{\mathrm{1}}}  \SCsym{,}  \SCnt{X_{{\mathrm{1}}}}  \SCsym{,}  \SCnt{X_{{\mathrm{2}}}}  \SCsym{,}  \Phi_{{\mathrm{2}}}  \vdash_\mathcal{C}  \SCnt{Y}}
              \end{array}
            }
          }{\Phi_{{\mathrm{1}}}  \SCsym{,}  \SCnt{X_{{\mathrm{1}}}}  \otimes  \SCnt{X_{{\mathrm{2}}}}  \SCsym{,}  \Phi_{{\mathrm{2}}}  \vdash_\mathcal{C}  \SCnt{Y}}
        \end{math}
        \qquad\qquad
        \begin{math}
          \begin{array}{c}
            \Pi_2 \\
            {\Psi_{{\mathrm{1}}}  \SCsym{,}  \SCnt{Y}  \SCsym{,}  \Psi_{{\mathrm{2}}}  \vdash_\mathcal{C}  \SCnt{Z}}
          \end{array}
        \end{math}
      \end{center}
      By assumption, $c(\Pi_1),c(\Pi_2)\leq |Y|$. By induction, there is a
      proof $\Pi'$ from $\pi$ and $\Pi_2$ for sequent
      $\Psi_{{\mathrm{1}}}  \SCsym{,}  \Phi_{{\mathrm{1}}}  \SCsym{,}  \SCnt{X_{{\mathrm{1}}}}  \SCsym{,}  \SCnt{X_{{\mathrm{2}}}}  \SCsym{,}  \Phi_{{\mathrm{2}}}  \SCsym{,}  \Psi_{{\mathrm{2}}}  \vdash_\mathcal{C}  \SCnt{Z}$ s.t. $c(\Pi')\leq |Y|$. Therefore,
      the proof $\Pi$ can be constructed as follows with $c(\Pi)\leq |Y|$.
      \begin{center}
        \scriptsize
        \begin{math}
          $$\mprset{flushleft}
          \inferrule* [right={\tiny tenL}] {
            $$\mprset{flushleft}
            \inferrule* [right={\tiny cut}] {
              {
                \begin{array}{cc}
                  \pi & \Pi_2 \\
                  {\Phi_{{\mathrm{1}}}  \SCsym{,}  \SCnt{X_{{\mathrm{1}}}}  \SCsym{,}  \SCnt{X_{{\mathrm{2}}}}  \SCsym{,}  \Phi_{{\mathrm{2}}}  \vdash_\mathcal{C}  \SCnt{Y}} & {\Psi_{{\mathrm{1}}}  \SCsym{,}  \SCnt{Y}  \SCsym{,}  \Psi_{{\mathrm{2}}}  \vdash_\mathcal{C}  \SCnt{Z}}
                \end{array}
              }
            }{\Psi_{{\mathrm{1}}}  \SCsym{,}  \Phi_{{\mathrm{1}}}  \SCsym{,}  \SCnt{X_{{\mathrm{1}}}}  \SCsym{,}  \SCnt{X_{{\mathrm{2}}}}  \SCsym{,}  \Phi_{{\mathrm{2}}}  \SCsym{,}  \Psi_{{\mathrm{2}}}  \vdash_\mathcal{C}  \SCnt{Z}}
          }{\Psi_{{\mathrm{1}}}  \SCsym{,}  \Phi_{{\mathrm{1}}}  \SCsym{,}  \SCnt{X_{{\mathrm{1}}}}  \otimes  \SCnt{X_{{\mathrm{2}}}}  \SCsym{,}  \Phi_{{\mathrm{2}}}  \SCsym{,}  \Psi_{{\mathrm{2}}}  \vdash_\mathcal{C}  \SCnt{Z}}
        \end{math}
      \end{center}

\item Case 2:
      \begin{center}
        \scriptsize
        $\Pi_1$:
        \begin{math}
          $$\mprset{flushleft}
          \inferrule* [right={\tiny tenL}] {
            {
              \begin{array}{c}
                \pi \\
                {\Phi_{{\mathrm{1}}}  \SCsym{,}  \SCnt{X_{{\mathrm{1}}}}  \SCsym{,}  \SCnt{X_{{\mathrm{2}}}}  \SCsym{,}  \Phi_{{\mathrm{2}}}  \vdash_\mathcal{C}  \SCnt{Y}}
              \end{array}
            }
          }{\Phi_{{\mathrm{1}}}  \SCsym{,}  \SCnt{X_{{\mathrm{1}}}}  \otimes  \SCnt{X_{{\mathrm{2}}}}  \SCsym{,}  \Phi_{{\mathrm{2}}}  \vdash_\mathcal{C}  \SCnt{Y}}
        \end{math}
        \qquad\qquad
        \begin{math}
          \begin{array}{c}
            \Pi_2 \\
            {\Gamma_{{\mathrm{1}}}  \SCsym{;}  \SCnt{Y}  \SCsym{;}  \Gamma_{{\mathrm{2}}}  \vdash_\mathcal{L}  \SCnt{A}}
          \end{array}
        \end{math}
      \end{center}
      By assumption, $c(\Pi_1),c(\Pi_2)\leq |Y|$. By induction, there is a
      proof $\Pi'$ from $\pi$ and $\Pi_2$ for sequent
      $\Gamma_{{\mathrm{1}}}  \SCsym{;}  \Phi_{{\mathrm{1}}}  \SCsym{;}  \SCnt{X_{{\mathrm{1}}}}  \SCsym{;}  \SCnt{X_{{\mathrm{2}}}}  \SCsym{;}  \Phi_{{\mathrm{2}}}  \SCsym{;}  \Gamma_{{\mathrm{2}}}  \vdash_\mathcal{L}  \SCnt{A}$ s.t. $c(\Pi')\leq |Y|$. Therefore,
      the proof $\Pi$ can be constructed as follows with $c(\Pi)\leq |Y|$.
      \begin{center}
        \scriptsize
        \begin{math}
          $$\mprset{flushleft}
          \inferrule* [right={\tiny tenL1}] {
            $$\mprset{flushleft}
            \inferrule* [right={\tiny cut1}] {
              {
                \begin{array}{cc}
                  \pi & \Pi_2 \\
                  {\Phi_{{\mathrm{1}}}  \SCsym{,}  \SCnt{X_{{\mathrm{1}}}}  \SCsym{,}  \SCnt{X_{{\mathrm{2}}}}  \SCsym{,}  \Phi_{{\mathrm{2}}}  \vdash_\mathcal{C}  \SCnt{Y}} & {\Gamma_{{\mathrm{1}}}  \SCsym{;}  \SCnt{Y}  \SCsym{;}  \Gamma_{{\mathrm{2}}}  \vdash_\mathcal{L}  \SCnt{A}}
                \end{array}
              }
            }{\Gamma_{{\mathrm{1}}}  \SCsym{;}  \Phi_{{\mathrm{1}}}  \SCsym{;}  \SCnt{X_{{\mathrm{1}}}}  \SCsym{;}  \SCnt{X_{{\mathrm{2}}}}  \SCsym{;}  \Phi_{{\mathrm{2}}}  \SCsym{;}  \Gamma_{{\mathrm{2}}}  \vdash_\mathcal{L}  \SCnt{A}}
          }{\Gamma_{{\mathrm{1}}}  \SCsym{;}  \Phi_{{\mathrm{1}}}  \SCsym{;}  \SCnt{X_{{\mathrm{1}}}}  \otimes  \SCnt{X_{{\mathrm{2}}}}  \SCsym{;}  \Phi_{{\mathrm{2}}}  \SCsym{;}  \Gamma_{{\mathrm{2}}}  \vdash_\mathcal{L}  \SCnt{A}}
        \end{math}
      \end{center}

\item Case 3:
      \begin{center}
        \scriptsize
        $\Pi_1$:
        \begin{math}
          $$\mprset{flushleft}
          \inferrule* [right={\tiny tenL}] {
            {
              \begin{array}{c}
                \pi \\
                {\Gamma_{{\mathrm{1}}}  \SCsym{;}  \SCnt{X}  \SCsym{;}  \SCnt{Y}  \SCsym{;}  \Gamma_{{\mathrm{2}}}  \vdash_\mathcal{L}  \SCnt{A}}
              \end{array}
            }
          }{\Gamma_{{\mathrm{1}}}  \SCsym{;}  \SCnt{X}  \otimes  \SCnt{Y}  \SCsym{;}  \Gamma_{{\mathrm{2}}}  \vdash_\mathcal{L}  \SCnt{A}}
        \end{math}
        \qquad\qquad
        \begin{math}
          \begin{array}{c}
            \Pi_2 \\
            {\Delta_{{\mathrm{1}}}  \SCsym{;}  \SCnt{A}  \SCsym{;}  \Delta_{{\mathrm{2}}}  \vdash_\mathcal{L}  \SCnt{B}}
          \end{array}
        \end{math}
      \end{center}
      By assumption, $c(\Pi_1),c(\Pi_2)\leq |A|$. By induction, there is a
      proof $\Pi'$ from $\pi$ and $\Pi_2$ for sequent
      $\Delta_{{\mathrm{1}}}  \SCsym{;}  \SCnt{X}  \SCsym{;}  \SCnt{Y}  \SCsym{;}  \Gamma_{{\mathrm{2}}}  \SCsym{;}  \Delta_{{\mathrm{2}}}  \vdash_\mathcal{L}  \SCnt{B}$ s.t. $c(\Pi')\leq |A|$. Therefore, the
      proof $\Pi$ can be constructed as follows with $c(\Pi)\leq |A|$.
      \begin{center}
        \scriptsize
        \begin{math}
          $$\mprset{flushleft}
          \inferrule* [right={\tiny tenL1}] {
            $$\mprset{flushleft}
            \inferrule* [right={\tiny cut2}] {
              {
                \begin{array}{cc}
                  \pi & \Pi_2 \\
                  {\Gamma_{{\mathrm{1}}}  \SCsym{;}  \SCnt{X}  \SCsym{;}  \SCnt{Y}  \SCsym{;}  \Gamma_{{\mathrm{2}}}  \vdash_\mathcal{L}  \SCnt{A}} & {\Delta_{{\mathrm{1}}}  \SCsym{;}  \SCnt{A}  \SCsym{;}  \Delta_{{\mathrm{2}}}  \vdash_\mathcal{L}  \SCnt{B}}
                \end{array}
              }
            }{\Delta_{{\mathrm{1}}}  \SCsym{;}  \Gamma_{{\mathrm{1}}}  \SCsym{;}  \SCnt{X}  \SCsym{;}  \SCnt{Y}  \SCsym{;}  \Gamma_{{\mathrm{2}}}  \SCsym{;}  \Delta_{{\mathrm{2}}}  \vdash_\mathcal{L}  \SCnt{B}}
          }{\Delta_{{\mathrm{1}}}  \SCsym{;}  \Gamma_{{\mathrm{1}}}  \SCsym{;}  \SCnt{X}  \otimes  \SCnt{Y}  \SCsym{;}  \Gamma_{{\mathrm{2}}}  \SCsym{;}  \Delta_{{\mathrm{2}}}  \vdash_\mathcal{L}  \SCnt{B}}
        \end{math}
      \end{center}
\end{itemize}

\subsubsection{Left introduction of the non-commutative tensor products $\tri$}
\begin{center}
  \scriptsize
  $\Pi_1$:
  \begin{math}
    $$\mprset{flushleft}
    \inferrule* [right={\tiny tenL2}] {
      {
        \begin{array}{c}
          \pi \\
          {\Gamma_{{\mathrm{1}}}  \SCsym{;}  \SCnt{A_{{\mathrm{1}}}}  \SCsym{;}  \SCnt{A_{{\mathrm{2}}}}  \SCsym{;}  \Gamma_{{\mathrm{2}}}  \vdash_\mathcal{L}  \SCnt{B}}
        \end{array}
      }
    }{\Gamma_{{\mathrm{1}}}  \SCsym{;}  \SCnt{A_{{\mathrm{1}}}}  \triangleright  \SCnt{A_{{\mathrm{2}}}}  \SCsym{;}  \Gamma_{{\mathrm{2}}}  \vdash_\mathcal{L}  \SCnt{B}}
  \end{math}
  \qquad\qquad
  \begin{math}
    \begin{array}{c}
      \Pi_2 \\
      {\Delta_{{\mathrm{1}}}  \SCsym{;}  \SCnt{B}  \SCsym{;}  \Delta_{{\mathrm{2}}}  \vdash_\mathcal{L}  \SCnt{C}}
    \end{array}
  \end{math}
\end{center}
By assumption, $c(\Pi_1),c(\Pi_2)\leq |B|$. By induction, there is a
proof $\Pi'$ from $\pi$ and $\Pi_2$ for sequent \\
$\Delta_{{\mathrm{1}}}  \SCsym{;}  \Gamma_{{\mathrm{1}}}  \SCsym{;}  \SCnt{A_{{\mathrm{1}}}}  \SCsym{;}  \SCnt{A_{{\mathrm{2}}}}  \SCsym{;}  \Gamma_{{\mathrm{2}}}  \SCsym{;}  \Delta_{{\mathrm{2}}}  \vdash_\mathcal{L}  \SCnt{C}$ s.t. $c(\Pi')\leq |B|$.
Therefore, the proof $\Pi$ can be constructed as follows with
$c(\Pi)\leq |B|$.
\begin{center}
  \scriptsize
  \begin{math}
    $$\mprset{flushleft}
    \inferrule* [right={\tiny tenL2}] {
      $$\mprset{flushleft}
      \inferrule* [right={\tiny cut2}] {
        {
          \begin{array}{cc}
            \pi & \Pi_2 \\
            {\Gamma_{{\mathrm{1}}}  \SCsym{;}  \SCnt{A_{{\mathrm{1}}}}  \SCsym{;}  \SCnt{A_{{\mathrm{2}}}}  \SCsym{;}  \Gamma_{{\mathrm{2}}}  \vdash_\mathcal{L}  \SCnt{B}} & {\Delta_{{\mathrm{1}}}  \SCsym{;}  \SCnt{B}  \SCsym{;}  \Delta_{{\mathrm{2}}}  \vdash_\mathcal{L}  \SCnt{C}}
          \end{array}
        }
      }{\Delta_{{\mathrm{1}}}  \SCsym{;}  \Gamma_{{\mathrm{1}}}  \SCsym{;}  \SCnt{A_{{\mathrm{1}}}}  \SCsym{;}  \SCnt{A_{{\mathrm{2}}}}  \SCsym{;}  \Gamma_{{\mathrm{2}}}  \SCsym{;}  \Delta_{{\mathrm{2}}}  \vdash_\mathcal{L}  \SCnt{C}}
    }{\Delta_{{\mathrm{1}}}  \SCsym{;}  \Gamma_{{\mathrm{1}}}  \SCsym{;}  \SCnt{A_{{\mathrm{1}}}}  \triangleright  \SCnt{A_{{\mathrm{2}}}}  \SCsym{;}  \Gamma_{{\mathrm{2}}}  \SCsym{;}  \Delta_{{\mathrm{2}}}  \vdash_\mathcal{L}  \SCnt{C}}
  \end{math}
\end{center}



\subsubsection{Left introduction of the commutative unit $ \mathsf{Unit} $}
\begin{itemize}
\item Case 1:
      \begin{center}
        \scriptsize
        $\Pi_1$:
        \begin{math}
          $$\mprset{flushleft}
          \inferrule* [right={\tiny unitL}] {
            {
              \begin{array}{c}
                \pi \\
                {\Phi_{{\mathrm{1}}}  \SCsym{,}  \Phi_{{\mathrm{2}}}  \vdash_\mathcal{C}  \SCnt{X}}
              \end{array}
            }
          }{\Phi_{{\mathrm{1}}}  \SCsym{,}   \mathsf{Unit}   \SCsym{,}  \Phi_{{\mathrm{2}}}  \vdash_\mathcal{C}  \SCnt{X}}
        \end{math}
        \qquad\qquad
        \begin{math}
          \begin{array}{c}
            \Pi_2 \\
            {\Psi_{{\mathrm{1}}}  \SCsym{,}  \SCnt{X}  \SCsym{,}  \Psi_{{\mathrm{2}}}  \vdash_\mathcal{C}  \SCnt{Y}}
          \end{array}
        \end{math}
      \end{center}
      By assumption, $c(\Pi_1),c(\Pi_2)\leq |X|$. By induction, there is a
      proof $\Pi'$ from $\pi$ and $\Pi_2$ for sequent
      $\Psi_{{\mathrm{1}}}  \SCsym{,}  \Phi_{{\mathrm{1}}}  \SCsym{,}  \Phi_{{\mathrm{2}}}  \SCsym{,}  \Psi_{{\mathrm{2}}}  \vdash_\mathcal{C}  \SCnt{Y}$
      s.t. $c(\Pi')\leq |X|$. Therefore, the proof $\Pi$ can be constructed
      as follows, and $c(\Pi)=c(\Pi')\leq |X|$.
      \begin{center}
        \scriptsize
        \begin{math}
          $$\mprset{flushleft}
          \inferrule* [right={\tiny unitL}] {
            {
              \begin{array}{c}
                \Pi' \\
                {\Psi_{{\mathrm{1}}}  \SCsym{,}  \Phi_{{\mathrm{1}}}  \SCsym{,}  \Phi_{{\mathrm{2}}}  \SCsym{,}  \Psi_{{\mathrm{2}}}  \vdash_\mathcal{C}  \SCnt{Y}}
              \end{array}
            }
          }{\Psi_{{\mathrm{1}}}  \SCsym{,}  \Phi_{{\mathrm{1}}}  \SCsym{,}   \mathsf{Unit}   \SCsym{,}  \Phi_{{\mathrm{2}}}  \SCsym{,}  \Psi_{{\mathrm{2}}}  \vdash_\mathcal{C}  \SCnt{Y}}
        \end{math}
      \end{center}

\item Case 2:
      \begin{center}
        \scriptsize
        $\Pi_1$:
        \begin{math}
          $$\mprset{flushleft}
          \inferrule* [right={\tiny unitL}] {
            {
              \begin{array}{c}
                \pi \\
                {\Phi_{{\mathrm{1}}}  \SCsym{,}  \Phi_{{\mathrm{2}}}  \vdash_\mathcal{C}  \SCnt{X}}
              \end{array}
            }
          }{\Phi_{{\mathrm{1}}}  \SCsym{,}   \mathsf{Unit}   \SCsym{,}  \Phi_{{\mathrm{2}}}  \vdash_\mathcal{C}  \SCnt{X}}
        \end{math}
        \qquad\qquad
        \begin{math}
          \begin{array}{c}
            \Pi_2 \\
            {\Gamma_{{\mathrm{1}}}  \SCsym{;}  \SCnt{X}  \SCsym{;}  \Gamma_{{\mathrm{2}}}  \vdash_\mathcal{L}  \SCnt{A}}
          \end{array}
        \end{math}
      \end{center}
      By assumption, $c(\Pi_1),c(\Pi_2)\leq |X|$. By induction, there is a
      proof $\Pi'$ from $\pi$ and $\Pi_2$ for sequent
      $\Gamma_{{\mathrm{1}}}  \SCsym{;}  \Phi_{{\mathrm{1}}}  \SCsym{;}  \Phi_{{\mathrm{2}}}  \SCsym{;}  \Gamma_{{\mathrm{2}}}  \vdash_\mathcal{L}  \SCnt{A}$
      s.t. $c(\Pi')\leq |X|$. Therefore, the proof $\Pi$ can be constructed
      as follows, and $c(\Pi)=c(\Pi')\leq |X|$.
      \begin{center}
        \scriptsize
        \begin{math}
          $$\mprset{flushleft}
          \inferrule* [right={\tiny unitL}] {
            {
              \begin{array}{c}
                \Pi' \\
                {\Gamma_{{\mathrm{1}}}  \SCsym{;}  \Phi_{{\mathrm{1}}}  \SCsym{;}  \Phi_{{\mathrm{2}}}  \SCsym{;}  \Gamma_{{\mathrm{2}}}  \vdash_\mathcal{L}  \SCnt{A}}
              \end{array}
            }
          }{\Gamma_{{\mathrm{1}}}  \SCsym{;}  \Phi_{{\mathrm{1}}}  \SCsym{;}   \mathsf{Unit}   \SCsym{;}  \Phi_{{\mathrm{2}}}  \SCsym{;}  \Gamma_{{\mathrm{2}}}  \vdash_\mathcal{L}  \SCnt{A}}
        \end{math}
      \end{center}

\item Case 3:
      \begin{center}
        \scriptsize
        $\Pi_1$:
        \begin{math}
          $$\mprset{flushleft}
          \inferrule* [right={\tiny unitL}] {
            {
              \begin{array}{c}
                \pi \\
                {\Delta_{{\mathrm{1}}}  \SCsym{;}  \Delta_{{\mathrm{2}}}  \vdash_\mathcal{L}  \SCnt{A}}
              \end{array}
            }
          }{\Delta_{{\mathrm{1}}}  \SCsym{;}   \mathsf{Unit}   \SCsym{;}  \Delta_{{\mathrm{2}}}  \vdash_\mathcal{L}  \SCnt{A}}
        \end{math}
        \qquad\qquad
        \begin{math}
          \begin{array}{c}
            \Pi_2 \\
            {\Gamma_{{\mathrm{1}}}  \SCsym{;}  \SCnt{A}  \SCsym{;}  \Gamma_{{\mathrm{2}}}  \vdash_\mathcal{L}  \SCnt{B}}
          \end{array}
        \end{math}
      \end{center}
      By assumption, $c(\Pi_1),c(\Pi_2)\leq |X|$. By induction, there is a
      proof $\Pi'$ from $\pi$ and $\Pi_2$ for sequent
      $\Gamma_{{\mathrm{1}}}  \SCsym{;}  \Delta_{{\mathrm{1}}}  \SCsym{;}  \Delta_{{\mathrm{2}}}  \SCsym{;}  \Gamma_{{\mathrm{2}}}  \vdash_\mathcal{L}  \SCnt{B}$
      s.t. $c(\Pi')\leq |X|$. Therefore, the proof $\Pi$ can be constructed
      as follows, and $c(\Pi)=c(\Pi')\leq |X|$.
      \begin{center}
        \scriptsize
        \begin{math}
          $$\mprset{flushleft}
          \inferrule* [right={\tiny unitL}] {
            {
              \begin{array}{c}
                \Pi' \\
                {\Gamma_{{\mathrm{1}}}  \SCsym{;}  \Delta_{{\mathrm{1}}}  \SCsym{;}  \Delta_{{\mathrm{2}}}  \SCsym{;}  \Gamma_{{\mathrm{2}}}  \vdash_\mathcal{L}  \SCnt{B}}
              \end{array}
            }
          }{\Gamma_{{\mathrm{1}}}  \SCsym{;}  \Delta_{{\mathrm{1}}}  \SCsym{;}   \mathsf{Unit}   \SCsym{;}  \Delta_{{\mathrm{2}}}  \SCsym{;}  \Gamma_{{\mathrm{2}}}  \vdash_\mathcal{L}  \SCnt{B}}
        \end{math}
      \end{center}
\end{itemize}



\subsubsection{Left introduction of the non-commutative unit $ \mathsf{Unit} $}
\begin{center}
  \scriptsize
  $\Pi_1$:
  \begin{math}
    $$\mprset{flushleft}
    \inferrule* [right={\tiny unitL}] {
      {
        \begin{array}{c}
          \pi \\
          {\Delta_{{\mathrm{1}}}  \SCsym{;}  \Delta_{{\mathrm{2}}}  \vdash_\mathcal{L}  \SCnt{A}}
        \end{array}
      }
    }{\Delta_{{\mathrm{1}}}  \SCsym{;}   \mathsf{Unit}   \SCsym{;}  \Delta_{{\mathrm{2}}}  \vdash_\mathcal{L}  \SCnt{A}}
  \end{math}
  \qquad\qquad
  \begin{math}
    \begin{array}{c}
      \Pi_2 \\
      {\Gamma_{{\mathrm{1}}}  \SCsym{;}  \SCnt{A}  \SCsym{;}  \Gamma_{{\mathrm{2}}}  \vdash_\mathcal{L}  \SCnt{B}}
    \end{array}
  \end{math}
\end{center}
By assumption, $c(\Pi_1),c(\Pi_2)\leq |X|$. By induction, there is a
proof $\Pi'$ from $\pi$ and $\Pi_2$ for sequent
$\Gamma_{{\mathrm{1}}}  \SCsym{;}  \Delta_{{\mathrm{1}}}  \SCsym{;}  \Delta_{{\mathrm{2}}}  \SCsym{;}  \Gamma_{{\mathrm{2}}}  \vdash_\mathcal{L}  \SCnt{B}$
s.t. $c(\Pi')\leq |X|$. Therefore, the proof $\Pi$ can be constructed
as follows, and $c(\Pi)=c(\Pi')\leq |X|$.
\begin{center}
  \scriptsize
  \begin{math}
    $$\mprset{flushleft}
    \inferrule* [right={\tiny unitL}] {
      {
        \begin{array}{c}
          \Pi' \\
          {\Gamma_{{\mathrm{1}}}  \SCsym{;}  \Delta_{{\mathrm{1}}}  \SCsym{;}  \Delta_{{\mathrm{2}}}  \SCsym{;}  \Gamma_{{\mathrm{2}}}  \vdash_\mathcal{L}  \SCnt{B}}
        \end{array}
      }
    }{\Gamma_{{\mathrm{1}}}  \SCsym{;}  \Delta_{{\mathrm{1}}}  \SCsym{;}   \mathsf{Unit}   \SCsym{;}  \Delta_{{\mathrm{2}}}  \SCsym{;}  \Gamma_{{\mathrm{2}}}  \vdash_\mathcal{L}  \SCnt{B}}
  \end{math}
\end{center}



\subsubsection{Left introduction of the functor $F$}
\begin{center}
  \scriptsize
  $\Pi_1$:
  \begin{math}
    $$\mprset{flushleft}
    \inferrule* [right={\tiny FL}] {
      {
        \begin{array}{c}
          \pi_1 \\
          {\Gamma_{{\mathrm{1}}}  \SCsym{;}  \SCnt{X}  \SCsym{;}  \Gamma_{{\mathrm{2}}}  \vdash_\mathcal{L}  \SCnt{A}}
        \end{array}
      }
    }{\Gamma_{{\mathrm{1}}}  \SCsym{;}   \mathsf{F} \SCnt{X}   \SCsym{;}  \Gamma_{{\mathrm{2}}}  \vdash_\mathcal{L}  \SCnt{A}}
  \end{math}
  \qquad\qquad
  \begin{math}
    \begin{array}{c}
      \Pi_2 \\
      {\Delta_{{\mathrm{1}}}  \SCsym{;}  \SCnt{A}  \SCsym{;}  \Delta_{{\mathrm{2}}}  \vdash_\mathcal{L}  \SCnt{B}}
    \end{array}
  \end{math}
\end{center}
By assumption, $c(\Pi_1),c(\Pi_2)\leq |A|$. By induction, there is a
proof $\Pi'$ from $\pi_2$ and $\Pi_2$ for sequent
$\Delta_{{\mathrm{1}}}  \SCsym{;}  \Gamma_{{\mathrm{1}}}  \SCsym{;}  \SCnt{X}  \SCsym{;}  \Gamma_{{\mathrm{2}}}  \SCsym{;}  \Delta_{{\mathrm{2}}}  \vdash_\mathcal{L}  \SCnt{B}$ s.t. $c(\Pi')\leq |A|$. Therefore, the
proof $\Pi$ can be constructed as follows with $c(\Pi)\leq |A|$.
\begin{center}
  \scriptsize
  \begin{math}
    $$\mprset{flushleft}
    \inferrule* [right={\tiny FL}] {
      $$\mprset{flushleft}
      \inferrule* [right={\tiny cut2}] {
        {
          \begin{array}{cc}
            \pi_2 & \Pi_2 \\
            {\Gamma_{{\mathrm{1}}}  \SCsym{;}  \SCnt{X}  \SCsym{;}  \Gamma_{{\mathrm{2}}}  \vdash_\mathcal{L}  \SCnt{A}} & {\Delta_{{\mathrm{1}}}  \SCsym{;}  \SCnt{A}  \SCsym{;}  \Delta_{{\mathrm{2}}}  \vdash_\mathcal{L}  \SCnt{B}}
          \end{array}
        }
      }{\Delta_{{\mathrm{1}}}  \SCsym{;}  \Gamma_{{\mathrm{1}}}  \SCsym{;}  \SCnt{X}  \SCsym{;}  \Gamma_{{\mathrm{2}}}  \SCsym{;}  \Delta_{{\mathrm{2}}}  \vdash_\mathcal{L}  \SCnt{B}}
    }{\Delta_{{\mathrm{1}}}  \SCsym{;}  \Gamma_{{\mathrm{1}}}  \SCsym{;}   \mathsf{F} \SCnt{X}   \SCsym{;}  \Gamma_{{\mathrm{2}}}  \SCsym{;}  \Delta_{{\mathrm{2}}}  \vdash_\mathcal{L}  \SCnt{B}}
  \end{math}
\end{center}

\subsubsection{Left introduction of the functor $G$}
\begin{center}
  \scriptsize
  $\Pi_1$:
  \begin{math}
    $$\mprset{flushleft}
    \inferrule* [right={\tiny GL}] {
      {
        \begin{array}{c}
          \pi_1 \\
          {\Gamma_{{\mathrm{1}}}  \SCsym{;}  \SCnt{A}  \SCsym{;}  \Gamma_{{\mathrm{2}}}  \vdash_\mathcal{L}  \SCnt{B}}
        \end{array}
      }
    }{\Gamma_{{\mathrm{1}}}  \SCsym{;}   \mathsf{G} \SCnt{A}   \SCsym{;}  \Gamma_{{\mathrm{2}}}  \vdash_\mathcal{L}  \SCnt{B}}
  \end{math}
  \qquad\qquad
  \begin{math}
    \begin{array}{c}
      \Pi_2 \\
      {\Delta_{{\mathrm{1}}}  \SCsym{;}  \SCnt{B}  \SCsym{;}  \Delta_{{\mathrm{2}}}  \vdash_\mathcal{L}  \SCnt{C}}
    \end{array}
  \end{math}
\end{center}
By assumption, $c(\Pi_1),c(\Pi_2)\leq |B|$. By induction, there is a
proof $\Pi'$ from $\pi_2$ and $\Pi_2$ for sequent
$\Delta_{{\mathrm{1}}}  \SCsym{;}  \Gamma_{{\mathrm{1}}}  \SCsym{;}  \SCnt{A}  \SCsym{;}  \Gamma_{{\mathrm{2}}}  \SCsym{;}  \Delta_{{\mathrm{2}}}  \vdash_\mathcal{L}  \SCnt{C}$ s.t. $c(\Pi')\leq |B|$. Therefore, the
proof $\Pi$ can be constructed as follows with $c(\Pi)\leq |B|$.
\begin{center}
  \scriptsize
  \begin{math}
    $$\mprset{flushleft}
    \inferrule* [right={\tiny GL}] {
      $$\mprset{flushleft}
      \inferrule* [right={\tiny cut2}] {
        {
          \begin{array}{cc}
            \pi_2 & \Pi_2 \\
            {\Gamma_{{\mathrm{1}}}  \SCsym{;}  \SCnt{A}  \SCsym{;}  \Gamma_{{\mathrm{2}}}  \vdash_\mathcal{L}  \SCnt{B}} & {\Delta_{{\mathrm{1}}}  \SCsym{;}  \SCnt{B}  \SCsym{;}  \Delta_{{\mathrm{2}}}  \vdash_\mathcal{L}  \SCnt{C}}
          \end{array}
        }
      }{\Delta_{{\mathrm{1}}}  \SCsym{;}  \Gamma_{{\mathrm{1}}}  \SCsym{;}  \SCnt{A}  \SCsym{;}  \Gamma_{{\mathrm{2}}}  \SCsym{;}  \Delta_{{\mathrm{2}}}  \vdash_\mathcal{L}  \SCnt{C}}
    }{\Delta_{{\mathrm{1}}}  \SCsym{;}  \Gamma_{{\mathrm{1}}}  \SCsym{;}   \mathsf{G} \SCnt{A}   \SCsym{;}  \Gamma_{{\mathrm{2}}}  \SCsym{;}  \Delta_{{\mathrm{2}}}  \vdash_\mathcal{L}  \SCnt{C}}
  \end{math}
\end{center}



\subsection{Secondary Hypothesis}

\subsubsection{Right introduction of the commutative tensor product $\otimes$}
\begin{itemize}
\item Case 1:
      \begin{center}
        \scriptsize
        \begin{math}
          \begin{array}{c}
            \Pi_1 \\
            {\Phi_{{\mathrm{2}}}  \vdash_\mathcal{C}  \SCnt{X}}
          \end{array}
        \end{math}
        \qquad\qquad
        $\Pi_2$:
        \begin{math}
          $$\mprset{flushleft}
          \inferrule* [right={\tiny tenR}] {
            {
              \begin{array}{cc}
                \pi_1 & \pi_2 \\
                {\Psi_{{\mathrm{1}}}  \SCsym{,}  \SCnt{X}  \SCsym{,}  \Psi_{{\mathrm{2}}}  \vdash_\mathcal{C}  \SCnt{Y_{{\mathrm{1}}}}} & {\Phi_{{\mathrm{1}}}  \vdash_\mathcal{C}  \SCnt{Y_{{\mathrm{2}}}}}
              \end{array}
            }
          }{\Psi_{{\mathrm{1}}}  \SCsym{,}  \SCnt{X}  \SCsym{,}  \Psi_{{\mathrm{2}}}  \SCsym{,}  \Phi_{{\mathrm{1}}}  \vdash_\mathcal{C}  \SCnt{Y_{{\mathrm{1}}}}  \otimes  \SCnt{Y_{{\mathrm{2}}}}}
        \end{math}
      \end{center}
      By assumption, $c(\Pi_1),c(\Pi_2)\leq |X|$. By induction on $\Pi_1$
      and $\pi_1$, there is a proof $\Pi'$ for sequent
      $\Psi_{{\mathrm{1}}}  \SCsym{,}  \Phi_{{\mathrm{2}}}  \SCsym{,}  \Psi_{{\mathrm{2}}}  \vdash_\mathcal{C}  \SCnt{Y_{{\mathrm{1}}}}$ s.t. $c(\Pi') \leq |X|$. Therefore, the proof
      $\Pi$ can be constructed as follows with $c(\Pi) = c(\Pi') \leq |X|$.
      \begin{center}
        \scriptsize
        \begin{math}
          $$\mprset{flushleft}
          \inferrule* [right={\tiny tenR}] {
            {
              \begin{array}{cc}
                \Pi' & \pi_1 \\
                {\Psi_{{\mathrm{1}}}  \SCsym{,}  \Phi_{{\mathrm{2}}}  \SCsym{,}  \Psi_{{\mathrm{2}}}  \vdash_\mathcal{C}  \SCnt{Y_{{\mathrm{1}}}}} & {\Phi_{{\mathrm{1}}}  \vdash_\mathcal{C}  \SCnt{Y_{{\mathrm{2}}}}}
              \end{array}
            }
          }{\Psi_{{\mathrm{1}}}  \SCsym{,}  \Phi_{{\mathrm{2}}}  \SCsym{,}  \Psi_{{\mathrm{2}}}  \SCsym{,}  \Phi_{{\mathrm{1}}}  \vdash_\mathcal{C}  \SCnt{Y_{{\mathrm{1}}}}  \otimes  \SCnt{Y_{{\mathrm{2}}}}}
        \end{math}
      \end{center}

\item Case 2:
      \begin{center}
        \scriptsize
        \begin{math}
          \begin{array}{c}
            \Pi_1 \\
            {\Phi_{{\mathrm{2}}}  \vdash_\mathcal{C}  \SCnt{X}}
          \end{array}
        \end{math}
        \qquad\qquad
        $\Pi_2$:
        \begin{math}
          $$\mprset{flushleft}
          \inferrule* [right={\tiny tenR}] {
            {
              \begin{array}{cc}
                \pi_1 & \pi_2 \\
                {\Phi_{{\mathrm{1}}}  \vdash_\mathcal{C}  \SCnt{Y_{{\mathrm{1}}}}} & {\Psi_{{\mathrm{1}}}  \SCsym{,}  \SCnt{X}  \SCsym{,}  \Psi_{{\mathrm{2}}}  \vdash_\mathcal{C}  \SCnt{Y_{{\mathrm{2}}}}}
              \end{array}
            }
          }{\Phi_{{\mathrm{1}}}  \SCsym{,}  \Psi_{{\mathrm{1}}}  \SCsym{,}  \SCnt{X}  \SCsym{,}  \Psi_{{\mathrm{2}}}  \vdash_\mathcal{C}  \SCnt{Y_{{\mathrm{1}}}}  \otimes  \SCnt{Y_{{\mathrm{2}}}}}
        \end{math}
      \end{center}
      By assumption, $c(\Pi_1),c(\Pi_2)\leq |X|$. By induction on $\Pi_1$
      and $\pi_2$, there is a proof $\Pi'$ for sequent
      $\Psi_{{\mathrm{1}}}  \SCsym{,}  \Phi_{{\mathrm{2}}}  \SCsym{,}  \Psi_{{\mathrm{2}}}  \vdash_\mathcal{C}  \SCnt{Y_{{\mathrm{2}}}}$ s.t. $c(\Pi') \leq |X|$. Therefore, the proof
      $\Pi$ can be constructed as follows with $c(\Pi) = c(\Pi') \leq |X|$.
      \begin{center}
        \scriptsize
        \begin{math}
          $$\mprset{flushleft}
          \inferrule* [right={\tiny tenR}] {
            {
              \begin{array}{cc}
                \pi_1 & \Pi' \\
                {\Phi_{{\mathrm{1}}}  \vdash_\mathcal{C}  \SCnt{Y_{{\mathrm{1}}}}} & {\Psi_{{\mathrm{1}}}  \SCsym{,}  \Phi_{{\mathrm{2}}}  \SCsym{,}  \Psi_{{\mathrm{2}}}  \vdash_\mathcal{C}  \SCnt{Y_{{\mathrm{2}}}}}
              \end{array}
            }
          }{\Phi_{{\mathrm{1}}}  \SCsym{,}  \Psi_{{\mathrm{1}}}  \SCsym{,}  \Phi_{{\mathrm{2}}}  \SCsym{,}  \Psi_{{\mathrm{2}}}  \vdash_\mathcal{C}  \SCnt{Y_{{\mathrm{1}}}}  \otimes  \SCnt{Y_{{\mathrm{2}}}}}
        \end{math}
      \end{center}
\end{itemize}



\subsubsection{Right introduction of the non-commutative tensor product $\tri$}
\begin{itemize}
\item Case 1:
      \begin{center}
        \scriptsize
        \begin{math}
          \begin{array}{c}
            \Pi_1 \\
            {\Phi  \vdash_\mathcal{C}  \SCnt{X}}
          \end{array}
        \end{math}
        \qquad\qquad
        $\Pi_2$:
        \begin{math}
          $$\mprset{flushleft}
          \inferrule* [right={\tiny tenR}] {
            {
              \begin{array}{cc}
                \pi_1 & \pi_2 \\
                {\Gamma_{{\mathrm{1}}}  \SCsym{;}  \SCnt{X}  \SCsym{;}  \Gamma_{{\mathrm{2}}}  \vdash_\mathcal{L}  \SCnt{A}} & {\Gamma_{{\mathrm{3}}}  \vdash_\mathcal{L}  \SCnt{B}}
              \end{array}
            }
          }{\Gamma_{{\mathrm{1}}}  \SCsym{;}  \SCnt{X}  \SCsym{;}  \Gamma_{{\mathrm{2}}}  \SCsym{;}  \Gamma_{{\mathrm{3}}}  \vdash_\mathcal{L}  \SCnt{A}  \triangleright  \SCnt{B}}
        \end{math}
      \end{center}
      By assumption, $c(\Pi_1),c(\Pi_2)\leq |X|$. By induction on $\Pi_1$
      and $\pi_1$, there is a proof $\Pi'$ for sequent
      $\Gamma_{{\mathrm{1}}}  \SCsym{;}  \Phi  \SCsym{;}  \Gamma_{{\mathrm{2}}}  \vdash_\mathcal{L}  \SCnt{A}$ s.t. $c(\Pi') \leq |X|$. Therefore, the proof
      $\Pi$ can be constructed as follows with $c(\Pi) = c(\Pi') \leq |X|$.
      \begin{center}
        \scriptsize
        \begin{math}
          $$\mprset{flushleft}
          \inferrule* [right={\tiny tenR}] {
            {
              \begin{array}{cc}
                \Pi' & \pi_1 \\
                {\Gamma_{{\mathrm{1}}}  \SCsym{;}  \Phi  \SCsym{;}  \Gamma_{{\mathrm{2}}}  \vdash_\mathcal{L}  \SCnt{A}} & {\Gamma_{{\mathrm{3}}}  \vdash_\mathcal{L}  \SCnt{B}}
              \end{array}
            }
          }{\Gamma_{{\mathrm{1}}}  \SCsym{;}  \Phi  \SCsym{;}  \Gamma_{{\mathrm{2}}}  \SCsym{;}  \Gamma_{{\mathrm{3}}}  \vdash_\mathcal{L}  \SCnt{A}  \triangleright  \SCnt{B}}
        \end{math}
      \end{center}

\item Case 2:
      \begin{center}
        \scriptsize
        \begin{math}
          \begin{array}{c}
            \Pi_1 \\
            {\Delta  \vdash_\mathcal{L}  \SCnt{C}}
          \end{array}
        \end{math}
        \qquad\qquad
        $\Pi_2$:
        \begin{math}
          $$\mprset{flushleft}
          \inferrule* [right={\tiny tenR}] {
            {
              \begin{array}{cc}
                \pi_1 & \pi_2 \\
                {\Gamma_{{\mathrm{1}}}  \SCsym{;}  \SCnt{C}  \SCsym{;}  \Gamma_{{\mathrm{2}}}  \vdash_\mathcal{L}  \SCnt{A}} & {\Gamma_{{\mathrm{3}}}  \vdash_\mathcal{L}  \SCnt{B}}
              \end{array}
            }
          }{\Gamma_{{\mathrm{1}}}  \SCsym{;}  \SCnt{C}  \SCsym{;}  \Gamma_{{\mathrm{2}}}  \SCsym{;}  \Gamma_{{\mathrm{3}}}  \vdash_\mathcal{L}  \SCnt{A}  \triangleright  \SCnt{B}}
        \end{math}
      \end{center}
      By assumption, $c(\Pi_1),c(\Pi_2)\leq |C|$. By induction on $\Pi_1$
      and $\pi_1$, there is a proof $\Pi'$ for sequent
      $\Gamma_{{\mathrm{1}}}  \SCsym{;}  \Delta  \SCsym{;}  \Gamma_{{\mathrm{2}}}  \vdash_\mathcal{L}  \SCnt{A}$ s.t. $c(\Pi') \leq |C|$. Therefore, the proof
      $\Pi$ can be constructed as follows with $c(\Pi) = c(\Pi') \leq |C|$.
      \begin{center}
        \scriptsize
        \begin{math}
          $$\mprset{flushleft}
          \inferrule* [right={\tiny tenR}] {
            {
              \begin{array}{cc}
                \Pi' & \pi_1 \\
                {\Gamma_{{\mathrm{1}}}  \SCsym{;}  \Delta  \SCsym{;}  \Gamma_{{\mathrm{2}}}  \vdash_\mathcal{L}  \SCnt{A}} & {\Gamma_{{\mathrm{3}}}  \vdash_\mathcal{L}  \SCnt{B}}
              \end{array}
            }
          }{\Gamma_{{\mathrm{1}}}  \SCsym{;}  \Delta  \SCsym{;}  \Gamma_{{\mathrm{2}}}  \SCsym{;}  \Gamma_{{\mathrm{3}}}  \vdash_\mathcal{L}  \SCnt{A}  \triangleright  \SCnt{B}}
        \end{math}
      \end{center}

\item Case 3:
      \begin{center}
        \scriptsize
        \begin{math}
          \begin{array}{c}
            \Pi_1 \\
            {\Phi  \vdash_\mathcal{C}  \SCnt{X}}
          \end{array}
        \end{math}
        \qquad\qquad
        $\Pi_2$:
        \begin{math}
          $$\mprset{flushleft}
          \inferrule* [right={\tiny tenR}] {
            {
              \begin{array}{cc}
                \pi_1 & \pi_2 \\
                {\Gamma_{{\mathrm{1}}}  \vdash_\mathcal{L}  \SCnt{A}} & {\Gamma_{{\mathrm{2}}}  \SCsym{;}  \SCnt{X}  \SCsym{;}  \Gamma_{{\mathrm{3}}}  \vdash_\mathcal{L}  \SCnt{B}}
              \end{array}
            }
          }{\Gamma_{{\mathrm{1}}}  \SCsym{;}  \Gamma_{{\mathrm{2}}}  \SCsym{;}  \SCnt{X}  \SCsym{;}  \Gamma_{{\mathrm{3}}}  \vdash_\mathcal{L}  \SCnt{A}  \triangleright  \SCnt{B}}
        \end{math}
      \end{center}
      By assumption, $c(\Pi_1),c(\Pi_2)\leq |X|$. By induction on $\Pi_1$
      and $\pi_2$, there is a proof $\Pi'$ for sequent
      $\Gamma_{{\mathrm{2}}}  \SCsym{;}  \Phi  \SCsym{;}  \Gamma_{{\mathrm{3}}}  \vdash_\mathcal{L}  \SCnt{B}$ s.t. $c(\Pi') \leq |X|$. Therefore, the proof
      $\Pi$ can be constructed as follows with $c(\Pi) = c(\Pi') \leq |X|$.
      \begin{center}
        \scriptsize
        \begin{math}
          $$\mprset{flushleft}
          \inferrule* [right={\tiny tenR}] {
            {
              \begin{array}{cc}
                \pi_1 & \Pi' \\
                {\Gamma_{{\mathrm{1}}}  \vdash_\mathcal{L}  \SCnt{A}} & {\Gamma_{{\mathrm{2}}}  \SCsym{;}  \Phi  \SCsym{;}  \Gamma_{{\mathrm{3}}}  \vdash_\mathcal{L}  \SCnt{B}}
              \end{array}
            }
          }{\Gamma_{{\mathrm{1}}}  \SCsym{;}  \Gamma_{{\mathrm{2}}}  \SCsym{;}  \Phi  \SCsym{;}  \Gamma_{{\mathrm{3}}}  \vdash_\mathcal{L}  \SCnt{A}  \triangleright  \SCnt{B}}
        \end{math}
      \end{center}

\item Case 4:
      \begin{center}
        \scriptsize
        \begin{math}
          \begin{array}{c}
            \Pi_1 \\
            {\Delta  \vdash_\mathcal{L}  \SCnt{C}}
          \end{array}
        \end{math}
        \qquad\qquad
        $\Pi_2$:
        \begin{math}
          $$\mprset{flushleft}
          \inferrule* [right={\tiny tenR}] {
            {
              \begin{array}{cc}
                \pi_1 & \pi_2 \\
                {\Gamma_{{\mathrm{1}}}  \vdash_\mathcal{L}  \SCnt{A}} & {\Gamma_{{\mathrm{2}}}  \SCsym{;}  \SCnt{C}  \SCsym{;}  \Gamma_{{\mathrm{3}}}  \vdash_\mathcal{L}  \SCnt{B}}
              \end{array}
            }
          }{\Gamma_{{\mathrm{1}}}  \SCsym{;}  \Gamma_{{\mathrm{2}}}  \SCsym{;}  \SCnt{C}  \SCsym{;}  \Gamma_{{\mathrm{3}}}  \vdash_\mathcal{L}  \SCnt{A}  \triangleright  \SCnt{B}}
        \end{math}
      \end{center}
      By assumption, $c(\Pi_1),c(\Pi_2)\leq |C|$. By induction on $\Pi_1$
      and $\pi_2$, there is a proof $\Pi'$ for sequent
      $\Gamma_{{\mathrm{2}}}  \SCsym{;}  \Delta  \SCsym{;}  \Gamma_{{\mathrm{3}}}  \vdash_\mathcal{L}  \SCnt{B}$ s.t. $c(\Pi') \leq |C|$. Therefore, the proof
      $\Pi$ can be constructed as follows with $c(\Pi) = c(\Pi') \leq |C|$.
      \begin{center}
        \scriptsize
        \begin{math}
          $$\mprset{flushleft}
          \inferrule* [right={\tiny tenR}] {
            {
              \begin{array}{cc}
                \pi_1 & \Pi' \\
                {\Gamma_{{\mathrm{1}}}  \vdash_\mathcal{L}  \SCnt{A}} & {\Gamma_{{\mathrm{2}}}  \SCsym{;}  \Delta  \SCsym{;}  \Gamma_{{\mathrm{3}}}  \vdash_\mathcal{L}  \SCnt{B}}
              \end{array}
            }
          }{\Gamma_{{\mathrm{1}}}  \SCsym{;}  \Gamma_{{\mathrm{2}}}  \SCsym{;}  \Delta  \SCsym{;}  \Gamma_{{\mathrm{3}}}  \vdash_\mathcal{L}  \SCnt{A}  \triangleright  \SCnt{B}}
        \end{math}
      \end{center}
\end{itemize}



\subsubsection{Left introduction of the commutative implication $\multimap$}
\begin{itemize}
\item Case 1:
      \begin{center}
        \scriptsize
        \begin{math}
          \begin{array}{c}
            \Pi_1 \\
            {\Phi  \vdash_\mathcal{C}  \SCnt{X}}
          \end{array}
        \end{math}
        \qquad\qquad
        $\Pi_2$:
        \begin{math}
          $$\mprset{flushleft}
          \inferrule* [right={\tiny impL}] {
            {
              \begin{array}{cc}
                \pi_1 & \pi_2 \\
                {\Psi_{{\mathrm{2}}}  \SCsym{,}  \SCnt{X}  \SCsym{,}  \Psi_{{\mathrm{3}}}  \vdash_\mathcal{C}  \SCnt{Y_{{\mathrm{1}}}}} & {\Psi_{{\mathrm{1}}}  \SCsym{,}  \SCnt{Y_{{\mathrm{2}}}}  \SCsym{,}  \Psi_{{\mathrm{4}}}  \vdash_\mathcal{C}  \SCnt{Z}}
              \end{array}
            }
          }{\Psi_{{\mathrm{1}}}  \SCsym{,}  \SCnt{Y_{{\mathrm{1}}}}  \multimap  \SCnt{Y_{{\mathrm{2}}}}  \SCsym{,}  \Psi_{{\mathrm{2}}}  \SCsym{,}  \SCnt{X}  \SCsym{,}  \Psi_{{\mathrm{3}}}  \SCsym{,}  \Psi_{{\mathrm{4}}}  \vdash_\mathcal{C}  \SCnt{Z}}
        \end{math}
      \end{center}
      By assumption, $c(\Pi_1),c(\Pi_2)\leq |X|$. By induction on $\Pi_1$ and $\pi_1$, there is
      a proof $\Pi'$ for sequent $\Psi_{{\mathrm{2}}}  \SCsym{,}  \Phi  \SCsym{,}  \Psi_{{\mathrm{3}}}  \vdash_\mathcal{C}  \SCnt{Y_{{\mathrm{1}}}}$ s.t. $c(\Pi') \leq |X|$. Therefore, the
      proof $\Pi$ can be constructed as follows with $c(\Pi) = c(\Pi') \leq |X|$.
      \begin{center}
        \scriptsize
        \begin{math}
          $$\mprset{flushleft}
          \inferrule* [right={\tiny impL}] {
            {
              \begin{array}{cc}
                \Pi' & \pi_2 \\
                {\Psi_{{\mathrm{2}}}  \SCsym{,}  \Phi  \SCsym{,}  \Psi_{{\mathrm{3}}}  \vdash_\mathcal{C}  \SCnt{Y_{{\mathrm{1}}}}} & {\Psi_{{\mathrm{1}}}  \SCsym{,}  \SCnt{Y_{{\mathrm{2}}}}  \SCsym{,}  \Psi_{{\mathrm{4}}}  \vdash_\mathcal{C}  \SCnt{Z}}
              \end{array}
            }
          }{\Psi_{{\mathrm{1}}}  \SCsym{,}  \SCnt{Y_{{\mathrm{1}}}}  \multimap  \SCnt{Y_{{\mathrm{2}}}}  \SCsym{,}  \Psi_{{\mathrm{2}}}  \SCsym{,}  \Phi  \SCsym{,}  \Psi_{{\mathrm{3}}}  \SCsym{,}  \Psi_{{\mathrm{4}}}  \vdash_\mathcal{C}  \SCnt{Z}}
        \end{math}
      \end{center}

\item Case 2:
      \begin{center}
        \scriptsize
        \begin{math}
          \begin{array}{c}
            \Pi_1 \\
            {\Phi  \vdash_\mathcal{C}  \SCnt{X}}
          \end{array}
        \end{math}
        \qquad\qquad
        $\Pi_2$:
        \begin{math}
          $$\mprset{flushleft}
          \inferrule* [right={\tiny impL}] {
            {
              \begin{array}{cc}
                \pi_1 & \pi_2 \\
                {\Psi_{{\mathrm{3}}}  \vdash_\mathcal{C}  \SCnt{Y_{{\mathrm{1}}}}} & {\Psi_{{\mathrm{1}}}  \SCsym{,}  \SCnt{X}  \SCsym{,}  \Psi_{{\mathrm{2}}}  \SCsym{,}  \SCnt{Y_{{\mathrm{2}}}}  \SCsym{,}  \Psi_{{\mathrm{4}}}  \vdash_\mathcal{C}  \SCnt{Z}}
              \end{array}
            }
          }{\Psi_{{\mathrm{1}}}  \SCsym{,}  \SCnt{X}  \SCsym{,}  \Psi_{{\mathrm{2}}}  \SCsym{,}  \SCnt{Y_{{\mathrm{1}}}}  \multimap  \SCnt{Y_{{\mathrm{2}}}}  \SCsym{,}  \Psi_{{\mathrm{3}}}  \SCsym{,}  \Psi_{{\mathrm{4}}}  \vdash_\mathcal{C}  \SCnt{Z}}
        \end{math}
      \end{center}
      By assumption, $c(\Pi_1),c(\Pi_2)\leq |X|$. By induction on $\Pi_1$ and $\pi_2$, there is
      a proof $\Pi'$ for sequent $\Psi_{{\mathrm{1}}}  \SCsym{,}  \Phi  \SCsym{,}  \Psi_{{\mathrm{2}}}  \SCsym{,}  \SCnt{Y_{{\mathrm{2}}}}  \SCsym{,}  \Psi_{{\mathrm{4}}}  \vdash_\mathcal{C}  \SCnt{Z}$ s.t. $c(\Pi') \leq |X|$.
      Therefore, the proof $\Pi$ can be constructed as follows with
      $c(\Pi) = c(\Pi') \leq |X|$.
      \begin{center}
        \scriptsize
        \begin{math}
          $$\mprset{flushleft}
          \inferrule* [right={\tiny impL}] {
            {
              \begin{array}{cc}
                \pi_1 & \Pi' \\
                {\Psi_{{\mathrm{3}}}  \vdash_\mathcal{C}  \SCnt{Y_{{\mathrm{1}}}}} & {\Psi_{{\mathrm{1}}}  \SCsym{,}  \Phi  \SCsym{,}  \Psi_{{\mathrm{2}}}  \SCsym{,}  \SCnt{Y_{{\mathrm{2}}}}  \SCsym{,}  \Psi_{{\mathrm{4}}}  \vdash_\mathcal{C}  \SCnt{Z}}
              \end{array}
            }
          }{\Psi_{{\mathrm{1}}}  \SCsym{,}  \Phi_{{\mathrm{1}}}  \SCsym{,}  \Psi_{{\mathrm{2}}}  \SCsym{,}  \SCnt{Y_{{\mathrm{1}}}}  \multimap  \SCnt{Y_{{\mathrm{2}}}}  \SCsym{,}  \Psi_{{\mathrm{3}}}  \SCsym{,}  \Psi_{{\mathrm{4}}}  \vdash_\mathcal{C}  \SCnt{Z}}
        \end{math}
      \end{center}

\item Case 3:
      \begin{center}
        \scriptsize
        \begin{math}
          \begin{array}{c}
            \Pi_1 \\
            {\Phi  \vdash_\mathcal{C}  \SCnt{X}}
          \end{array}
        \end{math}
        \qquad\qquad
        $\Pi_2$:
        \begin{math}
          $$\mprset{flushleft}
          \inferrule* [right={\tiny impL}] {
            {
              \begin{array}{cc}
                \pi_1 & \pi_2 \\
                {\Psi_{{\mathrm{2}}}  \vdash_\mathcal{C}  \SCnt{Y_{{\mathrm{1}}}}} & {\Psi_{{\mathrm{1}}}  \SCsym{,}  \SCnt{Y_{{\mathrm{2}}}}  \SCsym{,}  \Psi_{{\mathrm{3}}}  \SCsym{,}  \SCnt{X}  \SCsym{,}  \Psi_{{\mathrm{4}}}  \vdash_\mathcal{C}  \SCnt{Z}}
              \end{array}
            }
          }{\Psi_{{\mathrm{1}}}  \SCsym{,}  \SCnt{Y_{{\mathrm{1}}}}  \multimap  \SCnt{Y_{{\mathrm{2}}}}  \SCsym{,}  \Psi_{{\mathrm{2}}}  \SCsym{,}  \Psi_{{\mathrm{3}}}  \SCsym{,}  \SCnt{X}  \SCsym{,}  \Psi_{{\mathrm{4}}}  \vdash_\mathcal{C}  \SCnt{Z}}
        \end{math}
      \end{center}
      By assumption, $c(\Pi_1),c(\Pi_2)\leq |X|$. By induction on $\Pi_1$
      and $\pi_2$, there is a proof $\Pi'$ for sequent
      $\Psi_{{\mathrm{1}}}  \SCsym{,}  \Phi  \SCsym{,}  \Psi_{{\mathrm{2}}}  \SCsym{,}  \SCnt{Y_{{\mathrm{2}}}}  \SCsym{,}  \Psi_{{\mathrm{4}}}  \vdash_\mathcal{C}  \SCnt{Z}$ s.t. $c(\Pi') \leq |X|$. Therefore,
      the proof $\Pi$ can be constructed as follows with
      $c(\Pi) = c(\Pi') \leq |X|$.
      \begin{center}
        \scriptsize
        \begin{math}
          $$\mprset{flushleft}
          \inferrule* [right={\tiny impL}] {
            {
              \begin{array}{cc}
                \pi_1 & \Pi' \\
                {\Psi_{{\mathrm{2}}}  \vdash_\mathcal{C}  \SCnt{Y_{{\mathrm{1}}}}} & {\Psi_{{\mathrm{1}}}  \SCsym{,}  \SCnt{Y_{{\mathrm{2}}}}  \SCsym{,}  \Psi_{{\mathrm{3}}}  \SCsym{,}  \Phi  \SCsym{,}  \Psi_{{\mathrm{4}}}  \vdash_\mathcal{C}  \SCnt{Z}}
              \end{array}
            }
          }{\Psi_{{\mathrm{1}}}  \SCsym{,}  \SCnt{Y_{{\mathrm{1}}}}  \multimap  \SCnt{Y_{{\mathrm{2}}}}  \SCsym{,}  \Psi_{{\mathrm{2}}}  \SCsym{,}  \Psi_{{\mathrm{3}}}  \SCsym{,}  \Phi  \SCsym{,}  \Psi_{{\mathrm{4}}}  \vdash_\mathcal{C}  \SCnt{Z}}
        \end{math}
      \end{center}

\item Case 4:
      \begin{center}
        \scriptsize
        \begin{math}
          \begin{array}{c}
            \Pi_1 \\
            {\Phi  \vdash_\mathcal{C}  \SCnt{X}}
          \end{array}
        \end{math}
        \qquad\qquad
        $\Pi_2$:
        \begin{math}
          $$\mprset{flushleft}
          \inferrule* [right={\tiny impL}] {
            {
              \begin{array}{cc}
                \pi_1 & \pi_2 \\
                {\Psi_{{\mathrm{1}}}  \SCsym{,}  \SCnt{X}  \SCsym{,}  \Psi_{{\mathrm{2}}}  \vdash_\mathcal{C}  \SCnt{Y_{{\mathrm{1}}}}} & {\Gamma_{{\mathrm{1}}}  \SCsym{;}  \SCnt{Y_{{\mathrm{2}}}}  \SCsym{;}  \Gamma_{{\mathrm{2}}}  \vdash_\mathcal{L}  \SCnt{A}}
              \end{array}
            }
          }{\Gamma_{{\mathrm{1}}}  \SCsym{;}  \SCnt{Y_{{\mathrm{1}}}}  \multimap  \SCnt{Y_{{\mathrm{2}}}}  \SCsym{;}  \Psi_{{\mathrm{1}}}  \SCsym{;}  \SCnt{X}  \SCsym{;}  \Psi_{{\mathrm{2}}}  \SCsym{;}  \Gamma_{{\mathrm{2}}}  \vdash_\mathcal{L}  \SCnt{A}}
        \end{math}
      \end{center}
      By assumption, $c(\Pi_1),c(\Pi_2)\leq |X|$. By induction on $\Pi_1$
      and $\pi_1$, there is a proof $\Pi'$ for sequent
      $\Psi_{{\mathrm{1}}}  \SCsym{,}  \Phi  \SCsym{,}  \Psi_{{\mathrm{2}}}  \vdash_\mathcal{C}  \SCnt{Y_{{\mathrm{1}}}}$ s.t. $c(\Pi') \leq |X|$. Therefore, the proof
      $\Pi$ can be constructed as follows with $c(\Pi) = c(\Pi') \leq |X|$.
      \begin{center}
        \scriptsize
        \begin{math}
          $$\mprset{flushleft}
          \inferrule* [right={\tiny impL}] {
            {
              \begin{array}{cc}
                \Pi' & \pi_2 \\
                {\Psi_{{\mathrm{1}}}  \SCsym{,}  \Phi  \SCsym{,}  \Psi_{{\mathrm{2}}}  \vdash_\mathcal{C}  \SCnt{Y_{{\mathrm{1}}}}} & {\Gamma_{{\mathrm{1}}}  \SCsym{;}  \SCnt{Y_{{\mathrm{2}}}}  \SCsym{;}  \Gamma_{{\mathrm{2}}}  \vdash_\mathcal{L}  \SCnt{A}}
              \end{array}
            }
          }{\Gamma_{{\mathrm{1}}}  \SCsym{;}  \SCnt{Y_{{\mathrm{1}}}}  \multimap  \SCnt{Y_{{\mathrm{2}}}}  \SCsym{;}  \Psi_{{\mathrm{1}}}  \SCsym{;}  \Phi  \SCsym{;}  \Psi_{{\mathrm{2}}}  \SCsym{;}  \Gamma_{{\mathrm{2}}}  \vdash_\mathcal{L}  \SCnt{A}}
        \end{math}
      \end{center}

\item Case 5:
      \begin{center}
        \scriptsize
        \begin{math}
          \begin{array}{c}
            \Pi_1 \\
            {\Phi  \vdash_\mathcal{C}  \SCnt{X}}
          \end{array}
        \end{math}
        \qquad\qquad
        $\Pi_2$:
        \begin{math}
          $$\mprset{flushleft}
          \inferrule* [right={\tiny impL}] {
            {
              \begin{array}{cc}
                \pi_1 & \pi_2 \\
                {\Psi  \vdash_\mathcal{C}  \SCnt{Y_{{\mathrm{1}}}}} & {\Gamma_{{\mathrm{1}}}  \SCsym{;}  \SCnt{X}  \SCsym{;}  \Gamma_{{\mathrm{2}}}  \SCsym{;}  \SCnt{Y_{{\mathrm{2}}}}  \SCsym{;}  \Gamma_{{\mathrm{3}}}  \vdash_\mathcal{L}  \SCnt{A}}
              \end{array}
            }
          }{\Gamma_{{\mathrm{1}}}  \SCsym{;}  \SCnt{X}  \SCsym{;}  \Gamma_{{\mathrm{2}}}  \SCsym{;}  \SCnt{Y_{{\mathrm{1}}}}  \multimap  \SCnt{Y_{{\mathrm{2}}}}  \SCsym{;}  \Psi  \SCsym{;}  \Gamma_{{\mathrm{3}}}  \vdash_\mathcal{L}  \SCnt{A}}
        \end{math}
      \end{center}
      By assumption, $c(\Pi_1),c(\Pi_2)\leq |X|$. By induction on $\Pi_1$
      and $\pi_2$, there is a proof $\Pi'$ for sequent
      $\Gamma_{{\mathrm{1}}}  \SCsym{;}  \Phi  \SCsym{;}  \Gamma_{{\mathrm{2}}}  \SCsym{;}  \SCnt{Y_{{\mathrm{2}}}}  \SCsym{;}  \Gamma_{{\mathrm{3}}}  \vdash_\mathcal{L}  \SCnt{A}$ s.t. $c(\Pi') \leq |X|$. Therefore, the
      proof $\Pi$ can be constructed as follows with
      $c(\Pi) = c(\Pi') \leq |X|$.
      \begin{center}
        \scriptsize
        \begin{math}
          $$\mprset{flushleft}
          \inferrule* [right={\tiny impL}] {
            {
              \begin{array}{cc}
                \pi_1 & \Pi' \\
                {\Psi  \vdash_\mathcal{C}  \SCnt{Y_{{\mathrm{1}}}}} & {\Gamma_{{\mathrm{1}}}  \SCsym{;}  \Phi  \SCsym{;}  \Gamma_{{\mathrm{2}}}  \SCsym{;}  \SCnt{Y_{{\mathrm{2}}}}  \SCsym{;}  \Gamma_{{\mathrm{3}}}  \vdash_\mathcal{L}  \SCnt{A}}
              \end{array}
            }
          }{\Gamma_{{\mathrm{1}}}  \SCsym{;}  \Phi  \SCsym{;}  \Gamma_{{\mathrm{2}}}  \SCsym{;}  \SCnt{Y_{{\mathrm{1}}}}  \multimap  \SCnt{Y_{{\mathrm{2}}}}  \SCsym{;}  \Psi  \SCsym{;}  \Gamma_{{\mathrm{3}}}  \vdash_\mathcal{L}  \SCnt{A}}
        \end{math}
      \end{center}

\item Case 6:
      \begin{center}
        \scriptsize
        \begin{math}
          \begin{array}{c}
            \Pi_1 \\
            {\Delta  \vdash_\mathcal{L}  \SCnt{B}}
          \end{array}
        \end{math}
        \qquad\qquad
        $\Pi_2$:
        \begin{math}
          $$\mprset{flushleft}
          \inferrule* [right={\tiny impL}] {
            {
              \begin{array}{cc}
                \pi_1 & \pi_2 \\
                {\Psi  \vdash_\mathcal{C}  \SCnt{Y_{{\mathrm{1}}}}} & {\Gamma_{{\mathrm{1}}}  \SCsym{;}  \SCnt{B}  \SCsym{;}  \Gamma_{{\mathrm{2}}}  \SCsym{;}  \SCnt{Y_{{\mathrm{2}}}}  \SCsym{;}  \Gamma_{{\mathrm{3}}}  \vdash_\mathcal{L}  \SCnt{A}}
              \end{array}
            }
          }{\Gamma_{{\mathrm{1}}}  \SCsym{;}  \SCnt{B}  \SCsym{;}  \Gamma_{{\mathrm{2}}}  \SCsym{;}  \SCnt{Y_{{\mathrm{1}}}}  \multimap  \SCnt{Y_{{\mathrm{2}}}}  \SCsym{;}  \Psi  \SCsym{;}  \Gamma_{{\mathrm{3}}}  \vdash_\mathcal{L}  \SCnt{A}}
        \end{math}
      \end{center}
      By assumption, $c(\Pi_1),c(\Pi_2)\leq |B|$. By induction on $\Pi_1$
      and $\pi_2$, there is a proof $\Pi'$ for sequent
      $\Gamma_{{\mathrm{1}}}  \SCsym{;}  \Delta  \SCsym{;}  \Gamma_{{\mathrm{2}}}  \SCsym{;}  \SCnt{Y_{{\mathrm{2}}}}  \SCsym{;}  \Gamma_{{\mathrm{3}}}  \vdash_\mathcal{L}  \SCnt{A}$ s.t. $c(\Pi') \leq |B|$. Therefore, the
      proof $\Pi$ can be constructed as follows with
      $c(\Pi) = c(\Pi') \leq |B|$.
      \begin{center}
        \scriptsize
        \begin{math}
          $$\mprset{flushleft}
          \inferrule* [right={\tiny impL}] {
            {
              \begin{array}{cc}
                \pi_1 & \Pi' \\
                {\Psi  \vdash_\mathcal{C}  \SCnt{Y_{{\mathrm{1}}}}} & {\Gamma_{{\mathrm{1}}}  \SCsym{;}  \Delta  \SCsym{;}  \Gamma_{{\mathrm{2}}}  \SCsym{;}  \SCnt{Y_{{\mathrm{2}}}}  \SCsym{;}  \Gamma_{{\mathrm{3}}}  \vdash_\mathcal{L}  \SCnt{A}}
              \end{array}
            }
          }{\Gamma_{{\mathrm{1}}}  \SCsym{;}  \Delta  \SCsym{;}  \Gamma_{{\mathrm{2}}}  \SCsym{;}  \SCnt{Y_{{\mathrm{1}}}}  \multimap  \SCnt{Y_{{\mathrm{2}}}}  \SCsym{;}  \Psi  \SCsym{;}  \Gamma_{{\mathrm{3}}}  \vdash_\mathcal{L}  \SCnt{A}}
        \end{math}
      \end{center}

\item Case 7:
      \begin{center}
        \scriptsize
        \begin{math}
          \begin{array}{c}
            \Pi_1 \\
            {\Phi  \vdash_\mathcal{C}  \SCnt{X}}
          \end{array}
        \end{math}
        \qquad\qquad
        $\Pi_2$:
        \begin{math}
          $$\mprset{flushleft}
          \inferrule* [right={\tiny impL}] {
            {
              \begin{array}{cc}
                \pi_1 & \pi_2 \\
                {\Psi  \vdash_\mathcal{C}  \SCnt{Y_{{\mathrm{1}}}}} & {\Gamma_{{\mathrm{1}}}  \SCsym{;}  \SCnt{Y_{{\mathrm{2}}}}  \SCsym{;}  \Gamma_{{\mathrm{2}}}  \SCsym{;}  \SCnt{X}  \SCsym{;}  \Gamma_{{\mathrm{3}}}  \vdash_\mathcal{L}  \SCnt{A}}
              \end{array}
            }
          }{\Gamma_{{\mathrm{1}}}  \SCsym{;}  \SCnt{Y_{{\mathrm{1}}}}  \multimap  \SCnt{Y_{{\mathrm{2}}}}  \SCsym{;}  \Psi  \SCsym{;}  \Gamma_{{\mathrm{2}}}  \SCsym{;}  \SCnt{X}  \SCsym{;}  \Gamma_{{\mathrm{3}}}  \vdash_\mathcal{L}  \SCnt{A}}
        \end{math}
      \end{center}
      By assumption, $c(\Pi_1),c(\Pi_2)\leq |X|$. By induction on $\Pi_1$
      and $\pi_2$, there is a proof $\Pi'$ for sequent
      $\Gamma_{{\mathrm{1}}}  \SCsym{;}  \SCnt{Y_{{\mathrm{2}}}}  \SCsym{;}  \Gamma_{{\mathrm{2}}}  \SCsym{;}  \Phi  \SCsym{;}  \Gamma_{{\mathrm{3}}}  \vdash_\mathcal{L}  \SCnt{A}$ s.t. $c(\Pi') \leq |X|$. Therefore, the
      proof $\Pi$ can be constructed as follows with
      $c(\Pi) = c(\Pi') \leq |X|$.
      \begin{center}
        \scriptsize
        \begin{math}
          $$\mprset{flushleft}
          \inferrule* [right={\tiny impL}] {
            {
              \begin{array}{cc}
                \pi_1 & \Pi' \\
                {\Psi  \vdash_\mathcal{C}  \SCnt{Y_{{\mathrm{1}}}}} & {\Gamma_{{\mathrm{1}}}  \SCsym{;}  \SCnt{Y_{{\mathrm{2}}}}  \SCsym{;}  \Gamma_{{\mathrm{2}}}  \SCsym{;}  \Phi  \SCsym{;}  \Gamma_{{\mathrm{3}}}  \vdash_\mathcal{L}  \SCnt{A}}
              \end{array}
            }
          }{\Gamma_{{\mathrm{1}}}  \SCsym{;}  \SCnt{Y_{{\mathrm{1}}}}  \multimap  \SCnt{Y_{{\mathrm{2}}}}  \SCsym{;}  \Psi  \SCsym{;}  \Gamma_{{\mathrm{2}}}  \SCsym{;}  \Phi  \SCsym{;}  \Gamma_{{\mathrm{3}}}  \vdash_\mathcal{L}  \SCnt{A}}
        \end{math}
      \end{center}

\item Case 8:
      \begin{center}
        \scriptsize
        \begin{math}
          \begin{array}{c}
            \Pi_1 \\
            {\Delta  \vdash_\mathcal{L}  \SCnt{B}}
          \end{array}
        \end{math}
        \qquad\qquad
        $\Pi_2$:
        \begin{math}
          $$\mprset{flushleft}
          \inferrule* [right={\tiny impL}] {
            {
              \begin{array}{cc}
                \pi_1 & \pi_2 \\
                {\Psi  \vdash_\mathcal{C}  \SCnt{Y_{{\mathrm{1}}}}} & {\Gamma_{{\mathrm{1}}}  \SCsym{;}  \SCnt{Y_{{\mathrm{2}}}}  \SCsym{;}  \Gamma_{{\mathrm{2}}}  \SCsym{;}  \SCnt{B}  \SCsym{;}  \Gamma_{{\mathrm{3}}}  \vdash_\mathcal{L}  \SCnt{A}}
              \end{array}
            }
          }{\Gamma_{{\mathrm{1}}}  \SCsym{;}  \SCnt{Y_{{\mathrm{1}}}}  \multimap  \SCnt{Y_{{\mathrm{2}}}}  \SCsym{;}  \Psi  \SCsym{;}  \Gamma_{{\mathrm{2}}}  \SCsym{;}  \SCnt{B}  \SCsym{;}  \Gamma_{{\mathrm{3}}}  \vdash_\mathcal{L}  \SCnt{A}}
        \end{math}
      \end{center}
      By assumption, $c(\Pi_1),c(\Pi_2)\leq |B|$. By induction on $\Pi_1$
      and $\pi_2$, there is a proof $\Pi'$ for sequent
      $\Gamma_{{\mathrm{1}}}  \SCsym{;}  \SCnt{Y_{{\mathrm{2}}}}  \SCsym{;}  \Gamma_{{\mathrm{2}}}  \SCsym{;}  \Delta  \SCsym{;}  \Gamma_{{\mathrm{3}}}  \vdash_\mathcal{L}  \SCnt{A}$ s.t. $c(\Pi') \leq |B|$. Therefore,
      the proof $\Pi$ can be constructed as follows with
      $c(\Pi) = c(\Pi') \leq |B|$.
      \begin{center}
        \scriptsize
        \begin{math}
          $$\mprset{flushleft}
          \inferrule* [right={\tiny impL}] {
            {
              \begin{array}{cc}
                \pi_1 & \Pi' \\
                {\Psi  \vdash_\mathcal{C}  \SCnt{Y_{{\mathrm{1}}}}} & {\Gamma_{{\mathrm{1}}}  \SCsym{;}  \SCnt{Y_{{\mathrm{2}}}}  \SCsym{;}  \Gamma_{{\mathrm{2}}}  \SCsym{;}  \Delta  \SCsym{;}  \Gamma_{{\mathrm{3}}}  \vdash_\mathcal{L}  \SCnt{A}}
              \end{array}
            }
          }{\Gamma_{{\mathrm{1}}}  \SCsym{;}  \SCnt{Y_{{\mathrm{1}}}}  \multimap  \SCnt{Y_{{\mathrm{2}}}}  \SCsym{;}  \Psi  \SCsym{;}  \Gamma_{{\mathrm{2}}}  \SCsym{;}  \Delta  \SCsym{;}  \Gamma_{{\mathrm{3}}}  \vdash_\mathcal{L}  \SCnt{A}}
        \end{math}
      \end{center}
\end{itemize}



\subsubsection{Left introduction of the non-commutative left implication $\lto$}
\begin{itemize}
\item Case 1:
      \begin{center}
        \scriptsize
        \begin{math}
          \begin{array}{c}
            \Pi_1 \\
            {\Phi  \vdash_\mathcal{C}  \SCnt{X}}
          \end{array}
        \end{math}
        \qquad\qquad
        $\Pi_2$:
        \begin{math}
          $$\mprset{flushleft}
          \inferrule* [right={\tiny imprL}] {
            {
              \begin{array}{cc}
                \pi_1 & \pi_2 \\
                {\Delta_{{\mathrm{1}}}  \SCsym{;}  \SCnt{X}  \SCsym{;}  \Delta_{{\mathrm{2}}}  \vdash_\mathcal{L}  \SCnt{A_{{\mathrm{1}}}}} & {\Gamma_{{\mathrm{1}}}  \SCsym{;}  \SCnt{A_{{\mathrm{2}}}}  \SCsym{;}  \Gamma_{{\mathrm{2}}}  \vdash_\mathcal{L}  \SCnt{B}}
              \end{array}
            }
          }{\Gamma_{{\mathrm{1}}}  \SCsym{;}  \SCnt{A_{{\mathrm{1}}}}  \rightharpoonup  \SCnt{A_{{\mathrm{2}}}}  \SCsym{;}  \Delta_{{\mathrm{1}}}  \SCsym{;}  \SCnt{X}  \SCsym{;}  \Delta_{{\mathrm{2}}}  \SCsym{;}  \Gamma_{{\mathrm{2}}}  \vdash_\mathcal{L}  \SCnt{B}}
        \end{math}
      \end{center}
      By assumption, $c(\Pi_1),c(\Pi_2)\leq |X|$. By induction on $\Pi_1$
      and $\pi_1$, there is a proof $\Pi'$ for sequent
      $\Delta_{{\mathrm{1}}}  \SCsym{;}  \Phi  \SCsym{;}  \Delta_{{\mathrm{2}}}  \vdash_\mathcal{L}  \SCnt{A_{{\mathrm{1}}}}$ s.t. $c(\Pi') \leq |X|$. Therefore, the proof
      $\Pi$ can be constructed as follows with $c(\Pi) = c(\Pi') \leq |X|$.
      \begin{center}
        \scriptsize
        \begin{math}
          $$\mprset{flushleft}
          \inferrule* [right={\tiny impL}] {
            {
              \begin{array}{cc}
                \Pi' & \pi_2 \\
                {\Delta_{{\mathrm{1}}}  \SCsym{;}  \Phi  \SCsym{;}  \Delta_{{\mathrm{2}}}  \vdash_\mathcal{L}  \SCnt{A_{{\mathrm{1}}}}} & {\Gamma_{{\mathrm{1}}}  \SCsym{;}  \SCnt{A_{{\mathrm{2}}}}  \SCsym{;}  \Gamma_{{\mathrm{2}}}  \vdash_\mathcal{L}  \SCnt{B}}
              \end{array}
            }
          }{\Gamma_{{\mathrm{1}}}  \SCsym{;}  \SCnt{A_{{\mathrm{1}}}}  \rightharpoonup  \SCnt{A_{{\mathrm{2}}}}  \SCsym{;}  \Delta_{{\mathrm{1}}}  \SCsym{;}  \Phi  \SCsym{;}  \Delta_{{\mathrm{2}}}  \SCsym{;}  \Gamma_{{\mathrm{2}}}  \vdash_\mathcal{L}  \SCnt{B}}
        \end{math}
      \end{center}

\item Case 2:
      \begin{center}
        \scriptsize
        \begin{math}
          \begin{array}{c}
            \Pi_1 \\
            {\Gamma  \vdash_\mathcal{L}  \SCnt{C}}
          \end{array}
        \end{math}
        \qquad\qquad
        $\Pi_2$:
        \begin{math}
          $$\mprset{flushleft}
          \inferrule* [right={\tiny imprL}] {
            {
              \begin{array}{cc}
                \pi_1 & \pi_2 \\
                {\Delta_{{\mathrm{1}}}  \SCsym{;}  \SCnt{C}  \SCsym{;}  \Delta_{{\mathrm{2}}}  \vdash_\mathcal{L}  \SCnt{A_{{\mathrm{1}}}}} & {\Gamma_{{\mathrm{1}}}  \SCsym{;}  \SCnt{A_{{\mathrm{2}}}}  \SCsym{;}  \Gamma_{{\mathrm{2}}}  \vdash_\mathcal{L}  \SCnt{B}}
              \end{array}
            }
          }{\Gamma_{{\mathrm{1}}}  \SCsym{;}  \SCnt{A_{{\mathrm{1}}}}  \rightharpoonup  \SCnt{A_{{\mathrm{2}}}}  \SCsym{;}  \Delta_{{\mathrm{1}}}  \SCsym{;}  \SCnt{C}  \SCsym{;}  \Delta_{{\mathrm{2}}}  \SCsym{;}  \Gamma_{{\mathrm{2}}}  \vdash_\mathcal{L}  \SCnt{B}}
        \end{math}
      \end{center}
      By assumption, $c(\Pi_1),c(\Pi_2)\leq |C|$. By induction on $\Pi_1$
      and $\pi_1$, there is a proof $\Pi'$ for sequent
      $\Delta_{{\mathrm{1}}}  \SCsym{;}  \Gamma  \SCsym{;}  \Delta_{{\mathrm{2}}}  \vdash_\mathcal{L}  \SCnt{A_{{\mathrm{1}}}}$ s.t. $c(\Pi') \leq |C|$. Therefore, the proof
      $\Pi$ can be constructed as follows with $c(\Pi) = c(\Pi') \leq |C|$.
      \begin{center}
        \scriptsize
        \begin{math}
          $$\mprset{flushleft}
          \inferrule* [right={\tiny imprL}] {
            {
              \begin{array}{cc}
                \Pi' & \pi_2 \\
                {\Delta_{{\mathrm{1}}}  \SCsym{;}  \Gamma  \SCsym{;}  \Delta_{{\mathrm{2}}}  \vdash_\mathcal{L}  \SCnt{A_{{\mathrm{1}}}}} & {\Gamma_{{\mathrm{1}}}  \SCsym{;}  \SCnt{A_{{\mathrm{2}}}}  \SCsym{;}  \Gamma_{{\mathrm{2}}}  \vdash_\mathcal{L}  \SCnt{B}}
              \end{array}
            }
          }{\Gamma_{{\mathrm{1}}}  \SCsym{;}  \SCnt{A_{{\mathrm{1}}}}  \rightharpoonup  \SCnt{A_{{\mathrm{2}}}}  \SCsym{;}  \Delta_{{\mathrm{1}}}  \SCsym{;}  \Gamma  \SCsym{;}  \Delta_{{\mathrm{2}}}  \SCsym{;}  \Gamma_{{\mathrm{2}}}  \vdash_\mathcal{L}  \SCnt{B}}
        \end{math}
      \end{center}

\item Case 3:
      \begin{center}
        \scriptsize
        \begin{math}
          \begin{array}{c}
            \Pi_1 \\
            {\Phi  \vdash_\mathcal{C}  \SCnt{X}}
          \end{array}
        \end{math}
        \qquad\qquad
        $\Pi_2$:
        \begin{math}
          $$\mprset{flushleft}
          \inferrule* [right={\tiny imprL}] {
            {
              \begin{array}{cc}
                \pi_1 & \pi_2 \\
                {\Delta  \vdash_\mathcal{L}  \SCnt{A_{{\mathrm{1}}}}} & {\Gamma_{{\mathrm{1}}}  \SCsym{;}  \SCnt{X}  \SCsym{;}  \Gamma_{{\mathrm{2}}}  \SCsym{;}  \SCnt{A_{{\mathrm{2}}}}  \SCsym{;}  \Gamma_{{\mathrm{3}}}  \vdash_\mathcal{L}  \SCnt{B}}
              \end{array}
            }
          }{\Gamma_{{\mathrm{1}}}  \SCsym{;}  \SCnt{X}  \SCsym{;}  \Gamma_{{\mathrm{2}}}  \SCsym{;}  \SCnt{A_{{\mathrm{1}}}}  \rightharpoonup  \SCnt{A_{{\mathrm{2}}}}  \SCsym{;}  \Delta  \SCsym{;}  \Gamma_{{\mathrm{3}}}  \vdash_\mathcal{L}  \SCnt{B}}
        \end{math}
      \end{center}
      By assumption, $c(\Pi_1),c(\Pi_2)\leq |X|$. By induction on $\Pi_1$
      and $\pi_2$, there is a proof $\Pi'$ for sequent
      $\Gamma_{{\mathrm{1}}}  \SCsym{;}  \Phi  \SCsym{;}  \Gamma_{{\mathrm{2}}}  \SCsym{;}  \SCnt{A_{{\mathrm{2}}}}  \SCsym{;}  \Gamma_{{\mathrm{3}}}  \vdash_\mathcal{L}  \SCnt{B}$ s.t. $c(\Pi') \leq |X|$. Therefore,
      the proof $\Pi$ can be constructed as follows with
      $c(\Pi) = c(\Pi') \leq |X|$.
      \begin{center}
        \scriptsize
        \begin{math}
          $$\mprset{flushleft}
          \inferrule* [right={\tiny imprL}] {
            {
              \begin{array}{cc}
                \pi_1 & \Pi' \\
                {\Delta  \vdash_\mathcal{L}  \SCnt{A_{{\mathrm{1}}}}} & {\Gamma_{{\mathrm{1}}}  \SCsym{;}  \Phi  \SCsym{;}  \Gamma_{{\mathrm{2}}}  \SCsym{;}  \SCnt{A_{{\mathrm{2}}}}  \SCsym{;}  \Gamma_{{\mathrm{3}}}  \vdash_\mathcal{L}  \SCnt{B}}
              \end{array}
            }
          }{\Gamma_{{\mathrm{1}}}  \SCsym{;}  \Phi  \SCsym{;}  \Gamma_{{\mathrm{2}}}  \SCsym{;}  \SCnt{A_{{\mathrm{1}}}}  \rightharpoonup  \SCnt{A_{{\mathrm{2}}}}  \SCsym{;}  \Delta  \SCsym{;}  \Gamma_{{\mathrm{3}}}  \vdash_\mathcal{L}  \SCnt{B}}
        \end{math}
      \end{center}

\item Case 4:
      \begin{center}
        \scriptsize
        \begin{math}
          \begin{array}{c}
            \Pi_1 \\
            {\Delta_{{\mathrm{1}}}  \vdash_\mathcal{L}  \SCnt{B}}
          \end{array}
        \end{math}
        \qquad\qquad
        $\Pi_2$:
        \begin{math}
          $$\mprset{flushleft}
          \inferrule* [right={\tiny imprL}] {
            {
              \begin{array}{cc}
                \pi_1 & \pi_2 \\
                {\Delta_{{\mathrm{2}}}  \vdash_\mathcal{L}  \SCnt{A_{{\mathrm{1}}}}} & {\Gamma_{{\mathrm{1}}}  \SCsym{;}  \SCnt{B}  \SCsym{;}  \Gamma_{{\mathrm{2}}}  \SCsym{;}  \SCnt{A_{{\mathrm{2}}}}  \SCsym{;}  \Gamma_{{\mathrm{3}}}  \vdash_\mathcal{L}  \SCnt{C}}
              \end{array}
            }
          }{\Gamma_{{\mathrm{1}}}  \SCsym{;}  \SCnt{B}  \SCsym{;}  \Gamma_{{\mathrm{2}}}  \SCsym{;}  \SCnt{A_{{\mathrm{1}}}}  \rightharpoonup  \SCnt{A_{{\mathrm{2}}}}  \SCsym{;}  \Delta_{{\mathrm{2}}}  \SCsym{;}  \Gamma_{{\mathrm{3}}}  \vdash_\mathcal{L}  \SCnt{C}}
        \end{math}
      \end{center}
      By assumption, $c(\Pi_1),c(\Pi_2)\leq |B|$. By induction on $\Pi_1$
      and $\pi_2$, there is a proof $\Pi'$ for sequent
      $\Gamma_{{\mathrm{1}}}  \SCsym{;}  \Delta_{{\mathrm{1}}}  \SCsym{;}  \Gamma_{{\mathrm{2}}}  \SCsym{;}  \SCnt{A_{{\mathrm{2}}}}  \SCsym{;}  \Gamma_{{\mathrm{3}}}  \vdash_\mathcal{L}  \SCnt{C}$ s.t. $c(\Pi') \leq |B|$. Therefore,
      the proof $\Pi$ can be constructed as follows with
      $c(\Pi) = c(\Pi') \leq |B|$.
      \begin{center}
        \scriptsize
        \begin{math}
          $$\mprset{flushleft}
          \inferrule* [right={\tiny imprL}] {
            {
              \begin{array}{cc}
                \pi_1 & \Pi' \\
                {\Delta_{{\mathrm{2}}}  \vdash_\mathcal{L}  \SCnt{A_{{\mathrm{1}}}}} & {\Gamma_{{\mathrm{1}}}  \SCsym{;}  \Delta_{{\mathrm{1}}}  \SCsym{;}  \Gamma_{{\mathrm{2}}}  \SCsym{;}  \SCnt{A_{{\mathrm{2}}}}  \SCsym{;}  \Gamma_{{\mathrm{3}}}  \vdash_\mathcal{L}  \SCnt{C}}
              \end{array}
            }
          }{\Gamma_{{\mathrm{1}}}  \SCsym{;}  \Delta_{{\mathrm{1}}}  \SCsym{;}  \Gamma_{{\mathrm{2}}}  \SCsym{;}  \SCnt{A_{{\mathrm{1}}}}  \rightharpoonup  \SCnt{A_{{\mathrm{2}}}}  \SCsym{;}  \Delta_{{\mathrm{2}}}  \SCsym{;}  \Gamma_{{\mathrm{3}}}  \vdash_\mathcal{L}  \SCnt{C}}
        \end{math}
      \end{center}

\item Case 5:
      \begin{center}
        \scriptsize
        \begin{math}
          \begin{array}{c}
            \Pi_1 \\
            {\Phi  \vdash_\mathcal{C}  \SCnt{X}}
          \end{array}
        \end{math}
        \qquad\qquad
        $\Pi_2$:
        \begin{math}
          $$\mprset{flushleft}
          \inferrule* [right={\tiny imprL}] {
            {
              \begin{array}{cc}
                \pi_1 & \pi_2 \\
                {\Delta  \vdash_\mathcal{L}  \SCnt{A_{{\mathrm{1}}}}} & {\Gamma_{{\mathrm{1}}}  \SCsym{;}  \SCnt{A_{{\mathrm{2}}}}  \SCsym{;}  \Gamma_{{\mathrm{2}}}  \SCsym{;}  \SCnt{X}  \SCsym{;}  \Gamma_{{\mathrm{3}}}  \vdash_\mathcal{L}  \SCnt{B}}
              \end{array}
            }
          }{\Gamma_{{\mathrm{1}}}  \SCsym{;}  \SCnt{A_{{\mathrm{1}}}}  \rightharpoonup  \SCnt{A_{{\mathrm{2}}}}  \SCsym{;}  \Delta  \SCsym{;}  \Gamma_{{\mathrm{2}}}  \SCsym{;}  \SCnt{X}  \SCsym{;}  \Gamma_{{\mathrm{3}}}  \vdash_\mathcal{L}  \SCnt{B}}
        \end{math}
      \end{center}
      By assumption, $c(\Pi_1),c(\Pi_2)\leq |X|$. By induction on $\Pi_1$
      and $\pi_2$, there is a proof $\Pi'$ for sequent
      $\Gamma_{{\mathrm{1}}}  \SCsym{;}  \SCnt{A_{{\mathrm{2}}}}  \SCsym{;}  \Gamma_{{\mathrm{2}}}  \SCsym{;}  \Phi  \SCsym{;}  \Gamma_{{\mathrm{3}}}  \vdash_\mathcal{L}  \SCnt{B}$ s.t. $c(\Pi') \leq |X|$. Therefore, the
      proof $\Pi$ can be constructed as follows with
      $c(\Pi) = c(\Pi') \leq |X|$.
      \begin{center}
        \scriptsize
        \begin{math}
          $$\mprset{flushleft}
          \inferrule* [right={\tiny imprL}] {
            {
              \begin{array}{cc}
                \pi_1 & \Pi' \\
                {\Delta  \vdash_\mathcal{L}  \SCnt{A_{{\mathrm{1}}}}} & {\Gamma_{{\mathrm{1}}}  \SCsym{;}  \SCnt{A_{{\mathrm{2}}}}  \SCsym{;}  \Gamma_{{\mathrm{2}}}  \SCsym{;}  \Phi  \SCsym{;}  \Gamma_{{\mathrm{3}}}  \vdash_\mathcal{L}  \SCnt{B}}
              \end{array}
            }
          }{\Gamma_{{\mathrm{1}}}  \SCsym{;}  \SCnt{A_{{\mathrm{1}}}}  \rightharpoonup  \SCnt{A_{{\mathrm{2}}}}  \SCsym{;}  \Delta  \SCsym{;}  \Gamma_{{\mathrm{2}}}  \SCsym{;}  \Phi  \SCsym{;}  \Gamma_{{\mathrm{3}}}  \vdash_\mathcal{L}  \SCnt{B}}
        \end{math}
      \end{center}

\item Case 6:
      \begin{center}
        \scriptsize
        \begin{math}
          \begin{array}{c}
            \Pi_1 \\
            {\Delta_{{\mathrm{1}}}  \vdash_\mathcal{L}  \SCnt{B}}
          \end{array}
        \end{math}
        \qquad\qquad
        $\Pi_2$:
        \begin{math}
          $$\mprset{flushleft}
          \inferrule* [right={\tiny imprL}] {
            {
              \begin{array}{cc}
                \pi_1 & \pi_2 \\
                {\Delta_{{\mathrm{2}}}  \vdash_\mathcal{L}  \SCnt{A_{{\mathrm{1}}}}} & {\Gamma_{{\mathrm{1}}}  \SCsym{;}  \SCnt{A_{{\mathrm{2}}}}  \SCsym{;}  \Gamma_{{\mathrm{2}}}  \SCsym{;}  \SCnt{B}  \SCsym{;}  \Gamma_{{\mathrm{3}}}  \vdash_\mathcal{L}  \SCnt{C}}
              \end{array}
            }
          }{\Gamma_{{\mathrm{1}}}  \SCsym{;}  \SCnt{A_{{\mathrm{1}}}}  \rightharpoonup  \SCnt{A_{{\mathrm{2}}}}  \SCsym{;}  \Delta_{{\mathrm{2}}}  \SCsym{;}  \Gamma_{{\mathrm{2}}}  \SCsym{;}  \SCnt{B}  \SCsym{;}  \Gamma_{{\mathrm{3}}}  \vdash_\mathcal{L}  \SCnt{C}}
        \end{math}
      \end{center}
      By assumption, $c(\Pi_1),c(\Pi_2)\leq |B|$. By induction on $\Pi_1$
      and $\pi_2$, there is a proof $\Pi'$ for sequent
      $\Gamma_{{\mathrm{1}}}  \SCsym{;}  \SCnt{A_{{\mathrm{2}}}}  \SCsym{;}  \Gamma_{{\mathrm{2}}}  \SCsym{;}  \Delta_{{\mathrm{1}}}  \SCsym{;}  \Gamma_{{\mathrm{3}}}  \vdash_\mathcal{L}  \SCnt{C}$ s.t. $c(\Pi') \leq |B|$. Therefore,
      the proof $\Pi$ can be constructed as follows with
      $c(\Pi) = c(\Pi') \leq |B|$.
      \begin{center}
        \scriptsize
        \begin{math}
          $$\mprset{flushleft}
          \inferrule* [right={\tiny imprL}] {
            {
              \begin{array}{cc}
                \pi_1 & \Pi' \\
                {\Delta_{{\mathrm{2}}}  \vdash_\mathcal{L}  \SCnt{A_{{\mathrm{1}}}}} & {\Gamma_{{\mathrm{1}}}  \SCsym{;}  \SCnt{A_{{\mathrm{2}}}}  \SCsym{;}  \Gamma_{{\mathrm{2}}}  \SCsym{;}  \Delta_{{\mathrm{1}}}  \SCsym{;}  \Gamma_{{\mathrm{3}}}  \vdash_\mathcal{L}  \SCnt{C}}
              \end{array}
            }
          }{\Gamma_{{\mathrm{1}}}  \SCsym{;}  \SCnt{A_{{\mathrm{1}}}}  \rightharpoonup  \SCnt{A_{{\mathrm{2}}}}  \SCsym{;}  \Delta_{{\mathrm{2}}}  \SCsym{;}  \Gamma_{{\mathrm{2}}}  \SCsym{;}  \Delta_{{\mathrm{1}}}  \SCsym{;}  \Gamma_{{\mathrm{3}}}  \vdash_\mathcal{L}  \SCnt{C}}
        \end{math}
      \end{center}
\end{itemize}


\subsubsection{Left introduction of the non-commutative right implication $\rto$}
\begin{itemize}
\item Case 1:
      \begin{center}
        \scriptsize
        \begin{math}
          \begin{array}{c}
            \Pi_1 \\
            {\Phi  \vdash_\mathcal{C}  \SCnt{X}}
          \end{array}
        \end{math}
        \qquad\qquad
        $\Pi_2$:
        \begin{math}
          $$\mprset{flushleft}
          \inferrule* [right={\tiny implL}] {
            {
              \begin{array}{cc}
                \pi_1 & \pi_2 \\
                {\Delta_{{\mathrm{1}}}  \SCsym{;}  \SCnt{X}  \SCsym{;}  \Delta_{{\mathrm{2}}}  \vdash_\mathcal{L}  \SCnt{A_{{\mathrm{1}}}}} & {\Gamma_{{\mathrm{1}}}  \SCsym{;}  \SCnt{A_{{\mathrm{2}}}}  \SCsym{;}  \Gamma_{{\mathrm{2}}}  \vdash_\mathcal{L}  \SCnt{B}}
              \end{array}
            }
          }{\Gamma_{{\mathrm{1}}}  \SCsym{;}  \Delta_{{\mathrm{1}}}  \SCsym{;}  \SCnt{A_{{\mathrm{2}}}}  \leftharpoonup  \SCnt{A_{{\mathrm{1}}}}  \SCsym{;}  \SCnt{X}  \SCsym{;}  \Delta_{{\mathrm{2}}}  \SCsym{;}  \Gamma_{{\mathrm{2}}}  \vdash_\mathcal{L}  \SCnt{B}}
        \end{math}
      \end{center}
      By assumption, $c(\Pi_1),c(\Pi_2)\leq |X|$. By induction on $\Pi_1$
      and $\pi_1$, there is a proof $\Pi'$ for sequent
      $\Delta_{{\mathrm{1}}}  \SCsym{;}  \Phi  \SCsym{;}  \Delta_{{\mathrm{2}}}  \vdash_\mathcal{L}  \SCnt{A_{{\mathrm{1}}}}$ s.t. $c(\Pi') \leq |X|$. Therefore, the proof
      $\Pi$ can be constructed as follows with $c(\Pi) = c(\Pi') \leq |X|$.
      \begin{center}
        \scriptsize
        \begin{math}
          $$\mprset{flushleft}
          \inferrule* [right={\tiny implL}] {
            {
              \begin{array}{cc}
                \Pi' & \pi_2 \\
                {\Delta_{{\mathrm{1}}}  \SCsym{;}  \Phi  \SCsym{;}  \Delta_{{\mathrm{2}}}  \vdash_\mathcal{L}  \SCnt{A_{{\mathrm{1}}}}} & {\Gamma_{{\mathrm{1}}}  \SCsym{;}  \SCnt{A_{{\mathrm{2}}}}  \SCsym{;}  \Gamma_{{\mathrm{2}}}  \vdash_\mathcal{L}  \SCnt{B}}
              \end{array}
            }
          }{\Gamma_{{\mathrm{1}}}  \SCsym{;}  \Delta_{{\mathrm{1}}}  \SCsym{;}  \SCnt{A_{{\mathrm{2}}}}  \leftharpoonup  \SCnt{A_{{\mathrm{1}}}}  \SCsym{;}  \Phi  \SCsym{;}  \Delta_{{\mathrm{2}}}  \SCsym{;}  \Gamma_{{\mathrm{2}}}  \vdash_\mathcal{L}  \SCnt{B}}
        \end{math}
      \end{center}

\item Case 2:
      \begin{center}
        \scriptsize
        \begin{math}
          \begin{array}{c}
            \Pi_1 \\
            {\Gamma  \vdash_\mathcal{L}  \SCnt{C}}
          \end{array}
        \end{math}
        \qquad\qquad
        $\Pi_2$:
        \begin{math}
          $$\mprset{flushleft}
          \inferrule* [right={\tiny implL}] {
            {
              \begin{array}{cc}
                \pi_1 & \pi_2 \\
                {\Delta_{{\mathrm{1}}}  \SCsym{;}  \SCnt{C}  \SCsym{;}  \Delta_{{\mathrm{2}}}  \vdash_\mathcal{L}  \SCnt{A_{{\mathrm{1}}}}} & {\Gamma_{{\mathrm{1}}}  \SCsym{;}  \SCnt{A_{{\mathrm{2}}}}  \SCsym{;}  \Gamma_{{\mathrm{2}}}  \vdash_\mathcal{L}  \SCnt{B}}
              \end{array}
            }
          }{\Gamma_{{\mathrm{1}}}  \SCsym{;}  \Delta_{{\mathrm{1}}}  \SCsym{;}  \SCnt{C}  \SCsym{;}  \Delta_{{\mathrm{2}}}  \SCsym{;}  \SCnt{A_{{\mathrm{2}}}}  \leftharpoonup  \SCnt{A_{{\mathrm{1}}}}  \SCsym{;}  \Gamma_{{\mathrm{2}}}  \vdash_\mathcal{L}  \SCnt{B}}
        \end{math}
      \end{center}
      By assumption, $c(\Pi_1),c(\Pi_2)\leq |C|$. By induction on $\Pi_1$
      and $\pi_1$, there is a proof $\Pi'$ for sequent
      $\Delta_{{\mathrm{1}}}  \SCsym{;}  \Gamma  \SCsym{;}  \Delta_{{\mathrm{2}}}  \vdash_\mathcal{L}  \SCnt{A_{{\mathrm{1}}}}$ s.t. $c(\Pi') \leq |C|$. Therefore, the proof
      $\Pi$ can be constructed as follows with $c(\Pi) = c(\Pi') \leq |C|$.
      \begin{center}
        \scriptsize
        \begin{math}
          $$\mprset{flushleft}
          \inferrule* [right={\tiny implL}] {
            {
              \begin{array}{cc}
                \Pi' & \pi_2 \\
                {\Delta_{{\mathrm{1}}}  \SCsym{;}  \Gamma  \SCsym{;}  \Delta_{{\mathrm{2}}}  \vdash_\mathcal{L}  \SCnt{A_{{\mathrm{1}}}}} & {\Gamma_{{\mathrm{1}}}  \SCsym{;}  \SCnt{A_{{\mathrm{2}}}}  \SCsym{;}  \Gamma_{{\mathrm{2}}}  \vdash_\mathcal{L}  \SCnt{B}}
              \end{array}
            }
          }{\Gamma_{{\mathrm{1}}}  \SCsym{;}  \Delta_{{\mathrm{1}}}  \SCsym{;}  \Gamma  \SCsym{;}  \Delta_{{\mathrm{2}}}  \SCsym{;}  \SCnt{A_{{\mathrm{2}}}}  \leftharpoonup  \SCnt{A_{{\mathrm{1}}}}  \SCsym{;}  \Gamma_{{\mathrm{2}}}  \vdash_\mathcal{L}  \SCnt{B}}
        \end{math}
      \end{center}

\item Case 3:
      \begin{center}
        \scriptsize
        \begin{math}
          \begin{array}{c}
            \Pi_1 \\
            {\Phi  \vdash_\mathcal{C}  \SCnt{X}}
          \end{array}
        \end{math}
        \qquad\qquad
        $\Pi_2$:
        \begin{math}
          $$\mprset{flushleft}
          \inferrule* [right={\tiny implL}] {
            {
              \begin{array}{cc}
                \pi_1 & \pi_2 \\
                {\Delta  \vdash_\mathcal{L}  \SCnt{A_{{\mathrm{1}}}}} & {\Gamma_{{\mathrm{1}}}  \SCsym{;}  \SCnt{X}  \SCsym{;}  \Gamma_{{\mathrm{2}}}  \SCsym{;}  \SCnt{A_{{\mathrm{2}}}}  \SCsym{;}  \Gamma_{{\mathrm{3}}}  \vdash_\mathcal{L}  \SCnt{B}}
              \end{array}
            }
          }{\Gamma_{{\mathrm{1}}}  \SCsym{;}  \SCnt{X}  \SCsym{;}  \Gamma_{{\mathrm{2}}}  \SCsym{;}  \Delta  \SCsym{;}  \SCnt{A_{{\mathrm{2}}}}  \leftharpoonup  \SCnt{A_{{\mathrm{1}}}}  \SCsym{;}  \Gamma_{{\mathrm{3}}}  \vdash_\mathcal{L}  \SCnt{B}}
        \end{math}
      \end{center}
      By assumption, $c(\Pi_1),c(\Pi_2)\leq |X|$. By induction on $\Pi_1$
      and $\pi_2$, there is a proof $\Pi'$ for sequent
      $\Gamma_{{\mathrm{1}}}  \SCsym{;}  \Phi  \SCsym{;}  \Gamma_{{\mathrm{2}}}  \SCsym{;}  \SCnt{A_{{\mathrm{2}}}}  \SCsym{;}  \Gamma_{{\mathrm{3}}}  \vdash_\mathcal{L}  \SCnt{B}$ s.t. $c(\Pi') \leq |X|$. Therefore, the
      proof $\Pi$ can be constructed as follows with
      $c(\Pi) = c(\Pi') \leq |X|$.
      \begin{center}
        \scriptsize
        \begin{math}
          $$\mprset{flushleft}
          \inferrule* [right={\tiny implL}] {
            {
              \begin{array}{cc}
                \pi_1 & \Pi' \\
                {\Delta  \vdash_\mathcal{L}  \SCnt{A_{{\mathrm{1}}}}} & {\Gamma_{{\mathrm{1}}}  \SCsym{;}  \Phi  \SCsym{;}  \Gamma_{{\mathrm{2}}}  \SCsym{;}  \SCnt{A_{{\mathrm{2}}}}  \SCsym{;}  \Gamma_{{\mathrm{3}}}  \vdash_\mathcal{L}  \SCnt{B}}
              \end{array}
            }
          }{\Gamma_{{\mathrm{1}}}  \SCsym{;}  \Phi  \SCsym{;}  \Gamma_{{\mathrm{2}}}  \SCsym{;}  \Delta  \SCsym{;}  \SCnt{A_{{\mathrm{2}}}}  \leftharpoonup  \SCnt{A_{{\mathrm{1}}}}  \SCsym{;}  \Gamma_{{\mathrm{3}}}  \vdash_\mathcal{L}  \SCnt{B}}
        \end{math}
      \end{center}

\item Case 4:
      \begin{center}
        \scriptsize
        \begin{math}
          \begin{array}{c}
            \Pi_1 \\
            {\Delta_{{\mathrm{1}}}  \vdash_\mathcal{L}  \SCnt{B}}
          \end{array}
        \end{math}
        \qquad\qquad
        $\Pi_2$:
        \begin{math}
          $$\mprset{flushleft}
          \inferrule* [right={\tiny implL}] {
            {
              \begin{array}{cc}
                \pi_1 & \pi_2 \\
                {\Delta_{{\mathrm{2}}}  \vdash_\mathcal{L}  \SCnt{A_{{\mathrm{1}}}}} & {\Gamma_{{\mathrm{1}}}  \SCsym{;}  \SCnt{B}  \SCsym{;}  \Gamma_{{\mathrm{2}}}  \SCsym{;}  \SCnt{A_{{\mathrm{2}}}}  \SCsym{;}  \Gamma_{{\mathrm{3}}}  \vdash_\mathcal{L}  \SCnt{C}}
              \end{array}
            }
          }{\Gamma_{{\mathrm{1}}}  \SCsym{;}  \SCnt{B}  \SCsym{;}  \Gamma_{{\mathrm{2}}}  \SCsym{;}  \Delta_{{\mathrm{2}}}  \SCsym{;}  \SCnt{A_{{\mathrm{2}}}}  \leftharpoonup  \SCnt{A_{{\mathrm{1}}}}  \SCsym{;}  \Gamma_{{\mathrm{3}}}  \vdash_\mathcal{L}  \SCnt{C}}
        \end{math}
      \end{center}
      By assumption, $c(\Pi_1),c(\Pi_2)\leq |B|$. By induction on $\Pi_1$
      and $\pi_2$, there is a proof $\Pi'$ for sequent
      $\Gamma_{{\mathrm{1}}}  \SCsym{;}  \Delta_{{\mathrm{1}}}  \SCsym{;}  \Gamma_{{\mathrm{2}}}  \SCsym{;}  \SCnt{A_{{\mathrm{2}}}}  \SCsym{;}  \Gamma_{{\mathrm{3}}}  \vdash_\mathcal{L}  \SCnt{C}$ s.t. $c(\Pi') \leq |B|$. Therefore,
      the proof $\Pi$ can be constructed as follows with
      $c(\Pi) = c(\Pi') \leq |B|$.
      \begin{center}
        \scriptsize
        \begin{math}
          $$\mprset{flushleft}
          \inferrule* [right={\tiny implL}] {
            {
              \begin{array}{cc}
                \pi_1 & \Pi' \\
                {\Delta_{{\mathrm{2}}}  \vdash_\mathcal{L}  \SCnt{A_{{\mathrm{1}}}}} & {\Gamma_{{\mathrm{1}}}  \SCsym{;}  \Delta_{{\mathrm{1}}}  \SCsym{;}  \Gamma_{{\mathrm{2}}}  \SCsym{;}  \SCnt{A_{{\mathrm{2}}}}  \SCsym{;}  \Gamma_{{\mathrm{3}}}  \vdash_\mathcal{L}  \SCnt{C}}
              \end{array}
            }
          }{\Gamma_{{\mathrm{1}}}  \SCsym{;}  \Delta_{{\mathrm{1}}}  \SCsym{;}  \Gamma_{{\mathrm{2}}}  \SCsym{;}  \Delta_{{\mathrm{2}}}  \SCsym{;}  \SCnt{A_{{\mathrm{2}}}}  \leftharpoonup  \SCnt{A_{{\mathrm{1}}}}  \SCsym{;}  \Gamma_{{\mathrm{3}}}  \vdash_\mathcal{L}  \SCnt{C}}
        \end{math}
      \end{center}

\item Case 5:
      \begin{center}
        \scriptsize
        \begin{math}
          \begin{array}{c}
            \Pi_1 \\
            {\Phi  \vdash_\mathcal{C}  \SCnt{X}}
          \end{array}
        \end{math}
        \qquad\qquad
        $\Pi_2$:
        \begin{math}
          $$\mprset{flushleft}
          \inferrule* [right={\tiny implL}] {
            {
              \begin{array}{cc}
                \pi_1 & \pi_2 \\
                {\Delta  \vdash_\mathcal{L}  \SCnt{A_{{\mathrm{1}}}}} & {\Gamma_{{\mathrm{1}}}  \SCsym{;}  \SCnt{A_{{\mathrm{2}}}}  \SCsym{;}  \Gamma_{{\mathrm{2}}}  \SCsym{;}  \SCnt{X}  \SCsym{;}  \Gamma_{{\mathrm{3}}}  \vdash_\mathcal{L}  \SCnt{B}}
              \end{array}
            }
          }{\Gamma_{{\mathrm{1}}}  \SCsym{;}  \Delta  \SCsym{;}  \SCnt{A_{{\mathrm{2}}}}  \leftharpoonup  \SCnt{A_{{\mathrm{1}}}}  \SCsym{;}  \Delta  \SCsym{;}  \Gamma_{{\mathrm{2}}}  \SCsym{;}  \SCnt{X}  \SCsym{;}  \Gamma_{{\mathrm{3}}}  \vdash_\mathcal{L}  \SCnt{B}}
        \end{math}
      \end{center}
      By assumption, $c(\Pi_1),c(\Pi_2)\leq |X|$. By induction on $\Pi_1$
      and $\pi_2$, there is a proof $\Pi'$ for sequent
      $\Gamma_{{\mathrm{1}}}  \SCsym{;}  \SCnt{A_{{\mathrm{2}}}}  \SCsym{;}  \Gamma_{{\mathrm{2}}}  \SCsym{;}  \Phi  \SCsym{;}  \Gamma_{{\mathrm{3}}}  \vdash_\mathcal{L}  \SCnt{B}$ s.t. $c(\Pi') \leq |X|$. Therefore, the
      proof $\Pi$ can be constructed as follows with
      $c(\Pi) = c(\Pi') \leq |X|$.
      \begin{center}
        \scriptsize
        \begin{math}
          $$\mprset{flushleft}
          \inferrule* [right={\tiny implL}] {
            {
              \begin{array}{cc}
                \pi_1 & \Pi' \\
                {\Delta  \vdash_\mathcal{L}  \SCnt{A_{{\mathrm{1}}}}} & {\Gamma_{{\mathrm{1}}}  \SCsym{;}  \SCnt{A_{{\mathrm{2}}}}  \SCsym{;}  \Gamma_{{\mathrm{2}}}  \SCsym{;}  \Phi  \SCsym{;}  \Gamma_{{\mathrm{3}}}  \vdash_\mathcal{L}  \SCnt{B}}
              \end{array}
            }
          }{\Gamma_{{\mathrm{1}}}  \SCsym{;}  \Delta  \SCsym{;}  \SCnt{A_{{\mathrm{2}}}}  \leftharpoonup  \SCnt{A_{{\mathrm{1}}}}  \SCsym{;}  \Gamma_{{\mathrm{2}}}  \SCsym{;}  \Phi  \SCsym{;}  \Gamma_{{\mathrm{3}}}  \vdash_\mathcal{L}  \SCnt{B}}
        \end{math}
      \end{center}

\item Case 6:
    \begin{center}
      \scriptsize
      \begin{math}
        \begin{array}{c}
          \Pi_1 \\
          {\Delta_{{\mathrm{1}}}  \vdash_\mathcal{L}  \SCnt{B}}
        \end{array}
      \end{math}
      \qquad\qquad
      $\Pi_2$:
      \begin{math}
        $$\mprset{flushleft}
        \inferrule* [right={\tiny implL}] {
          {
            \begin{array}{cc}
              \pi_1 & \pi_2 \\
              {\Delta_{{\mathrm{2}}}  \vdash_\mathcal{L}  \SCnt{A_{{\mathrm{1}}}}} & {\Gamma_{{\mathrm{1}}}  \SCsym{;}  \SCnt{A_{{\mathrm{2}}}}  \SCsym{;}  \Gamma_{{\mathrm{2}}}  \SCsym{;}  \SCnt{B}  \SCsym{;}  \Gamma_{{\mathrm{3}}}  \vdash_\mathcal{L}  \SCnt{C}}
            \end{array}
          }
        }{\Gamma_{{\mathrm{1}}}  \SCsym{;}  \Delta_{{\mathrm{2}}}  \SCsym{;}  \SCnt{A_{{\mathrm{2}}}}  \leftharpoonup  \SCnt{A_{{\mathrm{1}}}}  \SCsym{;}  \Gamma_{{\mathrm{2}}}  \SCsym{;}  \SCnt{B}  \SCsym{;}  \Gamma_{{\mathrm{3}}}  \vdash_\mathcal{L}  \SCnt{C}}
      \end{math}
    \end{center}
    By assumption, $c(\Pi_1),c(\Pi_2)\leq |B|$. By induction on $\Pi_1$ and
    $\pi_2$, there is a proof $\Pi'$ for sequent
    $\Gamma_{{\mathrm{1}}}  \SCsym{;}  \SCnt{A_{{\mathrm{2}}}}  \SCsym{;}  \Gamma_{{\mathrm{2}}}  \SCsym{;}  \Delta_{{\mathrm{1}}}  \SCsym{;}  \Gamma_{{\mathrm{3}}}  \vdash_\mathcal{L}  \SCnt{C}$ s.t. $c(\Pi') \leq |B|$. Therefore, the
    proof $\Pi$ can be constructed as follows with
    $c(\Pi) = c(\Pi') \leq |B|$.
    \begin{center}
      \scriptsize
      \begin{math}
        $$\mprset{flushleft}
        \inferrule* [right={\tiny implL}] {
          {
            \begin{array}{cc}
              \pi_1 & \Pi' \\
              {\Delta_{{\mathrm{2}}}  \vdash_\mathcal{L}  \SCnt{A_{{\mathrm{1}}}}} & {\Gamma_{{\mathrm{1}}}  \SCsym{;}  \SCnt{A_{{\mathrm{2}}}}  \SCsym{;}  \Gamma_{{\mathrm{2}}}  \SCsym{;}  \Delta_{{\mathrm{1}}}  \SCsym{;}  \Gamma_{{\mathrm{3}}}  \vdash_\mathcal{L}  \SCnt{C}}
            \end{array}
          }
        }{\Gamma_{{\mathrm{1}}}  \SCsym{;}  \Delta_{{\mathrm{2}}}  \SCsym{;}  \SCnt{A_{{\mathrm{2}}}}  \leftharpoonup  \SCnt{A_{{\mathrm{1}}}}  \SCsym{;}  \Gamma_{{\mathrm{2}}}  \SCsym{;}  \Delta_{{\mathrm{1}}}  \SCsym{;}  \Gamma_{{\mathrm{3}}}  \vdash_\mathcal{L}  \SCnt{C}}
      \end{math}
    \end{center}
\end{itemize}




\subsubsection{Left introduction of the commutative tensor $\otimes$ (with low priority)}
\begin{itemize}
\item Case 1:
      \begin{center}
        \scriptsize
        \begin{math}
          \begin{array}{c}
            \Pi_1 \\
            {\Phi  \vdash_\mathcal{C}  \SCnt{X}}
          \end{array}
        \end{math}
        \qquad\qquad
        $\Pi_2$:
        \begin{math}
          $$\mprset{flushleft}
          \inferrule* [right={\tiny tenL}] {
            {
              \begin{array}{c}
                \pi \\
                {\Psi_{{\mathrm{1}}}  \SCsym{,}  \SCnt{X}  \SCsym{,}  \Psi_{{\mathrm{2}}}  \SCsym{,}  \SCnt{Y_{{\mathrm{1}}}}  \SCsym{,}  \SCnt{Y_{{\mathrm{2}}}}  \SCsym{,}  \Psi_{{\mathrm{3}}}  \vdash_\mathcal{C}  \SCnt{Z}}
              \end{array}
            }
          }{\Psi_{{\mathrm{1}}}  \SCsym{,}  \SCnt{X}  \SCsym{,}  \Psi_{{\mathrm{2}}}  \SCsym{,}  \SCnt{Y_{{\mathrm{1}}}}  \otimes  \SCnt{Y_{{\mathrm{2}}}}  \SCsym{,}  \Psi_{{\mathrm{3}}}  \vdash_\mathcal{C}  \SCnt{Z}}
        \end{math}
      \end{center}
      By assumption, $c(\Pi_1),c(\Pi_2)\leq |X|$. By induction on $\Pi_1$
      and $\pi$, there is a proof $\Pi'$ for sequent
      $\Psi_{{\mathrm{1}}}  \SCsym{,}  \Phi  \SCsym{,}  \Psi_{{\mathrm{2}}}  \SCsym{,}  \SCnt{Y_{{\mathrm{1}}}}  \SCsym{,}  \SCnt{Y_{{\mathrm{2}}}}  \SCsym{,}  \Psi_{{\mathrm{3}}}  \vdash_\mathcal{C}  \SCnt{Z}$ s.t. $c(\Pi') \leq |X|$. Therefore,
      the proof $\Pi$ can be constructed as follows with
      $c(\Pi) = c(\Pi') \leq |X|$.
      \begin{center}
        \scriptsize
        \begin{math}
          $$\mprset{flushleft}
          \inferrule* [right={\tiny tenL}] {
            {
              \begin{array}{c}
                \Pi' \\
                {\Psi_{{\mathrm{1}}}  \SCsym{,}  \Phi  \SCsym{,}  \Psi_{{\mathrm{2}}}  \SCsym{,}  \SCnt{Y_{{\mathrm{1}}}}  \SCsym{,}  \SCnt{Y_{{\mathrm{2}}}}  \SCsym{,}  \Psi_{{\mathrm{3}}}  \vdash_\mathcal{C}  \SCnt{Z}}
              \end{array}
            }
          }{\Psi_{{\mathrm{1}}}  \SCsym{,}  \Phi  \SCsym{,}  \Psi_{{\mathrm{2}}}  \SCsym{,}  \SCnt{Y_{{\mathrm{1}}}}  \otimes  \SCnt{Y_{{\mathrm{2}}}}  \SCsym{,}  \Psi_{{\mathrm{3}}}  \vdash_\mathcal{C}  \SCnt{Z}}
        \end{math}
      \end{center}

\item Case 2:
      \begin{center}
        \scriptsize
        \begin{math}
          \begin{array}{c}
            \Pi_1 \\
            {\Phi  \vdash_\mathcal{C}  \SCnt{X}}
          \end{array}
        \end{math}
        \qquad\qquad
        $\Pi_2$:
        \begin{math}
          $$\mprset{flushleft}
          \inferrule* [right={\tiny tenL}] {
            {
              \begin{array}{c}
                \pi \\
                {\Psi_{{\mathrm{1}}}  \SCsym{,}  \SCnt{Y_{{\mathrm{1}}}}  \SCsym{,}  \SCnt{Y_{{\mathrm{2}}}}  \SCsym{,}  \Psi_{{\mathrm{2}}}  \SCsym{,}  \SCnt{X}  \SCsym{,}  \Psi_{{\mathrm{3}}}  \vdash_\mathcal{C}  \SCnt{Z}}
              \end{array}
            }
          }{\Psi_{{\mathrm{1}}}  \SCsym{,}  \SCnt{Y_{{\mathrm{1}}}}  \otimes  \SCnt{Y_{{\mathrm{2}}}}  \SCsym{,}  \Psi_{{\mathrm{2}}}  \SCsym{,}  \SCnt{X}  \SCsym{,}  \Psi_{{\mathrm{3}}}  \vdash_\mathcal{C}  \SCnt{Z}}
        \end{math}
      \end{center}
      By assumption, $c(\Pi_1),c(\Pi_2)\leq |X|$. By induction on $\Pi_1$
      and $\pi$, there is a proof $\Pi'$ for sequent
      $\Psi_{{\mathrm{1}}}  \SCsym{,}  \SCnt{Y_{{\mathrm{1}}}}  \SCsym{,}  \SCnt{Y_{{\mathrm{2}}}}  \SCsym{,}  \Psi_{{\mathrm{2}}}  \SCsym{,}  \Phi  \SCsym{,}  \Psi_{{\mathrm{3}}}  \vdash_\mathcal{C}  \SCnt{Z}$ s.t. $c(\Pi') \leq |X|$. Therefore,
      the proof $\Pi$ can be constructed as follows with
      $c(\Pi) = c(\Pi') \leq |X|$.
      \begin{center}
        \scriptsize
        \begin{math}
          $$\mprset{flushleft}
          \inferrule* [right={\tiny tenL}] {
            {
              \begin{array}{c}
                \Pi' \\
                {\Psi_{{\mathrm{1}}}  \SCsym{,}  \SCnt{Y_{{\mathrm{1}}}}  \SCsym{,}  \SCnt{Y_{{\mathrm{2}}}}  \SCsym{,}  \Psi_{{\mathrm{2}}}  \SCsym{,}  \Phi  \SCsym{,}  \Psi_{{\mathrm{3}}}  \vdash_\mathcal{C}  \SCnt{Z}}
              \end{array}
            }
          }{\Psi_{{\mathrm{1}}}  \SCsym{,}  \SCnt{Y_{{\mathrm{1}}}}  \otimes  \SCnt{Y_{{\mathrm{2}}}}  \SCsym{,}  \Psi_{{\mathrm{2}}}  \SCsym{,}  \Phi  \SCsym{,}  \Psi_{{\mathrm{3}}}  \vdash_\mathcal{C}  \SCnt{Z}}
        \end{math}
      \end{center}

\item Case 3:
      \begin{center}
        \scriptsize
        \begin{math}
          \begin{array}{c}
            \Pi_1 \\
            {\Phi  \vdash_\mathcal{C}  \SCnt{X}}
          \end{array}
        \end{math}
        \qquad\qquad
        $\Pi_2$:
        \begin{math}
          $$\mprset{flushleft}
          \inferrule* [right={\tiny tenL}] {
            {
              \begin{array}{c}
                \pi \\
                {\Gamma_{{\mathrm{1}}}  \SCsym{;}  \SCnt{X}  \SCsym{;}  \Gamma_{{\mathrm{2}}}  \SCsym{;}  \SCnt{Y_{{\mathrm{1}}}}  \SCsym{;}  \SCnt{Y_{{\mathrm{2}}}}  \SCsym{;}  \Gamma_{{\mathrm{3}}}  \vdash_\mathcal{L}  \SCnt{A}}
              \end{array}
            }
          }{\Gamma_{{\mathrm{1}}}  \SCsym{;}  \SCnt{X}  \SCsym{;}  \Gamma_{{\mathrm{2}}}  \SCsym{;}  \SCnt{Y_{{\mathrm{1}}}}  \otimes  \SCnt{Y_{{\mathrm{2}}}}  \SCsym{;}  \Gamma_{{\mathrm{3}}}  \vdash_\mathcal{L}  \SCnt{A}}
        \end{math}
      \end{center}
      By assumption, $c(\Pi_1),c(\Pi_2)\leq |X|$. By induction on $\Pi_1$
      and $\pi$, there is a proof $\Pi'$ for sequent
      $\Gamma_{{\mathrm{1}}}  \SCsym{;}  \Phi  \SCsym{;}  \Gamma_{{\mathrm{2}}}  \SCsym{;}  \SCnt{Y_{{\mathrm{1}}}}  \SCsym{;}  \SCnt{Y_{{\mathrm{2}}}}  \SCsym{;}  \Gamma_{{\mathrm{3}}}  \vdash_\mathcal{L}  \SCnt{A}$ s.t. $c(\Pi') \leq |X|$. Therefore,
      the proof $\Pi$ can be constructed as follows with
      $c(\Pi) = c(\Pi') \leq |X|$.
      \begin{center}
        \scriptsize
        \begin{math}
          $$\mprset{flushleft}
          \inferrule* [right={\tiny tenL}] {
            {
              \begin{array}{c}
                \Pi' \\
                {\Gamma_{{\mathrm{1}}}  \SCsym{;}  \Phi  \SCsym{;}  \Gamma_{{\mathrm{2}}}  \SCsym{;}  \SCnt{Y_{{\mathrm{1}}}}  \SCsym{;}  \SCnt{Y_{{\mathrm{2}}}}  \SCsym{;}  \Gamma_{{\mathrm{3}}}  \vdash_\mathcal{L}  \SCnt{A}}
              \end{array}
            }
          }{\Gamma_{{\mathrm{1}}}  \SCsym{;}  \Phi  \SCsym{;}  \Gamma_{{\mathrm{2}}}  \SCsym{;}  \SCnt{Y_{{\mathrm{1}}}}  \otimes  \SCnt{Y_{{\mathrm{2}}}}  \SCsym{;}  \Gamma_{{\mathrm{3}}}  \vdash_\mathcal{L}  \SCnt{A}}
        \end{math}
      \end{center}

\item Case 4:
      \begin{center}
        \scriptsize
        \begin{math}
          \begin{array}{c}
            \Pi_1 \\
            {\Delta  \vdash_\mathcal{L}  \SCnt{B}}
          \end{array}
        \end{math}
        \qquad\qquad
        $\Pi_2$:
        \begin{math}
          $$\mprset{flushleft}
          \inferrule* [right={\tiny tenL}] {
            {
              \begin{array}{c}
                \pi \\
                {\Gamma_{{\mathrm{1}}}  \SCsym{;}  \SCnt{B}  \SCsym{;}  \Gamma_{{\mathrm{2}}}  \SCsym{;}  \SCnt{Y_{{\mathrm{1}}}}  \SCsym{;}  \SCnt{Y_{{\mathrm{2}}}}  \SCsym{;}  \Gamma_{{\mathrm{3}}}  \vdash_\mathcal{L}  \SCnt{A}}
              \end{array}
            }
          }{\Gamma_{{\mathrm{1}}}  \SCsym{;}  \SCnt{B}  \SCsym{;}  \Gamma_{{\mathrm{2}}}  \SCsym{;}  \SCnt{Y_{{\mathrm{1}}}}  \otimes  \SCnt{Y_{{\mathrm{2}}}}  \SCsym{;}  \Gamma_{{\mathrm{3}}}  \vdash_\mathcal{L}  \SCnt{A}}
        \end{math}
      \end{center}
      By assumption, $c(\Pi_1),c(\Pi_2)\leq |B|$. By induction on $\Pi_1$
      and $\pi$, there is a proof $\Pi'$ for sequent
      $\Gamma_{{\mathrm{1}}}  \SCsym{;}  \SCnt{B}  \SCsym{;}  \Gamma_{{\mathrm{2}}}  \SCsym{;}  \SCnt{Y_{{\mathrm{1}}}}  \SCsym{;}  \SCnt{Y_{{\mathrm{2}}}}  \SCsym{;}  \Gamma_{{\mathrm{3}}}  \vdash_\mathcal{L}  \SCnt{A}$ s.t. $c(\Pi') \leq |B|$. Therefore,
      the proof $\Pi$ can be constructed as follows with
      $c(\Pi) = c(\Pi') \leq |B|$.
      \begin{center}
        \scriptsize
        \begin{math}
          $$\mprset{flushleft}
          \inferrule* [right={\tiny tenL}] {
            {
              \begin{array}{c}
                \Pi' \\
                {\Gamma_{{\mathrm{1}}}  \SCsym{;}  \Delta  \SCsym{;}  \Gamma_{{\mathrm{2}}}  \SCsym{;}  \SCnt{Y_{{\mathrm{1}}}}  \SCsym{;}  \SCnt{Y_{{\mathrm{2}}}}  \SCsym{;}  \Gamma_{{\mathrm{3}}}  \vdash_\mathcal{L}  \SCnt{A}}
              \end{array}
            }
          }{\Gamma_{{\mathrm{1}}}  \SCsym{;}  \Delta  \SCsym{;}  \Gamma_{{\mathrm{2}}}  \SCsym{;}  \SCnt{Y_{{\mathrm{1}}}}  \otimes  \SCnt{Y_{{\mathrm{2}}}}  \SCsym{;}  \Gamma_{{\mathrm{3}}}  \vdash_\mathcal{L}  \SCnt{A}}
        \end{math}
      \end{center}

\item Case 5:
      \begin{center}
        \scriptsize
        \begin{math}
          \begin{array}{c}
            \Pi_1 \\
            {\Phi  \vdash_\mathcal{C}  \SCnt{X}}
          \end{array}
        \end{math}
        \qquad\qquad
        $\Pi_2$:
        \begin{math}
          $$\mprset{flushleft}
          \inferrule* [right={\tiny tenL}] {
            {
              \begin{array}{c}
                \pi \\
                {\Gamma_{{\mathrm{1}}}  \SCsym{;}  \SCnt{Y_{{\mathrm{1}}}}  \SCsym{;}  \SCnt{Y_{{\mathrm{2}}}}  \SCsym{;}  \Gamma_{{\mathrm{2}}}  \SCsym{;}  \SCnt{X}  \SCsym{;}  \Gamma_{{\mathrm{3}}}  \vdash_\mathcal{L}  \SCnt{A}}
              \end{array}
            }
          }{\Gamma_{{\mathrm{1}}}  \SCsym{;}  \SCnt{Y_{{\mathrm{1}}}}  \otimes  \SCnt{Y_{{\mathrm{2}}}}  \SCsym{;}  \Gamma_{{\mathrm{2}}}  \SCsym{;}  \SCnt{X}  \SCsym{;}  \Gamma_{{\mathrm{3}}}  \vdash_\mathcal{L}  \SCnt{A}}
        \end{math}
      \end{center}
      By assumption, $c(\Pi_1),c(\Pi_2)\leq |X|$. By induction on $\Pi_1$
      and $\pi$, there is a proof $\Pi'$ for sequent
      $\Gamma_{{\mathrm{1}}}  \SCsym{;}  \SCnt{Y_{{\mathrm{1}}}}  \SCsym{;}  \SCnt{Y_{{\mathrm{2}}}}  \SCsym{;}  \Gamma_{{\mathrm{2}}}  \SCsym{;}  \Phi  \SCsym{;}  \Gamma_{{\mathrm{3}}}  \vdash_\mathcal{L}  \SCnt{A}$ s.t. $c(\Pi') \leq |X|$. Therefore,
      the proof $\Pi$ can be constructed as follows with
      $c(\Pi) = c(\Pi') \leq |X|$.
      \begin{center}
        \scriptsize
        \begin{math}
          $$\mprset{flushleft}
          \inferrule* [right={\tiny tenL}] {
            {
              \begin{array}{c}
                \Pi' \\
                {\Gamma_{{\mathrm{1}}}  \SCsym{;}  \SCnt{Y_{{\mathrm{1}}}}  \SCsym{;}  \SCnt{Y_{{\mathrm{2}}}}  \SCsym{;}  \Gamma_{{\mathrm{2}}}  \SCsym{;}  \Phi  \SCsym{;}  \Gamma_{{\mathrm{3}}}  \vdash_\mathcal{L}  \SCnt{A}}
              \end{array}
            }
          }{\Gamma_{{\mathrm{1}}}  \SCsym{;}  \SCnt{Y_{{\mathrm{1}}}}  \otimes  \SCnt{Y_{{\mathrm{2}}}}  \SCsym{;}  \Gamma_{{\mathrm{2}}}  \SCsym{;}  \Phi  \SCsym{;}  \Gamma_{{\mathrm{3}}}  \vdash_\mathcal{L}  \SCnt{A}}
        \end{math}
      \end{center}

\item Case 6:
      \begin{center}
        \scriptsize
        \begin{math}
          \begin{array}{c}
            \Pi_1 \\
            {\Delta  \vdash_\mathcal{L}  \SCnt{B}}
          \end{array}
        \end{math}
        \qquad\qquad
        $\Pi_2$:
        \begin{math}
          $$\mprset{flushleft}
          \inferrule* [right={\tiny tenL}] {
            {
              \begin{array}{c}
                \pi \\
                {\Gamma_{{\mathrm{1}}}  \SCsym{;}  \SCnt{Y_{{\mathrm{1}}}}  \SCsym{;}  \SCnt{Y_{{\mathrm{2}}}}  \SCsym{;}  \Gamma_{{\mathrm{2}}}  \SCsym{;}  \SCnt{B}  \SCsym{;}  \Gamma_{{\mathrm{3}}}  \vdash_\mathcal{L}  \SCnt{A}}
              \end{array}
            }
          }{\Gamma_{{\mathrm{1}}}  \SCsym{;}  \SCnt{Y_{{\mathrm{1}}}}  \otimes  \SCnt{Y_{{\mathrm{2}}}}  \SCsym{;}  \Gamma_{{\mathrm{2}}}  \SCsym{;}  \SCnt{B}  \SCsym{;}  \Gamma_{{\mathrm{3}}}  \vdash_\mathcal{L}  \SCnt{A}}
        \end{math}
      \end{center}
      By assumption, $c(\Pi_1),c(\Pi_2)\leq |B|$. By induction on $\Pi_1$
      and $\pi$, there is a proof $\Pi'$ for sequent
      $\Gamma_{{\mathrm{1}}}  \SCsym{;}  \SCnt{Y_{{\mathrm{1}}}}  \SCsym{;}  \SCnt{Y_{{\mathrm{2}}}}  \SCsym{;}  \Gamma_{{\mathrm{2}}}  \SCsym{;}  \Delta  \SCsym{;}  \Gamma_{{\mathrm{3}}}  \vdash_\mathcal{L}  \SCnt{A}$ s.t. $c(\Pi') \leq |B|$. Therefore,
      the proof $\Pi$ can be constructed as follows with
      $c(\Pi) = c(\Pi') \leq |B|$.
      \begin{center}
        \scriptsize
        \begin{math}
          $$\mprset{flushleft}
          \inferrule* [right={\tiny tenL}] {
            {
              \begin{array}{c}
                \Pi' \\
                {\Gamma_{{\mathrm{1}}}  \SCsym{;}  \SCnt{Y_{{\mathrm{1}}}}  \SCsym{;}  \SCnt{Y_{{\mathrm{2}}}}  \SCsym{;}  \Gamma_{{\mathrm{2}}}  \SCsym{;}  \Delta  \SCsym{;}  \Gamma_{{\mathrm{3}}}  \vdash_\mathcal{L}  \SCnt{A}}
              \end{array}
            }
          }{\Gamma_{{\mathrm{1}}}  \SCsym{;}  \SCnt{Y_{{\mathrm{1}}}}  \otimes  \SCnt{Y_{{\mathrm{2}}}}  \SCsym{;}  \Gamma_{{\mathrm{2}}}  \SCsym{;}  \Delta  \SCsym{;}  \Gamma_{{\mathrm{3}}}  \vdash_\mathcal{L}  \SCnt{A}}
        \end{math}
      \end{center}
\end{itemize}


\subsubsection{Left introduction of the non-commutative tensor $\tri$ (with low priority)}:
\begin{itemize}
\item Case 1:
      \begin{center}
        \scriptsize
        \begin{math}
          \begin{array}{c}
            \Pi_1 \\
            {\Phi  \vdash_\mathcal{C}  \SCnt{X}}
          \end{array}
        \end{math}
        \qquad\qquad
        $\Pi_2$:
        \begin{math}
          $$\mprset{flushleft}
          \inferrule* [right={\tiny tenL}] {
            {
              \begin{array}{c}
                \pi \\
                {\Gamma_{{\mathrm{1}}}  \SCsym{;}  \SCnt{X}  \SCsym{;}  \Gamma_{{\mathrm{2}}}  \SCsym{;}  \SCnt{A_{{\mathrm{1}}}}  \SCsym{;}  \SCnt{A_{{\mathrm{2}}}}  \SCsym{;}  \Gamma_{{\mathrm{3}}}  \vdash_\mathcal{L}  \SCnt{B}}
              \end{array}
            }
          }{\Gamma_{{\mathrm{1}}}  \SCsym{;}  \SCnt{X}  \SCsym{;}  \Gamma_{{\mathrm{2}}}  \SCsym{;}  \SCnt{A_{{\mathrm{1}}}}  \triangleright  \SCnt{A_{{\mathrm{2}}}}  \SCsym{;}  \Gamma_{{\mathrm{3}}}  \vdash_\mathcal{L}  \SCnt{B}}
        \end{math}
      \end{center}
      By assumption, $c(\Pi_1),c(\Pi_2)\leq |X|$. By induction on $\Pi_1$
      and $\pi$, there is a proof $\Pi'$ for sequent
      $\Gamma_{{\mathrm{1}}}  \SCsym{;}  \Phi  \SCsym{;}  \Gamma_{{\mathrm{2}}}  \SCsym{;}  \SCnt{A_{{\mathrm{1}}}}  \SCsym{;}  \SCnt{A_{{\mathrm{2}}}}  \SCsym{;}  \Gamma_{{\mathrm{3}}}  \vdash_\mathcal{L}  \SCnt{B}$ s.t. $c(\Pi') \leq |X|$. Therefore,
      the proof $\Pi$ can be constructed as follows with
      $c(\Pi) = c(\Pi') \leq |X|$.
      \begin{center}
        \scriptsize
        \begin{math}
          $$\mprset{flushleft}
          \inferrule* [right={\tiny tenL}] {
            {
              \begin{array}{c}
                \Pi' \\
                {\Gamma_{{\mathrm{1}}}  \SCsym{;}  \Phi  \SCsym{;}  \Gamma_{{\mathrm{2}}}  \SCsym{;}  \SCnt{A_{{\mathrm{1}}}}  \SCsym{;}  \SCnt{A_{{\mathrm{2}}}}  \SCsym{;}  \Gamma_{{\mathrm{3}}}  \vdash_\mathcal{L}  \SCnt{B}}
              \end{array}
            }
          }{\Gamma_{{\mathrm{1}}}  \SCsym{;}  \Phi  \SCsym{;}  \Gamma_{{\mathrm{2}}}  \SCsym{;}  \SCnt{A_{{\mathrm{1}}}}  \triangleright  \SCnt{A_{{\mathrm{2}}}}  \SCsym{;}  \Gamma_{{\mathrm{3}}}  \vdash_\mathcal{L}  \SCnt{B}}
        \end{math}
      \end{center}

\item Case 2:
      \begin{center}
        \scriptsize
        \begin{math}
          \begin{array}{c}
            \Pi_1 \\
            {\Delta  \vdash_\mathcal{L}  \SCnt{B}}
          \end{array}
        \end{math}
        \qquad\qquad
        $\Pi_2$:
        \begin{math}
          $$\mprset{flushleft}
          \inferrule* [right={\tiny tenL}] {
            {
              \begin{array}{c}
                \pi \\
                {\Gamma_{{\mathrm{1}}}  \SCsym{;}  \SCnt{B}  \SCsym{;}  \Gamma_{{\mathrm{2}}}  \SCsym{;}  \SCnt{A_{{\mathrm{1}}}}  \SCsym{;}  \SCnt{A_{{\mathrm{2}}}}  \SCsym{;}  \Gamma_{{\mathrm{3}}}  \vdash_\mathcal{L}  \SCnt{C}}
              \end{array}
            }
          }{\Gamma_{{\mathrm{1}}}  \SCsym{;}  \SCnt{B}  \SCsym{;}  \Gamma_{{\mathrm{2}}}  \SCsym{;}  \SCnt{A_{{\mathrm{1}}}}  \triangleright  \SCnt{A_{{\mathrm{2}}}}  \SCsym{;}  \Gamma_{{\mathrm{3}}}  \vdash_\mathcal{L}  \SCnt{C}}
        \end{math}
      \end{center}
      By assumption, $c(\Pi_1),c(\Pi_2)\leq |B|$. By induction on $\Pi_1$
      and $\pi$, there is a proof $\Pi'$ for sequent
      $\Gamma_{{\mathrm{1}}}  \SCsym{;}  \Delta  \SCsym{;}  \Gamma_{{\mathrm{2}}}  \SCsym{;}  \SCnt{A_{{\mathrm{1}}}}  \SCsym{;}  \SCnt{A_{{\mathrm{2}}}}  \SCsym{;}  \Gamma_{{\mathrm{3}}}  \vdash_\mathcal{L}  \SCnt{C}$ s.t. $c(\Pi') \leq |B|$. Therefore,
      the proof $\Pi$ can be constructed as follows with
      $c(\Pi) = c(\Pi') \leq |B|$.
      \begin{center}
        \scriptsize
        \begin{math}
          $$\mprset{flushleft}
          \inferrule* [right={\tiny tenL}] {
            {
              \begin{array}{c}
                \Pi' \\
                {\Gamma_{{\mathrm{1}}}  \SCsym{;}  \Delta  \SCsym{;}  \Gamma_{{\mathrm{2}}}  \SCsym{;}  \SCnt{A_{{\mathrm{1}}}}  \SCsym{;}  \SCnt{A_{{\mathrm{2}}}}  \SCsym{;}  \Gamma_{{\mathrm{3}}}  \vdash_\mathcal{L}  \SCnt{C}}
              \end{array}
            }
          }{\Gamma_{{\mathrm{1}}}  \SCsym{;}  \Delta  \SCsym{;}  \Gamma_{{\mathrm{2}}}  \SCsym{;}  \SCnt{A_{{\mathrm{1}}}}  \triangleright  \SCnt{A_{{\mathrm{2}}}}  \SCsym{;}  \Gamma_{{\mathrm{3}}}  \vdash_\mathcal{L}  \SCnt{C}}
        \end{math}
      \end{center}

\item Case 3:
      \begin{center}
        \scriptsize
        \begin{math}
          \begin{array}{c}
            \Pi_1 \\
            {\Phi  \vdash_\mathcal{C}  \SCnt{X}}
          \end{array}
        \end{math}
        \qquad\qquad
        $\Pi_2$:
        \begin{math}
          $$\mprset{flushleft}
          \inferrule* [right={\tiny tenL}] {
            {
              \begin{array}{c}
                \pi \\
                {\Gamma_{{\mathrm{1}}}  \SCsym{;}  \SCnt{A_{{\mathrm{1}}}}  \SCsym{;}  \SCnt{A_{{\mathrm{2}}}}  \SCsym{;}  \Gamma_{{\mathrm{2}}}  \SCsym{;}  \SCnt{X}  \SCsym{;}  \Gamma_{{\mathrm{3}}}  \vdash_\mathcal{L}  \SCnt{B}}
              \end{array}
            }
          }{\Gamma_{{\mathrm{1}}}  \SCsym{;}  \SCnt{A_{{\mathrm{1}}}}  \triangleright  \SCnt{A_{{\mathrm{2}}}}  \SCsym{;}  \Gamma_{{\mathrm{2}}}  \SCsym{;}  \SCnt{X}  \SCsym{;}  \Gamma_{{\mathrm{3}}}  \vdash_\mathcal{L}  \SCnt{B}}
        \end{math}
      \end{center}
      By assumption, $c(\Pi_1),c(\Pi_2)\leq |X|$. By induction on $\Pi_1$
      and $\pi$, there is a proof $\Pi'$ for sequent
      $\Gamma_{{\mathrm{1}}}  \SCsym{;}  \SCnt{A_{{\mathrm{1}}}}  \SCsym{;}  \SCnt{A_{{\mathrm{2}}}}  \SCsym{;}  \Gamma_{{\mathrm{2}}}  \SCsym{;}  \Phi  \SCsym{;}  \Gamma_{{\mathrm{3}}}  \vdash_\mathcal{L}  \SCnt{A}$ s.t. $c(\Pi') \leq |X|$. Therefore,
      the proof $\Pi$ can be constructed as follows with
      $c(\Pi) = c(\Pi') \leq |X|$.
      \begin{center}
        \scriptsize
        \begin{math}
          $$\mprset{flushleft}
          \inferrule* [right={\tiny tenL}] {
            {
              \begin{array}{c}
                \Pi' \\
                {\Gamma_{{\mathrm{1}}}  \SCsym{;}  \SCnt{A_{{\mathrm{1}}}}  \SCsym{;}  \SCnt{A_{{\mathrm{2}}}}  \SCsym{;}  \Gamma_{{\mathrm{2}}}  \SCsym{;}  \Phi  \SCsym{;}  \Gamma_{{\mathrm{3}}}  \vdash_\mathcal{L}  \SCnt{B}}
              \end{array}
            }
          }{\Gamma_{{\mathrm{1}}}  \SCsym{;}  \SCnt{A_{{\mathrm{1}}}}  \triangleright  \SCnt{A_{{\mathrm{2}}}}  \SCsym{;}  \Gamma_{{\mathrm{2}}}  \SCsym{;}  \Phi  \SCsym{;}  \Gamma_{{\mathrm{3}}}  \vdash_\mathcal{L}  \SCnt{B}}
        \end{math}
      \end{center}

\item Case 4:
      \begin{center}
        \scriptsize
        \begin{math}
          \begin{array}{c}
            \Pi_1 \\
            {\Delta  \vdash_\mathcal{L}  \SCnt{B}}
          \end{array}
        \end{math}
        \qquad\qquad
        $\Pi_2$:
        \begin{math}
          $$\mprset{flushleft}
          \inferrule* [right={\tiny tenL}] {
            {
              \begin{array}{c}
                \pi \\
                {\Gamma_{{\mathrm{1}}}  \SCsym{;}  \SCnt{A_{{\mathrm{1}}}}  \SCsym{;}  \SCnt{A_{{\mathrm{2}}}}  \SCsym{;}  \Gamma_{{\mathrm{2}}}  \SCsym{;}  \SCnt{B}  \SCsym{;}  \Gamma_{{\mathrm{3}}}  \vdash_\mathcal{L}  \SCnt{C}}
              \end{array}
            }
          }{\Gamma_{{\mathrm{1}}}  \SCsym{;}  \SCnt{A_{{\mathrm{1}}}}  \triangleright  \SCnt{A_{{\mathrm{2}}}}  \SCsym{;}  \Gamma_{{\mathrm{2}}}  \SCsym{;}  \SCnt{B}  \SCsym{;}  \Gamma_{{\mathrm{3}}}  \vdash_\mathcal{L}  \SCnt{C}}
        \end{math}
      \end{center}
      By assumption, $c(\Pi_1),c(\Pi_2)\leq |B|$. By induction on $\Pi_1$
      and $\pi$, there is a proof $\Pi'$ for sequent
      $\Gamma_{{\mathrm{1}}}  \SCsym{;}  \SCnt{A_{{\mathrm{1}}}}  \SCsym{;}  \SCnt{A_{{\mathrm{2}}}}  \SCsym{;}  \Gamma_{{\mathrm{2}}}  \SCsym{;}  \Delta  \SCsym{;}  \Gamma_{{\mathrm{3}}}  \vdash_\mathcal{L}  \SCnt{C}$ s.t. $c(\Pi') \leq |B|$. Therefore,
      the proof $\Pi$ can be constructed as follows with
      $c(\Pi) = c(\Pi') \leq |B|$.
      \begin{center}
        \scriptsize
        \begin{math}
          $$\mprset{flushleft}
          \inferrule* [right={\tiny tenL}] {
            {
              \begin{array}{c}
                \Pi' \\
                {\Gamma_{{\mathrm{1}}}  \SCsym{;}  \SCnt{A_{{\mathrm{1}}}}  \SCsym{;}  \SCnt{A_{{\mathrm{2}}}}  \SCsym{;}  \Gamma_{{\mathrm{2}}}  \SCsym{;}  \Delta  \SCsym{;}  \Gamma_{{\mathrm{3}}}  \vdash_\mathcal{L}  \SCnt{C}}
              \end{array}
            }
          }{\Gamma_{{\mathrm{1}}}  \SCsym{;}  \SCnt{A_{{\mathrm{1}}}}  \triangleright  \SCnt{A_{{\mathrm{2}}}}  \SCsym{;}  \Gamma_{{\mathrm{2}}}  \SCsym{;}  \Delta  \SCsym{;}  \Gamma_{{\mathrm{3}}}  \vdash_\mathcal{L}  \SCnt{C}}
        \end{math}
      \end{center}
\end{itemize}



\subsubsection{$\SCdruleTXXexName$}
\begin{itemize}
\item Case 1:
      \begin{center}
        \scriptsize
        \begin{math}
          \begin{array}{c}
            \Pi_1 \\
            {\Phi  \vdash_\mathcal{C}  \SCnt{X}}
          \end{array}
        \end{math}
        \qquad\qquad
        $\Pi_2$:
        \begin{math}
          $$\mprset{flushleft}
          \inferrule* [right={\tiny beta}] {
            {
              \begin{array}{c}
                \pi \\
                {\Psi_{{\mathrm{1}}}  \SCsym{,}  \SCnt{X}  \SCsym{,}  \Psi_{{\mathrm{2}}}  \SCsym{,}  \SCnt{Y_{{\mathrm{1}}}}  \SCsym{,}  \SCnt{Y_{{\mathrm{2}}}}  \SCsym{,}  \Psi_{{\mathrm{3}}}  \vdash_\mathcal{C}  \SCnt{Z}}
              \end{array}
            }
          }{\Psi_{{\mathrm{1}}}  \SCsym{,}  \SCnt{X}  \SCsym{,}  \Psi_{{\mathrm{2}}}  \SCsym{,}  \SCnt{Y_{{\mathrm{2}}}}  \SCsym{,}  \SCnt{Y_{{\mathrm{1}}}}  \SCsym{,}  \Psi_{{\mathrm{3}}}  \vdash_\mathcal{C}  \SCnt{Z}}
        \end{math}
      \end{center}
      By assumption, $c(\Pi_1),c(\Pi_2)\leq |X|$. By induction on $\Pi_1$
      and $\pi$, there is a proof $\Pi'$ for sequent
      $\Psi_{{\mathrm{1}}}  \SCsym{,}  \Phi  \SCsym{,}  \Psi_{{\mathrm{2}}}  \SCsym{,}  \SCnt{Y_{{\mathrm{1}}}}  \SCsym{,}  \SCnt{Y_{{\mathrm{2}}}}  \SCsym{,}  \Psi_{{\mathrm{3}}}  \vdash_\mathcal{C}  \SCnt{Z}$ s.t. $c(\Pi') \leq |X|$. Therefore,
      the proof $\Pi$ can be constructed as follows with
      $c(\Pi) = c(\Pi') \leq |X|$.
      \begin{center}
        \scriptsize
        \begin{math}
          $$\mprset{flushleft}
          \inferrule* [right={\tiny cut}] {
            {
              \begin{array}{cc}
                \Pi' \\
                {\Psi_{{\mathrm{1}}}  \SCsym{,}  \Phi  \SCsym{,}  \Psi_{{\mathrm{2}}}  \SCsym{,}  \SCnt{Y_{{\mathrm{1}}}}  \SCsym{,}  \SCnt{Y_{{\mathrm{2}}}}  \SCsym{,}  \Psi_{{\mathrm{3}}}  \vdash_\mathcal{C}  \SCnt{Z}}
              \end{array}
            }
          }{\Psi_{{\mathrm{1}}}  \SCsym{,}  \Phi  \SCsym{,}  \Psi_{{\mathrm{2}}}  \SCsym{,}  \SCnt{Y_{{\mathrm{2}}}}  \SCsym{,}  \SCnt{Y_{{\mathrm{1}}}}  \SCsym{,}  \Psi_{{\mathrm{3}}}  \vdash_\mathcal{C}  \SCnt{Z}}
        \end{math}
      \end{center}

\item Case 2:
      \begin{center}
        \scriptsize
        \begin{math}
          \begin{array}{c}
            \Pi_1 \\
            {\Phi  \vdash_\mathcal{C}  \SCnt{X}}
          \end{array}
        \end{math}
        \qquad\qquad
        $\Pi_2$:
        \begin{math}
          $$\mprset{flushleft}
          \inferrule* [right={\tiny beta}] {
            {
              \begin{array}{c}
                \pi \\
                {\Psi_{{\mathrm{1}}}  \SCsym{,}  \SCnt{Y_{{\mathrm{1}}}}  \SCsym{,}  \SCnt{Y_{{\mathrm{2}}}}  \SCsym{,}  \Psi_{{\mathrm{2}}}  \SCsym{,}  \SCnt{X}  \SCsym{,}  \Psi_{{\mathrm{3}}}  \vdash_\mathcal{C}  \SCnt{Z}}
              \end{array}
            }
          }{\Psi_{{\mathrm{1}}}  \SCsym{,}  \SCnt{X}  \SCsym{,}  \Psi_{{\mathrm{2}}}  \SCsym{,}  \SCnt{Y_{{\mathrm{2}}}}  \SCsym{,}  \SCnt{Y_{{\mathrm{1}}}}  \SCsym{,}  \Psi_{{\mathrm{3}}}  \vdash_\mathcal{C}  \SCnt{Z}}
        \end{math}
      \end{center}
      By assumption, $c(\Pi_1),c(\Pi_2)\leq |X|$. By induction on $\Pi_1$
      and $\pi$, there is a proof $\Pi'$ for sequent
      $\Psi_{{\mathrm{1}}}  \SCsym{,}  \SCnt{Y_{{\mathrm{1}}}}  \SCsym{,}  \SCnt{Y_{{\mathrm{2}}}}  \SCsym{,}  \Psi_{{\mathrm{2}}}  \SCsym{,}  \Phi  \SCsym{,}  \Psi_{{\mathrm{3}}}  \vdash_\mathcal{C}  \SCnt{Z}$ s.t. $c(\Pi') \leq |X|$. Therefore,
      the proof $\Pi$ can be constructed as follows with
      $c(\Pi) = c(\Pi') \leq |X|$.
      \begin{center}
        \scriptsize
        \begin{math}
          $$\mprset{flushleft}
          \inferrule* [right={\tiny cut}] {
            {
              \begin{array}{cc}
                \Pi' \\
                {\Psi_{{\mathrm{1}}}  \SCsym{,}  \SCnt{Y_{{\mathrm{1}}}}  \SCsym{,}  \SCnt{Y_{{\mathrm{2}}}}  \SCsym{,}  \Psi_{{\mathrm{2}}}  \SCsym{,}  \Phi  \SCsym{,}  \Psi_{{\mathrm{3}}}  \vdash_\mathcal{C}  \SCnt{Z}}
              \end{array}
            }
          }{\Psi_{{\mathrm{1}}}  \SCsym{,}  \SCnt{Y_{{\mathrm{2}}}}  \SCsym{,}  \SCnt{Y_{{\mathrm{1}}}}  \SCsym{,}  \Psi_{{\mathrm{2}}}  \SCsym{,}  \Phi  \SCsym{,}  \Psi_{{\mathrm{3}}}  \vdash_\mathcal{C}  \SCnt{Z}}
        \end{math}
      \end{center}
\end{itemize}


\subsubsection{$\SCdruleSXXexName$}
\begin{itemize}
\item Case 1:
      \begin{center}
        \scriptsize
        \begin{math}
          \begin{array}{c}
            \Pi_1 \\
            {\Phi  \vdash_\mathcal{C}  \SCnt{X}}
          \end{array}
        \end{math}
        \qquad\qquad
        $\Pi_2$:
        \begin{math}
          $$\mprset{flushleft}
          \inferrule* [right={\tiny beta}] {
            {
              \begin{array}{c}
                \pi \\
                {\Gamma_{{\mathrm{1}}}  \SCsym{;}  \SCnt{X}  \SCsym{;}  \Gamma_{{\mathrm{2}}}  \SCsym{;}  \SCnt{Y_{{\mathrm{1}}}}  \SCsym{;}  \SCnt{Y_{{\mathrm{2}}}}  \SCsym{;}  \Gamma_{{\mathrm{3}}}  \vdash_\mathcal{L}  \SCnt{A}}
              \end{array}
            }
          }{\Gamma_{{\mathrm{1}}}  \SCsym{;}  \SCnt{X}  \SCsym{;}  \Gamma_{{\mathrm{2}}}  \SCsym{;}  \SCnt{Y_{{\mathrm{2}}}}  \SCsym{;}  \SCnt{Y_{{\mathrm{1}}}}  \SCsym{;}  \Gamma_{{\mathrm{3}}}  \vdash_\mathcal{L}  \SCnt{A}}
        \end{math}
      \end{center}
      By assumption, $c(\Pi_1),c(\Pi_2)\leq |X|$. By induction on $\Pi_1$
      and $\pi$, there is a proof $\Pi'$ for sequent
      $\Gamma_{{\mathrm{1}}}  \SCsym{;}  \Phi  \SCsym{;}  \Gamma_{{\mathrm{2}}}  \SCsym{;}  \SCnt{Y_{{\mathrm{1}}}}  \SCsym{;}  \SCnt{Y_{{\mathrm{2}}}}  \SCsym{;}  \Gamma_{{\mathrm{3}}}  \vdash_\mathcal{L}  \SCnt{A}$ s.t. $c(\Pi') \leq |X|$. Therefore,
      the proof $\Pi$ can be constructed as follows with
      $c(\Pi) = c(\Pi') \leq |X|$.
      \begin{center}
        \scriptsize
        \begin{math}
          $$\mprset{flushleft}
          \inferrule* [right={\tiny cut}] {
            {
              \begin{array}{cc}
                \Pi' \\
                {\Gamma_{{\mathrm{1}}}  \SCsym{;}  \Phi  \SCsym{;}  \Gamma_{{\mathrm{2}}}  \SCsym{;}  \SCnt{Y_{{\mathrm{1}}}}  \SCsym{;}  \SCnt{Y_{{\mathrm{2}}}}  \SCsym{;}  \Gamma_{{\mathrm{3}}}  \vdash_\mathcal{L}  \SCnt{A}}
              \end{array}
            }
          }{\Gamma_{{\mathrm{1}}}  \SCsym{;}  \Phi  \SCsym{;}  \Gamma_{{\mathrm{2}}}  \SCsym{;}  \SCnt{Y_{{\mathrm{2}}}}  \SCsym{;}  \SCnt{Y_{{\mathrm{1}}}}  \SCsym{;}  \Gamma_{{\mathrm{3}}}  \vdash_\mathcal{L}  \SCnt{A}}
        \end{math}
      \end{center}

\item Case 2:
      \begin{center}
        \scriptsize
        \begin{math}
          \begin{array}{c}
            \Pi_1 \\
            {\Delta  \vdash_\mathcal{L}  \SCnt{B}}
          \end{array}
        \end{math}
        \qquad\qquad
        $\Pi_2$:
        \begin{math}
          $$\mprset{flushleft}
          \inferrule* [right={\tiny beta}] {
            {
              \begin{array}{c}
                \pi \\
                {\Gamma_{{\mathrm{1}}}  \SCsym{;}  \SCnt{B}  \SCsym{;}  \Gamma_{{\mathrm{2}}}  \SCsym{;}  \SCnt{Y_{{\mathrm{1}}}}  \SCsym{;}  \SCnt{Y_{{\mathrm{2}}}}  \SCsym{;}  \Gamma_{{\mathrm{3}}}  \vdash_\mathcal{L}  \SCnt{A}}
              \end{array}
            }
          }{\Gamma_{{\mathrm{1}}}  \SCsym{;}  \SCnt{B}  \SCsym{;}  \Gamma_{{\mathrm{2}}}  \SCsym{;}  \SCnt{Y_{{\mathrm{2}}}}  \SCsym{;}  \SCnt{Y_{{\mathrm{1}}}}  \SCsym{;}  \Gamma_{{\mathrm{3}}}  \vdash_\mathcal{L}  \SCnt{A}}
        \end{math}
      \end{center}
      By assumption, $c(\Pi_1),c(\Pi_2)\leq |X|$. By induction on $\Pi_1$
      and $\pi$, there is a proof $\Pi'$ for sequent
      $\Gamma_{{\mathrm{1}}}  \SCsym{;}  \Delta  \SCsym{;}  \Gamma_{{\mathrm{2}}}  \SCsym{;}  \SCnt{Y_{{\mathrm{1}}}}  \SCsym{;}  \SCnt{Y_{{\mathrm{2}}}}  \SCsym{;}  \Gamma_{{\mathrm{3}}}  \vdash_\mathcal{L}  \SCnt{A}$ s.t. $c(\Pi') \leq |X|$. Therefore,
      the proof $\Pi$ can be constructed as follows with
      $c(\Pi) = c(\Pi') \leq |X|$.
      \begin{center}
        \scriptsize
        \begin{math}
          $$\mprset{flushleft}
          \inferrule* [right={\tiny cut}] {
            {
              \begin{array}{cc}
                \Pi' \\
                {\Gamma_{{\mathrm{1}}}  \SCsym{;}  \Delta  \SCsym{;}  \Gamma_{{\mathrm{2}}}  \SCsym{;}  \SCnt{Y_{{\mathrm{1}}}}  \SCsym{;}  \SCnt{Y_{{\mathrm{2}}}}  \SCsym{;}  \Gamma_{{\mathrm{3}}}  \vdash_\mathcal{L}  \SCnt{A}}
              \end{array}
            }
          }{\Gamma_{{\mathrm{1}}}  \SCsym{;}  \Delta  \SCsym{;}  \Gamma_{{\mathrm{2}}}  \SCsym{;}  \SCnt{Y_{{\mathrm{2}}}}  \SCsym{;}  \SCnt{Y_{{\mathrm{1}}}}  \SCsym{;}  \Gamma_{{\mathrm{3}}}  \vdash_\mathcal{L}  \SCnt{A}}
        \end{math}
      \end{center}

\item Case 3:
      \begin{center}
        \scriptsize
        \begin{math}
          \begin{array}{c}
            \Pi_1 \\
            {\Phi  \vdash_\mathcal{C}  \SCnt{X}}
          \end{array}
        \end{math}
        \qquad\qquad
        $\Pi_2$:
        \begin{math}
          $$\mprset{flushleft}
          \inferrule* [right={\tiny beta}] {
            {
              \begin{array}{c}
                \pi \\
                {\Gamma_{{\mathrm{1}}}  \SCsym{;}  \SCnt{Y_{{\mathrm{1}}}}  \SCsym{;}  \SCnt{Y_{{\mathrm{2}}}}  \SCsym{;}  \Gamma_{{\mathrm{2}}}  \SCsym{;}  \SCnt{X}  \SCsym{;}  \Gamma_{{\mathrm{3}}}  \vdash_\mathcal{L}  \SCnt{A}}
              \end{array}
            }
          }{\Gamma_{{\mathrm{1}}}  \SCsym{;}  \SCnt{X}  \SCsym{;}  \Gamma_{{\mathrm{2}}}  \SCsym{;}  \SCnt{Y_{{\mathrm{2}}}}  \SCsym{;}  \SCnt{Y_{{\mathrm{1}}}}  \SCsym{;}  \Gamma_{{\mathrm{3}}}  \vdash_\mathcal{L}  \SCnt{A}}
        \end{math}
      \end{center}
      By assumption, $c(\Pi_1),c(\Pi_2)\leq |X|$. By induction on $\Pi_1$
      and $\pi$, there is a proof $\Pi'$ for sequent
      $\Gamma_{{\mathrm{1}}}  \SCsym{;}  \SCnt{Y_{{\mathrm{1}}}}  \SCsym{;}  \SCnt{Y_{{\mathrm{2}}}}  \SCsym{;}  \Gamma_{{\mathrm{2}}}  \SCsym{;}  \Phi  \SCsym{;}  \Gamma_{{\mathrm{3}}}  \vdash_\mathcal{L}  \SCnt{A}$ s.t. $c(\Pi') \leq |X|$. Therefore,
      the proof $\Pi$ can be constructed as follows with
      $c(\Pi) = c(\Pi') \leq |X|$.
      \begin{center}
        \scriptsize
        \begin{math}
          $$\mprset{flushleft}
          \inferrule* [right={\tiny cut}] {
            {
              \begin{array}{cc}
                \Pi' \\
                {\Gamma_{{\mathrm{1}}}  \SCsym{;}  \SCnt{Y_{{\mathrm{1}}}}  \SCsym{;}  \SCnt{Y_{{\mathrm{2}}}}  \SCsym{;}  \Gamma_{{\mathrm{2}}}  \SCsym{;}  \Phi  \SCsym{;}  \Gamma_{{\mathrm{3}}}  \vdash_\mathcal{L}  \SCnt{A}}
              \end{array}
            }
          }{\Gamma_{{\mathrm{1}}}  \SCsym{;}  \SCnt{Y_{{\mathrm{2}}}}  \SCsym{;}  \SCnt{Y_{{\mathrm{1}}}}  \SCsym{;}  \Gamma_{{\mathrm{2}}}  \SCsym{;}  \Phi  \SCsym{;}  \Gamma_{{\mathrm{3}}}  \vdash_\mathcal{L}  \SCnt{A}}
        \end{math}
      \end{center}

\item Case 4:
      \begin{center}
        \scriptsize
        \begin{math}
          \begin{array}{c}
            \Pi_1 \\
            {\Delta  \vdash_\mathcal{L}  \SCnt{B}}
          \end{array}
        \end{math}
        \qquad\qquad
        $\Pi_2$:
        \begin{math}
          $$\mprset{flushleft}
          \inferrule* [right={\tiny beta}] {
            {
              \begin{array}{c}
                \pi \\
                {\Gamma_{{\mathrm{1}}}  \SCsym{;}  \SCnt{Y_{{\mathrm{1}}}}  \SCsym{;}  \SCnt{Y_{{\mathrm{2}}}}  \SCsym{;}  \Gamma_{{\mathrm{2}}}  \SCsym{;}  \SCnt{B}  \SCsym{;}  \Gamma_{{\mathrm{3}}}  \vdash_\mathcal{L}  \SCnt{A}}
              \end{array}
            }
          }{\Gamma_{{\mathrm{1}}}  \SCsym{;}  \SCnt{Y_{{\mathrm{2}}}}  \SCsym{;}  \SCnt{Y_{{\mathrm{1}}}}  \SCsym{;}  \Gamma_{{\mathrm{2}}}  \SCsym{;}  \SCnt{B}  \SCsym{;}  \Gamma_{{\mathrm{3}}}  \vdash_\mathcal{L}  \SCnt{A}}
        \end{math}
      \end{center}
      By assumption, $c(\Pi_1),c(\Pi_2)\leq |X|$. By induction on $\Pi_1$
      and $\pi$, there is a proof $\Pi'$ for sequent
      $\Gamma_{{\mathrm{1}}}  \SCsym{;}  \SCnt{Y_{{\mathrm{1}}}}  \SCsym{;}  \SCnt{Y_{{\mathrm{2}}}}  \SCsym{;}  \Gamma_{{\mathrm{2}}}  \SCsym{;}  \Delta  \SCsym{;}  \Gamma_{{\mathrm{3}}}  \vdash_\mathcal{L}  \SCnt{A}$ s.t. $c(\Pi') \leq |X|$. Therefore,
      the proof $\Pi$ can be constructed as follows with
      $c(\Pi) = c(\Pi') \leq |X|$.
      \begin{center}
        \scriptsize
        \begin{math}
          $$\mprset{flushleft}
          \inferrule* [right={\tiny cut}] {
            {
              \begin{array}{cc}
                \Pi' \\
                {\Gamma_{{\mathrm{1}}}  \SCsym{;}  \SCnt{Y_{{\mathrm{1}}}}  \SCsym{;}  \SCnt{Y_{{\mathrm{2}}}}  \SCsym{;}  \Gamma_{{\mathrm{2}}}  \SCsym{;}  \Delta  \SCsym{;}  \Gamma_{{\mathrm{3}}}  \vdash_\mathcal{L}  \SCnt{A}}
              \end{array}
            }
          }{\Gamma_{{\mathrm{1}}}  \SCsym{;}  \SCnt{Y_{{\mathrm{2}}}}  \SCsym{;}  \SCnt{Y_{{\mathrm{1}}}}  \SCsym{;}  \Gamma_{{\mathrm{2}}}  \SCsym{;}  \Delta  \SCsym{;}  \Gamma_{{\mathrm{3}}}  \vdash_\mathcal{L}  \SCnt{A}}
        \end{math}
      \end{center}
\end{itemize}



\subsubsection{Left introduction of the commutative unit $ \mathsf{Unit} $ (with low priority)}
\begin{itemize}
\item Case 1:
      \begin{center}
        \scriptsize
        \begin{math}
          \begin{array}{c}
            \Pi_1 \\
            {\Psi  \vdash_\mathcal{C}  \SCnt{X}}
          \end{array}
        \end{math}
        \qquad\qquad
        $\Pi_2$:
        \begin{math}
          $$\mprset{flushleft}
          \inferrule* [right={\tiny unitL}] {
            {
              \begin{array}{c}
                \pi \\
                {\Phi_{{\mathrm{1}}}  \SCsym{,}  \Phi_{{\mathrm{2}}}  \SCsym{,}  \SCnt{X}  \SCsym{,}  \Phi_{{\mathrm{3}}}  \vdash_\mathcal{C}  \SCnt{Y}}
              \end{array}
            }
          }{\Phi_{{\mathrm{1}}}  \SCsym{,}   \mathsf{Unit}   \SCsym{,}  \Phi_{{\mathrm{2}}}  \SCsym{,}  \SCnt{X}  \SCsym{,}  \Phi_{{\mathrm{3}}}  \vdash_\mathcal{C}  \SCnt{Y}}
        \end{math}
      \end{center}
      By assumption, $c(\Pi_1),c(\Pi_2)\leq |X|$. By induction on $\Pi_1$
      and $\pi$, there is a proof $\Pi'$ for sequent
      $\Phi_{{\mathrm{1}}}  \SCsym{,}  \Phi_{{\mathrm{2}}}  \SCsym{,}  \Psi  \SCsym{,}  \Phi_{{\mathrm{3}}}  \vdash_\mathcal{C}  \SCnt{Y}$
      s.t. $c(\Pi') \leq |X|$. Therefore, the proof $\Pi$ can be
      constructed as follows with $c(\Pi) = c(\Pi') \leq |X|$.
      \begin{center}
        \scriptsize
        \begin{math}
          $$\mprset{flushleft}
          \inferrule* [right={\tiny unitL}] {
            {
              \begin{array}{c}
                \Pi' \\
                {\Phi_{{\mathrm{1}}}  \SCsym{,}  \Phi_{{\mathrm{2}}}  \SCsym{,}  \Psi  \SCsym{,}  \Phi_{{\mathrm{3}}}  \vdash_\mathcal{C}  \SCnt{Y}}
              \end{array}
            }
          }{\Phi_{{\mathrm{1}}}  \SCsym{,}   \mathsf{Unit}   \SCsym{,}  \Phi_{{\mathrm{2}}}  \SCsym{,}  \Psi  \SCsym{,}  \Phi_{{\mathrm{3}}}  \vdash_\mathcal{C}  \SCnt{Y}}
        \end{math}
      \end{center}

\item Case 2:
      \begin{center}
        \scriptsize
        \begin{math}
          \begin{array}{c}
            \Pi_1 \\
            {\Phi  \vdash_\mathcal{C}  \SCnt{X}}
          \end{array}
        \end{math}
        \qquad\qquad
        $\Pi_2$:
        \begin{math}
          $$\mprset{flushleft}
          \inferrule* [right={\tiny unitL1}] {
            {
              \begin{array}{c}
                \pi \\
                {\Gamma_{{\mathrm{1}}}  \SCsym{;}  \Gamma_{{\mathrm{2}}}  \SCsym{;}  \SCnt{X}  \SCsym{;}  \Gamma_{{\mathrm{3}}}  \vdash_\mathcal{L}  \SCnt{A}}
              \end{array}
            }
          }{\Gamma_{{\mathrm{1}}}  \SCsym{;}   \mathsf{Unit}   \SCsym{;}  \Gamma_{{\mathrm{2}}}  \SCsym{;}  \SCnt{X}  \SCsym{;}  \Gamma_{{\mathrm{3}}}  \vdash_\mathcal{L}  \SCnt{A}}
        \end{math}
      \end{center}
      By assumption, $c(\Pi_1),c(\Pi_2)\leq |X|$. By induction on $\Pi_1$
      and $\pi$, there is a proof $\Pi'$ for sequent
      $\Gamma_{{\mathrm{1}}}  \SCsym{;}  \Gamma_{{\mathrm{2}}}  \SCsym{;}  \Phi  \SCsym{;}  \Gamma_{{\mathrm{2}}}  \vdash_\mathcal{L}  \SCnt{A}$
      s.t. $c(\Pi') \leq |X|$. Therefore, the proof $\Pi$ can be
      constructed as follows with $c(\Pi) = c(\Pi') \leq |X|$.
      \begin{center}
        \scriptsize
        \begin{math}
          $$\mprset{flushleft}
          \inferrule* [right={\tiny unitL1}] {
            {
              \begin{array}{c}
                \Pi' \\
                {\Gamma_{{\mathrm{1}}}  \SCsym{;}  \Gamma_{{\mathrm{2}}}  \SCsym{;}  \Phi  \SCsym{;}  \Gamma_{{\mathrm{3}}}  \vdash_\mathcal{L}  \SCnt{A}}
              \end{array}
            }
          }{\Gamma_{{\mathrm{1}}}  \SCsym{;}   \mathsf{Unit}   \SCsym{;}  \Gamma_{{\mathrm{2}}}  \SCsym{;}  \Phi  \SCsym{;}  \Gamma_{{\mathrm{3}}}  \vdash_\mathcal{L}  \SCnt{A}}
        \end{math}
      \end{center}

\item Case 3:
      \begin{center}
        \scriptsize
        \begin{math}
          \begin{array}{c}
            \Pi_1 \\
            {\Delta  \vdash_\mathcal{L}  \SCnt{B}}
          \end{array}
        \end{math}
        \qquad\qquad
        $\Pi_2$:
        \begin{math}
          $$\mprset{flushleft}
          \inferrule* [right={\tiny unitL1}] {
            {
              \begin{array}{c}
                \pi \\
                {\Gamma_{{\mathrm{1}}}  \SCsym{;}  \Gamma_{{\mathrm{2}}}  \SCsym{;}  \SCnt{B}  \SCsym{;}  \Gamma_{{\mathrm{3}}}  \vdash_\mathcal{L}  \SCnt{A}}
              \end{array}
            }
          }{\Gamma_{{\mathrm{1}}}  \SCsym{;}   \mathsf{Unit}   \SCsym{;}  \Gamma_{{\mathrm{2}}}  \SCsym{;}  \SCnt{B}  \SCsym{;}  \Gamma_{{\mathrm{3}}}  \vdash_\mathcal{L}  \SCnt{A}}
        \end{math}
      \end{center}
      By assumption, $c(\Pi_1),c(\Pi_2)\leq |B|$. By induction on $\Pi_1$
      and $\pi$, there is a proof $\Pi'$ for sequent
      $\Gamma_{{\mathrm{1}}}  \SCsym{;}  \Gamma_{{\mathrm{2}}}  \SCsym{;}  \Delta  \SCsym{;}  \Gamma_{{\mathrm{3}}}  \vdash_\mathcal{L}  \SCnt{A}$
      s.t. $c(\Pi') \leq |B|$. Therefore, the proof $\Pi$ can be
      constructed as follows with $c(\Pi) = c(\Pi') \leq |B|$.
      \begin{center}
        \scriptsize
        \begin{math}
          $$\mprset{flushleft}
          \inferrule* [right={\tiny unitL1}] {
            {
              \begin{array}{c}
                \Pi' \\
                {\Gamma_{{\mathrm{1}}}  \SCsym{;}  \Gamma_{{\mathrm{2}}}  \SCsym{;}  \Delta  \SCsym{;}  \Gamma_{{\mathrm{3}}}  \vdash_\mathcal{L}  \SCnt{A}}
              \end{array}
            }
          }{\Gamma_{{\mathrm{1}}}  \SCsym{;}   \mathsf{Unit}   \SCsym{;}  \Gamma_{{\mathrm{2}}}  \SCsym{;}  \Delta  \SCsym{;}  \Gamma_{{\mathrm{3}}}  \vdash_\mathcal{L}  \SCnt{A}}
        \end{math}
      \end{center}
\end{itemize}

\subsubsection{Left introduction of the non-commutative unit $ \mathsf{Unit} $ (with low priority)}
\begin{itemize}
\item Case 1:
      \begin{center}
        \scriptsize
        \begin{math}
          \begin{array}{c}
            \Pi_1 \\
            {\Phi  \vdash_\mathcal{C}  \SCnt{X}}
          \end{array}
        \end{math}
        \qquad\qquad
        $\Pi_2$:
        \begin{math}
          $$\mprset{flushleft}
          \inferrule* [right={\tiny unitL2}] {
            {
              \begin{array}{c}
                \pi \\
                {\Gamma_{{\mathrm{1}}}  \SCsym{;}  \Gamma_{{\mathrm{2}}}  \SCsym{;}  \SCnt{X}  \SCsym{;}  \Gamma_{{\mathrm{3}}}  \vdash_\mathcal{L}  \SCnt{A}}
              \end{array}
            }
          }{\Gamma_{{\mathrm{1}}}  \SCsym{;}   \mathsf{Unit}   \SCsym{;}  \Gamma_{{\mathrm{2}}}  \SCsym{;}  \SCnt{X}  \SCsym{;}  \Gamma_{{\mathrm{3}}}  \vdash_\mathcal{L}  \SCnt{A}}
        \end{math}
      \end{center}
      By assumption, $c(\Pi_1),c(\Pi_2)\leq |X|$. By induction on $\Pi_1$
      and $\pi$, there is a proof $\Pi'$ for sequent
      $\Gamma_{{\mathrm{1}}}  \SCsym{;}  \Gamma_{{\mathrm{2}}}  \SCsym{;}  \Phi  \SCsym{;}  \Gamma_{{\mathrm{3}}}  \vdash_\mathcal{L}  \SCnt{A}$
      s.t. $c(\Pi') \leq |X|$. Therefore, the proof $\Pi$ can be
      constructed as follows with $c(\Pi) = c(\Pi') \leq |X|$.
      \begin{center}
        \scriptsize
        \begin{math}
          $$\mprset{flushleft}
          \inferrule* [right={\tiny unitL2}] {
            {
              \begin{array}{c}
                \Pi' \\
                {\Gamma_{{\mathrm{1}}}  \SCsym{;}  \Gamma_{{\mathrm{2}}}  \SCsym{;}  \Phi  \SCsym{;}  \Gamma_{{\mathrm{3}}}  \vdash_\mathcal{L}  \SCnt{A}}
              \end{array}
            }
          }{\Gamma_{{\mathrm{1}}}  \SCsym{;}   \mathsf{Unit}   \SCsym{;}  \Gamma_{{\mathrm{2}}}  \SCsym{;}  \Phi  \SCsym{;}  \Gamma_{{\mathrm{3}}}  \vdash_\mathcal{L}  \SCnt{A}}
        \end{math}
      \end{center}

\item Case 2:
      \begin{center}
        \scriptsize
        \begin{math}
          \begin{array}{c}
            \Pi_1 \\
            {\Delta  \vdash_\mathcal{L}  \SCnt{B}}
          \end{array}
        \end{math}
        \qquad\qquad
        $\Pi_2$:
        \begin{math}
          $$\mprset{flushleft}
          \inferrule* [right={\tiny unitL2}] {
            {
              \begin{array}{c}
                \pi \\
                {\Gamma_{{\mathrm{1}}}  \SCsym{;}  \Gamma_{{\mathrm{2}}}  \SCsym{;}  \SCnt{B}  \SCsym{;}  \Gamma_{{\mathrm{3}}}  \vdash_\mathcal{L}  \SCnt{A}}
              \end{array}
            }
          }{\Gamma_{{\mathrm{1}}}  \SCsym{;}   \mathsf{Unit}   \SCsym{;}  \Gamma_{{\mathrm{2}}}  \SCsym{;}  \SCnt{B}  \SCsym{;}  \Gamma_{{\mathrm{3}}}  \vdash_\mathcal{L}  \SCnt{A}}
        \end{math}
      \end{center}
      By assumption, $c(\Pi_1),c(\Pi_2)\leq |B|$. By induction on $\Pi_1$
      and $\pi$, there is a proof $\Pi'$ for sequent
      $\Gamma_{{\mathrm{1}}}  \SCsym{;}  \Gamma_{{\mathrm{2}}}  \SCsym{;}  \Delta  \SCsym{;}  \Gamma_{{\mathrm{3}}}  \vdash_\mathcal{L}  \SCnt{A}$
      s.t. $c(\Pi') \leq |B|$. Therefore, the proof $\Pi$ can be
      constructed as follows with $c(\Pi) = c(\Pi') \leq |B|$.
      \begin{center}
        \scriptsize
        \begin{math}
          $$\mprset{flushleft}
          \inferrule* [right={\tiny unitL2}] {
            {
              \begin{array}{c}
                \Pi' \\
                {\Gamma_{{\mathrm{1}}}  \SCsym{;}  \Gamma_{{\mathrm{2}}}  \SCsym{;}  \Delta  \SCsym{;}  \Gamma_{{\mathrm{3}}}  \vdash_\mathcal{L}  \SCnt{A}}
              \end{array}
            }
          }{\Gamma_{{\mathrm{1}}}  \SCsym{;}   \mathsf{Unit}   \SCsym{;}  \Gamma_{{\mathrm{2}}}  \SCsym{;}  \Delta  \SCsym{;}  \Gamma_{{\mathrm{3}}}  \vdash_\mathcal{L}  \SCnt{A}}
        \end{math}
      \end{center}
\end{itemize}



\subsubsection{Right introduction of the commutative implication $\multimap$ (with low priority)}
\begin{center}
  \scriptsize
  \begin{math}
    \begin{array}{c}
      \Pi_1 \\
      {\Phi  \vdash_\mathcal{C}  \SCnt{X}}
    \end{array}
  \end{math}
  \qquad\qquad
  $\Pi_2$:
  \begin{math}
    $$\mprset{flushleft}
    \inferrule* [right={\tiny impR}] {
      {
        \begin{array}{c}
          \pi \\
          {\Psi_{{\mathrm{1}}}  \SCsym{,}  \SCnt{X}  \SCsym{,}  \Psi_{{\mathrm{2}}}  \SCsym{,}  \SCnt{Y_{{\mathrm{1}}}}  \vdash_\mathcal{C}  \SCnt{Y_{{\mathrm{2}}}}}
        \end{array}
      }
    }{\Psi_{{\mathrm{1}}}  \SCsym{,}  \SCnt{X}  \SCsym{,}  \Psi_{{\mathrm{2}}}  \vdash_\mathcal{C}  \SCnt{Y_{{\mathrm{1}}}}  \multimap  \SCnt{Y_{{\mathrm{2}}}}}
  \end{math}
\end{center}
By assumption, $c(\Pi_1),c(\Pi_2)\leq |X|$. By induction on $\Pi_1$
and $\pi$, there is a proof $\Pi'$ for sequent
$\Psi_{{\mathrm{1}}}  \SCsym{,}  \Phi  \SCsym{,}  \Psi_{{\mathrm{2}}}  \SCsym{,}  \SCnt{Y_{{\mathrm{1}}}}  \vdash_\mathcal{C}  \SCnt{Y_{{\mathrm{2}}}}$ s.t. $c(\Pi') \leq |X|$. Therefore, the
proof $\Pi$ can be constructed as follows with
$c(\Pi) = c(\Pi') \leq |X|$.
\begin{center}
  \scriptsize
  \begin{math}
    $$\mprset{flushleft}
    \inferrule* [right={\tiny impR}] {
      {
        \begin{array}{c}
          \Pi' \\
          {\Psi_{{\mathrm{1}}}  \SCsym{,}  \Phi  \SCsym{,}  \Psi_{{\mathrm{2}}}  \SCsym{,}  \SCnt{Y_{{\mathrm{1}}}}  \vdash_\mathcal{C}  \SCnt{Y_{{\mathrm{2}}}}}
        \end{array}
      }
    }{\Psi_{{\mathrm{1}}}  \SCsym{,}  \Phi  \SCsym{,}  \Psi_{{\mathrm{2}}}  \vdash_\mathcal{C}  \SCnt{Y_{{\mathrm{1}}}}  \multimap  \SCnt{Y_{{\mathrm{2}}}}}
  \end{math}
\end{center}



\subsubsection{Right introduction of the non-commutative left implication $\lto$ (with low priority)}
\begin{itemize}
\item Case 1:
      \begin{center}
        \scriptsize
        \begin{math}
          \begin{array}{c}
            \Pi_1 \\
            {\Phi  \vdash_\mathcal{C}  \SCnt{X}}
          \end{array}
        \end{math}
        \qquad\qquad
        $\Pi_2$:
        \begin{math}
          $$\mprset{flushleft}
          \inferrule* [right={\tiny impR}] {
            {
              \begin{array}{c}
                \pi \\
                {\Gamma_{{\mathrm{1}}}  \SCsym{;}  \SCnt{X}  \SCsym{;}  \Gamma_{{\mathrm{2}}}  \SCsym{;}  \SCnt{A}  \vdash_\mathcal{L}  \SCnt{B}}
              \end{array}
            }
          }{\Gamma_{{\mathrm{1}}}  \SCsym{;}  \SCnt{X}  \SCsym{;}  \Gamma_{{\mathrm{2}}}  \vdash_\mathcal{L}  \SCnt{A}  \rightharpoonup  \SCnt{B}}
        \end{math}
      \end{center}
      By assumption, $c(\Pi_1),c(\Pi_2)\leq |X|$. By induction on $\Pi_1$
      and $\pi$, there is a proof $\Pi'$ for sequent
      $\Gamma_{{\mathrm{1}}}  \SCsym{;}  \Phi  \SCsym{;}  \Gamma_{{\mathrm{2}}}  \SCsym{;}  \SCnt{A}  \vdash_\mathcal{L}  \SCnt{B}$ s.t. $c(\Pi') \leq |X|$. Therefore, the
      proof $\Pi$ can be constructed as follows with
      $c(\Pi) = c(\Pi') \leq |X|$.
      \begin{center}
        \scriptsize
        \begin{math}
          $$\mprset{flushleft}
          \inferrule* [right={\tiny implR}] {
            {
              \begin{array}{c}
                \Pi' \\
                {\Gamma_{{\mathrm{1}}}  \SCsym{;}  \Phi  \SCsym{;}  \Gamma_{{\mathrm{2}}}  \SCsym{;}  \SCnt{A}  \vdash_\mathcal{L}  \SCnt{B}}
              \end{array}
            }
          }{\Gamma_{{\mathrm{1}}}  \SCsym{;}  \Phi  \SCsym{;}  \Gamma_{{\mathrm{2}}}  \vdash_\mathcal{L}  \SCnt{A}  \rightharpoonup  \SCnt{B}}
        \end{math}
      \end{center}

\item Case 2:
      \begin{center}
        \scriptsize
        \begin{math}
          \begin{array}{c}
            \Pi_1 \\
            {\Delta  \vdash_\mathcal{L}  \SCnt{C}}
          \end{array}
        \end{math}
        \qquad\qquad
        $\Pi_2$:
        \begin{math}
          $$\mprset{flushleft}
          \inferrule* [right={\tiny impR}] {
            {
              \begin{array}{c}
                \pi \\
                {\Gamma_{{\mathrm{1}}}  \SCsym{;}  \SCnt{C}  \SCsym{;}  \Gamma_{{\mathrm{2}}}  \SCsym{;}  \SCnt{A}  \vdash_\mathcal{L}  \SCnt{B}}
              \end{array}
            }
          }{\Gamma_{{\mathrm{1}}}  \SCsym{;}  \SCnt{C}  \SCsym{;}  \Gamma_{{\mathrm{2}}}  \vdash_\mathcal{L}  \SCnt{A}  \rightharpoonup  \SCnt{B}}
        \end{math}
      \end{center}
      By assumption, $c(\Pi_1),c(\Pi_2)\leq |C|$. By induction on $\Pi_1$
      and $\pi$, there is a proof $\Pi'$ for sequent
      $\Gamma_{{\mathrm{1}}}  \SCsym{;}  \Delta  \SCsym{;}  \Gamma_{{\mathrm{2}}}  \SCsym{;}  \SCnt{A}  \vdash_\mathcal{L}  \SCnt{B}$ s.t. $c(\Pi') \leq |C|$. Therefore, the
      proof $\Pi$ can be constructed as follows with
      $c(\Pi) = c(\Pi') \leq |C|$.
      \begin{center}
        \scriptsize
        \begin{math}
          $$\mprset{flushleft}
          \inferrule* [right={\tiny implR}] {
            {
              \begin{array}{c}
                \Pi' \\
                {\Gamma_{{\mathrm{1}}}  \SCsym{;}  \Delta  \SCsym{;}  \Gamma_{{\mathrm{2}}}  \SCsym{;}  \SCnt{A}  \vdash_\mathcal{L}  \SCnt{B}}
              \end{array}
            }
          }{\Gamma_{{\mathrm{1}}}  \SCsym{;}  \Delta  \SCsym{;}  \Gamma_{{\mathrm{2}}}  \vdash_\mathcal{L}  \SCnt{A}  \rightharpoonup  \SCnt{B}}
        \end{math}
      \end{center}
\end{itemize}




\subsubsection{Right introduction of the non-commutative right implication $\rto$ (with low priority)}
\begin{itemize}
\item Case 1:
      \begin{center}
        \scriptsize
        \begin{math}
          \begin{array}{c}
            \Pi_1 \\
            {\Phi  \vdash_\mathcal{C}  \SCnt{X}}
          \end{array}
        \end{math}
        \qquad\qquad
        $\Pi_2$:
        \begin{math}
          $$\mprset{flushleft}
          \inferrule* [right={\tiny impL}] {
            {
              \begin{array}{c}
                \pi \\
                {\SCnt{A}  \SCsym{;}  \Gamma_{{\mathrm{1}}}  \SCsym{;}  \SCnt{X}  \SCsym{;}  \Gamma_{{\mathrm{2}}}  \vdash_\mathcal{L}  \SCnt{B}}
              \end{array}
            }
          }{\Gamma_{{\mathrm{1}}}  \SCsym{;}  \SCnt{X}  \SCsym{;}  \Gamma_{{\mathrm{2}}}  \vdash_\mathcal{L}  \SCnt{B}  \leftharpoonup  \SCnt{A}}
        \end{math}
      \end{center}
      By assumption, $c(\Pi_1),c(\Pi_2)\leq |X|$. By induction on $\Pi_1$
      and $\pi$, there is a proof $\Pi'$ for sequent
      $\SCnt{A}  \SCsym{;}  \Gamma_{{\mathrm{1}}}  \SCsym{;}  \Phi  \SCsym{;}  \Gamma_{{\mathrm{2}}}  \vdash_\mathcal{L}  \SCnt{B}$ s.t. $c(\Pi') \leq |X|$. Therefore, the
      proof $\Pi$ can be constructed as follows with
      $c(\Pi) = c(\Pi') \leq |X|$.
      \begin{center}
        \scriptsize
        \begin{math}
          $$\mprset{flushleft}
          \inferrule* [right={\tiny impR}] {
            {
              \begin{array}{c}
                \Pi' \\
                {\SCnt{A}  \SCsym{;}  \Gamma_{{\mathrm{1}}}  \SCsym{;}  \Phi  \SCsym{;}  \Gamma_{{\mathrm{2}}}  \vdash_\mathcal{L}  \SCnt{B}}
              \end{array}
            }
          }{\Gamma_{{\mathrm{1}}}  \SCsym{;}  \Phi  \SCsym{;}  \Gamma_{{\mathrm{2}}}  \vdash_\mathcal{L}  \SCnt{B}  \leftharpoonup  \SCnt{A}}
        \end{math}
      \end{center}

\item Case 2:
      \begin{center}
        \scriptsize
        \begin{math}
          \begin{array}{c}
            \Pi_1 \\
            {\Delta  \vdash_\mathcal{L}  \SCnt{C}}
          \end{array}
        \end{math}
        \qquad\qquad
        $\Pi_2$:
        \begin{math}
          $$\mprset{flushleft}
          \inferrule* [right={\tiny impR}] {
            {
              \begin{array}{c}
                \pi \\
                {\SCnt{A}  \SCsym{;}  \Gamma_{{\mathrm{1}}}  \SCsym{;}  \SCnt{C}  \SCsym{;}  \Gamma_{{\mathrm{2}}}  \vdash_\mathcal{L}  \SCnt{B}}
              \end{array}
            }
          }{\Gamma_{{\mathrm{1}}}  \SCsym{;}  \SCnt{C}  \SCsym{;}  \Gamma_{{\mathrm{2}}}  \vdash_\mathcal{L}  \SCnt{B}  \leftharpoonup  \SCnt{A}}
        \end{math}
      \end{center}
      By assumption, $c(\Pi_1),c(\Pi_2)\leq |C|$. By induction on $\Pi_1$
      and $\pi$, there is a proof $\Pi'$ for sequent
      $\Gamma_{{\mathrm{1}}}  \SCsym{;}  \Delta  \SCsym{;}  \Gamma_{{\mathrm{2}}}  \SCsym{;}  \SCnt{A}  \vdash_\mathcal{L}  \SCnt{B}$ s.t. $c(\Pi') \leq |C|$. Therefore, the
      proof $\Pi$ can be constructed as follows with
      $c(\Pi) = c(\Pi') \leq |C|$.
      \begin{center}
        \scriptsize
        \begin{math}
          $$\mprset{flushleft}
          \inferrule* [right={\tiny impR}] {
            {
              \begin{array}{c}
                \Pi' \\
                {\SCnt{A}  \SCsym{;}  \Gamma_{{\mathrm{1}}}  \SCsym{;}  \Delta  \SCsym{;}  \Gamma_{{\mathrm{2}}}  \vdash_\mathcal{L}  \SCnt{B}}
              \end{array}
            }
          }{\Gamma_{{\mathrm{1}}}  \SCsym{;}  \Delta  \SCsym{;}  \Gamma_{{\mathrm{2}}}  \vdash_\mathcal{L}  \SCnt{B}  \leftharpoonup  \SCnt{A}}
        \end{math}
      \end{center}
\end{itemize}



\subsubsection{Right introduction of the functor $F$}
\begin{center}
  \scriptsize
  \begin{math}
    \begin{array}{c}
      \Pi_1 \\
      {\Phi  \vdash_\mathcal{C}  \SCnt{X}}
    \end{array}
  \end{math}
  \qquad\qquad
  $\Pi_2$:
  \begin{math}
    $$\mprset{flushleft}
    \inferrule* [right={\tiny Fr}] {
      {
        \begin{array}{c}
          \pi \\
          {\Psi_{{\mathrm{1}}}  \SCsym{,}  \SCnt{X}  \SCsym{,}  \Psi_{{\mathrm{2}}}  \vdash_\mathcal{C}  \SCnt{Y}}
        \end{array}
      }
    }{\Psi_{{\mathrm{1}}}  \SCsym{,}  \SCnt{X}  \SCsym{,}  \Psi_{{\mathrm{2}}}  \vdash_\mathcal{L}   \mathsf{F} \SCnt{Y} }
  \end{math}
\end{center}
By assumption, $c(\Pi_1),c(\Pi_2)\leq |X|$. By induction on $\Pi_1$
and $\pi$, there is a proof $\Pi'$ for sequent $\Psi_{{\mathrm{1}}}  \SCsym{,}  \Phi  \SCsym{,}  \Psi_{{\mathrm{2}}}  \vdash_\mathcal{C}  \SCnt{Y}$
s.t. $c(\Pi') \leq |X|$. Therefore, the proof $\Pi$ can be
constructed as follows with $c(\Pi) = c(\Pi') \leq |X|$.
\begin{center}
  \scriptsize
  \begin{math}
    $$\mprset{flushleft}
    \inferrule* [right={\tiny Fr}] {
      {
        \begin{array}{c}
          \Pi' \\
          {\Psi_{{\mathrm{1}}}  \SCsym{,}  \Phi  \SCsym{,}  \Psi_{{\mathrm{2}}}  \vdash_\mathcal{C}  \SCnt{Y}}
        \end{array}
      }
    }{\Psi_{{\mathrm{1}}}  \SCsym{,}  \Phi  \SCsym{,}  \Psi_{{\mathrm{2}}}  \vdash_\mathcal{L}   \mathsf{F} \SCnt{Y} }
  \end{math}
\end{center}



\subsubsection{Left introduction of the functor $F$ (with low priority)}
\begin{itemize}
\item Case 1:
      \begin{center}
        \scriptsize
        \begin{math}
          \begin{array}{c}
            \Pi_1 \\
            {\Phi  \vdash_\mathcal{C}  \SCnt{X}}
          \end{array}
        \end{math}
        \qquad\qquad
        $\Pi_2$:
        \begin{math}
          $$\mprset{flushleft}
          \inferrule* [right={\tiny Fl}] {
            {
              \begin{array}{c}
                \pi \\
                {\Gamma_{{\mathrm{1}}}  \SCsym{;}  \SCnt{X}  \SCsym{;}  \Gamma_{{\mathrm{2}}}  \SCsym{;}  \SCnt{Y}  \SCsym{;}  \Gamma_{{\mathrm{3}}}  \vdash_\mathcal{L}  \SCnt{A}}
              \end{array}
            }
          }{\Gamma_{{\mathrm{1}}}  \SCsym{;}  \SCnt{X}  \SCsym{;}  \Gamma_{{\mathrm{2}}}  \SCsym{;}   \mathsf{F} \SCnt{Y}   \SCsym{;}  \Gamma_{{\mathrm{3}}}  \vdash_\mathcal{L}  \SCnt{A}}
        \end{math}
      \end{center}
      By assumption, $c(\Pi_1),c(\Pi_2)\leq |X|$. By induction on $\Pi_1$
      and $\pi$, there is a proof $\Pi'$ for sequent
      $\Gamma_{{\mathrm{1}}}  \SCsym{;}  \Phi  \SCsym{;}  \Gamma_{{\mathrm{2}}}  \SCsym{;}  \SCnt{Y}  \SCsym{;}  \Gamma_{{\mathrm{3}}}  \vdash_\mathcal{L}  \SCnt{A}$ s.t. $c(\Pi') \leq |X|$. Therefore, the
      proof $\Pi$ can be constructed as follows with
      $c(\Pi) = c(\Pi') \leq |X|$.
      \begin{center}
        \scriptsize
        \begin{math}
          $$\mprset{flushleft}
          \inferrule* [right={\tiny Fl}] {
            {
              \begin{array}{c}
                \Pi' \\
                {\Gamma_{{\mathrm{1}}}  \SCsym{;}  \Phi  \SCsym{;}  \Gamma_{{\mathrm{2}}}  \SCsym{;}  \SCnt{Y}  \SCsym{;}  \Gamma_{{\mathrm{3}}}  \vdash_\mathcal{L}  \SCnt{A}}
              \end{array}
            }
          }{\Gamma_{{\mathrm{1}}}  \SCsym{;}  \Phi  \SCsym{;}  \Gamma_{{\mathrm{2}}}  \SCsym{;}   \mathsf{F} \SCnt{Y}   \SCsym{;}  \Gamma_{{\mathrm{3}}}  \vdash_\mathcal{L}  \SCnt{A}}
        \end{math}
      \end{center}

\item Case 2:
      \begin{center}
        \scriptsize
        \begin{math}
          \begin{array}{c}
            \Pi_1 \\
            {\Delta  \vdash_\mathcal{L}  \SCnt{B}}
          \end{array}
        \end{math}
        \qquad\qquad
        $\Pi_2$:
        \begin{math}
          $$\mprset{flushleft}
          \inferrule* [right={\tiny Fl}] {
            {
              \begin{array}{c}
                \pi \\
                {\Gamma_{{\mathrm{1}}}  \SCsym{;}  \SCnt{B}  \SCsym{;}  \Gamma_{{\mathrm{2}}}  \SCsym{;}  \SCnt{Y}  \SCsym{;}  \Gamma_{{\mathrm{3}}}  \vdash_\mathcal{L}  \SCnt{A}}
              \end{array}
            }
          }{\Gamma_{{\mathrm{1}}}  \SCsym{;}  \SCnt{B}  \SCsym{;}  \Gamma_{{\mathrm{2}}}  \SCsym{;}   \mathsf{F} \SCnt{Y}   \SCsym{;}  \Gamma_{{\mathrm{3}}}  \vdash_\mathcal{L}  \SCnt{A}}
        \end{math}
      \end{center}
      By assumption, $c(\Pi_1),c(\Pi_2)\leq |B|$. By induction on $\Pi_1$
      and $\pi$, there is a proof $\Pi'$ for sequent
      $\Gamma_{{\mathrm{1}}}  \SCsym{;}  \Delta  \SCsym{;}  \Gamma_{{\mathrm{2}}}  \SCsym{;}  \SCnt{Y}  \SCsym{;}  \Gamma_{{\mathrm{3}}}  \vdash_\mathcal{L}  \SCnt{A}$ s.t. $c(\Pi') \leq |B|$. Therefore, the
      proof $\Pi$ can be constructed as follows with
      $c(\Pi) = c(\Pi') \leq |B|$.
      \begin{center}
        \scriptsize
        \begin{math}
          $$\mprset{flushleft}
          \inferrule* [right={\tiny Fl}] {
            {
              \begin{array}{c}
                \Pi' \\
                {\Gamma_{{\mathrm{1}}}  \SCsym{;}  \Delta  \SCsym{;}  \Gamma_{{\mathrm{2}}}  \SCsym{;}  \SCnt{Y}  \SCsym{;}  \Gamma_{{\mathrm{3}}}  \vdash_\mathcal{L}  \SCnt{A}}
              \end{array}
            }
          }{\Gamma_{{\mathrm{1}}}  \SCsym{;}  \Delta  \SCsym{;}  \Gamma_{{\mathrm{2}}}  \SCsym{;}   \mathsf{F} \SCnt{Y}   \SCsym{;}  \Gamma_{{\mathrm{3}}}  \vdash_\mathcal{L}  \SCnt{A}}
        \end{math}
      \end{center}

\item Case 3:
      \begin{center}
        \scriptsize
        \begin{math}
          \begin{array}{c}
            \Pi_1 \\
            {\Phi  \vdash_\mathcal{C}  \SCnt{X}}
          \end{array}
        \end{math}
        \qquad\qquad
        $\Pi_2$:
        \begin{math}
          $$\mprset{flushleft}
          \inferrule* [right={\tiny Fl}] {
            {
              \begin{array}{c}
                \pi \\
                {\Gamma_{{\mathrm{1}}}  \SCsym{;}  \SCnt{Y}  \SCsym{;}  \Gamma_{{\mathrm{2}}}  \SCsym{;}  \SCnt{X}  \SCsym{;}  \Gamma_{{\mathrm{3}}}  \vdash_\mathcal{L}  \SCnt{A}}
              \end{array}
            }
          }{\Gamma_{{\mathrm{1}}}  \SCsym{;}   \mathsf{F} \SCnt{Y}   \SCsym{;}  \Gamma_{{\mathrm{2}}}  \SCsym{;}  \SCnt{X}  \SCsym{;}  \Gamma_{{\mathrm{3}}}  \vdash_\mathcal{L}  \SCnt{A}}
        \end{math}
      \end{center}
      By assumption, $c(\Pi_1),c(\Pi_2)\leq |X|$. By induction on $\Pi_1$
      and $\pi$, there is a proof $\Pi'$ for sequent
      $\Gamma_{{\mathrm{1}}}  \SCsym{;}  \SCnt{Y}  \SCsym{;}  \Gamma_{{\mathrm{2}}}  \SCsym{;}  \Phi  \SCsym{;}  \Gamma_{{\mathrm{3}}}  \vdash_\mathcal{L}  \SCnt{A}$ s.t. $c(\Pi') \leq |X|$. Therefore, the
      proof $\Pi$ can be constructed as follows with
      $c(\Pi) = c(\Pi') \leq |X|$.
      \begin{center}
        \scriptsize
        \begin{math}
          $$\mprset{flushleft}
          \inferrule* [right={\tiny Fl}] {
            {
              \begin{array}{c}
                \Pi' \\
                {\Gamma_{{\mathrm{1}}}  \SCsym{;}  \SCnt{Y}  \SCsym{;}  \Gamma_{{\mathrm{2}}}  \SCsym{;}  \Phi  \SCsym{;}  \Gamma_{{\mathrm{3}}}  \vdash_\mathcal{L}  \SCnt{A}}
              \end{array}
            }
          }{\Gamma_{{\mathrm{1}}}  \SCsym{;}   \mathsf{F} \SCnt{Y}   \SCsym{;}  \Gamma_{{\mathrm{2}}}  \SCsym{;}  \Phi  \SCsym{;}  \Gamma_{{\mathrm{3}}}  \vdash_\mathcal{L}  \SCnt{A}}
        \end{math}
      \end{center}

\item Case 4:
      \begin{center}
        \scriptsize
        \begin{math}
          \begin{array}{c}
            \Pi_1 \\
            {\Delta  \vdash_\mathcal{L}  \SCnt{B}}
          \end{array}
        \end{math}
        \qquad\qquad
        $\Pi_2$:
        \begin{math}
          $$\mprset{flushleft}
          \inferrule* [right={\tiny Fl}] {
            {
              \begin{array}{c}
                \pi \\
                {\Gamma_{{\mathrm{1}}}  \SCsym{;}  \SCnt{Y}  \SCsym{;}  \Gamma_{{\mathrm{2}}}  \SCsym{;}  \SCnt{B}  \SCsym{;}  \Gamma_{{\mathrm{3}}}  \vdash_\mathcal{L}  \SCnt{A}}
              \end{array}
            }
          }{\Gamma_{{\mathrm{1}}}  \SCsym{;}   \mathsf{F} \SCnt{Y}   \SCsym{;}  \Gamma_{{\mathrm{2}}}  \SCsym{;}  \Delta  \SCsym{;}  \Gamma_{{\mathrm{3}}}  \vdash_\mathcal{L}  \SCnt{A}}
        \end{math}
      \end{center}
      By assumption, $c(\Pi_1),c(\Pi_2)\leq |B|$. By induction on $\Pi_1$
      and $\pi$, there is a proof $\Pi'$ for sequent
      $\Gamma_{{\mathrm{1}}}  \SCsym{;}  \SCnt{Y}  \SCsym{;}  \Gamma_{{\mathrm{2}}}  \SCsym{;}  \Delta  \SCsym{;}  \Gamma_{{\mathrm{3}}}  \vdash_\mathcal{L}  \SCnt{A}$ s.t. $c(\Pi') \leq |B|$. Therefore, the
      proof $\Pi$ can be constructed as follows with
      $c(\Pi) = c(\Pi') \leq |B|$.
      \begin{center}
        \scriptsize
        \begin{math}
          $$\mprset{flushleft}
          \inferrule* [right={\tiny Fl}] {
            {
              \begin{array}{c}
                \Pi' \\
                {\Gamma_{{\mathrm{1}}}  \SCsym{;}  \SCnt{Y}  \SCsym{;}  \Gamma_{{\mathrm{2}}}  \SCsym{;}  \Delta  \SCsym{;}  \Gamma_{{\mathrm{3}}}  \vdash_\mathcal{L}  \SCnt{A}}
              \end{array}
            }
          }{\Gamma_{{\mathrm{1}}}  \SCsym{;}   \mathsf{F} \SCnt{Y}   \SCsym{;}  \Gamma_{{\mathrm{2}}}  \SCsym{;}  \Delta  \SCsym{;}  \Gamma_{{\mathrm{3}}}  \vdash_\mathcal{L}  \SCnt{A}}
        \end{math}
      \end{center}
\end{itemize}




\subsubsection{Right introduction of the functor $G$ (with low priority)}
\begin{center}
  \scriptsize
  \begin{math}
    \begin{array}{c}
      \Pi_1 \\
      {\Phi  \vdash_\mathcal{C}  \SCnt{X}}
    \end{array}
  \end{math}
  \qquad\qquad
  $\Pi_2$:
  \begin{math}
    $$\mprset{flushleft}
    \inferrule* [right={\tiny Gr}] {
      {
        \begin{array}{c}
          \pi \\
          {\Psi_{{\mathrm{1}}}  \SCsym{;}  \SCnt{X}  \SCsym{;}  \Psi_{{\mathrm{2}}}  \vdash_\mathcal{L}  \SCnt{A}}
        \end{array}
      }
    }{\Psi_{{\mathrm{1}}}  \SCsym{,}  \SCnt{X}  \SCsym{,}  \Psi_{{\mathrm{2}}}  \vdash_\mathcal{C}   \mathsf{G} \SCnt{A} }
  \end{math}
\end{center}
By assumption, $c(\Pi_1),c(\Pi_2)\leq |X|$. By induction on $\Pi_1$
and $\pi$, there is a proof $\Pi'$ for sequent $\Psi_{{\mathrm{1}}}  \SCsym{,}  \Phi  \SCsym{,}  \Psi_{{\mathrm{2}}}  \vdash_\mathcal{L}  \SCnt{A}$
s.t. $c(\Pi') \leq |X|$. Therefore, the proof $\Pi$ can be
constructed as follows with $c(\Pi) = c(\Pi') \leq |X|$.
\begin{center}
  \scriptsize
  \begin{math}
    $$\mprset{flushleft}
    \inferrule* [right={\tiny Gr}] {
      {
        \begin{array}{c}
          \Pi' \\
          {\Psi_{{\mathrm{1}}}  \SCsym{;}  \Phi  \SCsym{;}  \Psi_{{\mathrm{2}}}  \vdash_\mathcal{L}  \SCnt{A}}
        \end{array}
      }
    }{\Psi_{{\mathrm{1}}}  \SCsym{,}  \Phi  \SCsym{,}  \Psi_{{\mathrm{2}}}  \vdash_\mathcal{C}   \mathsf{G} \SCnt{A} }
  \end{math}
\end{center}




\subsubsection{Left introduction of the functor $G$ (with low priority)}
\begin{itemize}
\item Case 1:
      \begin{center}
        \scriptsize
        \begin{math}
          \begin{array}{c}
            \Pi_1 \\
            {\Phi  \vdash_\mathcal{C}  \SCnt{X}}
          \end{array}
        \end{math}
        \qquad\qquad
        $\Pi_2$:
        \begin{math}
          $$\mprset{flushleft}
          \inferrule* [right={\tiny Gl}] {
            {
              \begin{array}{c}
                \pi \\
                {\Gamma_{{\mathrm{1}}}  \SCsym{;}  \SCnt{X}  \SCsym{;}  \Gamma_{{\mathrm{2}}}  \SCsym{;}  \SCnt{B}  \SCsym{;}  \Gamma_{{\mathrm{3}}}  \vdash_\mathcal{L}  \SCnt{A}}
              \end{array}
            }
          }{\Gamma_{{\mathrm{1}}}  \SCsym{;}  \SCnt{X}  \SCsym{;}  \Gamma_{{\mathrm{2}}}  \SCsym{;}   \mathsf{G} \SCnt{B}   \SCsym{;}  \Gamma_{{\mathrm{3}}}  \vdash_\mathcal{L}  \SCnt{A}}
        \end{math}
      \end{center}
      By assumption, $c(\Pi_1),c(\Pi_2)\leq |X|$. By induction on $\Pi_1$
      and $\pi$, there is a proof $\Pi'$ for sequent
      $\Gamma_{{\mathrm{1}}}  \SCsym{;}  \Phi  \SCsym{;}  \Gamma_{{\mathrm{2}}}  \SCsym{;}  \SCnt{B}  \SCsym{;}  \Gamma_{{\mathrm{3}}}  \vdash_\mathcal{L}  \SCnt{A}$ s.t. $c(\Pi') \leq |X|$. Therefore, the
      proof $\Pi$ can be constructed as follows with
      $c(\Pi) = c(\Pi') \leq |X|$.
      \begin{center}
        \scriptsize
        \begin{math}
          $$\mprset{flushleft}
          \inferrule* [right={\tiny Gl}] {
            {
              \begin{array}{c}
                \Pi' \\
                {\Gamma_{{\mathrm{1}}}  \SCsym{;}  \Phi  \SCsym{;}  \Gamma_{{\mathrm{2}}}  \SCsym{;}  \SCnt{B}  \SCsym{;}  \Gamma_{{\mathrm{3}}}  \vdash_\mathcal{L}  \SCnt{A}}
              \end{array}
            }
          }{\Gamma_{{\mathrm{1}}}  \SCsym{;}  \Phi  \SCsym{;}  \Gamma_{{\mathrm{2}}}  \SCsym{;}   \mathsf{G} \SCnt{B}   \SCsym{;}  \Gamma_{{\mathrm{3}}}  \vdash_\mathcal{L}  \SCnt{A}}
        \end{math}
      \end{center}

\item Case 2:
      \begin{center}
        \scriptsize
        \begin{math}
          \begin{array}{c}
            \Pi_1 \\
            {\Delta  \vdash_\mathcal{L}  \SCnt{B}}
          \end{array}
        \end{math}
        \qquad\qquad
        $\Pi_2$:
        \begin{math}
          $$\mprset{flushleft}
          \inferrule* [right={\tiny Gl}] {
            {
              \begin{array}{c}
                \pi \\
                {\Gamma_{{\mathrm{1}}}  \SCsym{;}  \SCnt{B}  \SCsym{;}  \Gamma_{{\mathrm{2}}}  \SCsym{;}  \SCnt{C}  \SCsym{;}  \Gamma_{{\mathrm{3}}}  \vdash_\mathcal{L}  \SCnt{A}}
              \end{array}
            }
          }{\Gamma_{{\mathrm{1}}}  \SCsym{;}  \SCnt{B}  \SCsym{;}  \Gamma_{{\mathrm{2}}}  \SCsym{;}   \mathsf{G} \SCnt{C}   \SCsym{;}  \Gamma_{{\mathrm{3}}}  \vdash_\mathcal{L}  \SCnt{A}}
        \end{math}
      \end{center}
      By assumption, $c(\Pi_1),c(\Pi_2)\leq |B|$. By induction on $\Pi_1$
      and $\pi$, there is a proof $\Pi'$ for sequent
      $\Gamma_{{\mathrm{1}}}  \SCsym{;}  \Delta  \SCsym{;}  \Gamma_{{\mathrm{2}}}  \SCsym{;}  \SCnt{C}  \SCsym{;}  \Gamma_{{\mathrm{3}}}  \vdash_\mathcal{L}  \SCnt{A}$ s.t. $c(\Pi') \leq |B|$. Therefore, the
      proof $\Pi$ can be constructed as follows with
      $c(\Pi) = c(\Pi') \leq |B|$.
      \begin{center}
        \scriptsize
        \begin{math}
          $$\mprset{flushleft}
          \inferrule* [right={\tiny Gl}] {
            {
              \begin{array}{c}
                \Pi' \\
                {\Gamma_{{\mathrm{1}}}  \SCsym{;}  \Delta  \SCsym{;}  \Gamma_{{\mathrm{2}}}  \SCsym{;}  \SCnt{C}  \SCsym{;}  \Gamma_{{\mathrm{3}}}  \vdash_\mathcal{L}  \SCnt{A}}
              \end{array}
            }
          }{\Gamma_{{\mathrm{1}}}  \SCsym{;}  \Delta  \SCsym{;}  \Gamma_{{\mathrm{2}}}  \SCsym{;}   \mathsf{G} \SCnt{C}   \SCsym{;}  \Gamma_{{\mathrm{3}}}  \vdash_\mathcal{L}  \SCnt{A}}
        \end{math}
      \end{center}

\item Case 3:
      \begin{center}
        \scriptsize
        \begin{math}
          \begin{array}{c}
            \Pi_1 \\
            {\Phi  \vdash_\mathcal{C}  \SCnt{X}}
          \end{array}
        \end{math}
        \qquad\qquad
        $\Pi_2$:
        \begin{math}
          $$\mprset{flushleft}
          \inferrule* [right={\tiny Gl}] {
            {
              \begin{array}{c}
                \pi \\
                {\Gamma_{{\mathrm{1}}}  \SCsym{;}  \SCnt{B}  \SCsym{;}  \Gamma_{{\mathrm{2}}}  \SCsym{;}  \SCnt{X}  \SCsym{;}  \Gamma_{{\mathrm{3}}}  \vdash_\mathcal{L}  \SCnt{A}}
              \end{array}
            }
          }{\Gamma_{{\mathrm{1}}}  \SCsym{;}   \mathsf{G} \SCnt{B}   \SCsym{;}  \Gamma_{{\mathrm{2}}}  \SCsym{;}  \SCnt{X}  \SCsym{;}  \Gamma_{{\mathrm{3}}}  \vdash_\mathcal{L}  \SCnt{A}}
        \end{math}
      \end{center}
      By assumption, $c(\Pi_1),c(\Pi_2)\leq |X|$. By induction on $\Pi_1$
      and $\pi$, there is a proof $\Pi'$ for sequent
      $\Gamma_{{\mathrm{1}}}  \SCsym{;}  \SCnt{B}  \SCsym{;}  \Gamma_{{\mathrm{2}}}  \SCsym{;}  \Phi  \SCsym{;}  \Gamma_{{\mathrm{3}}}  \vdash_\mathcal{L}  \SCnt{A}$ s.t. $c(\Pi') \leq |X|$. Therefore, the
      proof $\Pi$ can be constructed as follows with
      $c(\Pi) = c(\Pi') \leq |X|$.
      \begin{center}
        \scriptsize
        \begin{math}
          $$\mprset{flushleft}
          \inferrule* [right={\tiny Gl}] {
            {
              \begin{array}{c}
                \Pi' \\
                {\Gamma_{{\mathrm{1}}}  \SCsym{;}  \SCnt{B}  \SCsym{;}  \Gamma_{{\mathrm{2}}}  \SCsym{;}  \Phi  \SCsym{;}  \Gamma_{{\mathrm{3}}}  \vdash_\mathcal{L}  \SCnt{A}}
              \end{array}
            }
          }{\Gamma_{{\mathrm{1}}}  \SCsym{;}   \mathsf{G} \SCnt{B}   \SCsym{;}  \Gamma_{{\mathrm{2}}}  \SCsym{;}  \Phi  \SCsym{;}  \Gamma_{{\mathrm{3}}}  \vdash_\mathcal{L}  \SCnt{A}}
        \end{math}
      \end{center}

\item Case 4:
      \begin{center}
        \scriptsize
        \begin{math}
          \begin{array}{c}
            \Pi_1 \\
            {\Delta  \vdash_\mathcal{L}  \SCnt{B}}
          \end{array}
        \end{math}
        \qquad\qquad
        $\Pi_2$:
        \begin{math}
          $$\mprset{flushleft}
          \inferrule* [right={\tiny Gl}] {
            {
              \begin{array}{c}
                \pi \\
                {\Gamma_{{\mathrm{1}}}  \SCsym{;}  \SCnt{C}  \SCsym{;}  \Gamma_{{\mathrm{2}}}  \SCsym{;}  \SCnt{B}  \SCsym{;}  \Gamma_{{\mathrm{3}}}  \vdash_\mathcal{L}  \SCnt{A}}
              \end{array}
            }
          }{\Gamma_{{\mathrm{1}}}  \SCsym{;}   \mathsf{G} \SCnt{C}   \SCsym{;}  \Gamma_{{\mathrm{2}}}  \SCsym{;}  \SCnt{B}  \SCsym{;}  \Gamma_{{\mathrm{3}}}  \vdash_\mathcal{L}  \SCnt{A}}
        \end{math}
      \end{center}
      By assumption, $c(\Pi_1),c(\Pi_2)\leq |B|$. By induction on $\Pi_1$
      and $\pi$, there is a proof $\Pi'$ for sequent
      $\Gamma_{{\mathrm{1}}}  \SCsym{;}  \SCnt{C}  \SCsym{;}  \Gamma_{{\mathrm{2}}}  \SCsym{;}  \Delta  \SCsym{;}  \Gamma_{{\mathrm{3}}}  \vdash_\mathcal{L}  \SCnt{A}$ s.t. $c(\Pi') \leq |B|$. Therefore, the
      proof $\Pi$ can be constructed as follows with
      $c(\Pi) = c(\Pi') \leq |B|$.
      \begin{center}
        \scriptsize
        \begin{math}
          $$\mprset{flushleft}
          \inferrule* [right={\tiny Gl}] {
            {
              \begin{array}{c}
                \Pi' \\
                {\Gamma_{{\mathrm{1}}}  \SCsym{;}  \SCnt{C}  \SCsym{;}  \Gamma_{{\mathrm{2}}}  \SCsym{;}  \Delta  \SCsym{;}  \Gamma_{{\mathrm{3}}}  \vdash_\mathcal{L}  \SCnt{A}}
              \end{array}
            }
          }{\Gamma_{{\mathrm{1}}}  \SCsym{;}   \mathsf{G} \SCnt{C}   \SCsym{;}  \Gamma_{{\mathrm{2}}}  \SCsym{;}  \Delta  \SCsym{;}  \Gamma_{{\mathrm{3}}}  \vdash_\mathcal{L}  \SCnt{A}}
        \end{math}
      \end{center}

\end{itemize}



%--------------------------------------------------
%--------------------------------------------------
\section{Proof For Lemma~\ref{lem:monoidal-monad}}
\label{app:monoidal-monad}

Let $(\cat{C},\cat{L},F,G,\eta,\varepsilon)$ be a LAM. We define the monad
$(T,\eta:id_\cat{C}\rightarrow T,\mu:T^2\rightarrow T)$ on $\cat{C}$ as
$T=GF$, $\eta_X:X\rightarrow GFX$, and
$\mu_X=G\varepsilon_{FX}:GFGFX\rightarrow GFX$. Since $(F,\m{})$ and
$(G,\n{})$ are monoidal functors, we have
$$\t{X,Y}=G\m{X,Y}\circ\n{FX,FY}:TX\otimes TY\rightarrow T(X\otimes Y) \qquad\mbox{and}\qquad\t{I}=G\m{I}\circ\n{I'}:I\rightarrow TI.$$
The monad $T$ being monoidal means:
\begin{enumerate}
\item $T$ is a monoidal functor, i.e. the following diagrams commute:
      \begin{mathpar}
      \bfig
        \hSquares/->`->`->``->`->`->/<400>[
          (TX\otimes TY)\otimes TZ`TX\otimes(TY\otimes TZ)`TX\otimes T(Y\otimes Z)`
          T(X\otimes Y)\otimes TZ`T((X\otimes Y)\otimes Z)`T(X\otimes(Y\otimes Z));
          \alpha_{TX,TY,TZ}`id_{TX}\otimes\t{Y,Z}`\t{X,Y}\otimes id_{TZ}``
          \t{X,Y\otimes Z}`\t{X\otimes Y,Z}`T\alpha_{X,Y,Z}]
        \morphism(1300,200)//<0,0>[`;(1)]
      \efig
      \and
      \bfig
        \square/->`->`<-`->/<600,400>[
          I\otimes TX`TX`TI\otimes TX`T(I\otimes X);
          \lambda_{TX}`\t{I}\otimes id_{TX}`T\lambda_X`\t{I,X}]
        \morphism(350,200)//<0,0>[`;(2)]
      \efig
      \and
      \bfig
        \square/->`->`<-`->/<600,400>[
          TX\otimes I`TX`TX\otimes TI`T(X\otimes I);
          \rho_{TX}`id_{TX}\otimes\t{I}`T\rho_X`\t{X,I}]
        \morphism(350,200)//<0,0>[`;(3)]
      \efig
      \end{mathpar}
      We write $GF$ instead of $T$ in the proof for clarity. \\
      By replacing $\t{X,Y}$ with its definition, diagram (1) above
      commutes by the following commutative diagram, in which the two
      hexagons commute because $G$ and $F$ are monoidal functors, and the
      two quadrilaterals commute by the naturality of $\n{}$.
      \begin{mathpar}
      \bfig
        \iiixiii/->`->`->``->```->`<-`->``/<1400,400>[
          (GFX\otimes GFY)\otimes GFZ`GFX\otimes(GFY\otimes GFZ)`GFX\otimes G(FY\tri FZ)`
          G(FX\tri FY)\otimes GFZ`G(FX\tri(FY\tri FZ))`GFX\otimes GF(Y\otimes Z)`
          GF(X\otimes Y)\otimes GFZ`G((FX\tri FY)\tri FZ)`G(FX\tri F(Y\otimes Z));
          \alpha_{GFX,GFY,GFZ}`id_{GFX}\otimes\n{FY,FZ}`\n{FX,FY}\otimes id_{GFZ}``
          id_{GFX}\otimes G\m{Y,Z}```G\m{X,Y}\otimes id_{GFZ}`G\alpha'_{FX,FY,FZ}`
          \n{FX,F(Y\otimes Z)}``]
        \morphism(2800,800)|m|<-1400,-400>[
          GFX\otimes G(FY\tri FZ)`G(FX\tri(FY\tri FZ));\n{FX,FY\tri FZ}]
        \morphism(0,400)|m|<1400,-400>[
          G(FX\tri FY)\otimes GFZ`G((FX\tri FY)\tri FZ);\n{FX\tri FY,FZ}]
        \morphism(1400,400)|m|<1400,-400>[
          G(FX\tri(FY\tri FZ))`G(FX\tri F(Y\otimes Z));G(id_{FX}\tri\m{Y,Z})]
        \ptriangle(0,-400)|mlm|/`->`->/<1400,400>[
          GF(X\otimes Y)\otimes GFZ`G((FX\tri FY)\tri FZ)`G(F(X\otimes Y)\tri FZ);
          `\n{F(X\otimes Y),FZ}`G(\m{X,Y}\otimes id_{FZ})]
        \morphism(0,-400)|b|<1400,0>[
          G(F(X\otimes Y)\tri FZ)`GF((X\otimes Y)\otimes Z);G\m{X\otimes Y,Z}]
        \dtriangle(1400,-400)|mrb|/`->`->/<1400,400>[
          G(FX\tri F(Y\otimes Z))`GF((X\otimes Y)\otimes Z)`GF(X\otimes(Y\otimes Z));
          `G\m{X,Y\otimes Z}`GF\alpha_{X,Y,Z}]
      \efig
      \end{mathpar}
      Diagram (2) commutes by the following commutative diagrams, in which
      the top quadrilateral commutes because $G$ is monoidal, the right
      quadrilateral commutes because $F$ is monoidal, and the left square
      commutes by the naturality of $\n{}$.
      \begin{mathpar}
      \bfig
        \ptriangle/->`->`/<1600,400>[
          I\otimes GFX`GFX`GI'\otimes GFX;\lambda_{GFX}`\n{I'}\otimes id_{GFX}`]
        \square(0,-400)|lmmb|<800,400>[
          GI'\otimes GFX`G(I'\tri FX)`GFI\otimes GFX`G(FI\tri FX);
          \n{I',FX}`G\m{I}\otimes id_{GFX}`G(\m{I}\tri id_{FX})`\n{FI,FX}]
        \morphism(800,0)|m|<800,400>[G(I'\tri FX)`GFX;G\lambda'_{FX}]
        \dtriangle(800,-400)/`<-`->/<800,800>[
          GFX`G(FI\tri FX)`GF(I\otimes X);
          `GF\lambda_X`G\m{I,X}]
      \efig
      \end{mathpar}
      Similarly, diagram (3) commutes as follows:
      \begin{mathpar}
      \bfig
        \ptriangle/->`->`/<1600,400>[
          GFX\otimes I`GFX`GFX\otimes GI';\rho_{GFX}`id_{GFX}\otimes\n{I'}`]
        \square(0,-400)|lmmb|<800,400>[
          GFX\otimes GI'`G(FX\tri I')`GFX\otimes GFI`G(FX\tri FI);
          \n{FX,I'}`id_{GFX}\otimes G\m{I}`G(id_{FX}\otimes\m{I})`\n{FX,FI}]
        \morphism(800,0)|m|<800,400>[G(FX\tri I')`GFX;G\rho'_{FX}]
        \dtriangle(800,-400)/`<-`->/<800,800>[
          GFX`G(FX\tri FI)`GF(X\otimes I);
          `GF\rho_X`G\m{X,I}]
      \efig
      \end{mathpar}
\item $\eta$ is a monoidal natural transformation. In fact, since $\eta$
      is the unit of the monoidal adjunction, $\eta$ is monoidal by
      definition and thus the following two diagrams commute.
      \begin{mathpar}
      \bfig
        \square/=`->`->`->/<600,400>[
          X\otimes Y`X\otimes Y`TX\otimes TY`T(X\otimes Y);
          `\eta_X\otimes\eta_Y`\eta_{X\otimes Y}`\t{X,Y}]
      \efig
      \and
      \bfig
        \Vtriangle/->`=`<-/<400,400>[I`TI`I;\eta_I``\t{I}]
      \efig
      \end{mathpar}
\item $\mu$ is a monoidal natural transformation. It is obvious that since
      $\mu=G\varepsilon_{FA}$ and $\varepsilon$ is monoidal, so is $\mu$.
      Thus the following diagrams commute.
      \begin{mathpar}
      \bfig
        \square/`->`->`->/<1500,400>[
          T^2X\otimes T^2Y`T^2(X\otimes Y)`TX\otimes TY`T(X\otimes Y);
          `\mu_X\otimes\mu_Y`\mu_{X\otimes Y}`\t{X,Y}]
        \morphism(0,400)<800,0>[T^2X\otimes T^2Y`T(TX\otimes TY);\t{TX,TY}]
        \morphism(800,400)<700,0>[T(TX\otimes TY)`T^2(X\otimes Y);T\t{X,Y}]
      \efig
      \and
      \bfig
        \square/->`<-`<-`<-/<400,400>[T^2I`TI`TI`I;\mu_I`T\t{I}`\t{I}`\t{I}]
      \efig
      \end{mathpar}
\end{enumerate}



%--------------------------------------------------
%--------------------------------------------------
\section{Proof For Lemma~\ref{lem:strong-monad}}
\label{app:strong-monad}

\begin{definition}
\label{def:strong-monad}
Let $(\cat{M},\tri,I,\alpha,\lambda,\rho)$ be a monoidal category and
$(T,\eta,\mu)$ be a monad on $\cat{M}$. $T$ is a \textbf{strong monad} if
there is natural transformation $\tau$, called the \textbf{tensorial
strength}, with components $\tau_{A,B}:A\tri TB\rightarrow T(A\tri B)$
such that the following diagrams commute:
\begin{mathpar}
\bfig
  \Vtriangle<400,400>[I\tri TA`T(I\tri A)`TA;\tau_{I,A}`\lambda_{TA}`T\lambda_A]
\efig
\and
\bfig
  \Vtriangle<400,400>[
    A\tri B`A\tri TB`T(A\tri B);id_A\tri\eta_B`\eta_{A\tri B}`\tau_{A,B}]
\efig
\and
\bfig
  \square/->`->`->`/<1800,400>[
    (A\tri B)\tri TC`T((A\tri B)\tri C)`
    A\tri(B\tri TC)`T(A\tri(B\tri C));
    \tau_{A\tri B,C}`\alpha_{A,B,TC}`T\alpha_{A,B,C}`]
  \morphism<900,0>[A\tri(B\tri TC)`A\tri T(B\tri C);id_A\tri\tau_{B,C}]
  \morphism(900,0)<900,0>[A\tri T(B\tri C)`T(A\tri(B\tri C));\tau_{A,B\tri C}]  \efig
\and
\bfig
  \square/`->`->`->/<1400,400>[
    A\tri T^2B`T^2(A\tri B)`A\tri TB`T(A\tri B);
    `id_A\tri\mu_B`\mu_{A\tri B}`\tau_{A,B}]
  \morphism(0,400)<700,0>[A\tri T^2B`T(A\tri TB);\tau_{A,TB}]
  \morphism(700,400)<700,0>[T(A\tri TB)`T^2(A\tri B);T\tau_{A,B}]
\efig
\end{mathpar}
\end{definition}
\noindent
The proof for Lemma~\ref{lem:strong-monad} goes as follows.
\noindent
Let $(\cat{C},\cat{L},F,G,\eta,\varepsilon)$ be a LAM, where
$(\cat{C},\otimes,I,\alpha,\lambda,\rho)$ is symmetric monoidal closed,
and \\ $(\cat{L},\tri,I',\alpha',\lambda',\rho')$ is Lambek. In
Lemma~\ref{lem:monoidal-monad}, we have proved that the monad
$(T=GF,\eta,\mu)$ is monoidal with the natural transformation
$\t{X,Y}:TX\otimes TY\rightarrow T(X\otimes Y)$ and the morphism
$\t{I}:I\rightarrow TI$.
\noindent
We define the tensorial strength
$\tau_{X,Y}:X\otimes TY\rightarrow T(X\otimes Y)$ as
$$\tau_{X,Y}=\t{X,Y}\circ(\eta_X\otimes id_{TY}).$$
Since $\eta$ is a monoidal natural transformation, we have
$\eta_I=G\m{I}\circ\n{I'}$, and thus $\eta_I=\t{I}$. The following diagram
commutes because $T$ is monoidal, where the composition
$\t{I,X}\circ(\t{I}\otimes id_{TX})$ is the definition of $\tau_{I,X}$. So
the first triangle in Definition~\ref{def:strong-monad} commutes.
\begin{mathpar}
\bfig
  \square/->`->`->`<-/<600,400>[
    I\otimes TX`TI\otimes TX`TX`T(I\otimes X);
    \t{I}\otimes id_{TX}`\lambda_{TX}`\t{I,X}`T\lambda_X]
\efig
\end{mathpar}
Similarly, by using the definition of $\tau$, the the second triangle in the definition is
equivalent to the following diagram, which commutes because $\eta$ is a monoidal natural
transformation:
\begin{mathpar}
\bfig
  \square/->`->`->`<-/<600,400>[
    X\otimes Y`X\otimes TY`T(X\otimes Y)`TX\otimes TY;
    id_X\otimes\eta_Y`\eta_{X\otimes Y}`\eta_X\otimes id_{TY}`\t{X,Y}]
  \morphism(0,400)|m|<600,-400>[X\otimes Y`TX\otimes TY;\eta_X\otimes\eta_Y]
\efig
\end{mathpar}
The first pentagon in the definition commutes by the following commutative diagrams, because
$\eta$ and $\alpha$ are natural transformations and $T$ is monoidal:
\begin{mathpar}
\bfig
  \qtriangle|amm|/->`->`<-/<1000,400>[
    (X\otimes Y)\otimes TZ`T(X\otimes Y)\otimes TZ`(TX\otimes TY)\otimes TZ;
    \eta_{X\otimes Y}\otimes id_{TZ}`
    (\eta_X\otimes\eta_Y)\otimes id_{TZ}`
    \t{X,Y}\otimes id_{TZ}]
  \morphism(0,400)<0,-400>[(X\otimes Y)\otimes TZ`X\otimes(Y\otimes TZ);\alpha_{X,Y,TZ}]
  \morphism(1000,0)|m|<0,-400>[
    (TX\otimes TY)\otimes TZ`TX\otimes(TY\otimes TZ);\alpha_{TX,TY,TZ}]
  \Dtriangle(0,-800)|lmm|/->`->`<-/<1000,400>[
    X\otimes(Y\otimes TZ)`TX\otimes(TY\otimes TZ)`X\otimes(TY\otimes TZ);
    id_X\otimes(\eta_Y\otimes id_{TZ})`
    \eta_X\otimes(\eta_Y\otimes id_{TZ})`
    \eta_X\otimes id_{TY\otimes TZ}]
  \morphism(0,-800)|b|<1000,0>[
    X\otimes(TY\otimes TZ)`X\otimes T(Y\otimes Z);id_X\otimes\t{Y,Z}]
  \qtriangle(1000,0)|amr|/->``->/<1000,400>[
    T(X\otimes Y)\otimes TZ`T((X\otimes Y)\otimes Z)`T(X\otimes(Y\otimes Z));
    \t{X\otimes Y,Z}``T\alpha_{X,Y,Z}]
  \morphism(2000,-800)<0,800>[
    TX\otimes T(Y\otimes Z)`T(X\otimes(Y\otimes Z));\t{X,Y\otimes Z}]
  \btriangle(1000,-800)|mmb|/`->`->/<1000,400>[
    TX\otimes(TY\otimes TZ)`X\otimes T(Y\otimes Z)`TX\otimes T(Y\otimes Z);
    `id_{TX}\otimes\t{Y,Z}`\eta_X\otimes id_{T(Y\otimes Z)}]
\efig
\end{mathpar}
The last diagram in the definition commutes by the following commutative diagram, because
$T$ is a monad, $\t{}$ is a natural transformation, and $\mu$ is a monoidal natural
transformation:
\begin{mathpar}
\bfig
  \ptriangle/->`->`/<700,400>[
    X\otimes T^2Y`TX\otimes T^2Y`X\otimes TY;\eta_X\otimes id_{T^2Y}`id_X\otimes\mu_Y`]
  \btriangle(0,-400)/->``->/<700,400>[
    X\otimes TY`TX\otimes TY`T(X\otimes Y);\eta_X\otimes id_{TY}``\t{X,Y}]
  \morphism(700,400)|m|<-700,-800>[TX\otimes T^2Y`TX\otimes TY;id_{TX}\otimes\mu_Y]
  \morphism(700,0)|m|<-700,-400>[TX\otimes T^2Y`TX\otimes TY;id_{TX}\otimes\mu_Y]
  \qtriangle(700,0)/->``->/<1800,400>[
    TX\otimes T^2Y`T(X\otimes TY)`T(TX\otimes TY);\t{X,TY}``T(\eta_X\otimes id_{TY})]
  \btriangle(700,0)|mmm|/=`->`<-/<900,400>[
    TX\otimes T^2Y`TX\otimes T^2Y`T^2X\otimes T^2Y;
    `T\eta_X\otimes id_{T^2Y}`\mu_X\otimes id_{T^2Y}]
  \morphism(1600,0)|m|<900,0>[T^2X\otimes T^2Y`T(TX\otimes TY);\t{TX,TY}]
  \morphism(1600,0)|m|<-1600,-400>[T^2X\otimes T^2Y`TX\otimes TY;\mu_X\otimes\mu_Y]
  \dtriangle(700,-400)/`->`<-/<1800,400>[
    T(TX\otimes TY)`T(X\otimes Y)`T^2(X\otimes Y);`T\t{X,Y}`\mu_{X\otimes Y}]
\efig
\end{mathpar}




\input{appendix2-output}


\end{document}

%%% Local Variables: 
%%% mode: latex
%%% TeX-master: t
%%% End:

