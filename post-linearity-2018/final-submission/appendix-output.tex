\section{Proof For Lemma~\ref{lem:cut-reduction}}
\label{app:cut-reduction}


\subsection{Commuting Conversion Cut vs. Cut}

\subsubsection{$\SCdruleTXXcutName$ vs. $\SCdruleTXXcutName$}
\begin{itemize}
% C-Cut vs. C-Cut Case 1
\item Case 1:
      \begin{center}
        \scriptsize
        \begin{math}
          \begin{array}{c}
            \Pi_1 \\
            {\Phi  \vdash_\mathcal{C}  \SCnt{X}}
          \end{array}
        \end{math}
        \qquad\qquad
        $\Pi_2:$
        \begin{math}
          $$\mprset{flushleft}
          \inferrule* [right={\tiny cut}] {
            {
              \begin{array}{cc}
                \pi_1 & \pi_2 \\
                {\Psi_{{\mathrm{2}}}  \SCsym{,}  \SCnt{X}  \SCsym{,}  \Psi_{{\mathrm{3}}}  \vdash_\mathcal{C}  \SCnt{Y}} & {\Psi_{{\mathrm{1}}}  \SCsym{,}  \SCnt{Y}  \SCsym{,}  \Psi_{{\mathrm{4}}}  \vdash_\mathcal{C}  \SCnt{Z}}
              \end{array}
            }
          }{\Psi_{{\mathrm{1}}}  \SCsym{,}  \Psi_{{\mathrm{2}}}  \SCsym{,}  \SCnt{X}  \SCsym{,}  \Psi_{{\mathrm{3}}}  \SCsym{,}  \Psi_{{\mathrm{4}}}  \vdash_\mathcal{C}  \SCnt{Z}}
        \end{math}
      \end{center}
      By assumption, $c(\Pi_1),c(\Pi_2)\leq |X|$. Therefore, $c(\pi_1)$,
      $c(\pi_2)\leq |X|$. Since $Y$ is the cut formula on $\pi_1$ and
      $\pi_2$, we have $|Y|+1\leq|X|$. By induction on $\Pi_1$ and $\pi_1$
      there exists a proof $\Pi'$ for sequent $\Psi_{{\mathrm{2}}}  \SCsym{,}  \Phi  \SCsym{,}  \Psi_{{\mathrm{3}}}  \vdash_\mathcal{C}  \SCnt{Y}$ s.t.
      $c(\Pi')\leq|X|$. So $\Pi$ can be constructed as follows, with
      $c(\Pi)\leq max\{c(\Pi'),c(\pi_2),|Y|+1\}\leq |X|$.
      \begin{center}
        \scriptsize
        \begin{math}
          $$\mprset{flushleft}
          \inferrule* [right={\tiny cut}] {
            {
              \begin{array}{cc}
                \Pi' & \pi_2 \\
                {\Psi_{{\mathrm{2}}}  \SCsym{,}  \Phi  \SCsym{,}  \Psi_{{\mathrm{3}}}  \vdash_\mathcal{C}  \SCnt{Y}} & {\Psi_{{\mathrm{1}}}  \SCsym{,}  \SCnt{Y}  \SCsym{,}  \Psi_{{\mathrm{4}}}  \vdash_\mathcal{C}  \SCnt{Z}}
              \end{array}
            }
          }{\Psi_{{\mathrm{1}}}  \SCsym{,}  \Psi_{{\mathrm{2}}}  \SCsym{,}  \Phi  \SCsym{,}  \Psi_{{\mathrm{3}}}  \SCsym{,}  \Psi_{{\mathrm{4}}}  \vdash_\mathcal{C}  \SCnt{Z}}
        \end{math}
      \end{center}

% C-Cut vs. C-Cut Case 2
\item Case 2:
      \begin{center}
        \scriptsize
        $\Pi_1$:
        \begin{math}
          $$\mprset{flushleft}
          \inferrule* [right={\tiny cut}] {
            {
              \begin{array}{cc}
                \pi_1 & \pi_2 \\
                {\Phi  \vdash_\mathcal{C}  \SCnt{X}} & {\Psi_{{\mathrm{2}}}  \SCsym{,}  \SCnt{X}  \SCsym{,}  \Psi_{{\mathrm{3}}}  \vdash_\mathcal{C}  \SCnt{Y}}
              \end{array}
            }
          }{\Psi_{{\mathrm{2}}}  \SCsym{,}  \Phi  \SCsym{,}  \Psi_{{\mathrm{3}}}  \vdash_\mathcal{C}  \SCnt{Y}}
        \end{math}
        \qquad\qquad
        \begin{math}
          \begin{array}{c}
            \Pi_2 \\
            {\Psi_{{\mathrm{1}}}  \SCsym{,}  \SCnt{Y}  \SCsym{,}  \Psi_{{\mathrm{4}}}  \vdash_\mathcal{C}  \SCnt{Z}}
          \end{array}
        \end{math}
      \end{center}
      By assumption, $c(\Pi_1),c(\Pi_2)\leq |Y|$. Since the cut rank of the last cut in
      $\Pi_1$ is $|X|+1$, then $|X|+1\leq |Y|$. By induction on $\Pi_1$ and $\Pi_2$, there is
      a proof $\Pi'$ for sequent $\Psi_{{\mathrm{1}}}  \SCsym{,}  \Psi_{{\mathrm{2}}}  \SCsym{,}  \SCnt{X}  \SCsym{,}  \Psi_{{\mathrm{3}}}  \SCsym{,}  \Psi_{{\mathrm{4}}}  \vdash_\mathcal{C}  \SCnt{Z}$ s.t. $c(\Pi')\leq|Y|$.
      Therefore, the proof $\Pi$ can be constructed as follows, and
      $c(\Pi)\leq max\{c(\pi_1),c(\Pi'),|X|+1\}\leq |Y|$.
      \begin{center}
        \scriptsize
        \begin{math}
          $$\mprset{flushleft}
          \inferrule* [right={\tiny cut}] {
            {
              \begin{array}{cc}
                \pi_1 & \Pi' \\
                {\Phi  \vdash_\mathcal{C}  \SCnt{X}} & {\Psi_{{\mathrm{1}}}  \SCsym{,}  \Psi_{{\mathrm{2}}}  \SCsym{,}  \SCnt{X}  \SCsym{,}  \Psi_{{\mathrm{3}}}  \SCsym{,}  \Psi_{{\mathrm{4}}}  \vdash_\mathcal{C}  \SCnt{Z}}
              \end{array}
            }
          }{\Psi_{{\mathrm{1}}}  \SCsym{,}  \Psi_{{\mathrm{2}}}  \SCsym{,}  \Phi  \SCsym{,}  \Psi_{{\mathrm{3}}}  \SCsym{,}  \Psi_{{\mathrm{4}}}  \vdash_\mathcal{C}  \SCnt{Z}}
        \end{math}
      \end{center}
\end{itemize}



% C-Cut vs. LC-Cut Case 1
\subsubsection{$\SCdruleTXXcutName$ vs. $\SCdruleSXXcutOneName$}
\begin{itemize}
\item Case 1:
      \begin{center}
        \scriptsize
        \begin{math}
          \begin{array}{c}
            \Pi_1 \\
            {\Phi  \vdash_\mathcal{C}  \SCnt{X}}
          \end{array}
        \end{math}
        \qquad\qquad
        $\Pi_2:$
        \begin{math}
          $$\mprset{flushleft}
          \inferrule* [right={\tiny cut1}] {
            {
              \begin{array}{cc}
                \pi_2 & \pi_3 \\
                {\Psi_{{\mathrm{1}}}  \SCsym{,}  \SCnt{X}  \SCsym{,}  \Psi_{{\mathrm{2}}}  \vdash_\mathcal{C}  \SCnt{Y}} & {\Gamma_{{\mathrm{1}}}  \SCsym{;}  \SCnt{Y}  \SCsym{;}  \Gamma_{{\mathrm{2}}}  \vdash_\mathcal{L}  \SCnt{A}}
              \end{array}
            }
          }{\Gamma_{{\mathrm{1}}}  \SCsym{;}  \Psi_{{\mathrm{1}}}  \SCsym{;}  \SCnt{X}  \SCsym{;}  \Psi_{{\mathrm{2}}}  \SCsym{;}  \Gamma_{{\mathrm{2}}}  \vdash_\mathcal{L}  \SCnt{A}}
        \end{math}
      \end{center}
      By assumption, $c(\Pi_1),c(\Pi_2)\leq |X|$. Therefore, $c(\pi_1)$,
      $c(\pi_2)\leq |X|$. Since $Y$ is the cut formula on $\pi_1$ and
      $\pi_2$, we have $|Y|+1\leq|X|$. By induction on $\Pi_1$ and $\pi_1$,
      there exists a proof $\Pi'$ for sequent $\Psi_{{\mathrm{1}}}  \SCsym{,}  \Phi  \SCsym{,}  \Psi_{{\mathrm{2}}}  \vdash_\mathcal{C}  \SCnt{Y}$ s.t.
      $c(\Pi')\leq|X|$. So $\Pi$ can be constructed as follows, with
      $c(\Pi)\leq max\{c(\Pi'),c(\pi_2),|Y|+1\}\leq |X|$.
      \begin{center}
        \scriptsize
        \begin{math}
          $$\mprset{flushleft}
          \inferrule* [right={\tiny cut1}] {
            {
              \begin{array}{cc}
                \Pi' & \pi_2 \\
                {\Psi_{{\mathrm{1}}}  \SCsym{,}  \Phi  \SCsym{,}  \Psi_{{\mathrm{2}}}  \vdash_\mathcal{C}  \SCnt{Y}} & {\Gamma_{{\mathrm{1}}}  \SCsym{;}  \SCnt{Y}  \SCsym{;}  \Gamma_{{\mathrm{2}}}  \vdash_\mathcal{L}  \SCnt{A}}
              \end{array}
            }
          }{\Gamma_{{\mathrm{1}}}  \SCsym{;}  \Psi_{{\mathrm{1}}}  \SCsym{;}  \Phi  \SCsym{;}  \Psi_{{\mathrm{2}}}  \SCsym{;}  \Gamma_{{\mathrm{2}}}  \vdash_\mathcal{L}  \SCnt{A}}
        \end{math}
      \end{center}

% C-Cut vs. LC-Cut Case 2
\item Case 2:
      \begin{center}
        \scriptsize
        $\Pi_1$:
        \begin{math}
          $$\mprset{flushleft}
          \inferrule* [right={\tiny cut}] {
            {
              \begin{array}{cc}
                \pi_1 & \pi_2 \\
                {\Phi  \vdash_\mathcal{C}  \SCnt{X}} & {\Psi_{{\mathrm{1}}}  \SCsym{,}  \SCnt{X}  \SCsym{,}  \Psi_{{\mathrm{2}}}  \vdash_\mathcal{C}  \SCnt{Y}}
              \end{array}
            }
          }{\Psi_{{\mathrm{1}}}  \SCsym{,}  \Phi  \SCsym{,}  \Psi_{{\mathrm{2}}}  \vdash_\mathcal{C}  \SCnt{Y}}
        \end{math}
        \qquad\qquad
        \begin{math}
          \begin{array}{c}
            \Pi_2 \\
            {\Gamma_{{\mathrm{1}}}  \SCsym{;}  \SCnt{Y}  \SCsym{;}  \Gamma_{{\mathrm{2}}}  \vdash_\mathcal{L}  \SCnt{A}}
          \end{array}
        \end{math}
      \end{center}
      By assumption, $c(\Pi_1),c(\Pi_2)\leq |Y|$. Similar as above,
      $|X|+1\leq |Y|$ and there is a proof $\Pi'$ constructed from $\pi_2$
      and $\Pi_2$ for sequent $\Gamma_{{\mathrm{1}}}  \SCsym{;}  \Psi_{{\mathrm{1}}}  \SCsym{;}  \SCnt{X}  \SCsym{;}  \Psi_{{\mathrm{2}}}  \SCsym{;}  \Gamma_{{\mathrm{2}}}  \vdash_\mathcal{L}  \SCnt{A}$ s.t.
      $c(\Pi')\leq|Y|$. Therefore, the proof $\Pi$ can be constructed as
      follows, and $c(\Pi)\leq max\{c(\pi_1),c(\Pi'),|X|+1\}\leq |Y|$.
      \begin{center}
        \scriptsize
        \begin{math}
          $$\mprset{flushleft}
          \inferrule* [right={\tiny cut}] {
            {
              \begin{array}{cc}
                \pi_1 & \Pi'\\
                {\Phi  \vdash_\mathcal{C}  \SCnt{X}} & {\Gamma_{{\mathrm{1}}}  \SCsym{;}  \Psi_{{\mathrm{1}}}  \SCsym{;}  \SCnt{X}  \SCsym{;}  \Psi_{{\mathrm{2}}}  \SCsym{;}  \Gamma_{{\mathrm{2}}}  \vdash_\mathcal{L}  \SCnt{A}}
              \end{array}
            }
          }{\Gamma_{{\mathrm{1}}}  \SCsym{;}  \Psi_{{\mathrm{1}}}  \SCsym{;}  \Phi  \SCsym{;}  \Psi_{{\mathrm{2}}}  \SCsym{;}  \Gamma_{{\mathrm{2}}}  \vdash_\mathcal{L}  \SCnt{A}}
        \end{math}
      \end{center}
\end{itemize}

% LC-Cut vs. L-Cut Case 1
\subsubsection{$\SCdruleSXXcutOneName$ vs. $\SCdruleSXXcutTwoName$}
\begin{itemize}
\item Case 1:
      \begin{center}
        \scriptsize
        \begin{math}
          \begin{array}{c}
            \Pi_1 \\
            {\Phi  \vdash_\mathcal{C}  \SCnt{X}}
          \end{array}
        \end{math}
        \qquad\qquad
        $\Pi_2:$
        \begin{math}
          $$\mprset{flushleft}
          \inferrule* [right={\tiny cut2}] {
            {
              \begin{array}{cc}
                \pi_1 & \pi_2 \\
                {\Gamma_{{\mathrm{2}}}  \SCsym{;}  \SCnt{X}  \SCsym{;}  \Gamma_{{\mathrm{3}}}  \vdash_\mathcal{L}  \SCnt{A}} & {\Gamma_{{\mathrm{1}}}  \SCsym{;}  \SCnt{A}  \SCsym{;}  \Gamma_{{\mathrm{4}}}  \vdash_\mathcal{L}  \SCnt{B}}
              \end{array}
            }
          }{\Gamma_{{\mathrm{1}}}  \SCsym{;}  \Gamma_{{\mathrm{2}}}  \SCsym{;}  \SCnt{X}  \SCsym{;}  \Gamma_{{\mathrm{3}}}  \SCsym{;}  \Gamma_{{\mathrm{4}}}  \vdash_\mathcal{L}  \SCnt{B}}
        \end{math}
      \end{center}
      By assumption, $c(\Pi_1),c(\Pi_2)\leq |X|$. Therefore, $c(\pi_1)$,
      $c(\pi_2)\leq |X|$. Since $A$ is the cut formula on $\pi_1$ and
      $\pi_2$, we have $|A|+1\leq|X|$. By induction on $\Pi_1$ and $\pi_1$,
      there exists a proof $\Pi'$ for sequent $\Gamma_{{\mathrm{2}}}  \SCsym{;}  \Phi  \SCsym{;}  \Gamma_{{\mathrm{3}}}  \vdash_\mathcal{L}  \SCnt{A}$ s.t.
      $c(\Pi')\leq|X|$. So $\Pi$ can be constructed as follows, with
      $c(\Pi)\leq max\{c(\Pi'),c(\pi_2),|A|+1\}\leq |X|$.
      \begin{center}
        \scriptsize
        \begin{math}
          $$\mprset{flushleft}
          \inferrule* [right={\tiny cut2}] {
            {
              \begin{array}{cc}
                \Pi' & \pi_2 \\
                {\Gamma_{{\mathrm{2}}}  \SCsym{;}  \Phi  \SCsym{;}  \Gamma_{{\mathrm{3}}}  \vdash_\mathcal{L}  \SCnt{A}} & {\Gamma_{{\mathrm{1}}}  \SCsym{;}  \SCnt{A}  \SCsym{;}  \Gamma_{{\mathrm{4}}}  \vdash_\mathcal{L}  \SCnt{B}}
              \end{array}
            }
          }{\Gamma_{{\mathrm{1}}}  \SCsym{;}  \Gamma_{{\mathrm{2}}}  \SCsym{;}  \Phi  \SCsym{;}  \Gamma_{{\mathrm{3}}}  \SCsym{;}  \Gamma_{{\mathrm{4}}}  \vdash_\mathcal{L}  \SCnt{B}}
        \end{math}
      \end{center}

% LC-Cut vs. L-Cut Case 2
\item Case 2:
      \begin{center}
        \scriptsize
        $\Pi_1$:
        \begin{math}
          $$\mprset{flushleft}
          \inferrule* [right={\tiny cut}] {
            {
              \begin{array}{cc}
                \pi_1 & \pi_2 \\
                {\Phi  \vdash_\mathcal{C}  \SCnt{X}} & {\Gamma_{{\mathrm{2}}}  \SCsym{;}  \SCnt{X}  \SCsym{;}  \Gamma_{{\mathrm{3}}}  \vdash_\mathcal{L}  \SCnt{A}}
              \end{array}
            }
          }{\Gamma_{{\mathrm{2}}}  \SCsym{;}  \Phi  \SCsym{;}  \Gamma_{{\mathrm{3}}}  \vdash_\mathcal{L}  \SCnt{A}}
        \end{math}
        \qquad\qquad
        \begin{math}
          \begin{array}{c}
            \Pi_2 \\
            {\Gamma_{{\mathrm{1}}}  \SCsym{;}  \SCnt{A}  \SCsym{;}  \Gamma_{{\mathrm{4}}}  \vdash_\mathcal{L}  \SCnt{B}}
          \end{array}
        \end{math}
      \end{center}
      By assumption, $c(\Pi_1),c(\Pi_2)\leq |A|$. Similar as above,
      $|X|+1\leq |A|$ and there is a proof $\Pi'$ constructed from'
      $\pi_2$ and $\Pi_2$ for sequent $\Gamma_{{\mathrm{1}}}  \SCsym{;}  \Gamma_{{\mathrm{2}}}  \SCsym{;}  \SCnt{X}  \SCsym{;}  \Gamma_{{\mathrm{3}}}  \SCsym{;}  \Gamma_{{\mathrm{4}}}  \vdash_\mathcal{L}  \SCnt{B}$ s.t.
      $c(\Pi')\leq|A|$. Therefore, the proof $\Pi$ can be constructed as
      follows, and $c(\Pi)\leq max\{c(\pi_1),c(\Pi'),|X|+1\}\leq |A|$.
      \begin{center}
        \scriptsize
        \begin{math}
          $$\mprset{flushleft}
          \inferrule* [right={\tiny cut}] {
            {
              \begin{array}{cc}
                \pi_1  & \Pi' \\
                {\Phi  \vdash_\mathcal{C}  \SCnt{X}} & {\Gamma_{{\mathrm{1}}}  \SCsym{;}  \Gamma_{{\mathrm{2}}}  \SCsym{;}  \SCnt{X}  \SCsym{;}  \Gamma_{{\mathrm{3}}}  \SCsym{;}  \Gamma_{{\mathrm{4}}}  \vdash_\mathcal{L}  \SCnt{B}}
              \end{array}
            }
          }{\Gamma_{{\mathrm{1}}}  \SCsym{;}  \Gamma_{{\mathrm{2}}}  \SCsym{;}  \Phi  \SCsym{;}  \Gamma_{{\mathrm{3}}}  \SCsym{;}  \Gamma_{{\mathrm{4}}}  \vdash_\mathcal{L}  \SCnt{B}}
        \end{math}
      \end{center}
\end{itemize}

% L-Cut vs. L-Cut Case 1
\subsubsection{$\SCdruleSXXcutTwoName$ vs. $\SCdruleSXXcutTwoName$}
\begin{itemize}
\item Case 1:
      \begin{center}
        \scriptsize
        \begin{math}
          \begin{array}{c}
            \Pi_1 \\
            {\Gamma  \vdash_\mathcal{L}  \SCnt{A}}
          \end{array}
        \end{math}
        \qquad\qquad
        $\Pi_2:$
        \begin{math}
          $$\mprset{flushleft}
          \inferrule* [right={\tiny cut2}] {
            {
              \begin{array}{cc}
                \pi_1 & \pi_2 \\
                {\Delta_{{\mathrm{2}}}  \SCsym{;}  \SCnt{A}  \SCsym{;}  \Delta_{{\mathrm{3}}}  \vdash_\mathcal{L}  \SCnt{B}} & {\Delta_{{\mathrm{1}}}  \SCsym{;}  \SCnt{B}  \SCsym{;}  \Delta_{{\mathrm{4}}}  \vdash_\mathcal{L}  \SCnt{C}}
              \end{array}
            }
          }{\Delta_{{\mathrm{1}}}  \SCsym{;}  \Delta_{{\mathrm{2}}}  \SCsym{;}  \SCnt{A}  \SCsym{;}  \Delta_{{\mathrm{3}}}  \SCsym{;}  \Delta_{{\mathrm{4}}}  \vdash_\mathcal{L}  \SCnt{C}}
        \end{math}
      \end{center}
      By assumption, $c(\Pi_1),c(\Pi_2)\leq |A|$. Therefore, $c(\pi_1)$,
      $c(\pi_2)\leq |A|$. Since $B$ is the cut formula on $\pi_1$ and
      $\pi_3$, we have $|B|+1\leq|A|$. By induction on $\Pi_1$ and
      $\pi_1$, there exists a proof $\Pi'$ for sequent
      $\Delta_{{\mathrm{2}}}  \SCsym{;}  \Gamma  \SCsym{;}  \Delta_{{\mathrm{3}}}  \vdash_\mathcal{L}  \SCnt{B}$ s.t. $c(\Pi')\leq|A|$. So $\Pi$ can be
      constructed as follows,  with
      $c(\Pi)\leq max\{c(\Pi'),c(\pi_2),|B|+1\}\leq |A|$.
      \begin{center}
        \scriptsize
        \begin{math}
          $$\mprset{flushleft}
          \inferrule* [right={\tiny cut}] {
            {
              \begin{array}{cc}
                \Pi' & \pi_2 \\
                {\Delta_{{\mathrm{2}}}  \SCsym{;}  \Gamma  \SCsym{;}  \Delta_{{\mathrm{3}}}  \vdash_\mathcal{L}  \SCnt{B}} & {\Delta_{{\mathrm{1}}}  \SCsym{;}  \SCnt{B}  \SCsym{;}  \Delta_{{\mathrm{4}}}  \vdash_\mathcal{L}  \SCnt{C}}
              \end{array}
            }
          }{\Delta_{{\mathrm{1}}}  \SCsym{;}  \Delta_{{\mathrm{2}}}  \SCsym{;}  \Gamma  \SCsym{;}  \Delta_{{\mathrm{3}}}  \SCsym{;}  \Delta_{{\mathrm{4}}}  \vdash_\mathcal{L}  \SCnt{C}}
        \end{math}
      \end{center}

% L-Cut vs. L-Cut Case 2
\item Case 2:
      \begin{center}
        \scriptsize
        $\Pi_1$:
        \begin{math}
          $$\mprset{flushleft}
          \inferrule* [right={\tiny cut}] {
            {
              \begin{array}{cc}
                \pi_1 & \pi_2 \\
                {\Delta  \vdash_\mathcal{L}  \SCnt{A}} & {\Delta_{{\mathrm{2}}}  \SCsym{;}  \SCnt{A}  \SCsym{;}  \Delta_{{\mathrm{3}}}  \vdash_\mathcal{L}  \SCnt{B}}
              \end{array}
            }
          }{\Delta_{{\mathrm{2}}}  \SCsym{;}  \Delta  \SCsym{;}  \Delta_{{\mathrm{3}}}  \vdash_\mathcal{L}  \SCnt{A}}
        \end{math}
        \qquad\qquad
        \begin{math}
          \begin{array}{c}
            \Pi_2 \\
            {\Delta_{{\mathrm{1}}}  \SCsym{;}  \SCnt{B}  \SCsym{;}  \Delta_{{\mathrm{4}}}  \vdash_\mathcal{L}  \SCnt{C}}
          \end{array}
        \end{math}
      \end{center}
      By assumption, $c(\Pi_1),c(\Pi_2)\leq |B|$. Similar as above,
      $|A|+1\leq |B|$ and there is a proof $\Pi'$ constructed from $\pi_2$ 
      and $\Pi_2$ for sequent $\Delta_{{\mathrm{1}}}  \SCsym{;}  \Delta_{{\mathrm{2}}}  \SCsym{;}  \SCnt{A}  \SCsym{;}  \Delta_{{\mathrm{3}}}  \SCsym{;}  \Delta_{{\mathrm{4}}}  \vdash_\mathcal{L}  \SCnt{C}$ s.t.
      $c(\Pi')\leq|A|$. Therefore, the proof $\Pi$ can be constructed as
      follows, and $c(\Pi)\leq max\{c(\pi_1),c(\Pi'),|A|+1\}\leq |B|$.
      \begin{center}
        \scriptsize
        \begin{math}
          $$\mprset{flushleft}
          \inferrule* [right={\tiny cut}] {
            {
              \begin{array}{cc}
                \pi_1 & \Pi' \\
                {\Gamma  \vdash_\mathcal{L}  \SCnt{A}} & {\Delta_{{\mathrm{1}}}  \SCsym{;}  \Delta_{{\mathrm{2}}}  \SCsym{;}  \SCnt{A}  \SCsym{;}  \Delta_{{\mathrm{3}}}  \SCsym{;}  \Delta_{{\mathrm{4}}}  \vdash_\mathcal{L}  \SCnt{C}}
              \end{array}
            }
          }{\Delta_{{\mathrm{1}}}  \SCsym{;}  \Delta_{{\mathrm{2}}}  \SCsym{;}  \Gamma  \SCsym{;}  \Delta_{{\mathrm{3}}}  \SCsym{;}  \Delta_{{\mathrm{4}}}  \vdash_\mathcal{L}  \SCnt{C}}
        \end{math}
      \end{center}

\end{itemize}
% End of subsubsection Commuting conversion cut vs. cut



\subsection{The Axiom Steps}

\subsubsection{$\SCdruleTXXaxName$}
\begin{itemize}
% C-id Case 1
\item Case 1:
      \begin{center}
        \scriptsize
        $\Pi_1$:
        \begin{math}
          $$\mprset{flushleft}
          \inferrule* [right={\tiny ax}] {
            \,
          }{\SCnt{X}  \vdash_\mathcal{C}  \SCnt{X}}
        \end{math}
        \qquad\qquad
        \begin{math}
          \begin{array}{c}
            \Pi_2 \\
            {\Phi_{{\mathrm{1}}}  \SCsym{,}  \SCnt{X}  \SCsym{,}  \Phi_{{\mathrm{2}}}  \vdash_\mathcal{C}  \SCnt{Y}}
          \end{array}
        \end{math}
      \end{center}
      By assumption, $c(\Pi_1),c(\Pi_2)\leq |X|$. The proof $\Pi$ is the
      same as $\Pi_2$.

% C-id Case 2
\item Case 2:
      \begin{center}
        \scriptsize
        $\Pi_1$:
        \begin{math}
          \begin{array}{c}
            \Pi_1 \\
            {\Phi  \vdash_\mathcal{C}  \SCnt{X}}
          \end{array}
        \end{math}
        \qquad\qquad
        $\Pi_2$:
        \begin{math}
          $$\mprset{flushleft}
          \inferrule* [right={\tiny ax}] {
            \,
          }{\SCnt{X}  \vdash_\mathcal{C}  \SCnt{X}}
        \end{math}
      \end{center}
      By assumption, $c(\Pi_1),c(\Pi_2)\leq |X|$. The proof $\Pi$ is the
      same as $\Pi_1$.

% C-id Case 3
\item Case 3:
      \begin{center}
        \scriptsize
        $\Pi_1$:
        \begin{math}
          $$\mprset{flushleft}
          \inferrule* [right={\tiny ax}] {
            \,
          }{\SCnt{X}  \vdash_\mathcal{C}  \SCnt{X}}
        \end{math}
        \qquad\qquad
        \begin{math}
          \begin{array}{c}
            \Pi_2 \\
            {\Gamma_{{\mathrm{1}}}  \SCsym{;}  \SCnt{X}  \SCsym{;}  \Gamma_{{\mathrm{2}}}  \vdash_\mathcal{L}  \SCnt{A}}
          \end{array}
        \end{math}
      \end{center}
      By assumption, $c(\Pi_1),c(\Pi_2)\leq |X|$. The proof $\Pi$ is the
      same as $\Pi_2$.
\end{itemize}

% L-id Case 1
\subsubsection{$\SCdruleTXXaxName$}
\begin{itemize}
\item Case 1:
      \begin{center}
        \scriptsize
        $\Pi_1$:
        \begin{math}
          $$\mprset{flushleft}
          \inferrule* [right={\tiny ax}] {
            \,
          }{\SCnt{A}  \vdash_\mathcal{L}  \SCnt{A}}
        \end{math}
        \qquad\qquad
        \begin{math}
          \begin{array}{c}
            \Pi_2 \\
            {\Gamma_{{\mathrm{1}}}  \SCsym{;}  \SCnt{A}  \SCsym{;}  \Gamma_{{\mathrm{2}}}  \vdash_\mathcal{L}  \SCnt{B}}
          \end{array}
        \end{math}
      \end{center}
      By assumption, $c(\Pi_1),c(\Pi_2)\leq |A|$. The proof $\Pi$ is the
      same as $\Pi_2$.

% L-id Case 2
\item Case 2:
      \begin{center}
        \scriptsize
        $\Pi_1$:
        \begin{math}
          \begin{array}{c}
            \Pi_1 \\
            {\Delta  \vdash_\mathcal{L}  \SCnt{A}}
          \end{array}
        \end{math}
        \qquad\qquad
        $\Pi_2$:
        \begin{math}
          $$\mprset{flushleft}
          \inferrule* [right={\tiny ax}] {
            \,
          }{\SCnt{A}  \vdash_\mathcal{L}  \SCnt{A}}
        \end{math}
      \end{center}
      By assumption, $c(\Pi_1),c(\Pi_2)\leq |X|$. The proof $\Pi$ is the
      same as $\Pi_1$.
\end{itemize}
% End of subsubsection Axiom steps



\subsection{The Exchange Steps}

\subsubsection{$\SCdruleTXXexName$}

\begin{itemize}
% Conclusion vs. C-ex Case 1
\item Case 1:
      \begin{center}
        \scriptsize
        \begin{math}
          \begin{array}{c}
            \Pi_1 \\
            {\Psi  \vdash_\mathcal{C}  \SCnt{X_{{\mathrm{1}}}}}
          \end{array}
        \end{math}
        \qquad\qquad
        $\Pi_2$:
        \begin{math}
          $$\mprset{flushleft}
          \inferrule* [right={\tiny ex}] {
            {
              \begin{array}{c}
                \pi \\
                {\Phi_{{\mathrm{1}}}  \SCsym{,}  \SCnt{X_{{\mathrm{1}}}}  \SCsym{,}  \SCnt{X_{{\mathrm{2}}}}  \SCsym{,}  \Phi_{{\mathrm{2}}}  \vdash_\mathcal{C}  \SCnt{Y}}
              \end{array}
            }
          }{\Phi_{{\mathrm{1}}}  \SCsym{,}  \SCnt{X_{{\mathrm{2}}}}  \SCsym{,}  \SCnt{X_{{\mathrm{1}}}}  \SCsym{,}  \Phi_{{\mathrm{2}}}  \vdash_\mathcal{C}  \SCnt{Y}}
        \end{math}
      \end{center}
      By assumption, $c(\Pi_1),c(\Pi_2)\leq |X_1|$. By induction on $\pi$
      and $\Pi_1$, there is a proof $\Pi'$ for sequent
      $\Phi_{{\mathrm{1}}}  \SCsym{,}  \Psi  \SCsym{,}  \SCnt{X_{{\mathrm{2}}}}  \SCsym{,}  \Phi_{{\mathrm{2}}}  \vdash_\mathcal{C}  \SCnt{Y}$ s.t. $c(\Pi')\leq|X_1|$. Therefore, the
      proof $\Pi$ can be constructed as follows, and
      $c(\Pi)=c(\Pi')\leq|X_1|$.
      \begin{center}
        \scriptsize
        \begin{math}
          $$\mprset{flushleft}
          \inferrule* [right={\tiny series of ex}] {
            {
              \begin{array}{c}
                \Pi' \\
                {\Phi_{{\mathrm{1}}}  \SCsym{,}  \Psi  \SCsym{,}  \SCnt{X_{{\mathrm{2}}}}  \SCsym{,}  \Phi_{{\mathrm{2}}}  \vdash_\mathcal{C}  \SCnt{Y}}
              \end{array}
            }
          }{\Phi_{{\mathrm{1}}}  \SCsym{,}  \SCnt{X_{{\mathrm{2}}}}  \SCsym{,}  \Psi  \SCsym{,}  \Phi_{{\mathrm{2}}}  \vdash_\mathcal{C}  \SCnt{Y}}
        \end{math}
      \end{center}

% Conclusion vs. C-ex Case 2
\item Case 2:
      \begin{center}
        \scriptsize
        \begin{math}
          \begin{array}{c}
            \Pi_1 \\
            {\Psi  \vdash_\mathcal{C}  \SCnt{X_{{\mathrm{2}}}}}
          \end{array}
        \end{math}
        \qquad\qquad
        $\Pi_2$:
        \begin{math}
          $$\mprset{flushleft}
          \inferrule* [right={\tiny ex}] {
            {
              \begin{array}{c}
                \pi \\
                {\Phi_{{\mathrm{1}}}  \SCsym{,}  \SCnt{X_{{\mathrm{1}}}}  \SCsym{,}  \SCnt{X_{{\mathrm{2}}}}  \SCsym{,}  \Phi_{{\mathrm{2}}}  \vdash_\mathcal{C}  \SCnt{Y}}
              \end{array}
            }
          }{\Phi_{{\mathrm{1}}}  \SCsym{,}  \SCnt{X_{{\mathrm{2}}}}  \SCsym{,}  \SCnt{X_{{\mathrm{1}}}}  \SCsym{,}  \Phi_{{\mathrm{2}}}  \vdash_\mathcal{C}  \SCnt{Y}}
        \end{math}
      \end{center}
      By assumption, $c(\Pi_1),c(\Pi_2)\leq |X_2|$. By induction on $\pi$
      and $\Pi_1$, there is a proof $\Pi'$ for sequent
      $\Phi_{{\mathrm{1}}}  \SCsym{,}  \SCnt{X_{{\mathrm{1}}}}  \SCsym{,}  \Psi  \SCsym{,}  \Phi_{{\mathrm{2}}}  \vdash_\mathcal{C}  \SCnt{Y}$ s.t. $c(\Pi')\leq|X_2|$. Therefore, the
      proof $\Pi$ can be constructed as follows, and
      $c(\Pi)=c(\Pi')\leq|X_2|$.
      \begin{center}
        \scriptsize
        \begin{math}
          $$\mprset{flushleft}
          \inferrule* [right={\tiny series of ex}] {
            {
              \begin{array}{c}
                \Pi' \\
                {\Phi_{{\mathrm{1}}}  \SCsym{,}  \SCnt{X_{{\mathrm{1}}}}  \SCsym{,}  \Psi  \SCsym{,}  \Phi_{{\mathrm{2}}}  \vdash_\mathcal{C}  \SCnt{Y}}
              \end{array}
            }
          }{\Phi_{{\mathrm{1}}}  \SCsym{,}  \Psi  \SCsym{,}  \SCnt{X_{{\mathrm{1}}}}  \SCsym{,}  \Phi_{{\mathrm{2}}}  \vdash_\mathcal{C}  \SCnt{Y}}
        \end{math}
      \end{center}
\end{itemize}

% Conclusion vs. LC-ex Case 1
\subsubsection{$\SCdruleSXXexName$}
\begin{itemize}
\item Case 1:
      \begin{center}
        \scriptsize
        \begin{math}
          \begin{array}{c}
            \Pi_1 \\
            {\Psi  \vdash_\mathcal{C}  \SCnt{X_{{\mathrm{1}}}}}
          \end{array}
        \end{math}
        \qquad\qquad
        $\Pi_2$:
        \begin{math}
          $$\mprset{flushleft}
          \inferrule* [right={\tiny ex}] {
            {
              \begin{array}{c}
                \pi \\
                {\Delta_{{\mathrm{1}}}  \SCsym{;}  \SCnt{X_{{\mathrm{1}}}}  \SCsym{;}  \SCnt{X_{{\mathrm{2}}}}  \SCsym{;}  \Delta_{{\mathrm{2}}}  \vdash_\mathcal{L}  \SCnt{A}}
              \end{array}
            }
          }{\Delta_{{\mathrm{1}}}  \SCsym{;}  \SCnt{X_{{\mathrm{2}}}}  \SCsym{;}  \SCnt{X_{{\mathrm{1}}}}  \SCsym{;}  \Delta_{{\mathrm{2}}}  \vdash_\mathcal{L}  \SCnt{A}}
        \end{math}
      \end{center}
      By assumption, $c(\Pi_1),c(\Pi_2)\leq |X_1|$. By induction on $\pi$
      and $\Pi_1$, there is a proof $\Pi'$ for sequent
      $\Delta_{{\mathrm{1}}}  \SCsym{;}  \Psi  \SCsym{;}  \SCnt{X_{{\mathrm{2}}}}  \SCsym{;}  \Delta_{{\mathrm{2}}}  \vdash_\mathcal{L}  \SCnt{A}$ s.t. $c(\Pi')\leq|X_1|$. Therefore, the
      proof $\Pi$ can be constructed as follows, and
      $c(\Pi)=c(\Pi')\leq|X_1|$.
      \begin{center}
        \scriptsize
        \begin{math}
          $$\mprset{flushleft}
          \inferrule* [right={\tiny series of ex}] {
            {
              \begin{array}{c}
                \Pi' \\
                {\Delta_{{\mathrm{1}}}  \SCsym{;}  \Psi  \SCsym{;}  \SCnt{X_{{\mathrm{2}}}}  \SCsym{;}  \Delta_{{\mathrm{2}}}  \vdash_\mathcal{L}  \SCnt{A}}
              \end{array}
            }
          }{\Delta_{{\mathrm{1}}}  \SCsym{;}  \SCnt{X_{{\mathrm{2}}}}  \SCsym{;}  \Psi  \SCsym{;}  \Delta_{{\mathrm{2}}}  \vdash_\mathcal{L}  \SCnt{A}}
        \end{math}
      \end{center}

% Conclusion vs. LC-ex Case 2
\item Case 2:
      \begin{center}
        \scriptsize
        \begin{math}
          \begin{array}{c}
            \Pi_1 \\
            {\Psi  \vdash_\mathcal{C}  \SCnt{X_{{\mathrm{2}}}}}
          \end{array}
        \end{math}
        \qquad\qquad
        $\Pi_2$:
        \begin{math}
          $$\mprset{flushleft}
          \inferrule* [right={\tiny ex}] {
            {
              \begin{array}{c}
                \pi \\
                {\Delta_{{\mathrm{1}}}  \SCsym{;}  \SCnt{X_{{\mathrm{1}}}}  \SCsym{;}  \SCnt{X_{{\mathrm{2}}}}  \SCsym{;}  \Delta_{{\mathrm{2}}}  \vdash_\mathcal{L}  \SCnt{A}}
              \end{array}
            }
          }{\Delta_{{\mathrm{1}}}  \SCsym{;}  \SCnt{X_{{\mathrm{2}}}}  \SCsym{;}  \SCnt{X_{{\mathrm{1}}}}  \SCsym{;}  \Delta_{{\mathrm{2}}}  \vdash_\mathcal{L}  \SCnt{A}}
        \end{math}
      \end{center}
      By assumption, $c(\Pi_1),c(\Pi_2)\leq |X_2|$. By induction on $\pi$
      and $\Pi_1$, there is a proof $\Pi'$ for sequent
      $\Delta_{{\mathrm{1}}}  \SCsym{;}  \SCnt{X_{{\mathrm{1}}}}  \SCsym{;}  \Psi  \SCsym{;}  \Delta_{{\mathrm{2}}}  \vdash_\mathcal{L}  \SCnt{A}$ s.t. $c(\Pi')\leq|X_2|$. Therefore, the
      proof $\Pi$ can be constructed as follows, and
      $c(\Pi)=c(\Pi')\leq|X_2|$.
      \begin{center}
        \scriptsize
        \begin{math}
          $$\mprset{flushleft}
          \inferrule* [right={\tiny series of ex}] {
            {
              \begin{array}{c}
                \Pi' \\
                {\Delta_{{\mathrm{1}}}  \SCsym{;}  \SCnt{X_{{\mathrm{1}}}}  \SCsym{;}  \Psi  \SCsym{;}  \Phi_{{\mathrm{2}}}  \vdash_\mathcal{L}  \SCnt{A}}
              \end{array}
            }
          }{\Phi_{{\mathrm{1}}}  \SCsym{;}  \Psi  \SCsym{;}  \SCnt{X_{{\mathrm{1}}}}  \SCsym{;}  \Phi_{{\mathrm{2}}}  \vdash_\mathcal{L}  \SCnt{A}}
        \end{math}
      \end{center}
\end{itemize}



\subsection{Principal Formula vs. Principal Formula} 

\subsubsection{The Commutative Tensor Product $\otimes$}
\begin{center}
  \scriptsize
  $\Pi_1:$
  \begin{math}
    $$\mprset{flushleft}
    \inferrule* [right={\tiny tenR}] {
      {
        \begin{array}{cc}
          \pi_1 & \pi_2 \\
          {\Phi_{{\mathrm{1}}}  \vdash_\mathcal{C}  \SCnt{X}} & {\Phi_{{\mathrm{2}}}  \vdash_\mathcal{C}  \SCnt{Y}}
        \end{array}
      }
    }{\Phi_{{\mathrm{1}}}  \SCsym{,}  \Phi_{{\mathrm{2}}}  \vdash_\mathcal{C}  \SCnt{X}  \otimes  \SCnt{Y}}
  \end{math}
  \qquad\qquad
  $\Pi_2:$
  \begin{math}
    $$\mprset{flushleft}
    \inferrule* [right={\tiny tenL}] {
      {
        \begin{array}{c}
          \pi_3 \\
          {\Psi_{{\mathrm{1}}}  \SCsym{,}  \SCnt{X}  \SCsym{,}  \SCnt{Y}  \SCsym{,}  \Psi_{{\mathrm{2}}}  \vdash_\mathcal{C}  \SCnt{Z}}
        \end{array}
      }
    }{\Psi_{{\mathrm{1}}}  \SCsym{,}  \SCnt{X}  \otimes  \SCnt{Y}  \SCsym{,}  \Psi_{{\mathrm{2}}}  \vdash_\mathcal{C}  \SCnt{Z}}
  \end{math}
\end{center}
By assumption, $c(\Pi_1),c(\Pi_2)\leq |\SCnt{X}  \otimes  \SCnt{Y}| = |X|+|Y|+1$. The proof
$\Pi$ can be constructed as follows, and
$c(\Pi)\leq max\{c(\pi_1),c(\pi_2),c(\pi_3),|X|+1,|Y|+1\}\leq |X|+|Y|+1 = |\SCnt{X}  \otimes  \SCnt{Y}|$.
\begin{center}
  \scriptsize
  \begin{math}
    $$\mprset{flushleft}
    \inferrule* [right={\tiny cut}] {
      {
        \begin{array}{c}
          \pi_1 \\
          {\Phi_{{\mathrm{1}}}  \vdash_\mathcal{C}  \SCnt{X}}
        \end{array}
      }
      $$\mprset{flushleft}
      \inferrule* [right={\tiny cut}] {
      {
        \begin{array}{cc}
          \pi_2 & \pi_3 \\
          {\Phi_{{\mathrm{2}}}  \vdash_\mathcal{C}  \SCnt{Y}} & {\Psi_{{\mathrm{1}}}  \SCsym{,}  \SCnt{X}  \SCsym{,}  \SCnt{Y}  \SCsym{,}  \Psi_{{\mathrm{2}}}  \vdash_\mathcal{C}  \SCnt{Z}}
        \end{array}
      }
      }{\Psi_{{\mathrm{1}}}  \SCsym{,}  \SCnt{X}  \SCsym{,}  \Phi_{{\mathrm{2}}}  \SCsym{,}  \Psi_{{\mathrm{2}}}  \vdash_\mathcal{C}  \SCnt{Z}}
    }{\Psi_{{\mathrm{1}}}  \SCsym{,}  \Phi_{{\mathrm{1}}}  \SCsym{,}  \Phi_{{\mathrm{2}}}  \SCsym{,}  \Psi_{{\mathrm{2}}}  \vdash_\mathcal{C}  \SCnt{Z}}
  \end{math}
\end{center}

\subsubsection{The Non-commutative Tensor Product $\tri$}
\begin{center}
  \scriptsize
  $\Pi_1:$
  \begin{math}
    $$\mprset{flushleft}
    \inferrule* [right={\tiny tenR}] {
      {
        \begin{array}{cc}
          \pi_1 & \pi_2 \\
          {\Gamma_{{\mathrm{1}}}  \vdash_\mathcal{L}  \SCnt{A}} & {\Gamma_{{\mathrm{2}}}  \vdash_\mathcal{L}  \SCnt{B}}
        \end{array}
      }
    }{\Gamma_{{\mathrm{1}}}  \SCsym{;}  \Gamma_{{\mathrm{2}}}  \vdash_\mathcal{L}  \SCnt{A}  \triangleright  \SCnt{B}}
  \end{math}
  \qquad\qquad
  $\Pi_2:$
  \begin{math}
    $$\mprset{flushleft}
    \inferrule* [right={\tiny tenL1}] {
      {
        \begin{array}{c}
          \pi_3 \\
          {\Delta_{{\mathrm{1}}}  \SCsym{;}  \SCnt{A}  \SCsym{;}  \SCnt{B}  \SCsym{;}  \Delta_{{\mathrm{2}}}  \vdash_\mathcal{L}  \SCnt{C}}
        \end{array}
      }
    }{\Delta_{{\mathrm{1}}}  \SCsym{;}  \SCnt{A}  \triangleright  \SCnt{B}  \SCsym{;}  \Delta_{{\mathrm{2}}}  \vdash_\mathcal{L}  \SCnt{C}}
  \end{math}
\end{center}
By assumption, $c(\Pi_1),c(\Pi_2)\leq |\SCnt{A}  \triangleright  \SCnt{B}| = |X|+|Y|+1$. The proof
$\Pi$ can be constructed as follows, and
$c(\Pi)\leq max\{c(\pi_1),c(\pi_2),c(\pi_3),|A|+1,|B|+1\}\leq |A|+|B|+1 = |\SCnt{A}  \triangleright  \SCnt{B}|$.
\begin{center}
  \scriptsize
  \begin{math}
    $$\mprset{flushleft}
    \inferrule* [right={\tiny cut2}] {
      {
        \begin{array}{c}
          \pi_1 \\
          {\Gamma_{{\mathrm{1}}}  \vdash_\mathcal{L}  \SCnt{A}}
        \end{array}
      }
      $$\mprset{flushleft}
      \inferrule* [right={\tiny cut2}] {
      {
        \begin{array}{cc}
          \pi_2 & \pi_3 \\
          {\Gamma_{{\mathrm{2}}}  \vdash_\mathcal{L}  \SCnt{B}} & {\Delta_{{\mathrm{1}}}  \SCsym{;}  \SCnt{A}  \SCsym{;}  \SCnt{B}  \SCsym{;}  \Delta_{{\mathrm{2}}}  \vdash_\mathcal{L}  \SCnt{C}}
        \end{array}
      }
      }{\Delta_{{\mathrm{1}}}  \SCsym{;}  \SCnt{A}  \SCsym{;}  \Gamma_{{\mathrm{2}}}  \SCsym{;}  \Delta_{{\mathrm{2}}}  \vdash_\mathcal{L}  \SCnt{C}}
    }{\Delta_{{\mathrm{1}}}  \SCsym{;}  \Gamma_{{\mathrm{1}}}  \SCsym{;}  \Gamma_{{\mathrm{2}}}  \SCsym{;}  \Psi_{{\mathrm{2}}}  \vdash_\mathcal{L}  \SCnt{C}}
  \end{math}
\end{center}

\subsubsection{The Commutative Implication $\multimap$}
\begin{center}
  \scriptsize
  $\Pi_1:$
  \begin{math}
    $$\mprset{flushleft}
    \inferrule* [right={\tiny tenR}] {
      {
        \begin{array}{c}
          \pi_1 \\
          {\Phi_{{\mathrm{1}}}  \SCsym{,}  \SCnt{X}  \vdash_\mathcal{C}  \SCnt{Y}}
        \end{array}
      }
    }{\Phi_{{\mathrm{1}}}  \vdash_\mathcal{C}  \SCnt{X}  \multimap  \SCnt{Y}}
  \end{math}
  \qquad\qquad
  $\Pi_2:$
  \begin{math}
    $$\mprset{flushleft}
    \inferrule* [right={\tiny tenL}] {
      {
        \begin{array}{cc}
          \pi_2 & \pi_3 \\
          {\Phi_{{\mathrm{2}}}  \vdash_\mathcal{C}  \SCnt{X}} & {\Psi_{{\mathrm{1}}}  \SCsym{,}  \SCnt{Y}  \SCsym{,}  \Psi_{{\mathrm{2}}}  \vdash_\mathcal{C}  \SCnt{Z}}
        \end{array}
      }
    }{\Psi_{{\mathrm{1}}}  \SCsym{,}  \SCnt{X}  \multimap  \SCnt{Y}  \SCsym{,}  \Phi  \SCsym{,}  \Psi_{{\mathrm{2}}}  \vdash_\mathcal{C}  \SCnt{Z}}
  \end{math}
\end{center}
By assumption, $c(\Pi_1),c(\Pi_2)\leq |\SCnt{X}  \multimap  \SCnt{Y}| = |X|+|Y|+1$. The proof 
$\Pi$ is constructed as follows
$c(\Pi)\leq max\{c(\pi_1),c(\pi_2),c(\pi_3),|X|+1,|Y|+1\}\leq |X|+|Y|+1 = |\SCnt{X}  \multimap  \SCnt{Y}|$.
\begin{center}
  \scriptsize
  \begin{math}
    $$\mprset{flushleft}
    \inferrule* [right={\tiny tenR}] {
      $$\mprset{flushleft}
      \inferrule* [right={\tiny tenR}] {
        {
          \begin{array}{cc}
            \pi_1 & \pi_2 \\
            {\Phi_{{\mathrm{1}}}  \SCsym{,}  \SCnt{X}  \vdash_\mathcal{C}  \SCnt{Y}} & {\Phi_{{\mathrm{2}}}  \vdash_\mathcal{C}  \SCnt{X}}
          \end{array}
        }
      }{\Phi_{{\mathrm{1}}}  \SCsym{,}  \Phi_{{\mathrm{2}}}  \vdash_\mathcal{C}  \SCnt{Y}} \\
       {
         \begin{array}{c}
           \pi_3 \\
           {\Psi_{{\mathrm{1}}}  \SCsym{,}  \SCnt{Y}  \SCsym{,}  \Psi_{{\mathrm{2}}}  \vdash_\mathcal{C}  \SCnt{Z}}
         \end{array}
       }
    }{\Psi_{{\mathrm{1}}}  \SCsym{,}  \Phi_{{\mathrm{1}}}  \SCsym{,}  \Phi_{{\mathrm{2}}}  \SCsym{,}  \Psi_{{\mathrm{2}}}  \vdash_\mathcal{C}  \SCnt{Z}}
  \end{math}
\end{center}

\subsubsection{The Non-commutative Right Implication $\lto$}
\begin{center}
  \scriptsize
  $\Pi_1:$
  \begin{math}
    $$\mprset{flushleft}
    \inferrule* [right={\tiny imprR}] {
      {
        \begin{array}{c}
          \pi_1 \\
          {\Gamma  \SCsym{;}  \SCnt{A}  \vdash_\mathcal{L}  \SCnt{B}}
        \end{array}
      }
    }{\Gamma  \vdash_\mathcal{L}  \SCnt{A}  \rightharpoonup  \SCnt{B}}
  \end{math}
  \qquad\qquad
  $\Pi_2:$
  \begin{math}
    $$\mprset{flushleft}
    \inferrule* [right={\tiny imprL}] {
      {
        \begin{array}{cc}
          \pi_2 & \pi_3 \\
          {\Delta_{{\mathrm{1}}}  \vdash_\mathcal{L}  \SCnt{A}} & {\Delta_{{\mathrm{2}}}  \SCsym{;}  \SCnt{B}  \vdash_\mathcal{L}  \SCnt{C}}
        \end{array}
      }
    }{\Delta_{{\mathrm{2}}}  \SCsym{;}  \SCnt{A}  \rightharpoonup  \SCnt{B}  \SCsym{;}  \Delta_{{\mathrm{1}}}  \vdash_\mathcal{L}  \SCnt{C}}
  \end{math}
\end{center}
By assumption, $c(\Pi_1),c(\Pi_2)\leq |\SCnt{A}  \rightharpoonup  \SCnt{B}| = |A|+|B|+1$. The proof
$\Pi$ is constructed as follows, and
$c(\Pi)\leq max\{c(\pi_1),c(\pi_2),c(\pi_3),|A|+1,|B|+1\}\leq |A|+|B|+1 = |\SCnt{A}  \rightharpoonup  \SCnt{B}|$.
\begin{center}
  \scriptsize
  \begin{math}
    $$\mprset{flushleft}
    \inferrule* [right={\tiny cut2}] {
      $$\mprset{flushleft}
      \inferrule* [right={\tiny cut2}] {
        {
          \begin{array}{cc}
            \pi_1 & \pi_2 \\
            {\Gamma  \SCsym{;}  \SCnt{A}  \vdash_\mathcal{L}  \SCnt{B}} & {\Delta_{{\mathrm{1}}}  \vdash_\mathcal{L}  \SCnt{A}}
          \end{array}
        }
      }{\Gamma  \SCsym{;}  \Delta_{{\mathrm{1}}}  \vdash_\mathcal{L}  \SCnt{B}}
       {
         \begin{array}{c}
           \pi_3 \\
           {\Delta_{{\mathrm{2}}}  \SCsym{;}  \SCnt{B}  \vdash_\mathcal{L}  \SCnt{C}}
         \end{array}
       }
    }{\Delta_{{\mathrm{2}}}  \SCsym{;}  \Gamma  \SCsym{;}  \Delta_{{\mathrm{1}}}  \vdash_\mathcal{L}  \SCnt{C}}
  \end{math}
\end{center}

\subsubsection{The Non-commutative Left Implication $\rto$}
\begin{center}
  \scriptsize
  $\Pi_1:$
  \begin{math}
    $$\mprset{flushleft}
    \inferrule* [right={\tiny implR}] {
      {
        \begin{array}{c}
          \pi_1 \\
          {\SCnt{A}  \SCsym{;}  \Gamma  \vdash_\mathcal{L}  \SCnt{B}}
        \end{array}
      }
    }{\Gamma  \vdash_\mathcal{L}  \SCnt{B}  \leftharpoonup  \SCnt{A}}
  \end{math}
  \qquad\qquad
  $\Pi_2:$
  \begin{math}
    $$\mprset{flushleft}
    \inferrule* [right={\tiny implL}] {
      {
        \begin{array}{cc}
          \pi_2 & \pi_3 \\
          {\Delta_{{\mathrm{1}}}  \vdash_\mathcal{L}  \SCnt{A}} & {\SCnt{B}  \SCsym{;}  \Delta_{{\mathrm{2}}}  \vdash_\mathcal{L}  \SCnt{C}}
        \end{array}
      }
    }{\Delta_{{\mathrm{1}}}  \SCsym{;}  \SCnt{B}  \leftharpoonup  \SCnt{A}  \SCsym{;}  \Delta_{{\mathrm{2}}}  \vdash_\mathcal{L}  \SCnt{C}}
  \end{math}
\end{center}
By assumption, $c(\Pi_1),c(\Pi_2)\leq |\SCnt{B}  \leftharpoonup  \SCnt{A}| = |A|+|B|+1$. The
proof $\Pi$ is constructed as follows, and
$c(\Pi)\leq max\{c(\pi_1),c(\pi_2),c(\pi_3),|A|+1,|B|+1\}\leq |A|+|B|+1 = |\SCnt{B}  \leftharpoonup  \SCnt{A}|$.
\begin{center}
  \scriptsize
  \begin{math}
    $$\mprset{flushleft}
    \inferrule* [right={\tiny cut1}] {
      $$\mprset{flushleft}
      \inferrule* [right={\tiny cut2}] {
        {
          \begin{array}{cc}
            \pi_1 & \pi_2 \\
            {\SCnt{A}  \SCsym{;}  \Gamma  \vdash_\mathcal{L}  \SCnt{B}} & {\Delta_{{\mathrm{1}}}  \vdash_\mathcal{L}  \SCnt{A}}
          \end{array}
        }
      }{\Delta_{{\mathrm{1}}}  \SCsym{;}  \Gamma  \vdash_\mathcal{L}  \SCnt{B}}
       {
         \begin{array}{c}
           \pi_3 \\
           {\SCnt{B}  \SCsym{;}  \Delta_{{\mathrm{2}}}  \vdash_\mathcal{L}  \SCnt{C}}
         \end{array}
       }
    }{\Delta_{{\mathrm{1}}}  \SCsym{;}  \Gamma  \SCsym{;}  \Delta_{{\mathrm{2}}}  \vdash_\mathcal{L}  \SCnt{C}}
  \end{math}
\end{center}



\subsubsection{The Commutative Unit $ \mathsf{Unit} $}
\begin{itemize}
\item Case 1:
      \begin{center}
        \scriptsize
        $\Pi_1:$
        \begin{math}
          $$\mprset{flushleft}
          \inferrule* [right={\tiny unitR}] {
            \,
          }{ \cdot   \vdash_\mathcal{C}   \mathsf{Unit} }
        \end{math}
        \qquad\qquad
        $\Pi_2:$
        \begin{math}
          $$\mprset{flushleft}
          \inferrule* [right={\tiny unitL}] {
            {
              \begin{array}{c}
                \pi \\
                {\Phi  \SCsym{,}  \Psi  \vdash_\mathcal{C}  \SCnt{X}}
              \end{array}
            }
          }{\Phi  \SCsym{,}   \mathsf{Unit}   \SCsym{,}  \Psi  \vdash_\mathcal{C}  \SCnt{X}}
        \end{math}
      \end{center}
      By assumption, $c(\Pi_1),c(\Pi_2)\leq | \mathsf{Unit} |$. The proof $\Pi$
      is the subproof $\pi$ in $\Pi_2$ for sequent $\Phi  \vdash_\mathcal{C}  \SCnt{X}$. So
      $c(\Pi)=c(\Pi_2)\leq | \mathsf{Unit} |$.

\item Case 2:
      \begin{center}
        \scriptsize
        $\Pi_1:$
        \begin{math}
          $$\mprset{flushleft}
          \inferrule* [right={\tiny unitR}] {
            \,
          }{ \cdot   \vdash_\mathcal{C}   \mathsf{Unit} }
        \end{math}
        \qquad\qquad
        $\Pi_2:$
        \begin{math}
          $$\mprset{flushleft}
          \inferrule* [right={\tiny unitL1}] {
            {
              \begin{array}{c}
                \pi \\
                {\Gamma  \SCsym{;}  \Delta  \vdash_\mathcal{L}  \SCnt{A}}
              \end{array}
            }
          }{\Gamma  \SCsym{;}   \mathsf{Unit}   \SCsym{;}  \Delta  \vdash_\mathcal{L}  \SCnt{A}}
        \end{math}
      \end{center}
      Similar as above, $\Pi$ is $\pi$.
\end{itemize}


\subsubsection{The Non-commutative Unit $ \mathsf{Unit} $}
\begin{center}
  \scriptsize
  $\Pi_1:$
  \begin{math}
    $$\mprset{flushleft}
    \inferrule* [right={\tiny unitR}] {
      \,
    }{ \cdot   \vdash_\mathcal{L}   \mathsf{Unit} }
  \end{math}
  \qquad\qquad
  $\Pi_2:$
  \begin{math}
    $$\mprset{flushleft}
    \inferrule* [right={\tiny unitL2}] {
      {
        \begin{array}{c}
          \pi \\
          {\Gamma  \SCsym{;}  \Delta  \vdash_\mathcal{L}  \SCnt{A}}
        \end{array}
      }
    }{\Gamma  \SCsym{;}   \mathsf{Unit}   \SCsym{;}  \Delta  \vdash_\mathcal{L}  \SCnt{A}}
  \end{math}
\end{center}
By assumption, $c(\Pi_1),c(\Pi_2)\leq | \mathsf{Unit} |$. The proof $\Pi$ is the
subproof $\pi$ in $\Pi_2$ for sequent $\Delta  \vdash_\mathcal{L}  \SCnt{A}$. So
$c(\Pi)=c(\Pi_2)\leq | \mathsf{Unit} |$.

\subsubsection{The Functor $F$}
\begin{center}
  \scriptsize
  $\Pi_1:$
  \begin{math}
    $$\mprset{flushleft}
    \inferrule* [right={\tiny FR}] {
      {
        \begin{array}{c}
          \pi_1 \\
          {\Phi  \vdash_\mathcal{C}  \SCnt{X}}
        \end{array}
      }
    }{\Phi  \vdash_\mathcal{L}   \mathsf{F} \SCnt{X} }
  \end{math}
  \qquad\qquad
  $\Pi_2:$
  \begin{math}
    $$\mprset{flushleft}
    \inferrule* [right={\tiny FL}] {
      {
        \begin{array}{c}
          \pi_2 \\
          {\Gamma  \SCsym{;}  \SCnt{X}  \SCsym{;}  \Delta  \vdash_\mathcal{L}  \SCnt{A}}
        \end{array}
      }
    }{\Gamma  \SCsym{;}   \mathsf{F} \SCnt{X}   \SCsym{;}  \Delta  \vdash_\mathcal{L}  \SCnt{A}}
  \end{math}
\end{center}
By assumption, $c(\Pi_1),c(\Pi_2)\leq | \mathsf{F} \SCnt{X} | = |X|+1$. The proof
$\Pi$ is constructed as follows, and \\
$c(\Pi)\leq max\{c(\pi_1),c(\pi_2),|X|+1\}\leq | \mathsf{F} \SCnt{X} |$.
\begin{center}
  \scriptsize
  \begin{math}
    $$\mprset{flushleft}
    \inferrule* [right={\tiny cut2}] {
      {
        \begin{array}{cc}
          \pi_1 & \pi_2 \\
          {\Phi  \vdash_\mathcal{C}  \SCnt{X}} & {\Gamma  \SCsym{;}  \SCnt{X}  \SCsym{;}  \Delta  \vdash_\mathcal{L}  \SCnt{A}}
        \end{array}
      }
    }{\Gamma  \SCsym{;}  \Phi  \SCsym{;}  \Delta  \vdash_\mathcal{L}  \SCnt{A}}
  \end{math}
\end{center}

\subsubsection{The Functor $G$}
\begin{center}
  \scriptsize
  $\Pi_1:$
  \begin{math}
    $$\mprset{flushleft}
    \inferrule* [right={\tiny GR}] {
      {
        \begin{array}{c}
          \pi_1 \\
          {\Phi  \vdash_\mathcal{L}  \SCnt{A}}
        \end{array}
      }
    }{\Phi  \vdash_\mathcal{C}   \mathsf{G} \SCnt{A} }
  \end{math}
  \qquad\qquad
  $\Pi_2:$
  \begin{math}
    $$\mprset{flushleft}
    \inferrule* [right={\tiny GL}] {
      {
        \begin{array}{c}
          \pi_2 \\
          {\Gamma  \SCsym{;}  \SCnt{A}  \SCsym{;}  \Delta  \vdash_\mathcal{L}  \SCnt{B}}
        \end{array}
      }
    }{\Gamma  \SCsym{;}   \mathsf{G} \SCnt{A}   \SCsym{;}  \Delta  \vdash_\mathcal{L}  \SCnt{B}}
  \end{math}
\end{center}
By assumption, $c(\Pi_1),c(\Pi_2)\leq | \mathsf{G} \SCnt{A} | = |A|+1$. The proof $\Pi$ 
is constructed as follows, and \\
$c(\Pi)\leq max\{c(\pi_1),c(\pi_2),|A|+1\}\leq | \mathsf{G} \SCnt{A} |$.
\begin{center}
  \scriptsize
  \begin{math}
    $$\mprset{flushleft}
    \inferrule* [right={\tiny GL}] {
      {
        \begin{array}{cc}
          \pi_1 & \pi_2 \\
          {\Phi  \vdash_\mathcal{L}  \SCnt{A}} & {\Gamma  \SCsym{;}  \SCnt{A}  \SCsym{;}  \Delta  \vdash_\mathcal{L}  \SCnt{B}}
        \end{array}
      }
    }{\Gamma  \SCsym{;}  \Phi  \SCsym{;}  \Delta  \vdash_\mathcal{L}  \SCnt{B}}
  \end{math}
\end{center}



\subsection{Secondary Conclusion}

\subsubsection{Left introduction of the commutative implication $\multimap$}
\begin{itemize}
\item Case 1:
      \begin{center}
        \scriptsize
        $\Pi_1$:
        \begin{math}
          $$\mprset{flushleft}
          \inferrule* [right={\tiny impL}] {
            {
              \begin{array}{cc}
                \pi_1 & \pi_2 \\
                {\Phi_{{\mathrm{1}}}  \vdash_\mathcal{C}  \SCnt{X_{{\mathrm{1}}}}} & {\Phi_{{\mathrm{2}}}  \SCsym{,}  \SCnt{X_{{\mathrm{2}}}}  \SCsym{,}  \Phi_{{\mathrm{3}}}  \vdash_\mathcal{C}  \SCnt{Y}}
              \end{array}
            }
          }{\Phi_{{\mathrm{2}}}  \SCsym{,}  \SCnt{X_{{\mathrm{1}}}}  \multimap  \SCnt{X_{{\mathrm{2}}}}  \SCsym{,}  \Phi_{{\mathrm{1}}}  \SCsym{,}  \Phi_{{\mathrm{3}}}  \vdash_\mathcal{C}  \SCnt{Y}}
        \end{math}
        \qquad\qquad
        \begin{math}
          \begin{array}{c}
            \Pi_2 \\
            {\Psi_{{\mathrm{1}}}  \SCsym{,}  \SCnt{Y}  \SCsym{,}  \Psi_{{\mathrm{2}}}  \vdash_\mathcal{C}  \SCnt{Z}}
          \end{array}
        \end{math}
      \end{center}
      By assumption, $c(\Pi_1),c(\Pi_2)\leq |Y|$. By induction, there is a
      proof $\Pi'$ from $\pi_2$ and $\Pi_2$ for sequent
      $\Psi_{{\mathrm{1}}}  \SCsym{,}  \Phi_{{\mathrm{2}}}  \SCsym{,}  \SCnt{X_{{\mathrm{2}}}}  \SCsym{,}  \Phi_{{\mathrm{3}}}  \SCsym{,}  \Psi_{{\mathrm{2}}}  \vdash_\mathcal{C}  \SCnt{Z}$ s.t. $c(\Pi')\leq |Y|$. Therefore,
      the proof $\Pi$ can be constructed as follows with $c(\Pi)\leq |Y|$.
      \begin{center}
        \scriptsize
        \begin{math}
          $$\mprset{flushleft}
          \inferrule* [right={\tiny impL}] {
            {
              \begin{array}{c}
                \pi_1 \\
                {\Phi_{{\mathrm{1}}}  \vdash_\mathcal{C}  \SCnt{X_{{\mathrm{1}}}}}
              \end{array}
            }
            $$\mprset{flushleft}
            \inferrule* [right={\tiny cut}] {
              {
                \begin{array}{cc}
                  \pi_2 & \Pi_2 \\
                  {\Phi_{{\mathrm{2}}}  \SCsym{,}  \SCnt{X_{{\mathrm{2}}}}  \SCsym{,}  \Phi_{{\mathrm{3}}}  \vdash_\mathcal{C}  \SCnt{Y}} & {\Psi_{{\mathrm{1}}}  \SCsym{,}  \SCnt{Y}  \SCsym{,}  \Psi_{{\mathrm{2}}}  \vdash_\mathcal{C}  \SCnt{Z}}
                \end{array}
              }
            }{\Psi_{{\mathrm{1}}}  \SCsym{,}  \Phi_{{\mathrm{2}}}  \SCsym{,}  \SCnt{X_{{\mathrm{2}}}}  \SCsym{,}  \Phi_{{\mathrm{3}}}  \SCsym{,}  \Psi_{{\mathrm{2}}}  \vdash_\mathcal{C}  \SCnt{Z}}
          }{\Psi_{{\mathrm{1}}}  \SCsym{,}  \Phi_{{\mathrm{2}}}  \SCsym{,}  \SCnt{X_{{\mathrm{1}}}}  \multimap  \SCnt{X_{{\mathrm{2}}}}  \SCsym{,}  \Phi_{{\mathrm{1}}}  \SCsym{,}  \Phi_{{\mathrm{3}}}  \SCsym{,}  \Psi_{{\mathrm{2}}}  \vdash_\mathcal{C}  \SCnt{Z}}
        \end{math}
      \end{center}

\item Case 2:
      \begin{center}
        \scriptsize
        $\Pi_1$:
        \begin{math}
          $$\mprset{flushleft}
          \inferrule* [right={\tiny impL}] {
            {
              \begin{array}{cc}
                \pi_1 & \pi_2 \\
                {\Phi_{{\mathrm{1}}}  \vdash_\mathcal{C}  \SCnt{X_{{\mathrm{1}}}}} & {\Phi_{{\mathrm{2}}}  \SCsym{,}  \SCnt{X_{{\mathrm{2}}}}  \SCsym{,}  \Phi_{{\mathrm{3}}}  \vdash_\mathcal{C}  \SCnt{Y}}
              \end{array}
            }
          }{\Phi_{{\mathrm{2}}}  \SCsym{,}  \SCnt{X_{{\mathrm{1}}}}  \multimap  \SCnt{X_{{\mathrm{2}}}}  \SCsym{,}  \Phi_{{\mathrm{1}}}  \SCsym{,}  \Phi_{{\mathrm{3}}}  \vdash_\mathcal{C}  \SCnt{Y}}
        \end{math}
        \qquad\qquad
        \begin{math}
          \begin{array}{c}
            \Pi_2 \\
            {\Gamma_{{\mathrm{1}}}  \SCsym{;}  \SCnt{Y}  \SCsym{;}  \Gamma_{{\mathrm{2}}}  \vdash_\mathcal{L}  \SCnt{A}}
          \end{array}
        \end{math}
      \end{center}
      By assumption, $c(\Pi_1),c(\Pi_2)\leq |Y|$. By induction, there is a
      proof $\Pi'$ from $\pi_2$ and $\Pi_2$ for sequent
      $\Gamma_{{\mathrm{1}}}  \SCsym{;}  \Phi_{{\mathrm{2}}}  \SCsym{;}  \SCnt{X_{{\mathrm{2}}}}  \SCsym{;}  \Phi_{{\mathrm{3}}}  \SCsym{;}  \Gamma_{{\mathrm{2}}}  \vdash_\mathcal{L}  \SCnt{A}$ s.t. $c(\Pi')\leq |Y|$. Therefore, the
      proof $\Pi$ can be constructed as follows with $c(\Pi)\leq |Y|$.
      \begin{center}
        \scriptsize
        \begin{math}
          $$\mprset{flushleft}
          \inferrule* [right={\tiny impL}] {
            {
              \begin{array}{c}
                \pi_1 \\
                {\Phi_{{\mathrm{1}}}  \vdash_\mathcal{C}  \SCnt{X_{{\mathrm{1}}}}}
              \end{array}
            }
            $$\mprset{flushleft}
            \inferrule* [right={\tiny cut}] {
              {
                \begin{array}{cc}
                  \pi_2 & \Pi_2 \\
                  {\Phi_{{\mathrm{2}}}  \SCsym{,}  \SCnt{X_{{\mathrm{2}}}}  \SCsym{,}  \Phi_{{\mathrm{3}}}  \vdash_\mathcal{C}  \SCnt{Y}} & {\Gamma_{{\mathrm{1}}}  \SCsym{;}  \SCnt{Y}  \SCsym{;}  \Gamma_{{\mathrm{2}}}  \vdash_\mathcal{L}  \SCnt{A}}
                \end{array}
              }
            }{\Gamma_{{\mathrm{1}}}  \SCsym{;}  \Phi_{{\mathrm{2}}}  \SCsym{;}  \SCnt{X_{{\mathrm{2}}}}  \SCsym{;}  \Phi_{{\mathrm{3}}}  \SCsym{;}  \Gamma_{{\mathrm{2}}}  \vdash_\mathcal{L}  \SCnt{A}}
          }{\Gamma_{{\mathrm{1}}}  \SCsym{;}  \Phi_{{\mathrm{2}}}  \SCsym{;}  \SCnt{X_{{\mathrm{1}}}}  \multimap  \SCnt{X_{{\mathrm{2}}}}  \SCsym{;}  \Phi_{{\mathrm{1}}}  \SCsym{;}  \Phi_{{\mathrm{3}}}  \SCsym{;}  \Gamma_{{\mathrm{2}}}  \vdash_\mathcal{L}  \SCnt{A}}
        \end{math}
      \end{center}
\end{itemize}



\subsubsection{Left introduction of the non-commutative left implication $\lto$}
\begin{center}
\scriptsize
  $\Pi_1$:
  \begin{math}
    $$\mprset{flushleft}
    \inferrule* [right={\tiny impL}] {
      {
        \begin{array}{cc}
          \pi_1 & \pi_2 \\
          {\Gamma_{{\mathrm{1}}}  \vdash_\mathcal{L}  \SCnt{A_{{\mathrm{1}}}}} & {\Gamma_{{\mathrm{2}}}  \SCsym{;}  \SCnt{A_{{\mathrm{2}}}}  \SCsym{;}  \Gamma_{{\mathrm{3}}}  \vdash_\mathcal{L}  \SCnt{B}}
        \end{array}
      }
    }{\Gamma_{{\mathrm{2}}}  \SCsym{;}  \SCnt{A_{{\mathrm{1}}}}  \rightharpoonup  \SCnt{A_{{\mathrm{2}}}}  \SCsym{;}  \Gamma_{{\mathrm{1}}}  \SCsym{;}  \Gamma_{{\mathrm{3}}}  \vdash_\mathcal{L}  \SCnt{B}}
  \end{math}
  \qquad\qquad
  \begin{math}
    \begin{array}{c}
      \Pi_2 \\
      {\Delta_{{\mathrm{1}}}  \SCsym{;}  \SCnt{B}  \SCsym{;}  \Delta_{{\mathrm{2}}}  \vdash_\mathcal{L}  \SCnt{C}}
    \end{array}
  \end{math}
\end{center}
By assumption, $c(\Pi_1),c(\Pi_2)\leq |B|$. By induction, there is a
proof $\Pi'$ from $\pi_2$ and $\Pi_2$ for sequent
$\Delta_{{\mathrm{1}}}  \SCsym{;}  \Gamma_{{\mathrm{2}}}  \SCsym{;}  \SCnt{A_{{\mathrm{2}}}}  \SCsym{;}  \Gamma_{{\mathrm{3}}}  \SCsym{;}  \Delta_{{\mathrm{2}}}  \vdash_\mathcal{L}  \SCnt{C}$ s.t. $c(\Pi')\leq |B|$.
Therefore, the proof $\Pi$ can be constructed as follows with
$c(\Pi)\leq |B|$.
\begin{center}
  \scriptsize
  \begin{math}
    $$\mprset{flushleft}
    \inferrule* [right={\tiny impL}] {
      {
        \begin{array}{c}
          \pi_1 \\
          {\Gamma_{{\mathrm{1}}}  \vdash_\mathcal{L}  \SCnt{A_{{\mathrm{1}}}}}
        \end{array}
      }
      $$\mprset{flushleft}
      \inferrule* [right={\tiny cut}] {
        {
          \begin{array}{cc}
            \pi_2 & \Pi_2 \\
            {\Gamma_{{\mathrm{2}}}  \SCsym{;}  \SCnt{A_{{\mathrm{2}}}}  \SCsym{;}  \Gamma_{{\mathrm{3}}}  \vdash_\mathcal{L}  \SCnt{B}} & {\Delta_{{\mathrm{1}}}  \SCsym{;}  \SCnt{B}  \SCsym{;}  \Delta_{{\mathrm{2}}}  \vdash_\mathcal{L}  \SCnt{C}}
          \end{array}
        }
      }{\Delta_{{\mathrm{1}}}  \SCsym{;}  \Gamma_{{\mathrm{2}}}  \SCsym{;}  \SCnt{A_{{\mathrm{2}}}}  \SCsym{;}  \Gamma_{{\mathrm{3}}}  \SCsym{;}  \Delta_{{\mathrm{2}}}  \vdash_\mathcal{L}  \SCnt{C}}
    }{\Delta_{{\mathrm{1}}}  \SCsym{;}  \Gamma_{{\mathrm{2}}}  \SCsym{;}  \SCnt{A_{{\mathrm{1}}}}  \rightharpoonup  \SCnt{A_{{\mathrm{2}}}}  \SCsym{;}  \Gamma_{{\mathrm{1}}}  \SCsym{;}  \Gamma_{{\mathrm{3}}}  \SCsym{;}  \Delta_{{\mathrm{2}}}  \vdash_\mathcal{L}  \SCnt{C}}
  \end{math}
\end{center}


\subsubsection{Left introduction of the non-commutative right implication $\rto$}
\begin{center}
  \scriptsize
  $\Pi_1$:
  \begin{math}
    $$\mprset{flushleft}
    \inferrule* [right={\tiny impL}] {
      {
        \begin{array}{cc}
          \pi_1 & \pi_2 \\
          {\Gamma_{{\mathrm{1}}}  \vdash_\mathcal{L}  \SCnt{A_{{\mathrm{1}}}}} & {\Gamma_{{\mathrm{2}}}  \SCsym{;}  \SCnt{A_{{\mathrm{2}}}}  \SCsym{;}  \Gamma_{{\mathrm{3}}}  \vdash_\mathcal{L}  \SCnt{B}}
        \end{array}
      }
    }{\Gamma_{{\mathrm{2}}}  \SCsym{;}  \Gamma_{{\mathrm{1}}}  \SCsym{;}  \SCnt{A_{{\mathrm{2}}}}  \leftharpoonup  \SCnt{A_{{\mathrm{1}}}}  \SCsym{;}  \Gamma_{{\mathrm{3}}}  \vdash_\mathcal{L}  \SCnt{B}}
  \end{math}
  \qquad\qquad
  \begin{math}
    \begin{array}{c}
      \Pi_2 \\
      {\Delta_{{\mathrm{1}}}  \SCsym{;}  \SCnt{B}  \SCsym{;}  \Delta_{{\mathrm{2}}}  \vdash_\mathcal{L}  \SCnt{C}}
    \end{array}
  \end{math}
\end{center}
By assumption, $c(\Pi_1),c(\Pi_2)\leq |B|$. By induction, there is a
proof $\Pi'$ from $\pi_2$ and $\Pi_2$ for sequent
$\Delta_{{\mathrm{1}}}  \SCsym{;}  \Gamma_{{\mathrm{2}}}  \SCsym{;}  \SCnt{A_{{\mathrm{2}}}}  \SCsym{;}  \Gamma_{{\mathrm{3}}}  \SCsym{;}  \Delta_{{\mathrm{2}}}  \vdash_\mathcal{L}  \SCnt{C}$ s.t. $c(\Pi')\leq |B|$. Therefore, the
proof $\Pi$ can be constructed as follows with $c(\Pi)\leq |B|$.
\begin{center}
  \scriptsize
  \begin{math}
    $$\mprset{flushleft}
    \inferrule* [right={\tiny impL}] {
      {
        \begin{array}{c}
          \pi_1 \\
          {\Gamma_{{\mathrm{1}}}  \vdash_\mathcal{L}  \SCnt{A_{{\mathrm{1}}}}}
        \end{array}
      }
      $$\mprset{flushleft}
      \inferrule* [right={\tiny cut}] {
        {
          \begin{array}{cc}
            \pi_2 & \Pi_2 \\
            {\Gamma_{{\mathrm{2}}}  \SCsym{;}  \SCnt{A_{{\mathrm{2}}}}  \SCsym{;}  \Gamma_{{\mathrm{3}}}  \vdash_\mathcal{L}  \SCnt{B}} & {\Delta_{{\mathrm{1}}}  \SCsym{;}  \SCnt{B}  \SCsym{;}  \Delta_{{\mathrm{2}}}  \vdash_\mathcal{L}  \SCnt{C}}
          \end{array}
        }
      }{\Delta_{{\mathrm{1}}}  \SCsym{;}  \Gamma_{{\mathrm{2}}}  \SCsym{;}  \SCnt{A_{{\mathrm{2}}}}  \SCsym{;}  \Gamma_{{\mathrm{3}}}  \SCsym{;}  \Delta_{{\mathrm{2}}}  \vdash_\mathcal{L}  \SCnt{C}}
    }{\Delta_{{\mathrm{1}}}  \SCsym{;}  \Gamma_{{\mathrm{2}}}  \SCsym{;}  \Gamma_{{\mathrm{1}}}  \SCsym{;}  \SCnt{A_{{\mathrm{2}}}}  \leftharpoonup  \SCnt{A_{{\mathrm{1}}}}  \SCsym{;}  \Gamma_{{\mathrm{3}}}  \SCsym{;}  \Delta_{{\mathrm{2}}}  \vdash_\mathcal{L}  \SCnt{C}}
  \end{math}
\end{center}

% C-ex Case 1
\subsubsection{$\SCdruleTXXexName$}
\begin{itemize}
\item Case 1:
      \begin{center}
        \scriptsize
        $\Pi_1$:
        \begin{math}
          $$\mprset{flushleft}
          \inferrule* [right={\tiny ex}] {
            {
              \begin{array}{c}
                \pi \\
                {\Phi_{{\mathrm{1}}}  \SCsym{,}  \SCnt{X_{{\mathrm{1}}}}  \SCsym{,}  \SCnt{X_{{\mathrm{2}}}}  \SCsym{,}  \Phi_{{\mathrm{2}}}  \vdash_\mathcal{C}  \SCnt{Y}}
              \end{array}
            }
          }{\Phi_{{\mathrm{1}}}  \SCsym{,}  \SCnt{X_{{\mathrm{2}}}}  \SCsym{,}  \SCnt{X_{{\mathrm{1}}}}  \SCsym{,}  \Phi_{{\mathrm{2}}}  \vdash_\mathcal{C}  \SCnt{Y}}
        \end{math}
        \qquad\qquad
        \begin{math}
          \begin{array}{c}
            \Pi_2 \\
            {\Psi_{{\mathrm{1}}}  \SCsym{,}  \SCnt{Y}  \SCsym{,}  \Psi_{{\mathrm{2}}}  \vdash_\mathcal{C}  \SCnt{Z}}
          \end{array}
        \end{math}
      \end{center}
      By assumption, $c(\Pi_1),c(\Pi_2)\leq |Y|$. By induction on $\pi$
      and $\Pi_2$, there is a proof $\Pi'$ for sequent
      $\Psi_{{\mathrm{1}}}  \SCsym{,}  \Phi_{{\mathrm{1}}}  \SCsym{,}  \SCnt{X_{{\mathrm{1}}}}  \SCsym{,}  \SCnt{X_{{\mathrm{2}}}}  \SCsym{,}  \Phi_{{\mathrm{2}}}  \SCsym{,}  \Psi_{{\mathrm{2}}}  \vdash_\mathcal{C}  \SCnt{Z}$ s.t. $c(\Pi')\leq|Y|$. Therefore,
      the proof $\Pi$ can be constructed as follows, and
      $c(\Pi)=c(\Pi')\leq|Y|$.
      \begin{center}
        \scriptsize
        \begin{math}
          $$\mprset{flushleft}
          \inferrule* [right={\tiny ex}] {
            {
              \begin{array}{c}
                \Pi' \\
                {\Psi_{{\mathrm{1}}}  \SCsym{,}  \Phi_{{\mathrm{1}}}  \SCsym{,}  \SCnt{X_{{\mathrm{1}}}}  \SCsym{,}  \SCnt{X_{{\mathrm{2}}}}  \SCsym{,}  \Phi_{{\mathrm{2}}}  \SCsym{,}  \Psi_{{\mathrm{2}}}  \vdash_\mathcal{C}  \SCnt{Z}}
              \end{array}
            }
          }{\Psi_{{\mathrm{1}}}  \SCsym{,}  \Phi_{{\mathrm{1}}}  \SCsym{,}  \SCnt{X_{{\mathrm{2}}}}  \SCsym{,}  \SCnt{X_{{\mathrm{1}}}}  \SCsym{,}  \Phi_{{\mathrm{2}}}  \SCsym{,}  \Psi_{{\mathrm{2}}}  \vdash_\mathcal{C}  \SCnt{Z}}
        \end{math}
      \end{center}

% C-ex Case 2
\item Case 2:
      \begin{center}
        \scriptsize
        $\Pi_1$:
        \begin{math}
          $$\mprset{flushleft}
          \inferrule* [right={\tiny beta}] {
            {
              \begin{array}{c}
                \pi \\
                {\Phi_{{\mathrm{1}}}  \SCsym{,}  \SCnt{X}  \SCsym{,}  \SCnt{Y}  \SCsym{,}  \Phi_{{\mathrm{2}}}  \vdash_\mathcal{C}  \SCnt{Z}}
              \end{array}
            }
          }{\Phi_{{\mathrm{1}}}  \SCsym{,}  \SCnt{Y}  \SCsym{,}  \SCnt{X}  \SCsym{,}  \Phi_{{\mathrm{2}}}  \vdash_\mathcal{C}  \SCnt{Z}}
        \end{math}
        \qquad\qquad
        \begin{math}
          \begin{array}{c}
            \Pi_2 \\
            {\Gamma_{{\mathrm{1}}}  \SCsym{;}  \SCnt{Z}  \SCsym{;}  \Gamma_{{\mathrm{2}}}  \vdash_\mathcal{L}  \SCnt{A}}
          \end{array}
        \end{math}
      \end{center}
      By assumption, $c(\Pi_1),c(\Pi_2)\leq |Z|$. Similar as above, there
      is a proof $\Pi'$ constructed from $\pi$ and $\Pi_2$ for 
      $\Gamma_{{\mathrm{1}}}  \SCsym{;}  \Phi_{{\mathrm{1}}}  \SCsym{;}  \SCnt{X}  \SCsym{;}  \SCnt{Y}  \SCsym{;}  \Phi_{{\mathrm{2}}}  \SCsym{;}  \Gamma_{{\mathrm{2}}}  \vdash_\mathcal{L}  \SCnt{A}$ s.t. $c(\Pi')\leq|Z|$. Therefore,
      the proof $\Pi$ can be constructed as follows, and
      $c(\Pi)=c(\Pi')\leq|Z|$.
      \begin{center}
        \scriptsize
        \begin{math}
          $$\mprset{flushleft}
          \inferrule* [right={\tiny beta}] {
            {
              \begin{array}{c}
                \Pi' \\
                {\Gamma_{{\mathrm{1}}}  \SCsym{;}  \Phi_{{\mathrm{1}}}  \SCsym{;}  \SCnt{X}  \SCsym{;}  \SCnt{Y}  \SCsym{;}  \Phi_{{\mathrm{2}}}  \SCsym{;}  \Gamma_{{\mathrm{2}}}  \vdash_\mathcal{L}  \SCnt{A}}
              \end{array}
            }
          }{\Gamma_{{\mathrm{1}}}  \SCsym{;}  \Phi_{{\mathrm{1}}}  \SCsym{;}  \SCnt{Y}  \SCsym{;}  \SCnt{X}  \SCsym{;}  \Phi_{{\mathrm{2}}}  \SCsym{;}  \Gamma_{{\mathrm{2}}}  \vdash_\mathcal{L}  \SCnt{A}}
        \end{math}
      \end{center}
\end{itemize}

% LC-ex
\subsubsection{$\SCdruleSXXexName$}
\begin{center}
  \scriptsize
  $\Pi_1$:
  \begin{math}
    $$\mprset{flushleft}
    \inferrule* [right={\tiny beta}] {
      {
        \begin{array}{c}
          \pi \\
          {\Gamma_{{\mathrm{1}}}  \SCsym{;}  \SCnt{X}  \SCsym{;}  \SCnt{Y}  \SCsym{;}  \Gamma_{{\mathrm{2}}}  \vdash_\mathcal{L}  \SCnt{A}}
        \end{array}
      }
    }{\Gamma_{{\mathrm{1}}}  \SCsym{;}  \SCnt{Y}  \SCsym{;}  \SCnt{X}  \SCsym{;}  \Gamma_{{\mathrm{2}}}  \vdash_\mathcal{L}  \SCnt{A}}
  \end{math}
  \qquad\qquad
  \begin{math}
    \begin{array}{c}
      \Pi_2 \\
      {\Delta_{{\mathrm{1}}}  \SCsym{;}  \SCnt{A}  \SCsym{;}  \Delta_{{\mathrm{2}}}  \vdash_\mathcal{L}  \SCnt{B}}
    \end{array}
  \end{math}
\end{center}
By assumption, $c(\Pi_1),c(\Pi_2)\leq |A|$. Similar as above, there
is a proof $\Pi'$ constructed from $\pi$ and $\Pi_2$ for sequent
$\Delta_{{\mathrm{1}}}  \SCsym{;}  \Gamma_{{\mathrm{1}}}  \SCsym{;}  \SCnt{X}  \SCsym{;}  \SCnt{Y}  \SCsym{;}  \Gamma_{{\mathrm{2}}}  \SCsym{;}  \Delta_{{\mathrm{2}}}  \vdash_\mathcal{L}  \SCnt{B}$ s.t. $c(\Pi')\leq|A|$. Therefore,
the proof $\Pi$ can be constructed as follows, and
$c(\Pi)=c(\Pi')\leq|A|$.
\begin{center}
  \scriptsize
  \begin{math}
    $$\mprset{flushleft}
    \inferrule* [right={\tiny beta}] {
      {
        \begin{array}{cc}
          \Pi' \\
          {\Delta_{{\mathrm{1}}}  \SCsym{;}  \Gamma_{{\mathrm{1}}}  \SCsym{;}  \SCnt{X}  \SCsym{;}  \SCnt{Y}  \SCsym{;}  \Gamma_{{\mathrm{2}}}  \SCsym{;}  \Delta_{{\mathrm{2}}}  \vdash_\mathcal{L}  \SCnt{B}}
        \end{array}
      }
    }{\Delta_{{\mathrm{1}}}  \SCsym{;}  \Gamma_{{\mathrm{1}}}  \SCsym{;}  \SCnt{Y}  \SCsym{;}  \SCnt{X}  \SCsym{;}  \Gamma_{{\mathrm{2}}}  \SCsym{;}  \Delta_{{\mathrm{2}}}  \vdash_\mathcal{L}  \SCnt{B}}
  \end{math}
\end{center}





\subsubsection{Left introduction of the commutative tensor product $\otimes$}
\begin{itemize}
\item Case 1:
      \begin{center}
        \scriptsize
        $\Pi_1$:
        \begin{math}
          $$\mprset{flushleft}
          \inferrule* [right={\tiny tenL}] {
            {
              \begin{array}{c}
                \pi \\
                {\Phi_{{\mathrm{1}}}  \SCsym{,}  \SCnt{X_{{\mathrm{1}}}}  \SCsym{,}  \SCnt{X_{{\mathrm{2}}}}  \SCsym{,}  \Phi_{{\mathrm{2}}}  \vdash_\mathcal{C}  \SCnt{Y}}
              \end{array}
            }
          }{\Phi_{{\mathrm{1}}}  \SCsym{,}  \SCnt{X_{{\mathrm{1}}}}  \otimes  \SCnt{X_{{\mathrm{2}}}}  \SCsym{,}  \Phi_{{\mathrm{2}}}  \vdash_\mathcal{C}  \SCnt{Y}}
        \end{math}
        \qquad\qquad
        \begin{math}
          \begin{array}{c}
            \Pi_2 \\
            {\Psi_{{\mathrm{1}}}  \SCsym{,}  \SCnt{Y}  \SCsym{,}  \Psi_{{\mathrm{2}}}  \vdash_\mathcal{C}  \SCnt{Z}}
          \end{array}
        \end{math}
      \end{center}
      By assumption, $c(\Pi_1),c(\Pi_2)\leq |Y|$. By induction, there is a
      proof $\Pi'$ from $\pi$ and $\Pi_2$ for sequent
      $\Psi_{{\mathrm{1}}}  \SCsym{,}  \Phi_{{\mathrm{1}}}  \SCsym{,}  \SCnt{X_{{\mathrm{1}}}}  \SCsym{,}  \SCnt{X_{{\mathrm{2}}}}  \SCsym{,}  \Phi_{{\mathrm{2}}}  \SCsym{,}  \Psi_{{\mathrm{2}}}  \vdash_\mathcal{C}  \SCnt{Z}$ s.t. $c(\Pi')\leq |Y|$. Therefore,
      the proof $\Pi$ can be constructed as follows with $c(\Pi)\leq |Y|$.
      \begin{center}
        \scriptsize
        \begin{math}
          $$\mprset{flushleft}
          \inferrule* [right={\tiny tenL}] {
            $$\mprset{flushleft}
            \inferrule* [right={\tiny cut}] {
              {
                \begin{array}{cc}
                  \pi & \Pi_2 \\
                  {\Phi_{{\mathrm{1}}}  \SCsym{,}  \SCnt{X_{{\mathrm{1}}}}  \SCsym{,}  \SCnt{X_{{\mathrm{2}}}}  \SCsym{,}  \Phi_{{\mathrm{2}}}  \vdash_\mathcal{C}  \SCnt{Y}} & {\Psi_{{\mathrm{1}}}  \SCsym{,}  \SCnt{Y}  \SCsym{,}  \Psi_{{\mathrm{2}}}  \vdash_\mathcal{C}  \SCnt{Z}}
                \end{array}
              }
            }{\Psi_{{\mathrm{1}}}  \SCsym{,}  \Phi_{{\mathrm{1}}}  \SCsym{,}  \SCnt{X_{{\mathrm{1}}}}  \SCsym{,}  \SCnt{X_{{\mathrm{2}}}}  \SCsym{,}  \Phi_{{\mathrm{2}}}  \SCsym{,}  \Psi_{{\mathrm{2}}}  \vdash_\mathcal{C}  \SCnt{Z}}
          }{\Psi_{{\mathrm{1}}}  \SCsym{,}  \Phi_{{\mathrm{1}}}  \SCsym{,}  \SCnt{X_{{\mathrm{1}}}}  \otimes  \SCnt{X_{{\mathrm{2}}}}  \SCsym{,}  \Phi_{{\mathrm{2}}}  \SCsym{,}  \Psi_{{\mathrm{2}}}  \vdash_\mathcal{C}  \SCnt{Z}}
        \end{math}
      \end{center}

\item Case 2:
      \begin{center}
        \scriptsize
        $\Pi_1$:
        \begin{math}
          $$\mprset{flushleft}
          \inferrule* [right={\tiny tenL}] {
            {
              \begin{array}{c}
                \pi \\
                {\Phi_{{\mathrm{1}}}  \SCsym{,}  \SCnt{X_{{\mathrm{1}}}}  \SCsym{,}  \SCnt{X_{{\mathrm{2}}}}  \SCsym{,}  \Phi_{{\mathrm{2}}}  \vdash_\mathcal{C}  \SCnt{Y}}
              \end{array}
            }
          }{\Phi_{{\mathrm{1}}}  \SCsym{,}  \SCnt{X_{{\mathrm{1}}}}  \otimes  \SCnt{X_{{\mathrm{2}}}}  \SCsym{,}  \Phi_{{\mathrm{2}}}  \vdash_\mathcal{C}  \SCnt{Y}}
        \end{math}
        \qquad\qquad
        \begin{math}
          \begin{array}{c}
            \Pi_2 \\
            {\Gamma_{{\mathrm{1}}}  \SCsym{;}  \SCnt{Y}  \SCsym{;}  \Gamma_{{\mathrm{2}}}  \vdash_\mathcal{L}  \SCnt{A}}
          \end{array}
        \end{math}
      \end{center}
      By assumption, $c(\Pi_1),c(\Pi_2)\leq |Y|$. By induction, there is a
      proof $\Pi'$ from $\pi$ and $\Pi_2$ for sequent
      $\Gamma_{{\mathrm{1}}}  \SCsym{;}  \Phi_{{\mathrm{1}}}  \SCsym{;}  \SCnt{X_{{\mathrm{1}}}}  \SCsym{;}  \SCnt{X_{{\mathrm{2}}}}  \SCsym{;}  \Phi_{{\mathrm{2}}}  \SCsym{;}  \Gamma_{{\mathrm{2}}}  \vdash_\mathcal{L}  \SCnt{A}$ s.t. $c(\Pi')\leq |Y|$. Therefore,
      the proof $\Pi$ can be constructed as follows with $c(\Pi)\leq |Y|$.
      \begin{center}
        \scriptsize
        \begin{math}
          $$\mprset{flushleft}
          \inferrule* [right={\tiny tenL1}] {
            $$\mprset{flushleft}
            \inferrule* [right={\tiny cut1}] {
              {
                \begin{array}{cc}
                  \pi & \Pi_2 \\
                  {\Phi_{{\mathrm{1}}}  \SCsym{,}  \SCnt{X_{{\mathrm{1}}}}  \SCsym{,}  \SCnt{X_{{\mathrm{2}}}}  \SCsym{,}  \Phi_{{\mathrm{2}}}  \vdash_\mathcal{C}  \SCnt{Y}} & {\Gamma_{{\mathrm{1}}}  \SCsym{;}  \SCnt{Y}  \SCsym{;}  \Gamma_{{\mathrm{2}}}  \vdash_\mathcal{L}  \SCnt{A}}
                \end{array}
              }
            }{\Gamma_{{\mathrm{1}}}  \SCsym{;}  \Phi_{{\mathrm{1}}}  \SCsym{;}  \SCnt{X_{{\mathrm{1}}}}  \SCsym{;}  \SCnt{X_{{\mathrm{2}}}}  \SCsym{;}  \Phi_{{\mathrm{2}}}  \SCsym{;}  \Gamma_{{\mathrm{2}}}  \vdash_\mathcal{L}  \SCnt{A}}
          }{\Gamma_{{\mathrm{1}}}  \SCsym{;}  \Phi_{{\mathrm{1}}}  \SCsym{;}  \SCnt{X_{{\mathrm{1}}}}  \otimes  \SCnt{X_{{\mathrm{2}}}}  \SCsym{;}  \Phi_{{\mathrm{2}}}  \SCsym{;}  \Gamma_{{\mathrm{2}}}  \vdash_\mathcal{L}  \SCnt{A}}
        \end{math}
      \end{center}

\item Case 3:
      \begin{center}
        \scriptsize
        $\Pi_1$:
        \begin{math}
          $$\mprset{flushleft}
          \inferrule* [right={\tiny tenL}] {
            {
              \begin{array}{c}
                \pi \\
                {\Gamma_{{\mathrm{1}}}  \SCsym{;}  \SCnt{X}  \SCsym{;}  \SCnt{Y}  \SCsym{;}  \Gamma_{{\mathrm{2}}}  \vdash_\mathcal{L}  \SCnt{A}}
              \end{array}
            }
          }{\Gamma_{{\mathrm{1}}}  \SCsym{;}  \SCnt{X}  \otimes  \SCnt{Y}  \SCsym{;}  \Gamma_{{\mathrm{2}}}  \vdash_\mathcal{L}  \SCnt{A}}
        \end{math}
        \qquad\qquad
        \begin{math}
          \begin{array}{c}
            \Pi_2 \\
            {\Delta_{{\mathrm{1}}}  \SCsym{;}  \SCnt{A}  \SCsym{;}  \Delta_{{\mathrm{2}}}  \vdash_\mathcal{L}  \SCnt{B}}
          \end{array}
        \end{math}
      \end{center}
      By assumption, $c(\Pi_1),c(\Pi_2)\leq |A|$. By induction, there is a
      proof $\Pi'$ from $\pi$ and $\Pi_2$ for sequent
      $\Delta_{{\mathrm{1}}}  \SCsym{;}  \SCnt{X}  \SCsym{;}  \SCnt{Y}  \SCsym{;}  \Gamma_{{\mathrm{2}}}  \SCsym{;}  \Delta_{{\mathrm{2}}}  \vdash_\mathcal{L}  \SCnt{B}$ s.t. $c(\Pi')\leq |A|$. Therefore, the
      proof $\Pi$ can be constructed as follows with $c(\Pi)\leq |A|$.
      \begin{center}
        \scriptsize
        \begin{math}
          $$\mprset{flushleft}
          \inferrule* [right={\tiny tenL1}] {
            $$\mprset{flushleft}
            \inferrule* [right={\tiny cut2}] {
              {
                \begin{array}{cc}
                  \pi & \Pi_2 \\
                  {\Gamma_{{\mathrm{1}}}  \SCsym{;}  \SCnt{X}  \SCsym{;}  \SCnt{Y}  \SCsym{;}  \Gamma_{{\mathrm{2}}}  \vdash_\mathcal{L}  \SCnt{A}} & {\Delta_{{\mathrm{1}}}  \SCsym{;}  \SCnt{A}  \SCsym{;}  \Delta_{{\mathrm{2}}}  \vdash_\mathcal{L}  \SCnt{B}}
                \end{array}
              }
            }{\Delta_{{\mathrm{1}}}  \SCsym{;}  \Gamma_{{\mathrm{1}}}  \SCsym{;}  \SCnt{X}  \SCsym{;}  \SCnt{Y}  \SCsym{;}  \Gamma_{{\mathrm{2}}}  \SCsym{;}  \Delta_{{\mathrm{2}}}  \vdash_\mathcal{L}  \SCnt{B}}
          }{\Delta_{{\mathrm{1}}}  \SCsym{;}  \Gamma_{{\mathrm{1}}}  \SCsym{;}  \SCnt{X}  \otimes  \SCnt{Y}  \SCsym{;}  \Gamma_{{\mathrm{2}}}  \SCsym{;}  \Delta_{{\mathrm{2}}}  \vdash_\mathcal{L}  \SCnt{B}}
        \end{math}
      \end{center}
\end{itemize}

\subsubsection{Left introduction of the non-commutative tensor products $\tri$}
\begin{center}
  \scriptsize
  $\Pi_1$:
  \begin{math}
    $$\mprset{flushleft}
    \inferrule* [right={\tiny tenL2}] {
      {
        \begin{array}{c}
          \pi \\
          {\Gamma_{{\mathrm{1}}}  \SCsym{;}  \SCnt{A_{{\mathrm{1}}}}  \SCsym{;}  \SCnt{A_{{\mathrm{2}}}}  \SCsym{;}  \Gamma_{{\mathrm{2}}}  \vdash_\mathcal{L}  \SCnt{B}}
        \end{array}
      }
    }{\Gamma_{{\mathrm{1}}}  \SCsym{;}  \SCnt{A_{{\mathrm{1}}}}  \triangleright  \SCnt{A_{{\mathrm{2}}}}  \SCsym{;}  \Gamma_{{\mathrm{2}}}  \vdash_\mathcal{L}  \SCnt{B}}
  \end{math}
  \qquad\qquad
  \begin{math}
    \begin{array}{c}
      \Pi_2 \\
      {\Delta_{{\mathrm{1}}}  \SCsym{;}  \SCnt{B}  \SCsym{;}  \Delta_{{\mathrm{2}}}  \vdash_\mathcal{L}  \SCnt{C}}
    \end{array}
  \end{math}
\end{center}
By assumption, $c(\Pi_1),c(\Pi_2)\leq |B|$. By induction, there is a
proof $\Pi'$ from $\pi$ and $\Pi_2$ for sequent \\
$\Delta_{{\mathrm{1}}}  \SCsym{;}  \Gamma_{{\mathrm{1}}}  \SCsym{;}  \SCnt{A_{{\mathrm{1}}}}  \SCsym{;}  \SCnt{A_{{\mathrm{2}}}}  \SCsym{;}  \Gamma_{{\mathrm{2}}}  \SCsym{;}  \Delta_{{\mathrm{2}}}  \vdash_\mathcal{L}  \SCnt{C}$ s.t. $c(\Pi')\leq |B|$.
Therefore, the proof $\Pi$ can be constructed as follows with
$c(\Pi)\leq |B|$.
\begin{center}
  \scriptsize
  \begin{math}
    $$\mprset{flushleft}
    \inferrule* [right={\tiny tenL2}] {
      $$\mprset{flushleft}
      \inferrule* [right={\tiny cut2}] {
        {
          \begin{array}{cc}
            \pi & \Pi_2 \\
            {\Gamma_{{\mathrm{1}}}  \SCsym{;}  \SCnt{A_{{\mathrm{1}}}}  \SCsym{;}  \SCnt{A_{{\mathrm{2}}}}  \SCsym{;}  \Gamma_{{\mathrm{2}}}  \vdash_\mathcal{L}  \SCnt{B}} & {\Delta_{{\mathrm{1}}}  \SCsym{;}  \SCnt{B}  \SCsym{;}  \Delta_{{\mathrm{2}}}  \vdash_\mathcal{L}  \SCnt{C}}
          \end{array}
        }
      }{\Delta_{{\mathrm{1}}}  \SCsym{;}  \Gamma_{{\mathrm{1}}}  \SCsym{;}  \SCnt{A_{{\mathrm{1}}}}  \SCsym{;}  \SCnt{A_{{\mathrm{2}}}}  \SCsym{;}  \Gamma_{{\mathrm{2}}}  \SCsym{;}  \Delta_{{\mathrm{2}}}  \vdash_\mathcal{L}  \SCnt{C}}
    }{\Delta_{{\mathrm{1}}}  \SCsym{;}  \Gamma_{{\mathrm{1}}}  \SCsym{;}  \SCnt{A_{{\mathrm{1}}}}  \triangleright  \SCnt{A_{{\mathrm{2}}}}  \SCsym{;}  \Gamma_{{\mathrm{2}}}  \SCsym{;}  \Delta_{{\mathrm{2}}}  \vdash_\mathcal{L}  \SCnt{C}}
  \end{math}
\end{center}



\subsubsection{Left introduction of the commutative unit $ \mathsf{Unit} $}
\begin{itemize}
\item Case 1:
      \begin{center}
        \scriptsize
        $\Pi_1$:
        \begin{math}
          $$\mprset{flushleft}
          \inferrule* [right={\tiny unitL}] {
            {
              \begin{array}{c}
                \pi \\
                {\Phi_{{\mathrm{1}}}  \SCsym{,}  \Phi_{{\mathrm{2}}}  \vdash_\mathcal{C}  \SCnt{X}}
              \end{array}
            }
          }{\Phi_{{\mathrm{1}}}  \SCsym{,}   \mathsf{Unit}   \SCsym{,}  \Phi_{{\mathrm{2}}}  \vdash_\mathcal{C}  \SCnt{X}}
        \end{math}
        \qquad\qquad
        \begin{math}
          \begin{array}{c}
            \Pi_2 \\
            {\Psi_{{\mathrm{1}}}  \SCsym{,}  \SCnt{X}  \SCsym{,}  \Psi_{{\mathrm{2}}}  \vdash_\mathcal{C}  \SCnt{Y}}
          \end{array}
        \end{math}
      \end{center}
      By assumption, $c(\Pi_1),c(\Pi_2)\leq |X|$. By induction, there is a
      proof $\Pi'$ from $\pi$ and $\Pi_2$ for sequent
      $\Psi_{{\mathrm{1}}}  \SCsym{,}  \Phi_{{\mathrm{1}}}  \SCsym{,}  \Phi_{{\mathrm{2}}}  \SCsym{,}  \Psi_{{\mathrm{2}}}  \vdash_\mathcal{C}  \SCnt{Y}$
      s.t. $c(\Pi')\leq |X|$. Therefore, the proof $\Pi$ can be constructed
      as follows, and $c(\Pi)=c(\Pi')\leq |X|$.
      \begin{center}
        \scriptsize
        \begin{math}
          $$\mprset{flushleft}
          \inferrule* [right={\tiny unitL}] {
            {
              \begin{array}{c}
                \Pi' \\
                {\Psi_{{\mathrm{1}}}  \SCsym{,}  \Phi_{{\mathrm{1}}}  \SCsym{,}  \Phi_{{\mathrm{2}}}  \SCsym{,}  \Psi_{{\mathrm{2}}}  \vdash_\mathcal{C}  \SCnt{Y}}
              \end{array}
            }
          }{\Psi_{{\mathrm{1}}}  \SCsym{,}  \Phi_{{\mathrm{1}}}  \SCsym{,}   \mathsf{Unit}   \SCsym{,}  \Phi_{{\mathrm{2}}}  \SCsym{,}  \Psi_{{\mathrm{2}}}  \vdash_\mathcal{C}  \SCnt{Y}}
        \end{math}
      \end{center}

\item Case 2:
      \begin{center}
        \scriptsize
        $\Pi_1$:
        \begin{math}
          $$\mprset{flushleft}
          \inferrule* [right={\tiny unitL}] {
            {
              \begin{array}{c}
                \pi \\
                {\Phi_{{\mathrm{1}}}  \SCsym{,}  \Phi_{{\mathrm{2}}}  \vdash_\mathcal{C}  \SCnt{X}}
              \end{array}
            }
          }{\Phi_{{\mathrm{1}}}  \SCsym{,}   \mathsf{Unit}   \SCsym{,}  \Phi_{{\mathrm{2}}}  \vdash_\mathcal{C}  \SCnt{X}}
        \end{math}
        \qquad\qquad
        \begin{math}
          \begin{array}{c}
            \Pi_2 \\
            {\Gamma_{{\mathrm{1}}}  \SCsym{;}  \SCnt{X}  \SCsym{;}  \Gamma_{{\mathrm{2}}}  \vdash_\mathcal{L}  \SCnt{A}}
          \end{array}
        \end{math}
      \end{center}
      By assumption, $c(\Pi_1),c(\Pi_2)\leq |X|$. By induction, there is a
      proof $\Pi'$ from $\pi$ and $\Pi_2$ for sequent
      $\Gamma_{{\mathrm{1}}}  \SCsym{;}  \Phi_{{\mathrm{1}}}  \SCsym{;}  \Phi_{{\mathrm{2}}}  \SCsym{;}  \Gamma_{{\mathrm{2}}}  \vdash_\mathcal{L}  \SCnt{A}$
      s.t. $c(\Pi')\leq |X|$. Therefore, the proof $\Pi$ can be constructed
      as follows, and $c(\Pi)=c(\Pi')\leq |X|$.
      \begin{center}
        \scriptsize
        \begin{math}
          $$\mprset{flushleft}
          \inferrule* [right={\tiny unitL}] {
            {
              \begin{array}{c}
                \Pi' \\
                {\Gamma_{{\mathrm{1}}}  \SCsym{;}  \Phi_{{\mathrm{1}}}  \SCsym{;}  \Phi_{{\mathrm{2}}}  \SCsym{;}  \Gamma_{{\mathrm{2}}}  \vdash_\mathcal{L}  \SCnt{A}}
              \end{array}
            }
          }{\Gamma_{{\mathrm{1}}}  \SCsym{;}  \Phi_{{\mathrm{1}}}  \SCsym{;}   \mathsf{Unit}   \SCsym{;}  \Phi_{{\mathrm{2}}}  \SCsym{;}  \Gamma_{{\mathrm{2}}}  \vdash_\mathcal{L}  \SCnt{A}}
        \end{math}
      \end{center}

\item Case 3:
      \begin{center}
        \scriptsize
        $\Pi_1$:
        \begin{math}
          $$\mprset{flushleft}
          \inferrule* [right={\tiny unitL}] {
            {
              \begin{array}{c}
                \pi \\
                {\Delta_{{\mathrm{1}}}  \SCsym{;}  \Delta_{{\mathrm{2}}}  \vdash_\mathcal{L}  \SCnt{A}}
              \end{array}
            }
          }{\Delta_{{\mathrm{1}}}  \SCsym{;}   \mathsf{Unit}   \SCsym{;}  \Delta_{{\mathrm{2}}}  \vdash_\mathcal{L}  \SCnt{A}}
        \end{math}
        \qquad\qquad
        \begin{math}
          \begin{array}{c}
            \Pi_2 \\
            {\Gamma_{{\mathrm{1}}}  \SCsym{;}  \SCnt{A}  \SCsym{;}  \Gamma_{{\mathrm{2}}}  \vdash_\mathcal{L}  \SCnt{B}}
          \end{array}
        \end{math}
      \end{center}
      By assumption, $c(\Pi_1),c(\Pi_2)\leq |X|$. By induction, there is a
      proof $\Pi'$ from $\pi$ and $\Pi_2$ for sequent
      $\Gamma_{{\mathrm{1}}}  \SCsym{;}  \Delta_{{\mathrm{1}}}  \SCsym{;}  \Delta_{{\mathrm{2}}}  \SCsym{;}  \Gamma_{{\mathrm{2}}}  \vdash_\mathcal{L}  \SCnt{B}$
      s.t. $c(\Pi')\leq |X|$. Therefore, the proof $\Pi$ can be constructed
      as follows, and $c(\Pi)=c(\Pi')\leq |X|$.
      \begin{center}
        \scriptsize
        \begin{math}
          $$\mprset{flushleft}
          \inferrule* [right={\tiny unitL}] {
            {
              \begin{array}{c}
                \Pi' \\
                {\Gamma_{{\mathrm{1}}}  \SCsym{;}  \Delta_{{\mathrm{1}}}  \SCsym{;}  \Delta_{{\mathrm{2}}}  \SCsym{;}  \Gamma_{{\mathrm{2}}}  \vdash_\mathcal{L}  \SCnt{B}}
              \end{array}
            }
          }{\Gamma_{{\mathrm{1}}}  \SCsym{;}  \Delta_{{\mathrm{1}}}  \SCsym{;}   \mathsf{Unit}   \SCsym{;}  \Delta_{{\mathrm{2}}}  \SCsym{;}  \Gamma_{{\mathrm{2}}}  \vdash_\mathcal{L}  \SCnt{B}}
        \end{math}
      \end{center}
\end{itemize}



\subsubsection{Left introduction of the non-commutative unit $ \mathsf{Unit} $}
\begin{center}
  \scriptsize
  $\Pi_1$:
  \begin{math}
    $$\mprset{flushleft}
    \inferrule* [right={\tiny unitL}] {
      {
        \begin{array}{c}
          \pi \\
          {\Delta_{{\mathrm{1}}}  \SCsym{;}  \Delta_{{\mathrm{2}}}  \vdash_\mathcal{L}  \SCnt{A}}
        \end{array}
      }
    }{\Delta_{{\mathrm{1}}}  \SCsym{;}   \mathsf{Unit}   \SCsym{;}  \Delta_{{\mathrm{2}}}  \vdash_\mathcal{L}  \SCnt{A}}
  \end{math}
  \qquad\qquad
  \begin{math}
    \begin{array}{c}
      \Pi_2 \\
      {\Gamma_{{\mathrm{1}}}  \SCsym{;}  \SCnt{A}  \SCsym{;}  \Gamma_{{\mathrm{2}}}  \vdash_\mathcal{L}  \SCnt{B}}
    \end{array}
  \end{math}
\end{center}
By assumption, $c(\Pi_1),c(\Pi_2)\leq |X|$. By induction, there is a
proof $\Pi'$ from $\pi$ and $\Pi_2$ for sequent
$\Gamma_{{\mathrm{1}}}  \SCsym{;}  \Delta_{{\mathrm{1}}}  \SCsym{;}  \Delta_{{\mathrm{2}}}  \SCsym{;}  \Gamma_{{\mathrm{2}}}  \vdash_\mathcal{L}  \SCnt{B}$
s.t. $c(\Pi')\leq |X|$. Therefore, the proof $\Pi$ can be constructed
as follows, and $c(\Pi)=c(\Pi')\leq |X|$.
\begin{center}
  \scriptsize
  \begin{math}
    $$\mprset{flushleft}
    \inferrule* [right={\tiny unitL}] {
      {
        \begin{array}{c}
          \Pi' \\
          {\Gamma_{{\mathrm{1}}}  \SCsym{;}  \Delta_{{\mathrm{1}}}  \SCsym{;}  \Delta_{{\mathrm{2}}}  \SCsym{;}  \Gamma_{{\mathrm{2}}}  \vdash_\mathcal{L}  \SCnt{B}}
        \end{array}
      }
    }{\Gamma_{{\mathrm{1}}}  \SCsym{;}  \Delta_{{\mathrm{1}}}  \SCsym{;}   \mathsf{Unit}   \SCsym{;}  \Delta_{{\mathrm{2}}}  \SCsym{;}  \Gamma_{{\mathrm{2}}}  \vdash_\mathcal{L}  \SCnt{B}}
  \end{math}
\end{center}



\subsubsection{Left introduction of the functor $F$}
\begin{center}
  \scriptsize
  $\Pi_1$:
  \begin{math}
    $$\mprset{flushleft}
    \inferrule* [right={\tiny FL}] {
      {
        \begin{array}{c}
          \pi_1 \\
          {\Gamma_{{\mathrm{1}}}  \SCsym{;}  \SCnt{X}  \SCsym{;}  \Gamma_{{\mathrm{2}}}  \vdash_\mathcal{L}  \SCnt{A}}
        \end{array}
      }
    }{\Gamma_{{\mathrm{1}}}  \SCsym{;}   \mathsf{F} \SCnt{X}   \SCsym{;}  \Gamma_{{\mathrm{2}}}  \vdash_\mathcal{L}  \SCnt{A}}
  \end{math}
  \qquad\qquad
  \begin{math}
    \begin{array}{c}
      \Pi_2 \\
      {\Delta_{{\mathrm{1}}}  \SCsym{;}  \SCnt{A}  \SCsym{;}  \Delta_{{\mathrm{2}}}  \vdash_\mathcal{L}  \SCnt{B}}
    \end{array}
  \end{math}
\end{center}
By assumption, $c(\Pi_1),c(\Pi_2)\leq |A|$. By induction, there is a
proof $\Pi'$ from $\pi_2$ and $\Pi_2$ for sequent
$\Delta_{{\mathrm{1}}}  \SCsym{;}  \Gamma_{{\mathrm{1}}}  \SCsym{;}  \SCnt{X}  \SCsym{;}  \Gamma_{{\mathrm{2}}}  \SCsym{;}  \Delta_{{\mathrm{2}}}  \vdash_\mathcal{L}  \SCnt{B}$ s.t. $c(\Pi')\leq |A|$. Therefore, the
proof $\Pi$ can be constructed as follows with $c(\Pi)\leq |A|$.
\begin{center}
  \scriptsize
  \begin{math}
    $$\mprset{flushleft}
    \inferrule* [right={\tiny FL}] {
      $$\mprset{flushleft}
      \inferrule* [right={\tiny cut2}] {
        {
          \begin{array}{cc}
            \pi_2 & \Pi_2 \\
            {\Gamma_{{\mathrm{1}}}  \SCsym{;}  \SCnt{X}  \SCsym{;}  \Gamma_{{\mathrm{2}}}  \vdash_\mathcal{L}  \SCnt{A}} & {\Delta_{{\mathrm{1}}}  \SCsym{;}  \SCnt{A}  \SCsym{;}  \Delta_{{\mathrm{2}}}  \vdash_\mathcal{L}  \SCnt{B}}
          \end{array}
        }
      }{\Delta_{{\mathrm{1}}}  \SCsym{;}  \Gamma_{{\mathrm{1}}}  \SCsym{;}  \SCnt{X}  \SCsym{;}  \Gamma_{{\mathrm{2}}}  \SCsym{;}  \Delta_{{\mathrm{2}}}  \vdash_\mathcal{L}  \SCnt{B}}
    }{\Delta_{{\mathrm{1}}}  \SCsym{;}  \Gamma_{{\mathrm{1}}}  \SCsym{;}   \mathsf{F} \SCnt{X}   \SCsym{;}  \Gamma_{{\mathrm{2}}}  \SCsym{;}  \Delta_{{\mathrm{2}}}  \vdash_\mathcal{L}  \SCnt{B}}
  \end{math}
\end{center}

\subsubsection{Left introduction of the functor $G$}
\begin{center}
  \scriptsize
  $\Pi_1$:
  \begin{math}
    $$\mprset{flushleft}
    \inferrule* [right={\tiny GL}] {
      {
        \begin{array}{c}
          \pi_1 \\
          {\Gamma_{{\mathrm{1}}}  \SCsym{;}  \SCnt{A}  \SCsym{;}  \Gamma_{{\mathrm{2}}}  \vdash_\mathcal{L}  \SCnt{B}}
        \end{array}
      }
    }{\Gamma_{{\mathrm{1}}}  \SCsym{;}   \mathsf{G} \SCnt{A}   \SCsym{;}  \Gamma_{{\mathrm{2}}}  \vdash_\mathcal{L}  \SCnt{B}}
  \end{math}
  \qquad\qquad
  \begin{math}
    \begin{array}{c}
      \Pi_2 \\
      {\Delta_{{\mathrm{1}}}  \SCsym{;}  \SCnt{B}  \SCsym{;}  \Delta_{{\mathrm{2}}}  \vdash_\mathcal{L}  \SCnt{C}}
    \end{array}
  \end{math}
\end{center}
By assumption, $c(\Pi_1),c(\Pi_2)\leq |B|$. By induction, there is a
proof $\Pi'$ from $\pi_2$ and $\Pi_2$ for sequent
$\Delta_{{\mathrm{1}}}  \SCsym{;}  \Gamma_{{\mathrm{1}}}  \SCsym{;}  \SCnt{A}  \SCsym{;}  \Gamma_{{\mathrm{2}}}  \SCsym{;}  \Delta_{{\mathrm{2}}}  \vdash_\mathcal{L}  \SCnt{C}$ s.t. $c(\Pi')\leq |B|$. Therefore, the
proof $\Pi$ can be constructed as follows with $c(\Pi)\leq |B|$.
\begin{center}
  \scriptsize
  \begin{math}
    $$\mprset{flushleft}
    \inferrule* [right={\tiny GL}] {
      $$\mprset{flushleft}
      \inferrule* [right={\tiny cut2}] {
        {
          \begin{array}{cc}
            \pi_2 & \Pi_2 \\
            {\Gamma_{{\mathrm{1}}}  \SCsym{;}  \SCnt{A}  \SCsym{;}  \Gamma_{{\mathrm{2}}}  \vdash_\mathcal{L}  \SCnt{B}} & {\Delta_{{\mathrm{1}}}  \SCsym{;}  \SCnt{B}  \SCsym{;}  \Delta_{{\mathrm{2}}}  \vdash_\mathcal{L}  \SCnt{C}}
          \end{array}
        }
      }{\Delta_{{\mathrm{1}}}  \SCsym{;}  \Gamma_{{\mathrm{1}}}  \SCsym{;}  \SCnt{A}  \SCsym{;}  \Gamma_{{\mathrm{2}}}  \SCsym{;}  \Delta_{{\mathrm{2}}}  \vdash_\mathcal{L}  \SCnt{C}}
    }{\Delta_{{\mathrm{1}}}  \SCsym{;}  \Gamma_{{\mathrm{1}}}  \SCsym{;}   \mathsf{G} \SCnt{A}   \SCsym{;}  \Gamma_{{\mathrm{2}}}  \SCsym{;}  \Delta_{{\mathrm{2}}}  \vdash_\mathcal{L}  \SCnt{C}}
  \end{math}
\end{center}



\subsection{Secondary Hypothesis}

\subsubsection{Right introduction of the commutative tensor product $\otimes$}
\begin{itemize}
\item Case 1:
      \begin{center}
        \scriptsize
        \begin{math}
          \begin{array}{c}
            \Pi_1 \\
            {\Phi_{{\mathrm{2}}}  \vdash_\mathcal{C}  \SCnt{X}}
          \end{array}
        \end{math}
        \qquad\qquad
        $\Pi_2$:
        \begin{math}
          $$\mprset{flushleft}
          \inferrule* [right={\tiny tenR}] {
            {
              \begin{array}{cc}
                \pi_1 & \pi_2 \\
                {\Psi_{{\mathrm{1}}}  \SCsym{,}  \SCnt{X}  \SCsym{,}  \Psi_{{\mathrm{2}}}  \vdash_\mathcal{C}  \SCnt{Y_{{\mathrm{1}}}}} & {\Phi_{{\mathrm{1}}}  \vdash_\mathcal{C}  \SCnt{Y_{{\mathrm{2}}}}}
              \end{array}
            }
          }{\Psi_{{\mathrm{1}}}  \SCsym{,}  \SCnt{X}  \SCsym{,}  \Psi_{{\mathrm{2}}}  \SCsym{,}  \Phi_{{\mathrm{1}}}  \vdash_\mathcal{C}  \SCnt{Y_{{\mathrm{1}}}}  \otimes  \SCnt{Y_{{\mathrm{2}}}}}
        \end{math}
      \end{center}
      By assumption, $c(\Pi_1),c(\Pi_2)\leq |X|$. By induction on $\Pi_1$
      and $\pi_1$, there is a proof $\Pi'$ for sequent
      $\Psi_{{\mathrm{1}}}  \SCsym{,}  \Phi_{{\mathrm{2}}}  \SCsym{,}  \Psi_{{\mathrm{2}}}  \vdash_\mathcal{C}  \SCnt{Y_{{\mathrm{1}}}}$ s.t. $c(\Pi') \leq |X|$. Therefore, the proof
      $\Pi$ can be constructed as follows with $c(\Pi) = c(\Pi') \leq |X|$.
      \begin{center}
        \scriptsize
        \begin{math}
          $$\mprset{flushleft}
          \inferrule* [right={\tiny tenR}] {
            {
              \begin{array}{cc}
                \Pi' & \pi_1 \\
                {\Psi_{{\mathrm{1}}}  \SCsym{,}  \Phi_{{\mathrm{2}}}  \SCsym{,}  \Psi_{{\mathrm{2}}}  \vdash_\mathcal{C}  \SCnt{Y_{{\mathrm{1}}}}} & {\Phi_{{\mathrm{1}}}  \vdash_\mathcal{C}  \SCnt{Y_{{\mathrm{2}}}}}
              \end{array}
            }
          }{\Psi_{{\mathrm{1}}}  \SCsym{,}  \Phi_{{\mathrm{2}}}  \SCsym{,}  \Psi_{{\mathrm{2}}}  \SCsym{,}  \Phi_{{\mathrm{1}}}  \vdash_\mathcal{C}  \SCnt{Y_{{\mathrm{1}}}}  \otimes  \SCnt{Y_{{\mathrm{2}}}}}
        \end{math}
      \end{center}

\item Case 2:
      \begin{center}
        \scriptsize
        \begin{math}
          \begin{array}{c}
            \Pi_1 \\
            {\Phi_{{\mathrm{2}}}  \vdash_\mathcal{C}  \SCnt{X}}
          \end{array}
        \end{math}
        \qquad\qquad
        $\Pi_2$:
        \begin{math}
          $$\mprset{flushleft}
          \inferrule* [right={\tiny tenR}] {
            {
              \begin{array}{cc}
                \pi_1 & \pi_2 \\
                {\Phi_{{\mathrm{1}}}  \vdash_\mathcal{C}  \SCnt{Y_{{\mathrm{1}}}}} & {\Psi_{{\mathrm{1}}}  \SCsym{,}  \SCnt{X}  \SCsym{,}  \Psi_{{\mathrm{2}}}  \vdash_\mathcal{C}  \SCnt{Y_{{\mathrm{2}}}}}
              \end{array}
            }
          }{\Phi_{{\mathrm{1}}}  \SCsym{,}  \Psi_{{\mathrm{1}}}  \SCsym{,}  \SCnt{X}  \SCsym{,}  \Psi_{{\mathrm{2}}}  \vdash_\mathcal{C}  \SCnt{Y_{{\mathrm{1}}}}  \otimes  \SCnt{Y_{{\mathrm{2}}}}}
        \end{math}
      \end{center}
      By assumption, $c(\Pi_1),c(\Pi_2)\leq |X|$. By induction on $\Pi_1$
      and $\pi_2$, there is a proof $\Pi'$ for sequent
      $\Psi_{{\mathrm{1}}}  \SCsym{,}  \Phi_{{\mathrm{2}}}  \SCsym{,}  \Psi_{{\mathrm{2}}}  \vdash_\mathcal{C}  \SCnt{Y_{{\mathrm{2}}}}$ s.t. $c(\Pi') \leq |X|$. Therefore, the proof
      $\Pi$ can be constructed as follows with $c(\Pi) = c(\Pi') \leq |X|$.
      \begin{center}
        \scriptsize
        \begin{math}
          $$\mprset{flushleft}
          \inferrule* [right={\tiny tenR}] {
            {
              \begin{array}{cc}
                \pi_1 & \Pi' \\
                {\Phi_{{\mathrm{1}}}  \vdash_\mathcal{C}  \SCnt{Y_{{\mathrm{1}}}}} & {\Psi_{{\mathrm{1}}}  \SCsym{,}  \Phi_{{\mathrm{2}}}  \SCsym{,}  \Psi_{{\mathrm{2}}}  \vdash_\mathcal{C}  \SCnt{Y_{{\mathrm{2}}}}}
              \end{array}
            }
          }{\Phi_{{\mathrm{1}}}  \SCsym{,}  \Psi_{{\mathrm{1}}}  \SCsym{,}  \Phi_{{\mathrm{2}}}  \SCsym{,}  \Psi_{{\mathrm{2}}}  \vdash_\mathcal{C}  \SCnt{Y_{{\mathrm{1}}}}  \otimes  \SCnt{Y_{{\mathrm{2}}}}}
        \end{math}
      \end{center}
\end{itemize}



\subsubsection{Right introduction of the non-commutative tensor product $\tri$}
\begin{itemize}
\item Case 1:
      \begin{center}
        \scriptsize
        \begin{math}
          \begin{array}{c}
            \Pi_1 \\
            {\Phi  \vdash_\mathcal{C}  \SCnt{X}}
          \end{array}
        \end{math}
        \qquad\qquad
        $\Pi_2$:
        \begin{math}
          $$\mprset{flushleft}
          \inferrule* [right={\tiny tenR}] {
            {
              \begin{array}{cc}
                \pi_1 & \pi_2 \\
                {\Gamma_{{\mathrm{1}}}  \SCsym{;}  \SCnt{X}  \SCsym{;}  \Gamma_{{\mathrm{2}}}  \vdash_\mathcal{L}  \SCnt{A}} & {\Gamma_{{\mathrm{3}}}  \vdash_\mathcal{L}  \SCnt{B}}
              \end{array}
            }
          }{\Gamma_{{\mathrm{1}}}  \SCsym{;}  \SCnt{X}  \SCsym{;}  \Gamma_{{\mathrm{2}}}  \SCsym{;}  \Gamma_{{\mathrm{3}}}  \vdash_\mathcal{L}  \SCnt{A}  \triangleright  \SCnt{B}}
        \end{math}
      \end{center}
      By assumption, $c(\Pi_1),c(\Pi_2)\leq |X|$. By induction on $\Pi_1$
      and $\pi_1$, there is a proof $\Pi'$ for sequent
      $\Gamma_{{\mathrm{1}}}  \SCsym{;}  \Phi  \SCsym{;}  \Gamma_{{\mathrm{2}}}  \vdash_\mathcal{L}  \SCnt{A}$ s.t. $c(\Pi') \leq |X|$. Therefore, the proof
      $\Pi$ can be constructed as follows with $c(\Pi) = c(\Pi') \leq |X|$.
      \begin{center}
        \scriptsize
        \begin{math}
          $$\mprset{flushleft}
          \inferrule* [right={\tiny tenR}] {
            {
              \begin{array}{cc}
                \Pi' & \pi_1 \\
                {\Gamma_{{\mathrm{1}}}  \SCsym{;}  \Phi  \SCsym{;}  \Gamma_{{\mathrm{2}}}  \vdash_\mathcal{L}  \SCnt{A}} & {\Gamma_{{\mathrm{3}}}  \vdash_\mathcal{L}  \SCnt{B}}
              \end{array}
            }
          }{\Gamma_{{\mathrm{1}}}  \SCsym{;}  \Phi  \SCsym{;}  \Gamma_{{\mathrm{2}}}  \SCsym{;}  \Gamma_{{\mathrm{3}}}  \vdash_\mathcal{L}  \SCnt{A}  \triangleright  \SCnt{B}}
        \end{math}
      \end{center}

\item Case 2:
      \begin{center}
        \scriptsize
        \begin{math}
          \begin{array}{c}
            \Pi_1 \\
            {\Delta  \vdash_\mathcal{L}  \SCnt{C}}
          \end{array}
        \end{math}
        \qquad\qquad
        $\Pi_2$:
        \begin{math}
          $$\mprset{flushleft}
          \inferrule* [right={\tiny tenR}] {
            {
              \begin{array}{cc}
                \pi_1 & \pi_2 \\
                {\Gamma_{{\mathrm{1}}}  \SCsym{;}  \SCnt{C}  \SCsym{;}  \Gamma_{{\mathrm{2}}}  \vdash_\mathcal{L}  \SCnt{A}} & {\Gamma_{{\mathrm{3}}}  \vdash_\mathcal{L}  \SCnt{B}}
              \end{array}
            }
          }{\Gamma_{{\mathrm{1}}}  \SCsym{;}  \SCnt{C}  \SCsym{;}  \Gamma_{{\mathrm{2}}}  \SCsym{;}  \Gamma_{{\mathrm{3}}}  \vdash_\mathcal{L}  \SCnt{A}  \triangleright  \SCnt{B}}
        \end{math}
      \end{center}
      By assumption, $c(\Pi_1),c(\Pi_2)\leq |C|$. By induction on $\Pi_1$
      and $\pi_1$, there is a proof $\Pi'$ for sequent
      $\Gamma_{{\mathrm{1}}}  \SCsym{;}  \Delta  \SCsym{;}  \Gamma_{{\mathrm{2}}}  \vdash_\mathcal{L}  \SCnt{A}$ s.t. $c(\Pi') \leq |C|$. Therefore, the proof
      $\Pi$ can be constructed as follows with $c(\Pi) = c(\Pi') \leq |C|$.
      \begin{center}
        \scriptsize
        \begin{math}
          $$\mprset{flushleft}
          \inferrule* [right={\tiny tenR}] {
            {
              \begin{array}{cc}
                \Pi' & \pi_1 \\
                {\Gamma_{{\mathrm{1}}}  \SCsym{;}  \Delta  \SCsym{;}  \Gamma_{{\mathrm{2}}}  \vdash_\mathcal{L}  \SCnt{A}} & {\Gamma_{{\mathrm{3}}}  \vdash_\mathcal{L}  \SCnt{B}}
              \end{array}
            }
          }{\Gamma_{{\mathrm{1}}}  \SCsym{;}  \Delta  \SCsym{;}  \Gamma_{{\mathrm{2}}}  \SCsym{;}  \Gamma_{{\mathrm{3}}}  \vdash_\mathcal{L}  \SCnt{A}  \triangleright  \SCnt{B}}
        \end{math}
      \end{center}

\item Case 3:
      \begin{center}
        \scriptsize
        \begin{math}
          \begin{array}{c}
            \Pi_1 \\
            {\Phi  \vdash_\mathcal{C}  \SCnt{X}}
          \end{array}
        \end{math}
        \qquad\qquad
        $\Pi_2$:
        \begin{math}
          $$\mprset{flushleft}
          \inferrule* [right={\tiny tenR}] {
            {
              \begin{array}{cc}
                \pi_1 & \pi_2 \\
                {\Gamma_{{\mathrm{1}}}  \vdash_\mathcal{L}  \SCnt{A}} & {\Gamma_{{\mathrm{2}}}  \SCsym{;}  \SCnt{X}  \SCsym{;}  \Gamma_{{\mathrm{3}}}  \vdash_\mathcal{L}  \SCnt{B}}
              \end{array}
            }
          }{\Gamma_{{\mathrm{1}}}  \SCsym{;}  \Gamma_{{\mathrm{2}}}  \SCsym{;}  \SCnt{X}  \SCsym{;}  \Gamma_{{\mathrm{3}}}  \vdash_\mathcal{L}  \SCnt{A}  \triangleright  \SCnt{B}}
        \end{math}
      \end{center}
      By assumption, $c(\Pi_1),c(\Pi_2)\leq |X|$. By induction on $\Pi_1$
      and $\pi_2$, there is a proof $\Pi'$ for sequent
      $\Gamma_{{\mathrm{2}}}  \SCsym{;}  \Phi  \SCsym{;}  \Gamma_{{\mathrm{3}}}  \vdash_\mathcal{L}  \SCnt{B}$ s.t. $c(\Pi') \leq |X|$. Therefore, the proof
      $\Pi$ can be constructed as follows with $c(\Pi) = c(\Pi') \leq |X|$.
      \begin{center}
        \scriptsize
        \begin{math}
          $$\mprset{flushleft}
          \inferrule* [right={\tiny tenR}] {
            {
              \begin{array}{cc}
                \pi_1 & \Pi' \\
                {\Gamma_{{\mathrm{1}}}  \vdash_\mathcal{L}  \SCnt{A}} & {\Gamma_{{\mathrm{2}}}  \SCsym{;}  \Phi  \SCsym{;}  \Gamma_{{\mathrm{3}}}  \vdash_\mathcal{L}  \SCnt{B}}
              \end{array}
            }
          }{\Gamma_{{\mathrm{1}}}  \SCsym{;}  \Gamma_{{\mathrm{2}}}  \SCsym{;}  \Phi  \SCsym{;}  \Gamma_{{\mathrm{3}}}  \vdash_\mathcal{L}  \SCnt{A}  \triangleright  \SCnt{B}}
        \end{math}
      \end{center}

\item Case 4:
      \begin{center}
        \scriptsize
        \begin{math}
          \begin{array}{c}
            \Pi_1 \\
            {\Delta  \vdash_\mathcal{L}  \SCnt{C}}
          \end{array}
        \end{math}
        \qquad\qquad
        $\Pi_2$:
        \begin{math}
          $$\mprset{flushleft}
          \inferrule* [right={\tiny tenR}] {
            {
              \begin{array}{cc}
                \pi_1 & \pi_2 \\
                {\Gamma_{{\mathrm{1}}}  \vdash_\mathcal{L}  \SCnt{A}} & {\Gamma_{{\mathrm{2}}}  \SCsym{;}  \SCnt{C}  \SCsym{;}  \Gamma_{{\mathrm{3}}}  \vdash_\mathcal{L}  \SCnt{B}}
              \end{array}
            }
          }{\Gamma_{{\mathrm{1}}}  \SCsym{;}  \Gamma_{{\mathrm{2}}}  \SCsym{;}  \SCnt{C}  \SCsym{;}  \Gamma_{{\mathrm{3}}}  \vdash_\mathcal{L}  \SCnt{A}  \triangleright  \SCnt{B}}
        \end{math}
      \end{center}
      By assumption, $c(\Pi_1),c(\Pi_2)\leq |C|$. By induction on $\Pi_1$
      and $\pi_2$, there is a proof $\Pi'$ for sequent
      $\Gamma_{{\mathrm{2}}}  \SCsym{;}  \Delta  \SCsym{;}  \Gamma_{{\mathrm{3}}}  \vdash_\mathcal{L}  \SCnt{B}$ s.t. $c(\Pi') \leq |C|$. Therefore, the proof
      $\Pi$ can be constructed as follows with $c(\Pi) = c(\Pi') \leq |C|$.
      \begin{center}
        \scriptsize
        \begin{math}
          $$\mprset{flushleft}
          \inferrule* [right={\tiny tenR}] {
            {
              \begin{array}{cc}
                \pi_1 & \Pi' \\
                {\Gamma_{{\mathrm{1}}}  \vdash_\mathcal{L}  \SCnt{A}} & {\Gamma_{{\mathrm{2}}}  \SCsym{;}  \Delta  \SCsym{;}  \Gamma_{{\mathrm{3}}}  \vdash_\mathcal{L}  \SCnt{B}}
              \end{array}
            }
          }{\Gamma_{{\mathrm{1}}}  \SCsym{;}  \Gamma_{{\mathrm{2}}}  \SCsym{;}  \Delta  \SCsym{;}  \Gamma_{{\mathrm{3}}}  \vdash_\mathcal{L}  \SCnt{A}  \triangleright  \SCnt{B}}
        \end{math}
      \end{center}
\end{itemize}



\subsubsection{Left introduction of the commutative implication $\multimap$}
\begin{itemize}
\item Case 1:
      \begin{center}
        \scriptsize
        \begin{math}
          \begin{array}{c}
            \Pi_1 \\
            {\Phi  \vdash_\mathcal{C}  \SCnt{X}}
          \end{array}
        \end{math}
        \qquad\qquad
        $\Pi_2$:
        \begin{math}
          $$\mprset{flushleft}
          \inferrule* [right={\tiny impL}] {
            {
              \begin{array}{cc}
                \pi_1 & \pi_2 \\
                {\Psi_{{\mathrm{2}}}  \SCsym{,}  \SCnt{X}  \SCsym{,}  \Psi_{{\mathrm{3}}}  \vdash_\mathcal{C}  \SCnt{Y_{{\mathrm{1}}}}} & {\Psi_{{\mathrm{1}}}  \SCsym{,}  \SCnt{Y_{{\mathrm{2}}}}  \SCsym{,}  \Psi_{{\mathrm{4}}}  \vdash_\mathcal{C}  \SCnt{Z}}
              \end{array}
            }
          }{\Psi_{{\mathrm{1}}}  \SCsym{,}  \SCnt{Y_{{\mathrm{1}}}}  \multimap  \SCnt{Y_{{\mathrm{2}}}}  \SCsym{,}  \Psi_{{\mathrm{2}}}  \SCsym{,}  \SCnt{X}  \SCsym{,}  \Psi_{{\mathrm{3}}}  \SCsym{,}  \Psi_{{\mathrm{4}}}  \vdash_\mathcal{C}  \SCnt{Z}}
        \end{math}
      \end{center}
      By assumption, $c(\Pi_1),c(\Pi_2)\leq |X|$. By induction on $\Pi_1$ and $\pi_1$, there is
      a proof $\Pi'$ for sequent $\Psi_{{\mathrm{2}}}  \SCsym{,}  \Phi  \SCsym{,}  \Psi_{{\mathrm{3}}}  \vdash_\mathcal{C}  \SCnt{Y_{{\mathrm{1}}}}$ s.t. $c(\Pi') \leq |X|$. Therefore, the
      proof $\Pi$ can be constructed as follows with $c(\Pi) = c(\Pi') \leq |X|$.
      \begin{center}
        \scriptsize
        \begin{math}
          $$\mprset{flushleft}
          \inferrule* [right={\tiny impL}] {
            {
              \begin{array}{cc}
                \Pi' & \pi_2 \\
                {\Psi_{{\mathrm{2}}}  \SCsym{,}  \Phi  \SCsym{,}  \Psi_{{\mathrm{3}}}  \vdash_\mathcal{C}  \SCnt{Y_{{\mathrm{1}}}}} & {\Psi_{{\mathrm{1}}}  \SCsym{,}  \SCnt{Y_{{\mathrm{2}}}}  \SCsym{,}  \Psi_{{\mathrm{4}}}  \vdash_\mathcal{C}  \SCnt{Z}}
              \end{array}
            }
          }{\Psi_{{\mathrm{1}}}  \SCsym{,}  \SCnt{Y_{{\mathrm{1}}}}  \multimap  \SCnt{Y_{{\mathrm{2}}}}  \SCsym{,}  \Psi_{{\mathrm{2}}}  \SCsym{,}  \Phi  \SCsym{,}  \Psi_{{\mathrm{3}}}  \SCsym{,}  \Psi_{{\mathrm{4}}}  \vdash_\mathcal{C}  \SCnt{Z}}
        \end{math}
      \end{center}

\item Case 2:
      \begin{center}
        \scriptsize
        \begin{math}
          \begin{array}{c}
            \Pi_1 \\
            {\Phi  \vdash_\mathcal{C}  \SCnt{X}}
          \end{array}
        \end{math}
        \qquad\qquad
        $\Pi_2$:
        \begin{math}
          $$\mprset{flushleft}
          \inferrule* [right={\tiny impL}] {
            {
              \begin{array}{cc}
                \pi_1 & \pi_2 \\
                {\Psi_{{\mathrm{3}}}  \vdash_\mathcal{C}  \SCnt{Y_{{\mathrm{1}}}}} & {\Psi_{{\mathrm{1}}}  \SCsym{,}  \SCnt{X}  \SCsym{,}  \Psi_{{\mathrm{2}}}  \SCsym{,}  \SCnt{Y_{{\mathrm{2}}}}  \SCsym{,}  \Psi_{{\mathrm{4}}}  \vdash_\mathcal{C}  \SCnt{Z}}
              \end{array}
            }
          }{\Psi_{{\mathrm{1}}}  \SCsym{,}  \SCnt{X}  \SCsym{,}  \Psi_{{\mathrm{2}}}  \SCsym{,}  \SCnt{Y_{{\mathrm{1}}}}  \multimap  \SCnt{Y_{{\mathrm{2}}}}  \SCsym{,}  \Psi_{{\mathrm{3}}}  \SCsym{,}  \Psi_{{\mathrm{4}}}  \vdash_\mathcal{C}  \SCnt{Z}}
        \end{math}
      \end{center}
      By assumption, $c(\Pi_1),c(\Pi_2)\leq |X|$. By induction on $\Pi_1$ and $\pi_2$, there is
      a proof $\Pi'$ for sequent $\Psi_{{\mathrm{1}}}  \SCsym{,}  \Phi  \SCsym{,}  \Psi_{{\mathrm{2}}}  \SCsym{,}  \SCnt{Y_{{\mathrm{2}}}}  \SCsym{,}  \Psi_{{\mathrm{4}}}  \vdash_\mathcal{C}  \SCnt{Z}$ s.t. $c(\Pi') \leq |X|$.
      Therefore, the proof $\Pi$ can be constructed as follows with
      $c(\Pi) = c(\Pi') \leq |X|$.
      \begin{center}
        \scriptsize
        \begin{math}
          $$\mprset{flushleft}
          \inferrule* [right={\tiny impL}] {
            {
              \begin{array}{cc}
                \pi_1 & \Pi' \\
                {\Psi_{{\mathrm{3}}}  \vdash_\mathcal{C}  \SCnt{Y_{{\mathrm{1}}}}} & {\Psi_{{\mathrm{1}}}  \SCsym{,}  \Phi  \SCsym{,}  \Psi_{{\mathrm{2}}}  \SCsym{,}  \SCnt{Y_{{\mathrm{2}}}}  \SCsym{,}  \Psi_{{\mathrm{4}}}  \vdash_\mathcal{C}  \SCnt{Z}}
              \end{array}
            }
          }{\Psi_{{\mathrm{1}}}  \SCsym{,}  \Phi_{{\mathrm{1}}}  \SCsym{,}  \Psi_{{\mathrm{2}}}  \SCsym{,}  \SCnt{Y_{{\mathrm{1}}}}  \multimap  \SCnt{Y_{{\mathrm{2}}}}  \SCsym{,}  \Psi_{{\mathrm{3}}}  \SCsym{,}  \Psi_{{\mathrm{4}}}  \vdash_\mathcal{C}  \SCnt{Z}}
        \end{math}
      \end{center}

\item Case 3:
      \begin{center}
        \scriptsize
        \begin{math}
          \begin{array}{c}
            \Pi_1 \\
            {\Phi  \vdash_\mathcal{C}  \SCnt{X}}
          \end{array}
        \end{math}
        \qquad\qquad
        $\Pi_2$:
        \begin{math}
          $$\mprset{flushleft}
          \inferrule* [right={\tiny impL}] {
            {
              \begin{array}{cc}
                \pi_1 & \pi_2 \\
                {\Psi_{{\mathrm{2}}}  \vdash_\mathcal{C}  \SCnt{Y_{{\mathrm{1}}}}} & {\Psi_{{\mathrm{1}}}  \SCsym{,}  \SCnt{Y_{{\mathrm{2}}}}  \SCsym{,}  \Psi_{{\mathrm{3}}}  \SCsym{,}  \SCnt{X}  \SCsym{,}  \Psi_{{\mathrm{4}}}  \vdash_\mathcal{C}  \SCnt{Z}}
              \end{array}
            }
          }{\Psi_{{\mathrm{1}}}  \SCsym{,}  \SCnt{Y_{{\mathrm{1}}}}  \multimap  \SCnt{Y_{{\mathrm{2}}}}  \SCsym{,}  \Psi_{{\mathrm{2}}}  \SCsym{,}  \Psi_{{\mathrm{3}}}  \SCsym{,}  \SCnt{X}  \SCsym{,}  \Psi_{{\mathrm{4}}}  \vdash_\mathcal{C}  \SCnt{Z}}
        \end{math}
      \end{center}
      By assumption, $c(\Pi_1),c(\Pi_2)\leq |X|$. By induction on $\Pi_1$
      and $\pi_2$, there is a proof $\Pi'$ for sequent
      $\Psi_{{\mathrm{1}}}  \SCsym{,}  \Phi  \SCsym{,}  \Psi_{{\mathrm{2}}}  \SCsym{,}  \SCnt{Y_{{\mathrm{2}}}}  \SCsym{,}  \Psi_{{\mathrm{4}}}  \vdash_\mathcal{C}  \SCnt{Z}$ s.t. $c(\Pi') \leq |X|$. Therefore,
      the proof $\Pi$ can be constructed as follows with
      $c(\Pi) = c(\Pi') \leq |X|$.
      \begin{center}
        \scriptsize
        \begin{math}
          $$\mprset{flushleft}
          \inferrule* [right={\tiny impL}] {
            {
              \begin{array}{cc}
                \pi_1 & \Pi' \\
                {\Psi_{{\mathrm{2}}}  \vdash_\mathcal{C}  \SCnt{Y_{{\mathrm{1}}}}} & {\Psi_{{\mathrm{1}}}  \SCsym{,}  \SCnt{Y_{{\mathrm{2}}}}  \SCsym{,}  \Psi_{{\mathrm{3}}}  \SCsym{,}  \Phi  \SCsym{,}  \Psi_{{\mathrm{4}}}  \vdash_\mathcal{C}  \SCnt{Z}}
              \end{array}
            }
          }{\Psi_{{\mathrm{1}}}  \SCsym{,}  \SCnt{Y_{{\mathrm{1}}}}  \multimap  \SCnt{Y_{{\mathrm{2}}}}  \SCsym{,}  \Psi_{{\mathrm{2}}}  \SCsym{,}  \Psi_{{\mathrm{3}}}  \SCsym{,}  \Phi  \SCsym{,}  \Psi_{{\mathrm{4}}}  \vdash_\mathcal{C}  \SCnt{Z}}
        \end{math}
      \end{center}

\item Case 4:
      \begin{center}
        \scriptsize
        \begin{math}
          \begin{array}{c}
            \Pi_1 \\
            {\Phi  \vdash_\mathcal{C}  \SCnt{X}}
          \end{array}
        \end{math}
        \qquad\qquad
        $\Pi_2$:
        \begin{math}
          $$\mprset{flushleft}
          \inferrule* [right={\tiny impL}] {
            {
              \begin{array}{cc}
                \pi_1 & \pi_2 \\
                {\Psi_{{\mathrm{1}}}  \SCsym{,}  \SCnt{X}  \SCsym{,}  \Psi_{{\mathrm{2}}}  \vdash_\mathcal{C}  \SCnt{Y_{{\mathrm{1}}}}} & {\Gamma_{{\mathrm{1}}}  \SCsym{;}  \SCnt{Y_{{\mathrm{2}}}}  \SCsym{;}  \Gamma_{{\mathrm{2}}}  \vdash_\mathcal{L}  \SCnt{A}}
              \end{array}
            }
          }{\Gamma_{{\mathrm{1}}}  \SCsym{;}  \SCnt{Y_{{\mathrm{1}}}}  \multimap  \SCnt{Y_{{\mathrm{2}}}}  \SCsym{;}  \Psi_{{\mathrm{1}}}  \SCsym{;}  \SCnt{X}  \SCsym{;}  \Psi_{{\mathrm{2}}}  \SCsym{;}  \Gamma_{{\mathrm{2}}}  \vdash_\mathcal{L}  \SCnt{A}}
        \end{math}
      \end{center}
      By assumption, $c(\Pi_1),c(\Pi_2)\leq |X|$. By induction on $\Pi_1$
      and $\pi_1$, there is a proof $\Pi'$ for sequent
      $\Psi_{{\mathrm{1}}}  \SCsym{,}  \Phi  \SCsym{,}  \Psi_{{\mathrm{2}}}  \vdash_\mathcal{C}  \SCnt{Y_{{\mathrm{1}}}}$ s.t. $c(\Pi') \leq |X|$. Therefore, the proof
      $\Pi$ can be constructed as follows with $c(\Pi) = c(\Pi') \leq |X|$.
      \begin{center}
        \scriptsize
        \begin{math}
          $$\mprset{flushleft}
          \inferrule* [right={\tiny impL}] {
            {
              \begin{array}{cc}
                \Pi' & \pi_2 \\
                {\Psi_{{\mathrm{1}}}  \SCsym{,}  \Phi  \SCsym{,}  \Psi_{{\mathrm{2}}}  \vdash_\mathcal{C}  \SCnt{Y_{{\mathrm{1}}}}} & {\Gamma_{{\mathrm{1}}}  \SCsym{;}  \SCnt{Y_{{\mathrm{2}}}}  \SCsym{;}  \Gamma_{{\mathrm{2}}}  \vdash_\mathcal{L}  \SCnt{A}}
              \end{array}
            }
          }{\Gamma_{{\mathrm{1}}}  \SCsym{;}  \SCnt{Y_{{\mathrm{1}}}}  \multimap  \SCnt{Y_{{\mathrm{2}}}}  \SCsym{;}  \Psi_{{\mathrm{1}}}  \SCsym{;}  \Phi  \SCsym{;}  \Psi_{{\mathrm{2}}}  \SCsym{;}  \Gamma_{{\mathrm{2}}}  \vdash_\mathcal{L}  \SCnt{A}}
        \end{math}
      \end{center}

\item Case 5:
      \begin{center}
        \scriptsize
        \begin{math}
          \begin{array}{c}
            \Pi_1 \\
            {\Phi  \vdash_\mathcal{C}  \SCnt{X}}
          \end{array}
        \end{math}
        \qquad\qquad
        $\Pi_2$:
        \begin{math}
          $$\mprset{flushleft}
          \inferrule* [right={\tiny impL}] {
            {
              \begin{array}{cc}
                \pi_1 & \pi_2 \\
                {\Psi  \vdash_\mathcal{C}  \SCnt{Y_{{\mathrm{1}}}}} & {\Gamma_{{\mathrm{1}}}  \SCsym{;}  \SCnt{X}  \SCsym{;}  \Gamma_{{\mathrm{2}}}  \SCsym{;}  \SCnt{Y_{{\mathrm{2}}}}  \SCsym{;}  \Gamma_{{\mathrm{3}}}  \vdash_\mathcal{L}  \SCnt{A}}
              \end{array}
            }
          }{\Gamma_{{\mathrm{1}}}  \SCsym{;}  \SCnt{X}  \SCsym{;}  \Gamma_{{\mathrm{2}}}  \SCsym{;}  \SCnt{Y_{{\mathrm{1}}}}  \multimap  \SCnt{Y_{{\mathrm{2}}}}  \SCsym{;}  \Psi  \SCsym{;}  \Gamma_{{\mathrm{3}}}  \vdash_\mathcal{L}  \SCnt{A}}
        \end{math}
      \end{center}
      By assumption, $c(\Pi_1),c(\Pi_2)\leq |X|$. By induction on $\Pi_1$
      and $\pi_2$, there is a proof $\Pi'$ for sequent
      $\Gamma_{{\mathrm{1}}}  \SCsym{;}  \Phi  \SCsym{;}  \Gamma_{{\mathrm{2}}}  \SCsym{;}  \SCnt{Y_{{\mathrm{2}}}}  \SCsym{;}  \Gamma_{{\mathrm{3}}}  \vdash_\mathcal{L}  \SCnt{A}$ s.t. $c(\Pi') \leq |X|$. Therefore, the
      proof $\Pi$ can be constructed as follows with
      $c(\Pi) = c(\Pi') \leq |X|$.
      \begin{center}
        \scriptsize
        \begin{math}
          $$\mprset{flushleft}
          \inferrule* [right={\tiny impL}] {
            {
              \begin{array}{cc}
                \pi_1 & \Pi' \\
                {\Psi  \vdash_\mathcal{C}  \SCnt{Y_{{\mathrm{1}}}}} & {\Gamma_{{\mathrm{1}}}  \SCsym{;}  \Phi  \SCsym{;}  \Gamma_{{\mathrm{2}}}  \SCsym{;}  \SCnt{Y_{{\mathrm{2}}}}  \SCsym{;}  \Gamma_{{\mathrm{3}}}  \vdash_\mathcal{L}  \SCnt{A}}
              \end{array}
            }
          }{\Gamma_{{\mathrm{1}}}  \SCsym{;}  \Phi  \SCsym{;}  \Gamma_{{\mathrm{2}}}  \SCsym{;}  \SCnt{Y_{{\mathrm{1}}}}  \multimap  \SCnt{Y_{{\mathrm{2}}}}  \SCsym{;}  \Psi  \SCsym{;}  \Gamma_{{\mathrm{3}}}  \vdash_\mathcal{L}  \SCnt{A}}
        \end{math}
      \end{center}

\item Case 6:
      \begin{center}
        \scriptsize
        \begin{math}
          \begin{array}{c}
            \Pi_1 \\
            {\Delta  \vdash_\mathcal{L}  \SCnt{B}}
          \end{array}
        \end{math}
        \qquad\qquad
        $\Pi_2$:
        \begin{math}
          $$\mprset{flushleft}
          \inferrule* [right={\tiny impL}] {
            {
              \begin{array}{cc}
                \pi_1 & \pi_2 \\
                {\Psi  \vdash_\mathcal{C}  \SCnt{Y_{{\mathrm{1}}}}} & {\Gamma_{{\mathrm{1}}}  \SCsym{;}  \SCnt{B}  \SCsym{;}  \Gamma_{{\mathrm{2}}}  \SCsym{;}  \SCnt{Y_{{\mathrm{2}}}}  \SCsym{;}  \Gamma_{{\mathrm{3}}}  \vdash_\mathcal{L}  \SCnt{A}}
              \end{array}
            }
          }{\Gamma_{{\mathrm{1}}}  \SCsym{;}  \SCnt{B}  \SCsym{;}  \Gamma_{{\mathrm{2}}}  \SCsym{;}  \SCnt{Y_{{\mathrm{1}}}}  \multimap  \SCnt{Y_{{\mathrm{2}}}}  \SCsym{;}  \Psi  \SCsym{;}  \Gamma_{{\mathrm{3}}}  \vdash_\mathcal{L}  \SCnt{A}}
        \end{math}
      \end{center}
      By assumption, $c(\Pi_1),c(\Pi_2)\leq |B|$. By induction on $\Pi_1$
      and $\pi_2$, there is a proof $\Pi'$ for sequent
      $\Gamma_{{\mathrm{1}}}  \SCsym{;}  \Delta  \SCsym{;}  \Gamma_{{\mathrm{2}}}  \SCsym{;}  \SCnt{Y_{{\mathrm{2}}}}  \SCsym{;}  \Gamma_{{\mathrm{3}}}  \vdash_\mathcal{L}  \SCnt{A}$ s.t. $c(\Pi') \leq |B|$. Therefore, the
      proof $\Pi$ can be constructed as follows with
      $c(\Pi) = c(\Pi') \leq |B|$.
      \begin{center}
        \scriptsize
        \begin{math}
          $$\mprset{flushleft}
          \inferrule* [right={\tiny impL}] {
            {
              \begin{array}{cc}
                \pi_1 & \Pi' \\
                {\Psi  \vdash_\mathcal{C}  \SCnt{Y_{{\mathrm{1}}}}} & {\Gamma_{{\mathrm{1}}}  \SCsym{;}  \Delta  \SCsym{;}  \Gamma_{{\mathrm{2}}}  \SCsym{;}  \SCnt{Y_{{\mathrm{2}}}}  \SCsym{;}  \Gamma_{{\mathrm{3}}}  \vdash_\mathcal{L}  \SCnt{A}}
              \end{array}
            }
          }{\Gamma_{{\mathrm{1}}}  \SCsym{;}  \Delta  \SCsym{;}  \Gamma_{{\mathrm{2}}}  \SCsym{;}  \SCnt{Y_{{\mathrm{1}}}}  \multimap  \SCnt{Y_{{\mathrm{2}}}}  \SCsym{;}  \Psi  \SCsym{;}  \Gamma_{{\mathrm{3}}}  \vdash_\mathcal{L}  \SCnt{A}}
        \end{math}
      \end{center}

\item Case 7:
      \begin{center}
        \scriptsize
        \begin{math}
          \begin{array}{c}
            \Pi_1 \\
            {\Phi  \vdash_\mathcal{C}  \SCnt{X}}
          \end{array}
        \end{math}
        \qquad\qquad
        $\Pi_2$:
        \begin{math}
          $$\mprset{flushleft}
          \inferrule* [right={\tiny impL}] {
            {
              \begin{array}{cc}
                \pi_1 & \pi_2 \\
                {\Psi  \vdash_\mathcal{C}  \SCnt{Y_{{\mathrm{1}}}}} & {\Gamma_{{\mathrm{1}}}  \SCsym{;}  \SCnt{Y_{{\mathrm{2}}}}  \SCsym{;}  \Gamma_{{\mathrm{2}}}  \SCsym{;}  \SCnt{X}  \SCsym{;}  \Gamma_{{\mathrm{3}}}  \vdash_\mathcal{L}  \SCnt{A}}
              \end{array}
            }
          }{\Gamma_{{\mathrm{1}}}  \SCsym{;}  \SCnt{Y_{{\mathrm{1}}}}  \multimap  \SCnt{Y_{{\mathrm{2}}}}  \SCsym{;}  \Psi  \SCsym{;}  \Gamma_{{\mathrm{2}}}  \SCsym{;}  \SCnt{X}  \SCsym{;}  \Gamma_{{\mathrm{3}}}  \vdash_\mathcal{L}  \SCnt{A}}
        \end{math}
      \end{center}
      By assumption, $c(\Pi_1),c(\Pi_2)\leq |X|$. By induction on $\Pi_1$
      and $\pi_2$, there is a proof $\Pi'$ for sequent
      $\Gamma_{{\mathrm{1}}}  \SCsym{;}  \SCnt{Y_{{\mathrm{2}}}}  \SCsym{;}  \Gamma_{{\mathrm{2}}}  \SCsym{;}  \Phi  \SCsym{;}  \Gamma_{{\mathrm{3}}}  \vdash_\mathcal{L}  \SCnt{A}$ s.t. $c(\Pi') \leq |X|$. Therefore, the
      proof $\Pi$ can be constructed as follows with
      $c(\Pi) = c(\Pi') \leq |X|$.
      \begin{center}
        \scriptsize
        \begin{math}
          $$\mprset{flushleft}
          \inferrule* [right={\tiny impL}] {
            {
              \begin{array}{cc}
                \pi_1 & \Pi' \\
                {\Psi  \vdash_\mathcal{C}  \SCnt{Y_{{\mathrm{1}}}}} & {\Gamma_{{\mathrm{1}}}  \SCsym{;}  \SCnt{Y_{{\mathrm{2}}}}  \SCsym{;}  \Gamma_{{\mathrm{2}}}  \SCsym{;}  \Phi  \SCsym{;}  \Gamma_{{\mathrm{3}}}  \vdash_\mathcal{L}  \SCnt{A}}
              \end{array}
            }
          }{\Gamma_{{\mathrm{1}}}  \SCsym{;}  \SCnt{Y_{{\mathrm{1}}}}  \multimap  \SCnt{Y_{{\mathrm{2}}}}  \SCsym{;}  \Psi  \SCsym{;}  \Gamma_{{\mathrm{2}}}  \SCsym{;}  \Phi  \SCsym{;}  \Gamma_{{\mathrm{3}}}  \vdash_\mathcal{L}  \SCnt{A}}
        \end{math}
      \end{center}

\item Case 8:
      \begin{center}
        \scriptsize
        \begin{math}
          \begin{array}{c}
            \Pi_1 \\
            {\Delta  \vdash_\mathcal{L}  \SCnt{B}}
          \end{array}
        \end{math}
        \qquad\qquad
        $\Pi_2$:
        \begin{math}
          $$\mprset{flushleft}
          \inferrule* [right={\tiny impL}] {
            {
              \begin{array}{cc}
                \pi_1 & \pi_2 \\
                {\Psi  \vdash_\mathcal{C}  \SCnt{Y_{{\mathrm{1}}}}} & {\Gamma_{{\mathrm{1}}}  \SCsym{;}  \SCnt{Y_{{\mathrm{2}}}}  \SCsym{;}  \Gamma_{{\mathrm{2}}}  \SCsym{;}  \SCnt{B}  \SCsym{;}  \Gamma_{{\mathrm{3}}}  \vdash_\mathcal{L}  \SCnt{A}}
              \end{array}
            }
          }{\Gamma_{{\mathrm{1}}}  \SCsym{;}  \SCnt{Y_{{\mathrm{1}}}}  \multimap  \SCnt{Y_{{\mathrm{2}}}}  \SCsym{;}  \Psi  \SCsym{;}  \Gamma_{{\mathrm{2}}}  \SCsym{;}  \SCnt{B}  \SCsym{;}  \Gamma_{{\mathrm{3}}}  \vdash_\mathcal{L}  \SCnt{A}}
        \end{math}
      \end{center}
      By assumption, $c(\Pi_1),c(\Pi_2)\leq |B|$. By induction on $\Pi_1$
      and $\pi_2$, there is a proof $\Pi'$ for sequent
      $\Gamma_{{\mathrm{1}}}  \SCsym{;}  \SCnt{Y_{{\mathrm{2}}}}  \SCsym{;}  \Gamma_{{\mathrm{2}}}  \SCsym{;}  \Delta  \SCsym{;}  \Gamma_{{\mathrm{3}}}  \vdash_\mathcal{L}  \SCnt{A}$ s.t. $c(\Pi') \leq |B|$. Therefore,
      the proof $\Pi$ can be constructed as follows with
      $c(\Pi) = c(\Pi') \leq |B|$.
      \begin{center}
        \scriptsize
        \begin{math}
          $$\mprset{flushleft}
          \inferrule* [right={\tiny impL}] {
            {
              \begin{array}{cc}
                \pi_1 & \Pi' \\
                {\Psi  \vdash_\mathcal{C}  \SCnt{Y_{{\mathrm{1}}}}} & {\Gamma_{{\mathrm{1}}}  \SCsym{;}  \SCnt{Y_{{\mathrm{2}}}}  \SCsym{;}  \Gamma_{{\mathrm{2}}}  \SCsym{;}  \Delta  \SCsym{;}  \Gamma_{{\mathrm{3}}}  \vdash_\mathcal{L}  \SCnt{A}}
              \end{array}
            }
          }{\Gamma_{{\mathrm{1}}}  \SCsym{;}  \SCnt{Y_{{\mathrm{1}}}}  \multimap  \SCnt{Y_{{\mathrm{2}}}}  \SCsym{;}  \Psi  \SCsym{;}  \Gamma_{{\mathrm{2}}}  \SCsym{;}  \Delta  \SCsym{;}  \Gamma_{{\mathrm{3}}}  \vdash_\mathcal{L}  \SCnt{A}}
        \end{math}
      \end{center}
\end{itemize}



\subsubsection{Left introduction of the non-commutative left implication $\lto$}
\begin{itemize}
\item Case 1:
      \begin{center}
        \scriptsize
        \begin{math}
          \begin{array}{c}
            \Pi_1 \\
            {\Phi  \vdash_\mathcal{C}  \SCnt{X}}
          \end{array}
        \end{math}
        \qquad\qquad
        $\Pi_2$:
        \begin{math}
          $$\mprset{flushleft}
          \inferrule* [right={\tiny imprL}] {
            {
              \begin{array}{cc}
                \pi_1 & \pi_2 \\
                {\Delta_{{\mathrm{1}}}  \SCsym{;}  \SCnt{X}  \SCsym{;}  \Delta_{{\mathrm{2}}}  \vdash_\mathcal{L}  \SCnt{A_{{\mathrm{1}}}}} & {\Gamma_{{\mathrm{1}}}  \SCsym{;}  \SCnt{A_{{\mathrm{2}}}}  \SCsym{;}  \Gamma_{{\mathrm{2}}}  \vdash_\mathcal{L}  \SCnt{B}}
              \end{array}
            }
          }{\Gamma_{{\mathrm{1}}}  \SCsym{;}  \SCnt{A_{{\mathrm{1}}}}  \rightharpoonup  \SCnt{A_{{\mathrm{2}}}}  \SCsym{;}  \Delta_{{\mathrm{1}}}  \SCsym{;}  \SCnt{X}  \SCsym{;}  \Delta_{{\mathrm{2}}}  \SCsym{;}  \Gamma_{{\mathrm{2}}}  \vdash_\mathcal{L}  \SCnt{B}}
        \end{math}
      \end{center}
      By assumption, $c(\Pi_1),c(\Pi_2)\leq |X|$. By induction on $\Pi_1$
      and $\pi_1$, there is a proof $\Pi'$ for sequent
      $\Delta_{{\mathrm{1}}}  \SCsym{;}  \Phi  \SCsym{;}  \Delta_{{\mathrm{2}}}  \vdash_\mathcal{L}  \SCnt{A_{{\mathrm{1}}}}$ s.t. $c(\Pi') \leq |X|$. Therefore, the proof
      $\Pi$ can be constructed as follows with $c(\Pi) = c(\Pi') \leq |X|$.
      \begin{center}
        \scriptsize
        \begin{math}
          $$\mprset{flushleft}
          \inferrule* [right={\tiny impL}] {
            {
              \begin{array}{cc}
                \Pi' & \pi_2 \\
                {\Delta_{{\mathrm{1}}}  \SCsym{;}  \Phi  \SCsym{;}  \Delta_{{\mathrm{2}}}  \vdash_\mathcal{L}  \SCnt{A_{{\mathrm{1}}}}} & {\Gamma_{{\mathrm{1}}}  \SCsym{;}  \SCnt{A_{{\mathrm{2}}}}  \SCsym{;}  \Gamma_{{\mathrm{2}}}  \vdash_\mathcal{L}  \SCnt{B}}
              \end{array}
            }
          }{\Gamma_{{\mathrm{1}}}  \SCsym{;}  \SCnt{A_{{\mathrm{1}}}}  \rightharpoonup  \SCnt{A_{{\mathrm{2}}}}  \SCsym{;}  \Delta_{{\mathrm{1}}}  \SCsym{;}  \Phi  \SCsym{;}  \Delta_{{\mathrm{2}}}  \SCsym{;}  \Gamma_{{\mathrm{2}}}  \vdash_\mathcal{L}  \SCnt{B}}
        \end{math}
      \end{center}

\item Case 2:
      \begin{center}
        \scriptsize
        \begin{math}
          \begin{array}{c}
            \Pi_1 \\
            {\Gamma  \vdash_\mathcal{L}  \SCnt{C}}
          \end{array}
        \end{math}
        \qquad\qquad
        $\Pi_2$:
        \begin{math}
          $$\mprset{flushleft}
          \inferrule* [right={\tiny imprL}] {
            {
              \begin{array}{cc}
                \pi_1 & \pi_2 \\
                {\Delta_{{\mathrm{1}}}  \SCsym{;}  \SCnt{C}  \SCsym{;}  \Delta_{{\mathrm{2}}}  \vdash_\mathcal{L}  \SCnt{A_{{\mathrm{1}}}}} & {\Gamma_{{\mathrm{1}}}  \SCsym{;}  \SCnt{A_{{\mathrm{2}}}}  \SCsym{;}  \Gamma_{{\mathrm{2}}}  \vdash_\mathcal{L}  \SCnt{B}}
              \end{array}
            }
          }{\Gamma_{{\mathrm{1}}}  \SCsym{;}  \SCnt{A_{{\mathrm{1}}}}  \rightharpoonup  \SCnt{A_{{\mathrm{2}}}}  \SCsym{;}  \Delta_{{\mathrm{1}}}  \SCsym{;}  \SCnt{C}  \SCsym{;}  \Delta_{{\mathrm{2}}}  \SCsym{;}  \Gamma_{{\mathrm{2}}}  \vdash_\mathcal{L}  \SCnt{B}}
        \end{math}
      \end{center}
      By assumption, $c(\Pi_1),c(\Pi_2)\leq |C|$. By induction on $\Pi_1$
      and $\pi_1$, there is a proof $\Pi'$ for sequent
      $\Delta_{{\mathrm{1}}}  \SCsym{;}  \Gamma  \SCsym{;}  \Delta_{{\mathrm{2}}}  \vdash_\mathcal{L}  \SCnt{A_{{\mathrm{1}}}}$ s.t. $c(\Pi') \leq |C|$. Therefore, the proof
      $\Pi$ can be constructed as follows with $c(\Pi) = c(\Pi') \leq |C|$.
      \begin{center}
        \scriptsize
        \begin{math}
          $$\mprset{flushleft}
          \inferrule* [right={\tiny imprL}] {
            {
              \begin{array}{cc}
                \Pi' & \pi_2 \\
                {\Delta_{{\mathrm{1}}}  \SCsym{;}  \Gamma  \SCsym{;}  \Delta_{{\mathrm{2}}}  \vdash_\mathcal{L}  \SCnt{A_{{\mathrm{1}}}}} & {\Gamma_{{\mathrm{1}}}  \SCsym{;}  \SCnt{A_{{\mathrm{2}}}}  \SCsym{;}  \Gamma_{{\mathrm{2}}}  \vdash_\mathcal{L}  \SCnt{B}}
              \end{array}
            }
          }{\Gamma_{{\mathrm{1}}}  \SCsym{;}  \SCnt{A_{{\mathrm{1}}}}  \rightharpoonup  \SCnt{A_{{\mathrm{2}}}}  \SCsym{;}  \Delta_{{\mathrm{1}}}  \SCsym{;}  \Gamma  \SCsym{;}  \Delta_{{\mathrm{2}}}  \SCsym{;}  \Gamma_{{\mathrm{2}}}  \vdash_\mathcal{L}  \SCnt{B}}
        \end{math}
      \end{center}

\item Case 3:
      \begin{center}
        \scriptsize
        \begin{math}
          \begin{array}{c}
            \Pi_1 \\
            {\Phi  \vdash_\mathcal{C}  \SCnt{X}}
          \end{array}
        \end{math}
        \qquad\qquad
        $\Pi_2$:
        \begin{math}
          $$\mprset{flushleft}
          \inferrule* [right={\tiny imprL}] {
            {
              \begin{array}{cc}
                \pi_1 & \pi_2 \\
                {\Delta  \vdash_\mathcal{L}  \SCnt{A_{{\mathrm{1}}}}} & {\Gamma_{{\mathrm{1}}}  \SCsym{;}  \SCnt{X}  \SCsym{;}  \Gamma_{{\mathrm{2}}}  \SCsym{;}  \SCnt{A_{{\mathrm{2}}}}  \SCsym{;}  \Gamma_{{\mathrm{3}}}  \vdash_\mathcal{L}  \SCnt{B}}
              \end{array}
            }
          }{\Gamma_{{\mathrm{1}}}  \SCsym{;}  \SCnt{X}  \SCsym{;}  \Gamma_{{\mathrm{2}}}  \SCsym{;}  \SCnt{A_{{\mathrm{1}}}}  \rightharpoonup  \SCnt{A_{{\mathrm{2}}}}  \SCsym{;}  \Delta  \SCsym{;}  \Gamma_{{\mathrm{3}}}  \vdash_\mathcal{L}  \SCnt{B}}
        \end{math}
      \end{center}
      By assumption, $c(\Pi_1),c(\Pi_2)\leq |X|$. By induction on $\Pi_1$
      and $\pi_2$, there is a proof $\Pi'$ for sequent
      $\Gamma_{{\mathrm{1}}}  \SCsym{;}  \Phi  \SCsym{;}  \Gamma_{{\mathrm{2}}}  \SCsym{;}  \SCnt{A_{{\mathrm{2}}}}  \SCsym{;}  \Gamma_{{\mathrm{3}}}  \vdash_\mathcal{L}  \SCnt{B}$ s.t. $c(\Pi') \leq |X|$. Therefore,
      the proof $\Pi$ can be constructed as follows with
      $c(\Pi) = c(\Pi') \leq |X|$.
      \begin{center}
        \scriptsize
        \begin{math}
          $$\mprset{flushleft}
          \inferrule* [right={\tiny imprL}] {
            {
              \begin{array}{cc}
                \pi_1 & \Pi' \\
                {\Delta  \vdash_\mathcal{L}  \SCnt{A_{{\mathrm{1}}}}} & {\Gamma_{{\mathrm{1}}}  \SCsym{;}  \Phi  \SCsym{;}  \Gamma_{{\mathrm{2}}}  \SCsym{;}  \SCnt{A_{{\mathrm{2}}}}  \SCsym{;}  \Gamma_{{\mathrm{3}}}  \vdash_\mathcal{L}  \SCnt{B}}
              \end{array}
            }
          }{\Gamma_{{\mathrm{1}}}  \SCsym{;}  \Phi  \SCsym{;}  \Gamma_{{\mathrm{2}}}  \SCsym{;}  \SCnt{A_{{\mathrm{1}}}}  \rightharpoonup  \SCnt{A_{{\mathrm{2}}}}  \SCsym{;}  \Delta  \SCsym{;}  \Gamma_{{\mathrm{3}}}  \vdash_\mathcal{L}  \SCnt{B}}
        \end{math}
      \end{center}

\item Case 4:
      \begin{center}
        \scriptsize
        \begin{math}
          \begin{array}{c}
            \Pi_1 \\
            {\Delta_{{\mathrm{1}}}  \vdash_\mathcal{L}  \SCnt{B}}
          \end{array}
        \end{math}
        \qquad\qquad
        $\Pi_2$:
        \begin{math}
          $$\mprset{flushleft}
          \inferrule* [right={\tiny imprL}] {
            {
              \begin{array}{cc}
                \pi_1 & \pi_2 \\
                {\Delta_{{\mathrm{2}}}  \vdash_\mathcal{L}  \SCnt{A_{{\mathrm{1}}}}} & {\Gamma_{{\mathrm{1}}}  \SCsym{;}  \SCnt{B}  \SCsym{;}  \Gamma_{{\mathrm{2}}}  \SCsym{;}  \SCnt{A_{{\mathrm{2}}}}  \SCsym{;}  \Gamma_{{\mathrm{3}}}  \vdash_\mathcal{L}  \SCnt{C}}
              \end{array}
            }
          }{\Gamma_{{\mathrm{1}}}  \SCsym{;}  \SCnt{B}  \SCsym{;}  \Gamma_{{\mathrm{2}}}  \SCsym{;}  \SCnt{A_{{\mathrm{1}}}}  \rightharpoonup  \SCnt{A_{{\mathrm{2}}}}  \SCsym{;}  \Delta_{{\mathrm{2}}}  \SCsym{;}  \Gamma_{{\mathrm{3}}}  \vdash_\mathcal{L}  \SCnt{C}}
        \end{math}
      \end{center}
      By assumption, $c(\Pi_1),c(\Pi_2)\leq |B|$. By induction on $\Pi_1$
      and $\pi_2$, there is a proof $\Pi'$ for sequent
      $\Gamma_{{\mathrm{1}}}  \SCsym{;}  \Delta_{{\mathrm{1}}}  \SCsym{;}  \Gamma_{{\mathrm{2}}}  \SCsym{;}  \SCnt{A_{{\mathrm{2}}}}  \SCsym{;}  \Gamma_{{\mathrm{3}}}  \vdash_\mathcal{L}  \SCnt{C}$ s.t. $c(\Pi') \leq |B|$. Therefore,
      the proof $\Pi$ can be constructed as follows with
      $c(\Pi) = c(\Pi') \leq |B|$.
      \begin{center}
        \scriptsize
        \begin{math}
          $$\mprset{flushleft}
          \inferrule* [right={\tiny imprL}] {
            {
              \begin{array}{cc}
                \pi_1 & \Pi' \\
                {\Delta_{{\mathrm{2}}}  \vdash_\mathcal{L}  \SCnt{A_{{\mathrm{1}}}}} & {\Gamma_{{\mathrm{1}}}  \SCsym{;}  \Delta_{{\mathrm{1}}}  \SCsym{;}  \Gamma_{{\mathrm{2}}}  \SCsym{;}  \SCnt{A_{{\mathrm{2}}}}  \SCsym{;}  \Gamma_{{\mathrm{3}}}  \vdash_\mathcal{L}  \SCnt{C}}
              \end{array}
            }
          }{\Gamma_{{\mathrm{1}}}  \SCsym{;}  \Delta_{{\mathrm{1}}}  \SCsym{;}  \Gamma_{{\mathrm{2}}}  \SCsym{;}  \SCnt{A_{{\mathrm{1}}}}  \rightharpoonup  \SCnt{A_{{\mathrm{2}}}}  \SCsym{;}  \Delta_{{\mathrm{2}}}  \SCsym{;}  \Gamma_{{\mathrm{3}}}  \vdash_\mathcal{L}  \SCnt{C}}
        \end{math}
      \end{center}

\item Case 5:
      \begin{center}
        \scriptsize
        \begin{math}
          \begin{array}{c}
            \Pi_1 \\
            {\Phi  \vdash_\mathcal{C}  \SCnt{X}}
          \end{array}
        \end{math}
        \qquad\qquad
        $\Pi_2$:
        \begin{math}
          $$\mprset{flushleft}
          \inferrule* [right={\tiny imprL}] {
            {
              \begin{array}{cc}
                \pi_1 & \pi_2 \\
                {\Delta  \vdash_\mathcal{L}  \SCnt{A_{{\mathrm{1}}}}} & {\Gamma_{{\mathrm{1}}}  \SCsym{;}  \SCnt{A_{{\mathrm{2}}}}  \SCsym{;}  \Gamma_{{\mathrm{2}}}  \SCsym{;}  \SCnt{X}  \SCsym{;}  \Gamma_{{\mathrm{3}}}  \vdash_\mathcal{L}  \SCnt{B}}
              \end{array}
            }
          }{\Gamma_{{\mathrm{1}}}  \SCsym{;}  \SCnt{A_{{\mathrm{1}}}}  \rightharpoonup  \SCnt{A_{{\mathrm{2}}}}  \SCsym{;}  \Delta  \SCsym{;}  \Gamma_{{\mathrm{2}}}  \SCsym{;}  \SCnt{X}  \SCsym{;}  \Gamma_{{\mathrm{3}}}  \vdash_\mathcal{L}  \SCnt{B}}
        \end{math}
      \end{center}
      By assumption, $c(\Pi_1),c(\Pi_2)\leq |X|$. By induction on $\Pi_1$
      and $\pi_2$, there is a proof $\Pi'$ for sequent
      $\Gamma_{{\mathrm{1}}}  \SCsym{;}  \SCnt{A_{{\mathrm{2}}}}  \SCsym{;}  \Gamma_{{\mathrm{2}}}  \SCsym{;}  \Phi  \SCsym{;}  \Gamma_{{\mathrm{3}}}  \vdash_\mathcal{L}  \SCnt{B}$ s.t. $c(\Pi') \leq |X|$. Therefore, the
      proof $\Pi$ can be constructed as follows with
      $c(\Pi) = c(\Pi') \leq |X|$.
      \begin{center}
        \scriptsize
        \begin{math}
          $$\mprset{flushleft}
          \inferrule* [right={\tiny imprL}] {
            {
              \begin{array}{cc}
                \pi_1 & \Pi' \\
                {\Delta  \vdash_\mathcal{L}  \SCnt{A_{{\mathrm{1}}}}} & {\Gamma_{{\mathrm{1}}}  \SCsym{;}  \SCnt{A_{{\mathrm{2}}}}  \SCsym{;}  \Gamma_{{\mathrm{2}}}  \SCsym{;}  \Phi  \SCsym{;}  \Gamma_{{\mathrm{3}}}  \vdash_\mathcal{L}  \SCnt{B}}
              \end{array}
            }
          }{\Gamma_{{\mathrm{1}}}  \SCsym{;}  \SCnt{A_{{\mathrm{1}}}}  \rightharpoonup  \SCnt{A_{{\mathrm{2}}}}  \SCsym{;}  \Delta  \SCsym{;}  \Gamma_{{\mathrm{2}}}  \SCsym{;}  \Phi  \SCsym{;}  \Gamma_{{\mathrm{3}}}  \vdash_\mathcal{L}  \SCnt{B}}
        \end{math}
      \end{center}

\item Case 6:
      \begin{center}
        \scriptsize
        \begin{math}
          \begin{array}{c}
            \Pi_1 \\
            {\Delta_{{\mathrm{1}}}  \vdash_\mathcal{L}  \SCnt{B}}
          \end{array}
        \end{math}
        \qquad\qquad
        $\Pi_2$:
        \begin{math}
          $$\mprset{flushleft}
          \inferrule* [right={\tiny imprL}] {
            {
              \begin{array}{cc}
                \pi_1 & \pi_2 \\
                {\Delta_{{\mathrm{2}}}  \vdash_\mathcal{L}  \SCnt{A_{{\mathrm{1}}}}} & {\Gamma_{{\mathrm{1}}}  \SCsym{;}  \SCnt{A_{{\mathrm{2}}}}  \SCsym{;}  \Gamma_{{\mathrm{2}}}  \SCsym{;}  \SCnt{B}  \SCsym{;}  \Gamma_{{\mathrm{3}}}  \vdash_\mathcal{L}  \SCnt{C}}
              \end{array}
            }
          }{\Gamma_{{\mathrm{1}}}  \SCsym{;}  \SCnt{A_{{\mathrm{1}}}}  \rightharpoonup  \SCnt{A_{{\mathrm{2}}}}  \SCsym{;}  \Delta_{{\mathrm{2}}}  \SCsym{;}  \Gamma_{{\mathrm{2}}}  \SCsym{;}  \SCnt{B}  \SCsym{;}  \Gamma_{{\mathrm{3}}}  \vdash_\mathcal{L}  \SCnt{C}}
        \end{math}
      \end{center}
      By assumption, $c(\Pi_1),c(\Pi_2)\leq |B|$. By induction on $\Pi_1$
      and $\pi_2$, there is a proof $\Pi'$ for sequent
      $\Gamma_{{\mathrm{1}}}  \SCsym{;}  \SCnt{A_{{\mathrm{2}}}}  \SCsym{;}  \Gamma_{{\mathrm{2}}}  \SCsym{;}  \Delta_{{\mathrm{1}}}  \SCsym{;}  \Gamma_{{\mathrm{3}}}  \vdash_\mathcal{L}  \SCnt{C}$ s.t. $c(\Pi') \leq |B|$. Therefore,
      the proof $\Pi$ can be constructed as follows with
      $c(\Pi) = c(\Pi') \leq |B|$.
      \begin{center}
        \scriptsize
        \begin{math}
          $$\mprset{flushleft}
          \inferrule* [right={\tiny imprL}] {
            {
              \begin{array}{cc}
                \pi_1 & \Pi' \\
                {\Delta_{{\mathrm{2}}}  \vdash_\mathcal{L}  \SCnt{A_{{\mathrm{1}}}}} & {\Gamma_{{\mathrm{1}}}  \SCsym{;}  \SCnt{A_{{\mathrm{2}}}}  \SCsym{;}  \Gamma_{{\mathrm{2}}}  \SCsym{;}  \Delta_{{\mathrm{1}}}  \SCsym{;}  \Gamma_{{\mathrm{3}}}  \vdash_\mathcal{L}  \SCnt{C}}
              \end{array}
            }
          }{\Gamma_{{\mathrm{1}}}  \SCsym{;}  \SCnt{A_{{\mathrm{1}}}}  \rightharpoonup  \SCnt{A_{{\mathrm{2}}}}  \SCsym{;}  \Delta_{{\mathrm{2}}}  \SCsym{;}  \Gamma_{{\mathrm{2}}}  \SCsym{;}  \Delta_{{\mathrm{1}}}  \SCsym{;}  \Gamma_{{\mathrm{3}}}  \vdash_\mathcal{L}  \SCnt{C}}
        \end{math}
      \end{center}
\end{itemize}


\subsubsection{Left introduction of the non-commutative right implication $\rto$}
\begin{itemize}
\item Case 1:
      \begin{center}
        \scriptsize
        \begin{math}
          \begin{array}{c}
            \Pi_1 \\
            {\Phi  \vdash_\mathcal{C}  \SCnt{X}}
          \end{array}
        \end{math}
        \qquad\qquad
        $\Pi_2$:
        \begin{math}
          $$\mprset{flushleft}
          \inferrule* [right={\tiny implL}] {
            {
              \begin{array}{cc}
                \pi_1 & \pi_2 \\
                {\Delta_{{\mathrm{1}}}  \SCsym{;}  \SCnt{X}  \SCsym{;}  \Delta_{{\mathrm{2}}}  \vdash_\mathcal{L}  \SCnt{A_{{\mathrm{1}}}}} & {\Gamma_{{\mathrm{1}}}  \SCsym{;}  \SCnt{A_{{\mathrm{2}}}}  \SCsym{;}  \Gamma_{{\mathrm{2}}}  \vdash_\mathcal{L}  \SCnt{B}}
              \end{array}
            }
          }{\Gamma_{{\mathrm{1}}}  \SCsym{;}  \Delta_{{\mathrm{1}}}  \SCsym{;}  \SCnt{A_{{\mathrm{2}}}}  \leftharpoonup  \SCnt{A_{{\mathrm{1}}}}  \SCsym{;}  \SCnt{X}  \SCsym{;}  \Delta_{{\mathrm{2}}}  \SCsym{;}  \Gamma_{{\mathrm{2}}}  \vdash_\mathcal{L}  \SCnt{B}}
        \end{math}
      \end{center}
      By assumption, $c(\Pi_1),c(\Pi_2)\leq |X|$. By induction on $\Pi_1$
      and $\pi_1$, there is a proof $\Pi'$ for sequent
      $\Delta_{{\mathrm{1}}}  \SCsym{;}  \Phi  \SCsym{;}  \Delta_{{\mathrm{2}}}  \vdash_\mathcal{L}  \SCnt{A_{{\mathrm{1}}}}$ s.t. $c(\Pi') \leq |X|$. Therefore, the proof
      $\Pi$ can be constructed as follows with $c(\Pi) = c(\Pi') \leq |X|$.
      \begin{center}
        \scriptsize
        \begin{math}
          $$\mprset{flushleft}
          \inferrule* [right={\tiny implL}] {
            {
              \begin{array}{cc}
                \Pi' & \pi_2 \\
                {\Delta_{{\mathrm{1}}}  \SCsym{;}  \Phi  \SCsym{;}  \Delta_{{\mathrm{2}}}  \vdash_\mathcal{L}  \SCnt{A_{{\mathrm{1}}}}} & {\Gamma_{{\mathrm{1}}}  \SCsym{;}  \SCnt{A_{{\mathrm{2}}}}  \SCsym{;}  \Gamma_{{\mathrm{2}}}  \vdash_\mathcal{L}  \SCnt{B}}
              \end{array}
            }
          }{\Gamma_{{\mathrm{1}}}  \SCsym{;}  \Delta_{{\mathrm{1}}}  \SCsym{;}  \SCnt{A_{{\mathrm{2}}}}  \leftharpoonup  \SCnt{A_{{\mathrm{1}}}}  \SCsym{;}  \Phi  \SCsym{;}  \Delta_{{\mathrm{2}}}  \SCsym{;}  \Gamma_{{\mathrm{2}}}  \vdash_\mathcal{L}  \SCnt{B}}
        \end{math}
      \end{center}

\item Case 2:
      \begin{center}
        \scriptsize
        \begin{math}
          \begin{array}{c}
            \Pi_1 \\
            {\Gamma  \vdash_\mathcal{L}  \SCnt{C}}
          \end{array}
        \end{math}
        \qquad\qquad
        $\Pi_2$:
        \begin{math}
          $$\mprset{flushleft}
          \inferrule* [right={\tiny implL}] {
            {
              \begin{array}{cc}
                \pi_1 & \pi_2 \\
                {\Delta_{{\mathrm{1}}}  \SCsym{;}  \SCnt{C}  \SCsym{;}  \Delta_{{\mathrm{2}}}  \vdash_\mathcal{L}  \SCnt{A_{{\mathrm{1}}}}} & {\Gamma_{{\mathrm{1}}}  \SCsym{;}  \SCnt{A_{{\mathrm{2}}}}  \SCsym{;}  \Gamma_{{\mathrm{2}}}  \vdash_\mathcal{L}  \SCnt{B}}
              \end{array}
            }
          }{\Gamma_{{\mathrm{1}}}  \SCsym{;}  \Delta_{{\mathrm{1}}}  \SCsym{;}  \SCnt{C}  \SCsym{;}  \Delta_{{\mathrm{2}}}  \SCsym{;}  \SCnt{A_{{\mathrm{2}}}}  \leftharpoonup  \SCnt{A_{{\mathrm{1}}}}  \SCsym{;}  \Gamma_{{\mathrm{2}}}  \vdash_\mathcal{L}  \SCnt{B}}
        \end{math}
      \end{center}
      By assumption, $c(\Pi_1),c(\Pi_2)\leq |C|$. By induction on $\Pi_1$
      and $\pi_1$, there is a proof $\Pi'$ for sequent
      $\Delta_{{\mathrm{1}}}  \SCsym{;}  \Gamma  \SCsym{;}  \Delta_{{\mathrm{2}}}  \vdash_\mathcal{L}  \SCnt{A_{{\mathrm{1}}}}$ s.t. $c(\Pi') \leq |C|$. Therefore, the proof
      $\Pi$ can be constructed as follows with $c(\Pi) = c(\Pi') \leq |C|$.
      \begin{center}
        \scriptsize
        \begin{math}
          $$\mprset{flushleft}
          \inferrule* [right={\tiny implL}] {
            {
              \begin{array}{cc}
                \Pi' & \pi_2 \\
                {\Delta_{{\mathrm{1}}}  \SCsym{;}  \Gamma  \SCsym{;}  \Delta_{{\mathrm{2}}}  \vdash_\mathcal{L}  \SCnt{A_{{\mathrm{1}}}}} & {\Gamma_{{\mathrm{1}}}  \SCsym{;}  \SCnt{A_{{\mathrm{2}}}}  \SCsym{;}  \Gamma_{{\mathrm{2}}}  \vdash_\mathcal{L}  \SCnt{B}}
              \end{array}
            }
          }{\Gamma_{{\mathrm{1}}}  \SCsym{;}  \Delta_{{\mathrm{1}}}  \SCsym{;}  \Gamma  \SCsym{;}  \Delta_{{\mathrm{2}}}  \SCsym{;}  \SCnt{A_{{\mathrm{2}}}}  \leftharpoonup  \SCnt{A_{{\mathrm{1}}}}  \SCsym{;}  \Gamma_{{\mathrm{2}}}  \vdash_\mathcal{L}  \SCnt{B}}
        \end{math}
      \end{center}

\item Case 3:
      \begin{center}
        \scriptsize
        \begin{math}
          \begin{array}{c}
            \Pi_1 \\
            {\Phi  \vdash_\mathcal{C}  \SCnt{X}}
          \end{array}
        \end{math}
        \qquad\qquad
        $\Pi_2$:
        \begin{math}
          $$\mprset{flushleft}
          \inferrule* [right={\tiny implL}] {
            {
              \begin{array}{cc}
                \pi_1 & \pi_2 \\
                {\Delta  \vdash_\mathcal{L}  \SCnt{A_{{\mathrm{1}}}}} & {\Gamma_{{\mathrm{1}}}  \SCsym{;}  \SCnt{X}  \SCsym{;}  \Gamma_{{\mathrm{2}}}  \SCsym{;}  \SCnt{A_{{\mathrm{2}}}}  \SCsym{;}  \Gamma_{{\mathrm{3}}}  \vdash_\mathcal{L}  \SCnt{B}}
              \end{array}
            }
          }{\Gamma_{{\mathrm{1}}}  \SCsym{;}  \SCnt{X}  \SCsym{;}  \Gamma_{{\mathrm{2}}}  \SCsym{;}  \Delta  \SCsym{;}  \SCnt{A_{{\mathrm{2}}}}  \leftharpoonup  \SCnt{A_{{\mathrm{1}}}}  \SCsym{;}  \Gamma_{{\mathrm{3}}}  \vdash_\mathcal{L}  \SCnt{B}}
        \end{math}
      \end{center}
      By assumption, $c(\Pi_1),c(\Pi_2)\leq |X|$. By induction on $\Pi_1$
      and $\pi_2$, there is a proof $\Pi'$ for sequent
      $\Gamma_{{\mathrm{1}}}  \SCsym{;}  \Phi  \SCsym{;}  \Gamma_{{\mathrm{2}}}  \SCsym{;}  \SCnt{A_{{\mathrm{2}}}}  \SCsym{;}  \Gamma_{{\mathrm{3}}}  \vdash_\mathcal{L}  \SCnt{B}$ s.t. $c(\Pi') \leq |X|$. Therefore, the
      proof $\Pi$ can be constructed as follows with
      $c(\Pi) = c(\Pi') \leq |X|$.
      \begin{center}
        \scriptsize
        \begin{math}
          $$\mprset{flushleft}
          \inferrule* [right={\tiny implL}] {
            {
              \begin{array}{cc}
                \pi_1 & \Pi' \\
                {\Delta  \vdash_\mathcal{L}  \SCnt{A_{{\mathrm{1}}}}} & {\Gamma_{{\mathrm{1}}}  \SCsym{;}  \Phi  \SCsym{;}  \Gamma_{{\mathrm{2}}}  \SCsym{;}  \SCnt{A_{{\mathrm{2}}}}  \SCsym{;}  \Gamma_{{\mathrm{3}}}  \vdash_\mathcal{L}  \SCnt{B}}
              \end{array}
            }
          }{\Gamma_{{\mathrm{1}}}  \SCsym{;}  \Phi  \SCsym{;}  \Gamma_{{\mathrm{2}}}  \SCsym{;}  \Delta  \SCsym{;}  \SCnt{A_{{\mathrm{2}}}}  \leftharpoonup  \SCnt{A_{{\mathrm{1}}}}  \SCsym{;}  \Gamma_{{\mathrm{3}}}  \vdash_\mathcal{L}  \SCnt{B}}
        \end{math}
      \end{center}

\item Case 4:
      \begin{center}
        \scriptsize
        \begin{math}
          \begin{array}{c}
            \Pi_1 \\
            {\Delta_{{\mathrm{1}}}  \vdash_\mathcal{L}  \SCnt{B}}
          \end{array}
        \end{math}
        \qquad\qquad
        $\Pi_2$:
        \begin{math}
          $$\mprset{flushleft}
          \inferrule* [right={\tiny implL}] {
            {
              \begin{array}{cc}
                \pi_1 & \pi_2 \\
                {\Delta_{{\mathrm{2}}}  \vdash_\mathcal{L}  \SCnt{A_{{\mathrm{1}}}}} & {\Gamma_{{\mathrm{1}}}  \SCsym{;}  \SCnt{B}  \SCsym{;}  \Gamma_{{\mathrm{2}}}  \SCsym{;}  \SCnt{A_{{\mathrm{2}}}}  \SCsym{;}  \Gamma_{{\mathrm{3}}}  \vdash_\mathcal{L}  \SCnt{C}}
              \end{array}
            }
          }{\Gamma_{{\mathrm{1}}}  \SCsym{;}  \SCnt{B}  \SCsym{;}  \Gamma_{{\mathrm{2}}}  \SCsym{;}  \Delta_{{\mathrm{2}}}  \SCsym{;}  \SCnt{A_{{\mathrm{2}}}}  \leftharpoonup  \SCnt{A_{{\mathrm{1}}}}  \SCsym{;}  \Gamma_{{\mathrm{3}}}  \vdash_\mathcal{L}  \SCnt{C}}
        \end{math}
      \end{center}
      By assumption, $c(\Pi_1),c(\Pi_2)\leq |B|$. By induction on $\Pi_1$
      and $\pi_2$, there is a proof $\Pi'$ for sequent
      $\Gamma_{{\mathrm{1}}}  \SCsym{;}  \Delta_{{\mathrm{1}}}  \SCsym{;}  \Gamma_{{\mathrm{2}}}  \SCsym{;}  \SCnt{A_{{\mathrm{2}}}}  \SCsym{;}  \Gamma_{{\mathrm{3}}}  \vdash_\mathcal{L}  \SCnt{C}$ s.t. $c(\Pi') \leq |B|$. Therefore,
      the proof $\Pi$ can be constructed as follows with
      $c(\Pi) = c(\Pi') \leq |B|$.
      \begin{center}
        \scriptsize
        \begin{math}
          $$\mprset{flushleft}
          \inferrule* [right={\tiny implL}] {
            {
              \begin{array}{cc}
                \pi_1 & \Pi' \\
                {\Delta_{{\mathrm{2}}}  \vdash_\mathcal{L}  \SCnt{A_{{\mathrm{1}}}}} & {\Gamma_{{\mathrm{1}}}  \SCsym{;}  \Delta_{{\mathrm{1}}}  \SCsym{;}  \Gamma_{{\mathrm{2}}}  \SCsym{;}  \SCnt{A_{{\mathrm{2}}}}  \SCsym{;}  \Gamma_{{\mathrm{3}}}  \vdash_\mathcal{L}  \SCnt{C}}
              \end{array}
            }
          }{\Gamma_{{\mathrm{1}}}  \SCsym{;}  \Delta_{{\mathrm{1}}}  \SCsym{;}  \Gamma_{{\mathrm{2}}}  \SCsym{;}  \Delta_{{\mathrm{2}}}  \SCsym{;}  \SCnt{A_{{\mathrm{2}}}}  \leftharpoonup  \SCnt{A_{{\mathrm{1}}}}  \SCsym{;}  \Gamma_{{\mathrm{3}}}  \vdash_\mathcal{L}  \SCnt{C}}
        \end{math}
      \end{center}

\item Case 5:
      \begin{center}
        \scriptsize
        \begin{math}
          \begin{array}{c}
            \Pi_1 \\
            {\Phi  \vdash_\mathcal{C}  \SCnt{X}}
          \end{array}
        \end{math}
        \qquad\qquad
        $\Pi_2$:
        \begin{math}
          $$\mprset{flushleft}
          \inferrule* [right={\tiny implL}] {
            {
              \begin{array}{cc}
                \pi_1 & \pi_2 \\
                {\Delta  \vdash_\mathcal{L}  \SCnt{A_{{\mathrm{1}}}}} & {\Gamma_{{\mathrm{1}}}  \SCsym{;}  \SCnt{A_{{\mathrm{2}}}}  \SCsym{;}  \Gamma_{{\mathrm{2}}}  \SCsym{;}  \SCnt{X}  \SCsym{;}  \Gamma_{{\mathrm{3}}}  \vdash_\mathcal{L}  \SCnt{B}}
              \end{array}
            }
          }{\Gamma_{{\mathrm{1}}}  \SCsym{;}  \Delta  \SCsym{;}  \SCnt{A_{{\mathrm{2}}}}  \leftharpoonup  \SCnt{A_{{\mathrm{1}}}}  \SCsym{;}  \Delta  \SCsym{;}  \Gamma_{{\mathrm{2}}}  \SCsym{;}  \SCnt{X}  \SCsym{;}  \Gamma_{{\mathrm{3}}}  \vdash_\mathcal{L}  \SCnt{B}}
        \end{math}
      \end{center}
      By assumption, $c(\Pi_1),c(\Pi_2)\leq |X|$. By induction on $\Pi_1$
      and $\pi_2$, there is a proof $\Pi'$ for sequent
      $\Gamma_{{\mathrm{1}}}  \SCsym{;}  \SCnt{A_{{\mathrm{2}}}}  \SCsym{;}  \Gamma_{{\mathrm{2}}}  \SCsym{;}  \Phi  \SCsym{;}  \Gamma_{{\mathrm{3}}}  \vdash_\mathcal{L}  \SCnt{B}$ s.t. $c(\Pi') \leq |X|$. Therefore, the
      proof $\Pi$ can be constructed as follows with
      $c(\Pi) = c(\Pi') \leq |X|$.
      \begin{center}
        \scriptsize
        \begin{math}
          $$\mprset{flushleft}
          \inferrule* [right={\tiny implL}] {
            {
              \begin{array}{cc}
                \pi_1 & \Pi' \\
                {\Delta  \vdash_\mathcal{L}  \SCnt{A_{{\mathrm{1}}}}} & {\Gamma_{{\mathrm{1}}}  \SCsym{;}  \SCnt{A_{{\mathrm{2}}}}  \SCsym{;}  \Gamma_{{\mathrm{2}}}  \SCsym{;}  \Phi  \SCsym{;}  \Gamma_{{\mathrm{3}}}  \vdash_\mathcal{L}  \SCnt{B}}
              \end{array}
            }
          }{\Gamma_{{\mathrm{1}}}  \SCsym{;}  \Delta  \SCsym{;}  \SCnt{A_{{\mathrm{2}}}}  \leftharpoonup  \SCnt{A_{{\mathrm{1}}}}  \SCsym{;}  \Gamma_{{\mathrm{2}}}  \SCsym{;}  \Phi  \SCsym{;}  \Gamma_{{\mathrm{3}}}  \vdash_\mathcal{L}  \SCnt{B}}
        \end{math}
      \end{center}

\item Case 6:
    \begin{center}
      \scriptsize
      \begin{math}
        \begin{array}{c}
          \Pi_1 \\
          {\Delta_{{\mathrm{1}}}  \vdash_\mathcal{L}  \SCnt{B}}
        \end{array}
      \end{math}
      \qquad\qquad
      $\Pi_2$:
      \begin{math}
        $$\mprset{flushleft}
        \inferrule* [right={\tiny implL}] {
          {
            \begin{array}{cc}
              \pi_1 & \pi_2 \\
              {\Delta_{{\mathrm{2}}}  \vdash_\mathcal{L}  \SCnt{A_{{\mathrm{1}}}}} & {\Gamma_{{\mathrm{1}}}  \SCsym{;}  \SCnt{A_{{\mathrm{2}}}}  \SCsym{;}  \Gamma_{{\mathrm{2}}}  \SCsym{;}  \SCnt{B}  \SCsym{;}  \Gamma_{{\mathrm{3}}}  \vdash_\mathcal{L}  \SCnt{C}}
            \end{array}
          }
        }{\Gamma_{{\mathrm{1}}}  \SCsym{;}  \Delta_{{\mathrm{2}}}  \SCsym{;}  \SCnt{A_{{\mathrm{2}}}}  \leftharpoonup  \SCnt{A_{{\mathrm{1}}}}  \SCsym{;}  \Gamma_{{\mathrm{2}}}  \SCsym{;}  \SCnt{B}  \SCsym{;}  \Gamma_{{\mathrm{3}}}  \vdash_\mathcal{L}  \SCnt{C}}
      \end{math}
    \end{center}
    By assumption, $c(\Pi_1),c(\Pi_2)\leq |B|$. By induction on $\Pi_1$ and
    $\pi_2$, there is a proof $\Pi'$ for sequent
    $\Gamma_{{\mathrm{1}}}  \SCsym{;}  \SCnt{A_{{\mathrm{2}}}}  \SCsym{;}  \Gamma_{{\mathrm{2}}}  \SCsym{;}  \Delta_{{\mathrm{1}}}  \SCsym{;}  \Gamma_{{\mathrm{3}}}  \vdash_\mathcal{L}  \SCnt{C}$ s.t. $c(\Pi') \leq |B|$. Therefore, the
    proof $\Pi$ can be constructed as follows with
    $c(\Pi) = c(\Pi') \leq |B|$.
    \begin{center}
      \scriptsize
      \begin{math}
        $$\mprset{flushleft}
        \inferrule* [right={\tiny implL}] {
          {
            \begin{array}{cc}
              \pi_1 & \Pi' \\
              {\Delta_{{\mathrm{2}}}  \vdash_\mathcal{L}  \SCnt{A_{{\mathrm{1}}}}} & {\Gamma_{{\mathrm{1}}}  \SCsym{;}  \SCnt{A_{{\mathrm{2}}}}  \SCsym{;}  \Gamma_{{\mathrm{2}}}  \SCsym{;}  \Delta_{{\mathrm{1}}}  \SCsym{;}  \Gamma_{{\mathrm{3}}}  \vdash_\mathcal{L}  \SCnt{C}}
            \end{array}
          }
        }{\Gamma_{{\mathrm{1}}}  \SCsym{;}  \Delta_{{\mathrm{2}}}  \SCsym{;}  \SCnt{A_{{\mathrm{2}}}}  \leftharpoonup  \SCnt{A_{{\mathrm{1}}}}  \SCsym{;}  \Gamma_{{\mathrm{2}}}  \SCsym{;}  \Delta_{{\mathrm{1}}}  \SCsym{;}  \Gamma_{{\mathrm{3}}}  \vdash_\mathcal{L}  \SCnt{C}}
      \end{math}
    \end{center}
\end{itemize}




\subsubsection{Left introduction of the commutative tensor $\otimes$ (with low priority)}
\begin{itemize}
\item Case 1:
      \begin{center}
        \scriptsize
        \begin{math}
          \begin{array}{c}
            \Pi_1 \\
            {\Phi  \vdash_\mathcal{C}  \SCnt{X}}
          \end{array}
        \end{math}
        \qquad\qquad
        $\Pi_2$:
        \begin{math}
          $$\mprset{flushleft}
          \inferrule* [right={\tiny tenL}] {
            {
              \begin{array}{c}
                \pi \\
                {\Psi_{{\mathrm{1}}}  \SCsym{,}  \SCnt{X}  \SCsym{,}  \Psi_{{\mathrm{2}}}  \SCsym{,}  \SCnt{Y_{{\mathrm{1}}}}  \SCsym{,}  \SCnt{Y_{{\mathrm{2}}}}  \SCsym{,}  \Psi_{{\mathrm{3}}}  \vdash_\mathcal{C}  \SCnt{Z}}
              \end{array}
            }
          }{\Psi_{{\mathrm{1}}}  \SCsym{,}  \SCnt{X}  \SCsym{,}  \Psi_{{\mathrm{2}}}  \SCsym{,}  \SCnt{Y_{{\mathrm{1}}}}  \otimes  \SCnt{Y_{{\mathrm{2}}}}  \SCsym{,}  \Psi_{{\mathrm{3}}}  \vdash_\mathcal{C}  \SCnt{Z}}
        \end{math}
      \end{center}
      By assumption, $c(\Pi_1),c(\Pi_2)\leq |X|$. By induction on $\Pi_1$
      and $\pi$, there is a proof $\Pi'$ for sequent
      $\Psi_{{\mathrm{1}}}  \SCsym{,}  \Phi  \SCsym{,}  \Psi_{{\mathrm{2}}}  \SCsym{,}  \SCnt{Y_{{\mathrm{1}}}}  \SCsym{,}  \SCnt{Y_{{\mathrm{2}}}}  \SCsym{,}  \Psi_{{\mathrm{3}}}  \vdash_\mathcal{C}  \SCnt{Z}$ s.t. $c(\Pi') \leq |X|$. Therefore,
      the proof $\Pi$ can be constructed as follows with
      $c(\Pi) = c(\Pi') \leq |X|$.
      \begin{center}
        \scriptsize
        \begin{math}
          $$\mprset{flushleft}
          \inferrule* [right={\tiny tenL}] {
            {
              \begin{array}{c}
                \Pi' \\
                {\Psi_{{\mathrm{1}}}  \SCsym{,}  \Phi  \SCsym{,}  \Psi_{{\mathrm{2}}}  \SCsym{,}  \SCnt{Y_{{\mathrm{1}}}}  \SCsym{,}  \SCnt{Y_{{\mathrm{2}}}}  \SCsym{,}  \Psi_{{\mathrm{3}}}  \vdash_\mathcal{C}  \SCnt{Z}}
              \end{array}
            }
          }{\Psi_{{\mathrm{1}}}  \SCsym{,}  \Phi  \SCsym{,}  \Psi_{{\mathrm{2}}}  \SCsym{,}  \SCnt{Y_{{\mathrm{1}}}}  \otimes  \SCnt{Y_{{\mathrm{2}}}}  \SCsym{,}  \Psi_{{\mathrm{3}}}  \vdash_\mathcal{C}  \SCnt{Z}}
        \end{math}
      \end{center}

\item Case 2:
      \begin{center}
        \scriptsize
        \begin{math}
          \begin{array}{c}
            \Pi_1 \\
            {\Phi  \vdash_\mathcal{C}  \SCnt{X}}
          \end{array}
        \end{math}
        \qquad\qquad
        $\Pi_2$:
        \begin{math}
          $$\mprset{flushleft}
          \inferrule* [right={\tiny tenL}] {
            {
              \begin{array}{c}
                \pi \\
                {\Psi_{{\mathrm{1}}}  \SCsym{,}  \SCnt{Y_{{\mathrm{1}}}}  \SCsym{,}  \SCnt{Y_{{\mathrm{2}}}}  \SCsym{,}  \Psi_{{\mathrm{2}}}  \SCsym{,}  \SCnt{X}  \SCsym{,}  \Psi_{{\mathrm{3}}}  \vdash_\mathcal{C}  \SCnt{Z}}
              \end{array}
            }
          }{\Psi_{{\mathrm{1}}}  \SCsym{,}  \SCnt{Y_{{\mathrm{1}}}}  \otimes  \SCnt{Y_{{\mathrm{2}}}}  \SCsym{,}  \Psi_{{\mathrm{2}}}  \SCsym{,}  \SCnt{X}  \SCsym{,}  \Psi_{{\mathrm{3}}}  \vdash_\mathcal{C}  \SCnt{Z}}
        \end{math}
      \end{center}
      By assumption, $c(\Pi_1),c(\Pi_2)\leq |X|$. By induction on $\Pi_1$
      and $\pi$, there is a proof $\Pi'$ for sequent
      $\Psi_{{\mathrm{1}}}  \SCsym{,}  \SCnt{Y_{{\mathrm{1}}}}  \SCsym{,}  \SCnt{Y_{{\mathrm{2}}}}  \SCsym{,}  \Psi_{{\mathrm{2}}}  \SCsym{,}  \Phi  \SCsym{,}  \Psi_{{\mathrm{3}}}  \vdash_\mathcal{C}  \SCnt{Z}$ s.t. $c(\Pi') \leq |X|$. Therefore,
      the proof $\Pi$ can be constructed as follows with
      $c(\Pi) = c(\Pi') \leq |X|$.
      \begin{center}
        \scriptsize
        \begin{math}
          $$\mprset{flushleft}
          \inferrule* [right={\tiny tenL}] {
            {
              \begin{array}{c}
                \Pi' \\
                {\Psi_{{\mathrm{1}}}  \SCsym{,}  \SCnt{Y_{{\mathrm{1}}}}  \SCsym{,}  \SCnt{Y_{{\mathrm{2}}}}  \SCsym{,}  \Psi_{{\mathrm{2}}}  \SCsym{,}  \Phi  \SCsym{,}  \Psi_{{\mathrm{3}}}  \vdash_\mathcal{C}  \SCnt{Z}}
              \end{array}
            }
          }{\Psi_{{\mathrm{1}}}  \SCsym{,}  \SCnt{Y_{{\mathrm{1}}}}  \otimes  \SCnt{Y_{{\mathrm{2}}}}  \SCsym{,}  \Psi_{{\mathrm{2}}}  \SCsym{,}  \Phi  \SCsym{,}  \Psi_{{\mathrm{3}}}  \vdash_\mathcal{C}  \SCnt{Z}}
        \end{math}
      \end{center}

\item Case 3:
      \begin{center}
        \scriptsize
        \begin{math}
          \begin{array}{c}
            \Pi_1 \\
            {\Phi  \vdash_\mathcal{C}  \SCnt{X}}
          \end{array}
        \end{math}
        \qquad\qquad
        $\Pi_2$:
        \begin{math}
          $$\mprset{flushleft}
          \inferrule* [right={\tiny tenL}] {
            {
              \begin{array}{c}
                \pi \\
                {\Gamma_{{\mathrm{1}}}  \SCsym{;}  \SCnt{X}  \SCsym{;}  \Gamma_{{\mathrm{2}}}  \SCsym{;}  \SCnt{Y_{{\mathrm{1}}}}  \SCsym{;}  \SCnt{Y_{{\mathrm{2}}}}  \SCsym{;}  \Gamma_{{\mathrm{3}}}  \vdash_\mathcal{L}  \SCnt{A}}
              \end{array}
            }
          }{\Gamma_{{\mathrm{1}}}  \SCsym{;}  \SCnt{X}  \SCsym{;}  \Gamma_{{\mathrm{2}}}  \SCsym{;}  \SCnt{Y_{{\mathrm{1}}}}  \otimes  \SCnt{Y_{{\mathrm{2}}}}  \SCsym{;}  \Gamma_{{\mathrm{3}}}  \vdash_\mathcal{L}  \SCnt{A}}
        \end{math}
      \end{center}
      By assumption, $c(\Pi_1),c(\Pi_2)\leq |X|$. By induction on $\Pi_1$
      and $\pi$, there is a proof $\Pi'$ for sequent
      $\Gamma_{{\mathrm{1}}}  \SCsym{;}  \Phi  \SCsym{;}  \Gamma_{{\mathrm{2}}}  \SCsym{;}  \SCnt{Y_{{\mathrm{1}}}}  \SCsym{;}  \SCnt{Y_{{\mathrm{2}}}}  \SCsym{;}  \Gamma_{{\mathrm{3}}}  \vdash_\mathcal{L}  \SCnt{A}$ s.t. $c(\Pi') \leq |X|$. Therefore,
      the proof $\Pi$ can be constructed as follows with
      $c(\Pi) = c(\Pi') \leq |X|$.
      \begin{center}
        \scriptsize
        \begin{math}
          $$\mprset{flushleft}
          \inferrule* [right={\tiny tenL}] {
            {
              \begin{array}{c}
                \Pi' \\
                {\Gamma_{{\mathrm{1}}}  \SCsym{;}  \Phi  \SCsym{;}  \Gamma_{{\mathrm{2}}}  \SCsym{;}  \SCnt{Y_{{\mathrm{1}}}}  \SCsym{;}  \SCnt{Y_{{\mathrm{2}}}}  \SCsym{;}  \Gamma_{{\mathrm{3}}}  \vdash_\mathcal{L}  \SCnt{A}}
              \end{array}
            }
          }{\Gamma_{{\mathrm{1}}}  \SCsym{;}  \Phi  \SCsym{;}  \Gamma_{{\mathrm{2}}}  \SCsym{;}  \SCnt{Y_{{\mathrm{1}}}}  \otimes  \SCnt{Y_{{\mathrm{2}}}}  \SCsym{;}  \Gamma_{{\mathrm{3}}}  \vdash_\mathcal{L}  \SCnt{A}}
        \end{math}
      \end{center}

\item Case 4:
      \begin{center}
        \scriptsize
        \begin{math}
          \begin{array}{c}
            \Pi_1 \\
            {\Delta  \vdash_\mathcal{L}  \SCnt{B}}
          \end{array}
        \end{math}
        \qquad\qquad
        $\Pi_2$:
        \begin{math}
          $$\mprset{flushleft}
          \inferrule* [right={\tiny tenL}] {
            {
              \begin{array}{c}
                \pi \\
                {\Gamma_{{\mathrm{1}}}  \SCsym{;}  \SCnt{B}  \SCsym{;}  \Gamma_{{\mathrm{2}}}  \SCsym{;}  \SCnt{Y_{{\mathrm{1}}}}  \SCsym{;}  \SCnt{Y_{{\mathrm{2}}}}  \SCsym{;}  \Gamma_{{\mathrm{3}}}  \vdash_\mathcal{L}  \SCnt{A}}
              \end{array}
            }
          }{\Gamma_{{\mathrm{1}}}  \SCsym{;}  \SCnt{B}  \SCsym{;}  \Gamma_{{\mathrm{2}}}  \SCsym{;}  \SCnt{Y_{{\mathrm{1}}}}  \otimes  \SCnt{Y_{{\mathrm{2}}}}  \SCsym{;}  \Gamma_{{\mathrm{3}}}  \vdash_\mathcal{L}  \SCnt{A}}
        \end{math}
      \end{center}
      By assumption, $c(\Pi_1),c(\Pi_2)\leq |B|$. By induction on $\Pi_1$
      and $\pi$, there is a proof $\Pi'$ for sequent
      $\Gamma_{{\mathrm{1}}}  \SCsym{;}  \SCnt{B}  \SCsym{;}  \Gamma_{{\mathrm{2}}}  \SCsym{;}  \SCnt{Y_{{\mathrm{1}}}}  \SCsym{;}  \SCnt{Y_{{\mathrm{2}}}}  \SCsym{;}  \Gamma_{{\mathrm{3}}}  \vdash_\mathcal{L}  \SCnt{A}$ s.t. $c(\Pi') \leq |B|$. Therefore,
      the proof $\Pi$ can be constructed as follows with
      $c(\Pi) = c(\Pi') \leq |B|$.
      \begin{center}
        \scriptsize
        \begin{math}
          $$\mprset{flushleft}
          \inferrule* [right={\tiny tenL}] {
            {
              \begin{array}{c}
                \Pi' \\
                {\Gamma_{{\mathrm{1}}}  \SCsym{;}  \Delta  \SCsym{;}  \Gamma_{{\mathrm{2}}}  \SCsym{;}  \SCnt{Y_{{\mathrm{1}}}}  \SCsym{;}  \SCnt{Y_{{\mathrm{2}}}}  \SCsym{;}  \Gamma_{{\mathrm{3}}}  \vdash_\mathcal{L}  \SCnt{A}}
              \end{array}
            }
          }{\Gamma_{{\mathrm{1}}}  \SCsym{;}  \Delta  \SCsym{;}  \Gamma_{{\mathrm{2}}}  \SCsym{;}  \SCnt{Y_{{\mathrm{1}}}}  \otimes  \SCnt{Y_{{\mathrm{2}}}}  \SCsym{;}  \Gamma_{{\mathrm{3}}}  \vdash_\mathcal{L}  \SCnt{A}}
        \end{math}
      \end{center}

\item Case 5:
      \begin{center}
        \scriptsize
        \begin{math}
          \begin{array}{c}
            \Pi_1 \\
            {\Phi  \vdash_\mathcal{C}  \SCnt{X}}
          \end{array}
        \end{math}
        \qquad\qquad
        $\Pi_2$:
        \begin{math}
          $$\mprset{flushleft}
          \inferrule* [right={\tiny tenL}] {
            {
              \begin{array}{c}
                \pi \\
                {\Gamma_{{\mathrm{1}}}  \SCsym{;}  \SCnt{Y_{{\mathrm{1}}}}  \SCsym{;}  \SCnt{Y_{{\mathrm{2}}}}  \SCsym{;}  \Gamma_{{\mathrm{2}}}  \SCsym{;}  \SCnt{X}  \SCsym{;}  \Gamma_{{\mathrm{3}}}  \vdash_\mathcal{L}  \SCnt{A}}
              \end{array}
            }
          }{\Gamma_{{\mathrm{1}}}  \SCsym{;}  \SCnt{Y_{{\mathrm{1}}}}  \otimes  \SCnt{Y_{{\mathrm{2}}}}  \SCsym{;}  \Gamma_{{\mathrm{2}}}  \SCsym{;}  \SCnt{X}  \SCsym{;}  \Gamma_{{\mathrm{3}}}  \vdash_\mathcal{L}  \SCnt{A}}
        \end{math}
      \end{center}
      By assumption, $c(\Pi_1),c(\Pi_2)\leq |X|$. By induction on $\Pi_1$
      and $\pi$, there is a proof $\Pi'$ for sequent
      $\Gamma_{{\mathrm{1}}}  \SCsym{;}  \SCnt{Y_{{\mathrm{1}}}}  \SCsym{;}  \SCnt{Y_{{\mathrm{2}}}}  \SCsym{;}  \Gamma_{{\mathrm{2}}}  \SCsym{;}  \Phi  \SCsym{;}  \Gamma_{{\mathrm{3}}}  \vdash_\mathcal{L}  \SCnt{A}$ s.t. $c(\Pi') \leq |X|$. Therefore,
      the proof $\Pi$ can be constructed as follows with
      $c(\Pi) = c(\Pi') \leq |X|$.
      \begin{center}
        \scriptsize
        \begin{math}
          $$\mprset{flushleft}
          \inferrule* [right={\tiny tenL}] {
            {
              \begin{array}{c}
                \Pi' \\
                {\Gamma_{{\mathrm{1}}}  \SCsym{;}  \SCnt{Y_{{\mathrm{1}}}}  \SCsym{;}  \SCnt{Y_{{\mathrm{2}}}}  \SCsym{;}  \Gamma_{{\mathrm{2}}}  \SCsym{;}  \Phi  \SCsym{;}  \Gamma_{{\mathrm{3}}}  \vdash_\mathcal{L}  \SCnt{A}}
              \end{array}
            }
          }{\Gamma_{{\mathrm{1}}}  \SCsym{;}  \SCnt{Y_{{\mathrm{1}}}}  \otimes  \SCnt{Y_{{\mathrm{2}}}}  \SCsym{;}  \Gamma_{{\mathrm{2}}}  \SCsym{;}  \Phi  \SCsym{;}  \Gamma_{{\mathrm{3}}}  \vdash_\mathcal{L}  \SCnt{A}}
        \end{math}
      \end{center}

\item Case 6:
      \begin{center}
        \scriptsize
        \begin{math}
          \begin{array}{c}
            \Pi_1 \\
            {\Delta  \vdash_\mathcal{L}  \SCnt{B}}
          \end{array}
        \end{math}
        \qquad\qquad
        $\Pi_2$:
        \begin{math}
          $$\mprset{flushleft}
          \inferrule* [right={\tiny tenL}] {
            {
              \begin{array}{c}
                \pi \\
                {\Gamma_{{\mathrm{1}}}  \SCsym{;}  \SCnt{Y_{{\mathrm{1}}}}  \SCsym{;}  \SCnt{Y_{{\mathrm{2}}}}  \SCsym{;}  \Gamma_{{\mathrm{2}}}  \SCsym{;}  \SCnt{B}  \SCsym{;}  \Gamma_{{\mathrm{3}}}  \vdash_\mathcal{L}  \SCnt{A}}
              \end{array}
            }
          }{\Gamma_{{\mathrm{1}}}  \SCsym{;}  \SCnt{Y_{{\mathrm{1}}}}  \otimes  \SCnt{Y_{{\mathrm{2}}}}  \SCsym{;}  \Gamma_{{\mathrm{2}}}  \SCsym{;}  \SCnt{B}  \SCsym{;}  \Gamma_{{\mathrm{3}}}  \vdash_\mathcal{L}  \SCnt{A}}
        \end{math}
      \end{center}
      By assumption, $c(\Pi_1),c(\Pi_2)\leq |B|$. By induction on $\Pi_1$
      and $\pi$, there is a proof $\Pi'$ for sequent
      $\Gamma_{{\mathrm{1}}}  \SCsym{;}  \SCnt{Y_{{\mathrm{1}}}}  \SCsym{;}  \SCnt{Y_{{\mathrm{2}}}}  \SCsym{;}  \Gamma_{{\mathrm{2}}}  \SCsym{;}  \Delta  \SCsym{;}  \Gamma_{{\mathrm{3}}}  \vdash_\mathcal{L}  \SCnt{A}$ s.t. $c(\Pi') \leq |B|$. Therefore,
      the proof $\Pi$ can be constructed as follows with
      $c(\Pi) = c(\Pi') \leq |B|$.
      \begin{center}
        \scriptsize
        \begin{math}
          $$\mprset{flushleft}
          \inferrule* [right={\tiny tenL}] {
            {
              \begin{array}{c}
                \Pi' \\
                {\Gamma_{{\mathrm{1}}}  \SCsym{;}  \SCnt{Y_{{\mathrm{1}}}}  \SCsym{;}  \SCnt{Y_{{\mathrm{2}}}}  \SCsym{;}  \Gamma_{{\mathrm{2}}}  \SCsym{;}  \Delta  \SCsym{;}  \Gamma_{{\mathrm{3}}}  \vdash_\mathcal{L}  \SCnt{A}}
              \end{array}
            }
          }{\Gamma_{{\mathrm{1}}}  \SCsym{;}  \SCnt{Y_{{\mathrm{1}}}}  \otimes  \SCnt{Y_{{\mathrm{2}}}}  \SCsym{;}  \Gamma_{{\mathrm{2}}}  \SCsym{;}  \Delta  \SCsym{;}  \Gamma_{{\mathrm{3}}}  \vdash_\mathcal{L}  \SCnt{A}}
        \end{math}
      \end{center}
\end{itemize}


\subsubsection{Left introduction of the non-commutative tensor $\tri$ (with low priority)}:
\begin{itemize}
\item Case 1:
      \begin{center}
        \scriptsize
        \begin{math}
          \begin{array}{c}
            \Pi_1 \\
            {\Phi  \vdash_\mathcal{C}  \SCnt{X}}
          \end{array}
        \end{math}
        \qquad\qquad
        $\Pi_2$:
        \begin{math}
          $$\mprset{flushleft}
          \inferrule* [right={\tiny tenL}] {
            {
              \begin{array}{c}
                \pi \\
                {\Gamma_{{\mathrm{1}}}  \SCsym{;}  \SCnt{X}  \SCsym{;}  \Gamma_{{\mathrm{2}}}  \SCsym{;}  \SCnt{A_{{\mathrm{1}}}}  \SCsym{;}  \SCnt{A_{{\mathrm{2}}}}  \SCsym{;}  \Gamma_{{\mathrm{3}}}  \vdash_\mathcal{L}  \SCnt{B}}
              \end{array}
            }
          }{\Gamma_{{\mathrm{1}}}  \SCsym{;}  \SCnt{X}  \SCsym{;}  \Gamma_{{\mathrm{2}}}  \SCsym{;}  \SCnt{A_{{\mathrm{1}}}}  \triangleright  \SCnt{A_{{\mathrm{2}}}}  \SCsym{;}  \Gamma_{{\mathrm{3}}}  \vdash_\mathcal{L}  \SCnt{B}}
        \end{math}
      \end{center}
      By assumption, $c(\Pi_1),c(\Pi_2)\leq |X|$. By induction on $\Pi_1$
      and $\pi$, there is a proof $\Pi'$ for sequent
      $\Gamma_{{\mathrm{1}}}  \SCsym{;}  \Phi  \SCsym{;}  \Gamma_{{\mathrm{2}}}  \SCsym{;}  \SCnt{A_{{\mathrm{1}}}}  \SCsym{;}  \SCnt{A_{{\mathrm{2}}}}  \SCsym{;}  \Gamma_{{\mathrm{3}}}  \vdash_\mathcal{L}  \SCnt{B}$ s.t. $c(\Pi') \leq |X|$. Therefore,
      the proof $\Pi$ can be constructed as follows with
      $c(\Pi) = c(\Pi') \leq |X|$.
      \begin{center}
        \scriptsize
        \begin{math}
          $$\mprset{flushleft}
          \inferrule* [right={\tiny tenL}] {
            {
              \begin{array}{c}
                \Pi' \\
                {\Gamma_{{\mathrm{1}}}  \SCsym{;}  \Phi  \SCsym{;}  \Gamma_{{\mathrm{2}}}  \SCsym{;}  \SCnt{A_{{\mathrm{1}}}}  \SCsym{;}  \SCnt{A_{{\mathrm{2}}}}  \SCsym{;}  \Gamma_{{\mathrm{3}}}  \vdash_\mathcal{L}  \SCnt{B}}
              \end{array}
            }
          }{\Gamma_{{\mathrm{1}}}  \SCsym{;}  \Phi  \SCsym{;}  \Gamma_{{\mathrm{2}}}  \SCsym{;}  \SCnt{A_{{\mathrm{1}}}}  \triangleright  \SCnt{A_{{\mathrm{2}}}}  \SCsym{;}  \Gamma_{{\mathrm{3}}}  \vdash_\mathcal{L}  \SCnt{B}}
        \end{math}
      \end{center}

\item Case 2:
      \begin{center}
        \scriptsize
        \begin{math}
          \begin{array}{c}
            \Pi_1 \\
            {\Delta  \vdash_\mathcal{L}  \SCnt{B}}
          \end{array}
        \end{math}
        \qquad\qquad
        $\Pi_2$:
        \begin{math}
          $$\mprset{flushleft}
          \inferrule* [right={\tiny tenL}] {
            {
              \begin{array}{c}
                \pi \\
                {\Gamma_{{\mathrm{1}}}  \SCsym{;}  \SCnt{B}  \SCsym{;}  \Gamma_{{\mathrm{2}}}  \SCsym{;}  \SCnt{A_{{\mathrm{1}}}}  \SCsym{;}  \SCnt{A_{{\mathrm{2}}}}  \SCsym{;}  \Gamma_{{\mathrm{3}}}  \vdash_\mathcal{L}  \SCnt{C}}
              \end{array}
            }
          }{\Gamma_{{\mathrm{1}}}  \SCsym{;}  \SCnt{B}  \SCsym{;}  \Gamma_{{\mathrm{2}}}  \SCsym{;}  \SCnt{A_{{\mathrm{1}}}}  \triangleright  \SCnt{A_{{\mathrm{2}}}}  \SCsym{;}  \Gamma_{{\mathrm{3}}}  \vdash_\mathcal{L}  \SCnt{C}}
        \end{math}
      \end{center}
      By assumption, $c(\Pi_1),c(\Pi_2)\leq |B|$. By induction on $\Pi_1$
      and $\pi$, there is a proof $\Pi'$ for sequent
      $\Gamma_{{\mathrm{1}}}  \SCsym{;}  \Delta  \SCsym{;}  \Gamma_{{\mathrm{2}}}  \SCsym{;}  \SCnt{A_{{\mathrm{1}}}}  \SCsym{;}  \SCnt{A_{{\mathrm{2}}}}  \SCsym{;}  \Gamma_{{\mathrm{3}}}  \vdash_\mathcal{L}  \SCnt{C}$ s.t. $c(\Pi') \leq |B|$. Therefore,
      the proof $\Pi$ can be constructed as follows with
      $c(\Pi) = c(\Pi') \leq |B|$.
      \begin{center}
        \scriptsize
        \begin{math}
          $$\mprset{flushleft}
          \inferrule* [right={\tiny tenL}] {
            {
              \begin{array}{c}
                \Pi' \\
                {\Gamma_{{\mathrm{1}}}  \SCsym{;}  \Delta  \SCsym{;}  \Gamma_{{\mathrm{2}}}  \SCsym{;}  \SCnt{A_{{\mathrm{1}}}}  \SCsym{;}  \SCnt{A_{{\mathrm{2}}}}  \SCsym{;}  \Gamma_{{\mathrm{3}}}  \vdash_\mathcal{L}  \SCnt{C}}
              \end{array}
            }
          }{\Gamma_{{\mathrm{1}}}  \SCsym{;}  \Delta  \SCsym{;}  \Gamma_{{\mathrm{2}}}  \SCsym{;}  \SCnt{A_{{\mathrm{1}}}}  \triangleright  \SCnt{A_{{\mathrm{2}}}}  \SCsym{;}  \Gamma_{{\mathrm{3}}}  \vdash_\mathcal{L}  \SCnt{C}}
        \end{math}
      \end{center}

\item Case 3:
      \begin{center}
        \scriptsize
        \begin{math}
          \begin{array}{c}
            \Pi_1 \\
            {\Phi  \vdash_\mathcal{C}  \SCnt{X}}
          \end{array}
        \end{math}
        \qquad\qquad
        $\Pi_2$:
        \begin{math}
          $$\mprset{flushleft}
          \inferrule* [right={\tiny tenL}] {
            {
              \begin{array}{c}
                \pi \\
                {\Gamma_{{\mathrm{1}}}  \SCsym{;}  \SCnt{A_{{\mathrm{1}}}}  \SCsym{;}  \SCnt{A_{{\mathrm{2}}}}  \SCsym{;}  \Gamma_{{\mathrm{2}}}  \SCsym{;}  \SCnt{X}  \SCsym{;}  \Gamma_{{\mathrm{3}}}  \vdash_\mathcal{L}  \SCnt{B}}
              \end{array}
            }
          }{\Gamma_{{\mathrm{1}}}  \SCsym{;}  \SCnt{A_{{\mathrm{1}}}}  \triangleright  \SCnt{A_{{\mathrm{2}}}}  \SCsym{;}  \Gamma_{{\mathrm{2}}}  \SCsym{;}  \SCnt{X}  \SCsym{;}  \Gamma_{{\mathrm{3}}}  \vdash_\mathcal{L}  \SCnt{B}}
        \end{math}
      \end{center}
      By assumption, $c(\Pi_1),c(\Pi_2)\leq |X|$. By induction on $\Pi_1$
      and $\pi$, there is a proof $\Pi'$ for sequent
      $\Gamma_{{\mathrm{1}}}  \SCsym{;}  \SCnt{A_{{\mathrm{1}}}}  \SCsym{;}  \SCnt{A_{{\mathrm{2}}}}  \SCsym{;}  \Gamma_{{\mathrm{2}}}  \SCsym{;}  \Phi  \SCsym{;}  \Gamma_{{\mathrm{3}}}  \vdash_\mathcal{L}  \SCnt{A}$ s.t. $c(\Pi') \leq |X|$. Therefore,
      the proof $\Pi$ can be constructed as follows with
      $c(\Pi) = c(\Pi') \leq |X|$.
      \begin{center}
        \scriptsize
        \begin{math}
          $$\mprset{flushleft}
          \inferrule* [right={\tiny tenL}] {
            {
              \begin{array}{c}
                \Pi' \\
                {\Gamma_{{\mathrm{1}}}  \SCsym{;}  \SCnt{A_{{\mathrm{1}}}}  \SCsym{;}  \SCnt{A_{{\mathrm{2}}}}  \SCsym{;}  \Gamma_{{\mathrm{2}}}  \SCsym{;}  \Phi  \SCsym{;}  \Gamma_{{\mathrm{3}}}  \vdash_\mathcal{L}  \SCnt{B}}
              \end{array}
            }
          }{\Gamma_{{\mathrm{1}}}  \SCsym{;}  \SCnt{A_{{\mathrm{1}}}}  \triangleright  \SCnt{A_{{\mathrm{2}}}}  \SCsym{;}  \Gamma_{{\mathrm{2}}}  \SCsym{;}  \Phi  \SCsym{;}  \Gamma_{{\mathrm{3}}}  \vdash_\mathcal{L}  \SCnt{B}}
        \end{math}
      \end{center}

\item Case 4:
      \begin{center}
        \scriptsize
        \begin{math}
          \begin{array}{c}
            \Pi_1 \\
            {\Delta  \vdash_\mathcal{L}  \SCnt{B}}
          \end{array}
        \end{math}
        \qquad\qquad
        $\Pi_2$:
        \begin{math}
          $$\mprset{flushleft}
          \inferrule* [right={\tiny tenL}] {
            {
              \begin{array}{c}
                \pi \\
                {\Gamma_{{\mathrm{1}}}  \SCsym{;}  \SCnt{A_{{\mathrm{1}}}}  \SCsym{;}  \SCnt{A_{{\mathrm{2}}}}  \SCsym{;}  \Gamma_{{\mathrm{2}}}  \SCsym{;}  \SCnt{B}  \SCsym{;}  \Gamma_{{\mathrm{3}}}  \vdash_\mathcal{L}  \SCnt{C}}
              \end{array}
            }
          }{\Gamma_{{\mathrm{1}}}  \SCsym{;}  \SCnt{A_{{\mathrm{1}}}}  \triangleright  \SCnt{A_{{\mathrm{2}}}}  \SCsym{;}  \Gamma_{{\mathrm{2}}}  \SCsym{;}  \SCnt{B}  \SCsym{;}  \Gamma_{{\mathrm{3}}}  \vdash_\mathcal{L}  \SCnt{C}}
        \end{math}
      \end{center}
      By assumption, $c(\Pi_1),c(\Pi_2)\leq |B|$. By induction on $\Pi_1$
      and $\pi$, there is a proof $\Pi'$ for sequent
      $\Gamma_{{\mathrm{1}}}  \SCsym{;}  \SCnt{A_{{\mathrm{1}}}}  \SCsym{;}  \SCnt{A_{{\mathrm{2}}}}  \SCsym{;}  \Gamma_{{\mathrm{2}}}  \SCsym{;}  \Delta  \SCsym{;}  \Gamma_{{\mathrm{3}}}  \vdash_\mathcal{L}  \SCnt{C}$ s.t. $c(\Pi') \leq |B|$. Therefore,
      the proof $\Pi$ can be constructed as follows with
      $c(\Pi) = c(\Pi') \leq |B|$.
      \begin{center}
        \scriptsize
        \begin{math}
          $$\mprset{flushleft}
          \inferrule* [right={\tiny tenL}] {
            {
              \begin{array}{c}
                \Pi' \\
                {\Gamma_{{\mathrm{1}}}  \SCsym{;}  \SCnt{A_{{\mathrm{1}}}}  \SCsym{;}  \SCnt{A_{{\mathrm{2}}}}  \SCsym{;}  \Gamma_{{\mathrm{2}}}  \SCsym{;}  \Delta  \SCsym{;}  \Gamma_{{\mathrm{3}}}  \vdash_\mathcal{L}  \SCnt{C}}
              \end{array}
            }
          }{\Gamma_{{\mathrm{1}}}  \SCsym{;}  \SCnt{A_{{\mathrm{1}}}}  \triangleright  \SCnt{A_{{\mathrm{2}}}}  \SCsym{;}  \Gamma_{{\mathrm{2}}}  \SCsym{;}  \Delta  \SCsym{;}  \Gamma_{{\mathrm{3}}}  \vdash_\mathcal{L}  \SCnt{C}}
        \end{math}
      \end{center}
\end{itemize}



\subsubsection{$\SCdruleTXXexName$}
\begin{itemize}
\item Case 1:
      \begin{center}
        \scriptsize
        \begin{math}
          \begin{array}{c}
            \Pi_1 \\
            {\Phi  \vdash_\mathcal{C}  \SCnt{X}}
          \end{array}
        \end{math}
        \qquad\qquad
        $\Pi_2$:
        \begin{math}
          $$\mprset{flushleft}
          \inferrule* [right={\tiny beta}] {
            {
              \begin{array}{c}
                \pi \\
                {\Psi_{{\mathrm{1}}}  \SCsym{,}  \SCnt{X}  \SCsym{,}  \Psi_{{\mathrm{2}}}  \SCsym{,}  \SCnt{Y_{{\mathrm{1}}}}  \SCsym{,}  \SCnt{Y_{{\mathrm{2}}}}  \SCsym{,}  \Psi_{{\mathrm{3}}}  \vdash_\mathcal{C}  \SCnt{Z}}
              \end{array}
            }
          }{\Psi_{{\mathrm{1}}}  \SCsym{,}  \SCnt{X}  \SCsym{,}  \Psi_{{\mathrm{2}}}  \SCsym{,}  \SCnt{Y_{{\mathrm{2}}}}  \SCsym{,}  \SCnt{Y_{{\mathrm{1}}}}  \SCsym{,}  \Psi_{{\mathrm{3}}}  \vdash_\mathcal{C}  \SCnt{Z}}
        \end{math}
      \end{center}
      By assumption, $c(\Pi_1),c(\Pi_2)\leq |X|$. By induction on $\Pi_1$
      and $\pi$, there is a proof $\Pi'$ for sequent
      $\Psi_{{\mathrm{1}}}  \SCsym{,}  \Phi  \SCsym{,}  \Psi_{{\mathrm{2}}}  \SCsym{,}  \SCnt{Y_{{\mathrm{1}}}}  \SCsym{,}  \SCnt{Y_{{\mathrm{2}}}}  \SCsym{,}  \Psi_{{\mathrm{3}}}  \vdash_\mathcal{C}  \SCnt{Z}$ s.t. $c(\Pi') \leq |X|$. Therefore,
      the proof $\Pi$ can be constructed as follows with
      $c(\Pi) = c(\Pi') \leq |X|$.
      \begin{center}
        \scriptsize
        \begin{math}
          $$\mprset{flushleft}
          \inferrule* [right={\tiny cut}] {
            {
              \begin{array}{cc}
                \Pi' \\
                {\Psi_{{\mathrm{1}}}  \SCsym{,}  \Phi  \SCsym{,}  \Psi_{{\mathrm{2}}}  \SCsym{,}  \SCnt{Y_{{\mathrm{1}}}}  \SCsym{,}  \SCnt{Y_{{\mathrm{2}}}}  \SCsym{,}  \Psi_{{\mathrm{3}}}  \vdash_\mathcal{C}  \SCnt{Z}}
              \end{array}
            }
          }{\Psi_{{\mathrm{1}}}  \SCsym{,}  \Phi  \SCsym{,}  \Psi_{{\mathrm{2}}}  \SCsym{,}  \SCnt{Y_{{\mathrm{2}}}}  \SCsym{,}  \SCnt{Y_{{\mathrm{1}}}}  \SCsym{,}  \Psi_{{\mathrm{3}}}  \vdash_\mathcal{C}  \SCnt{Z}}
        \end{math}
      \end{center}

\item Case 2:
      \begin{center}
        \scriptsize
        \begin{math}
          \begin{array}{c}
            \Pi_1 \\
            {\Phi  \vdash_\mathcal{C}  \SCnt{X}}
          \end{array}
        \end{math}
        \qquad\qquad
        $\Pi_2$:
        \begin{math}
          $$\mprset{flushleft}
          \inferrule* [right={\tiny beta}] {
            {
              \begin{array}{c}
                \pi \\
                {\Psi_{{\mathrm{1}}}  \SCsym{,}  \SCnt{Y_{{\mathrm{1}}}}  \SCsym{,}  \SCnt{Y_{{\mathrm{2}}}}  \SCsym{,}  \Psi_{{\mathrm{2}}}  \SCsym{,}  \SCnt{X}  \SCsym{,}  \Psi_{{\mathrm{3}}}  \vdash_\mathcal{C}  \SCnt{Z}}
              \end{array}
            }
          }{\Psi_{{\mathrm{1}}}  \SCsym{,}  \SCnt{X}  \SCsym{,}  \Psi_{{\mathrm{2}}}  \SCsym{,}  \SCnt{Y_{{\mathrm{2}}}}  \SCsym{,}  \SCnt{Y_{{\mathrm{1}}}}  \SCsym{,}  \Psi_{{\mathrm{3}}}  \vdash_\mathcal{C}  \SCnt{Z}}
        \end{math}
      \end{center}
      By assumption, $c(\Pi_1),c(\Pi_2)\leq |X|$. By induction on $\Pi_1$
      and $\pi$, there is a proof $\Pi'$ for sequent
      $\Psi_{{\mathrm{1}}}  \SCsym{,}  \SCnt{Y_{{\mathrm{1}}}}  \SCsym{,}  \SCnt{Y_{{\mathrm{2}}}}  \SCsym{,}  \Psi_{{\mathrm{2}}}  \SCsym{,}  \Phi  \SCsym{,}  \Psi_{{\mathrm{3}}}  \vdash_\mathcal{C}  \SCnt{Z}$ s.t. $c(\Pi') \leq |X|$. Therefore,
      the proof $\Pi$ can be constructed as follows with
      $c(\Pi) = c(\Pi') \leq |X|$.
      \begin{center}
        \scriptsize
        \begin{math}
          $$\mprset{flushleft}
          \inferrule* [right={\tiny cut}] {
            {
              \begin{array}{cc}
                \Pi' \\
                {\Psi_{{\mathrm{1}}}  \SCsym{,}  \SCnt{Y_{{\mathrm{1}}}}  \SCsym{,}  \SCnt{Y_{{\mathrm{2}}}}  \SCsym{,}  \Psi_{{\mathrm{2}}}  \SCsym{,}  \Phi  \SCsym{,}  \Psi_{{\mathrm{3}}}  \vdash_\mathcal{C}  \SCnt{Z}}
              \end{array}
            }
          }{\Psi_{{\mathrm{1}}}  \SCsym{,}  \SCnt{Y_{{\mathrm{2}}}}  \SCsym{,}  \SCnt{Y_{{\mathrm{1}}}}  \SCsym{,}  \Psi_{{\mathrm{2}}}  \SCsym{,}  \Phi  \SCsym{,}  \Psi_{{\mathrm{3}}}  \vdash_\mathcal{C}  \SCnt{Z}}
        \end{math}
      \end{center}
\end{itemize}


\subsubsection{$\SCdruleSXXexName$}
\begin{itemize}
\item Case 1:
      \begin{center}
        \scriptsize
        \begin{math}
          \begin{array}{c}
            \Pi_1 \\
            {\Phi  \vdash_\mathcal{C}  \SCnt{X}}
          \end{array}
        \end{math}
        \qquad\qquad
        $\Pi_2$:
        \begin{math}
          $$\mprset{flushleft}
          \inferrule* [right={\tiny beta}] {
            {
              \begin{array}{c}
                \pi \\
                {\Gamma_{{\mathrm{1}}}  \SCsym{;}  \SCnt{X}  \SCsym{;}  \Gamma_{{\mathrm{2}}}  \SCsym{;}  \SCnt{Y_{{\mathrm{1}}}}  \SCsym{;}  \SCnt{Y_{{\mathrm{2}}}}  \SCsym{;}  \Gamma_{{\mathrm{3}}}  \vdash_\mathcal{L}  \SCnt{A}}
              \end{array}
            }
          }{\Gamma_{{\mathrm{1}}}  \SCsym{;}  \SCnt{X}  \SCsym{;}  \Gamma_{{\mathrm{2}}}  \SCsym{;}  \SCnt{Y_{{\mathrm{2}}}}  \SCsym{;}  \SCnt{Y_{{\mathrm{1}}}}  \SCsym{;}  \Gamma_{{\mathrm{3}}}  \vdash_\mathcal{L}  \SCnt{A}}
        \end{math}
      \end{center}
      By assumption, $c(\Pi_1),c(\Pi_2)\leq |X|$. By induction on $\Pi_1$
      and $\pi$, there is a proof $\Pi'$ for sequent
      $\Gamma_{{\mathrm{1}}}  \SCsym{;}  \Phi  \SCsym{;}  \Gamma_{{\mathrm{2}}}  \SCsym{;}  \SCnt{Y_{{\mathrm{1}}}}  \SCsym{;}  \SCnt{Y_{{\mathrm{2}}}}  \SCsym{;}  \Gamma_{{\mathrm{3}}}  \vdash_\mathcal{L}  \SCnt{A}$ s.t. $c(\Pi') \leq |X|$. Therefore,
      the proof $\Pi$ can be constructed as follows with
      $c(\Pi) = c(\Pi') \leq |X|$.
      \begin{center}
        \scriptsize
        \begin{math}
          $$\mprset{flushleft}
          \inferrule* [right={\tiny cut}] {
            {
              \begin{array}{cc}
                \Pi' \\
                {\Gamma_{{\mathrm{1}}}  \SCsym{;}  \Phi  \SCsym{;}  \Gamma_{{\mathrm{2}}}  \SCsym{;}  \SCnt{Y_{{\mathrm{1}}}}  \SCsym{;}  \SCnt{Y_{{\mathrm{2}}}}  \SCsym{;}  \Gamma_{{\mathrm{3}}}  \vdash_\mathcal{L}  \SCnt{A}}
              \end{array}
            }
          }{\Gamma_{{\mathrm{1}}}  \SCsym{;}  \Phi  \SCsym{;}  \Gamma_{{\mathrm{2}}}  \SCsym{;}  \SCnt{Y_{{\mathrm{2}}}}  \SCsym{;}  \SCnt{Y_{{\mathrm{1}}}}  \SCsym{;}  \Gamma_{{\mathrm{3}}}  \vdash_\mathcal{L}  \SCnt{A}}
        \end{math}
      \end{center}

\item Case 2:
      \begin{center}
        \scriptsize
        \begin{math}
          \begin{array}{c}
            \Pi_1 \\
            {\Delta  \vdash_\mathcal{L}  \SCnt{B}}
          \end{array}
        \end{math}
        \qquad\qquad
        $\Pi_2$:
        \begin{math}
          $$\mprset{flushleft}
          \inferrule* [right={\tiny beta}] {
            {
              \begin{array}{c}
                \pi \\
                {\Gamma_{{\mathrm{1}}}  \SCsym{;}  \SCnt{B}  \SCsym{;}  \Gamma_{{\mathrm{2}}}  \SCsym{;}  \SCnt{Y_{{\mathrm{1}}}}  \SCsym{;}  \SCnt{Y_{{\mathrm{2}}}}  \SCsym{;}  \Gamma_{{\mathrm{3}}}  \vdash_\mathcal{L}  \SCnt{A}}
              \end{array}
            }
          }{\Gamma_{{\mathrm{1}}}  \SCsym{;}  \SCnt{B}  \SCsym{;}  \Gamma_{{\mathrm{2}}}  \SCsym{;}  \SCnt{Y_{{\mathrm{2}}}}  \SCsym{;}  \SCnt{Y_{{\mathrm{1}}}}  \SCsym{;}  \Gamma_{{\mathrm{3}}}  \vdash_\mathcal{L}  \SCnt{A}}
        \end{math}
      \end{center}
      By assumption, $c(\Pi_1),c(\Pi_2)\leq |X|$. By induction on $\Pi_1$
      and $\pi$, there is a proof $\Pi'$ for sequent
      $\Gamma_{{\mathrm{1}}}  \SCsym{;}  \Delta  \SCsym{;}  \Gamma_{{\mathrm{2}}}  \SCsym{;}  \SCnt{Y_{{\mathrm{1}}}}  \SCsym{;}  \SCnt{Y_{{\mathrm{2}}}}  \SCsym{;}  \Gamma_{{\mathrm{3}}}  \vdash_\mathcal{L}  \SCnt{A}$ s.t. $c(\Pi') \leq |X|$. Therefore,
      the proof $\Pi$ can be constructed as follows with
      $c(\Pi) = c(\Pi') \leq |X|$.
      \begin{center}
        \scriptsize
        \begin{math}
          $$\mprset{flushleft}
          \inferrule* [right={\tiny cut}] {
            {
              \begin{array}{cc}
                \Pi' \\
                {\Gamma_{{\mathrm{1}}}  \SCsym{;}  \Delta  \SCsym{;}  \Gamma_{{\mathrm{2}}}  \SCsym{;}  \SCnt{Y_{{\mathrm{1}}}}  \SCsym{;}  \SCnt{Y_{{\mathrm{2}}}}  \SCsym{;}  \Gamma_{{\mathrm{3}}}  \vdash_\mathcal{L}  \SCnt{A}}
              \end{array}
            }
          }{\Gamma_{{\mathrm{1}}}  \SCsym{;}  \Delta  \SCsym{;}  \Gamma_{{\mathrm{2}}}  \SCsym{;}  \SCnt{Y_{{\mathrm{2}}}}  \SCsym{;}  \SCnt{Y_{{\mathrm{1}}}}  \SCsym{;}  \Gamma_{{\mathrm{3}}}  \vdash_\mathcal{L}  \SCnt{A}}
        \end{math}
      \end{center}

\item Case 3:
      \begin{center}
        \scriptsize
        \begin{math}
          \begin{array}{c}
            \Pi_1 \\
            {\Phi  \vdash_\mathcal{C}  \SCnt{X}}
          \end{array}
        \end{math}
        \qquad\qquad
        $\Pi_2$:
        \begin{math}
          $$\mprset{flushleft}
          \inferrule* [right={\tiny beta}] {
            {
              \begin{array}{c}
                \pi \\
                {\Gamma_{{\mathrm{1}}}  \SCsym{;}  \SCnt{Y_{{\mathrm{1}}}}  \SCsym{;}  \SCnt{Y_{{\mathrm{2}}}}  \SCsym{;}  \Gamma_{{\mathrm{2}}}  \SCsym{;}  \SCnt{X}  \SCsym{;}  \Gamma_{{\mathrm{3}}}  \vdash_\mathcal{L}  \SCnt{A}}
              \end{array}
            }
          }{\Gamma_{{\mathrm{1}}}  \SCsym{;}  \SCnt{X}  \SCsym{;}  \Gamma_{{\mathrm{2}}}  \SCsym{;}  \SCnt{Y_{{\mathrm{2}}}}  \SCsym{;}  \SCnt{Y_{{\mathrm{1}}}}  \SCsym{;}  \Gamma_{{\mathrm{3}}}  \vdash_\mathcal{L}  \SCnt{A}}
        \end{math}
      \end{center}
      By assumption, $c(\Pi_1),c(\Pi_2)\leq |X|$. By induction on $\Pi_1$
      and $\pi$, there is a proof $\Pi'$ for sequent
      $\Gamma_{{\mathrm{1}}}  \SCsym{;}  \SCnt{Y_{{\mathrm{1}}}}  \SCsym{;}  \SCnt{Y_{{\mathrm{2}}}}  \SCsym{;}  \Gamma_{{\mathrm{2}}}  \SCsym{;}  \Phi  \SCsym{;}  \Gamma_{{\mathrm{3}}}  \vdash_\mathcal{L}  \SCnt{A}$ s.t. $c(\Pi') \leq |X|$. Therefore,
      the proof $\Pi$ can be constructed as follows with
      $c(\Pi) = c(\Pi') \leq |X|$.
      \begin{center}
        \scriptsize
        \begin{math}
          $$\mprset{flushleft}
          \inferrule* [right={\tiny cut}] {
            {
              \begin{array}{cc}
                \Pi' \\
                {\Gamma_{{\mathrm{1}}}  \SCsym{;}  \SCnt{Y_{{\mathrm{1}}}}  \SCsym{;}  \SCnt{Y_{{\mathrm{2}}}}  \SCsym{;}  \Gamma_{{\mathrm{2}}}  \SCsym{;}  \Phi  \SCsym{;}  \Gamma_{{\mathrm{3}}}  \vdash_\mathcal{L}  \SCnt{A}}
              \end{array}
            }
          }{\Gamma_{{\mathrm{1}}}  \SCsym{;}  \SCnt{Y_{{\mathrm{2}}}}  \SCsym{;}  \SCnt{Y_{{\mathrm{1}}}}  \SCsym{;}  \Gamma_{{\mathrm{2}}}  \SCsym{;}  \Phi  \SCsym{;}  \Gamma_{{\mathrm{3}}}  \vdash_\mathcal{L}  \SCnt{A}}
        \end{math}
      \end{center}

\item Case 4:
      \begin{center}
        \scriptsize
        \begin{math}
          \begin{array}{c}
            \Pi_1 \\
            {\Delta  \vdash_\mathcal{L}  \SCnt{B}}
          \end{array}
        \end{math}
        \qquad\qquad
        $\Pi_2$:
        \begin{math}
          $$\mprset{flushleft}
          \inferrule* [right={\tiny beta}] {
            {
              \begin{array}{c}
                \pi \\
                {\Gamma_{{\mathrm{1}}}  \SCsym{;}  \SCnt{Y_{{\mathrm{1}}}}  \SCsym{;}  \SCnt{Y_{{\mathrm{2}}}}  \SCsym{;}  \Gamma_{{\mathrm{2}}}  \SCsym{;}  \SCnt{B}  \SCsym{;}  \Gamma_{{\mathrm{3}}}  \vdash_\mathcal{L}  \SCnt{A}}
              \end{array}
            }
          }{\Gamma_{{\mathrm{1}}}  \SCsym{;}  \SCnt{Y_{{\mathrm{2}}}}  \SCsym{;}  \SCnt{Y_{{\mathrm{1}}}}  \SCsym{;}  \Gamma_{{\mathrm{2}}}  \SCsym{;}  \SCnt{B}  \SCsym{;}  \Gamma_{{\mathrm{3}}}  \vdash_\mathcal{L}  \SCnt{A}}
        \end{math}
      \end{center}
      By assumption, $c(\Pi_1),c(\Pi_2)\leq |X|$. By induction on $\Pi_1$
      and $\pi$, there is a proof $\Pi'$ for sequent
      $\Gamma_{{\mathrm{1}}}  \SCsym{;}  \SCnt{Y_{{\mathrm{1}}}}  \SCsym{;}  \SCnt{Y_{{\mathrm{2}}}}  \SCsym{;}  \Gamma_{{\mathrm{2}}}  \SCsym{;}  \Delta  \SCsym{;}  \Gamma_{{\mathrm{3}}}  \vdash_\mathcal{L}  \SCnt{A}$ s.t. $c(\Pi') \leq |X|$. Therefore,
      the proof $\Pi$ can be constructed as follows with
      $c(\Pi) = c(\Pi') \leq |X|$.
      \begin{center}
        \scriptsize
        \begin{math}
          $$\mprset{flushleft}
          \inferrule* [right={\tiny cut}] {
            {
              \begin{array}{cc}
                \Pi' \\
                {\Gamma_{{\mathrm{1}}}  \SCsym{;}  \SCnt{Y_{{\mathrm{1}}}}  \SCsym{;}  \SCnt{Y_{{\mathrm{2}}}}  \SCsym{;}  \Gamma_{{\mathrm{2}}}  \SCsym{;}  \Delta  \SCsym{;}  \Gamma_{{\mathrm{3}}}  \vdash_\mathcal{L}  \SCnt{A}}
              \end{array}
            }
          }{\Gamma_{{\mathrm{1}}}  \SCsym{;}  \SCnt{Y_{{\mathrm{2}}}}  \SCsym{;}  \SCnt{Y_{{\mathrm{1}}}}  \SCsym{;}  \Gamma_{{\mathrm{2}}}  \SCsym{;}  \Delta  \SCsym{;}  \Gamma_{{\mathrm{3}}}  \vdash_\mathcal{L}  \SCnt{A}}
        \end{math}
      \end{center}
\end{itemize}



\subsubsection{Left introduction of the commutative unit $ \mathsf{Unit} $ (with low priority)}
\begin{itemize}
\item Case 1:
      \begin{center}
        \scriptsize
        \begin{math}
          \begin{array}{c}
            \Pi_1 \\
            {\Psi  \vdash_\mathcal{C}  \SCnt{X}}
          \end{array}
        \end{math}
        \qquad\qquad
        $\Pi_2$:
        \begin{math}
          $$\mprset{flushleft}
          \inferrule* [right={\tiny unitL}] {
            {
              \begin{array}{c}
                \pi \\
                {\Phi_{{\mathrm{1}}}  \SCsym{,}  \Phi_{{\mathrm{2}}}  \SCsym{,}  \SCnt{X}  \SCsym{,}  \Phi_{{\mathrm{3}}}  \vdash_\mathcal{C}  \SCnt{Y}}
              \end{array}
            }
          }{\Phi_{{\mathrm{1}}}  \SCsym{,}   \mathsf{Unit}   \SCsym{,}  \Phi_{{\mathrm{2}}}  \SCsym{,}  \SCnt{X}  \SCsym{,}  \Phi_{{\mathrm{3}}}  \vdash_\mathcal{C}  \SCnt{Y}}
        \end{math}
      \end{center}
      By assumption, $c(\Pi_1),c(\Pi_2)\leq |X|$. By induction on $\Pi_1$
      and $\pi$, there is a proof $\Pi'$ for sequent
      $\Phi_{{\mathrm{1}}}  \SCsym{,}  \Phi_{{\mathrm{2}}}  \SCsym{,}  \Psi  \SCsym{,}  \Phi_{{\mathrm{3}}}  \vdash_\mathcal{C}  \SCnt{Y}$
      s.t. $c(\Pi') \leq |X|$. Therefore, the proof $\Pi$ can be
      constructed as follows with $c(\Pi) = c(\Pi') \leq |X|$.
      \begin{center}
        \scriptsize
        \begin{math}
          $$\mprset{flushleft}
          \inferrule* [right={\tiny unitL}] {
            {
              \begin{array}{c}
                \Pi' \\
                {\Phi_{{\mathrm{1}}}  \SCsym{,}  \Phi_{{\mathrm{2}}}  \SCsym{,}  \Psi  \SCsym{,}  \Phi_{{\mathrm{3}}}  \vdash_\mathcal{C}  \SCnt{Y}}
              \end{array}
            }
          }{\Phi_{{\mathrm{1}}}  \SCsym{,}   \mathsf{Unit}   \SCsym{,}  \Phi_{{\mathrm{2}}}  \SCsym{,}  \Psi  \SCsym{,}  \Phi_{{\mathrm{3}}}  \vdash_\mathcal{C}  \SCnt{Y}}
        \end{math}
      \end{center}

\item Case 2:
      \begin{center}
        \scriptsize
        \begin{math}
          \begin{array}{c}
            \Pi_1 \\
            {\Phi  \vdash_\mathcal{C}  \SCnt{X}}
          \end{array}
        \end{math}
        \qquad\qquad
        $\Pi_2$:
        \begin{math}
          $$\mprset{flushleft}
          \inferrule* [right={\tiny unitL1}] {
            {
              \begin{array}{c}
                \pi \\
                {\Gamma_{{\mathrm{1}}}  \SCsym{;}  \Gamma_{{\mathrm{2}}}  \SCsym{;}  \SCnt{X}  \SCsym{;}  \Gamma_{{\mathrm{3}}}  \vdash_\mathcal{L}  \SCnt{A}}
              \end{array}
            }
          }{\Gamma_{{\mathrm{1}}}  \SCsym{;}   \mathsf{Unit}   \SCsym{;}  \Gamma_{{\mathrm{2}}}  \SCsym{;}  \SCnt{X}  \SCsym{;}  \Gamma_{{\mathrm{3}}}  \vdash_\mathcal{L}  \SCnt{A}}
        \end{math}
      \end{center}
      By assumption, $c(\Pi_1),c(\Pi_2)\leq |X|$. By induction on $\Pi_1$
      and $\pi$, there is a proof $\Pi'$ for sequent
      $\Gamma_{{\mathrm{1}}}  \SCsym{;}  \Gamma_{{\mathrm{2}}}  \SCsym{;}  \Phi  \SCsym{;}  \Gamma_{{\mathrm{2}}}  \vdash_\mathcal{L}  \SCnt{A}$
      s.t. $c(\Pi') \leq |X|$. Therefore, the proof $\Pi$ can be
      constructed as follows with $c(\Pi) = c(\Pi') \leq |X|$.
      \begin{center}
        \scriptsize
        \begin{math}
          $$\mprset{flushleft}
          \inferrule* [right={\tiny unitL1}] {
            {
              \begin{array}{c}
                \Pi' \\
                {\Gamma_{{\mathrm{1}}}  \SCsym{;}  \Gamma_{{\mathrm{2}}}  \SCsym{;}  \Phi  \SCsym{;}  \Gamma_{{\mathrm{3}}}  \vdash_\mathcal{L}  \SCnt{A}}
              \end{array}
            }
          }{\Gamma_{{\mathrm{1}}}  \SCsym{;}   \mathsf{Unit}   \SCsym{;}  \Gamma_{{\mathrm{2}}}  \SCsym{;}  \Phi  \SCsym{;}  \Gamma_{{\mathrm{3}}}  \vdash_\mathcal{L}  \SCnt{A}}
        \end{math}
      \end{center}

\item Case 3:
      \begin{center}
        \scriptsize
        \begin{math}
          \begin{array}{c}
            \Pi_1 \\
            {\Delta  \vdash_\mathcal{L}  \SCnt{B}}
          \end{array}
        \end{math}
        \qquad\qquad
        $\Pi_2$:
        \begin{math}
          $$\mprset{flushleft}
          \inferrule* [right={\tiny unitL1}] {
            {
              \begin{array}{c}
                \pi \\
                {\Gamma_{{\mathrm{1}}}  \SCsym{;}  \Gamma_{{\mathrm{2}}}  \SCsym{;}  \SCnt{B}  \SCsym{;}  \Gamma_{{\mathrm{3}}}  \vdash_\mathcal{L}  \SCnt{A}}
              \end{array}
            }
          }{\Gamma_{{\mathrm{1}}}  \SCsym{;}   \mathsf{Unit}   \SCsym{;}  \Gamma_{{\mathrm{2}}}  \SCsym{;}  \SCnt{B}  \SCsym{;}  \Gamma_{{\mathrm{3}}}  \vdash_\mathcal{L}  \SCnt{A}}
        \end{math}
      \end{center}
      By assumption, $c(\Pi_1),c(\Pi_2)\leq |B|$. By induction on $\Pi_1$
      and $\pi$, there is a proof $\Pi'$ for sequent
      $\Gamma_{{\mathrm{1}}}  \SCsym{;}  \Gamma_{{\mathrm{2}}}  \SCsym{;}  \Delta  \SCsym{;}  \Gamma_{{\mathrm{3}}}  \vdash_\mathcal{L}  \SCnt{A}$
      s.t. $c(\Pi') \leq |B|$. Therefore, the proof $\Pi$ can be
      constructed as follows with $c(\Pi) = c(\Pi') \leq |B|$.
      \begin{center}
        \scriptsize
        \begin{math}
          $$\mprset{flushleft}
          \inferrule* [right={\tiny unitL1}] {
            {
              \begin{array}{c}
                \Pi' \\
                {\Gamma_{{\mathrm{1}}}  \SCsym{;}  \Gamma_{{\mathrm{2}}}  \SCsym{;}  \Delta  \SCsym{;}  \Gamma_{{\mathrm{3}}}  \vdash_\mathcal{L}  \SCnt{A}}
              \end{array}
            }
          }{\Gamma_{{\mathrm{1}}}  \SCsym{;}   \mathsf{Unit}   \SCsym{;}  \Gamma_{{\mathrm{2}}}  \SCsym{;}  \Delta  \SCsym{;}  \Gamma_{{\mathrm{3}}}  \vdash_\mathcal{L}  \SCnt{A}}
        \end{math}
      \end{center}
\end{itemize}

\subsubsection{Left introduction of the non-commutative unit $ \mathsf{Unit} $ (with low priority)}
\begin{itemize}
\item Case 1:
      \begin{center}
        \scriptsize
        \begin{math}
          \begin{array}{c}
            \Pi_1 \\
            {\Phi  \vdash_\mathcal{C}  \SCnt{X}}
          \end{array}
        \end{math}
        \qquad\qquad
        $\Pi_2$:
        \begin{math}
          $$\mprset{flushleft}
          \inferrule* [right={\tiny unitL2}] {
            {
              \begin{array}{c}
                \pi \\
                {\Gamma_{{\mathrm{1}}}  \SCsym{;}  \Gamma_{{\mathrm{2}}}  \SCsym{;}  \SCnt{X}  \SCsym{;}  \Gamma_{{\mathrm{3}}}  \vdash_\mathcal{L}  \SCnt{A}}
              \end{array}
            }
          }{\Gamma_{{\mathrm{1}}}  \SCsym{;}   \mathsf{Unit}   \SCsym{;}  \Gamma_{{\mathrm{2}}}  \SCsym{;}  \SCnt{X}  \SCsym{;}  \Gamma_{{\mathrm{3}}}  \vdash_\mathcal{L}  \SCnt{A}}
        \end{math}
      \end{center}
      By assumption, $c(\Pi_1),c(\Pi_2)\leq |X|$. By induction on $\Pi_1$
      and $\pi$, there is a proof $\Pi'$ for sequent
      $\Gamma_{{\mathrm{1}}}  \SCsym{;}  \Gamma_{{\mathrm{2}}}  \SCsym{;}  \Phi  \SCsym{;}  \Gamma_{{\mathrm{3}}}  \vdash_\mathcal{L}  \SCnt{A}$
      s.t. $c(\Pi') \leq |X|$. Therefore, the proof $\Pi$ can be
      constructed as follows with $c(\Pi) = c(\Pi') \leq |X|$.
      \begin{center}
        \scriptsize
        \begin{math}
          $$\mprset{flushleft}
          \inferrule* [right={\tiny unitL2}] {
            {
              \begin{array}{c}
                \Pi' \\
                {\Gamma_{{\mathrm{1}}}  \SCsym{;}  \Gamma_{{\mathrm{2}}}  \SCsym{;}  \Phi  \SCsym{;}  \Gamma_{{\mathrm{3}}}  \vdash_\mathcal{L}  \SCnt{A}}
              \end{array}
            }
          }{\Gamma_{{\mathrm{1}}}  \SCsym{;}   \mathsf{Unit}   \SCsym{;}  \Gamma_{{\mathrm{2}}}  \SCsym{;}  \Phi  \SCsym{;}  \Gamma_{{\mathrm{3}}}  \vdash_\mathcal{L}  \SCnt{A}}
        \end{math}
      \end{center}

\item Case 2:
      \begin{center}
        \scriptsize
        \begin{math}
          \begin{array}{c}
            \Pi_1 \\
            {\Delta  \vdash_\mathcal{L}  \SCnt{B}}
          \end{array}
        \end{math}
        \qquad\qquad
        $\Pi_2$:
        \begin{math}
          $$\mprset{flushleft}
          \inferrule* [right={\tiny unitL2}] {
            {
              \begin{array}{c}
                \pi \\
                {\Gamma_{{\mathrm{1}}}  \SCsym{;}  \Gamma_{{\mathrm{2}}}  \SCsym{;}  \SCnt{B}  \SCsym{;}  \Gamma_{{\mathrm{3}}}  \vdash_\mathcal{L}  \SCnt{A}}
              \end{array}
            }
          }{\Gamma_{{\mathrm{1}}}  \SCsym{;}   \mathsf{Unit}   \SCsym{;}  \Gamma_{{\mathrm{2}}}  \SCsym{;}  \SCnt{B}  \SCsym{;}  \Gamma_{{\mathrm{3}}}  \vdash_\mathcal{L}  \SCnt{A}}
        \end{math}
      \end{center}
      By assumption, $c(\Pi_1),c(\Pi_2)\leq |B|$. By induction on $\Pi_1$
      and $\pi$, there is a proof $\Pi'$ for sequent
      $\Gamma_{{\mathrm{1}}}  \SCsym{;}  \Gamma_{{\mathrm{2}}}  \SCsym{;}  \Delta  \SCsym{;}  \Gamma_{{\mathrm{3}}}  \vdash_\mathcal{L}  \SCnt{A}$
      s.t. $c(\Pi') \leq |B|$. Therefore, the proof $\Pi$ can be
      constructed as follows with $c(\Pi) = c(\Pi') \leq |B|$.
      \begin{center}
        \scriptsize
        \begin{math}
          $$\mprset{flushleft}
          \inferrule* [right={\tiny unitL2}] {
            {
              \begin{array}{c}
                \Pi' \\
                {\Gamma_{{\mathrm{1}}}  \SCsym{;}  \Gamma_{{\mathrm{2}}}  \SCsym{;}  \Delta  \SCsym{;}  \Gamma_{{\mathrm{3}}}  \vdash_\mathcal{L}  \SCnt{A}}
              \end{array}
            }
          }{\Gamma_{{\mathrm{1}}}  \SCsym{;}   \mathsf{Unit}   \SCsym{;}  \Gamma_{{\mathrm{2}}}  \SCsym{;}  \Delta  \SCsym{;}  \Gamma_{{\mathrm{3}}}  \vdash_\mathcal{L}  \SCnt{A}}
        \end{math}
      \end{center}
\end{itemize}



\subsubsection{Right introduction of the commutative implication $\multimap$ (with low priority)}
\begin{center}
  \scriptsize
  \begin{math}
    \begin{array}{c}
      \Pi_1 \\
      {\Phi  \vdash_\mathcal{C}  \SCnt{X}}
    \end{array}
  \end{math}
  \qquad\qquad
  $\Pi_2$:
  \begin{math}
    $$\mprset{flushleft}
    \inferrule* [right={\tiny impR}] {
      {
        \begin{array}{c}
          \pi \\
          {\Psi_{{\mathrm{1}}}  \SCsym{,}  \SCnt{X}  \SCsym{,}  \Psi_{{\mathrm{2}}}  \SCsym{,}  \SCnt{Y_{{\mathrm{1}}}}  \vdash_\mathcal{C}  \SCnt{Y_{{\mathrm{2}}}}}
        \end{array}
      }
    }{\Psi_{{\mathrm{1}}}  \SCsym{,}  \SCnt{X}  \SCsym{,}  \Psi_{{\mathrm{2}}}  \vdash_\mathcal{C}  \SCnt{Y_{{\mathrm{1}}}}  \multimap  \SCnt{Y_{{\mathrm{2}}}}}
  \end{math}
\end{center}
By assumption, $c(\Pi_1),c(\Pi_2)\leq |X|$. By induction on $\Pi_1$
and $\pi$, there is a proof $\Pi'$ for sequent
$\Psi_{{\mathrm{1}}}  \SCsym{,}  \Phi  \SCsym{,}  \Psi_{{\mathrm{2}}}  \SCsym{,}  \SCnt{Y_{{\mathrm{1}}}}  \vdash_\mathcal{C}  \SCnt{Y_{{\mathrm{2}}}}$ s.t. $c(\Pi') \leq |X|$. Therefore, the
proof $\Pi$ can be constructed as follows with
$c(\Pi) = c(\Pi') \leq |X|$.
\begin{center}
  \scriptsize
  \begin{math}
    $$\mprset{flushleft}
    \inferrule* [right={\tiny impR}] {
      {
        \begin{array}{c}
          \Pi' \\
          {\Psi_{{\mathrm{1}}}  \SCsym{,}  \Phi  \SCsym{,}  \Psi_{{\mathrm{2}}}  \SCsym{,}  \SCnt{Y_{{\mathrm{1}}}}  \vdash_\mathcal{C}  \SCnt{Y_{{\mathrm{2}}}}}
        \end{array}
      }
    }{\Psi_{{\mathrm{1}}}  \SCsym{,}  \Phi  \SCsym{,}  \Psi_{{\mathrm{2}}}  \vdash_\mathcal{C}  \SCnt{Y_{{\mathrm{1}}}}  \multimap  \SCnt{Y_{{\mathrm{2}}}}}
  \end{math}
\end{center}



\subsubsection{Right introduction of the non-commutative left implication $\lto$ (with low priority)}
\begin{itemize}
\item Case 1:
      \begin{center}
        \scriptsize
        \begin{math}
          \begin{array}{c}
            \Pi_1 \\
            {\Phi  \vdash_\mathcal{C}  \SCnt{X}}
          \end{array}
        \end{math}
        \qquad\qquad
        $\Pi_2$:
        \begin{math}
          $$\mprset{flushleft}
          \inferrule* [right={\tiny impR}] {
            {
              \begin{array}{c}
                \pi \\
                {\Gamma_{{\mathrm{1}}}  \SCsym{;}  \SCnt{X}  \SCsym{;}  \Gamma_{{\mathrm{2}}}  \SCsym{;}  \SCnt{A}  \vdash_\mathcal{L}  \SCnt{B}}
              \end{array}
            }
          }{\Gamma_{{\mathrm{1}}}  \SCsym{;}  \SCnt{X}  \SCsym{;}  \Gamma_{{\mathrm{2}}}  \vdash_\mathcal{L}  \SCnt{A}  \rightharpoonup  \SCnt{B}}
        \end{math}
      \end{center}
      By assumption, $c(\Pi_1),c(\Pi_2)\leq |X|$. By induction on $\Pi_1$
      and $\pi$, there is a proof $\Pi'$ for sequent
      $\Gamma_{{\mathrm{1}}}  \SCsym{;}  \Phi  \SCsym{;}  \Gamma_{{\mathrm{2}}}  \SCsym{;}  \SCnt{A}  \vdash_\mathcal{L}  \SCnt{B}$ s.t. $c(\Pi') \leq |X|$. Therefore, the
      proof $\Pi$ can be constructed as follows with
      $c(\Pi) = c(\Pi') \leq |X|$.
      \begin{center}
        \scriptsize
        \begin{math}
          $$\mprset{flushleft}
          \inferrule* [right={\tiny implR}] {
            {
              \begin{array}{c}
                \Pi' \\
                {\Gamma_{{\mathrm{1}}}  \SCsym{;}  \Phi  \SCsym{;}  \Gamma_{{\mathrm{2}}}  \SCsym{;}  \SCnt{A}  \vdash_\mathcal{L}  \SCnt{B}}
              \end{array}
            }
          }{\Gamma_{{\mathrm{1}}}  \SCsym{;}  \Phi  \SCsym{;}  \Gamma_{{\mathrm{2}}}  \vdash_\mathcal{L}  \SCnt{A}  \rightharpoonup  \SCnt{B}}
        \end{math}
      \end{center}

\item Case 2:
      \begin{center}
        \scriptsize
        \begin{math}
          \begin{array}{c}
            \Pi_1 \\
            {\Delta  \vdash_\mathcal{L}  \SCnt{C}}
          \end{array}
        \end{math}
        \qquad\qquad
        $\Pi_2$:
        \begin{math}
          $$\mprset{flushleft}
          \inferrule* [right={\tiny impR}] {
            {
              \begin{array}{c}
                \pi \\
                {\Gamma_{{\mathrm{1}}}  \SCsym{;}  \SCnt{C}  \SCsym{;}  \Gamma_{{\mathrm{2}}}  \SCsym{;}  \SCnt{A}  \vdash_\mathcal{L}  \SCnt{B}}
              \end{array}
            }
          }{\Gamma_{{\mathrm{1}}}  \SCsym{;}  \SCnt{C}  \SCsym{;}  \Gamma_{{\mathrm{2}}}  \vdash_\mathcal{L}  \SCnt{A}  \rightharpoonup  \SCnt{B}}
        \end{math}
      \end{center}
      By assumption, $c(\Pi_1),c(\Pi_2)\leq |C|$. By induction on $\Pi_1$
      and $\pi$, there is a proof $\Pi'$ for sequent
      $\Gamma_{{\mathrm{1}}}  \SCsym{;}  \Delta  \SCsym{;}  \Gamma_{{\mathrm{2}}}  \SCsym{;}  \SCnt{A}  \vdash_\mathcal{L}  \SCnt{B}$ s.t. $c(\Pi') \leq |C|$. Therefore, the
      proof $\Pi$ can be constructed as follows with
      $c(\Pi) = c(\Pi') \leq |C|$.
      \begin{center}
        \scriptsize
        \begin{math}
          $$\mprset{flushleft}
          \inferrule* [right={\tiny implR}] {
            {
              \begin{array}{c}
                \Pi' \\
                {\Gamma_{{\mathrm{1}}}  \SCsym{;}  \Delta  \SCsym{;}  \Gamma_{{\mathrm{2}}}  \SCsym{;}  \SCnt{A}  \vdash_\mathcal{L}  \SCnt{B}}
              \end{array}
            }
          }{\Gamma_{{\mathrm{1}}}  \SCsym{;}  \Delta  \SCsym{;}  \Gamma_{{\mathrm{2}}}  \vdash_\mathcal{L}  \SCnt{A}  \rightharpoonup  \SCnt{B}}
        \end{math}
      \end{center}
\end{itemize}




\subsubsection{Right introduction of the non-commutative right implication $\rto$ (with low priority)}
\begin{itemize}
\item Case 1:
      \begin{center}
        \scriptsize
        \begin{math}
          \begin{array}{c}
            \Pi_1 \\
            {\Phi  \vdash_\mathcal{C}  \SCnt{X}}
          \end{array}
        \end{math}
        \qquad\qquad
        $\Pi_2$:
        \begin{math}
          $$\mprset{flushleft}
          \inferrule* [right={\tiny impL}] {
            {
              \begin{array}{c}
                \pi \\
                {\SCnt{A}  \SCsym{;}  \Gamma_{{\mathrm{1}}}  \SCsym{;}  \SCnt{X}  \SCsym{;}  \Gamma_{{\mathrm{2}}}  \vdash_\mathcal{L}  \SCnt{B}}
              \end{array}
            }
          }{\Gamma_{{\mathrm{1}}}  \SCsym{;}  \SCnt{X}  \SCsym{;}  \Gamma_{{\mathrm{2}}}  \vdash_\mathcal{L}  \SCnt{B}  \leftharpoonup  \SCnt{A}}
        \end{math}
      \end{center}
      By assumption, $c(\Pi_1),c(\Pi_2)\leq |X|$. By induction on $\Pi_1$
      and $\pi$, there is a proof $\Pi'$ for sequent
      $\SCnt{A}  \SCsym{;}  \Gamma_{{\mathrm{1}}}  \SCsym{;}  \Phi  \SCsym{;}  \Gamma_{{\mathrm{2}}}  \vdash_\mathcal{L}  \SCnt{B}$ s.t. $c(\Pi') \leq |X|$. Therefore, the
      proof $\Pi$ can be constructed as follows with
      $c(\Pi) = c(\Pi') \leq |X|$.
      \begin{center}
        \scriptsize
        \begin{math}
          $$\mprset{flushleft}
          \inferrule* [right={\tiny impR}] {
            {
              \begin{array}{c}
                \Pi' \\
                {\SCnt{A}  \SCsym{;}  \Gamma_{{\mathrm{1}}}  \SCsym{;}  \Phi  \SCsym{;}  \Gamma_{{\mathrm{2}}}  \vdash_\mathcal{L}  \SCnt{B}}
              \end{array}
            }
          }{\Gamma_{{\mathrm{1}}}  \SCsym{;}  \Phi  \SCsym{;}  \Gamma_{{\mathrm{2}}}  \vdash_\mathcal{L}  \SCnt{B}  \leftharpoonup  \SCnt{A}}
        \end{math}
      \end{center}

\item Case 2:
      \begin{center}
        \scriptsize
        \begin{math}
          \begin{array}{c}
            \Pi_1 \\
            {\Delta  \vdash_\mathcal{L}  \SCnt{C}}
          \end{array}
        \end{math}
        \qquad\qquad
        $\Pi_2$:
        \begin{math}
          $$\mprset{flushleft}
          \inferrule* [right={\tiny impR}] {
            {
              \begin{array}{c}
                \pi \\
                {\SCnt{A}  \SCsym{;}  \Gamma_{{\mathrm{1}}}  \SCsym{;}  \SCnt{C}  \SCsym{;}  \Gamma_{{\mathrm{2}}}  \vdash_\mathcal{L}  \SCnt{B}}
              \end{array}
            }
          }{\Gamma_{{\mathrm{1}}}  \SCsym{;}  \SCnt{C}  \SCsym{;}  \Gamma_{{\mathrm{2}}}  \vdash_\mathcal{L}  \SCnt{B}  \leftharpoonup  \SCnt{A}}
        \end{math}
      \end{center}
      By assumption, $c(\Pi_1),c(\Pi_2)\leq |C|$. By induction on $\Pi_1$
      and $\pi$, there is a proof $\Pi'$ for sequent
      $\Gamma_{{\mathrm{1}}}  \SCsym{;}  \Delta  \SCsym{;}  \Gamma_{{\mathrm{2}}}  \SCsym{;}  \SCnt{A}  \vdash_\mathcal{L}  \SCnt{B}$ s.t. $c(\Pi') \leq |C|$. Therefore, the
      proof $\Pi$ can be constructed as follows with
      $c(\Pi) = c(\Pi') \leq |C|$.
      \begin{center}
        \scriptsize
        \begin{math}
          $$\mprset{flushleft}
          \inferrule* [right={\tiny impR}] {
            {
              \begin{array}{c}
                \Pi' \\
                {\SCnt{A}  \SCsym{;}  \Gamma_{{\mathrm{1}}}  \SCsym{;}  \Delta  \SCsym{;}  \Gamma_{{\mathrm{2}}}  \vdash_\mathcal{L}  \SCnt{B}}
              \end{array}
            }
          }{\Gamma_{{\mathrm{1}}}  \SCsym{;}  \Delta  \SCsym{;}  \Gamma_{{\mathrm{2}}}  \vdash_\mathcal{L}  \SCnt{B}  \leftharpoonup  \SCnt{A}}
        \end{math}
      \end{center}
\end{itemize}



\subsubsection{Right introduction of the functor $F$}
\begin{center}
  \scriptsize
  \begin{math}
    \begin{array}{c}
      \Pi_1 \\
      {\Phi  \vdash_\mathcal{C}  \SCnt{X}}
    \end{array}
  \end{math}
  \qquad\qquad
  $\Pi_2$:
  \begin{math}
    $$\mprset{flushleft}
    \inferrule* [right={\tiny Fr}] {
      {
        \begin{array}{c}
          \pi \\
          {\Psi_{{\mathrm{1}}}  \SCsym{,}  \SCnt{X}  \SCsym{,}  \Psi_{{\mathrm{2}}}  \vdash_\mathcal{C}  \SCnt{Y}}
        \end{array}
      }
    }{\Psi_{{\mathrm{1}}}  \SCsym{,}  \SCnt{X}  \SCsym{,}  \Psi_{{\mathrm{2}}}  \vdash_\mathcal{L}   \mathsf{F} \SCnt{Y} }
  \end{math}
\end{center}
By assumption, $c(\Pi_1),c(\Pi_2)\leq |X|$. By induction on $\Pi_1$
and $\pi$, there is a proof $\Pi'$ for sequent $\Psi_{{\mathrm{1}}}  \SCsym{,}  \Phi  \SCsym{,}  \Psi_{{\mathrm{2}}}  \vdash_\mathcal{C}  \SCnt{Y}$
s.t. $c(\Pi') \leq |X|$. Therefore, the proof $\Pi$ can be
constructed as follows with $c(\Pi) = c(\Pi') \leq |X|$.
\begin{center}
  \scriptsize
  \begin{math}
    $$\mprset{flushleft}
    \inferrule* [right={\tiny Fr}] {
      {
        \begin{array}{c}
          \Pi' \\
          {\Psi_{{\mathrm{1}}}  \SCsym{,}  \Phi  \SCsym{,}  \Psi_{{\mathrm{2}}}  \vdash_\mathcal{C}  \SCnt{Y}}
        \end{array}
      }
    }{\Psi_{{\mathrm{1}}}  \SCsym{,}  \Phi  \SCsym{,}  \Psi_{{\mathrm{2}}}  \vdash_\mathcal{L}   \mathsf{F} \SCnt{Y} }
  \end{math}
\end{center}



\subsubsection{Left introduction of the functor $F$ (with low priority)}
\begin{itemize}
\item Case 1:
      \begin{center}
        \scriptsize
        \begin{math}
          \begin{array}{c}
            \Pi_1 \\
            {\Phi  \vdash_\mathcal{C}  \SCnt{X}}
          \end{array}
        \end{math}
        \qquad\qquad
        $\Pi_2$:
        \begin{math}
          $$\mprset{flushleft}
          \inferrule* [right={\tiny Fl}] {
            {
              \begin{array}{c}
                \pi \\
                {\Gamma_{{\mathrm{1}}}  \SCsym{;}  \SCnt{X}  \SCsym{;}  \Gamma_{{\mathrm{2}}}  \SCsym{;}  \SCnt{Y}  \SCsym{;}  \Gamma_{{\mathrm{3}}}  \vdash_\mathcal{L}  \SCnt{A}}
              \end{array}
            }
          }{\Gamma_{{\mathrm{1}}}  \SCsym{;}  \SCnt{X}  \SCsym{;}  \Gamma_{{\mathrm{2}}}  \SCsym{;}   \mathsf{F} \SCnt{Y}   \SCsym{;}  \Gamma_{{\mathrm{3}}}  \vdash_\mathcal{L}  \SCnt{A}}
        \end{math}
      \end{center}
      By assumption, $c(\Pi_1),c(\Pi_2)\leq |X|$. By induction on $\Pi_1$
      and $\pi$, there is a proof $\Pi'$ for sequent
      $\Gamma_{{\mathrm{1}}}  \SCsym{;}  \Phi  \SCsym{;}  \Gamma_{{\mathrm{2}}}  \SCsym{;}  \SCnt{Y}  \SCsym{;}  \Gamma_{{\mathrm{3}}}  \vdash_\mathcal{L}  \SCnt{A}$ s.t. $c(\Pi') \leq |X|$. Therefore, the
      proof $\Pi$ can be constructed as follows with
      $c(\Pi) = c(\Pi') \leq |X|$.
      \begin{center}
        \scriptsize
        \begin{math}
          $$\mprset{flushleft}
          \inferrule* [right={\tiny Fl}] {
            {
              \begin{array}{c}
                \Pi' \\
                {\Gamma_{{\mathrm{1}}}  \SCsym{;}  \Phi  \SCsym{;}  \Gamma_{{\mathrm{2}}}  \SCsym{;}  \SCnt{Y}  \SCsym{;}  \Gamma_{{\mathrm{3}}}  \vdash_\mathcal{L}  \SCnt{A}}
              \end{array}
            }
          }{\Gamma_{{\mathrm{1}}}  \SCsym{;}  \Phi  \SCsym{;}  \Gamma_{{\mathrm{2}}}  \SCsym{;}   \mathsf{F} \SCnt{Y}   \SCsym{;}  \Gamma_{{\mathrm{3}}}  \vdash_\mathcal{L}  \SCnt{A}}
        \end{math}
      \end{center}

\item Case 2:
      \begin{center}
        \scriptsize
        \begin{math}
          \begin{array}{c}
            \Pi_1 \\
            {\Delta  \vdash_\mathcal{L}  \SCnt{B}}
          \end{array}
        \end{math}
        \qquad\qquad
        $\Pi_2$:
        \begin{math}
          $$\mprset{flushleft}
          \inferrule* [right={\tiny Fl}] {
            {
              \begin{array}{c}
                \pi \\
                {\Gamma_{{\mathrm{1}}}  \SCsym{;}  \SCnt{B}  \SCsym{;}  \Gamma_{{\mathrm{2}}}  \SCsym{;}  \SCnt{Y}  \SCsym{;}  \Gamma_{{\mathrm{3}}}  \vdash_\mathcal{L}  \SCnt{A}}
              \end{array}
            }
          }{\Gamma_{{\mathrm{1}}}  \SCsym{;}  \SCnt{B}  \SCsym{;}  \Gamma_{{\mathrm{2}}}  \SCsym{;}   \mathsf{F} \SCnt{Y}   \SCsym{;}  \Gamma_{{\mathrm{3}}}  \vdash_\mathcal{L}  \SCnt{A}}
        \end{math}
      \end{center}
      By assumption, $c(\Pi_1),c(\Pi_2)\leq |B|$. By induction on $\Pi_1$
      and $\pi$, there is a proof $\Pi'$ for sequent
      $\Gamma_{{\mathrm{1}}}  \SCsym{;}  \Delta  \SCsym{;}  \Gamma_{{\mathrm{2}}}  \SCsym{;}  \SCnt{Y}  \SCsym{;}  \Gamma_{{\mathrm{3}}}  \vdash_\mathcal{L}  \SCnt{A}$ s.t. $c(\Pi') \leq |B|$. Therefore, the
      proof $\Pi$ can be constructed as follows with
      $c(\Pi) = c(\Pi') \leq |B|$.
      \begin{center}
        \scriptsize
        \begin{math}
          $$\mprset{flushleft}
          \inferrule* [right={\tiny Fl}] {
            {
              \begin{array}{c}
                \Pi' \\
                {\Gamma_{{\mathrm{1}}}  \SCsym{;}  \Delta  \SCsym{;}  \Gamma_{{\mathrm{2}}}  \SCsym{;}  \SCnt{Y}  \SCsym{;}  \Gamma_{{\mathrm{3}}}  \vdash_\mathcal{L}  \SCnt{A}}
              \end{array}
            }
          }{\Gamma_{{\mathrm{1}}}  \SCsym{;}  \Delta  \SCsym{;}  \Gamma_{{\mathrm{2}}}  \SCsym{;}   \mathsf{F} \SCnt{Y}   \SCsym{;}  \Gamma_{{\mathrm{3}}}  \vdash_\mathcal{L}  \SCnt{A}}
        \end{math}
      \end{center}

\item Case 3:
      \begin{center}
        \scriptsize
        \begin{math}
          \begin{array}{c}
            \Pi_1 \\
            {\Phi  \vdash_\mathcal{C}  \SCnt{X}}
          \end{array}
        \end{math}
        \qquad\qquad
        $\Pi_2$:
        \begin{math}
          $$\mprset{flushleft}
          \inferrule* [right={\tiny Fl}] {
            {
              \begin{array}{c}
                \pi \\
                {\Gamma_{{\mathrm{1}}}  \SCsym{;}  \SCnt{Y}  \SCsym{;}  \Gamma_{{\mathrm{2}}}  \SCsym{;}  \SCnt{X}  \SCsym{;}  \Gamma_{{\mathrm{3}}}  \vdash_\mathcal{L}  \SCnt{A}}
              \end{array}
            }
          }{\Gamma_{{\mathrm{1}}}  \SCsym{;}   \mathsf{F} \SCnt{Y}   \SCsym{;}  \Gamma_{{\mathrm{2}}}  \SCsym{;}  \SCnt{X}  \SCsym{;}  \Gamma_{{\mathrm{3}}}  \vdash_\mathcal{L}  \SCnt{A}}
        \end{math}
      \end{center}
      By assumption, $c(\Pi_1),c(\Pi_2)\leq |X|$. By induction on $\Pi_1$
      and $\pi$, there is a proof $\Pi'$ for sequent
      $\Gamma_{{\mathrm{1}}}  \SCsym{;}  \SCnt{Y}  \SCsym{;}  \Gamma_{{\mathrm{2}}}  \SCsym{;}  \Phi  \SCsym{;}  \Gamma_{{\mathrm{3}}}  \vdash_\mathcal{L}  \SCnt{A}$ s.t. $c(\Pi') \leq |X|$. Therefore, the
      proof $\Pi$ can be constructed as follows with
      $c(\Pi) = c(\Pi') \leq |X|$.
      \begin{center}
        \scriptsize
        \begin{math}
          $$\mprset{flushleft}
          \inferrule* [right={\tiny Fl}] {
            {
              \begin{array}{c}
                \Pi' \\
                {\Gamma_{{\mathrm{1}}}  \SCsym{;}  \SCnt{Y}  \SCsym{;}  \Gamma_{{\mathrm{2}}}  \SCsym{;}  \Phi  \SCsym{;}  \Gamma_{{\mathrm{3}}}  \vdash_\mathcal{L}  \SCnt{A}}
              \end{array}
            }
          }{\Gamma_{{\mathrm{1}}}  \SCsym{;}   \mathsf{F} \SCnt{Y}   \SCsym{;}  \Gamma_{{\mathrm{2}}}  \SCsym{;}  \Phi  \SCsym{;}  \Gamma_{{\mathrm{3}}}  \vdash_\mathcal{L}  \SCnt{A}}
        \end{math}
      \end{center}

\item Case 4:
      \begin{center}
        \scriptsize
        \begin{math}
          \begin{array}{c}
            \Pi_1 \\
            {\Delta  \vdash_\mathcal{L}  \SCnt{B}}
          \end{array}
        \end{math}
        \qquad\qquad
        $\Pi_2$:
        \begin{math}
          $$\mprset{flushleft}
          \inferrule* [right={\tiny Fl}] {
            {
              \begin{array}{c}
                \pi \\
                {\Gamma_{{\mathrm{1}}}  \SCsym{;}  \SCnt{Y}  \SCsym{;}  \Gamma_{{\mathrm{2}}}  \SCsym{;}  \SCnt{B}  \SCsym{;}  \Gamma_{{\mathrm{3}}}  \vdash_\mathcal{L}  \SCnt{A}}
              \end{array}
            }
          }{\Gamma_{{\mathrm{1}}}  \SCsym{;}   \mathsf{F} \SCnt{Y}   \SCsym{;}  \Gamma_{{\mathrm{2}}}  \SCsym{;}  \Delta  \SCsym{;}  \Gamma_{{\mathrm{3}}}  \vdash_\mathcal{L}  \SCnt{A}}
        \end{math}
      \end{center}
      By assumption, $c(\Pi_1),c(\Pi_2)\leq |B|$. By induction on $\Pi_1$
      and $\pi$, there is a proof $\Pi'$ for sequent
      $\Gamma_{{\mathrm{1}}}  \SCsym{;}  \SCnt{Y}  \SCsym{;}  \Gamma_{{\mathrm{2}}}  \SCsym{;}  \Delta  \SCsym{;}  \Gamma_{{\mathrm{3}}}  \vdash_\mathcal{L}  \SCnt{A}$ s.t. $c(\Pi') \leq |B|$. Therefore, the
      proof $\Pi$ can be constructed as follows with
      $c(\Pi) = c(\Pi') \leq |B|$.
      \begin{center}
        \scriptsize
        \begin{math}
          $$\mprset{flushleft}
          \inferrule* [right={\tiny Fl}] {
            {
              \begin{array}{c}
                \Pi' \\
                {\Gamma_{{\mathrm{1}}}  \SCsym{;}  \SCnt{Y}  \SCsym{;}  \Gamma_{{\mathrm{2}}}  \SCsym{;}  \Delta  \SCsym{;}  \Gamma_{{\mathrm{3}}}  \vdash_\mathcal{L}  \SCnt{A}}
              \end{array}
            }
          }{\Gamma_{{\mathrm{1}}}  \SCsym{;}   \mathsf{F} \SCnt{Y}   \SCsym{;}  \Gamma_{{\mathrm{2}}}  \SCsym{;}  \Delta  \SCsym{;}  \Gamma_{{\mathrm{3}}}  \vdash_\mathcal{L}  \SCnt{A}}
        \end{math}
      \end{center}
\end{itemize}




\subsubsection{Right introduction of the functor $G$ (with low priority)}
\begin{center}
  \scriptsize
  \begin{math}
    \begin{array}{c}
      \Pi_1 \\
      {\Phi  \vdash_\mathcal{C}  \SCnt{X}}
    \end{array}
  \end{math}
  \qquad\qquad
  $\Pi_2$:
  \begin{math}
    $$\mprset{flushleft}
    \inferrule* [right={\tiny Gr}] {
      {
        \begin{array}{c}
          \pi \\
          {\Psi_{{\mathrm{1}}}  \SCsym{;}  \SCnt{X}  \SCsym{;}  \Psi_{{\mathrm{2}}}  \vdash_\mathcal{L}  \SCnt{A}}
        \end{array}
      }
    }{\Psi_{{\mathrm{1}}}  \SCsym{,}  \SCnt{X}  \SCsym{,}  \Psi_{{\mathrm{2}}}  \vdash_\mathcal{C}   \mathsf{G} \SCnt{A} }
  \end{math}
\end{center}
By assumption, $c(\Pi_1),c(\Pi_2)\leq |X|$. By induction on $\Pi_1$
and $\pi$, there is a proof $\Pi'$ for sequent $\Psi_{{\mathrm{1}}}  \SCsym{,}  \Phi  \SCsym{,}  \Psi_{{\mathrm{2}}}  \vdash_\mathcal{L}  \SCnt{A}$
s.t. $c(\Pi') \leq |X|$. Therefore, the proof $\Pi$ can be
constructed as follows with $c(\Pi) = c(\Pi') \leq |X|$.
\begin{center}
  \scriptsize
  \begin{math}
    $$\mprset{flushleft}
    \inferrule* [right={\tiny Gr}] {
      {
        \begin{array}{c}
          \Pi' \\
          {\Psi_{{\mathrm{1}}}  \SCsym{;}  \Phi  \SCsym{;}  \Psi_{{\mathrm{2}}}  \vdash_\mathcal{L}  \SCnt{A}}
        \end{array}
      }
    }{\Psi_{{\mathrm{1}}}  \SCsym{,}  \Phi  \SCsym{,}  \Psi_{{\mathrm{2}}}  \vdash_\mathcal{C}   \mathsf{G} \SCnt{A} }
  \end{math}
\end{center}




\subsubsection{Left introduction of the functor $G$ (with low priority)}
\begin{itemize}
\item Case 1:
      \begin{center}
        \scriptsize
        \begin{math}
          \begin{array}{c}
            \Pi_1 \\
            {\Phi  \vdash_\mathcal{C}  \SCnt{X}}
          \end{array}
        \end{math}
        \qquad\qquad
        $\Pi_2$:
        \begin{math}
          $$\mprset{flushleft}
          \inferrule* [right={\tiny Gl}] {
            {
              \begin{array}{c}
                \pi \\
                {\Gamma_{{\mathrm{1}}}  \SCsym{;}  \SCnt{X}  \SCsym{;}  \Gamma_{{\mathrm{2}}}  \SCsym{;}  \SCnt{B}  \SCsym{;}  \Gamma_{{\mathrm{3}}}  \vdash_\mathcal{L}  \SCnt{A}}
              \end{array}
            }
          }{\Gamma_{{\mathrm{1}}}  \SCsym{;}  \SCnt{X}  \SCsym{;}  \Gamma_{{\mathrm{2}}}  \SCsym{;}   \mathsf{G} \SCnt{B}   \SCsym{;}  \Gamma_{{\mathrm{3}}}  \vdash_\mathcal{L}  \SCnt{A}}
        \end{math}
      \end{center}
      By assumption, $c(\Pi_1),c(\Pi_2)\leq |X|$. By induction on $\Pi_1$
      and $\pi$, there is a proof $\Pi'$ for sequent
      $\Gamma_{{\mathrm{1}}}  \SCsym{;}  \Phi  \SCsym{;}  \Gamma_{{\mathrm{2}}}  \SCsym{;}  \SCnt{B}  \SCsym{;}  \Gamma_{{\mathrm{3}}}  \vdash_\mathcal{L}  \SCnt{A}$ s.t. $c(\Pi') \leq |X|$. Therefore, the
      proof $\Pi$ can be constructed as follows with
      $c(\Pi) = c(\Pi') \leq |X|$.
      \begin{center}
        \scriptsize
        \begin{math}
          $$\mprset{flushleft}
          \inferrule* [right={\tiny Gl}] {
            {
              \begin{array}{c}
                \Pi' \\
                {\Gamma_{{\mathrm{1}}}  \SCsym{;}  \Phi  \SCsym{;}  \Gamma_{{\mathrm{2}}}  \SCsym{;}  \SCnt{B}  \SCsym{;}  \Gamma_{{\mathrm{3}}}  \vdash_\mathcal{L}  \SCnt{A}}
              \end{array}
            }
          }{\Gamma_{{\mathrm{1}}}  \SCsym{;}  \Phi  \SCsym{;}  \Gamma_{{\mathrm{2}}}  \SCsym{;}   \mathsf{G} \SCnt{B}   \SCsym{;}  \Gamma_{{\mathrm{3}}}  \vdash_\mathcal{L}  \SCnt{A}}
        \end{math}
      \end{center}

\item Case 2:
      \begin{center}
        \scriptsize
        \begin{math}
          \begin{array}{c}
            \Pi_1 \\
            {\Delta  \vdash_\mathcal{L}  \SCnt{B}}
          \end{array}
        \end{math}
        \qquad\qquad
        $\Pi_2$:
        \begin{math}
          $$\mprset{flushleft}
          \inferrule* [right={\tiny Gl}] {
            {
              \begin{array}{c}
                \pi \\
                {\Gamma_{{\mathrm{1}}}  \SCsym{;}  \SCnt{B}  \SCsym{;}  \Gamma_{{\mathrm{2}}}  \SCsym{;}  \SCnt{C}  \SCsym{;}  \Gamma_{{\mathrm{3}}}  \vdash_\mathcal{L}  \SCnt{A}}
              \end{array}
            }
          }{\Gamma_{{\mathrm{1}}}  \SCsym{;}  \SCnt{B}  \SCsym{;}  \Gamma_{{\mathrm{2}}}  \SCsym{;}   \mathsf{G} \SCnt{C}   \SCsym{;}  \Gamma_{{\mathrm{3}}}  \vdash_\mathcal{L}  \SCnt{A}}
        \end{math}
      \end{center}
      By assumption, $c(\Pi_1),c(\Pi_2)\leq |B|$. By induction on $\Pi_1$
      and $\pi$, there is a proof $\Pi'$ for sequent
      $\Gamma_{{\mathrm{1}}}  \SCsym{;}  \Delta  \SCsym{;}  \Gamma_{{\mathrm{2}}}  \SCsym{;}  \SCnt{C}  \SCsym{;}  \Gamma_{{\mathrm{3}}}  \vdash_\mathcal{L}  \SCnt{A}$ s.t. $c(\Pi') \leq |B|$. Therefore, the
      proof $\Pi$ can be constructed as follows with
      $c(\Pi) = c(\Pi') \leq |B|$.
      \begin{center}
        \scriptsize
        \begin{math}
          $$\mprset{flushleft}
          \inferrule* [right={\tiny Gl}] {
            {
              \begin{array}{c}
                \Pi' \\
                {\Gamma_{{\mathrm{1}}}  \SCsym{;}  \Delta  \SCsym{;}  \Gamma_{{\mathrm{2}}}  \SCsym{;}  \SCnt{C}  \SCsym{;}  \Gamma_{{\mathrm{3}}}  \vdash_\mathcal{L}  \SCnt{A}}
              \end{array}
            }
          }{\Gamma_{{\mathrm{1}}}  \SCsym{;}  \Delta  \SCsym{;}  \Gamma_{{\mathrm{2}}}  \SCsym{;}   \mathsf{G} \SCnt{C}   \SCsym{;}  \Gamma_{{\mathrm{3}}}  \vdash_\mathcal{L}  \SCnt{A}}
        \end{math}
      \end{center}

\item Case 3:
      \begin{center}
        \scriptsize
        \begin{math}
          \begin{array}{c}
            \Pi_1 \\
            {\Phi  \vdash_\mathcal{C}  \SCnt{X}}
          \end{array}
        \end{math}
        \qquad\qquad
        $\Pi_2$:
        \begin{math}
          $$\mprset{flushleft}
          \inferrule* [right={\tiny Gl}] {
            {
              \begin{array}{c}
                \pi \\
                {\Gamma_{{\mathrm{1}}}  \SCsym{;}  \SCnt{B}  \SCsym{;}  \Gamma_{{\mathrm{2}}}  \SCsym{;}  \SCnt{X}  \SCsym{;}  \Gamma_{{\mathrm{3}}}  \vdash_\mathcal{L}  \SCnt{A}}
              \end{array}
            }
          }{\Gamma_{{\mathrm{1}}}  \SCsym{;}   \mathsf{G} \SCnt{B}   \SCsym{;}  \Gamma_{{\mathrm{2}}}  \SCsym{;}  \SCnt{X}  \SCsym{;}  \Gamma_{{\mathrm{3}}}  \vdash_\mathcal{L}  \SCnt{A}}
        \end{math}
      \end{center}
      By assumption, $c(\Pi_1),c(\Pi_2)\leq |X|$. By induction on $\Pi_1$
      and $\pi$, there is a proof $\Pi'$ for sequent
      $\Gamma_{{\mathrm{1}}}  \SCsym{;}  \SCnt{B}  \SCsym{;}  \Gamma_{{\mathrm{2}}}  \SCsym{;}  \Phi  \SCsym{;}  \Gamma_{{\mathrm{3}}}  \vdash_\mathcal{L}  \SCnt{A}$ s.t. $c(\Pi') \leq |X|$. Therefore, the
      proof $\Pi$ can be constructed as follows with
      $c(\Pi) = c(\Pi') \leq |X|$.
      \begin{center}
        \scriptsize
        \begin{math}
          $$\mprset{flushleft}
          \inferrule* [right={\tiny Gl}] {
            {
              \begin{array}{c}
                \Pi' \\
                {\Gamma_{{\mathrm{1}}}  \SCsym{;}  \SCnt{B}  \SCsym{;}  \Gamma_{{\mathrm{2}}}  \SCsym{;}  \Phi  \SCsym{;}  \Gamma_{{\mathrm{3}}}  \vdash_\mathcal{L}  \SCnt{A}}
              \end{array}
            }
          }{\Gamma_{{\mathrm{1}}}  \SCsym{;}   \mathsf{G} \SCnt{B}   \SCsym{;}  \Gamma_{{\mathrm{2}}}  \SCsym{;}  \Phi  \SCsym{;}  \Gamma_{{\mathrm{3}}}  \vdash_\mathcal{L}  \SCnt{A}}
        \end{math}
      \end{center}

\item Case 4:
      \begin{center}
        \scriptsize
        \begin{math}
          \begin{array}{c}
            \Pi_1 \\
            {\Delta  \vdash_\mathcal{L}  \SCnt{B}}
          \end{array}
        \end{math}
        \qquad\qquad
        $\Pi_2$:
        \begin{math}
          $$\mprset{flushleft}
          \inferrule* [right={\tiny Gl}] {
            {
              \begin{array}{c}
                \pi \\
                {\Gamma_{{\mathrm{1}}}  \SCsym{;}  \SCnt{C}  \SCsym{;}  \Gamma_{{\mathrm{2}}}  \SCsym{;}  \SCnt{B}  \SCsym{;}  \Gamma_{{\mathrm{3}}}  \vdash_\mathcal{L}  \SCnt{A}}
              \end{array}
            }
          }{\Gamma_{{\mathrm{1}}}  \SCsym{;}   \mathsf{G} \SCnt{C}   \SCsym{;}  \Gamma_{{\mathrm{2}}}  \SCsym{;}  \SCnt{B}  \SCsym{;}  \Gamma_{{\mathrm{3}}}  \vdash_\mathcal{L}  \SCnt{A}}
        \end{math}
      \end{center}
      By assumption, $c(\Pi_1),c(\Pi_2)\leq |B|$. By induction on $\Pi_1$
      and $\pi$, there is a proof $\Pi'$ for sequent
      $\Gamma_{{\mathrm{1}}}  \SCsym{;}  \SCnt{C}  \SCsym{;}  \Gamma_{{\mathrm{2}}}  \SCsym{;}  \Delta  \SCsym{;}  \Gamma_{{\mathrm{3}}}  \vdash_\mathcal{L}  \SCnt{A}$ s.t. $c(\Pi') \leq |B|$. Therefore, the
      proof $\Pi$ can be constructed as follows with
      $c(\Pi) = c(\Pi') \leq |B|$.
      \begin{center}
        \scriptsize
        \begin{math}
          $$\mprset{flushleft}
          \inferrule* [right={\tiny Gl}] {
            {
              \begin{array}{c}
                \Pi' \\
                {\Gamma_{{\mathrm{1}}}  \SCsym{;}  \SCnt{C}  \SCsym{;}  \Gamma_{{\mathrm{2}}}  \SCsym{;}  \Delta  \SCsym{;}  \Gamma_{{\mathrm{3}}}  \vdash_\mathcal{L}  \SCnt{A}}
              \end{array}
            }
          }{\Gamma_{{\mathrm{1}}}  \SCsym{;}   \mathsf{G} \SCnt{C}   \SCsym{;}  \Gamma_{{\mathrm{2}}}  \SCsym{;}  \Delta  \SCsym{;}  \Gamma_{{\mathrm{3}}}  \vdash_\mathcal{L}  \SCnt{A}}
        \end{math}
      \end{center}

\end{itemize}



%--------------------------------------------------
%--------------------------------------------------
\section{Proof For Lemma~\ref{lem:monoidal-monad}}
\label{app:monoidal-monad}

Let $(\cat{C},\cat{L},F,G,\eta,\varepsilon)$ be a LAM. We define the monad
$(T,\eta:id_\cat{C}\rightarrow T,\mu:T^2\rightarrow T)$ on $\cat{C}$ as
$T=GF$, $\eta_X:X\rightarrow GFX$, and
$\mu_X=G\varepsilon_{FX}:GFGFX\rightarrow GFX$. Since $(F,\m{})$ and
$(G,\n{})$ are monoidal functors, we have
$$\t{X,Y}=G\m{X,Y}\circ\n{FX,FY}:TX\otimes TY\rightarrow T(X\otimes Y) \qquad\mbox{and}\qquad\t{I}=G\m{I}\circ\n{I'}:I\rightarrow TI.$$
The monad $T$ being monoidal means:
\begin{enumerate}
\item $T$ is a monoidal functor, i.e. the following diagrams commute:
      \begin{mathpar}
      \bfig
        \hSquares/->`->`->``->`->`->/<400>[
          (TX\otimes TY)\otimes TZ`TX\otimes(TY\otimes TZ)`TX\otimes T(Y\otimes Z)`
          T(X\otimes Y)\otimes TZ`T((X\otimes Y)\otimes Z)`T(X\otimes(Y\otimes Z));
          \alpha_{TX,TY,TZ}`id_{TX}\otimes\t{Y,Z}`\t{X,Y}\otimes id_{TZ}``
          \t{X,Y\otimes Z}`\t{X\otimes Y,Z}`T\alpha_{X,Y,Z}]
        \morphism(1300,200)//<0,0>[`;(1)]
      \efig
      \and
      \bfig
        \square/->`->`<-`->/<600,400>[
          I\otimes TX`TX`TI\otimes TX`T(I\otimes X);
          \lambda_{TX}`\t{I}\otimes id_{TX}`T\lambda_X`\t{I,X}]
        \morphism(350,200)//<0,0>[`;(2)]
      \efig
      \and
      \bfig
        \square/->`->`<-`->/<600,400>[
          TX\otimes I`TX`TX\otimes TI`T(X\otimes I);
          \rho_{TX}`id_{TX}\otimes\t{I}`T\rho_X`\t{X,I}]
        \morphism(350,200)//<0,0>[`;(3)]
      \efig
      \end{mathpar}
      We write $GF$ instead of $T$ in the proof for clarity. \\
      By replacing $\t{X,Y}$ with its definition, diagram (1) above
      commutes by the following commutative diagram, in which the two
      hexagons commute because $G$ and $F$ are monoidal functors, and the
      two quadrilaterals commute by the naturality of $\n{}$.
      \begin{mathpar}
      \bfig
        \iiixiii/->`->`->``->```->`<-`->``/<1400,400>[
          (GFX\otimes GFY)\otimes GFZ`GFX\otimes(GFY\otimes GFZ)`GFX\otimes G(FY\tri FZ)`
          G(FX\tri FY)\otimes GFZ`G(FX\tri(FY\tri FZ))`GFX\otimes GF(Y\otimes Z)`
          GF(X\otimes Y)\otimes GFZ`G((FX\tri FY)\tri FZ)`G(FX\tri F(Y\otimes Z));
          \alpha_{GFX,GFY,GFZ}`id_{GFX}\otimes\n{FY,FZ}`\n{FX,FY}\otimes id_{GFZ}``
          id_{GFX}\otimes G\m{Y,Z}```G\m{X,Y}\otimes id_{GFZ}`G\alpha'_{FX,FY,FZ}`
          \n{FX,F(Y\otimes Z)}``]
        \morphism(2800,800)|m|<-1400,-400>[
          GFX\otimes G(FY\tri FZ)`G(FX\tri(FY\tri FZ));\n{FX,FY\tri FZ}]
        \morphism(0,400)|m|<1400,-400>[
          G(FX\tri FY)\otimes GFZ`G((FX\tri FY)\tri FZ);\n{FX\tri FY,FZ}]
        \morphism(1400,400)|m|<1400,-400>[
          G(FX\tri(FY\tri FZ))`G(FX\tri F(Y\otimes Z));G(id_{FX}\tri\m{Y,Z})]
        \ptriangle(0,-400)|mlm|/`->`->/<1400,400>[
          GF(X\otimes Y)\otimes GFZ`G((FX\tri FY)\tri FZ)`G(F(X\otimes Y)\tri FZ);
          `\n{F(X\otimes Y),FZ}`G(\m{X,Y}\otimes id_{FZ})]
        \morphism(0,-400)|b|<1400,0>[
          G(F(X\otimes Y)\tri FZ)`GF((X\otimes Y)\otimes Z);G\m{X\otimes Y,Z}]
        \dtriangle(1400,-400)|mrb|/`->`->/<1400,400>[
          G(FX\tri F(Y\otimes Z))`GF((X\otimes Y)\otimes Z)`GF(X\otimes(Y\otimes Z));
          `G\m{X,Y\otimes Z}`GF\alpha_{X,Y,Z}]
      \efig
      \end{mathpar}
      Diagram (2) commutes by the following commutative diagrams, in which
      the top quadrilateral commutes because $G$ is monoidal, the right
      quadrilateral commutes because $F$ is monoidal, and the left square
      commutes by the naturality of $\n{}$.
      \begin{mathpar}
      \bfig
        \ptriangle/->`->`/<1600,400>[
          I\otimes GFX`GFX`GI'\otimes GFX;\lambda_{GFX}`\n{I'}\otimes id_{GFX}`]
        \square(0,-400)|lmmb|<800,400>[
          GI'\otimes GFX`G(I'\tri FX)`GFI\otimes GFX`G(FI\tri FX);
          \n{I',FX}`G\m{I}\otimes id_{GFX}`G(\m{I}\tri id_{FX})`\n{FI,FX}]
        \morphism(800,0)|m|<800,400>[G(I'\tri FX)`GFX;G\lambda'_{FX}]
        \dtriangle(800,-400)/`<-`->/<800,800>[
          GFX`G(FI\tri FX)`GF(I\otimes X);
          `GF\lambda_X`G\m{I,X}]
      \efig
      \end{mathpar}
      Similarly, diagram (3) commutes as follows:
      \begin{mathpar}
      \bfig
        \ptriangle/->`->`/<1600,400>[
          GFX\otimes I`GFX`GFX\otimes GI';\rho_{GFX}`id_{GFX}\otimes\n{I'}`]
        \square(0,-400)|lmmb|<800,400>[
          GFX\otimes GI'`G(FX\tri I')`GFX\otimes GFI`G(FX\tri FI);
          \n{FX,I'}`id_{GFX}\otimes G\m{I}`G(id_{FX}\otimes\m{I})`\n{FX,FI}]
        \morphism(800,0)|m|<800,400>[G(FX\tri I')`GFX;G\rho'_{FX}]
        \dtriangle(800,-400)/`<-`->/<800,800>[
          GFX`G(FX\tri FI)`GF(X\otimes I);
          `GF\rho_X`G\m{X,I}]
      \efig
      \end{mathpar}
\item $\eta$ is a monoidal natural transformation. In fact, since $\eta$
      is the unit of the monoidal adjunction, $\eta$ is monoidal by
      definition and thus the following two diagrams commute.
      \begin{mathpar}
      \bfig
        \square/=`->`->`->/<600,400>[
          X\otimes Y`X\otimes Y`TX\otimes TY`T(X\otimes Y);
          `\eta_X\otimes\eta_Y`\eta_{X\otimes Y}`\t{X,Y}]
      \efig
      \and
      \bfig
        \Vtriangle/->`=`<-/<400,400>[I`TI`I;\eta_I``\t{I}]
      \efig
      \end{mathpar}
\item $\mu$ is a monoidal natural transformation. It is obvious that since
      $\mu=G\varepsilon_{FA}$ and $\varepsilon$ is monoidal, so is $\mu$.
      Thus the following diagrams commute.
      \begin{mathpar}
      \bfig
        \square/`->`->`->/<1500,400>[
          T^2X\otimes T^2Y`T^2(X\otimes Y)`TX\otimes TY`T(X\otimes Y);
          `\mu_X\otimes\mu_Y`\mu_{X\otimes Y}`\t{X,Y}]
        \morphism(0,400)<800,0>[T^2X\otimes T^2Y`T(TX\otimes TY);\t{TX,TY}]
        \morphism(800,400)<700,0>[T(TX\otimes TY)`T^2(X\otimes Y);T\t{X,Y}]
      \efig
      \and
      \bfig
        \square/->`<-`<-`<-/<400,400>[T^2I`TI`TI`I;\mu_I`T\t{I}`\t{I}`\t{I}]
      \efig
      \end{mathpar}
\end{enumerate}



%--------------------------------------------------
%--------------------------------------------------
\section{Proof For Lemma~\ref{lem:strong-monad}}
\label{app:strong-monad}

\begin{definition}
\label{def:strong-monad}
Let $(\cat{M},\tri,I,\alpha,\lambda,\rho)$ be a monoidal category and
$(T,\eta,\mu)$ be a monad on $\cat{M}$. $T$ is a \textbf{strong monad} if
there is natural transformation $\tau$, called the \textbf{tensorial
strength}, with components $\tau_{A,B}:A\tri TB\rightarrow T(A\tri B)$
such that the following diagrams commute:
\begin{mathpar}
\bfig
  \Vtriangle<400,400>[I\tri TA`T(I\tri A)`TA;\tau_{I,A}`\lambda_{TA}`T\lambda_A]
\efig
\and
\bfig
  \Vtriangle<400,400>[
    A\tri B`A\tri TB`T(A\tri B);id_A\tri\eta_B`\eta_{A\tri B}`\tau_{A,B}]
\efig
\and
\bfig
  \square/->`->`->`/<1800,400>[
    (A\tri B)\tri TC`T((A\tri B)\tri C)`
    A\tri(B\tri TC)`T(A\tri(B\tri C));
    \tau_{A\tri B,C}`\alpha_{A,B,TC}`T\alpha_{A,B,C}`]
  \morphism<900,0>[A\tri(B\tri TC)`A\tri T(B\tri C);id_A\tri\tau_{B,C}]
  \morphism(900,0)<900,0>[A\tri T(B\tri C)`T(A\tri(B\tri C));\tau_{A,B\tri C}]  \efig
\and
\bfig
  \square/`->`->`->/<1400,400>[
    A\tri T^2B`T^2(A\tri B)`A\tri TB`T(A\tri B);
    `id_A\tri\mu_B`\mu_{A\tri B}`\tau_{A,B}]
  \morphism(0,400)<700,0>[A\tri T^2B`T(A\tri TB);\tau_{A,TB}]
  \morphism(700,400)<700,0>[T(A\tri TB)`T^2(A\tri B);T\tau_{A,B}]
\efig
\end{mathpar}
\end{definition}
\noindent
The proof for Lemma~\ref{lem:strong-monad} goes as follows.
\noindent
Let $(\cat{C},\cat{L},F,G,\eta,\varepsilon)$ be a LAM, where
$(\cat{C},\otimes,I,\alpha,\lambda,\rho)$ is symmetric monoidal closed,
and \\ $(\cat{L},\tri,I',\alpha',\lambda',\rho')$ is Lambek. In
Lemma~\ref{lem:monoidal-monad}, we have proved that the monad
$(T=GF,\eta,\mu)$ is monoidal with the natural transformation
$\t{X,Y}:TX\otimes TY\rightarrow T(X\otimes Y)$ and the morphism
$\t{I}:I\rightarrow TI$.
\noindent
We define the tensorial strength
$\tau_{X,Y}:X\otimes TY\rightarrow T(X\otimes Y)$ as
$$\tau_{X,Y}=\t{X,Y}\circ(\eta_X\otimes id_{TY}).$$
Since $\eta$ is a monoidal natural transformation, we have
$\eta_I=G\m{I}\circ\n{I'}$, and thus $\eta_I=\t{I}$. The following diagram
commutes because $T$ is monoidal, where the composition
$\t{I,X}\circ(\t{I}\otimes id_{TX})$ is the definition of $\tau_{I,X}$. So
the first triangle in Definition~\ref{def:strong-monad} commutes.
\begin{mathpar}
\bfig
  \square/->`->`->`<-/<600,400>[
    I\otimes TX`TI\otimes TX`TX`T(I\otimes X);
    \t{I}\otimes id_{TX}`\lambda_{TX}`\t{I,X}`T\lambda_X]
\efig
\end{mathpar}
Similarly, by using the definition of $\tau$, the the second triangle in the definition is
equivalent to the following diagram, which commutes because $\eta$ is a monoidal natural
transformation:
\begin{mathpar}
\bfig
  \square/->`->`->`<-/<600,400>[
    X\otimes Y`X\otimes TY`T(X\otimes Y)`TX\otimes TY;
    id_X\otimes\eta_Y`\eta_{X\otimes Y}`\eta_X\otimes id_{TY}`\t{X,Y}]
  \morphism(0,400)|m|<600,-400>[X\otimes Y`TX\otimes TY;\eta_X\otimes\eta_Y]
\efig
\end{mathpar}
The first pentagon in the definition commutes by the following commutative diagrams, because
$\eta$ and $\alpha$ are natural transformations and $T$ is monoidal:
\begin{mathpar}
\bfig
  \qtriangle|amm|/->`->`<-/<1000,400>[
    (X\otimes Y)\otimes TZ`T(X\otimes Y)\otimes TZ`(TX\otimes TY)\otimes TZ;
    \eta_{X\otimes Y}\otimes id_{TZ}`
    (\eta_X\otimes\eta_Y)\otimes id_{TZ}`
    \t{X,Y}\otimes id_{TZ}]
  \morphism(0,400)<0,-400>[(X\otimes Y)\otimes TZ`X\otimes(Y\otimes TZ);\alpha_{X,Y,TZ}]
  \morphism(1000,0)|m|<0,-400>[
    (TX\otimes TY)\otimes TZ`TX\otimes(TY\otimes TZ);\alpha_{TX,TY,TZ}]
  \Dtriangle(0,-800)|lmm|/->`->`<-/<1000,400>[
    X\otimes(Y\otimes TZ)`TX\otimes(TY\otimes TZ)`X\otimes(TY\otimes TZ);
    id_X\otimes(\eta_Y\otimes id_{TZ})`
    \eta_X\otimes(\eta_Y\otimes id_{TZ})`
    \eta_X\otimes id_{TY\otimes TZ}]
  \morphism(0,-800)|b|<1000,0>[
    X\otimes(TY\otimes TZ)`X\otimes T(Y\otimes Z);id_X\otimes\t{Y,Z}]
  \qtriangle(1000,0)|amr|/->``->/<1000,400>[
    T(X\otimes Y)\otimes TZ`T((X\otimes Y)\otimes Z)`T(X\otimes(Y\otimes Z));
    \t{X\otimes Y,Z}``T\alpha_{X,Y,Z}]
  \morphism(2000,-800)<0,800>[
    TX\otimes T(Y\otimes Z)`T(X\otimes(Y\otimes Z));\t{X,Y\otimes Z}]
  \btriangle(1000,-800)|mmb|/`->`->/<1000,400>[
    TX\otimes(TY\otimes TZ)`X\otimes T(Y\otimes Z)`TX\otimes T(Y\otimes Z);
    `id_{TX}\otimes\t{Y,Z}`\eta_X\otimes id_{T(Y\otimes Z)}]
\efig
\end{mathpar}
The last diagram in the definition commutes by the following commutative diagram, because
$T$ is a monad, $\t{}$ is a natural transformation, and $\mu$ is a monoidal natural
transformation:
\begin{mathpar}
\bfig
  \ptriangle/->`->`/<700,400>[
    X\otimes T^2Y`TX\otimes T^2Y`X\otimes TY;\eta_X\otimes id_{T^2Y}`id_X\otimes\mu_Y`]
  \btriangle(0,-400)/->``->/<700,400>[
    X\otimes TY`TX\otimes TY`T(X\otimes Y);\eta_X\otimes id_{TY}``\t{X,Y}]
  \morphism(700,400)|m|<-700,-800>[TX\otimes T^2Y`TX\otimes TY;id_{TX}\otimes\mu_Y]
  \morphism(700,0)|m|<-700,-400>[TX\otimes T^2Y`TX\otimes TY;id_{TX}\otimes\mu_Y]
  \qtriangle(700,0)/->``->/<1800,400>[
    TX\otimes T^2Y`T(X\otimes TY)`T(TX\otimes TY);\t{X,TY}``T(\eta_X\otimes id_{TY})]
  \btriangle(700,0)|mmm|/=`->`<-/<900,400>[
    TX\otimes T^2Y`TX\otimes T^2Y`T^2X\otimes T^2Y;
    `T\eta_X\otimes id_{T^2Y}`\mu_X\otimes id_{T^2Y}]
  \morphism(1600,0)|m|<900,0>[T^2X\otimes T^2Y`T(TX\otimes TY);\t{TX,TY}]
  \morphism(1600,0)|m|<-1600,-400>[T^2X\otimes T^2Y`TX\otimes TY;\mu_X\otimes\mu_Y]
  \dtriangle(700,-400)/`->`<-/<1800,400>[
    T(TX\otimes TY)`T(X\otimes Y)`T^2(X\otimes Y);`T\t{X,Y}`\mu_{X\otimes Y}]
\efig
\end{mathpar}



