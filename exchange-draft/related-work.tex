Polakow and Pfenning discuseed Ordered Linear Logic (OLL) \cite{}, which combines
intuitionistic, commutative linear and non-commutative linear logic, OLL contains sequents of
the form $\Gamma,\Delta,\Omega\vdash A$, where $\Gamma$ is a multiset of intuitionistic
assumptions, $\Delta$ is a multiset of commutative linear assumptions, and $\Omega$ is a list of
non-commutative linear assumptions. OLL contains logical connectives from all three the logics.
Therefore, our non-commutative adjoint model is a part of OLL and after combining with Benton's
commutative adjoint model, we would get a simplification of OLL.

Greco and Palmigiano \cite{} also presents a variant of the multiplicative fragment of
non-commutative ILL. But they focus on proper display calculi while we use sequent calculi.

de Paiva and Eades \cite{} also developed categorical models for the non-commutative ILL by
adapting the Dialectica categorical models for linear logic. 
