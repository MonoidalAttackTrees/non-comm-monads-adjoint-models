
\begin{definition}
\label{def:mc}
  A \textbf{monoidal category} $(\cat{M},\tri,I,\alpha,\lambda,\rho)$ is a category $\cat{M}$
  consists of
  \begin{itemize}
  \item a bifunctor $\tri:\cat{M}\times\cat{M}\rightarrow\cat{M}$, called the tensor product;
  \item an object $I$, called the unit object;
  \item three natural isomorphisms $\alpha$, $\lambda$, and $\rho$ with components
        $$\alpha_{A,B,C}:(A\tri B)\tri C\rightarrow A\tri(B\tri C)$$
        $$\lambda_A:I\tri A\rightarrow A$$
        $$\rho_A:A\tri I\rightarrow A$$
        where $\alpha$ is called associator, $\lambda$ is left unitor, and $\rho$ is right
        unitor,
  \end{itemize}
  such that the following diagrams commute for any objects $A$, $B$, $C$ in $\cat{M}$:
  \begin{mathpar}
  \bfig
    \square/`->`->`->/<2100,400>[
      ((A\tri B)\tri C)\tri D`
      A\tri((B\tri C)\tri D)`
      (A\tri B)\tri(C\tri D)`
      A\tri(B\tri(C\tri D));
      `
      \alpha_{A\tri B,C,D}`
      id_A\tri\alpha_{B,C,D}`
      \alpha_{A,B,C\tri D}]
    \morphism(0,400)<1100,0>[
      ((A\tri B)\tri C)\tri D`
      (A\tri(B\tri C))\tri D;
      \alpha_{A,B,C}\tri id_D]
    \morphism(1100,400)<1000,0>[
      (A\tri(B\tri C))\tri D`
      A\tri((B\tri C)\tri D);
      \alpha_{A,B\tri C,D}]
  \efig
  \and
  \bfig
    \Vtriangle<400,400>[
      (A\tri I)\tri B`
      A\tri(I\tri B)`
      A\tri B;
      \alpha_{A,I,B}`
      \rho_A\tri id_B`
      id_A\tri\lambda_B]
  \efig
  \end{mathpar}
\end{definition}

\begin{definition}
  A \textbf{Lambek category} (or a \textbf{biclosed monoidal category}) is a monoidal category
  $(\cat{M},\tri,I,\alpha,\lambda,\rho)$ equipped with two bifunctors
  $\rightharpoonup:\cat{M}^{op}\times\cat{M}\rightarrow\cat{M}$ and
  $\leftharpoonup:\cat{M}\times\cat{M}^{op}\rightarrow\cat{M}$ that are both right adjoint to
  the tensor product. That is, the following natural bijections hold:
  \begin{center}
  \begin{math}
  \begin{array}{lll}
    \Hom{L}{X\tri A}{B}\cong\Hom{L}{X}{A\lto B} & \quad\quad\quad\quad & 
    \Hom{L}{A\tri X}{B}\cong\Hom{L}{X}{B\rto A}
  \end{array}
  \end{math}
  \end{center}
\end{definition}

\begin{definition}
  \label{def:smcc}
  A \textbf{symmetric monoidal category} (SMCC) is a monoidal category
  $(\cat{M},\otimes,I,\alpha,\lambda,\rho)$ together with a natural transformation with
  components $\e{A,B}:A\otimes B\rightarrow B\otimes A$, called \textbf{exchange}, such that the
  following diagrams commute:
  \begin{mathpar}
  \bfig
    \Vtriangle<300,400>[A\otimes I`I\otimes A`A;\e{A,I}`\rho_A`\lambda_A]
  \efig
  \and
  \bfig
    \Vtriangle/=`->`<-/<300,400>[
      A\otimes B`A\otimes B`B\otimes A;
      id_{A\otimes B}`\e{A,B}`\e{B,A}]
  \efig
  \and
  \bfig
    \hSquares/->`->`->``->`->`->/<400>[
      (A\otimes B)\otimes C`A\otimes(B\otimes C)`(B\otimes C)\otimes A`
      (B\otimes A)\otimes C`B\otimes(A\otimes C)`B\otimes(C\otimes A);
      \alpha_{A,B,C}`\e{A,B\otimes C}`\e{A,B}\otimes id_C``
      \alpha_{B,A,C}`\alpha_{B,A,C}`id_B\otimes\e{A,C}]
  \efig
  \end{mathpar}
\end{definition}

We use $\tri$ for non-symmetric monoidal categories while $\otimes$ for symmetric ones.

\begin{definition}
  A \textbf{symmetric monoidal closed category} $(\cat{M},\otimes,I,\alpha,\lambda,\rho)$ is a
  symmetric monoidal category equipped with a bifunctor
  $\limp:\cat{M}^{op}\times\cat{M}\rightarrow\cat{M}$ that is right adjoint to the tensor
  product. That is, the following natural bijection
  $\Hom{\cat{M}}{X\otimes A}{B}\cong\Hom{\cat{M}}{X}{A\limp B}$ holds.
\end{definition}

\begin{lemma}
  \label{lemma:internal-homs-collapse}
  Let $A$ and $B$ be two objects in a Lambek category with the exchange natural transformation.
  Then $(A \lto B) \cong (B \rto A)$.
\end{lemma}
\begin{proof}
  First, notice that for any object $C$ we have
  \begin{center}
  \begin{math}
  \small
  \begin{array}{lllll}
    Hom[C,A\lto B]
    & \cong & Hom[C\otimes A,B] & \cat{L}\text{ is a Lambek category}\\
    & \cong & Hom[A\otimes C,B] & \text{By the exchange }\e{C,A}\\
    & \cong & Hom[C,B\rto A]    & \cat{L}\text{ is a Lambek category}
  \end{array}
  \end{math}
  \end{center}  
  Thus, $A\lto B\cong B\rto A$ by the Yoneda lemma.
\end{proof}
\begin{corollary}
  \label{corollary:LC-with-ex-mc}
  A Lambek category with exchange is symmetric monoidal closed.
\end{corollary}

\begin{definition}
  Let $(\cat{M},\tri,I,\alpha,\lambda,\rho)$ and
  $(\cat{M'},\tri',I',\alpha',\lambda',\rho')$ be monoidal categories. A \textbf{monoidal
  functor} $(F,\m{})$ from $\cat{M}$ to $\cat{M'}$ is a functor $F:\cat{M}\rightarrow\cat{M'}$
  together with a morphism $\m{I}:I'\rightarrow F(I)$ and a natural transformation
  $\m{A,B}:FA'\tri FB'\rightarrow F(A\tri B)$, such that the following diagrams commute
  for any objects $A$, $B$, and $C$ in $\cat{M}$:
  \begin{mathpar}
  \bfig
    \hSquares/->`->`->``->`->`->/<400>[
      (FA\tri'FB)\tri'FC`FA\tri'(FB\tri'FC)`FA\tri'F(B\tri C)`
      F(A\tri B)\tri'FC`F((A\tri B)\tri C)`F(A\tri(B\tri C));
      \alpha'_{FA,FB,FC}`id_{FA}\tri'\m{A,B}`\m{A,B}\tri'id_{FC}``
      \m{A,B\tri C}`\m{A\tri B,C}`F\alpha_{A,B,C}]
  \efig
  \and
  \bfig
    \square/->`->`<-`->/<600,400>[
      I'\tri'FA`FA`FI\tri'FA`F(I\tri A);
      \lambda'_{FA}`\m{I}\tri id_{FA}`F\lambda_A`\m{I,A}]
  \efig
  \and
  \bfig
    \square/->`->`<-`->/<600,400>[
      FA\tri'I'`FA`FA\tri'FI`F(A\tri I);
      \rho'_{FA}`\id_{FA}\tri\m{I}`F\rho_A`\m{A,I}]
  \efig
  \end{mathpar}
\end{definition}

\begin{definition}
  Let $(\cat{M},\otimes,I,\alpha,\lambda,\rho)$ and
  $(\cat{M'},\otimes',I',\alpha',\lambda',\rho')$ be symmetric monoidal categories. A
  \textbf{symmetric monoidal functor} $F:\cat{M}\rightarrow\cat{M'}$ is a monoidal functor
  $(F,\m{})$ that satisfies the following coherence diagram:
  \begin{mathpar}
  \bfig
    \square<700,400>[
      FA\otimes'FB`FB\otimes'FA`F(A\otimes B)`F(B\otimes A);
      \e{FA,FB}`\m{A,B}`\m{B,A}`F\e{A,B}]
  \efig
  \end{mathpar}
\end{definition}

\begin{definition}
  An \textbf{adjunction} between categories $\cat{C}$ and $\cat{D}$ consists of two functors
  $F:\cat{D}\rightarrow\cat{C}$, called the \textbf{left adjoint}, and
  $G:\cat{C}\rightarrow\cat{D}$, called the \textbf{right adjoint}, and two natural
  transformations $\eta:id_\cat{D}\rightarrow GF$, called the \textbf{unit}, and
  $\varepsilon:FG\rightarrow id_\cat{C}$, called the \textbf{counit}, such that the following
  diagrams commute for any object $A$ in $\cat{C}$ and $B$ in $\cat{D}$:
  \begin{mathpar}
  \bfig
    \Vtriangle/->`=`->/<400,400>[FB`FGFB`FB;F\eta_B``\varepsilon_{FB}]
  \efig
  \and
  \bfig
    \Vtriangle/->`=`->/<400,400>[GA`GFGA`GA;\eta_{GA}``G\varepsilon_A]
  \efig
  \end{mathpar}
\end{definition}

\begin{definition}
  Let $(F,\m{})$ and $(G,\n{})$ be monoidal functors from a monoidal category
  $(\cat{M},\otimes,I,\alpha,\lambda,\rho)$ to a monoidal category
  $(\cat{M'},\otimes',I',\alpha',\lambda',\rho')$. A \textbf{monoidal natural transformation}
  from $(F,\m{})$ to $(G,\n{})$ is a natural transformation $\theta:(F,\m{})\rightarrow(G,\n{})$
  such that the following diagrams commute for any objects $A$ and $B$ in $\cat{M}$:
  \begin{mathpar}
  \bfig
    \square<700,400>[
      FA\tri'FB`F(A\tri B)`GA\tri'GB`G(A\tri B);
      \m{A,B}`\theta_A\tri'\theta_B`\theta_{A\tri B}`\n{A,B}]
  \efig
  \and
  \bfig
    \Vtriangle/->`<-`<-/<400,400>[FI`GI`I';\theta_I`\m{I}`\n{I}]
  \efig
  \end{mathpar}
\end{definition}

\begin{definition}
  Let $(\cat{M},\tri,I,\alpha,\lambda,\rho)$ and
  $(\cat{M'},\tri',I',\alpha',\lambda',\rho')$ be monoidal categories,
  $F:\cat{M}\rightarrow\cat{M'}$ and $G:\cat{M}'\rightarrow\cat{M}$ be functors. The adjunction
  $F:\cat{M}\dashv\cat{M'}:G$ is a \textbf{monoidal adjunction} if $F$ and $G$ are monoidal
  functors, and the unit $\eta$ and the counit $\varepsilon$ are monoidal natural
  transformations.
\end{definition}

\begin{definition}
  Let $\cat{C}$ be a category. A \textbf{monad} on $\cat{C}$ consists of an endofunctor
  $T:\cat{C}\rightarrow\cat{C}$ together with two natural transformations
  $\eta:id_\cat{C}\rightarrow T$ and $\mu:T^2\rightarrow T$, where $id_\cat{C}$ is the identity
  functor on $\cat{C}$, such that the following diagrams commute:
  \begin{mathpar}
  \bfig
    \square<400,400>[T^3`T^2`T^2`T;T\mu`\mu_T`\mu`\mu]
  \efig
  \and
  \bfig
    \square<400,400>[T`T^2`T^2`T;\eta_T`T\eta`\mu`\mu]
    \morphism(0,400)/=/<400,-400>[T`T;]
  \efig
  \end{mathpar}
\end{definition}

\begin{definition}
  \label{def:strong-monad}
  Let $(\cat{M},\tri,I,\alpha,\lambda,\rho)$ be a monoidal category and $(T,\eta,\mu)$ be a
  monad on $\cat{M}$. $T$ is a \textbf{strong monad} if there is natural transformation $\tau$, 
  called the \textbf{tensorial strength}, with components
  $\tau_{A,B}:A\tri TB\rightarrow T(A\tri B)$ such that the following diagrams commute:
  \begin{mathpar}
  \bfig
    \Vtriangle<400,400>[I\tri TA`T(I\tri A)`TA;\tau_{I,A}`\lambda_{TA}`T\lambda_A]
  \efig
  \and
  \bfig
    \Vtriangle<400,400>[
      A\tri B`A\tri TB`T(A\tri B);id_A\tri\eta_B`\eta_{A\tri B}`\tau_{A,B}]
  \efig
  \and
  \bfig
    \square/->`->`->`/<1800,400>[
      (A\tri B)\tri TC`T((A\tri B)\tri C)`
      A\tri(B\tri TC)`T(A\tri(B\tri C));
      \tau_{A\tri B,C}`\alpha_{A,B,TC}`T\alpha_{A,B,C}`]
    \morphism<900,0>[A\tri(B\tri TC)`A\tri T(B\tri C);id_A\tri\tau_{B,C}]
    \morphism(900,0)<900,0>[A\tri T(B\tri C)`T(A\tri(B\tri C));\tau_{A,B\tri C}]
  \efig
  \and
  \bfig
    \square/`->`->`->/<1400,400>[
      A\tri T^2B`T^2(A\tri B)`A\tri TB`T(A\tri B);
      `id_A\tri\mu_B`\mu_{A\tri B}`\tau_{A,B}]
    \morphism(0,400)<700,0>[A\tri T^2B`T(A\tri TB);\tau_{A,TB}]
    \morphism(700,400)<700,0>[T(A\tri TB)`T^2(A\tri B);T\tau_{A,B}]
  \efig
  \end{mathpar}
\end{definition}

\begin{definition}
  Let $(\cat{M},\otimes,I,\alpha,\lambda,\rho)$ be a symmetric monoidal category with exchange
  $\e{}$, and $(T,\eta,\mu)$ be a strong monad on $\cat{M}$. Then there is a \textbf{``twisted''
  tensorial strength} $\tau'_{A,B}:TA\otimes B\rightarrow T(A\otimes B)$ defined as
  $\tau'_{A,B}=T\e{}\circ\tau_{B,A}\circ\e{}$. We can construct a pair of natural
  transformations $\Phi$, $\Phi'$ with components
  $\Phi_{A,B},\Phi'_{A,B}:TA\otimes TB\rightarrow T(A\otimes B)$ defined as
  $\Phi_{A,B}=\mu_{A\otimes B}\circ T\tau'_{A,B}\circ\tau_{TA,B}$ and
  $\Phi'_{A,B}=\mu_{A\otimes B}\circ T\tau_{A,B}\circ\tau'_{A,TB}$. If $\Phi=\Phi'$, then the
  monad $T$ is \textbf{commutative}.
\end{definition}

\begin{definition}
  Let $\cat{L}$ be a category. A \textbf{comonad} on $\cat{L}$ consists of an endofunctor
  $S:\cat{L}\rightarrow\cat{L}$ together with two natural transformations
  $\varepsilon:S\rightarrow id_\cat{L}$ and $\delta:S^2\rightarrow S$ such that the following
  diagrams commute:
  \begin{mathpar}
  \bfig
    \square<400,400>[S`S^2`S^2`S^3;\delta`\delta`S\delta`\delta_S]
  \efig
  \and
  \bfig
    \square<400,400>[S^2`S`S`S^2;S\varepsilon`\varepsilon_S`\delta`\delta]
  \efig
  \end{mathpar}
\end{definition}

