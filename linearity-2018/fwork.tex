We introduce the idea above of having a modality for exchange, but
what about individual modalities for weakening and contraction?
Indeed it is possible to give modalities for these structural rules as
well using adjoint models.  Now that we have each structural rule
isolated into their own modality is it possible to put them together
to form new modalities that combine structural rules?  The answer to
this question has already been shown to be positive, at least for
weakening and contraction, by Melli{\'e}s~\cite{Mellies:2004}, but we
plan to extend this line of work to include exchange.

The monads induced by the adjunction in CNC logic is non-commutative,
but Benton and Wadler show that the monads induced by the adjunction
in LNL logic \cite{Benton:1996} are commutative.  Using the extension
of Melli{\'e}s' work we mention above would allow us to combine both
CNC logic with LNL logic, and then be able to support both commutative
monads as well as non-commutative monads.  We plan on exploring this
in the future.

Hasegawa~\cite{EPTCS238.6} studies the linear of-course modality,
$!A$, as a comonad induced by an adjunction between a cartesian closed
category a (non-symmetric) monoidal category.  The results here
generalizes his by generalizing the cartesian closed category to a
symmetric monoidal closed category.  However, his approach focuses on
the comonad rather than the adjunctions.  It would be interesting to
do the same for LAM as well.
