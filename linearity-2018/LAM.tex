Joachim Lambek first introduced the Syntactic Calculus, now known as
the Lambek Calculus, in 1958 \cite{Lambek1958}.  Since then the Lambek
Calculus has largely been motivated by providing an explanation of the
mathematics of sentence structure, and can be found at the core of
Categorial Grammar; a term first used in the title of Bar-Hillel,
Gaifman and Shamir (1960), but categorical grammar began with
Ajdukiewicz (1935) quite a few years earlier. For many years the
Lambek Calculus went without use, but around 1980 it was taken up by
logicians interested in Computational Linguistics, especially the ones
interested in Categorial Grammars.

In Computational Linguistics, the Lambek Calculus has seen a
significant number of works written about it, apart from a number of
monographs that deal with logical and linguistic aspects of the
generalized type-logical approach.  For a shorter introduction, see
Moortgat's entry on the Stanford Encyclopedia of Philosophy on Type
Logical Grammar \cite{MoortgatSEP}.  Type Logical Grammar situates the
type-logical approach within the framework of Montague's Universal
Grammar and presents detailed linguistic analyses for a substantive
fragment of syntactic and semantic phenomena in the grammar of
English.  Type Logical Semantics offers a general introduction to
natural language semantics studied from a type-logical perspective.

The Lambek Calculus also has the potential for many applications in
other areas of computer science, such as, modeling processes.  Linear
Logic has been at the forefront of the study of process calculi for
many years \cite{HONDA20102223,Pratt:1997,ABRAMSKY19945}. We can think
of the commutative tensor product of linear logic as a parallel
operator.  For example, given a process $[[A]]$ and a process $[[B]]$,
then we can form the process $[[A (x) B]]$ which runs both processes
in parallel.  If we remove commutativity from the tensor product we
obtain a sequential composition instead of parallel composition.  That
is, the process $[[A (>) B]]$ first runs process $[[A]]$ and then
process $[[B]]$ in that order.  Paraphrasing Vaughan Pratt, ``The
sequential composition operation has no evident counterpart in type
theory'' see page 11 of \cite{Pratt:1997}.  We believe that the Lambek
Calculus will lead to filling this hole, and the results of this paper
as a means of obtaining a theory with both a parallel operator and a
sequential composition operator.  This work thus has a potential to
impact research in programming languages and computer security where
both linear logic and sequential composition play important roles.

