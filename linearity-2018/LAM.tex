The Syntactic Calculus was first introduced by
Joachim Lambek in 1958 \cite{Lambek1958}. Since then the rechristened
Lambek Calculus has had as its main motivation providing an
explanation of the mathematics of sentence structure, starting from
the author's algebraic intuitions. The Lambek Calculus is the core of
logical Categorial Grammar.  The first use of the term “categorial
grammar” seems to be in the title of Bar-Hillel, Gaifman and Shamir
(1960), but categorial grammar began with Ajdukiewicz (1935) quite a
few years earlier. After a period of ostracism, around 1980 the Lambek
Calculus was taken up by logicians interested in Computational
Linguistics, especially the ones interested in Categorial
Grammars.

In Computational Linguistics, the Lambek Calculus has seen a
significant number of works written about it, apart from a number of
monographs that deal with logical and linguistic aspects of the
generalized type-logical approach.  For a shorter introduction, see
Moortgat's entry on the Stanford Encyclopedia of Philosophy on Type
Logical Grammar \cite{MoortgatSEP}.  Type Logical Grammar situates the
type-logical approach within the framework of Montague's Universal
Grammar and presents detailed linguistic analyses for a substantive
fragment of syntactic and semantic phenomena in the grammar of
English.  Type Logical Semantics offers a general introduction to
natural language semantics studied from a type-logical perspective.

The Lambek Calculus also has the potential for many applications in
other areas of computer science, such as, modeling processes.  Linear
Logic has been at the forefront of the study of process calculi for
many years \cite{HONDA20102223,Pratt:1997,ABRAMSKY19945}. We can think
of the commutative tensor product of linear logic as a parallel
operator.  For example, given a process $[[A]]$ and a process $[[B]]$,
then we can form the process $[[A (x) B]]$ which runs both processes
in parallel.  If we remove commutativity from the tensor product we
obtain a sequential composition instead of parallel composition.  That
is, the process $[[A (>) B]]$ first runs process $[[A]]$ and then
process $[[B]]$ in that order.  Paraphrasing Vaughan Pratt, ``The
sequential composition operation has no evident counterpart in type
theory'' see page 11 of \cite{Pratt:1997}.  We believe that the Lambek
Calculus will lead to filling this hole, and the results of this paper
as a means of obtaining a theory with both a parallel operator and a
sequential composition operator.  This work thus has a potential to
impact research in programming languages and computer security where
both linear logic and sequential composition play important roles.

There are several interesting questions, considered for Linear Logic,
that could also be asked of the Lambek Calculus or its extensions.
One of them, posed by Morrill et al is whether we can extend the
Lambek Calculus with a modality that does for the structural rule of
\textit{(exchange)} what the modality \textit{of course} `!' does for
the rules of \textit{(weakening)} and \textit{(contraction)}.  A
preliminary proposal, which answers this question affirmatively, is
set forward in this paper. The answer was provided in semantical terms
in the first version of this work. Here we provide also the more
syntactic description of these modalities. Building up from work of
Ciabattoni, Galatos and Terui in \cite{Ciabattoni2012} and others that
describe how to transform systems of axioms into cut-free sequent
rules, we aim to refine the algebraization of proof theory.

