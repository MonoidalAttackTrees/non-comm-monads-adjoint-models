We first introduce adjoint models, and the two particular adjoint
models that will be used to model the Lambek Calculus with and without
associativity.  However, before we can do this we have to define a
very simple categorical structure known as a magmoidal category with a
unit.
\begin{definition}
  \label{def:magmoidal-categories}
  A \textbf{magmoidal category with a unit (MCU)}, $(\cat{M}, \otimes,
  I, u_l,u_r)$, is a category $\cat{M}$ with a distinguished object,
  $I$, called the unit, a bifunctor $\otimes : \cat{M} \times \cat{M}
  \to \cat{M}$, and two natural isomorphisms $u_l : I \otimes A \mto
  A$ and $u_r : A \otimes I \mto A$ called the left and right unitors
  respectively.
\end{definition}
\noindent
A MCU is then a generalization of a monoidal category by removing the
associator.  This is necessary in order to model the non-associative
Lambek calculus.

