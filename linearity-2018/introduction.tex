Joachim Lambek first introduced the Syntactic Calculus, now known as
the Lambek Calculus, in 1958 \cite{Lambek1958}.  Since then the Lambek
Calculus has largely been motivated by providing an explanation of the
mathematics of sentence structure, and can be found at the core of
Categorial Grammar; a term first used in the title of Bar-Hillel,
Gaifman and Shamir (1960), but categorical grammar began with
Ajdukiewicz (1935) quite a few years earlier. For many years the
Lambek Calculus went without use, but around 1980 it was taken up by
logicians interested in Computational Linguistics, especially the ones
interested in Categorial Grammars.

It was computational linguists, ???, who posed the question of whether
it is possible to isolate exchange using a modality in the same way
that the of-course modality of linear logic, $!A$, isolates weakening
and contraction.  de Paiva and Eades III \cite{dePaiva2018} propose
one solution to this problem by extending the Lambek calculus with the
modality characterized by the following sequent calculus inference
rules:
\[
\small
\begin{array}{ccccccccccccccccccccc}  
  \LdruleEr{} & & \LdruleEl{} & & \LdruleEOne{} & & \LdruleETwo{} 
\end{array}
\]
The thing to note is that the modality $\kappa A$ appears on only one
of the operands being exchanged.  That is, these rules along with
those for the tensor product allow one to prove that $\kappa A \otimes
B \multimap B \otimes \kappa A$ holds.  This is somewhat at odds with
algebraic intuition, and it is unclear how this modality could be
decomposed into adjoint functors in a linear/non-linear (LNL)
formalization of the Lambek calculus.

In this paper we show how to add an exchange modality, $eA$, where the
modality now occurs on both operands being exchanged. That is, one can
show that $eA \otimes eB \multimap eB \otimes eA$ holds.  We give a
LNL natural deduction formalization for the Lambek calculus with this
new modality, and two categorical models: a LNL model and a concrete
model in dialectica spaces.  Thus giving a second solution to the
problem proposed above.

The Lambek Calculus also has the potential for many applications in
other areas of computer science, such as, modeling processes.  Linear
Logic has been at the forefront of the study of process calculi for
many years \cite{HONDA20102223,Pratt:1997,ABRAMSKY19945}. We can think
of the commutative tensor product of linear logic as a parallel
operator.  For example, given a process $A$ and a process $B$, then we
can form the process $A \otimes B$ which runs both processes in
parallel.  If we remove commutativity from the tensor product we
obtain a sequential composition instead of parallel composition.  That
is, the process $A \rhd B$ first runs process $A$ and then process $B$
in that order.  Vaughan Pratt has stated that , ``sequential
composition has no evident counterpart in type theory'' see page 11 of
\cite{Pratt:1997}.  We believe that the Lambek Calculus will lead to
filling this hole.  

We introduce the idea above of having a modality for exchange, but
what about a individual modalities for weakening and contraction?
Indeed it is possible to give modalities for these structural rules as
well using adjoint models.  Now that we have each structural rule
isolated into their own modality is it possible to put them together
to form new modalities that combine structural rules?  The answer to
this question has already been show to be positive, at least for
weakening and contraction, but we extend this line of work to include
exchange, and in the future associativity as well.
Jacobs~\cite{JACOBS199473} used monads and distributive laws to relate
modalities, but Melli{\'e}s~\cite{Mellies:2004} shows how to use
adjunctions to accomplish the same, but in a more intuitive and
natural way.  Thus, this part of our work is a natural extension of
Melli{\'e}s'.

This work can be used to define a general core logic that is capable
of encoding several different substructural logics.  Some examples are
intuitionistic linear logic, affine logic, contraction logic, and the
Lambek Calculus.  This does have some similarities to Licata et
al.~\cite{licata2017fibrational}, but there they embrace the tree
based structures with multiple different branching node for contexts
from the logic of bunched implications.  However, our work is trying
to understand if this is truly necessary, and if adjunctions provide a
means of, at the very least, hiding these multiple branching nodes.
Thus, adjunctions may lead to a simplification.

%%% Local Variables: 
%%% mode: latex
%%% TeX-master: main.tex
%%% End:
