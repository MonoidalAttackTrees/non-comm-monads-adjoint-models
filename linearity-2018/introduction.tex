Perhaps the most elegant model of intuitionistic linear logic is
Benton's linear/non-linear (LNL) models \cite{Benton:1994}. On the
semantic side, LNL models are very compact and intuitive, but on the
syntactic side, they correspond to a logic and type theory, called LNL
logic, that allows one to reason with or without weakening and
contraction without the need to annotate every formula of the logic
with the of-course!  modality.  This is possible, because formulas
like $A \to B$ are primitives of the logic, and not encoded using the
of-course! modality.  Therefore, we ask the question, ``can a similar
elegant, intuitive, and flexible model and corresponding logic be
defined for the Lambek Calculus?''

LNL models are symmetric monoidal adjunctions, $\cat{C} : F \dashv G :
\cat{I}$, where $\cat{C}$ is a symmetric monoidal closed category with
weakening and contraction, hence is a cartesian closed category, and
$\cat{I}$ is a symmetric monoidal closed category.  Thus, a LNL model
consists of a model of intuitionistic logic on the left and a model of
intuitionistic linear logic on the right related via a pair of adjoint
functors.  Then Benton shows that the of-course! modality can be
recovered by $!A = F(G\,A)$.  These models suggest a more general
framework for working with structural rules.

Our main contributions in this paper are to the Lambek Calculus, but
we make use of a new more general framework that will put this work in
line with planned future work.  Suppose $\cat{M}_1$ and $\cat{M}_2$
are two categories with a bifunctor $\odot_i : \cat{M}_i \times
\cat{M}_i \mto \cat{M}_i$, a distinguished object $I_i \in
\mathsf{Obj}(\cat{M}_i)$, and the two natural isomorphisms $\lambda_A
: A \odot I_i \mto A$ and $\rho_A : I_i \odot A \mto A$.  We call
these categories \emph{magmoidal categories with a unit}.  Examples of
magmoidal categories with a unit are monoidal categories, symmetric
monoidal categories, and cartesian closed categories.  Benton's LNL
models can be generalized to magmodial categories.  Simply, take an
adjunction $\cat{M}_1 : F \dashv G : \cat{M}_2$ called an
\emph{adjoint model} where the functors $F$ and $G$ preserve the
magmoidal structure similarly to monoidal functors, but without the
coherence diagram for the associator.  Now if we add to $\cat{M}_1$
and $\cat{M}_2$ some structural rules, for example, by making
$\cat{M}_1$ a symmetric monoidal category and $\cat{M}_2$ a monoidal
category, then we obtain an adjoint model that corresponds to a logic
where intuitionistic linear logic ($\cat{M}_1$) and the Lambek
calculus ($\cat{M}_2$) are mixed similarly to LNL models.  However,
this also leads to a new solution to an existing problem.

Linguists have wondered if a modality for exchange can be added to the
Lambek Calculus similarly to how the of-course! modality adds
weakening and contraction to linear logic.  de Paiva and
Eades~\cite{?} show that this is possible by adding a modality $\kappa
A$ with the property that $(\kappa A \otimes B) \multimap (B \otimes
\kappa A)$.  The adjoint model introduced above induces the modality
$eA = F(G\,A)$ with the property $eA \otimes eB \multimap eB \otimes
eA$.  Thus providing a second solution to the problem, but we go
further, and show such a modality can be given for associativity. It
arises as the induced comonad of the adjoint model where $\cat{M}_1$
is a monoidal category, and $\cat{M}_2$ is a magmoidal category with a
unit.

%% Security, concurrency, and PL motivation
