\documentclass{beamer}

\usetheme{simple}

\usepackage{lmodern}
\usepackage[scale=2]{ccicons}

% TODO: 
%   position adjustement
%   change colours
%       

% Watermark background (simple theme)

%\setwatermark{\includegraphics[height=8cm]{img/Heckert_GNU_white.png}}


\title{On the Lambek Calculus with an Exchange Modality}
\subtitle{}
\date{July 5, 2018}
\author{Jiaming Jiang$^1$, Harley Eades III$^2$, Valeria de Paiva$^3$}
\institute{$^1$North Carolina State University; $^2$Augusta University; $^3$Nuance Communications}

\begin{document}

\maketitle


%--------------------------------------------------
% (Temporary slide)
%--------------------------------------------------
\begin{frame}{Outline 1}
\begin{itemize}
\item Of-course modality of linear logic
      \begin{enumerate}
      \item Importance/applicion areas of linear logic: linguistics, and
            process modeling
      \item Brief introduction to the of-course modality: \\
            include the weakening and contraction rules
      \item Example usage of the of-course modality from linguistics and
            process modeling
      \end{enumerate}
\item Objective
      \begin{enumerate}
      \item To isolate exchange using a modality in the same way that the
            of-course modality of linear logic, $!A$, isolates weakening
            and contraction
      \item Include example rules for exchange that look similar to the
            weakening and contraction rules for $!A$
      \end{enumerate}
\item Why do we care/why do we want to isolate exchange: example from
      process modeling
\end{itemize}
\end{frame}



%--------------------------------------------------
% (Temporary slide)
%--------------------------------------------------
\begin{frame}{Outline 2}
\begin{itemize}
\item General approach
      \begin{enumerate}
      \item Introduce the term CNC logic
      \item Brief description of CNC logic: abstracts Benton's LNL logic by
            removing the existince of the exchange structural rule
      \item A diagram for Benton's LNL logic, and LNL categorical model
      \item A diagram for CNC logic and the categorical model, compare with 
            LNL
      \end{enumerate}
\item Previous approach in isolating exchange
      \begin{itemize}
      \item Brief introduction to the inference rules of reference 8
      \item Limitations of the approach
      \end{itemize}
\item More detailed introduction to CNC logic: include some example rules
\item One slide for other results: $\beta$-reductions, commuting
      conversions, sequent calculus, natural deduction, etc
\item One slide for dialectica model
\item Conclusion ``take away'' slide
\end{itemize}
\end{frame}


%--------------------------------------------------
%--------------------------------------------------
%--------------------------------------------------



\end{document}
