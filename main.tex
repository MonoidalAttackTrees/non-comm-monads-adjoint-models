\documentclass[a4paper,UKenglish]{lipics-v2016}

\usepackage{amssymb,amsmath}
\usepackage{amsthm}
\usepackage{cmll}
\usepackage{txfonts}
\usepackage{graphicx}
\usepackage{stmaryrd}
\usepackage{todonotes}
\usepackage{mathpartir}
\usepackage{hyperref}
\usepackage{mdframed}
\usepackage[barr]{xy}
\usepackage{comment}
\usepackage{graphicx}
\usepackage[inline]{enumitem}

\usepackage{caption}
\captionsetup[figure]{name=Figure}

% generated by Ott 0.25 from: Elle-ND/Elle-ND.ott
\newcommand{\NDdrule}[4][]{{\displaystyle\frac{\begin{array}{l}#2\end{array}}{#3}\quad\NDdrulename{#4}}}
\newcommand{\NDusedrule}[1]{\[#1\]}
\newcommand{\NDpremise}[1]{ #1 \\}
\newenvironment{NDdefnblock}[3][]{ \framebox{\mbox{#2}} \quad #3 \\[0pt]}{}
\newenvironment{NDfundefnblock}[3][]{ \framebox{\mbox{#2}} \quad #3 \\[0pt]\begin{displaymath}\begin{array}{l}}{\end{array}\end{displaymath}}
\newcommand{\NDfunclause}[2]{ #1 \equiv #2 \\}
\newcommand{\NDnt}[1]{\mathit{#1}}
\newcommand{\NDmv}[1]{\mathit{#1}}
\newcommand{\NDkw}[1]{\mathbf{#1}}
\newcommand{\NDsym}[1]{#1}
\newcommand{\NDcom}[1]{\text{#1}}
\newcommand{\NDdrulename}[1]{\textsc{#1}}
\newcommand{\NDcomplu}[5]{\overline{#1}^{\,#2\in #3 #4 #5}}
\newcommand{\NDcompu}[3]{\overline{#1}^{\,#2<#3}}
\newcommand{\NDcomp}[2]{\overline{#1}^{\,#2}}
\newcommand{\NDgrammartabular}[1]{\begin{supertabular}{llcllllll}#1\end{supertabular}}
\newcommand{\NDmetavartabular}[1]{\begin{supertabular}{ll}#1\end{supertabular}}
\newcommand{\NDrulehead}[3]{$#1$ & & $#2$ & & & \multicolumn{2}{l}{#3}}
\newcommand{\NDprodline}[6]{& & $#1$ & $#2$ & $#3 #4$ & $#5$ & $#6$}
\newcommand{\NDfirstprodline}[6]{\NDprodline{#1}{#2}{#3}{#4}{#5}{#6}}
\newcommand{\NDlongprodline}[2]{& & $#1$ & \multicolumn{4}{l}{$#2$}}
\newcommand{\NDfirstlongprodline}[2]{\NDlongprodline{#1}{#2}}
\newcommand{\NDbindspecprodline}[6]{\NDprodline{#1}{#2}{#3}{#4}{#5}{#6}}
\newcommand{\NDprodnewline}{\\}
\newcommand{\NDinterrule}{\\[5.0mm]}
\newcommand{\NDafterlastrule}{\\}
\newcommand{\NDmetavars}{
\NDmetavartabular{
 $ \NDmv{vars} ,\, \NDmv{n} ,\, \NDmv{a} ,\, \NDmv{x} ,\, \NDmv{y} ,\, \NDmv{z} ,\, \NDmv{w} ,\, \NDmv{m} ,\, \NDmv{o} $ &  \\
 $ \NDmv{ivar} ,\, \NDmv{i} ,\, \NDmv{k} ,\, \NDmv{j} ,\, \NDmv{l} $ &  \\
 $ \NDmv{const} ,\, \NDmv{b} $ &  \\
}}

\newcommand{\NDA}{
\NDrulehead{\NDnt{A}  ,\ \NDnt{B}  ,\ \NDnt{C}  ,\ D}{::=}{}\NDprodnewline
\NDfirstprodline{|}{ \mathsf{B} }{}{}{}{}\NDprodnewline
\NDprodline{|}{ \mathsf{Unit} }{}{}{}{}\NDprodnewline
\NDprodline{|}{\NDnt{A}  \triangleright  \NDnt{B}}{}{}{}{}\NDprodnewline
\NDprodline{|}{\NDnt{A}  \rightharpoonup  \NDnt{B}}{}{}{}{}\NDprodnewline
\NDprodline{|}{\NDnt{A}  \leftharpoonup  \NDnt{B}}{}{}{}{}\NDprodnewline
\NDprodline{|}{\NDsym{(}  \NDnt{A}  \NDsym{)}} {\textsf{M}}{}{}{}\NDprodnewline
\NDprodline{|}{ \NDnt{A} } {\textsf{M}}{}{}{}\NDprodnewline
\NDprodline{|}{ \mathsf{F} \NDnt{X} }{}{}{}{}}

\newcommand{\NDW}{
\NDrulehead{\NDnt{W}  ,\ \NDnt{X}  ,\ \NDnt{Y}  ,\ \NDnt{Z}}{::=}{}\NDprodnewline
\NDfirstprodline{|}{ \mathsf{B} }{}{}{}{}\NDprodnewline
\NDprodline{|}{ \mathsf{Unit} }{}{}{}{}\NDprodnewline
\NDprodline{|}{\NDnt{X}  \otimes  \NDnt{Y}}{}{}{}{}\NDprodnewline
\NDprodline{|}{\NDnt{X}  \multimap  \NDnt{Y}}{}{}{}{}\NDprodnewline
\NDprodline{|}{\NDsym{(}  \NDnt{X}  \NDsym{)}} {\textsf{M}}{}{}{}\NDprodnewline
\NDprodline{|}{ \NDnt{X} } {\textsf{M}}{}{}{}\NDprodnewline
\NDprodline{|}{ \mathsf{G} \NDnt{A} }{}{}{}{}}

\newcommand{\NDT}{
\NDrulehead{\NDnt{T}}{::=}{}\NDprodnewline
\NDfirstprodline{|}{\NDnt{A}}{}{}{}{}\NDprodnewline
\NDprodline{|}{\NDnt{X}}{}{}{}{}}

\newcommand{\NDp}{
\NDrulehead{\NDnt{p}  ,\ \NDnt{q}}{::=}{}\NDprodnewline
\NDfirstprodline{|}{ \star }{}{}{}{}\NDprodnewline
\NDprodline{|}{\NDmv{x}}{}{}{}{}\NDprodnewline
\NDprodline{|}{ \mathsf{triv} }{}{}{}{}\NDprodnewline
\NDprodline{|}{ \mathsf{triv} }{}{}{}{}\NDprodnewline
\NDprodline{|}{\NDnt{p}  \otimes  \NDnt{p'}}{}{}{}{}\NDprodnewline
\NDprodline{|}{\NDnt{p}  \triangleright  \NDnt{p'}}{}{}{}{}\NDprodnewline
\NDprodline{|}{ \mathsf{F}\, \NDnt{p} }{}{}{}{}\NDprodnewline
\NDprodline{|}{ \mathsf{G}\, \NDnt{p} }{}{}{}{}}

\newcommand{\NDs}{
\NDrulehead{\NDnt{s}}{::=}{}\NDprodnewline
\NDfirstprodline{|}{\NDmv{x}}{}{}{}{}\NDprodnewline
\NDprodline{|}{\NDmv{b}}{}{}{}{}\NDprodnewline
\NDprodline{|}{ \mathsf{triv} }{}{}{}{}\NDprodnewline
\NDprodline{|}{ \mathsf{let}\, \NDnt{s_{{\mathrm{1}}}}  :  \NDnt{A} \,\mathsf{be}\, \NDnt{p} \,\mathsf{in}\, \NDnt{s_{{\mathrm{2}}}} }{}{}{}{}\NDprodnewline
\NDprodline{|}{ \mathsf{let}\, \NDnt{t}  :  \NDnt{X} \,\mathsf{be}\, \NDnt{p} \,\mathsf{in}\, \NDnt{s} }{}{}{}{}\NDprodnewline
\NDprodline{|}{\NDnt{s_{{\mathrm{1}}}}  \triangleright  \NDnt{s_{{\mathrm{2}}}}}{}{}{}{}\NDprodnewline
\NDprodline{|}{ \lambda_l  \NDmv{x}  :  \NDnt{A} . \NDnt{s} }{}{}{}{}\NDprodnewline
\NDprodline{|}{ \lambda_r  \NDmv{x}  :  \NDnt{A} . \NDnt{s} }{}{}{}{}\NDprodnewline
\NDprodline{|}{ \mathsf{app}_l\, \NDnt{s_{{\mathrm{1}}}} \, \NDnt{s_{{\mathrm{2}}}} }{}{}{}{}\NDprodnewline
\NDprodline{|}{ \mathsf{app}_r\, \NDnt{s_{{\mathrm{1}}}} \, \NDnt{s_{{\mathrm{2}}}} }{}{}{}{}\NDprodnewline
\NDprodline{|}{ \mathsf{derelict}\, \NDnt{t} }{}{}{}{}\NDprodnewline
\NDprodline{|}{ \mathsf{ex}\, \NDnt{s_{{\mathrm{1}}}} , \NDnt{s_{{\mathrm{2}}}} \,\mathsf{with}\, \NDmv{x_{{\mathrm{1}}}} , \NDmv{x_{{\mathrm{2}}}} \,\mathsf{in}\, \NDnt{s_{{\mathrm{3}}}} }{}{}{}{}\NDprodnewline
\NDprodline{|}{\NDsym{[}  \NDnt{s_{{\mathrm{1}}}}  \NDsym{/}  \NDmv{x}  \NDsym{]}  \NDnt{s_{{\mathrm{2}}}}} {\textsf{M}}{}{}{}\NDprodnewline
\NDprodline{|}{\NDsym{[}  \NDnt{t}  \NDsym{/}  \NDmv{x}  \NDsym{]}  \NDnt{s}} {\textsf{M}}{}{}{}\NDprodnewline
\NDprodline{|}{\NDsym{(}  \NDnt{s}  \NDsym{)}} {\textsf{S}}{}{}{}\NDprodnewline
\NDprodline{|}{ \NDnt{s} } {\textsf{M}}{}{}{}\NDprodnewline
\NDprodline{|}{ \mathsf{F} \NDnt{t} }{}{}{}{}}

\newcommand{\NDt}{
\NDrulehead{\NDnt{t}}{::=}{}\NDprodnewline
\NDfirstprodline{|}{\NDmv{x}}{}{}{}{}\NDprodnewline
\NDprodline{|}{\NDmv{b}}{}{}{}{}\NDprodnewline
\NDprodline{|}{ \mathsf{triv} }{}{}{}{}\NDprodnewline
\NDprodline{|}{ \mathsf{let}\, \NDnt{t_{{\mathrm{1}}}}  :  \NDnt{X} \,\mathsf{be}\, \NDnt{p} \,\mathsf{in}\, \NDnt{t_{{\mathrm{2}}}} }{}{}{}{}\NDprodnewline
\NDprodline{|}{\NDnt{t_{{\mathrm{1}}}}  \otimes  \NDnt{t_{{\mathrm{2}}}}}{}{}{}{}\NDprodnewline
\NDprodline{|}{ \lambda  \NDmv{x}  :  \NDnt{X} . \NDnt{t} }{}{}{}{}\NDprodnewline
\NDprodline{|}{ \NDnt{t_{{\mathrm{1}}}}   \NDnt{t_{{\mathrm{2}}}} }{}{}{}{}\NDprodnewline
\NDprodline{|}{ \mathsf{ex}\, \NDnt{t_{{\mathrm{1}}}} , \NDnt{t_{{\mathrm{2}}}} \,\mathsf{with}\, \NDmv{x_{{\mathrm{1}}}} , \NDmv{x_{{\mathrm{2}}}} \,\mathsf{in}\, \NDnt{t_{{\mathrm{3}}}} }{}{}{}{}\NDprodnewline
\NDprodline{|}{\NDsym{[}  \NDnt{t_{{\mathrm{1}}}}  \NDsym{/}  \NDmv{x}  \NDsym{]}  \NDnt{t_{{\mathrm{2}}}}} {\textsf{M}}{}{}{}\NDprodnewline
\NDprodline{|}{\NDsym{(}  \NDnt{t}  \NDsym{)}} {\textsf{S}}{}{}{}\NDprodnewline
\NDprodline{|}{\NDsym{h(}  \NDnt{t}  \NDsym{)}} {\textsf{M}}{}{}{}\NDprodnewline
\NDprodline{|}{ \mathsf{G}\, \NDnt{s} }{}{}{}{}}

\newcommand{\NDI}{
\NDrulehead{\Phi  ,\ \Psi}{::=}{}\NDprodnewline
\NDfirstprodline{|}{ \cdot }{}{}{}{}\NDprodnewline
\NDprodline{|}{\Phi_{{\mathrm{1}}}  \NDsym{,}  \Phi_{{\mathrm{2}}}}{}{}{}{}\NDprodnewline
\NDprodline{|}{\NDmv{x}  \NDsym{:}  \NDnt{X}}{}{}{}{}\NDprodnewline
\NDprodline{|}{\NDsym{(}  \Phi  \NDsym{)}} {\textsf{S}}{}{}{}}

\newcommand{\NDG}{
\NDrulehead{\Gamma  ,\ \Delta}{::=}{}\NDprodnewline
\NDfirstprodline{|}{ \cdot }{}{}{}{}\NDprodnewline
\NDprodline{|}{\NDmv{x}  \NDsym{:}  \NDnt{A}}{}{}{}{}\NDprodnewline
\NDprodline{|}{\Phi}{}{}{}{}\NDprodnewline
\NDprodline{|}{\Gamma_{{\mathrm{1}}}  \NDsym{;}  \Gamma_{{\mathrm{2}}}}{}{}{}{}\NDprodnewline
\NDprodline{|}{\NDsym{(}  \Gamma  \NDsym{)}} {\textsf{S}}{}{}{}}

\newcommand{\NDformula}{
\NDrulehead{\NDnt{formula}}{::=}{}\NDprodnewline
\NDfirstprodline{|}{\NDnt{judgement}}{}{}{}{}\NDprodnewline
\NDprodline{|}{ \NDnt{formula_{{\mathrm{1}}}}  \quad  \NDnt{formula_{{\mathrm{2}}}} } {\textsf{M}}{}{}{}\NDprodnewline
\NDprodline{|}{\NDnt{formula_{{\mathrm{1}}}} \, ... \, \NDnt{formula_{\NDmv{i}}}} {\textsf{M}}{}{}{}\NDprodnewline
\NDprodline{|}{ \NDnt{formula} } {\textsf{S}}{}{}{}\NDprodnewline
\NDprodline{|}{ \NDmv{x}  \not\in \mathsf{FV}( \NDnt{s} ) }{}{}{}{}\NDprodnewline
\NDprodline{|}{ \NDmv{x}  \not\in |  \Gamma ,  \Delta ,  \Psi  | }{}{}{}{}\NDprodnewline
\NDprodline{|}{ \NDmv{x}  \not\in |  \Gamma ,  \Delta  | }{}{}{}{}}

\newcommand{\NDterminals}{
\NDrulehead{\NDnt{terminals}}{::=}{}\NDprodnewline
\NDfirstprodline{|}{ \otimes }{}{}{}{}\NDprodnewline
\NDprodline{|}{ \triangleright }{}{}{}{}\NDprodnewline
\NDprodline{|}{ \circop{e} }{}{}{}{}\NDprodnewline
\NDprodline{|}{ \circop{w} }{}{}{}{}\NDprodnewline
\NDprodline{|}{ \circop{c} }{}{}{}{}\NDprodnewline
\NDprodline{|}{ \rightharpoonup }{}{}{}{}\NDprodnewline
\NDprodline{|}{ \leftharpoonup }{}{}{}{}\NDprodnewline
\NDprodline{|}{ \multimap }{}{}{}{}\NDprodnewline
\NDprodline{|}{ \vdash_\mathcal{C} }{}{}{}{}\NDprodnewline
\NDprodline{|}{ \vdash_\mathcal{L} }{}{}{}{}\NDprodnewline
\NDprodline{|}{ \leadsto_\beta }{}{}{}{}\NDprodnewline
\NDprodline{|}{ \leadsto_\mathsf{c} }{}{}{}{}}

\newcommand{\NDJtype}{
\NDrulehead{\NDnt{Jtype}}{::=}{}\NDprodnewline
\NDfirstprodline{|}{\Phi  \vdash_\mathcal{C}  \NDnt{t}  \NDsym{:}  \NDnt{X}}{}{}{}{}\NDprodnewline
\NDprodline{|}{\Gamma  \vdash_\mathcal{L}  \NDnt{s}  \NDsym{:}  \NDnt{A}}{}{}{}{}}

\newcommand{\NDReduction}{
\NDrulehead{\NDnt{Reduction}}{::=}{}\NDprodnewline
\NDfirstprodline{|}{\NDnt{t_{{\mathrm{1}}}}  \leadsto_\beta  \NDnt{t_{{\mathrm{2}}}}}{}{}{}{}\NDprodnewline
\NDprodline{|}{\NDnt{s_{{\mathrm{1}}}}  \leadsto_\beta  \NDnt{s_{{\mathrm{2}}}}}{}{}{}{}}

\newcommand{\NDCommuting}{
\NDrulehead{\NDnt{Commuting}}{::=}{}\NDprodnewline
\NDfirstprodline{|}{\NDnt{t_{{\mathrm{1}}}}  \leadsto_\mathsf{c}  \NDnt{t_{{\mathrm{2}}}}}{}{}{}{}\NDprodnewline
\NDprodline{|}{\NDnt{s_{{\mathrm{1}}}}  \leadsto_\mathsf{c}  \NDnt{s_{{\mathrm{2}}}}}{}{}{}{}}

\newcommand{\NDjudgement}{
\NDrulehead{\NDnt{judgement}}{::=}{}\NDprodnewline
\NDfirstprodline{|}{\NDnt{Jtype}}{}{}{}{}\NDprodnewline
\NDprodline{|}{\NDnt{Reduction}}{}{}{}{}\NDprodnewline
\NDprodline{|}{\NDnt{Commuting}}{}{}{}{}}

\newcommand{\NDuserXXsyntax}{
\NDrulehead{\NDnt{user\_syntax}}{::=}{}\NDprodnewline
\NDfirstprodline{|}{\NDmv{vars}}{}{}{}{}\NDprodnewline
\NDprodline{|}{\NDmv{ivar}}{}{}{}{}\NDprodnewline
\NDprodline{|}{\NDmv{const}}{}{}{}{}\NDprodnewline
\NDprodline{|}{\NDnt{A}}{}{}{}{}\NDprodnewline
\NDprodline{|}{\NDnt{W}}{}{}{}{}\NDprodnewline
\NDprodline{|}{\NDnt{T}}{}{}{}{}\NDprodnewline
\NDprodline{|}{\NDnt{p}}{}{}{}{}\NDprodnewline
\NDprodline{|}{\NDnt{s}}{}{}{}{}\NDprodnewline
\NDprodline{|}{\NDnt{t}}{}{}{}{}\NDprodnewline
\NDprodline{|}{\Phi}{}{}{}{}\NDprodnewline
\NDprodline{|}{\Gamma}{}{}{}{}\NDprodnewline
\NDprodline{|}{\NDnt{formula}}{}{}{}{}\NDprodnewline
\NDprodline{|}{\NDnt{terminals}}{}{}{}{}}

\newcommand{\NDgrammar}{\NDgrammartabular{
\NDA\NDinterrule
\NDW\NDinterrule
\NDT\NDinterrule
\NDp\NDinterrule
\NDs\NDinterrule
\NDt\NDinterrule
\NDI\NDinterrule
\NDG\NDinterrule
\NDformula\NDinterrule
\NDterminals\NDinterrule
\NDJtype\NDinterrule
\NDReduction\NDinterrule
\NDCommuting\NDinterrule
\NDjudgement\NDinterrule
\NDuserXXsyntax\NDafterlastrule
}}

% defnss
% defns Jtype
%% defn tty
\newcommand{\NDdruleTXXidName}[0]{\NDdrulename{T\_id}}
\newcommand{\NDdruleTXXid}[1]{\NDdrule[#1]{%
}{
\NDmv{x}  \NDsym{:}  \NDnt{X}  \vdash_\mathcal{C}  \NDmv{x}  \NDsym{:}  \NDnt{X}}{%
{\NDdruleTXXidName}{}%
}}


\newcommand{\NDdruleTXXunitIName}[0]{\NDdrulename{T\_unitI}}
\newcommand{\NDdruleTXXunitI}[1]{\NDdrule[#1]{%
}{
 \cdot   \vdash_\mathcal{C}   \mathsf{triv}   \NDsym{:}   \mathsf{Unit} }{%
{\NDdruleTXXunitIName}{}%
}}


\newcommand{\NDdruleTXXunitEName}[0]{\NDdrulename{T\_unitE}}
\newcommand{\NDdruleTXXunitE}[1]{\NDdrule[#1]{%
\NDpremise{ \Phi  \vdash_\mathcal{C}  \NDnt{t_{{\mathrm{1}}}}  \NDsym{:}   \mathsf{Unit}   \quad  \Psi  \vdash_\mathcal{C}  \NDnt{t_{{\mathrm{2}}}}  \NDsym{:}  \NDnt{Y} }%
}{
\Phi  \NDsym{,}  \Psi  \vdash_\mathcal{C}   \mathsf{let}\, \NDnt{t_{{\mathrm{1}}}}  :   \mathsf{Unit}  \,\mathsf{be}\,  \mathsf{triv}  \,\mathsf{in}\, \NDnt{t_{{\mathrm{2}}}}   \NDsym{:}  \NDnt{Y}}{%
{\NDdruleTXXunitEName}{}%
}}


\newcommand{\NDdruleTXXtenIName}[0]{\NDdrulename{T\_tenI}}
\newcommand{\NDdruleTXXtenI}[1]{\NDdrule[#1]{%
\NDpremise{ \Phi  \vdash_\mathcal{C}  \NDnt{t_{{\mathrm{1}}}}  \NDsym{:}  \NDnt{X}  \quad  \Psi  \vdash_\mathcal{C}  \NDnt{t_{{\mathrm{2}}}}  \NDsym{:}  \NDnt{Y} }%
}{
\Phi  \NDsym{,}  \Psi  \vdash_\mathcal{C}  \NDnt{t_{{\mathrm{1}}}}  \otimes  \NDnt{t_{{\mathrm{2}}}}  \NDsym{:}  \NDnt{X}  \otimes  \NDnt{Y}}{%
{\NDdruleTXXtenIName}{}%
}}


\newcommand{\NDdruleTXXtenEName}[0]{\NDdrulename{T\_tenE}}
\newcommand{\NDdruleTXXtenE}[1]{\NDdrule[#1]{%
\NDpremise{ \Phi  \vdash_\mathcal{C}  \NDnt{t_{{\mathrm{1}}}}  \NDsym{:}  \NDnt{X}  \otimes  \NDnt{Y}  \quad  \Psi_{{\mathrm{1}}}  \NDsym{,}  \NDmv{x}  \NDsym{:}  \NDnt{X}  \NDsym{,}  \NDmv{y}  \NDsym{:}  \NDnt{Y}  \NDsym{,}  \Psi_{{\mathrm{2}}}  \vdash_\mathcal{C}  \NDnt{t_{{\mathrm{2}}}}  \NDsym{:}  \NDnt{Z} }%
}{
\Psi_{{\mathrm{1}}}  \NDsym{,}  \Phi  \NDsym{,}  \Psi_{{\mathrm{2}}}  \vdash_\mathcal{C}   \mathsf{let}\, \NDnt{t_{{\mathrm{1}}}}  :  \NDnt{X}  \otimes  \NDnt{Y} \,\mathsf{be}\, \NDmv{x}  \otimes  \NDmv{y} \,\mathsf{in}\, \NDnt{t_{{\mathrm{2}}}}   \NDsym{:}  \NDnt{Z}}{%
{\NDdruleTXXtenEName}{}%
}}


\newcommand{\NDdruleTXXimpIName}[0]{\NDdrulename{T\_impI}}
\newcommand{\NDdruleTXXimpI}[1]{\NDdrule[#1]{%
\NDpremise{\Phi  \NDsym{,}  \NDmv{x}  \NDsym{:}  \NDnt{X}  \vdash_\mathcal{C}  \NDnt{t}  \NDsym{:}  \NDnt{Y}}%
}{
\Phi  \vdash_\mathcal{C}   \lambda  \NDmv{x}  :  \NDnt{X} . \NDnt{t}   \NDsym{:}  \NDnt{X}  \multimap  \NDnt{Y}}{%
{\NDdruleTXXimpIName}{}%
}}


\newcommand{\NDdruleTXXimpEName}[0]{\NDdrulename{T\_impE}}
\newcommand{\NDdruleTXXimpE}[1]{\NDdrule[#1]{%
\NDpremise{ \Phi  \vdash_\mathcal{C}  \NDnt{t_{{\mathrm{1}}}}  \NDsym{:}  \NDnt{X}  \multimap  \NDnt{Y}  \quad  \Psi  \vdash_\mathcal{C}  \NDnt{t_{{\mathrm{2}}}}  \NDsym{:}  \NDnt{X} }%
}{
\Phi  \NDsym{,}  \Psi  \vdash_\mathcal{C}   \NDnt{t_{{\mathrm{1}}}}   \NDnt{t_{{\mathrm{2}}}}   \NDsym{:}  \NDnt{Y}}{%
{\NDdruleTXXimpEName}{}%
}}


\newcommand{\NDdruleTXXGIName}[0]{\NDdrulename{T\_GI}}
\newcommand{\NDdruleTXXGI}[1]{\NDdrule[#1]{%
\NDpremise{\Phi  \vdash_\mathcal{L}  \NDnt{s}  \NDsym{:}  \NDnt{A}}%
}{
\Phi  \vdash_\mathcal{C}   \mathsf{G}\, \NDnt{s}   \NDsym{:}   \mathsf{G} \NDnt{A} }{%
{\NDdruleTXXGIName}{}%
}}


\newcommand{\NDdruleTXXbetaName}[0]{\NDdrulename{T\_beta}}
\newcommand{\NDdruleTXXbeta}[1]{\NDdrule[#1]{%
\NDpremise{\Phi  \NDsym{,}  \NDmv{x}  \NDsym{:}  \NDnt{X}  \NDsym{,}  \NDmv{y}  \NDsym{:}  \NDnt{Y}  \NDsym{,}  \Psi  \vdash_\mathcal{C}  \NDnt{t}  \NDsym{:}  \NDnt{Z}}%
}{
\Phi  \NDsym{,}  \NDmv{z}  \NDsym{:}  \NDnt{Y}  \NDsym{,}  \NDmv{w}  \NDsym{:}  \NDnt{X}  \NDsym{,}  \Psi  \vdash_\mathcal{C}   \mathsf{ex}\, \NDmv{w} , \NDmv{z} \,\mathsf{with}\, \NDmv{x} , \NDmv{y} \,\mathsf{in}\, \NDnt{t}   \NDsym{:}  \NDnt{Z}}{%
{\NDdruleTXXbetaName}{}%
}}


\newcommand{\NDdruleTXXcutName}[0]{\NDdrulename{T\_cut}}
\newcommand{\NDdruleTXXcut}[1]{\NDdrule[#1]{%
\NDpremise{ \Phi  \vdash_\mathcal{C}  \NDnt{t_{{\mathrm{1}}}}  \NDsym{:}  \NDnt{X}  \quad  \Psi_{{\mathrm{1}}}  \NDsym{,}  \NDmv{x}  \NDsym{:}  \NDnt{X}  \NDsym{,}  \Psi_{{\mathrm{2}}}  \vdash_\mathcal{C}  \NDnt{t_{{\mathrm{2}}}}  \NDsym{:}  \NDnt{Y} }%
}{
\Psi_{{\mathrm{1}}}  \NDsym{,}  \Phi  \NDsym{,}  \Psi_{{\mathrm{2}}}  \vdash_\mathcal{C}  \NDsym{[}  \NDnt{t_{{\mathrm{1}}}}  \NDsym{/}  \NDmv{x}  \NDsym{]}  \NDnt{t_{{\mathrm{2}}}}  \NDsym{:}  \NDnt{Y}}{%
{\NDdruleTXXcutName}{}%
}}

\newcommand{\NDdefntty}[1]{\begin{NDdefnblock}[#1]{$\Phi  \vdash_\mathcal{C}  \NDnt{t}  \NDsym{:}  \NDnt{X}$}{}
\NDusedrule{\NDdruleTXXid{}}
\NDusedrule{\NDdruleTXXunitI{}}
\NDusedrule{\NDdruleTXXunitE{}}
\NDusedrule{\NDdruleTXXtenI{}}
\NDusedrule{\NDdruleTXXtenE{}}
\NDusedrule{\NDdruleTXXimpI{}}
\NDusedrule{\NDdruleTXXimpE{}}
\NDusedrule{\NDdruleTXXGI{}}
\NDusedrule{\NDdruleTXXbeta{}}
\NDusedrule{\NDdruleTXXcut{}}
\end{NDdefnblock}}

%% defn sty
\newcommand{\NDdruleSXXidName}[0]{\NDdrulename{S\_id}}
\newcommand{\NDdruleSXXid}[1]{\NDdrule[#1]{%
}{
\NDmv{x}  \NDsym{:}  \NDnt{A}  \vdash_\mathcal{L}  \NDmv{x}  \NDsym{:}  \NDnt{A}}{%
{\NDdruleSXXidName}{}%
}}


\newcommand{\NDdruleSXXunitIName}[0]{\NDdrulename{S\_unitI}}
\newcommand{\NDdruleSXXunitI}[1]{\NDdrule[#1]{%
}{
 \cdot   \vdash_\mathcal{L}   \mathsf{triv}   \NDsym{:}   \mathsf{Unit} }{%
{\NDdruleSXXunitIName}{}%
}}


\newcommand{\NDdruleSXXunitEOneName}[0]{\NDdrulename{S\_unitE1}}
\newcommand{\NDdruleSXXunitEOne}[1]{\NDdrule[#1]{%
\NDpremise{ \Phi  \vdash_\mathcal{C}  \NDnt{t}  \NDsym{:}   \mathsf{Unit}   \quad  \Gamma  \vdash_\mathcal{L}  \NDnt{s}  \NDsym{:}  \NDnt{A} }%
}{
\Phi  \NDsym{;}  \Gamma  \vdash_\mathcal{L}   \mathsf{let}\, \NDnt{t}  :   \mathsf{Unit}  \,\mathsf{be}\,  \mathsf{triv}  \,\mathsf{in}\, \NDnt{s}   \NDsym{:}  \NDnt{A}}{%
{\NDdruleSXXunitEOneName}{}%
}}


\newcommand{\NDdruleSXXunitETwoName}[0]{\NDdrulename{S\_unitE2}}
\newcommand{\NDdruleSXXunitETwo}[1]{\NDdrule[#1]{%
\NDpremise{ \Gamma  \vdash_\mathcal{L}  \NDnt{s_{{\mathrm{1}}}}  \NDsym{:}   \mathsf{Unit}   \quad  \Delta  \vdash_\mathcal{L}  \NDnt{s_{{\mathrm{2}}}}  \NDsym{:}  \NDnt{A} }%
}{
\Gamma  \NDsym{;}  \Delta  \vdash_\mathcal{L}   \mathsf{let}\, \NDnt{s_{{\mathrm{1}}}}  :   \mathsf{Unit}  \,\mathsf{be}\,  \mathsf{triv}  \,\mathsf{in}\, \NDnt{s_{{\mathrm{2}}}}   \NDsym{:}  \NDnt{A}}{%
{\NDdruleSXXunitETwoName}{}%
}}


\newcommand{\NDdruleSXXtenIName}[0]{\NDdrulename{S\_tenI}}
\newcommand{\NDdruleSXXtenI}[1]{\NDdrule[#1]{%
\NDpremise{ \Gamma  \vdash_\mathcal{L}  \NDnt{s_{{\mathrm{1}}}}  \NDsym{:}  \NDnt{A}  \quad  \Delta  \vdash_\mathcal{L}  \NDnt{s_{{\mathrm{2}}}}  \NDsym{:}  \NDnt{B} }%
}{
\Gamma  \NDsym{;}  \Delta  \vdash_\mathcal{L}  \NDnt{s_{{\mathrm{1}}}}  \triangleright  \NDnt{s_{{\mathrm{2}}}}  \NDsym{:}  \NDnt{A}  \triangleright  \NDnt{B}}{%
{\NDdruleSXXtenIName}{}%
}}


\newcommand{\NDdruleSXXtenEOneName}[0]{\NDdrulename{S\_tenE1}}
\newcommand{\NDdruleSXXtenEOne}[1]{\NDdrule[#1]{%
\NDpremise{ \Phi  \vdash_\mathcal{C}  \NDnt{t}  \NDsym{:}  \NDnt{X}  \otimes  \NDnt{Y}  \quad  \Gamma_{{\mathrm{1}}}  \NDsym{;}  \NDmv{x}  \NDsym{:}  \NDnt{X}  \NDsym{;}  \NDmv{y}  \NDsym{:}  \NDnt{Y}  \NDsym{;}  \Gamma_{{\mathrm{2}}}  \vdash_\mathcal{L}  \NDnt{s}  \NDsym{:}  \NDnt{A} }%
}{
\Gamma_{{\mathrm{1}}}  \NDsym{;}  \Phi  \NDsym{;}  \Gamma_{{\mathrm{2}}}  \vdash_\mathcal{L}   \mathsf{let}\, \NDnt{t}  :  \NDnt{X}  \otimes  \NDnt{Y} \,\mathsf{be}\, \NDmv{x}  \otimes  \NDmv{y} \,\mathsf{in}\, \NDnt{s}   \NDsym{:}  \NDnt{A}}{%
{\NDdruleSXXtenEOneName}{}%
}}


\newcommand{\NDdruleSXXtenETwoName}[0]{\NDdrulename{S\_tenE2}}
\newcommand{\NDdruleSXXtenETwo}[1]{\NDdrule[#1]{%
\NDpremise{ \Gamma  \vdash_\mathcal{L}  \NDnt{s_{{\mathrm{1}}}}  \NDsym{:}  \NDnt{A}  \triangleright  \NDnt{B}  \quad  \Delta_{{\mathrm{1}}}  \NDsym{;}  \NDmv{x}  \NDsym{:}  \NDnt{A}  \NDsym{;}  \NDmv{y}  \NDsym{:}  \NDnt{B}  \NDsym{;}  \Delta_{{\mathrm{2}}}  \vdash_\mathcal{L}  \NDnt{s_{{\mathrm{2}}}}  \NDsym{:}  \NDnt{C} }%
}{
\Delta_{{\mathrm{1}}}  \NDsym{;}  \Gamma  \NDsym{;}  \Delta_{{\mathrm{2}}}  \vdash_\mathcal{L}   \mathsf{let}\, \NDnt{s_{{\mathrm{1}}}}  :  \NDnt{A}  \triangleright  \NDnt{B} \,\mathsf{be}\, \NDmv{x}  \triangleright  \NDmv{y} \,\mathsf{in}\, \NDnt{s_{{\mathrm{2}}}}   \NDsym{:}  \NDnt{C}}{%
{\NDdruleSXXtenETwoName}{}%
}}


\newcommand{\NDdruleSXXimprIName}[0]{\NDdrulename{S\_imprI}}
\newcommand{\NDdruleSXXimprI}[1]{\NDdrule[#1]{%
\NDpremise{\Gamma  \NDsym{;}  \NDmv{x}  \NDsym{:}  \NDnt{A}  \vdash_\mathcal{L}  \NDnt{s}  \NDsym{:}  \NDnt{B}}%
}{
\Gamma  \vdash_\mathcal{L}   \lambda_r  \NDmv{x}  :  \NDnt{A} . \NDnt{s}   \NDsym{:}  \NDnt{A}  \rightharpoonup  \NDnt{B}}{%
{\NDdruleSXXimprIName}{}%
}}


\newcommand{\NDdruleSXXimprEName}[0]{\NDdrulename{S\_imprE}}
\newcommand{\NDdruleSXXimprE}[1]{\NDdrule[#1]{%
\NDpremise{ \Gamma  \vdash_\mathcal{L}  \NDnt{s_{{\mathrm{1}}}}  \NDsym{:}  \NDnt{A}  \rightharpoonup  \NDnt{B}  \quad  \Delta  \vdash_\mathcal{L}  \NDnt{s_{{\mathrm{2}}}}  \NDsym{:}  \NDnt{A} }%
}{
\Gamma  \NDsym{;}  \Delta  \vdash_\mathcal{L}   \mathsf{app}_r\, \NDnt{s_{{\mathrm{1}}}} \, \NDnt{s_{{\mathrm{2}}}}   \NDsym{:}  \NDnt{B}}{%
{\NDdruleSXXimprEName}{}%
}}


\newcommand{\NDdruleSXXimplIName}[0]{\NDdrulename{S\_implI}}
\newcommand{\NDdruleSXXimplI}[1]{\NDdrule[#1]{%
\NDpremise{\NDmv{x}  \NDsym{:}  \NDnt{A}  \NDsym{;}  \Gamma  \vdash_\mathcal{L}  \NDnt{s}  \NDsym{:}  \NDnt{B}}%
}{
\Gamma  \vdash_\mathcal{L}   \lambda_l  \NDmv{x}  :  \NDnt{A} . \NDnt{s}   \NDsym{:}  \NDnt{B}  \leftharpoonup  \NDnt{A}}{%
{\NDdruleSXXimplIName}{}%
}}


\newcommand{\NDdruleSXXimplEName}[0]{\NDdrulename{S\_implE}}
\newcommand{\NDdruleSXXimplE}[1]{\NDdrule[#1]{%
\NDpremise{ \Gamma  \vdash_\mathcal{L}  \NDnt{s_{{\mathrm{1}}}}  \NDsym{:}  \NDnt{B}  \leftharpoonup  \NDnt{A}  \quad  \Delta  \vdash_\mathcal{L}  \NDnt{s_{{\mathrm{2}}}}  \NDsym{:}  \NDnt{A} }%
}{
\Delta  \NDsym{;}  \Gamma  \vdash_\mathcal{L}   \mathsf{app}_l\, \NDnt{s_{{\mathrm{1}}}} \, \NDnt{s_{{\mathrm{2}}}}   \NDsym{:}  \NDnt{B}}{%
{\NDdruleSXXimplEName}{}%
}}


\newcommand{\NDdruleSXXFIName}[0]{\NDdrulename{S\_FI}}
\newcommand{\NDdruleSXXFI}[1]{\NDdrule[#1]{%
\NDpremise{\Phi  \vdash_\mathcal{C}  \NDnt{t}  \NDsym{:}  \NDnt{X}}%
}{
\Phi  \vdash_\mathcal{L}   \mathsf{F} \NDnt{t}   \NDsym{:}   \mathsf{F} \NDnt{X} }{%
{\NDdruleSXXFIName}{}%
}}


\newcommand{\NDdruleSXXFEName}[0]{\NDdrulename{S\_FE}}
\newcommand{\NDdruleSXXFE}[1]{\NDdrule[#1]{%
\NDpremise{ \Gamma  \vdash_\mathcal{L}  \NDnt{s_{{\mathrm{1}}}}  \NDsym{:}   \mathsf{F} \NDnt{X}   \quad  \Delta_{{\mathrm{1}}}  \NDsym{;}  \NDmv{x}  \NDsym{:}  \NDnt{X}  \NDsym{;}  \Delta_{{\mathrm{2}}}  \vdash_\mathcal{L}  \NDnt{s_{{\mathrm{2}}}}  \NDsym{:}  \NDnt{A} }%
}{
\Delta_{{\mathrm{1}}}  \NDsym{;}  \Gamma  \NDsym{;}  \Delta_{{\mathrm{2}}}  \vdash_\mathcal{L}   \mathsf{let}\, \NDnt{s_{{\mathrm{1}}}}  :   \mathsf{F} \NDnt{X}  \,\mathsf{be}\,  \mathsf{F}\, \NDmv{x}  \,\mathsf{in}\, \NDnt{s_{{\mathrm{2}}}}   \NDsym{:}  \NDnt{A}}{%
{\NDdruleSXXFEName}{}%
}}


\newcommand{\NDdruleSXXGEName}[0]{\NDdrulename{S\_GE}}
\newcommand{\NDdruleSXXGE}[1]{\NDdrule[#1]{%
\NDpremise{\Phi  \vdash_\mathcal{C}  \NDnt{t}  \NDsym{:}   \mathsf{G} \NDnt{A} }%
}{
\Phi  \vdash_\mathcal{L}   \mathsf{derelict}\, \NDnt{t}   \NDsym{:}  \NDnt{A}}{%
{\NDdruleSXXGEName}{}%
}}


\newcommand{\NDdruleSXXbetaName}[0]{\NDdrulename{S\_beta}}
\newcommand{\NDdruleSXXbeta}[1]{\NDdrule[#1]{%
\NDpremise{\Gamma  \NDsym{;}  \NDmv{x}  \NDsym{:}  \NDnt{X}  \NDsym{;}  \NDmv{y}  \NDsym{:}  \NDnt{Y}  \NDsym{;}  \Delta  \vdash_\mathcal{L}  \NDnt{s}  \NDsym{:}  \NDnt{A}}%
}{
\Gamma  \NDsym{;}  \NDmv{z}  \NDsym{:}  \NDnt{Y}  \NDsym{;}  \NDmv{w}  \NDsym{:}  \NDnt{X}  \NDsym{;}  \Delta  \vdash_\mathcal{L}   \mathsf{ex}\, \NDmv{w} , \NDmv{z} \,\mathsf{with}\, \NDmv{x} , \NDmv{y} \,\mathsf{in}\, \NDnt{s}   \NDsym{:}  \NDnt{A}}{%
{\NDdruleSXXbetaName}{}%
}}


\newcommand{\NDdruleSXXcutOneName}[0]{\NDdrulename{S\_cut1}}
\newcommand{\NDdruleSXXcutOne}[1]{\NDdrule[#1]{%
\NDpremise{ \Phi  \vdash_\mathcal{C}  \NDnt{t}  \NDsym{:}  \NDnt{X}  \quad  \Gamma_{{\mathrm{1}}}  \NDsym{;}  \NDmv{x}  \NDsym{:}  \NDnt{X}  \NDsym{;}  \Gamma_{{\mathrm{2}}}  \vdash_\mathcal{L}  \NDnt{s}  \NDsym{:}  \NDnt{A} }%
}{
\Gamma_{{\mathrm{1}}}  \NDsym{;}  \Phi  \NDsym{;}  \Gamma_{{\mathrm{1}}}  \vdash_\mathcal{L}  \NDsym{[}  \NDnt{t}  \NDsym{/}  \NDmv{x}  \NDsym{]}  \NDnt{s}  \NDsym{:}  \NDnt{A}}{%
{\NDdruleSXXcutOneName}{}%
}}


\newcommand{\NDdruleSXXcutTwoName}[0]{\NDdrulename{S\_cut2}}
\newcommand{\NDdruleSXXcutTwo}[1]{\NDdrule[#1]{%
\NDpremise{ \Gamma  \vdash_\mathcal{L}  \NDnt{s_{{\mathrm{1}}}}  \NDsym{:}  \NDnt{A}  \quad  \Delta_{{\mathrm{1}}}  \NDsym{;}  \NDmv{x}  \NDsym{:}  \NDnt{A}  \NDsym{;}  \Delta_{{\mathrm{2}}}  \vdash_\mathcal{L}  \NDnt{s_{{\mathrm{2}}}}  \NDsym{:}  \NDnt{B} }%
}{
\Delta_{{\mathrm{1}}}  \NDsym{;}  \Gamma  \NDsym{;}  \Delta_{{\mathrm{2}}}  \vdash_\mathcal{L}  \NDsym{[}  \NDnt{s_{{\mathrm{1}}}}  \NDsym{/}  \NDmv{x}  \NDsym{]}  \NDnt{s_{{\mathrm{2}}}}  \NDsym{:}  \NDnt{B}}{%
{\NDdruleSXXcutTwoName}{}%
}}

\newcommand{\NDdefnsty}[1]{\begin{NDdefnblock}[#1]{$\Gamma  \vdash_\mathcal{L}  \NDnt{s}  \NDsym{:}  \NDnt{A}$}{}
\NDusedrule{\NDdruleSXXid{}}
\NDusedrule{\NDdruleSXXunitI{}}
\NDusedrule{\NDdruleSXXunitEOne{}}
\NDusedrule{\NDdruleSXXunitETwo{}}
\NDusedrule{\NDdruleSXXtenI{}}
\NDusedrule{\NDdruleSXXtenEOne{}}
\NDusedrule{\NDdruleSXXtenETwo{}}
\NDusedrule{\NDdruleSXXimprI{}}
\NDusedrule{\NDdruleSXXimprE{}}
\NDusedrule{\NDdruleSXXimplI{}}
\NDusedrule{\NDdruleSXXimplE{}}
\NDusedrule{\NDdruleSXXFI{}}
\NDusedrule{\NDdruleSXXFE{}}
\NDusedrule{\NDdruleSXXGE{}}
\NDusedrule{\NDdruleSXXbeta{}}
\NDusedrule{\NDdruleSXXcutOne{}}
\NDusedrule{\NDdruleSXXcutTwo{}}
\end{NDdefnblock}}


\newcommand{\NDdefnsJtype}{
\NDdefntty{}\NDdefnsty{}}

% defns Reduction
%% defn tred
\newcommand{\NDdruleTbetaXXletUName}[0]{\NDdrulename{Tbeta\_letU}}
\newcommand{\NDdruleTbetaXXletU}[1]{\NDdrule[#1]{%
}{
 \mathsf{let}\,  \mathsf{triv}   :   \mathsf{Unit}  \,\mathsf{be}\,  \mathsf{triv}  \,\mathsf{in}\, \NDnt{t}   \leadsto_\beta  \NDnt{t}}{%
{\NDdruleTbetaXXletUName}{}%
}}


\newcommand{\NDdruleTbetaXXletTName}[0]{\NDdrulename{Tbeta\_letT}}
\newcommand{\NDdruleTbetaXXletT}[1]{\NDdrule[#1]{%
}{
 \mathsf{let}\, \NDnt{t_{{\mathrm{1}}}}  \otimes  \NDnt{t_{{\mathrm{2}}}}  :  \NDnt{X}  \otimes  \NDnt{Y} \,\mathsf{be}\, \NDmv{x}  \otimes  \NDmv{y} \,\mathsf{in}\, \NDnt{t_{{\mathrm{3}}}}   \leadsto_\beta  \NDsym{[}  \NDnt{t_{{\mathrm{1}}}}  \NDsym{/}  \NDmv{x}  \NDsym{]}  \NDsym{[}  \NDnt{t_{{\mathrm{2}}}}  \NDsym{/}  \NDmv{y}  \NDsym{]}  \NDnt{t_{{\mathrm{3}}}}}{%
{\NDdruleTbetaXXletTName}{}%
}}


\newcommand{\NDdruleTbetaXXlamName}[0]{\NDdrulename{Tbeta\_lam}}
\newcommand{\NDdruleTbetaXXlam}[1]{\NDdrule[#1]{%
}{
 \NDsym{(}   \lambda  \NDmv{x}  :  \NDnt{X} . \NDnt{t_{{\mathrm{1}}}}   \NDsym{)}   \NDnt{t_{{\mathrm{2}}}}   \leadsto_\beta  \NDsym{[}  \NDnt{t_{{\mathrm{2}}}}  \NDsym{/}  \NDmv{x}  \NDsym{]}  \NDnt{t_{{\mathrm{1}}}}}{%
{\NDdruleTbetaXXlamName}{}%
}}


\newcommand{\NDdruleTbetaXXappOneName}[0]{\NDdrulename{Tbeta\_app1}}
\newcommand{\NDdruleTbetaXXappOne}[1]{\NDdrule[#1]{%
\NDpremise{\NDnt{t_{{\mathrm{1}}}}  \leadsto_\beta  \NDnt{t'_{{\mathrm{1}}}}}%
}{
 \NDnt{t_{{\mathrm{1}}}}   \NDnt{t_{{\mathrm{2}}}}   \leadsto_\beta   \NDnt{t'_{{\mathrm{1}}}}   \NDnt{t_{{\mathrm{2}}}} }{%
{\NDdruleTbetaXXappOneName}{}%
}}


\newcommand{\NDdruleTbetaXXappTwoName}[0]{\NDdrulename{Tbeta\_app2}}
\newcommand{\NDdruleTbetaXXappTwo}[1]{\NDdrule[#1]{%
\NDpremise{\NDnt{t_{{\mathrm{2}}}}  \leadsto_\beta  \NDnt{t'_{{\mathrm{2}}}}}%
}{
 \NDnt{t_{{\mathrm{1}}}}   \NDnt{t_{{\mathrm{2}}}}   \leadsto_\beta   \NDnt{t_{{\mathrm{1}}}}   \NDnt{t'_{{\mathrm{2}}}} }{%
{\NDdruleTbetaXXappTwoName}{}%
}}


\newcommand{\NDdruleTbetaXXappLetName}[0]{\NDdrulename{Tbeta\_appLet}}
\newcommand{\NDdruleTbetaXXappLet}[1]{\NDdrule[#1]{%
}{
 \NDsym{(}   \mathsf{let}\, \NDnt{t}  :  \NDnt{X} \,\mathsf{be}\, \NDnt{p} \,\mathsf{in}\, \NDnt{t_{{\mathrm{1}}}}   \NDsym{)}   \NDnt{t_{{\mathrm{2}}}}   \leadsto_\beta   \mathsf{let}\, \NDnt{t}  :  \NDnt{X} \,\mathsf{be}\, \NDnt{p} \,\mathsf{in}\, \NDsym{(}   \NDnt{t_{{\mathrm{1}}}}   \NDnt{t_{{\mathrm{2}}}}   \NDsym{)} }{%
{\NDdruleTbetaXXappLetName}{}%
}}


\newcommand{\NDdruleTbetaXXletLetName}[0]{\NDdrulename{Tbeta\_letLet}}
\newcommand{\NDdruleTbetaXXletLet}[1]{\NDdrule[#1]{%
}{
 \mathsf{let}\, \NDsym{(}   \mathsf{let}\, \NDnt{t_{{\mathrm{2}}}}  :  \NDnt{X} \,\mathsf{be}\, \NDnt{p_{{\mathrm{1}}}} \,\mathsf{in}\, \NDnt{t_{{\mathrm{1}}}}   \NDsym{)}  :  \NDnt{Y} \,\mathsf{be}\, \NDnt{p_{{\mathrm{2}}}} \,\mathsf{in}\, \NDnt{t_{{\mathrm{3}}}}   \leadsto_\beta   \mathsf{let}\, \NDnt{t_{{\mathrm{2}}}}  :  \NDnt{X} \,\mathsf{be}\, \NDnt{p_{{\mathrm{1}}}} \,\mathsf{in}\,  \mathsf{let}\, \NDnt{t_{{\mathrm{1}}}}  :  \NDnt{Y} \,\mathsf{be}\, \NDnt{p_{{\mathrm{2}}}} \,\mathsf{in}\, \NDnt{t_{{\mathrm{3}}}}  }{%
{\NDdruleTbetaXXletLetName}{}%
}}


\newcommand{\NDdruleTbetaXXletAppName}[0]{\NDdrulename{Tbeta\_letApp}}
\newcommand{\NDdruleTbetaXXletApp}[1]{\NDdrule[#1]{%
}{
 \mathsf{let}\, \NDnt{t_{{\mathrm{1}}}}  :  \NDnt{X} \,\mathsf{be}\, \NDnt{p} \,\mathsf{in}\, \NDsym{(}   \NDnt{t_{{\mathrm{1}}}}   \NDnt{t_{{\mathrm{2}}}}   \NDsym{)}   \leadsto_\beta   \NDsym{(}   \mathsf{let}\, \NDnt{t_{{\mathrm{1}}}}  :  \NDnt{X} \,\mathsf{be}\, \NDnt{p} \,\mathsf{in}\, \NDnt{t_{{\mathrm{1}}}}   \NDsym{)}   \NDsym{(}   \mathsf{let}\, \NDnt{t_{{\mathrm{1}}}}  :  \NDnt{X} \,\mathsf{be}\, \NDnt{p} \,\mathsf{in}\, \NDnt{t_{{\mathrm{2}}}}   \NDsym{)} }{%
{\NDdruleTbetaXXletAppName}{}%
}}

\newcommand{\NDdefntred}[1]{\begin{NDdefnblock}[#1]{$\NDnt{t_{{\mathrm{1}}}}  \leadsto_\beta  \NDnt{t_{{\mathrm{2}}}}$}{}
\NDusedrule{\NDdruleTbetaXXletU{}}
\NDusedrule{\NDdruleTbetaXXletT{}}
\NDusedrule{\NDdruleTbetaXXlam{}}
\NDusedrule{\NDdruleTbetaXXappOne{}}
\NDusedrule{\NDdruleTbetaXXappTwo{}}
\NDusedrule{\NDdruleTbetaXXappLet{}}
\NDusedrule{\NDdruleTbetaXXletLet{}}
\NDusedrule{\NDdruleTbetaXXletApp{}}
\end{NDdefnblock}}

%% defn sred
\newcommand{\NDdruleSbetaXXletUOneName}[0]{\NDdrulename{Sbeta\_letU1}}
\newcommand{\NDdruleSbetaXXletUOne}[1]{\NDdrule[#1]{%
}{
 \mathsf{let}\,  \mathsf{triv}   :   \mathsf{Unit}  \,\mathsf{be}\,  \mathsf{triv}  \,\mathsf{in}\, \NDnt{s}   \leadsto_\beta  \NDnt{s}}{%
{\NDdruleSbetaXXletUOneName}{}%
}}


\newcommand{\NDdruleSbetaXXletTOneName}[0]{\NDdrulename{Sbeta\_letT1}}
\newcommand{\NDdruleSbetaXXletTOne}[1]{\NDdrule[#1]{%
}{
 \mathsf{let}\, \NDnt{t_{{\mathrm{1}}}}  \otimes  \NDnt{t_{{\mathrm{2}}}}  :  \NDnt{X}  \otimes  \NDnt{Y} \,\mathsf{be}\, \NDmv{x}  \triangleright  \NDmv{y} \,\mathsf{in}\, \NDnt{s}   \leadsto_\beta  \NDsym{[}  \NDnt{t_{{\mathrm{1}}}}  \NDsym{/}  \NDmv{x}  \NDsym{]}  \NDsym{[}  \NDnt{t_{{\mathrm{2}}}}  \NDsym{/}  \NDmv{y}  \NDsym{]}  \NDnt{s}}{%
{\NDdruleSbetaXXletTOneName}{}%
}}


\newcommand{\NDdruleSbetaXXletTTwoName}[0]{\NDdrulename{Sbeta\_letT2}}
\newcommand{\NDdruleSbetaXXletTTwo}[1]{\NDdrule[#1]{%
}{
 \mathsf{let}\, \NDnt{s_{{\mathrm{1}}}}  \triangleright  \NDnt{s_{{\mathrm{2}}}}  :  \NDnt{A}  \triangleright  \NDnt{B} \,\mathsf{be}\, \NDmv{x}  \triangleright  \NDmv{y} \,\mathsf{in}\, \NDnt{s_{{\mathrm{3}}}}   \leadsto_\beta  \NDsym{[}  \NDnt{s_{{\mathrm{1}}}}  \NDsym{/}  \NDmv{x}  \NDsym{]}  \NDsym{[}  \NDnt{s_{{\mathrm{2}}}}  \NDsym{/}  \NDmv{y}  \NDsym{]}  \NDnt{s_{{\mathrm{3}}}}}{%
{\NDdruleSbetaXXletTTwoName}{}%
}}


\newcommand{\NDdruleSbetaXXletFName}[0]{\NDdrulename{Sbeta\_letF}}
\newcommand{\NDdruleSbetaXXletF}[1]{\NDdrule[#1]{%
}{
 \mathsf{let}\,  \mathsf{F} \NDnt{t}   :   \mathsf{F} \NDnt{X}  \,\mathsf{be}\,  \mathsf{F}\, \NDmv{x}  \,\mathsf{in}\, \NDnt{s}   \leadsto_\beta  \NDsym{[}  \NDnt{t}  \NDsym{/}  \NDmv{x}  \NDsym{]}  \NDnt{s}}{%
{\NDdruleSbetaXXletFName}{}%
}}


\newcommand{\NDdruleSbetaXXlamLName}[0]{\NDdrulename{Sbeta\_lamL}}
\newcommand{\NDdruleSbetaXXlamL}[1]{\NDdrule[#1]{%
}{
 \mathsf{app}_l\, \NDsym{(}   \lambda_l  \NDmv{x}  :  \NDnt{A} . \NDnt{s_{{\mathrm{1}}}}   \NDsym{)} \, \NDnt{s_{{\mathrm{2}}}}   \leadsto_\beta  \NDsym{[}  \NDnt{s_{{\mathrm{2}}}}  \NDsym{/}  \NDmv{x}  \NDsym{]}  \NDnt{s_{{\mathrm{1}}}}}{%
{\NDdruleSbetaXXlamLName}{}%
}}


\newcommand{\NDdruleSbetaXXlamRName}[0]{\NDdrulename{Sbeta\_lamR}}
\newcommand{\NDdruleSbetaXXlamR}[1]{\NDdrule[#1]{%
}{
 \mathsf{app}_r\, \NDsym{(}   \lambda_r  \NDmv{x}  :  \NDnt{A} . \NDnt{s_{{\mathrm{1}}}}   \NDsym{)} \, \NDnt{s_{{\mathrm{2}}}}   \leadsto_\beta  \NDsym{[}  \NDnt{s_{{\mathrm{2}}}}  \NDsym{/}  \NDmv{x}  \NDsym{]}  \NDnt{s_{{\mathrm{1}}}}}{%
{\NDdruleSbetaXXlamRName}{}%
}}


\newcommand{\NDdruleSbetaXXapplOneName}[0]{\NDdrulename{Sbeta\_appl1}}
\newcommand{\NDdruleSbetaXXapplOne}[1]{\NDdrule[#1]{%
\NDpremise{\NDnt{s_{{\mathrm{1}}}}  \leadsto_\beta  \NDnt{s'_{{\mathrm{1}}}}}%
}{
 \mathsf{app}_l\, \NDnt{s_{{\mathrm{1}}}} \, \NDnt{s_{{\mathrm{2}}}}   \leadsto_\beta   \mathsf{app}_l\, \NDnt{s'_{{\mathrm{1}}}} \, \NDnt{s_{{\mathrm{2}}}} }{%
{\NDdruleSbetaXXapplOneName}{}%
}}


\newcommand{\NDdruleSbetaXXapplTwoName}[0]{\NDdrulename{Sbeta\_appl2}}
\newcommand{\NDdruleSbetaXXapplTwo}[1]{\NDdrule[#1]{%
\NDpremise{\NDnt{s_{{\mathrm{2}}}}  \leadsto_\beta  \NDnt{s'_{{\mathrm{2}}}}}%
}{
 \mathsf{app}_l\, \NDnt{s_{{\mathrm{1}}}} \, \NDnt{s_{{\mathrm{2}}}}   \leadsto_\beta   \mathsf{app}_l\, \NDnt{s_{{\mathrm{1}}}} \, \NDnt{s'_{{\mathrm{2}}}} }{%
{\NDdruleSbetaXXapplTwoName}{}%
}}


\newcommand{\NDdruleSbetaXXapprOneName}[0]{\NDdrulename{Sbeta\_appr1}}
\newcommand{\NDdruleSbetaXXapprOne}[1]{\NDdrule[#1]{%
\NDpremise{\NDnt{s_{{\mathrm{1}}}}  \leadsto_\beta  \NDnt{s'_{{\mathrm{1}}}}}%
}{
 \mathsf{app}_r\, \NDnt{s_{{\mathrm{1}}}} \, \NDnt{s_{{\mathrm{2}}}}   \leadsto_\beta   \mathsf{app}_r\, \NDnt{s'_{{\mathrm{1}}}} \, \NDnt{s_{{\mathrm{2}}}} }{%
{\NDdruleSbetaXXapprOneName}{}%
}}


\newcommand{\NDdruleSbetaXXapprTwoName}[0]{\NDdrulename{Sbeta\_appr2}}
\newcommand{\NDdruleSbetaXXapprTwo}[1]{\NDdrule[#1]{%
\NDpremise{\NDnt{s_{{\mathrm{2}}}}  \leadsto_\beta  \NDnt{s'_{{\mathrm{2}}}}}%
}{
 \mathsf{app}_r\, \NDnt{s_{{\mathrm{1}}}} \, \NDnt{s_{{\mathrm{2}}}}   \leadsto_\beta   \mathsf{app}_r\, \NDnt{s_{{\mathrm{1}}}} \, \NDnt{s'_{{\mathrm{2}}}} }{%
{\NDdruleSbetaXXapprTwoName}{}%
}}


\newcommand{\NDdruleSbetaXXderelictName}[0]{\NDdrulename{Sbeta\_derelict}}
\newcommand{\NDdruleSbetaXXderelict}[1]{\NDdrule[#1]{%
}{
 \mathsf{derelict}\, \NDsym{(}   \mathsf{G}\, \NDnt{s}   \NDsym{)}   \leadsto_\beta  \NDnt{s}}{%
{\NDdruleSbetaXXderelictName}{}%
}}


\newcommand{\NDdruleSbetaXXapplLetName}[0]{\NDdrulename{Sbeta\_applLet}}
\newcommand{\NDdruleSbetaXXapplLet}[1]{\NDdrule[#1]{%
}{
 \mathsf{app}_l\, \NDsym{(}   \mathsf{let}\, \NDnt{s}  :  \NDnt{A} \,\mathsf{be}\, \NDnt{p} \,\mathsf{in}\, \NDnt{s_{{\mathrm{1}}}}   \NDsym{)} \, \NDnt{s_{{\mathrm{2}}}}   \leadsto_\beta   \mathsf{let}\, \NDnt{s}  :  \NDnt{A} \,\mathsf{be}\, \NDnt{p} \,\mathsf{in}\, \NDsym{(}   \mathsf{app}_l\, \NDnt{s_{{\mathrm{1}}}} \, \NDnt{s_{{\mathrm{2}}}}   \NDsym{)} }{%
{\NDdruleSbetaXXapplLetName}{}%
}}


\newcommand{\NDdruleSbetaXXapprLetName}[0]{\NDdrulename{Sbeta\_apprLet}}
\newcommand{\NDdruleSbetaXXapprLet}[1]{\NDdrule[#1]{%
}{
 \mathsf{app}_r\, \NDsym{(}   \mathsf{let}\, \NDnt{s}  :  \NDnt{A} \,\mathsf{be}\, \NDnt{p} \,\mathsf{in}\, \NDnt{s_{{\mathrm{1}}}}   \NDsym{)} \, \NDnt{s_{{\mathrm{2}}}}   \leadsto_\beta   \mathsf{let}\, \NDnt{s}  :  \NDnt{A} \,\mathsf{be}\, \NDnt{p} \,\mathsf{in}\, \NDsym{(}   \mathsf{app}_r\, \NDnt{s_{{\mathrm{1}}}} \, \NDnt{s_{{\mathrm{2}}}}   \NDsym{)} }{%
{\NDdruleSbetaXXapprLetName}{}%
}}


\newcommand{\NDdruleSbetaXXletLetName}[0]{\NDdrulename{Sbeta\_letLet}}
\newcommand{\NDdruleSbetaXXletLet}[1]{\NDdrule[#1]{%
}{
 \mathsf{let}\, \NDsym{(}   \mathsf{let}\, \NDnt{s_{{\mathrm{2}}}}  :  \NDnt{A} \,\mathsf{be}\, \NDnt{p_{{\mathrm{1}}}} \,\mathsf{in}\, \NDnt{s_{{\mathrm{1}}}}   \NDsym{)}  :  \NDnt{B} \,\mathsf{be}\, \NDnt{p_{{\mathrm{2}}}} \,\mathsf{in}\, \NDnt{s_{{\mathrm{3}}}}   \leadsto_\beta   \mathsf{let}\, \NDnt{s_{{\mathrm{2}}}}  :  \NDnt{A} \,\mathsf{be}\, \NDnt{p_{{\mathrm{1}}}} \,\mathsf{in}\,  \mathsf{let}\, \NDnt{s_{{\mathrm{1}}}}  :  \NDnt{B} \,\mathsf{be}\, \NDnt{p_{{\mathrm{2}}}} \,\mathsf{in}\, \NDnt{s_{{\mathrm{3}}}}  }{%
{\NDdruleSbetaXXletLetName}{}%
}}


\newcommand{\NDdruleSbetaXXletApplName}[0]{\NDdrulename{Sbeta\_letAppl}}
\newcommand{\NDdruleSbetaXXletAppl}[1]{\NDdrule[#1]{%
}{
 \mathsf{let}\, \NDnt{s_{{\mathrm{1}}}}  :  \NDnt{A} \,\mathsf{be}\, \NDnt{p} \,\mathsf{in}\, \NDsym{(}   \mathsf{app}_l\, \NDnt{s_{{\mathrm{1}}}} \, \NDnt{s_{{\mathrm{2}}}}   \NDsym{)}   \leadsto_\beta   \mathsf{app}_l\, \NDsym{(}   \mathsf{let}\, \NDnt{s_{{\mathrm{1}}}}  :  \NDnt{A} \,\mathsf{be}\, \NDnt{p} \,\mathsf{in}\, \NDnt{s_{{\mathrm{1}}}}   \NDsym{)} \, \NDsym{(}   \mathsf{let}\, \NDnt{s_{{\mathrm{1}}}}  :  \NDnt{A} \,\mathsf{be}\, \NDnt{p} \,\mathsf{in}\, \NDnt{s_{{\mathrm{2}}}}   \NDsym{)} }{%
{\NDdruleSbetaXXletApplName}{}%
}}


\newcommand{\NDdruleSbetaXXletApprName}[0]{\NDdrulename{Sbeta\_letAppr}}
\newcommand{\NDdruleSbetaXXletAppr}[1]{\NDdrule[#1]{%
}{
 \mathsf{let}\, \NDnt{s_{{\mathrm{1}}}}  :  \NDnt{A} \,\mathsf{be}\, \NDnt{p} \,\mathsf{in}\, \NDsym{(}   \mathsf{app}_r\, \NDnt{s_{{\mathrm{1}}}} \, \NDnt{s_{{\mathrm{2}}}}   \NDsym{)}   \leadsto_\beta   \mathsf{app}_r\, \NDsym{(}   \mathsf{let}\, \NDnt{s_{{\mathrm{1}}}}  :  \NDnt{A} \,\mathsf{be}\, \NDnt{p} \,\mathsf{in}\, \NDnt{s_{{\mathrm{1}}}}   \NDsym{)} \, \NDsym{(}   \mathsf{let}\, \NDnt{s_{{\mathrm{1}}}}  :  \NDnt{A} \,\mathsf{be}\, \NDnt{p} \,\mathsf{in}\, \NDnt{s_{{\mathrm{2}}}}   \NDsym{)} }{%
{\NDdruleSbetaXXletApprName}{}%
}}

\newcommand{\NDdefnsred}[1]{\begin{NDdefnblock}[#1]{$\NDnt{s_{{\mathrm{1}}}}  \leadsto_\beta  \NDnt{s_{{\mathrm{2}}}}$}{}
\NDusedrule{\NDdruleSbetaXXletUOne{}}
\NDusedrule{\NDdruleSbetaXXletTOne{}}
\NDusedrule{\NDdruleSbetaXXletTTwo{}}
\NDusedrule{\NDdruleSbetaXXletF{}}
\NDusedrule{\NDdruleSbetaXXlamL{}}
\NDusedrule{\NDdruleSbetaXXlamR{}}
\NDusedrule{\NDdruleSbetaXXapplOne{}}
\NDusedrule{\NDdruleSbetaXXapplTwo{}}
\NDusedrule{\NDdruleSbetaXXapprOne{}}
\NDusedrule{\NDdruleSbetaXXapprTwo{}}
\NDusedrule{\NDdruleSbetaXXderelict{}}
\NDusedrule{\NDdruleSbetaXXapplLet{}}
\NDusedrule{\NDdruleSbetaXXapprLet{}}
\NDusedrule{\NDdruleSbetaXXletLet{}}
\NDusedrule{\NDdruleSbetaXXletAppl{}}
\NDusedrule{\NDdruleSbetaXXletAppr{}}
\end{NDdefnblock}}


\newcommand{\NDdefnsReduction}{
\NDdefntred{}\NDdefnsred{}}

% defns Commuting
%% defn tcom
\newcommand{\NDdruleTcomXXunitEXXunitEName}[0]{\NDdrulename{Tcom\_unitE\_unitE}}
\newcommand{\NDdruleTcomXXunitEXXunitE}[1]{\NDdrule[#1]{%
}{
 \mathsf{let}\, \NDsym{(}   \mathsf{let}\, \NDnt{t_{{\mathrm{2}}}}  :   \mathsf{Unit}  \,\mathsf{be}\,  \mathsf{triv}  \,\mathsf{in}\, \NDnt{t_{{\mathrm{1}}}}   \NDsym{)}  :   \mathsf{Unit}  \,\mathsf{be}\,  \mathsf{triv}  \,\mathsf{in}\, \NDnt{t_{{\mathrm{3}}}}   \leadsto_\mathsf{c}   \mathsf{let}\, \NDnt{t_{{\mathrm{2}}}}  :   \mathsf{Unit}  \,\mathsf{be}\,  \mathsf{triv}  \,\mathsf{in}\, \NDsym{(}   \mathsf{let}\, \NDnt{t_{{\mathrm{1}}}}  :   \mathsf{Unit}  \,\mathsf{be}\,  \mathsf{triv}  \,\mathsf{in}\, \NDnt{t_{{\mathrm{3}}}}   \NDsym{)} }{%
{\NDdruleTcomXXunitEXXunitEName}{}%
}}


\newcommand{\NDdruleTcomXXunitEXXtenEName}[0]{\NDdrulename{Tcom\_unitE\_tenE}}
\newcommand{\NDdruleTcomXXunitEXXtenE}[1]{\NDdrule[#1]{%
}{
 \mathsf{let}\, \NDsym{(}   \mathsf{let}\, \NDnt{t_{{\mathrm{2}}}}  :   \mathsf{Unit}  \,\mathsf{be}\,  \mathsf{triv}  \,\mathsf{in}\, \NDnt{t_{{\mathrm{1}}}}   \NDsym{)}  :  \NDnt{X}  \otimes  \NDnt{Y} \,\mathsf{be}\, \NDmv{x}  \otimes  \NDmv{y} \,\mathsf{in}\, \NDnt{t_{{\mathrm{3}}}}   \leadsto_\mathsf{c}   \mathsf{let}\, \NDnt{t_{{\mathrm{2}}}}  :   \mathsf{Unit}  \,\mathsf{be}\,  \mathsf{triv}  \,\mathsf{in}\, \NDsym{(}   \mathsf{let}\, \NDnt{t_{{\mathrm{1}}}}  :  \NDnt{X}  \otimes  \NDnt{Y} \,\mathsf{be}\, \NDmv{x}  \otimes  \NDmv{y} \,\mathsf{in}\, \NDnt{t_{{\mathrm{3}}}}   \NDsym{)} }{%
{\NDdruleTcomXXunitEXXtenEName}{}%
}}


\newcommand{\NDdruleTcomXXunitEXXimpEName}[0]{\NDdrulename{Tcom\_unitE\_impE}}
\newcommand{\NDdruleTcomXXunitEXXimpE}[1]{\NDdrule[#1]{%
}{
 \NDsym{(}   \mathsf{let}\, \NDnt{t_{{\mathrm{2}}}}  :   \mathsf{Unit}  \,\mathsf{be}\,  \mathsf{triv}  \,\mathsf{in}\, \NDnt{t_{{\mathrm{1}}}}   \NDsym{)}   \NDnt{t_{{\mathrm{3}}}}   \leadsto_\mathsf{c}   \mathsf{let}\, \NDnt{t_{{\mathrm{2}}}}  :   \mathsf{Unit}  \,\mathsf{be}\,  \mathsf{triv}  \,\mathsf{in}\, \NDsym{(}   \NDnt{t_{{\mathrm{1}}}}   \NDnt{t_{{\mathrm{3}}}}   \NDsym{)} }{%
{\NDdruleTcomXXunitEXXimpEName}{}%
}}


\newcommand{\NDdruleTcomXXtenEXXunitEName}[0]{\NDdrulename{Tcom\_tenE\_unitE}}
\newcommand{\NDdruleTcomXXtenEXXunitE}[1]{\NDdrule[#1]{%
}{
 \mathsf{let}\, \NDsym{(}   \mathsf{let}\, \NDnt{t_{{\mathrm{2}}}}  :  \NDnt{X}  \otimes  \NDnt{Y} \,\mathsf{be}\, \NDmv{x}  \otimes  \NDmv{y} \,\mathsf{in}\, \NDnt{t_{{\mathrm{1}}}}   \NDsym{)}  :   \mathsf{Unit}  \,\mathsf{be}\,  \mathsf{triv}  \,\mathsf{in}\, \NDnt{t_{{\mathrm{3}}}}   \leadsto_\mathsf{c}   \mathsf{let}\, \NDnt{t_{{\mathrm{2}}}}  :  \NDnt{X}  \otimes  \NDnt{Y} \,\mathsf{be}\, \NDmv{x}  \otimes  \NDmv{y} \,\mathsf{in}\, \NDsym{(}   \mathsf{let}\, \NDnt{t_{{\mathrm{1}}}}  :   \mathsf{Unit}  \,\mathsf{be}\,  \mathsf{triv}  \,\mathsf{in}\, \NDnt{t_{{\mathrm{3}}}}   \NDsym{)} }{%
{\NDdruleTcomXXtenEXXunitEName}{}%
}}


\newcommand{\NDdruleTcomXXtenEXXtenEName}[0]{\NDdrulename{Tcom\_tenE\_tenE}}
\newcommand{\NDdruleTcomXXtenEXXtenE}[1]{\NDdrule[#1]{%
}{
 \mathsf{let}\, \NDsym{(}   \mathsf{let}\, \NDnt{t_{{\mathrm{2}}}}  :  \NDnt{X_{{\mathrm{2}}}}  \otimes  \NDnt{Y_{{\mathrm{2}}}} \,\mathsf{be}\, \NDmv{x}  \otimes  \NDmv{y} \,\mathsf{in}\, \NDnt{t_{{\mathrm{1}}}}   \NDsym{)}  :  \NDnt{X_{{\mathrm{1}}}}  \otimes  \NDnt{Y_{{\mathrm{1}}}} \,\mathsf{be}\, \NDmv{w}  \otimes  \NDmv{z} \,\mathsf{in}\, \NDnt{t_{{\mathrm{3}}}}   \leadsto_\mathsf{c}   \mathsf{let}\, \NDnt{t_{{\mathrm{2}}}}  :  \NDnt{X_{{\mathrm{2}}}}  \otimes  \NDnt{Y_{{\mathrm{2}}}} \,\mathsf{be}\, \NDmv{x}  \otimes  \NDmv{y} \,\mathsf{in}\, \NDsym{(}   \mathsf{let}\, \NDnt{t_{{\mathrm{1}}}}  :  \NDnt{X_{{\mathrm{1}}}}  \otimes  \NDnt{Y_{{\mathrm{1}}}} \,\mathsf{be}\, \NDmv{w}  \otimes  \NDmv{z} \,\mathsf{in}\, \NDnt{t_{{\mathrm{3}}}}   \NDsym{)} }{%
{\NDdruleTcomXXtenEXXtenEName}{}%
}}


\newcommand{\NDdruleTcomXXtenEXXimpEName}[0]{\NDdrulename{Tcom\_tenE\_impE}}
\newcommand{\NDdruleTcomXXtenEXXimpE}[1]{\NDdrule[#1]{%
}{
 \NDsym{(}   \mathsf{let}\, \NDnt{t_{{\mathrm{2}}}}  :  \NDnt{X_{{\mathrm{2}}}}  \otimes  \NDnt{Y_{{\mathrm{2}}}} \,\mathsf{be}\, \NDmv{x}  \otimes  \NDmv{y} \,\mathsf{in}\, \NDnt{t_{{\mathrm{1}}}}   \NDsym{)}   \NDnt{t_{{\mathrm{3}}}}   \leadsto_\mathsf{c}   \mathsf{let}\, \NDnt{t_{{\mathrm{2}}}}  :  \NDnt{X_{{\mathrm{2}}}}  \otimes  \NDnt{Y_{{\mathrm{2}}}} \,\mathsf{be}\, \NDmv{x}  \otimes  \NDmv{y} \,\mathsf{in}\, \NDsym{(}   \NDnt{t_{{\mathrm{1}}}}   \NDnt{t_{{\mathrm{3}}}}   \NDsym{)} }{%
{\NDdruleTcomXXtenEXXimpEName}{}%
}}


\newcommand{\NDdruleTcomXXimpEXXunitEName}[0]{\NDdrulename{Tcom\_impE\_unitE}}
\newcommand{\NDdruleTcomXXimpEXXunitE}[1]{\NDdrule[#1]{%
}{
 \mathsf{let}\, \NDsym{(}   \NDnt{t_{{\mathrm{1}}}}   \NDnt{t_{{\mathrm{2}}}}   \NDsym{)}  :   \mathsf{Unit}  \,\mathsf{be}\,  \mathsf{triv}  \,\mathsf{in}\, \NDnt{t_{{\mathrm{3}}}}   \leadsto_\mathsf{c}   \NDnt{t_{{\mathrm{1}}}}   \NDsym{(}   \mathsf{let}\, \NDnt{t_{{\mathrm{2}}}}  :   \mathsf{Unit}  \,\mathsf{be}\,  \mathsf{triv}  \,\mathsf{in}\, \NDnt{t_{{\mathrm{3}}}}   \NDsym{)} }{%
{\NDdruleTcomXXimpEXXunitEName}{}%
}}

\newcommand{\NDdefntcom}[1]{\begin{NDdefnblock}[#1]{$\NDnt{t_{{\mathrm{1}}}}  \leadsto_\mathsf{c}  \NDnt{t_{{\mathrm{2}}}}$}{}
\NDusedrule{\NDdruleTcomXXunitEXXunitE{}}
\NDusedrule{\NDdruleTcomXXunitEXXtenE{}}
\NDusedrule{\NDdruleTcomXXunitEXXimpE{}}
\NDusedrule{\NDdruleTcomXXtenEXXunitE{}}
\NDusedrule{\NDdruleTcomXXtenEXXtenE{}}
\NDusedrule{\NDdruleTcomXXtenEXXimpE{}}
\NDusedrule{\NDdruleTcomXXimpEXXunitE{}}
\end{NDdefnblock}}

%% defn scom
\newcommand{\NDdruleScomXXunitEXXunitEName}[0]{\NDdrulename{Scom\_unitE\_unitE}}
\newcommand{\NDdruleScomXXunitEXXunitE}[1]{\NDdrule[#1]{%
}{
 \mathsf{let}\, \NDsym{(}   \mathsf{let}\, \NDnt{s_{{\mathrm{2}}}}  :   \mathsf{Unit}  \,\mathsf{be}\,  \mathsf{triv}  \,\mathsf{in}\, \NDnt{s_{{\mathrm{1}}}}   \NDsym{)}  :   \mathsf{Unit}  \,\mathsf{be}\,  \mathsf{triv}  \,\mathsf{in}\, \NDnt{s_{{\mathrm{3}}}}   \leadsto_\mathsf{c}   \mathsf{let}\, \NDnt{s_{{\mathrm{2}}}}  :   \mathsf{Unit}  \,\mathsf{be}\,  \mathsf{triv}  \,\mathsf{in}\, \NDsym{(}   \mathsf{let}\, \NDnt{s_{{\mathrm{1}}}}  :   \mathsf{Unit}  \,\mathsf{be}\,  \mathsf{triv}  \,\mathsf{in}\, \NDnt{s_{{\mathrm{3}}}}   \NDsym{)} }{%
{\NDdruleScomXXunitEXXunitEName}{}%
}}


\newcommand{\NDdruleScomXXunitETwoXXunitEName}[0]{\NDdrulename{Scom\_unitE2\_unitE}}
\newcommand{\NDdruleScomXXunitETwoXXunitE}[1]{\NDdrule[#1]{%
}{
 \mathsf{let}\, \NDsym{(}   \mathsf{let}\, \NDnt{t}  :   \mathsf{Unit}  \,\mathsf{be}\,  \mathsf{triv}  \,\mathsf{in}\, \NDnt{s_{{\mathrm{1}}}}   \NDsym{)}  :   \mathsf{Unit}  \,\mathsf{be}\,  \mathsf{triv}  \,\mathsf{in}\, \NDnt{s_{{\mathrm{2}}}}   \leadsto_\mathsf{c}   \mathsf{let}\, \NDnt{t}  :   \mathsf{Unit}  \,\mathsf{be}\,  \mathsf{triv}  \,\mathsf{in}\, \NDsym{(}   \mathsf{let}\, \NDnt{s_{{\mathrm{1}}}}  :   \mathsf{Unit}  \,\mathsf{be}\,  \mathsf{triv}  \,\mathsf{in}\, \NDnt{s_{{\mathrm{2}}}}   \NDsym{)} }{%
{\NDdruleScomXXunitETwoXXunitEName}{}%
}}


\newcommand{\NDdruleScomXXunitEXXimprEName}[0]{\NDdrulename{Scom\_unitE\_imprE}}
\newcommand{\NDdruleScomXXunitEXXimprE}[1]{\NDdrule[#1]{%
}{
 \mathsf{app}_r\, \NDsym{(}   \mathsf{let}\, \NDnt{s_{{\mathrm{2}}}}  :   \mathsf{Unit}  \,\mathsf{be}\,  \mathsf{triv}  \,\mathsf{in}\, \NDnt{s_{{\mathrm{1}}}}   \NDsym{)} \, \NDnt{s_{{\mathrm{3}}}}   \leadsto_\mathsf{c}   \mathsf{let}\, \NDnt{s_{{\mathrm{2}}}}  :   \mathsf{Unit}  \,\mathsf{be}\,  \mathsf{triv}  \,\mathsf{in}\, \NDsym{(}   \mathsf{app}_r\, \NDnt{s_{{\mathrm{1}}}} \, \NDnt{s_{{\mathrm{3}}}}   \NDsym{)} }{%
{\NDdruleScomXXunitEXXimprEName}{}%
}}


\newcommand{\NDdruleScomXXunitETwoXXimprEName}[0]{\NDdrulename{Scom\_unitE2\_imprE}}
\newcommand{\NDdruleScomXXunitETwoXXimprE}[1]{\NDdrule[#1]{%
}{
 \mathsf{app}_r\, \NDsym{(}   \mathsf{let}\, \NDnt{t}  :   \mathsf{Unit}  \,\mathsf{be}\,  \mathsf{triv}  \,\mathsf{in}\, \NDnt{s_{{\mathrm{1}}}}   \NDsym{)} \, \NDnt{s_{{\mathrm{2}}}}   \leadsto_\mathsf{c}   \mathsf{let}\, \NDnt{t}  :   \mathsf{Unit}  \,\mathsf{be}\,  \mathsf{triv}  \,\mathsf{in}\, \NDsym{(}   \mathsf{app}_r\, \NDnt{s_{{\mathrm{1}}}} \, \NDnt{s_{{\mathrm{2}}}}   \NDsym{)} }{%
{\NDdruleScomXXunitETwoXXimprEName}{}%
}}


\newcommand{\NDdruleScomXXunitEXXFEName}[0]{\NDdrulename{Scom\_unitE\_FE}}
\newcommand{\NDdruleScomXXunitEXXFE}[1]{\NDdrule[#1]{%
}{
 \mathsf{let}\, \NDsym{(}   \mathsf{let}\, \NDnt{s_{{\mathrm{2}}}}  :   \mathsf{Unit}  \,\mathsf{be}\,  \mathsf{triv}  \,\mathsf{in}\, \NDnt{s_{{\mathrm{1}}}}   \NDsym{)}  :   \mathsf{F} \NDnt{X}  \,\mathsf{be}\,  \mathsf{F}\, \NDmv{x}  \,\mathsf{in}\, \NDnt{s_{{\mathrm{3}}}}   \leadsto_\mathsf{c}   \mathsf{let}\, \NDnt{s_{{\mathrm{2}}}}  :   \mathsf{Unit}  \,\mathsf{be}\,  \mathsf{triv}  \,\mathsf{in}\, \NDsym{(}   \mathsf{let}\, \NDnt{s_{{\mathrm{1}}}}  :   \mathsf{F} \NDnt{X}  \,\mathsf{be}\,  \mathsf{F}\, \NDmv{x}  \,\mathsf{in}\, \NDnt{s_{{\mathrm{3}}}}   \NDsym{)} }{%
{\NDdruleScomXXunitEXXFEName}{}%
}}


\newcommand{\NDdruleScomXXunitETwoXXFEName}[0]{\NDdrulename{Scom\_unitE2\_FE}}
\newcommand{\NDdruleScomXXunitETwoXXFE}[1]{\NDdrule[#1]{%
}{
 \mathsf{let}\, \NDsym{(}   \mathsf{let}\, \NDnt{t}  :   \mathsf{Unit}  \,\mathsf{be}\,  \mathsf{triv}  \,\mathsf{in}\, \NDnt{s_{{\mathrm{1}}}}   \NDsym{)}  :   \mathsf{F} \NDnt{X}  \,\mathsf{be}\,  \mathsf{F}\, \NDmv{x}  \,\mathsf{in}\, \NDnt{s_{{\mathrm{2}}}}   \leadsto_\mathsf{c}   \mathsf{let}\, \NDnt{t}  :   \mathsf{Unit}  \,\mathsf{be}\,  \mathsf{triv}  \,\mathsf{in}\, \NDsym{(}   \mathsf{let}\, \NDnt{s_{{\mathrm{1}}}}  :   \mathsf{F} \NDnt{X}  \,\mathsf{be}\,  \mathsf{F}\, \NDmv{x}  \,\mathsf{in}\, \NDnt{s_{{\mathrm{2}}}}   \NDsym{)} }{%
{\NDdruleScomXXunitETwoXXFEName}{}%
}}


\newcommand{\NDdruleScomXXtenEXXunitEName}[0]{\NDdrulename{Scom\_tenE\_unitE}}
\newcommand{\NDdruleScomXXtenEXXunitE}[1]{\NDdrule[#1]{%
}{
 \mathsf{let}\, \NDsym{(}   \mathsf{let}\, \NDnt{s_{{\mathrm{2}}}}  :  \NDnt{A}  \triangleright  \NDnt{B} \,\mathsf{be}\, \NDmv{x}  \triangleright  \NDmv{y} \,\mathsf{in}\, \NDnt{s_{{\mathrm{1}}}}   \NDsym{)}  :   \mathsf{Unit}  \,\mathsf{be}\,  \mathsf{triv}  \,\mathsf{in}\, \NDnt{s_{{\mathrm{3}}}}   \leadsto_\mathsf{c}   \mathsf{let}\, \NDnt{s_{{\mathrm{2}}}}  :  \NDnt{A}  \triangleright  \NDnt{B} \,\mathsf{be}\, \NDmv{x}  \triangleright  \NDmv{y} \,\mathsf{in}\, \NDsym{(}   \mathsf{let}\, \NDnt{s_{{\mathrm{1}}}}  :   \mathsf{Unit}  \,\mathsf{be}\,  \mathsf{triv}  \,\mathsf{in}\, \NDnt{s_{{\mathrm{3}}}}   \NDsym{)} }{%
{\NDdruleScomXXtenEXXunitEName}{}%
}}


\newcommand{\NDdruleScomXXtenETwoXXunitEName}[0]{\NDdrulename{Scom\_tenE2\_unitE}}
\newcommand{\NDdruleScomXXtenETwoXXunitE}[1]{\NDdrule[#1]{%
}{
 \mathsf{let}\, \NDsym{(}   \mathsf{let}\, \NDnt{t}  :  \NDnt{X}  \otimes  \NDnt{Y} \,\mathsf{be}\, \NDmv{x}  \otimes  \NDmv{y} \,\mathsf{in}\, \NDnt{s_{{\mathrm{1}}}}   \NDsym{)}  :   \mathsf{Unit}  \,\mathsf{be}\,  \mathsf{triv}  \,\mathsf{in}\, \NDnt{s_{{\mathrm{2}}}}   \leadsto_\mathsf{c}   \mathsf{let}\, \NDnt{t}  :  \NDnt{X}  \otimes  \NDnt{Y} \,\mathsf{be}\, \NDmv{x}  \otimes  \NDmv{y} \,\mathsf{in}\, \NDsym{(}   \mathsf{let}\, \NDnt{s_{{\mathrm{1}}}}  :   \mathsf{Unit}  \,\mathsf{be}\,  \mathsf{triv}  \,\mathsf{in}\, \NDnt{s_{{\mathrm{2}}}}   \NDsym{)} }{%
{\NDdruleScomXXtenETwoXXunitEName}{}%
}}


\newcommand{\NDdruleScomXXtenEXXtenEName}[0]{\NDdrulename{Scom\_tenE\_tenE}}
\newcommand{\NDdruleScomXXtenEXXtenE}[1]{\NDdrule[#1]{%
}{
 \mathsf{let}\, \NDsym{(}   \mathsf{let}\, \NDnt{s_{{\mathrm{2}}}}  :  \NDnt{A_{{\mathrm{2}}}}  \triangleright  \NDnt{B_{{\mathrm{2}}}} \,\mathsf{be}\, \NDmv{x}  \triangleright  \NDmv{y} \,\mathsf{in}\, \NDnt{s_{{\mathrm{1}}}}   \NDsym{)}  :  \NDnt{A_{{\mathrm{1}}}}  \triangleright  \NDnt{B_{{\mathrm{1}}}} \,\mathsf{be}\, \NDmv{w}  \triangleright  \NDmv{z} \,\mathsf{in}\, \NDnt{s_{{\mathrm{3}}}}   \leadsto_\mathsf{c}   \mathsf{let}\, \NDnt{s_{{\mathrm{2}}}}  :  \NDnt{A_{{\mathrm{2}}}}  \triangleright  \NDnt{B_{{\mathrm{2}}}} \,\mathsf{be}\, \NDmv{x}  \triangleright  \NDmv{y} \,\mathsf{in}\, \NDsym{(}   \mathsf{let}\, \NDnt{s_{{\mathrm{1}}}}  :  \NDnt{A_{{\mathrm{1}}}}  \triangleright  \NDnt{B_{{\mathrm{1}}}} \,\mathsf{be}\, \NDmv{w}  \triangleright  \NDmv{z} \,\mathsf{in}\, \NDnt{s_{{\mathrm{3}}}}   \NDsym{)} }{%
{\NDdruleScomXXtenEXXtenEName}{}%
}}


\newcommand{\NDdruleScomXXtenETwoXXtenEName}[0]{\NDdrulename{Scom\_tenE2\_tenE}}
\newcommand{\NDdruleScomXXtenETwoXXtenE}[1]{\NDdrule[#1]{%
}{
 \mathsf{let}\, \NDsym{(}   \mathsf{let}\, \NDnt{t}  :  \NDnt{X}  \otimes  \NDnt{Y} \,\mathsf{be}\, \NDmv{x}  \otimes  \NDmv{y} \,\mathsf{in}\, \NDnt{s_{{\mathrm{1}}}}   \NDsym{)}  :  \NDnt{A_{{\mathrm{1}}}}  \triangleright  \NDnt{B_{{\mathrm{1}}}} \,\mathsf{be}\, \NDmv{w}  \triangleright  \NDmv{z} \,\mathsf{in}\, \NDnt{s_{{\mathrm{2}}}}   \leadsto_\mathsf{c}   \mathsf{let}\, \NDnt{t}  :  \NDnt{X}  \otimes  \NDnt{Y} \,\mathsf{be}\, \NDmv{x}  \otimes  \NDmv{y} \,\mathsf{in}\, \NDsym{(}   \mathsf{let}\, \NDnt{s_{{\mathrm{1}}}}  :  \NDnt{A_{{\mathrm{1}}}}  \triangleright  \NDnt{B_{{\mathrm{1}}}} \,\mathsf{be}\, \NDmv{w}  \triangleright  \NDmv{z} \,\mathsf{in}\, \NDnt{s_{{\mathrm{2}}}}   \NDsym{)} }{%
{\NDdruleScomXXtenETwoXXtenEName}{}%
}}


\newcommand{\NDdruleScomXXtenEXXimprEName}[0]{\NDdrulename{Scom\_tenE\_imprE}}
\newcommand{\NDdruleScomXXtenEXXimprE}[1]{\NDdrule[#1]{%
}{
 \mathsf{app}_r\, \NDsym{(}   \mathsf{let}\, \NDnt{s_{{\mathrm{2}}}}  :  \NDnt{A_{{\mathrm{2}}}}  \triangleright  \NDnt{B_{{\mathrm{2}}}} \,\mathsf{be}\, \NDmv{x}  \triangleright  \NDmv{y} \,\mathsf{in}\, \NDnt{s_{{\mathrm{1}}}}   \NDsym{)} \, \NDnt{s_{{\mathrm{3}}}}   \leadsto_\mathsf{c}   \mathsf{let}\, \NDnt{s_{{\mathrm{2}}}}  :  \NDnt{A_{{\mathrm{2}}}}  \triangleright  \NDnt{B_{{\mathrm{2}}}} \,\mathsf{be}\, \NDmv{x}  \triangleright  \NDmv{y} \,\mathsf{in}\, \NDsym{(}   \mathsf{app}_r\, \NDnt{s_{{\mathrm{1}}}} \, \NDnt{s_{{\mathrm{3}}}}   \NDsym{)} }{%
{\NDdruleScomXXtenEXXimprEName}{}%
}}


\newcommand{\NDdruleScomXXtenETwoXXimprEName}[0]{\NDdrulename{Scom\_tenE2\_imprE}}
\newcommand{\NDdruleScomXXtenETwoXXimprE}[1]{\NDdrule[#1]{%
}{
 \mathsf{app}_r\, \NDsym{(}   \mathsf{let}\, \NDnt{t}  :  \NDnt{X}  \otimes  \NDnt{Y} \,\mathsf{be}\, \NDmv{x}  \otimes  \NDmv{y} \,\mathsf{in}\, \NDnt{s_{{\mathrm{1}}}}   \NDsym{)} \, \NDnt{s_{{\mathrm{2}}}}   \leadsto_\mathsf{c}   \mathsf{let}\, \NDnt{t}  :  \NDnt{X}  \otimes  \NDnt{Y} \,\mathsf{be}\, \NDmv{x}  \otimes  \NDmv{y} \,\mathsf{in}\, \NDsym{(}   \mathsf{app}_r\, \NDnt{s_{{\mathrm{1}}}} \, \NDnt{s_{{\mathrm{2}}}}   \NDsym{)} }{%
{\NDdruleScomXXtenETwoXXimprEName}{}%
}}


\newcommand{\NDdruleScomXXtenEXXimplEName}[0]{\NDdrulename{Scom\_tenE\_implE}}
\newcommand{\NDdruleScomXXtenEXXimplE}[1]{\NDdrule[#1]{%
}{
 \mathsf{app}_l\, \NDsym{(}   \mathsf{let}\, \NDnt{s_{{\mathrm{2}}}}  :  \NDnt{A_{{\mathrm{2}}}}  \triangleright  \NDnt{B_{{\mathrm{2}}}} \,\mathsf{be}\, \NDmv{x}  \triangleright  \NDmv{y} \,\mathsf{in}\, \NDnt{s_{{\mathrm{1}}}}   \NDsym{)} \, \NDnt{s_{{\mathrm{3}}}}   \leadsto_\mathsf{c}   \mathsf{let}\, \NDnt{s_{{\mathrm{2}}}}  :  \NDnt{A_{{\mathrm{2}}}}  \triangleright  \NDnt{B_{{\mathrm{2}}}} \,\mathsf{be}\, \NDmv{x}  \triangleright  \NDmv{y} \,\mathsf{in}\, \NDsym{(}   \mathsf{app}_l\, \NDnt{s_{{\mathrm{1}}}} \, \NDnt{s_{{\mathrm{3}}}}   \NDsym{)} }{%
{\NDdruleScomXXtenEXXimplEName}{}%
}}


\newcommand{\NDdruleScomXXtenETwoXXimplEName}[0]{\NDdrulename{Scom\_tenE2\_implE}}
\newcommand{\NDdruleScomXXtenETwoXXimplE}[1]{\NDdrule[#1]{%
}{
 \mathsf{app}_l\, \NDsym{(}   \mathsf{let}\, \NDnt{t}  :  \NDnt{X}  \otimes  \NDnt{Y} \,\mathsf{be}\, \NDmv{x}  \otimes  \NDmv{y} \,\mathsf{in}\, \NDnt{s_{{\mathrm{1}}}}   \NDsym{)} \, \NDnt{s_{{\mathrm{2}}}}   \leadsto_\mathsf{c}   \mathsf{let}\, \NDnt{t}  :  \NDnt{X}  \otimes  \NDnt{Y} \,\mathsf{be}\, \NDmv{x}  \otimes  \NDmv{y} \,\mathsf{in}\, \NDsym{(}   \mathsf{app}_l\, \NDnt{s_{{\mathrm{1}}}} \, \NDnt{s_{{\mathrm{2}}}}   \NDsym{)} }{%
{\NDdruleScomXXtenETwoXXimplEName}{}%
}}


\newcommand{\NDdruleScomXXtenEXXFEName}[0]{\NDdrulename{Scom\_tenE\_FE}}
\newcommand{\NDdruleScomXXtenEXXFE}[1]{\NDdrule[#1]{%
}{
 \mathsf{let}\, \NDsym{(}   \mathsf{let}\, \NDnt{s_{{\mathrm{2}}}}  :  \NDnt{A}  \triangleright  \NDnt{B} \,\mathsf{be}\, \NDmv{x}  \triangleright  \NDmv{y} \,\mathsf{in}\, \NDnt{s_{{\mathrm{1}}}}   \NDsym{)}  :   \mathsf{F} \NDnt{X}  \,\mathsf{be}\,  \mathsf{F}\, \NDmv{z}  \,\mathsf{in}\, \NDnt{s_{{\mathrm{3}}}}   \leadsto_\mathsf{c}   \mathsf{let}\, \NDnt{s_{{\mathrm{2}}}}  :  \NDnt{A}  \triangleright  \NDnt{B} \,\mathsf{be}\, \NDmv{x}  \triangleright  \NDmv{y} \,\mathsf{in}\, \NDsym{(}   \mathsf{let}\, \NDnt{s_{{\mathrm{1}}}}  :   \mathsf{F} \NDnt{X}  \,\mathsf{be}\,  \mathsf{F}\, \NDmv{z}  \,\mathsf{in}\, \NDnt{s_{{\mathrm{3}}}}   \NDsym{)} }{%
{\NDdruleScomXXtenEXXFEName}{}%
}}


\newcommand{\NDdruleScomXXtenETwoXXFEName}[0]{\NDdrulename{Scom\_tenE2\_FE}}
\newcommand{\NDdruleScomXXtenETwoXXFE}[1]{\NDdrule[#1]{%
}{
 \mathsf{let}\, \NDsym{(}   \mathsf{let}\, \NDnt{t}  :  \NDnt{X}  \otimes  \NDnt{Y} \,\mathsf{be}\, \NDmv{x}  \otimes  \NDmv{y} \,\mathsf{in}\, \NDnt{s_{{\mathrm{1}}}}   \NDsym{)}  :   \mathsf{F} \NDnt{Z}  \,\mathsf{be}\,  \mathsf{F}\, \NDmv{z}  \,\mathsf{in}\, \NDnt{s_{{\mathrm{3}}}}   \leadsto_\mathsf{c}   \mathsf{let}\, \NDnt{t}  :  \NDnt{X}  \otimes  \NDnt{Y} \,\mathsf{be}\, \NDmv{x}  \otimes  \NDmv{y} \,\mathsf{in}\, \NDsym{(}   \mathsf{let}\, \NDnt{s_{{\mathrm{1}}}}  :   \mathsf{F} \NDnt{Z}  \,\mathsf{be}\,  \mathsf{F}\, \NDmv{z}  \,\mathsf{in}\, \NDnt{s_{{\mathrm{3}}}}   \NDsym{)} }{%
{\NDdruleScomXXtenETwoXXFEName}{}%
}}


\newcommand{\NDdruleScomXXFEXXunitEName}[0]{\NDdrulename{Scom\_FE\_unitE}}
\newcommand{\NDdruleScomXXFEXXunitE}[1]{\NDdrule[#1]{%
}{
 \mathsf{let}\, \NDsym{(}   \mathsf{let}\, \NDnt{s_{{\mathrm{2}}}}  :   \mathsf{F} \NDnt{X}  \,\mathsf{be}\,  \mathsf{F}\, \NDmv{x}  \,\mathsf{in}\, \NDnt{s_{{\mathrm{1}}}}   \NDsym{)}  :   \mathsf{Unit}  \,\mathsf{be}\,  \mathsf{triv}  \,\mathsf{in}\, \NDnt{s_{{\mathrm{3}}}}   \leadsto_\mathsf{c}   \mathsf{let}\, \NDnt{s_{{\mathrm{2}}}}  :   \mathsf{F} \NDnt{X}  \,\mathsf{be}\,  \mathsf{F}\, \NDmv{x}  \,\mathsf{in}\, \NDsym{(}   \mathsf{let}\, \NDnt{s_{{\mathrm{1}}}}  :   \mathsf{Unit}  \,\mathsf{be}\,  \mathsf{triv}  \,\mathsf{in}\, \NDnt{s_{{\mathrm{2}}}}   \NDsym{)} }{%
{\NDdruleScomXXFEXXunitEName}{}%
}}


\newcommand{\NDdruleScomXXFEXXtenEName}[0]{\NDdrulename{Scom\_FE\_tenE}}
\newcommand{\NDdruleScomXXFEXXtenE}[1]{\NDdrule[#1]{%
}{
 \mathsf{let}\, \NDsym{(}   \mathsf{let}\, \NDnt{s_{{\mathrm{2}}}}  :   \mathsf{F} \NDnt{X}  \,\mathsf{be}\,  \mathsf{F}\, \NDmv{x}  \,\mathsf{in}\, \NDnt{s_{{\mathrm{1}}}}   \NDsym{)}  :  \NDnt{A}  \triangleright  \NDnt{B} \,\mathsf{be}\, \NDmv{x}  \triangleright  \NDmv{y} \,\mathsf{in}\, \NDnt{s_{{\mathrm{3}}}}   \leadsto_\mathsf{c}   \mathsf{let}\, \NDnt{s_{{\mathrm{2}}}}  :   \mathsf{F} \NDnt{X}  \,\mathsf{be}\,  \mathsf{F}\, \NDmv{x}  \,\mathsf{in}\, \NDsym{(}   \mathsf{let}\, \NDnt{s_{{\mathrm{1}}}}  :  \NDnt{A}  \triangleright  \NDnt{B} \,\mathsf{be}\, \NDmv{x}  \triangleright  \NDmv{y} \,\mathsf{in}\, \NDnt{s_{{\mathrm{3}}}}   \NDsym{)} }{%
{\NDdruleScomXXFEXXtenEName}{}%
}}


\newcommand{\NDdruleScomXXFEXXimprEName}[0]{\NDdrulename{Scom\_FE\_imprE}}
\newcommand{\NDdruleScomXXFEXXimprE}[1]{\NDdrule[#1]{%
}{
 \mathsf{app}_r\, \NDsym{(}   \mathsf{let}\, \NDnt{s_{{\mathrm{2}}}}  :   \mathsf{F} \NDnt{X}  \,\mathsf{be}\,  \mathsf{F}\, \NDmv{x}  \,\mathsf{in}\, \NDnt{s_{{\mathrm{1}}}}   \NDsym{)} \, \NDnt{s_{{\mathrm{3}}}}   \leadsto_\mathsf{c}   \mathsf{let}\, \NDnt{s_{{\mathrm{2}}}}  :   \mathsf{F} \NDnt{X}  \,\mathsf{be}\,  \mathsf{F}\, \NDmv{x}  \,\mathsf{in}\, \NDsym{(}   \mathsf{app}_r\, \NDnt{s_{{\mathrm{1}}}} \, \NDnt{s_{{\mathrm{3}}}}   \NDsym{)} }{%
{\NDdruleScomXXFEXXimprEName}{}%
}}


\newcommand{\NDdruleScomXXFEXXimplEName}[0]{\NDdrulename{Scom\_FE\_implE}}
\newcommand{\NDdruleScomXXFEXXimplE}[1]{\NDdrule[#1]{%
}{
 \mathsf{app}_l\, \NDsym{(}   \mathsf{let}\, \NDnt{s_{{\mathrm{2}}}}  :   \mathsf{F} \NDnt{X}  \,\mathsf{be}\,  \mathsf{F}\, \NDmv{x}  \,\mathsf{in}\, \NDnt{s_{{\mathrm{1}}}}   \NDsym{)} \, \NDnt{s_{{\mathrm{3}}}}   \leadsto_\mathsf{c}   \mathsf{let}\, \NDnt{s_{{\mathrm{2}}}}  :   \mathsf{F} \NDnt{X}  \,\mathsf{be}\,  \mathsf{F}\, \NDmv{x}  \,\mathsf{in}\, \NDsym{(}   \mathsf{app}_l\, \NDnt{s_{{\mathrm{1}}}} \, \NDnt{s_{{\mathrm{3}}}}   \NDsym{)} }{%
{\NDdruleScomXXFEXXimplEName}{}%
}}


\newcommand{\NDdruleScomXXFEXXFEName}[0]{\NDdrulename{Scom\_FE\_FE}}
\newcommand{\NDdruleScomXXFEXXFE}[1]{\NDdrule[#1]{%
}{
 \mathsf{let}\, \NDsym{(}   \mathsf{let}\, \NDnt{s_{{\mathrm{2}}}}  :   \mathsf{F} \NDnt{X}  \,\mathsf{be}\,  \mathsf{F}\, \NDmv{x}  \,\mathsf{in}\, \NDnt{s_{{\mathrm{1}}}}   \NDsym{)}  :   \mathsf{F} \NDnt{Y}  \,\mathsf{be}\,  \mathsf{F}\, \NDmv{y}  \,\mathsf{in}\, \NDnt{s_{{\mathrm{3}}}}   \leadsto_\mathsf{c}   \mathsf{let}\, \NDnt{s_{{\mathrm{2}}}}  :   \mathsf{F} \NDnt{X}  \,\mathsf{be}\,  \mathsf{F}\, \NDmv{x}  \,\mathsf{in}\, \NDsym{(}   \mathsf{let}\, \NDnt{s_{{\mathrm{1}}}}  :   \mathsf{F} \NDnt{Y}  \,\mathsf{be}\,  \mathsf{F}\, \NDmv{y}  \,\mathsf{in}\, \NDnt{s_{{\mathrm{3}}}}   \NDsym{)} }{%
{\NDdruleScomXXFEXXFEName}{}%
}}

\newcommand{\NDdefnscom}[1]{\begin{NDdefnblock}[#1]{$\NDnt{s_{{\mathrm{1}}}}  \leadsto_\mathsf{c}  \NDnt{s_{{\mathrm{2}}}}$}{}
\NDusedrule{\NDdruleScomXXunitEXXunitE{}}
\NDusedrule{\NDdruleScomXXunitETwoXXunitE{}}
\NDusedrule{\NDdruleScomXXunitEXXimprE{}}
\NDusedrule{\NDdruleScomXXunitETwoXXimprE{}}
\NDusedrule{\NDdruleScomXXunitEXXFE{}}
\NDusedrule{\NDdruleScomXXunitETwoXXFE{}}
\NDusedrule{\NDdruleScomXXtenEXXunitE{}}
\NDusedrule{\NDdruleScomXXtenETwoXXunitE{}}
\NDusedrule{\NDdruleScomXXtenEXXtenE{}}
\NDusedrule{\NDdruleScomXXtenETwoXXtenE{}}
\NDusedrule{\NDdruleScomXXtenEXXimprE{}}
\NDusedrule{\NDdruleScomXXtenETwoXXimprE{}}
\NDusedrule{\NDdruleScomXXtenEXXimplE{}}
\NDusedrule{\NDdruleScomXXtenETwoXXimplE{}}
\NDusedrule{\NDdruleScomXXtenEXXFE{}}
\NDusedrule{\NDdruleScomXXtenETwoXXFE{}}
\NDusedrule{\NDdruleScomXXFEXXunitE{}}
\NDusedrule{\NDdruleScomXXFEXXtenE{}}
\NDusedrule{\NDdruleScomXXFEXXimprE{}}
\NDusedrule{\NDdruleScomXXFEXXimplE{}}
\NDusedrule{\NDdruleScomXXFEXXFE{}}
\end{NDdefnblock}}


\newcommand{\NDdefnsCommuting}{
\NDdefntcom{}\NDdefnscom{}}

\newcommand{\NDdefnss}{
\NDdefnsJtype
\NDdefnsReduction
\NDdefnsCommuting
}

\newcommand{\NDall}{\NDmetavars\\[0pt]
\NDgrammar\\[5.0mm]
\NDdefnss}


% generated by Ott 0.25 from: Elle/Elle.ott
\newcommand{\Elledrule}[4][]{{\displaystyle\frac{\begin{array}{l}#2\end{array}}{#3}\quad\Elledrulename{#4}}}
\newcommand{\Elleusedrule}[1]{\[#1\]}
\newcommand{\Ellepremise}[1]{ #1 \\}
\newenvironment{Elledefnblock}[3][]{ \framebox{\mbox{#2}} \quad #3 \\[0pt]}{}
\newenvironment{Ellefundefnblock}[3][]{ \framebox{\mbox{#2}} \quad #3 \\[0pt]\begin{displaymath}\begin{array}{l}}{\end{array}\end{displaymath}}
\newcommand{\Ellefunclause}[2]{ #1 \equiv #2 \\}
\newcommand{\Ellent}[1]{\mathit{#1}}
\newcommand{\Ellemv}[1]{\mathit{#1}}
\newcommand{\Ellekw}[1]{\mathbf{#1}}
\newcommand{\Ellesym}[1]{#1}
\newcommand{\Ellecom}[1]{\text{#1}}
\newcommand{\Elledrulename}[1]{\textsc{#1}}
\newcommand{\Ellecomplu}[5]{\overline{#1}^{\,#2\in #3 #4 #5}}
\newcommand{\Ellecompu}[3]{\overline{#1}^{\,#2<#3}}
\newcommand{\Ellecomp}[2]{\overline{#1}^{\,#2}}
\newcommand{\Ellegrammartabular}[1]{\begin{supertabular}{llcllllll}#1\end{supertabular}}
\newcommand{\Ellemetavartabular}[1]{\begin{supertabular}{ll}#1\end{supertabular}}
\newcommand{\Ellerulehead}[3]{$#1$ & & $#2$ & & & \multicolumn{2}{l}{#3}}
\newcommand{\Elleprodline}[6]{& & $#1$ & $#2$ & $#3 #4$ & $#5$ & $#6$}
\newcommand{\Ellefirstprodline}[6]{\Elleprodline{#1}{#2}{#3}{#4}{#5}{#6}}
\newcommand{\Ellelongprodline}[2]{& & $#1$ & \multicolumn{4}{l}{$#2$}}
\newcommand{\Ellefirstlongprodline}[2]{\Ellelongprodline{#1}{#2}}
\newcommand{\Ellebindspecprodline}[6]{\Elleprodline{#1}{#2}{#3}{#4}{#5}{#6}}
\newcommand{\Elleprodnewline}{\\}
\newcommand{\Elleinterrule}{\\[5.0mm]}
\newcommand{\Elleafterlastrule}{\\}
\newcommand{\Ellemetavars}{
\Ellemetavartabular{
 $ \Ellemv{vars} ,\, \Ellemv{n} ,\, \Ellemv{a} ,\, \Ellemv{x} ,\, \Ellemv{y} ,\, \Ellemv{z} ,\, \Ellemv{w} ,\, \Ellemv{m} ,\, \Ellemv{o} $ &  \\
 $ \Ellemv{ivar} ,\, \Ellemv{i} ,\, \Ellemv{k} ,\, \Ellemv{j} ,\, \Ellemv{l} $ &  \\
 $ \Ellemv{const} ,\, \Ellemv{b} $ &  \\
}}

\newcommand{\ElleA}{
\Ellerulehead{\Ellent{A}  ,\ \Ellent{B}  ,\ \Ellent{C}  ,\ D}{::=}{}\Elleprodnewline
\Ellefirstprodline{|}{ \mathsf{B} }{}{}{}{}\Elleprodnewline
\Elleprodline{|}{ \mathsf{Unit} }{}{}{}{}\Elleprodnewline
\Elleprodline{|}{\Ellent{A}  \triangleright  \Ellent{B}}{}{}{}{}\Elleprodnewline
\Elleprodline{|}{\Ellent{A}  \rightharpoonup  \Ellent{B}}{}{}{}{}\Elleprodnewline
\Elleprodline{|}{\Ellent{A}  \leftharpoonup  \Ellent{B}}{}{}{}{}\Elleprodnewline
\Elleprodline{|}{\Ellesym{(}  \Ellent{A}  \Ellesym{)}} {\textsf{M}}{}{}{}\Elleprodnewline
\Elleprodline{|}{ \Ellent{A} } {\textsf{M}}{}{}{}\Elleprodnewline
\Elleprodline{|}{ \mathsf{F} \Ellent{X} }{}{}{}{}}

\newcommand{\ElleW}{
\Ellerulehead{\Ellent{W}  ,\ \Ellent{X}  ,\ \Ellent{Y}  ,\ \Ellent{Z}}{::=}{}\Elleprodnewline
\Ellefirstprodline{|}{ \mathsf{B} }{}{}{}{}\Elleprodnewline
\Elleprodline{|}{ \mathsf{Unit} }{}{}{}{}\Elleprodnewline
\Elleprodline{|}{\Ellent{X}  \otimes  \Ellent{Y}}{}{}{}{}\Elleprodnewline
\Elleprodline{|}{\Ellent{X}  \multimap  \Ellent{Y}}{}{}{}{}\Elleprodnewline
\Elleprodline{|}{\Ellesym{(}  \Ellent{X}  \Ellesym{)}} {\textsf{M}}{}{}{}\Elleprodnewline
\Elleprodline{|}{ \Ellent{X} } {\textsf{M}}{}{}{}\Elleprodnewline
\Elleprodline{|}{ \mathsf{G} \Ellent{A} }{}{}{}{}}

\newcommand{\ElleT}{
\Ellerulehead{\Ellent{T}}{::=}{}\Elleprodnewline
\Ellefirstprodline{|}{\Ellent{A}}{}{}{}{}\Elleprodnewline
\Elleprodline{|}{\Ellent{X}}{}{}{}{}}

\newcommand{\Ellep}{
\Ellerulehead{\Ellent{p}  ,\ \Ellent{q}}{::=}{}\Elleprodnewline
\Ellefirstprodline{|}{ \star }{}{}{}{}\Elleprodnewline
\Elleprodline{|}{\Ellemv{x}}{}{}{}{}\Elleprodnewline
\Elleprodline{|}{ \mathsf{triv} }{}{}{}{}\Elleprodnewline
\Elleprodline{|}{ \mathsf{triv} }{}{}{}{}\Elleprodnewline
\Elleprodline{|}{\Ellent{p}  \otimes  \Ellent{p'}}{}{}{}{}\Elleprodnewline
\Elleprodline{|}{\Ellent{p}  \triangleright  \Ellent{p'}}{}{}{}{}\Elleprodnewline
\Elleprodline{|}{ \mathsf{F}\, \Ellent{p} }{}{}{}{}\Elleprodnewline
\Elleprodline{|}{ \mathsf{G}\, \Ellent{p} }{}{}{}{}}

\newcommand{\Elles}{
\Ellerulehead{\Ellent{s}}{::=}{}\Elleprodnewline
\Ellefirstprodline{|}{\Ellemv{x}}{}{}{}{}\Elleprodnewline
\Elleprodline{|}{\Ellemv{b}}{}{}{}{}\Elleprodnewline
\Elleprodline{|}{ \mathsf{triv} }{}{}{}{}\Elleprodnewline
\Elleprodline{|}{ \mathsf{let}\, \Ellent{s_{{\mathrm{1}}}}  :  \Ellent{A} \,\mathsf{be}\, \Ellent{p} \,\mathsf{in}\, \Ellent{s_{{\mathrm{2}}}} }{}{}{}{}\Elleprodnewline
\Elleprodline{|}{ \mathsf{let}\, \Ellent{t}  :  \Ellent{X} \,\mathsf{be}\, \Ellent{p} \,\mathsf{in}\, \Ellent{s} }{}{}{}{}\Elleprodnewline
\Elleprodline{|}{\Ellent{s_{{\mathrm{1}}}}  \triangleright  \Ellent{s_{{\mathrm{2}}}}}{}{}{}{}\Elleprodnewline
\Elleprodline{|}{ \lambda_l  \Ellemv{x}  :  \Ellent{A} . \Ellent{s} }{}{}{}{}\Elleprodnewline
\Elleprodline{|}{ \lambda_r  \Ellemv{x}  :  \Ellent{A} . \Ellent{s} }{}{}{}{}\Elleprodnewline
\Elleprodline{|}{ \mathsf{app}_l\, \Ellent{s_{{\mathrm{1}}}} \, \Ellent{s_{{\mathrm{2}}}} }{}{}{}{}\Elleprodnewline
\Elleprodline{|}{ \mathsf{app}_r\, \Ellent{s_{{\mathrm{1}}}} \, \Ellent{s_{{\mathrm{2}}}} }{}{}{}{}\Elleprodnewline
\Elleprodline{|}{ \mathsf{derelict}\, \Ellent{t} }{}{}{}{}\Elleprodnewline
\Elleprodline{|}{ \mathsf{ex}\, \Ellent{s_{{\mathrm{1}}}} , \Ellent{s_{{\mathrm{2}}}} \,\mathsf{with}\, \Ellemv{x_{{\mathrm{1}}}} , \Ellemv{x_{{\mathrm{2}}}} \,\mathsf{in}\, \Ellent{s_{{\mathrm{3}}}} }{}{}{}{}\Elleprodnewline
\Elleprodline{|}{\Ellesym{[}  \Ellent{s_{{\mathrm{1}}}}  \Ellesym{/}  \Ellemv{x}  \Ellesym{]}  \Ellent{s_{{\mathrm{2}}}}} {\textsf{M}}{}{}{}\Elleprodnewline
\Elleprodline{|}{\Ellesym{[}  \Ellent{t}  \Ellesym{/}  \Ellemv{x}  \Ellesym{]}  \Ellent{s}} {\textsf{M}}{}{}{}\Elleprodnewline
\Elleprodline{|}{\Ellesym{(}  \Ellent{s}  \Ellesym{)}} {\textsf{S}}{}{}{}\Elleprodnewline
\Elleprodline{|}{ \Ellent{s} } {\textsf{M}}{}{}{}\Elleprodnewline
\Elleprodline{|}{ \mathsf{F} \Ellent{t} }{}{}{}{}}

\newcommand{\Ellet}{
\Ellerulehead{\Ellent{t}}{::=}{}\Elleprodnewline
\Ellefirstprodline{|}{\Ellemv{x}}{}{}{}{}\Elleprodnewline
\Elleprodline{|}{\Ellemv{b}}{}{}{}{}\Elleprodnewline
\Elleprodline{|}{ \mathsf{triv} }{}{}{}{}\Elleprodnewline
\Elleprodline{|}{ \mathsf{let}\, \Ellent{t_{{\mathrm{1}}}}  :  \Ellent{X} \,\mathsf{be}\, \Ellent{p} \,\mathsf{in}\, \Ellent{t_{{\mathrm{2}}}} }{}{}{}{}\Elleprodnewline
\Elleprodline{|}{\Ellent{t_{{\mathrm{1}}}}  \otimes  \Ellent{t_{{\mathrm{2}}}}}{}{}{}{}\Elleprodnewline
\Elleprodline{|}{ \lambda  \Ellemv{x}  :  \Ellent{X} . \Ellent{t} }{}{}{}{}\Elleprodnewline
\Elleprodline{|}{ \Ellent{t_{{\mathrm{1}}}}   \Ellent{t_{{\mathrm{2}}}} }{}{}{}{}\Elleprodnewline
\Elleprodline{|}{ \mathsf{ex}\, \Ellent{t_{{\mathrm{1}}}} , \Ellent{t_{{\mathrm{2}}}} \,\mathsf{with}\, \Ellemv{x_{{\mathrm{1}}}} , \Ellemv{x_{{\mathrm{2}}}} \,\mathsf{in}\, \Ellent{t_{{\mathrm{3}}}} }{}{}{}{}\Elleprodnewline
\Elleprodline{|}{\Ellesym{[}  \Ellent{t_{{\mathrm{1}}}}  \Ellesym{/}  \Ellemv{x}  \Ellesym{]}  \Ellent{t_{{\mathrm{2}}}}} {\textsf{M}}{}{}{}\Elleprodnewline
\Elleprodline{|}{\Ellesym{(}  \Ellent{t}  \Ellesym{)}} {\textsf{S}}{}{}{}\Elleprodnewline
\Elleprodline{|}{\Ellesym{h(}  \Ellent{t}  \Ellesym{)}} {\textsf{M}}{}{}{}\Elleprodnewline
\Elleprodline{|}{ \mathsf{G} \Ellent{s} }{}{}{}{}}

\newcommand{\ElleI}{
\Ellerulehead{\Phi  ,\ \Psi}{::=}{}\Elleprodnewline
\Ellefirstprodline{|}{ \cdot }{}{}{}{}\Elleprodnewline
\Elleprodline{|}{\Phi_{{\mathrm{1}}}  \Ellesym{,}  \Phi_{{\mathrm{2}}}}{}{}{}{}\Elleprodnewline
\Elleprodline{|}{\Ellemv{x}  \Ellesym{:}  \Ellent{X}}{}{}{}{}\Elleprodnewline
\Elleprodline{|}{\Ellesym{(}  \Phi  \Ellesym{)}} {\textsf{S}}{}{}{}}

\newcommand{\ElleG}{
\Ellerulehead{\Gamma  ,\ \Delta}{::=}{}\Elleprodnewline
\Ellefirstprodline{|}{ \cdot }{}{}{}{}\Elleprodnewline
\Elleprodline{|}{\Ellemv{x}  \Ellesym{:}  \Ellent{A}}{}{}{}{}\Elleprodnewline
\Elleprodline{|}{\Phi}{}{}{}{}\Elleprodnewline
\Elleprodline{|}{\Gamma_{{\mathrm{1}}}  \Ellesym{;}  \Gamma_{{\mathrm{2}}}}{}{}{}{}\Elleprodnewline
\Elleprodline{|}{\Ellesym{(}  \Gamma  \Ellesym{)}} {\textsf{S}}{}{}{}}

\newcommand{\Elleformula}{
\Ellerulehead{\Ellent{formula}}{::=}{}\Elleprodnewline
\Ellefirstprodline{|}{\Ellent{judgement}}{}{}{}{}\Elleprodnewline
\Elleprodline{|}{ \Ellent{formula_{{\mathrm{1}}}}  \quad  \Ellent{formula_{{\mathrm{2}}}} } {\textsf{M}}{}{}{}\Elleprodnewline
\Elleprodline{|}{\Ellent{formula_{{\mathrm{1}}}} \, ... \, \Ellent{formula_{\Ellemv{i}}}} {\textsf{M}}{}{}{}\Elleprodnewline
\Elleprodline{|}{ \Ellent{formula} } {\textsf{S}}{}{}{}\Elleprodnewline
\Elleprodline{|}{ \Ellemv{x}  \not\in \mathsf{FV}( \Ellent{s} ) }{}{}{}{}\Elleprodnewline
\Elleprodline{|}{ \Ellemv{x}  \not\in |  \Gamma ,  \Delta ,  \Psi  | }{}{}{}{}\Elleprodnewline
\Elleprodline{|}{ \Ellemv{x}  \not\in |  \Gamma ,  \Delta  | }{}{}{}{}}

\newcommand{\Elleterminals}{
\Ellerulehead{\Ellent{terminals}}{::=}{}\Elleprodnewline
\Ellefirstprodline{|}{ \otimes }{}{}{}{}\Elleprodnewline
\Elleprodline{|}{ \triangleright }{}{}{}{}\Elleprodnewline
\Elleprodline{|}{ \circop{e} }{}{}{}{}\Elleprodnewline
\Elleprodline{|}{ \circop{w} }{}{}{}{}\Elleprodnewline
\Elleprodline{|}{ \circop{c} }{}{}{}{}\Elleprodnewline
\Elleprodline{|}{ \rightharpoonup }{}{}{}{}\Elleprodnewline
\Elleprodline{|}{ \leftharpoonup }{}{}{}{}\Elleprodnewline
\Elleprodline{|}{ \multimap }{}{}{}{}\Elleprodnewline
\Elleprodline{|}{ \vdash_\mathcal{C} }{}{}{}{}\Elleprodnewline
\Elleprodline{|}{ \vdash_\mathcal{L} }{}{}{}{}\Elleprodnewline
\Elleprodline{|}{ \leadsto }{}{}{}{}\Elleprodnewline
\Elleprodline{|}{ \leadsto_\mathsf{c} }{}{}{}{}}

\newcommand{\ElleJtype}{
\Ellerulehead{\Ellent{Jtype}}{::=}{}\Elleprodnewline
\Ellefirstprodline{|}{\Phi  \vdash_\mathcal{C}  \Ellent{t}  \Ellesym{:}  \Ellent{X}}{}{}{}{}\Elleprodnewline
\Elleprodline{|}{\Gamma  \vdash_\mathcal{L}  \Ellent{s}  \Ellesym{:}  \Ellent{A}}{}{}{}{}}

\newcommand{\Ellejudgement}{
\Ellerulehead{\Ellent{judgement}}{::=}{}\Elleprodnewline
\Ellefirstprodline{|}{\Ellent{Jtype}}{}{}{}{}}

\newcommand{\ElleuserXXsyntax}{
\Ellerulehead{\Ellent{user\_syntax}}{::=}{}\Elleprodnewline
\Ellefirstprodline{|}{\Ellemv{vars}}{}{}{}{}\Elleprodnewline
\Elleprodline{|}{\Ellemv{ivar}}{}{}{}{}\Elleprodnewline
\Elleprodline{|}{\Ellemv{const}}{}{}{}{}\Elleprodnewline
\Elleprodline{|}{\Ellent{A}}{}{}{}{}\Elleprodnewline
\Elleprodline{|}{\Ellent{W}}{}{}{}{}\Elleprodnewline
\Elleprodline{|}{\Ellent{T}}{}{}{}{}\Elleprodnewline
\Elleprodline{|}{\Ellent{p}}{}{}{}{}\Elleprodnewline
\Elleprodline{|}{\Ellent{s}}{}{}{}{}\Elleprodnewline
\Elleprodline{|}{\Ellent{t}}{}{}{}{}\Elleprodnewline
\Elleprodline{|}{\Phi}{}{}{}{}\Elleprodnewline
\Elleprodline{|}{\Gamma}{}{}{}{}\Elleprodnewline
\Elleprodline{|}{\Ellent{formula}}{}{}{}{}\Elleprodnewline
\Elleprodline{|}{\Ellent{terminals}}{}{}{}{}}

\newcommand{\Ellegrammar}{\Ellegrammartabular{
\ElleA\Elleinterrule
\ElleW\Elleinterrule
\ElleT\Elleinterrule
\Ellep\Elleinterrule
\Elles\Elleinterrule
\Ellet\Elleinterrule
\ElleI\Elleinterrule
\ElleG\Elleinterrule
\Elleformula\Elleinterrule
\Elleterminals\Elleinterrule
\ElleJtype\Elleinterrule
\Ellejudgement\Elleinterrule
\ElleuserXXsyntax\Elleafterlastrule
}}

% defnss
% defns Jtype
%% defn tty
\newcommand{\ElledruleTXXaxName}[0]{\Elledrulename{T\_ax}}
\newcommand{\ElledruleTXXax}[1]{\Elledrule[#1]{%
}{
\Ellemv{x}  \Ellesym{:}  \Ellent{X}  \vdash_\mathcal{C}  \Ellemv{x}  \Ellesym{:}  \Ellent{X}}{%
{\ElledruleTXXaxName}{}%
}}


\newcommand{\ElledruleTXXunitLName}[0]{\Elledrulename{T\_unitL}}
\newcommand{\ElledruleTXXunitL}[1]{\Elledrule[#1]{%
\Ellepremise{\Phi  \Ellesym{,}  \Psi  \vdash_\mathcal{C}  \Ellent{t}  \Ellesym{:}  \Ellent{X}}%
}{
\Phi  \Ellesym{,}  \Ellemv{x}  \Ellesym{:}   \mathsf{Unit}   \Ellesym{,}  \Psi  \vdash_\mathcal{C}   \mathsf{let}\, \Ellemv{x}  :   \mathsf{Unit}  \,\mathsf{be}\,  \mathsf{triv}  \,\mathsf{in}\, \Ellent{t}   \Ellesym{:}  \Ellent{X}}{%
{\ElledruleTXXunitLName}{}%
}}


\newcommand{\ElledruleTXXunitRName}[0]{\Elledrulename{T\_unitR}}
\newcommand{\ElledruleTXXunitR}[1]{\Elledrule[#1]{%
}{
 \cdot   \vdash_\mathcal{C}   \mathsf{triv}   \Ellesym{:}   \mathsf{Unit} }{%
{\ElledruleTXXunitRName}{}%
}}


\newcommand{\ElledruleTXXtenLName}[0]{\Elledrulename{T\_tenL}}
\newcommand{\ElledruleTXXtenL}[1]{\Elledrule[#1]{%
\Ellepremise{\Phi  \Ellesym{,}  \Ellemv{x}  \Ellesym{:}  \Ellent{X}  \Ellesym{,}  \Ellemv{y}  \Ellesym{:}  \Ellent{Y}  \Ellesym{,}  \Psi  \vdash_\mathcal{C}  \Ellent{t}  \Ellesym{:}  \Ellent{Z}}%
}{
\Phi  \Ellesym{,}  \Ellemv{z}  \Ellesym{:}  \Ellent{X}  \otimes  \Ellent{Y}  \Ellesym{,}  \Psi  \vdash_\mathcal{C}   \mathsf{let}\, \Ellemv{z}  :  \Ellent{X}  \otimes  \Ellent{Y} \,\mathsf{be}\, \Ellemv{x}  \otimes  \Ellemv{y} \,\mathsf{in}\, \Ellent{t}   \Ellesym{:}  \Ellent{Z}}{%
{\ElledruleTXXtenLName}{}%
}}


\newcommand{\ElledruleTXXtenRName}[0]{\Elledrulename{T\_tenR}}
\newcommand{\ElledruleTXXtenR}[1]{\Elledrule[#1]{%
\Ellepremise{ \Phi  \vdash_\mathcal{C}  \Ellent{t_{{\mathrm{1}}}}  \Ellesym{:}  \Ellent{X}  \quad  \Psi  \vdash_\mathcal{C}  \Ellent{t_{{\mathrm{2}}}}  \Ellesym{:}  \Ellent{Y} }%
}{
\Phi  \Ellesym{,}  \Psi  \vdash_\mathcal{C}  \Ellent{t_{{\mathrm{1}}}}  \otimes  \Ellent{t_{{\mathrm{2}}}}  \Ellesym{:}  \Ellent{X}  \otimes  \Ellent{Y}}{%
{\ElledruleTXXtenRName}{}%
}}


\newcommand{\ElledruleTXXimpLName}[0]{\Elledrulename{T\_impL}}
\newcommand{\ElledruleTXXimpL}[1]{\Elledrule[#1]{%
\Ellepremise{ \Phi  \vdash_\mathcal{C}  \Ellent{t_{{\mathrm{1}}}}  \Ellesym{:}  \Ellent{X}  \quad  \Psi_{{\mathrm{1}}}  \Ellesym{,}  \Ellemv{x}  \Ellesym{:}  \Ellent{Y}  \Ellesym{,}  \Psi_{{\mathrm{2}}}  \vdash_\mathcal{C}  \Ellent{t_{{\mathrm{2}}}}  \Ellesym{:}  \Ellent{Z} }%
}{
\Psi_{{\mathrm{1}}}  \Ellesym{,}  \Ellemv{y}  \Ellesym{:}  \Ellent{X}  \multimap  \Ellent{Y}  \Ellesym{,}  \Phi  \Ellesym{,}  \Psi_{{\mathrm{2}}}  \vdash_\mathcal{C}  \Ellesym{[}   \Ellemv{y}   \Ellent{t_{{\mathrm{1}}}}   \Ellesym{/}  \Ellemv{x}  \Ellesym{]}  \Ellent{t_{{\mathrm{2}}}}  \Ellesym{:}  \Ellent{Z}}{%
{\ElledruleTXXimpLName}{}%
}}


\newcommand{\ElledruleTXXimpRName}[0]{\Elledrulename{T\_impR}}
\newcommand{\ElledruleTXXimpR}[1]{\Elledrule[#1]{%
\Ellepremise{\Phi  \Ellesym{,}  \Ellemv{x}  \Ellesym{:}  \Ellent{X}  \Ellesym{,}  \Psi  \vdash_\mathcal{C}  \Ellent{t}  \Ellesym{:}  \Ellent{Y}}%
}{
\Phi  \Ellesym{,}  \Psi  \vdash_\mathcal{C}   \lambda  \Ellemv{x}  :  \Ellent{X} . \Ellent{t}   \Ellesym{:}  \Ellent{X}  \multimap  \Ellent{Y}}{%
{\ElledruleTXXimpRName}{}%
}}


\newcommand{\ElledruleTXXGrName}[0]{\Elledrulename{T\_Gr}}
\newcommand{\ElledruleTXXGr}[1]{\Elledrule[#1]{%
\Ellepremise{\Phi  \vdash_\mathcal{L}  \Ellent{s}  \Ellesym{:}  \Ellent{A}}%
}{
\Phi  \vdash_\mathcal{C}   \mathsf{G} \Ellent{s}   \Ellesym{:}   \mathsf{G} \Ellent{A} }{%
{\ElledruleTXXGrName}{}%
}}


\newcommand{\ElledruleTXXcutName}[0]{\Elledrulename{T\_cut}}
\newcommand{\ElledruleTXXcut}[1]{\Elledrule[#1]{%
\Ellepremise{ \Phi  \vdash_\mathcal{C}  \Ellent{t_{{\mathrm{1}}}}  \Ellesym{:}  \Ellent{X}  \quad  \Psi_{{\mathrm{1}}}  \Ellesym{,}  \Ellemv{x}  \Ellesym{:}  \Ellent{X}  \Ellesym{,}  \Psi_{{\mathrm{2}}}  \vdash_\mathcal{C}  \Ellent{t_{{\mathrm{2}}}}  \Ellesym{:}  \Ellent{Y} }%
}{
\Psi_{{\mathrm{1}}}  \Ellesym{,}  \Phi  \Ellesym{,}  \Psi_{{\mathrm{2}}}  \vdash_\mathcal{C}  \Ellesym{[}  \Ellent{t_{{\mathrm{1}}}}  \Ellesym{/}  \Ellemv{x}  \Ellesym{]}  \Ellent{t_{{\mathrm{2}}}}  \Ellesym{:}  \Ellent{Y}}{%
{\ElledruleTXXcutName}{}%
}}


\newcommand{\ElledruleTXXexName}[0]{\Elledrulename{T\_ex}}
\newcommand{\ElledruleTXXex}[1]{\Elledrule[#1]{%
\Ellepremise{\Phi  \Ellesym{,}  \Ellemv{x}  \Ellesym{:}  \Ellent{X}  \Ellesym{,}  \Ellemv{y}  \Ellesym{:}  \Ellent{Y}  \Ellesym{,}  \Psi  \vdash_\mathcal{C}  \Ellent{t}  \Ellesym{:}  \Ellent{Z}}%
}{
\Phi  \Ellesym{,}  \Ellemv{z}  \Ellesym{:}  \Ellent{Y}  \Ellesym{,}  \Ellemv{w}  \Ellesym{:}  \Ellent{X}  \Ellesym{,}  \Psi  \vdash_\mathcal{C}   \mathsf{ex}\, \Ellemv{w} , \Ellemv{z} \,\mathsf{with}\, \Ellemv{x} , \Ellemv{y} \,\mathsf{in}\, \Ellent{t}   \Ellesym{:}  \Ellent{Z}}{%
{\ElledruleTXXexName}{}%
}}

\newcommand{\Elledefntty}[1]{\begin{Elledefnblock}[#1]{$\Phi  \vdash_\mathcal{C}  \Ellent{t}  \Ellesym{:}  \Ellent{X}$}{}
\Elleusedrule{\ElledruleTXXax{}}
\Elleusedrule{\ElledruleTXXunitL{}}
\Elleusedrule{\ElledruleTXXunitR{}}
\Elleusedrule{\ElledruleTXXtenL{}}
\Elleusedrule{\ElledruleTXXtenR{}}
\Elleusedrule{\ElledruleTXXimpL{}}
\Elleusedrule{\ElledruleTXXimpR{}}
\Elleusedrule{\ElledruleTXXGr{}}
\Elleusedrule{\ElledruleTXXcut{}}
\Elleusedrule{\ElledruleTXXex{}}
\end{Elledefnblock}}

%% defn sty
\newcommand{\ElledruleSXXaxName}[0]{\Elledrulename{S\_ax}}
\newcommand{\ElledruleSXXax}[1]{\Elledrule[#1]{%
}{
\Ellemv{x}  \Ellesym{:}  \Ellent{A}  \vdash_\mathcal{L}  \Ellemv{x}  \Ellesym{:}  \Ellent{A}}{%
{\ElledruleSXXaxName}{}%
}}


\newcommand{\ElledruleSXXunitLOneName}[0]{\Elledrulename{S\_unitL1}}
\newcommand{\ElledruleSXXunitLOne}[1]{\Elledrule[#1]{%
\Ellepremise{\Gamma  \Ellesym{;}  \Delta  \vdash_\mathcal{L}  \Ellent{s}  \Ellesym{:}  \Ellent{A}}%
}{
\Gamma  \Ellesym{;}  \Ellemv{x}  \Ellesym{:}   \mathsf{Unit}   \Ellesym{;}  \Delta  \vdash_\mathcal{L}   \mathsf{let}\, \Ellemv{x}  :   \mathsf{Unit}  \,\mathsf{be}\,  \mathsf{triv}  \,\mathsf{in}\, \Ellent{s}   \Ellesym{:}  \Ellent{A}}{%
{\ElledruleSXXunitLOneName}{}%
}}


\newcommand{\ElledruleSXXunitLTwoName}[0]{\Elledrulename{S\_unitL2}}
\newcommand{\ElledruleSXXunitLTwo}[1]{\Elledrule[#1]{%
\Ellepremise{\Gamma  \Ellesym{;}  \Delta  \vdash_\mathcal{L}  \Ellent{s}  \Ellesym{:}  \Ellent{A}}%
}{
\Gamma  \Ellesym{;}  \Ellemv{x}  \Ellesym{:}   \mathsf{Unit}   \Ellesym{;}  \Delta  \vdash_\mathcal{L}   \mathsf{let}\, \Ellemv{x}  :   \mathsf{Unit}  \,\mathsf{be}\,  \mathsf{triv}  \,\mathsf{in}\, \Ellent{s}   \Ellesym{:}  \Ellent{A}}{%
{\ElledruleSXXunitLTwoName}{}%
}}


\newcommand{\ElledruleSXXunitRName}[0]{\Elledrulename{S\_unitR}}
\newcommand{\ElledruleSXXunitR}[1]{\Elledrule[#1]{%
}{
 \cdot   \vdash_\mathcal{L}   \mathsf{triv}   \Ellesym{:}   \mathsf{Unit} }{%
{\ElledruleSXXunitRName}{}%
}}


\newcommand{\ElledruleSXXexName}[0]{\Elledrulename{S\_ex}}
\newcommand{\ElledruleSXXex}[1]{\Elledrule[#1]{%
\Ellepremise{\Gamma  \Ellesym{;}  \Ellemv{x}  \Ellesym{:}  \Ellent{X}  \Ellesym{;}  \Ellemv{y}  \Ellesym{:}  \Ellent{Y}  \Ellesym{;}  \Delta  \vdash_\mathcal{L}  \Ellent{s}  \Ellesym{:}  \Ellent{A}}%
}{
\Gamma  \Ellesym{;}  \Ellemv{z}  \Ellesym{:}  \Ellent{Y}  \Ellesym{;}  \Ellemv{w}  \Ellesym{:}  \Ellent{X}  \Ellesym{;}  \Delta  \vdash_\mathcal{L}   \mathsf{ex}\, \Ellemv{w} , \Ellemv{z} \,\mathsf{with}\, \Ellemv{x} , \Ellemv{y} \,\mathsf{in}\, \Ellent{s}   \Ellesym{:}  \Ellent{A}}{%
{\ElledruleSXXexName}{}%
}}


\newcommand{\ElledruleSXXtenLOneName}[0]{\Elledrulename{S\_tenL1}}
\newcommand{\ElledruleSXXtenLOne}[1]{\Elledrule[#1]{%
\Ellepremise{\Gamma  \Ellesym{;}  \Ellemv{x}  \Ellesym{:}  \Ellent{X}  \Ellesym{;}  \Ellemv{y}  \Ellesym{:}  \Ellent{Y}  \Ellesym{;}  \Delta  \vdash_\mathcal{L}  \Ellent{s}  \Ellesym{:}  \Ellent{A}}%
}{
\Gamma  \Ellesym{;}  \Ellemv{z}  \Ellesym{:}  \Ellent{X}  \otimes  \Ellent{Y}  \Ellesym{;}  \Delta  \vdash_\mathcal{L}   \mathsf{let}\, \Ellemv{z}  :  \Ellent{X}  \otimes  \Ellent{Y} \,\mathsf{be}\, \Ellemv{x}  \otimes  \Ellemv{y} \,\mathsf{in}\, \Ellent{s}   \Ellesym{:}  \Ellent{A}}{%
{\ElledruleSXXtenLOneName}{}%
}}


\newcommand{\ElledruleSXXtenLTwoName}[0]{\Elledrulename{S\_tenL2}}
\newcommand{\ElledruleSXXtenLTwo}[1]{\Elledrule[#1]{%
\Ellepremise{\Gamma  \Ellesym{;}  \Ellemv{x}  \Ellesym{:}  \Ellent{A}  \Ellesym{;}  \Ellemv{y}  \Ellesym{:}  \Ellent{B}  \Ellesym{;}  \Delta  \vdash_\mathcal{L}  \Ellent{s}  \Ellesym{:}  \Ellent{C}}%
}{
\Gamma  \Ellesym{;}  \Ellemv{z}  \Ellesym{:}  \Ellent{A}  \triangleright  \Ellent{B}  \Ellesym{;}  \Delta  \vdash_\mathcal{L}   \mathsf{let}\, \Ellemv{z}  :  \Ellent{A}  \triangleright  \Ellent{B} \,\mathsf{be}\, \Ellemv{x}  \triangleright  \Ellemv{y} \,\mathsf{in}\, \Ellent{s}   \Ellesym{:}  \Ellent{C}}{%
{\ElledruleSXXtenLTwoName}{}%
}}


\newcommand{\ElledruleSXXtenRName}[0]{\Elledrulename{S\_tenR}}
\newcommand{\ElledruleSXXtenR}[1]{\Elledrule[#1]{%
\Ellepremise{ \Gamma  \vdash_\mathcal{L}  \Ellent{s_{{\mathrm{1}}}}  \Ellesym{:}  \Ellent{A}  \quad  \Delta  \vdash_\mathcal{L}  \Ellent{s_{{\mathrm{2}}}}  \Ellesym{:}  \Ellent{B} }%
}{
\Gamma  \Ellesym{;}  \Delta  \vdash_\mathcal{L}  \Ellent{s_{{\mathrm{1}}}}  \triangleright  \Ellent{s_{{\mathrm{2}}}}  \Ellesym{:}  \Ellent{A}  \triangleright  \Ellent{B}}{%
{\ElledruleSXXtenRName}{}%
}}


\newcommand{\ElledruleSXXimpLName}[0]{\Elledrulename{S\_impL}}
\newcommand{\ElledruleSXXimpL}[1]{\Elledrule[#1]{%
\Ellepremise{ \Phi  \vdash_\mathcal{C}  \Ellent{t}  \Ellesym{:}  \Ellent{X}  \quad  \Gamma  \Ellesym{;}  \Ellemv{x}  \Ellesym{:}  \Ellent{Y}  \Ellesym{;}  \Delta  \vdash_\mathcal{L}  \Ellent{s}  \Ellesym{:}  \Ellent{A} }%
}{
\Gamma  \Ellesym{;}  \Ellemv{y}  \Ellesym{:}  \Ellent{X}  \multimap  \Ellent{Y}  \Ellesym{;}  \Phi  \Ellesym{;}  \Delta  \vdash_\mathcal{L}  \Ellesym{[}   \Ellemv{y}   \Ellent{t}   \Ellesym{/}  \Ellemv{x}  \Ellesym{]}  \Ellent{s}  \Ellesym{:}  \Ellent{A}}{%
{\ElledruleSXXimpLName}{}%
}}


\newcommand{\ElledruleSXXimprLName}[0]{\Elledrulename{S\_imprL}}
\newcommand{\ElledruleSXXimprL}[1]{\Elledrule[#1]{%
\Ellepremise{ \Gamma  \vdash_\mathcal{L}  \Ellent{s_{{\mathrm{1}}}}  \Ellesym{:}  \Ellent{A}  \quad  \Delta_{{\mathrm{1}}}  \Ellesym{;}  \Ellemv{x}  \Ellesym{:}  \Ellent{B}  \Ellesym{;}  \Delta_{{\mathrm{2}}}  \vdash_\mathcal{L}  \Ellent{s_{{\mathrm{2}}}}  \Ellesym{:}  \Ellent{C} }%
}{
\Delta_{{\mathrm{1}}}  \Ellesym{;}  \Ellemv{y}  \Ellesym{:}  \Ellent{A}  \rightharpoonup  \Ellent{B}  \Ellesym{;}  \Gamma  \Ellesym{;}  \Delta_{{\mathrm{2}}}  \vdash_\mathcal{L}  \Ellesym{[}   \mathsf{app}_r\, \Ellemv{y} \, \Ellent{s_{{\mathrm{1}}}}   \Ellesym{/}  \Ellemv{x}  \Ellesym{]}  \Ellent{s_{{\mathrm{2}}}}  \Ellesym{:}  \Ellent{C}}{%
{\ElledruleSXXimprLName}{}%
}}


\newcommand{\ElledruleSXXimprRName}[0]{\Elledrulename{S\_imprR}}
\newcommand{\ElledruleSXXimprR}[1]{\Elledrule[#1]{%
\Ellepremise{\Gamma  \Ellesym{;}  \Ellemv{x}  \Ellesym{:}  \Ellent{A}  \vdash_\mathcal{L}  \Ellent{s}  \Ellesym{:}  \Ellent{B}}%
}{
\Gamma  \vdash_\mathcal{L}   \lambda_r  \Ellemv{x}  :  \Ellent{A} . \Ellent{s}   \Ellesym{:}  \Ellent{A}  \rightharpoonup  \Ellent{B}}{%
{\ElledruleSXXimprRName}{}%
}}


\newcommand{\ElledruleSXXimplLName}[0]{\Elledrulename{S\_implL}}
\newcommand{\ElledruleSXXimplL}[1]{\Elledrule[#1]{%
\Ellepremise{ \Gamma  \vdash_\mathcal{L}  \Ellent{s_{{\mathrm{1}}}}  \Ellesym{:}  \Ellent{A}  \quad  \Delta_{{\mathrm{1}}}  \Ellesym{;}  \Ellemv{x}  \Ellesym{:}  \Ellent{B}  \Ellesym{;}  \Delta_{{\mathrm{2}}}  \vdash_\mathcal{L}  \Ellent{s_{{\mathrm{2}}}}  \Ellesym{:}  \Ellent{C} }%
}{
\Delta_{{\mathrm{1}}}  \Ellesym{;}  \Gamma  \Ellesym{;}  \Ellemv{y}  \Ellesym{:}  \Ellent{B}  \leftharpoonup  \Ellent{A}  \Ellesym{;}  \Delta_{{\mathrm{2}}}  \vdash_\mathcal{L}  \Ellesym{[}   \mathsf{app}_l\, \Ellemv{y} \, \Ellent{s_{{\mathrm{1}}}}   \Ellesym{/}  \Ellemv{x}  \Ellesym{]}  \Ellent{s_{{\mathrm{2}}}}  \Ellesym{:}  \Ellent{C}}{%
{\ElledruleSXXimplLName}{}%
}}


\newcommand{\ElledruleSXXimplRName}[0]{\Elledrulename{S\_implR}}
\newcommand{\ElledruleSXXimplR}[1]{\Elledrule[#1]{%
\Ellepremise{\Ellemv{x}  \Ellesym{:}  \Ellent{A}  \Ellesym{;}  \Gamma  \vdash_\mathcal{L}  \Ellent{s}  \Ellesym{:}  \Ellent{B}}%
}{
\Gamma  \vdash_\mathcal{L}   \lambda_l  \Ellemv{x}  :  \Ellent{A} . \Ellent{s}   \Ellesym{:}  \Ellent{B}  \leftharpoonup  \Ellent{A}}{%
{\ElledruleSXXimplRName}{}%
}}


\newcommand{\ElledruleSXXFlName}[0]{\Elledrulename{S\_Fl}}
\newcommand{\ElledruleSXXFl}[1]{\Elledrule[#1]{%
\Ellepremise{\Gamma  \Ellesym{;}  \Ellemv{x}  \Ellesym{:}  \Ellent{X}  \Ellesym{;}  \Delta  \vdash_\mathcal{L}  \Ellent{s}  \Ellesym{:}  \Ellent{A}}%
}{
\Gamma  \Ellesym{;}  \Ellemv{y}  \Ellesym{:}   \mathsf{F} \Ellent{X}   \Ellesym{;}  \Delta  \vdash_\mathcal{L}   \mathsf{let}\, \Ellemv{y}  :   \mathsf{F} \Ellent{X}  \,\mathsf{be}\,  \mathsf{F}\, \Ellemv{x}  \,\mathsf{in}\, \Ellent{s}   \Ellesym{:}  \Ellent{A}}{%
{\ElledruleSXXFlName}{}%
}}


\newcommand{\ElledruleSXXFrName}[0]{\Elledrulename{S\_Fr}}
\newcommand{\ElledruleSXXFr}[1]{\Elledrule[#1]{%
\Ellepremise{\Phi  \vdash_\mathcal{C}  \Ellent{t}  \Ellesym{:}  \Ellent{X}}%
}{
\Phi  \vdash_\mathcal{L}   \mathsf{F} \Ellent{t}   \Ellesym{:}   \mathsf{F} \Ellent{X} }{%
{\ElledruleSXXFrName}{}%
}}


\newcommand{\ElledruleSXXGlName}[0]{\Elledrulename{S\_Gl}}
\newcommand{\ElledruleSXXGl}[1]{\Elledrule[#1]{%
\Ellepremise{\Gamma  \Ellesym{;}  \Ellemv{x}  \Ellesym{:}  \Ellent{A}  \Ellesym{;}  \Delta  \vdash_\mathcal{L}  \Ellent{s}  \Ellesym{:}  \Ellent{B}}%
}{
\Gamma  \Ellesym{;}  \Ellemv{y}  \Ellesym{:}   \mathsf{G} \Ellent{A}   \Ellesym{;}  \Delta  \vdash_\mathcal{L}   \mathsf{let}\, \Ellemv{y}  :   \mathsf{G} \Ellent{A}  \,\mathsf{be}\,  \mathsf{G}\, \Ellemv{x}  \,\mathsf{in}\, \Ellent{s}   \Ellesym{:}  \Ellent{B}}{%
{\ElledruleSXXGlName}{}%
}}


\newcommand{\ElledruleSXXcutOneName}[0]{\Elledrulename{S\_cut1}}
\newcommand{\ElledruleSXXcutOne}[1]{\Elledrule[#1]{%
\Ellepremise{ \Phi  \vdash_\mathcal{C}  \Ellent{t}  \Ellesym{:}  \Ellent{X}  \quad  \Gamma_{{\mathrm{1}}}  \Ellesym{;}  \Ellemv{x}  \Ellesym{:}  \Ellent{X}  \Ellesym{;}  \Gamma_{{\mathrm{2}}}  \vdash_\mathcal{L}  \Ellent{s}  \Ellesym{:}  \Ellent{A} }%
}{
\Gamma_{{\mathrm{1}}}  \Ellesym{;}  \Phi  \Ellesym{;}  \Gamma_{{\mathrm{1}}}  \vdash_\mathcal{L}  \Ellesym{[}  \Ellent{t}  \Ellesym{/}  \Ellemv{x}  \Ellesym{]}  \Ellent{s}  \Ellesym{:}  \Ellent{A}}{%
{\ElledruleSXXcutOneName}{}%
}}


\newcommand{\ElledruleSXXcutTwoName}[0]{\Elledrulename{S\_cut2}}
\newcommand{\ElledruleSXXcutTwo}[1]{\Elledrule[#1]{%
\Ellepremise{ \Gamma  \vdash_\mathcal{L}  \Ellent{s_{{\mathrm{1}}}}  \Ellesym{:}  \Ellent{A}  \quad  \Delta_{{\mathrm{1}}}  \Ellesym{;}  \Ellemv{x}  \Ellesym{:}  \Ellent{A}  \Ellesym{;}  \Delta_{{\mathrm{2}}}  \vdash_\mathcal{L}  \Ellent{s_{{\mathrm{2}}}}  \Ellesym{:}  \Ellent{B} }%
}{
\Delta_{{\mathrm{1}}}  \Ellesym{;}  \Gamma  \Ellesym{;}  \Delta_{{\mathrm{2}}}  \vdash_\mathcal{L}  \Ellesym{[}  \Ellent{s_{{\mathrm{1}}}}  \Ellesym{/}  \Ellemv{x}  \Ellesym{]}  \Ellent{s_{{\mathrm{2}}}}  \Ellesym{:}  \Ellent{B}}{%
{\ElledruleSXXcutTwoName}{}%
}}

\newcommand{\Elledefnsty}[1]{\begin{Elledefnblock}[#1]{$\Gamma  \vdash_\mathcal{L}  \Ellent{s}  \Ellesym{:}  \Ellent{A}$}{}
\Elleusedrule{\ElledruleSXXax{}}
\Elleusedrule{\ElledruleSXXunitLOne{}}
\Elleusedrule{\ElledruleSXXunitLTwo{}}
\Elleusedrule{\ElledruleSXXunitR{}}
\Elleusedrule{\ElledruleSXXex{}}
\Elleusedrule{\ElledruleSXXtenLOne{}}
\Elleusedrule{\ElledruleSXXtenLTwo{}}
\Elleusedrule{\ElledruleSXXtenR{}}
\Elleusedrule{\ElledruleSXXimpL{}}
\Elleusedrule{\ElledruleSXXimprL{}}
\Elleusedrule{\ElledruleSXXimprR{}}
\Elleusedrule{\ElledruleSXXimplL{}}
\Elleusedrule{\ElledruleSXXimplR{}}
\Elleusedrule{\ElledruleSXXFl{}}
\Elleusedrule{\ElledruleSXXFr{}}
\Elleusedrule{\ElledruleSXXGl{}}
\Elleusedrule{\ElledruleSXXcutOne{}}
\Elleusedrule{\ElledruleSXXcutTwo{}}
\end{Elledefnblock}}


\newcommand{\ElledefnsJtype}{
\Elledefntty{}\Elledefnsty{}}

\newcommand{\Elledefnss}{
\ElledefnsJtype
}

\newcommand{\Elleall}{\Ellemetavars\\[0pt]
\Ellegrammar\\[5.0mm]
\Elledefnss}


% generated by Ott 0.25 from: LNL/LNL.ott
\newcommand{\LNLdrule}[4][]{{\displaystyle\frac{\begin{array}{l}#2\end{array}}{#3}\quad\LNLdrulename{#4}}}
\newcommand{\LNLusedrule}[1]{\[#1\]}
\newcommand{\LNLpremise}[1]{ #1 \\}
\newenvironment{LNLdefnblock}[3][]{ \framebox{\mbox{#2}} \quad #3 \\[0pt]}{}
\newenvironment{LNLfundefnblock}[3][]{ \framebox{\mbox{#2}} \quad #3 \\[0pt]\begin{displaymath}\begin{array}{l}}{\end{array}\end{displaymath}}
\newcommand{\LNLfunclause}[2]{ #1 \equiv #2 \\}
\newcommand{\LNLnt}[1]{\mathit{#1}}
\newcommand{\LNLmv}[1]{\mathit{#1}}
\newcommand{\LNLkw}[1]{\mathbf{#1}}
\newcommand{\LNLsym}[1]{#1}
\newcommand{\LNLcom}[1]{\text{#1}}
\newcommand{\LNLdrulename}[1]{\textsc{#1}}
\newcommand{\LNLcomplu}[5]{\overline{#1}^{\,#2\in #3 #4 #5}}
\newcommand{\LNLcompu}[3]{\overline{#1}^{\,#2<#3}}
\newcommand{\LNLcomp}[2]{\overline{#1}^{\,#2}}
\newcommand{\LNLgrammartabular}[1]{\begin{supertabular}{llcllllll}#1\end{supertabular}}
\newcommand{\LNLmetavartabular}[1]{\begin{supertabular}{ll}#1\end{supertabular}}
\newcommand{\LNLrulehead}[3]{$#1$ & & $#2$ & & & \multicolumn{2}{l}{#3}}
\newcommand{\LNLprodline}[6]{& & $#1$ & $#2$ & $#3 #4$ & $#5$ & $#6$}
\newcommand{\LNLfirstprodline}[6]{\LNLprodline{#1}{#2}{#3}{#4}{#5}{#6}}
\newcommand{\LNLlongprodline}[2]{& & $#1$ & \multicolumn{4}{l}{$#2$}}
\newcommand{\LNLfirstlongprodline}[2]{\LNLlongprodline{#1}{#2}}
\newcommand{\LNLbindspecprodline}[6]{\LNLprodline{#1}{#2}{#3}{#4}{#5}{#6}}
\newcommand{\LNLprodnewline}{\\}
\newcommand{\LNLinterrule}{\\[5.0mm]}
\newcommand{\LNLafterlastrule}{\\}
\newcommand{\LNLmetavars}{
\LNLmetavartabular{
 $ \LNLmv{vars} ,\, \LNLmv{n} ,\, \LNLmv{a} ,\, \LNLmv{x} ,\, \LNLmv{y} ,\, \LNLmv{z} ,\, \LNLmv{w} ,\, \LNLmv{m} ,\, \LNLmv{o} $ &  \\
 $ \LNLmv{ivar} ,\, \LNLmv{i} ,\, \LNLmv{k} ,\, \LNLmv{j} ,\, \LNLmv{l} $ &  \\
 $ \LNLmv{const} ,\, \LNLmv{b} $ &  \\
}}

\newcommand{\LNLA}{
\LNLrulehead{\LNLnt{A}  ,\ \LNLnt{B}  ,\ \LNLnt{C}}{::=}{}\LNLprodnewline
\LNLfirstprodline{|}{ \mathsf{B} }{}{}{}{}\LNLprodnewline
\LNLprodline{|}{ \mathrm{I} }{}{}{}{}\LNLprodnewline
\LNLprodline{|}{\LNLnt{A}  \otimes  \LNLnt{B}}{}{}{}{}\LNLprodnewline
\LNLprodline{|}{\LNLnt{A}  \multimap  \LNLnt{B}}{}{}{}{}\LNLprodnewline
\LNLprodline{|}{\LNLsym{(}  \LNLnt{A}  \LNLsym{)}} {\textsf{M}}{}{}{}\LNLprodnewline
\LNLprodline{|}{ \LNLnt{A} } {\textsf{M}}{}{}{}\LNLprodnewline
\LNLprodline{|}{ \mathsf{F} \LNLnt{X} }{}{}{}{}\LNLprodnewline
\LNLprodline{|}{ CL( \LNLnt{A} ) } {\textsf{M}}{}{}{}}

\newcommand{\LNLX}{
\LNLrulehead{\LNLnt{X}  ,\ \LNLnt{Y}  ,\ \LNLnt{Z}}{::=}{}\LNLprodnewline
\LNLfirstprodline{|}{ \mathsf{B} }{}{}{}{}\LNLprodnewline
\LNLprodline{|}{ \mathsf{1} }{}{}{}{}\LNLprodnewline
\LNLprodline{|}{\LNLnt{X}  \times  \LNLnt{Y}}{}{}{}{}\LNLprodnewline
\LNLprodline{|}{\LNLnt{X}  \rightarrow  \LNLnt{Y}}{}{}{}{}\LNLprodnewline
\LNLprodline{|}{\LNLsym{(}  \LNLnt{X}  \LNLsym{)}} {\textsf{M}}{}{}{}\LNLprodnewline
\LNLprodline{|}{ \LNLnt{X} } {\textsf{M}}{}{}{}\LNLprodnewline
\LNLprodline{|}{ \mathsf{G} \LNLnt{A} }{}{}{}{}}

\newcommand{\LNLT}{
\LNLrulehead{\LNLnt{T}}{::=}{}\LNLprodnewline
\LNLfirstprodline{|}{\LNLnt{A}}{}{}{}{}\LNLprodnewline
\LNLprodline{|}{\LNLnt{X}}{}{}{}{}}

\newcommand{\LNLp}{
\LNLrulehead{\LNLnt{p}}{::=}{}\LNLprodnewline
\LNLfirstprodline{|}{ \star }{}{}{}{}\LNLprodnewline
\LNLprodline{|}{\LNLmv{x}}{}{}{}{}\LNLprodnewline
\LNLprodline{|}{ \mathsf{u} }{}{}{}{}\LNLprodnewline
\LNLprodline{|}{ \ast }{}{}{}{}\LNLprodnewline
\LNLprodline{|}{\LNLnt{p}  \otimes  \LNLnt{p'}}{}{}{}{}\LNLprodnewline
\LNLprodline{|}{\LNLnt{p}  \times  \LNLnt{p'}}{}{}{}{}\LNLprodnewline
\LNLprodline{|}{ \mathsf{F}\, \LNLnt{p} }{}{}{}{}\LNLprodnewline
\LNLprodline{|}{ \mathsf{G}\, \LNLnt{p} }{}{}{}{}}

\newcommand{\LNLs}{
\LNLrulehead{\LNLnt{s}}{::=}{}\LNLprodnewline
\LNLfirstprodline{|}{\LNLmv{x}}{}{}{}{}\LNLprodnewline
\LNLprodline{|}{\LNLmv{b}}{}{}{}{}\LNLprodnewline
\LNLprodline{|}{ \ast }{}{}{}{}\LNLprodnewline
\LNLprodline{|}{ \mathsf{let}\, \LNLnt{s_{{\mathrm{1}}}}  :  \LNLnt{T} \,\mathsf{be}\, \LNLnt{p} \,\mathsf{in}\, \LNLnt{s_{{\mathrm{2}}}} }{}{}{}{}\LNLprodnewline
\LNLprodline{|}{ \mathsf{let}\, \LNLnt{t}  :  \LNLnt{T} \,\mathsf{be}\, \LNLnt{p} \,\mathsf{in}\, \LNLnt{s} }{}{}{}{}\LNLprodnewline
\LNLprodline{|}{\LNLnt{s_{{\mathrm{1}}}}  \otimes  \LNLnt{s_{{\mathrm{2}}}}}{}{}{}{}\LNLprodnewline
\LNLprodline{|}{ \lambda  \LNLmv{x}  :  \LNLnt{A} . \LNLnt{s} }{}{}{}{}\LNLprodnewline
\LNLprodline{|}{ \mathsf{app}\, \LNLnt{s_{{\mathrm{1}}}} \, \LNLnt{s_{{\mathrm{2}}}} }{}{}{}{}\LNLprodnewline
\LNLprodline{|}{ \mathsf{derelict}\, \LNLnt{t} }{}{}{}{}\LNLprodnewline
\LNLprodline{|}{\LNLsym{[}  \LNLnt{s_{{\mathrm{1}}}}  \LNLsym{/}  \LNLmv{x}  \LNLsym{]}  \LNLnt{s_{{\mathrm{2}}}}} {\textsf{M}}{}{}{}\LNLprodnewline
\LNLprodline{|}{\LNLsym{[}  \LNLnt{t}  \LNLsym{/}  \LNLmv{x}  \LNLsym{]}  \LNLnt{s}} {\textsf{M}}{}{}{}\LNLprodnewline
\LNLprodline{|}{\LNLsym{(}  \LNLnt{s}  \LNLsym{)}} {\textsf{S}}{}{}{}\LNLprodnewline
\LNLprodline{|}{ \LNLnt{s} } {\textsf{M}}{}{}{}\LNLprodnewline
\LNLprodline{|}{ \mathsf{F} \LNLnt{t} }{}{}{}{}\LNLprodnewline
\LNLprodline{|}{ CL( \LNLnt{s} ) } {\textsf{M}}{}{}{}}

\newcommand{\LNLt}{
\LNLrulehead{\LNLnt{t}}{::=}{}\LNLprodnewline
\LNLfirstprodline{|}{\LNLmv{x}}{}{}{}{}\LNLprodnewline
\LNLprodline{|}{\LNLmv{b}}{}{}{}{}\LNLprodnewline
\LNLprodline{|}{ \mathsf{u} }{}{}{}{}\LNLprodnewline
\LNLprodline{|}{ \mathsf{let}\, \LNLnt{t_{{\mathrm{1}}}}  :  \LNLnt{X} \,\mathsf{be}\, \LNLnt{p} \,\mathsf{in}\, \LNLnt{t_{{\mathrm{2}}}} }{}{}{}{}\LNLprodnewline
\LNLprodline{|}{\LNLnt{t_{{\mathrm{1}}}}  \times  \LNLnt{t_{{\mathrm{2}}}}}{}{}{}{}\LNLprodnewline
\LNLprodline{|}{ \lambda  \LNLmv{x}  :  \LNLnt{X} . \LNLnt{t} }{}{}{}{}\LNLprodnewline
\LNLprodline{|}{ \mathsf{app}\, \LNLnt{t_{{\mathrm{1}}}} \, \LNLnt{t_{{\mathrm{2}}}} }{}{}{}{}\LNLprodnewline
\LNLprodline{|}{ \mathsf{(} \LNLnt{t_{{\mathrm{1}}}}  ,  \LNLnt{t_{{\mathrm{2}}}} \mathsf{)} }{}{}{}{}\LNLprodnewline
\LNLprodline{|}{ \mathsf{fst(} \LNLnt{t} \mathsf{)} }{}{}{}{}\LNLprodnewline
\LNLprodline{|}{ \mathsf{snd(} \LNLnt{t} \mathsf{)} }{}{}{}{}\LNLprodnewline
\LNLprodline{|}{\LNLsym{[}  \LNLnt{t_{{\mathrm{1}}}}  \LNLsym{/}  \LNLmv{x}  \LNLsym{]}  \LNLnt{t_{{\mathrm{2}}}}} {\textsf{M}}{}{}{}\LNLprodnewline
\LNLprodline{|}{\LNLsym{(}  \LNLnt{t}  \LNLsym{)}} {\textsf{S}}{}{}{}\LNLprodnewline
\LNLprodline{|}{\LNLsym{h(}  \LNLnt{t}  \LNLsym{)}} {\textsf{M}}{}{}{}\LNLprodnewline
\LNLprodline{|}{ \mathsf{G} \LNLnt{s} }{}{}{}{}}

\newcommand{\LNLPh}{
\LNLrulehead{\Phi  ,\ \Psi}{::=}{}\LNLprodnewline
\LNLfirstprodline{|}{ \cdot }{}{}{}{}\LNLprodnewline
\LNLprodline{|}{\Phi_{{\mathrm{1}}}  \LNLsym{,}  \Phi_{{\mathrm{2}}}}{}{}{}{}\LNLprodnewline
\LNLprodline{|}{\LNLmv{x}  \LNLsym{:}  \LNLnt{X}}{}{}{}{}\LNLprodnewline
\LNLprodline{|}{\LNLsym{(}  \Phi  \LNLsym{)}} {\textsf{S}}{}{}{}}

\newcommand{\LNLG}{
\LNLrulehead{\Gamma  ,\ \Delta}{::=}{}\LNLprodnewline
\LNLfirstprodline{|}{ \cdot }{}{}{}{}\LNLprodnewline
\LNLprodline{|}{\LNLmv{x}  \LNLsym{:}  \LNLnt{A}}{}{}{}{}\LNLprodnewline
\LNLprodline{|}{\Phi}{}{}{}{}\LNLprodnewline
\LNLprodline{|}{\Phi  \LNLsym{;}  \Gamma}{}{}{}{}\LNLprodnewline
\LNLprodline{|}{\Gamma_{{\mathrm{1}}}  \LNLsym{,}  \Gamma_{{\mathrm{2}}}}{}{}{}{}\LNLprodnewline
\LNLprodline{|}{\LNLsym{(}  \Gamma  \LNLsym{)}} {\textsf{S}}{}{}{}}

\newcommand{\LNLformula}{
\LNLrulehead{\LNLnt{formula}}{::=}{}\LNLprodnewline
\LNLfirstprodline{|}{\LNLnt{judgement}}{}{}{}{}\LNLprodnewline
\LNLprodline{|}{ \LNLnt{formula_{{\mathrm{1}}}}  \quad  \LNLnt{formula_{{\mathrm{2}}}} } {\textsf{M}}{}{}{}\LNLprodnewline
\LNLprodline{|}{\LNLnt{formula_{{\mathrm{1}}}} \, ... \, \LNLnt{formula_{\LNLmv{i}}}} {\textsf{M}}{}{}{}\LNLprodnewline
\LNLprodline{|}{ \LNLnt{formula} } {\textsf{S}}{}{}{}}

\newcommand{\LNLterminals}{
\LNLrulehead{\LNLnt{terminals}}{::=}{}\LNLprodnewline
\LNLfirstprodline{|}{ \mathsf{1} }{}{}{}{}\LNLprodnewline
\LNLprodline{|}{ \mathrm{I} }{}{}{}{}\LNLprodnewline
\LNLprodline{|}{ \otimes }{}{}{}{}\LNLprodnewline
\LNLprodline{|}{ \times }{}{}{}{}\LNLprodnewline
\LNLprodline{|}{ \circop{e} }{}{}{}{}\LNLprodnewline
\LNLprodline{|}{ \circop{w} }{}{}{}{}\LNLprodnewline
\LNLprodline{|}{ \circop{c} }{}{}{}{}\LNLprodnewline
\LNLprodline{|}{ \rightarrow }{}{}{}{}\LNLprodnewline
\LNLprodline{|}{ \multimap }{}{}{}{}\LNLprodnewline
\LNLprodline{|}{ \vdash_\mathcal{C} }{}{}{}{}\LNLprodnewline
\LNLprodline{|}{ \vdash_\mathcal{L} }{}{}{}{}\LNLprodnewline
\LNLprodline{|}{ \leadsto }{}{}{}{}}

\newcommand{\LNLJtype}{
\LNLrulehead{\LNLnt{Jtype}}{::=}{}\LNLprodnewline
\LNLfirstprodline{|}{\Phi  \vdash_\mathcal{C}  \LNLnt{t}  \LNLsym{:}  \LNLnt{X}}{}{}{}{}\LNLprodnewline
\LNLprodline{|}{\Gamma  \vdash_\mathcal{L}  \LNLnt{s}  \LNLsym{:}  \LNLnt{A}}{}{}{}{}}

\newcommand{\LNLReduction}{
\LNLrulehead{\LNLnt{Reduction}}{::=}{}\LNLprodnewline
\LNLfirstprodline{|}{\LNLnt{t_{{\mathrm{1}}}}  \leadsto  \LNLnt{t_{{\mathrm{2}}}}}{}{}{}{}\LNLprodnewline
\LNLprodline{|}{\LNLnt{s_{{\mathrm{1}}}}  \leadsto  \LNLnt{s_{{\mathrm{2}}}}}{}{}{}{}}

\newcommand{\LNLjudgement}{
\LNLrulehead{\LNLnt{judgement}}{::=}{}\LNLprodnewline
\LNLfirstprodline{|}{\LNLnt{Jtype}}{}{}{}{}\LNLprodnewline
\LNLprodline{|}{\LNLnt{Reduction}}{}{}{}{}}

\newcommand{\LNLuserXXsyntax}{
\LNLrulehead{\LNLnt{user\_syntax}}{::=}{}\LNLprodnewline
\LNLfirstprodline{|}{\LNLmv{vars}}{}{}{}{}\LNLprodnewline
\LNLprodline{|}{\LNLmv{ivar}}{}{}{}{}\LNLprodnewline
\LNLprodline{|}{\LNLmv{const}}{}{}{}{}\LNLprodnewline
\LNLprodline{|}{\LNLnt{A}}{}{}{}{}\LNLprodnewline
\LNLprodline{|}{\LNLnt{X}}{}{}{}{}\LNLprodnewline
\LNLprodline{|}{\LNLnt{T}}{}{}{}{}\LNLprodnewline
\LNLprodline{|}{\LNLnt{p}}{}{}{}{}\LNLprodnewline
\LNLprodline{|}{\LNLnt{s}}{}{}{}{}\LNLprodnewline
\LNLprodline{|}{\LNLnt{t}}{}{}{}{}\LNLprodnewline
\LNLprodline{|}{\Phi}{}{}{}{}\LNLprodnewline
\LNLprodline{|}{\Gamma}{}{}{}{}\LNLprodnewline
\LNLprodline{|}{\LNLnt{formula}}{}{}{}{}\LNLprodnewline
\LNLprodline{|}{\LNLnt{terminals}}{}{}{}{}}

\newcommand{\LNLgrammar}{\LNLgrammartabular{
\LNLA\LNLinterrule
\LNLX\LNLinterrule
\LNLT\LNLinterrule
\LNLp\LNLinterrule
\LNLs\LNLinterrule
\LNLt\LNLinterrule
\LNLPh\LNLinterrule
\LNLG\LNLinterrule
\LNLformula\LNLinterrule
\LNLterminals\LNLinterrule
\LNLJtype\LNLinterrule
\LNLReduction\LNLinterrule
\LNLjudgement\LNLinterrule
\LNLuserXXsyntax\LNLafterlastrule
}}

% defnss
% defns Jtype
%% defn tty
\newcommand{\LNLdruleTXXidName}[0]{\LNLdrulename{T\_id}}
\newcommand{\LNLdruleTXXid}[1]{\LNLdrule[#1]{%
}{
\Phi  \LNLsym{,}  \LNLmv{x}  \LNLsym{:}  \LNLnt{X}  \vdash_\mathcal{C}  \LNLmv{x}  \LNLsym{:}  \LNLnt{X}}{%
{\LNLdruleTXXidName}{}%
}}


\newcommand{\LNLdruleTXXOneIName}[0]{\LNLdrulename{T\_1I}}
\newcommand{\LNLdruleTXXOneI}[1]{\LNLdrule[#1]{%
}{
\Phi  \vdash_\mathcal{C}   \mathsf{u}   \LNLsym{:}   \mathsf{1} }{%
{\LNLdruleTXXOneIName}{}%
}}


\newcommand{\LNLdruleTXXprodIName}[0]{\LNLdrulename{T\_prodI}}
\newcommand{\LNLdruleTXXprodI}[1]{\LNLdrule[#1]{%
\LNLpremise{ \Phi  \vdash_\mathcal{C}  \LNLnt{t_{{\mathrm{1}}}}  \LNLsym{:}  \LNLnt{X}  \quad  \Phi  \vdash_\mathcal{C}  \LNLnt{t_{{\mathrm{2}}}}  \LNLsym{:}  \LNLnt{Y} }%
}{
\Phi  \vdash_\mathcal{C}   \mathsf{(} \LNLnt{t_{{\mathrm{1}}}}  ,  \LNLnt{t_{{\mathrm{2}}}} \mathsf{)}   \LNLsym{:}  \LNLnt{X}  \times  \LNLnt{Y}}{%
{\LNLdruleTXXprodIName}{}%
}}


\newcommand{\LNLdruleTXXprodEOneName}[0]{\LNLdrulename{T\_prodE1}}
\newcommand{\LNLdruleTXXprodEOne}[1]{\LNLdrule[#1]{%
\LNLpremise{\Phi  \vdash_\mathcal{C}  \LNLnt{t}  \LNLsym{:}  \LNLnt{X}  \times  \LNLnt{Y}}%
}{
\Phi  \vdash_\mathcal{C}   \mathsf{fst(} \LNLnt{t} \mathsf{)}   \LNLsym{:}  \LNLnt{X}}{%
{\LNLdruleTXXprodEOneName}{}%
}}


\newcommand{\LNLdruleTXXprodETwoName}[0]{\LNLdrulename{T\_prodE2}}
\newcommand{\LNLdruleTXXprodETwo}[1]{\LNLdrule[#1]{%
\LNLpremise{\Phi  \vdash_\mathcal{C}  \LNLnt{t}  \LNLsym{:}  \LNLnt{X}  \times  \LNLnt{Y}}%
}{
\Phi  \vdash_\mathcal{C}   \mathsf{snd(} \LNLnt{t} \mathsf{)}   \LNLsym{:}  \LNLnt{Y}}{%
{\LNLdruleTXXprodETwoName}{}%
}}


\newcommand{\LNLdruleTXXimpIName}[0]{\LNLdrulename{T\_impI}}
\newcommand{\LNLdruleTXXimpI}[1]{\LNLdrule[#1]{%
\LNLpremise{\Phi  \LNLsym{,}  \LNLmv{x}  \LNLsym{:}  \LNLnt{X}  \vdash_\mathcal{C}  \LNLnt{t}  \LNLsym{:}  \LNLnt{Y}}%
}{
\Phi  \vdash_\mathcal{C}   \lambda  \LNLmv{x}  :  \LNLnt{X} . \LNLnt{t}   \LNLsym{:}  \LNLnt{X}  \rightarrow  \LNLnt{Y}}{%
{\LNLdruleTXXimpIName}{}%
}}


\newcommand{\LNLdruleTXXimpEName}[0]{\LNLdrulename{T\_impE}}
\newcommand{\LNLdruleTXXimpE}[1]{\LNLdrule[#1]{%
\LNLpremise{ \Phi  \vdash_\mathcal{C}  \LNLnt{t_{{\mathrm{1}}}}  \LNLsym{:}  \LNLnt{X}  \rightarrow  \LNLnt{Y}  \quad  \Phi  \vdash_\mathcal{C}  \LNLnt{t_{{\mathrm{2}}}}  \LNLsym{:}  \LNLnt{X} }%
}{
\Phi  \vdash_\mathcal{C}   \mathsf{app}\, \LNLnt{t_{{\mathrm{1}}}} \, \LNLnt{t_{{\mathrm{2}}}}   \LNLsym{:}  \LNLnt{Y}}{%
{\LNLdruleTXXimpEName}{}%
}}


\newcommand{\LNLdruleTXXGIName}[0]{\LNLdrulename{T\_GI}}
\newcommand{\LNLdruleTXXGI}[1]{\LNLdrule[#1]{%
\LNLpremise{\Phi  \vdash_\mathcal{L}  \LNLnt{s}  \LNLsym{:}  \LNLnt{A}}%
}{
\Phi  \vdash_\mathcal{C}   \mathsf{G} \LNLnt{s}   \LNLsym{:}   \mathsf{G} \LNLnt{A} }{%
{\LNLdruleTXXGIName}{}%
}}


\newcommand{\LNLdruleTXXsubName}[0]{\LNLdrulename{T\_sub}}
\newcommand{\LNLdruleTXXsub}[1]{\LNLdrule[#1]{%
\LNLpremise{ \Phi  \vdash_\mathcal{C}  \LNLnt{t_{{\mathrm{1}}}}  \LNLsym{:}  \LNLnt{X}  \quad  \LNLmv{x}  \LNLsym{:}  \LNLnt{X}  \LNLsym{,}  \Phi  \vdash_\mathcal{C}  \LNLnt{t_{{\mathrm{2}}}}  \LNLsym{:}  \LNLnt{Y} }%
}{
\Phi  \vdash_\mathcal{C}  \LNLsym{[}  \LNLnt{t_{{\mathrm{1}}}}  \LNLsym{/}  \LNLmv{x}  \LNLsym{]}  \LNLnt{t_{{\mathrm{2}}}}  \LNLsym{:}  \LNLnt{Y}}{%
{\LNLdruleTXXsubName}{}%
}}


\newcommand{\LNLdruleTXXweakeningName}[0]{\LNLdrulename{T\_weakening}}
\newcommand{\LNLdruleTXXweakening}[1]{\LNLdrule[#1]{%
\LNLpremise{\Phi  \vdash_\mathcal{C}  \LNLnt{t}  \LNLsym{:}  \LNLnt{Y}}%
}{
\Phi  \LNLsym{,}  \LNLmv{x}  \LNLsym{:}  \LNLnt{X}  \vdash_\mathcal{C}  \LNLnt{t}  \LNLsym{:}  \LNLnt{Y}}{%
{\LNLdruleTXXweakeningName}{}%
}}

\newcommand{\LNLdefntty}[1]{\begin{LNLdefnblock}[#1]{$\Phi  \vdash_\mathcal{C}  \LNLnt{t}  \LNLsym{:}  \LNLnt{X}$}{}
\LNLusedrule{\LNLdruleTXXid{}}
\LNLusedrule{\LNLdruleTXXOneI{}}
\LNLusedrule{\LNLdruleTXXprodI{}}
\LNLusedrule{\LNLdruleTXXprodEOne{}}
\LNLusedrule{\LNLdruleTXXprodETwo{}}
\LNLusedrule{\LNLdruleTXXimpI{}}
\LNLusedrule{\LNLdruleTXXimpE{}}
\LNLusedrule{\LNLdruleTXXGI{}}
\LNLusedrule{\LNLdruleTXXsub{}}
\LNLusedrule{\LNLdruleTXXweakening{}}
\end{LNLdefnblock}}

%% defn sty
\newcommand{\LNLdruleSXXidName}[0]{\LNLdrulename{S\_id}}
\newcommand{\LNLdruleSXXid}[1]{\LNLdrule[#1]{%
}{
\Phi  \LNLsym{;}  \LNLmv{x}  \LNLsym{:}  \LNLnt{A}  \vdash_\mathcal{L}  \LNLmv{x}  \LNLsym{:}  \LNLnt{A}}{%
{\LNLdruleSXXidName}{}%
}}


\newcommand{\LNLdruleSXXtenIName}[0]{\LNLdrulename{S\_tenI}}
\newcommand{\LNLdruleSXXtenI}[1]{\LNLdrule[#1]{%
\LNLpremise{ \Phi  \LNLsym{;}  \Gamma  \vdash_\mathcal{L}  \LNLnt{s_{{\mathrm{1}}}}  \LNLsym{:}  \LNLnt{A}  \quad  \Phi  \LNLsym{;}  \Delta  \vdash_\mathcal{L}  \LNLnt{s_{{\mathrm{2}}}}  \LNLsym{:}  \LNLnt{B} }%
}{
\Phi  \LNLsym{;}  \Gamma  \LNLsym{,}  \Delta  \vdash_\mathcal{L}  \LNLnt{s_{{\mathrm{1}}}}  \otimes  \LNLnt{s_{{\mathrm{2}}}}  \LNLsym{:}  \LNLnt{A}  \otimes  \LNLnt{B}}{%
{\LNLdruleSXXtenIName}{}%
}}


\newcommand{\LNLdruleSXXtenEName}[0]{\LNLdrulename{S\_tenE}}
\newcommand{\LNLdruleSXXtenE}[1]{\LNLdrule[#1]{%
\LNLpremise{ \Phi  \LNLsym{;}  \Gamma  \vdash_\mathcal{L}  \LNLnt{s_{{\mathrm{1}}}}  \LNLsym{:}  \LNLnt{A}  \otimes  \LNLnt{B}  \quad  \Phi  \LNLsym{;}  \Delta  \LNLsym{,}  \LNLmv{x}  \LNLsym{:}  \LNLnt{A}  \LNLsym{,}  \LNLmv{y}  \LNLsym{:}  \LNLnt{B}  \vdash_\mathcal{L}  \LNLnt{s_{{\mathrm{2}}}}  \LNLsym{:}  \LNLnt{C} }%
}{
\Phi  \LNLsym{;}  \Gamma  \LNLsym{,}  \Delta  \vdash_\mathcal{L}   \mathsf{let}\, \LNLnt{s_{{\mathrm{1}}}}  :  \LNLnt{A}  \otimes  \LNLnt{B} \,\mathsf{be}\, \LNLmv{x}  \otimes  \LNLmv{y} \,\mathsf{in}\, \LNLnt{s_{{\mathrm{2}}}}   \LNLsym{:}  \LNLnt{C}}{%
{\LNLdruleSXXtenEName}{}%
}}


\newcommand{\LNLdruleSXXIIName}[0]{\LNLdrulename{S\_II}}
\newcommand{\LNLdruleSXXII}[1]{\LNLdrule[#1]{%
}{
\Phi  \vdash_\mathcal{L}   \ast   \LNLsym{:}   \mathrm{I} }{%
{\LNLdruleSXXIIName}{}%
}}


\newcommand{\LNLdruleSXXIEName}[0]{\LNLdrulename{S\_IE}}
\newcommand{\LNLdruleSXXIE}[1]{\LNLdrule[#1]{%
\LNLpremise{ \Phi  \LNLsym{;}  \Gamma  \vdash_\mathcal{L}  \LNLnt{s_{{\mathrm{1}}}}  \LNLsym{:}   \mathrm{I}   \quad  \Phi  \LNLsym{;}  \Delta  \vdash_\mathcal{L}  \LNLnt{s_{{\mathrm{2}}}}  \LNLsym{:}  \LNLnt{A} }%
}{
\Phi  \LNLsym{;}  \Gamma  \LNLsym{,}  \Delta  \vdash_\mathcal{L}   \mathsf{let}\, \LNLnt{s_{{\mathrm{1}}}}  :   \mathrm{I}  \,\mathsf{be}\,  \ast  \,\mathsf{in}\, \LNLnt{s_{{\mathrm{2}}}}   \LNLsym{:}  \LNLnt{A}}{%
{\LNLdruleSXXIEName}{}%
}}


\newcommand{\LNLdruleSXXimpIName}[0]{\LNLdrulename{S\_impI}}
\newcommand{\LNLdruleSXXimpI}[1]{\LNLdrule[#1]{%
\LNLpremise{\Phi  \LNLsym{;}  \Gamma  \LNLsym{,}  \LNLmv{x}  \LNLsym{:}  \LNLnt{A}  \vdash_\mathcal{L}  \LNLnt{s}  \LNLsym{:}  \LNLnt{B}}%
}{
\Phi  \LNLsym{;}  \Gamma  \vdash_\mathcal{L}   \lambda  \LNLmv{x}  :  \LNLnt{A} . \LNLnt{s}   \LNLsym{:}  \LNLnt{A}  \multimap  \LNLnt{B}}{%
{\LNLdruleSXXimpIName}{}%
}}


\newcommand{\LNLdruleSXXimpEName}[0]{\LNLdrulename{S\_impE}}
\newcommand{\LNLdruleSXXimpE}[1]{\LNLdrule[#1]{%
\LNLpremise{ \Phi  \LNLsym{;}  \Gamma  \vdash_\mathcal{L}  \LNLnt{s_{{\mathrm{1}}}}  \LNLsym{:}  \LNLnt{A}  \multimap  \LNLnt{B}  \quad  \Phi  \LNLsym{;}  \Delta  \vdash_\mathcal{L}  \LNLnt{s_{{\mathrm{2}}}}  \LNLsym{:}  \LNLnt{A} }%
}{
\Phi  \LNLsym{;}  \Gamma  \LNLsym{,}  \Delta  \vdash_\mathcal{L}   \mathsf{app}\, \LNLnt{s_{{\mathrm{1}}}} \, \LNLnt{s_{{\mathrm{2}}}}   \LNLsym{:}  \LNLnt{B}}{%
{\LNLdruleSXXimpEName}{}%
}}


\newcommand{\LNLdruleSXXFIName}[0]{\LNLdrulename{S\_FI}}
\newcommand{\LNLdruleSXXFI}[1]{\LNLdrule[#1]{%
\LNLpremise{\Phi  \vdash_\mathcal{C}  \LNLnt{t}  \LNLsym{:}  \LNLnt{X}}%
}{
\Phi  \vdash_\mathcal{L}   \mathsf{F} \LNLnt{t}   \LNLsym{:}   \mathsf{F} \LNLnt{X} }{%
{\LNLdruleSXXFIName}{}%
}}


\newcommand{\LNLdruleSXXFEName}[0]{\LNLdrulename{S\_FE}}
\newcommand{\LNLdruleSXXFE}[1]{\LNLdrule[#1]{%
\LNLpremise{ \Phi  \LNLsym{;}  \Gamma  \vdash_\mathcal{L}  \LNLnt{s_{{\mathrm{1}}}}  \LNLsym{:}   \mathsf{F} \LNLnt{X}   \quad  \Phi  \LNLsym{,}  \LNLmv{x}  \LNLsym{:}  \LNLnt{X}  \LNLsym{;}  \Delta  \vdash_\mathcal{L}  \LNLnt{s_{{\mathrm{2}}}}  \LNLsym{:}  \LNLnt{A} }%
}{
\Phi  \LNLsym{;}  \Gamma  \LNLsym{,}  \Delta  \vdash_\mathcal{L}   \mathsf{let}\, \LNLnt{s_{{\mathrm{1}}}}  :   \mathsf{F} \LNLnt{X}  \,\mathsf{be}\,  \mathsf{F}\, \LNLmv{x}  \,\mathsf{in}\, \LNLnt{s_{{\mathrm{2}}}}   \LNLsym{:}  \LNLnt{A}}{%
{\LNLdruleSXXFEName}{}%
}}


\newcommand{\LNLdruleSXXGEName}[0]{\LNLdrulename{S\_GE}}
\newcommand{\LNLdruleSXXGE}[1]{\LNLdrule[#1]{%
\LNLpremise{\Phi  \vdash_\mathcal{C}  \LNLnt{t}  \LNLsym{:}   \mathsf{G} \LNLnt{A} }%
}{
\Phi  \vdash_\mathcal{L}   \mathsf{derelict}\, \LNLnt{t}   \LNLsym{:}  \LNLnt{A}}{%
{\LNLdruleSXXGEName}{}%
}}


\newcommand{\LNLdruleSXXsubOneName}[0]{\LNLdrulename{S\_sub1}}
\newcommand{\LNLdruleSXXsubOne}[1]{\LNLdrule[#1]{%
\LNLpremise{ \Phi  \vdash_\mathcal{C}  \LNLnt{t}  \LNLsym{:}  \LNLnt{X}  \quad  \LNLmv{x}  \LNLsym{:}  \LNLnt{X}  \LNLsym{,}  \Phi  \LNLsym{;}  \Gamma  \vdash_\mathcal{L}  \LNLnt{s}  \LNLsym{:}  \LNLnt{A} }%
}{
\Phi  \LNLsym{;}  \Gamma  \vdash_\mathcal{L}  \LNLsym{[}  \LNLnt{t}  \LNLsym{/}  \LNLmv{x}  \LNLsym{]}  \LNLnt{s}  \LNLsym{:}  \LNLnt{A}}{%
{\LNLdruleSXXsubOneName}{}%
}}


\newcommand{\LNLdruleSXXsubTwoName}[0]{\LNLdrulename{S\_sub2}}
\newcommand{\LNLdruleSXXsubTwo}[1]{\LNLdrule[#1]{%
\LNLpremise{ \Phi  \LNLsym{;}  \Gamma  \vdash_\mathcal{L}  \LNLnt{s_{{\mathrm{1}}}}  \LNLsym{:}  \LNLnt{A}  \quad  \Phi  \LNLsym{;}  \LNLmv{x}  \LNLsym{:}  \LNLnt{A}  \LNLsym{,}  \Delta  \vdash_\mathcal{L}  \LNLnt{s_{{\mathrm{2}}}}  \LNLsym{:}  \LNLnt{B} }%
}{
\Phi  \LNLsym{;}  \Gamma  \LNLsym{,}  \Delta  \vdash_\mathcal{L}  \LNLsym{[}  \LNLnt{s_{{\mathrm{1}}}}  \LNLsym{/}  \LNLmv{x}  \LNLsym{]}  \LNLnt{s_{{\mathrm{2}}}}  \LNLsym{:}  \LNLnt{B}}{%
{\LNLdruleSXXsubTwoName}{}%
}}


\newcommand{\LNLdruleSXXweakeningName}[0]{\LNLdrulename{S\_weakening}}
\newcommand{\LNLdruleSXXweakening}[1]{\LNLdrule[#1]{%
\LNLpremise{\Phi  \LNLsym{;}  \Delta  \vdash_\mathcal{L}  \LNLnt{s}  \LNLsym{:}  \LNLnt{A}}%
}{
\Phi  \LNLsym{,}  \LNLmv{x}  \LNLsym{:}  \LNLnt{X}  \LNLsym{;}  \Delta  \vdash_\mathcal{L}  \LNLnt{s}  \LNLsym{:}  \LNLnt{A}}{%
{\LNLdruleSXXweakeningName}{}%
}}

\newcommand{\LNLdefnsty}[1]{\begin{LNLdefnblock}[#1]{$\Gamma  \vdash_\mathcal{L}  \LNLnt{s}  \LNLsym{:}  \LNLnt{A}$}{}
\LNLusedrule{\LNLdruleSXXid{}}
\LNLusedrule{\LNLdruleSXXtenI{}}
\LNLusedrule{\LNLdruleSXXtenE{}}
\LNLusedrule{\LNLdruleSXXII{}}
\LNLusedrule{\LNLdruleSXXIE{}}
\LNLusedrule{\LNLdruleSXXimpI{}}
\LNLusedrule{\LNLdruleSXXimpE{}}
\LNLusedrule{\LNLdruleSXXFI{}}
\LNLusedrule{\LNLdruleSXXFE{}}
\LNLusedrule{\LNLdruleSXXGE{}}
\LNLusedrule{\LNLdruleSXXsubOne{}}
\LNLusedrule{\LNLdruleSXXsubTwo{}}
\LNLusedrule{\LNLdruleSXXweakening{}}
\end{LNLdefnblock}}


\newcommand{\LNLdefnsJtype}{
\LNLdefntty{}\LNLdefnsty{}}

% defns Reduction
%% defn tred
\newcommand{\LNLdruleTredXXFstName}[0]{\LNLdrulename{Tred\_Fst}}
\newcommand{\LNLdruleTredXXFst}[1]{\LNLdrule[#1]{%
}{
 \mathsf{fst(}  \mathsf{(} \LNLnt{t_{{\mathrm{1}}}}  ,  \LNLnt{t_{{\mathrm{2}}}} \mathsf{)}  \mathsf{)}   \leadsto  \LNLnt{t_{{\mathrm{1}}}}}{%
{\LNLdruleTredXXFstName}{}%
}}


\newcommand{\LNLdruleTredXXSndName}[0]{\LNLdrulename{Tred\_Snd}}
\newcommand{\LNLdruleTredXXSnd}[1]{\LNLdrule[#1]{%
}{
 \mathsf{snd(}  \mathsf{(} \LNLnt{t_{{\mathrm{1}}}}  ,  \LNLnt{t_{{\mathrm{2}}}} \mathsf{)}  \mathsf{)}   \leadsto  \LNLnt{t_{{\mathrm{2}}}}}{%
{\LNLdruleTredXXSndName}{}%
}}


\newcommand{\LNLdruleTredXXImpOneName}[0]{\LNLdrulename{Tred\_Imp1}}
\newcommand{\LNLdruleTredXXImpOne}[1]{\LNLdrule[#1]{%
}{
 \mathsf{app}\, \LNLsym{(}   \lambda  \LNLmv{x}  :  \LNLnt{X} . \LNLnt{t_{{\mathrm{1}}}}   \LNLsym{)} \, \LNLnt{t_{{\mathrm{2}}}}   \leadsto  \LNLsym{[}  \LNLnt{t_{{\mathrm{2}}}}  \LNLsym{/}  \LNLmv{x}  \LNLsym{]}  \LNLnt{t_{{\mathrm{1}}}}}{%
{\LNLdruleTredXXImpOneName}{}%
}}

\newcommand{\LNLdefntred}[1]{\begin{LNLdefnblock}[#1]{$\LNLnt{t_{{\mathrm{1}}}}  \leadsto  \LNLnt{t_{{\mathrm{2}}}}$}{}
\LNLusedrule{\LNLdruleTredXXFst{}}
\LNLusedrule{\LNLdruleTredXXSnd{}}
\LNLusedrule{\LNLdruleTredXXImpOne{}}
\end{LNLdefnblock}}

%% defn sred
\newcommand{\LNLdruleSredXXTenName}[0]{\LNLdrulename{Sred\_Ten}}
\newcommand{\LNLdruleSredXXTen}[1]{\LNLdrule[#1]{%
}{
 \mathsf{let}\, \LNLnt{s_{{\mathrm{1}}}}  \otimes  \LNLnt{s_{{\mathrm{2}}}}  :  \LNLnt{A}  \otimes  \LNLnt{B} \,\mathsf{be}\, \LNLmv{x}  \otimes  \LNLmv{y} \,\mathsf{in}\, \LNLnt{s_{{\mathrm{3}}}}   \leadsto  \LNLsym{[}  \LNLnt{s_{{\mathrm{2}}}}  \LNLsym{/}  \LNLmv{y}  \LNLsym{]}  \LNLsym{[}  \LNLnt{s_{{\mathrm{1}}}}  \LNLsym{/}  \LNLmv{x}  \LNLsym{]}  \LNLnt{s_{{\mathrm{3}}}}}{%
{\LNLdruleSredXXTenName}{}%
}}


\newcommand{\LNLdruleSredXXUnitName}[0]{\LNLdrulename{Sred\_Unit}}
\newcommand{\LNLdruleSredXXUnit}[1]{\LNLdrule[#1]{%
}{
 \mathsf{let}\,  \ast   :   \mathrm{I}  \,\mathsf{be}\,  \ast  \,\mathsf{in}\, \LNLnt{s}   \leadsto  \LNLnt{s}}{%
{\LNLdruleSredXXUnitName}{}%
}}


\newcommand{\LNLdruleSredXXImpTwoName}[0]{\LNLdrulename{Sred\_Imp2}}
\newcommand{\LNLdruleSredXXImpTwo}[1]{\LNLdrule[#1]{%
}{
 \mathsf{app}\, \LNLsym{(}   \lambda  \LNLmv{x}  :  \LNLnt{A} . \LNLnt{s_{{\mathrm{1}}}}   \LNLsym{)} \, \LNLnt{s_{{\mathrm{2}}}}   \leadsto  \LNLsym{[}  \LNLnt{s_{{\mathrm{2}}}}  \LNLsym{/}  \LNLmv{x}  \LNLsym{]}  \LNLnt{s_{{\mathrm{1}}}}}{%
{\LNLdruleSredXXImpTwoName}{}%
}}


\newcommand{\LNLdruleSredXXGName}[0]{\LNLdrulename{Sred\_G}}
\newcommand{\LNLdruleSredXXG}[1]{\LNLdrule[#1]{%
}{
 \mathsf{derelict}\, \LNLsym{(}   \mathsf{G} \LNLnt{s}   \LNLsym{)}   \leadsto  \LNLnt{s}}{%
{\LNLdruleSredXXGName}{}%
}}


\newcommand{\LNLdruleSredXXFName}[0]{\LNLdrulename{Sred\_F}}
\newcommand{\LNLdruleSredXXF}[1]{\LNLdrule[#1]{%
}{
 \mathsf{let}\,  \mathsf{F} \LNLnt{t}   :   \mathsf{F} \LNLnt{X}  \,\mathsf{be}\,  \mathsf{F}\, \LNLmv{x}  \,\mathsf{in}\, \LNLnt{s}   \leadsto  \LNLsym{[}  \LNLnt{t}  \LNLsym{/}  \LNLmv{x}  \LNLsym{]}  \LNLnt{s}}{%
{\LNLdruleSredXXFName}{}%
}}

\newcommand{\LNLdefnsred}[1]{\begin{LNLdefnblock}[#1]{$\LNLnt{s_{{\mathrm{1}}}}  \leadsto  \LNLnt{s_{{\mathrm{2}}}}$}{}
\LNLusedrule{\LNLdruleSredXXTen{}}
\LNLusedrule{\LNLdruleSredXXUnit{}}
\LNLusedrule{\LNLdruleSredXXImpTwo{}}
\LNLusedrule{\LNLdruleSredXXG{}}
\LNLusedrule{\LNLdruleSredXXF{}}
\end{LNLdefnblock}}


\newcommand{\LNLdefnsReduction}{
\LNLdefntred{}\LNLdefnsred{}}

\newcommand{\LNLdefnss}{
\LNLdefnsJtype
\LNLdefnsReduction
}

\newcommand{\LNLall}{\LNLmetavars\\[0pt]
\LNLgrammar\\[5.0mm]
\LNLdefnss}



%% This renames Barr's \to to \mto.  This allows us to use \to for imp
%% and \mto for a inline morphism.
\let\mto\to
\let\to\relax
\newcommand{\to}{\rightarrow}
\newcommand{\ndto}[1]{\to_{#1}}
\newcommand{\ndwedge}[1]{\wedge_{#1}}
\newcommand{\rto}{\leftharpoonup}
\newcommand{\lto}{\rightharpoonup}
\newcommand{\tri}{\triangleright}

% Commands that are useful for writing about type theory and programming language design.
%% \newcommand{\case}[4]{\text{case}\ #1\ \text{of}\ #2\text{.}#3\text{,}#2\text{.}#4}
\newcommand{\interp}[1]{\llbracket #1 \rrbracket}
\newcommand{\normto}[0]{\rightsquigarrow^{!}}
\newcommand{\join}[0]{\downarrow}
\newcommand{\redto}[0]{\rightsquigarrow}
\newcommand{\nat}[0]{\mathbb{N}}
\newcommand{\fun}[2]{\lambda #1.#2}
\newcommand{\CRI}[0]{\text{CR-Norm}}
\newcommand{\CRII}[0]{\text{CR-Pres}}
\newcommand{\CRIII}[0]{\text{CR-Prog}}
\newcommand{\subexp}[0]{\sqsubseteq}
%% Must include \usepackage{mathrsfs} for this to work.

\date{}

\let\b\relax
\let\d\relax
\let\t\relax
\let\r\relax
\let\c\relax
\let\j\relax
\let\wn\relax
\let\H\relax

% Cat commands.
\newcommand{\powerset}[1]{\mathcal{P}(#1)}
\newcommand{\cat}[1]{\mathcal{#1}}
\newcommand{\func}[1]{\mathsf{#1}}
\newcommand{\iso}[0]{\mathsf{iso}}
\newcommand{\H}[0]{\func{H}}
\newcommand{\J}[0]{\func{J}}
\newcommand{\catop}[1]{\cat{#1}^{\mathsf{op}}}
\newcommand{\Hom}[3]{\mathsf{Hom}_{\cat{#1}}(#2,#3)}
\newcommand{\limp}[0]{\multimap}
\newcommand{\colimp}[0]{\multimapdotinv}
\newcommand{\dial}[1]{\mathsf{Dial_{#1}}(\mathsf{Sets^{op}})}
\newcommand{\dialSets}[1]{\mathsf{Dial_{#1}}(\mathsf{Sets})}
\newcommand{\dcSets}[1]{\mathsf{DC_{#1}}(\mathsf{Sets})}
\newcommand{\sets}[0]{\mathsf{Sets}}
\newcommand{\obj}[1]{\mathsf{Obj}(#1)}
\newcommand{\mor}[1]{\mathsf{Mor(#1)}}
\newcommand{\id}[0]{\mathsf{id}}
\newcommand{\lett}[0]{\mathsf{let}\,}
\newcommand{\inn}[0]{\,\mathsf{in}\,}
\newcommand{\cur}[1]{\mathsf{cur}(#1)}
\newcommand{\curi}[1]{\mathsf{cur}^{-1}(#1)}

\newcommand{\w}[1]{\mathsf{weak}_{#1}}
\newcommand{\c}[1]{\mathsf{contra}_{#1}}
\newcommand{\cL}[1]{\mathsf{contraL}_{#1}}
\newcommand{\cR}[1]{\mathsf{contraR}_{#1}}
\newcommand{\e}[1]{\mathsf{ex}_{#1}}

\newcommand{\m}[1]{\mathsf{m}_{#1}}
\newcommand{\n}[1]{\mathsf{n}_{#1}}
\newcommand{\b}[1]{\mathsf{b}_{#1}}
\newcommand{\d}[1]{\mathsf{d}_{#1}}
\newcommand{\h}[1]{\mathsf{h}_{#1}}
\newcommand{\p}[1]{\mathsf{p}_{#1}}
\newcommand{\q}[1]{\mathsf{q}_{#1}}
\newcommand{\t}[1]{\mathsf{t}_{#1}}
\newcommand{\r}[1]{\mathsf{r}_{#1}}
\newcommand{\s}[1]{\mathsf{s}_{#1}}
\newcommand{\j}[1]{\mathsf{j}_{#1}}
\newcommand{\jinv}[1]{\mathsf{j}^{-1}_{#1}}
\newcommand{\wn}[0]{\mathop{?}}
\newcommand{\codiag}[1]{\bigtriangledown_{#1}}

\newcommand{\seq}{\rhd}

\newenvironment{changemargin}[2]{%
  \begin{list}{}{%
    \setlength{\topsep}{0pt}%
    \setlength{\leftmargin}{#1}%
    \setlength{\rightmargin}{#2}%
    \setlength{\listparindent}{\parindent}%
    \setlength{\itemindent}{\parindent}%
    \setlength{\parsep}{\parskip}%
  }%
  \item[]}{\end{list}}

\newenvironment{diagram}{
  \begin{center}
    \begin{math}
      \bfig
}{
      \efig
    \end{math}
  \end{center}
}

%% Ott
% % generated by Ott 0.25 from: Elle/Elle.ott
\newcommand{\Elledrule}[4][]{{\displaystyle\frac{\begin{array}{l}#2\end{array}}{#3}\quad\Elledrulename{#4}}}
\newcommand{\Elleusedrule}[1]{\[#1\]}
\newcommand{\Ellepremise}[1]{ #1 \\}
\newenvironment{Elledefnblock}[3][]{ \framebox{\mbox{#2}} \quad #3 \\[0pt]}{}
\newenvironment{Ellefundefnblock}[3][]{ \framebox{\mbox{#2}} \quad #3 \\[0pt]\begin{displaymath}\begin{array}{l}}{\end{array}\end{displaymath}}
\newcommand{\Ellefunclause}[2]{ #1 \equiv #2 \\}
\newcommand{\Ellent}[1]{\mathit{#1}}
\newcommand{\Ellemv}[1]{\mathit{#1}}
\newcommand{\Ellekw}[1]{\mathbf{#1}}
\newcommand{\Ellesym}[1]{#1}
\newcommand{\Ellecom}[1]{\text{#1}}
\newcommand{\Elledrulename}[1]{\textsc{#1}}
\newcommand{\Ellecomplu}[5]{\overline{#1}^{\,#2\in #3 #4 #5}}
\newcommand{\Ellecompu}[3]{\overline{#1}^{\,#2<#3}}
\newcommand{\Ellecomp}[2]{\overline{#1}^{\,#2}}
\newcommand{\Ellegrammartabular}[1]{\begin{supertabular}{llcllllll}#1\end{supertabular}}
\newcommand{\Ellemetavartabular}[1]{\begin{supertabular}{ll}#1\end{supertabular}}
\newcommand{\Ellerulehead}[3]{$#1$ & & $#2$ & & & \multicolumn{2}{l}{#3}}
\newcommand{\Elleprodline}[6]{& & $#1$ & $#2$ & $#3 #4$ & $#5$ & $#6$}
\newcommand{\Ellefirstprodline}[6]{\Elleprodline{#1}{#2}{#3}{#4}{#5}{#6}}
\newcommand{\Ellelongprodline}[2]{& & $#1$ & \multicolumn{4}{l}{$#2$}}
\newcommand{\Ellefirstlongprodline}[2]{\Ellelongprodline{#1}{#2}}
\newcommand{\Ellebindspecprodline}[6]{\Elleprodline{#1}{#2}{#3}{#4}{#5}{#6}}
\newcommand{\Elleprodnewline}{\\}
\newcommand{\Elleinterrule}{\\[5.0mm]}
\newcommand{\Elleafterlastrule}{\\}
\newcommand{\Ellemetavars}{
\Ellemetavartabular{
 $ \Ellemv{vars} ,\, \Ellemv{n} ,\, \Ellemv{a} ,\, \Ellemv{x} ,\, \Ellemv{y} ,\, \Ellemv{z} ,\, \Ellemv{w} ,\, \Ellemv{m} ,\, \Ellemv{o} $ &  \\
 $ \Ellemv{ivar} ,\, \Ellemv{i} ,\, \Ellemv{k} ,\, \Ellemv{j} ,\, \Ellemv{l} $ &  \\
 $ \Ellemv{const} ,\, \Ellemv{b} $ &  \\
}}

\newcommand{\ElleA}{
\Ellerulehead{\Ellent{A}  ,\ \Ellent{B}  ,\ \Ellent{C}  ,\ D}{::=}{}\Elleprodnewline
\Ellefirstprodline{|}{ \mathsf{B} }{}{}{}{}\Elleprodnewline
\Elleprodline{|}{ \mathsf{Unit} }{}{}{}{}\Elleprodnewline
\Elleprodline{|}{\Ellent{A}  \triangleright  \Ellent{B}}{}{}{}{}\Elleprodnewline
\Elleprodline{|}{\Ellent{A}  \rightharpoonup  \Ellent{B}}{}{}{}{}\Elleprodnewline
\Elleprodline{|}{\Ellent{A}  \leftharpoonup  \Ellent{B}}{}{}{}{}\Elleprodnewline
\Elleprodline{|}{\Ellesym{(}  \Ellent{A}  \Ellesym{)}} {\textsf{M}}{}{}{}\Elleprodnewline
\Elleprodline{|}{ \Ellent{A} } {\textsf{M}}{}{}{}\Elleprodnewline
\Elleprodline{|}{ \mathsf{F} \Ellent{X} }{}{}{}{}}

\newcommand{\ElleW}{
\Ellerulehead{\Ellent{W}  ,\ \Ellent{X}  ,\ \Ellent{Y}  ,\ \Ellent{Z}}{::=}{}\Elleprodnewline
\Ellefirstprodline{|}{ \mathsf{B} }{}{}{}{}\Elleprodnewline
\Elleprodline{|}{ \mathsf{Unit} }{}{}{}{}\Elleprodnewline
\Elleprodline{|}{\Ellent{X}  \otimes  \Ellent{Y}}{}{}{}{}\Elleprodnewline
\Elleprodline{|}{\Ellent{X}  \multimap  \Ellent{Y}}{}{}{}{}\Elleprodnewline
\Elleprodline{|}{\Ellesym{(}  \Ellent{X}  \Ellesym{)}} {\textsf{M}}{}{}{}\Elleprodnewline
\Elleprodline{|}{ \Ellent{X} } {\textsf{M}}{}{}{}\Elleprodnewline
\Elleprodline{|}{ \mathsf{G} \Ellent{A} }{}{}{}{}}

\newcommand{\ElleT}{
\Ellerulehead{\Ellent{T}}{::=}{}\Elleprodnewline
\Ellefirstprodline{|}{\Ellent{A}}{}{}{}{}\Elleprodnewline
\Elleprodline{|}{\Ellent{X}}{}{}{}{}}

\newcommand{\Ellep}{
\Ellerulehead{\Ellent{p}  ,\ \Ellent{q}}{::=}{}\Elleprodnewline
\Ellefirstprodline{|}{ \star }{}{}{}{}\Elleprodnewline
\Elleprodline{|}{\Ellemv{x}}{}{}{}{}\Elleprodnewline
\Elleprodline{|}{ \mathsf{triv} }{}{}{}{}\Elleprodnewline
\Elleprodline{|}{ \mathsf{triv} }{}{}{}{}\Elleprodnewline
\Elleprodline{|}{\Ellent{p}  \otimes  \Ellent{p'}}{}{}{}{}\Elleprodnewline
\Elleprodline{|}{\Ellent{p}  \triangleright  \Ellent{p'}}{}{}{}{}\Elleprodnewline
\Elleprodline{|}{ \mathsf{F}\, \Ellent{p} }{}{}{}{}\Elleprodnewline
\Elleprodline{|}{ \mathsf{G}\, \Ellent{p} }{}{}{}{}}

\newcommand{\Elles}{
\Ellerulehead{\Ellent{s}}{::=}{}\Elleprodnewline
\Ellefirstprodline{|}{\Ellemv{x}}{}{}{}{}\Elleprodnewline
\Elleprodline{|}{\Ellemv{b}}{}{}{}{}\Elleprodnewline
\Elleprodline{|}{ \mathsf{triv} }{}{}{}{}\Elleprodnewline
\Elleprodline{|}{ \mathsf{let}\, \Ellent{s_{{\mathrm{1}}}}  :  \Ellent{A} \,\mathsf{be}\, \Ellent{p} \,\mathsf{in}\, \Ellent{s_{{\mathrm{2}}}} }{}{}{}{}\Elleprodnewline
\Elleprodline{|}{ \mathsf{let}\, \Ellent{t}  :  \Ellent{X} \,\mathsf{be}\, \Ellent{p} \,\mathsf{in}\, \Ellent{s} }{}{}{}{}\Elleprodnewline
\Elleprodline{|}{\Ellent{s_{{\mathrm{1}}}}  \triangleright  \Ellent{s_{{\mathrm{2}}}}}{}{}{}{}\Elleprodnewline
\Elleprodline{|}{ \lambda_l  \Ellemv{x}  :  \Ellent{A} . \Ellent{s} }{}{}{}{}\Elleprodnewline
\Elleprodline{|}{ \lambda_r  \Ellemv{x}  :  \Ellent{A} . \Ellent{s} }{}{}{}{}\Elleprodnewline
\Elleprodline{|}{ \mathsf{app}_l\, \Ellent{s_{{\mathrm{1}}}} \, \Ellent{s_{{\mathrm{2}}}} }{}{}{}{}\Elleprodnewline
\Elleprodline{|}{ \mathsf{app}_r\, \Ellent{s_{{\mathrm{1}}}} \, \Ellent{s_{{\mathrm{2}}}} }{}{}{}{}\Elleprodnewline
\Elleprodline{|}{ \mathsf{derelict}\, \Ellent{t} }{}{}{}{}\Elleprodnewline
\Elleprodline{|}{ \mathsf{ex}\, \Ellent{s_{{\mathrm{1}}}} , \Ellent{s_{{\mathrm{2}}}} \,\mathsf{with}\, \Ellemv{x_{{\mathrm{1}}}} , \Ellemv{x_{{\mathrm{2}}}} \,\mathsf{in}\, \Ellent{s_{{\mathrm{3}}}} }{}{}{}{}\Elleprodnewline
\Elleprodline{|}{\Ellesym{[}  \Ellent{s_{{\mathrm{1}}}}  \Ellesym{/}  \Ellemv{x}  \Ellesym{]}  \Ellent{s_{{\mathrm{2}}}}} {\textsf{M}}{}{}{}\Elleprodnewline
\Elleprodline{|}{\Ellesym{[}  \Ellent{t}  \Ellesym{/}  \Ellemv{x}  \Ellesym{]}  \Ellent{s}} {\textsf{M}}{}{}{}\Elleprodnewline
\Elleprodline{|}{\Ellesym{(}  \Ellent{s}  \Ellesym{)}} {\textsf{S}}{}{}{}\Elleprodnewline
\Elleprodline{|}{ \Ellent{s} } {\textsf{M}}{}{}{}\Elleprodnewline
\Elleprodline{|}{ \mathsf{F} \Ellent{t} }{}{}{}{}}

\newcommand{\Ellet}{
\Ellerulehead{\Ellent{t}}{::=}{}\Elleprodnewline
\Ellefirstprodline{|}{\Ellemv{x}}{}{}{}{}\Elleprodnewline
\Elleprodline{|}{\Ellemv{b}}{}{}{}{}\Elleprodnewline
\Elleprodline{|}{ \mathsf{triv} }{}{}{}{}\Elleprodnewline
\Elleprodline{|}{ \mathsf{let}\, \Ellent{t_{{\mathrm{1}}}}  :  \Ellent{X} \,\mathsf{be}\, \Ellent{p} \,\mathsf{in}\, \Ellent{t_{{\mathrm{2}}}} }{}{}{}{}\Elleprodnewline
\Elleprodline{|}{\Ellent{t_{{\mathrm{1}}}}  \otimes  \Ellent{t_{{\mathrm{2}}}}}{}{}{}{}\Elleprodnewline
\Elleprodline{|}{ \lambda  \Ellemv{x}  :  \Ellent{X} . \Ellent{t} }{}{}{}{}\Elleprodnewline
\Elleprodline{|}{ \Ellent{t_{{\mathrm{1}}}}   \Ellent{t_{{\mathrm{2}}}} }{}{}{}{}\Elleprodnewline
\Elleprodline{|}{ \mathsf{ex}\, \Ellent{t_{{\mathrm{1}}}} , \Ellent{t_{{\mathrm{2}}}} \,\mathsf{with}\, \Ellemv{x_{{\mathrm{1}}}} , \Ellemv{x_{{\mathrm{2}}}} \,\mathsf{in}\, \Ellent{t_{{\mathrm{3}}}} }{}{}{}{}\Elleprodnewline
\Elleprodline{|}{\Ellesym{[}  \Ellent{t_{{\mathrm{1}}}}  \Ellesym{/}  \Ellemv{x}  \Ellesym{]}  \Ellent{t_{{\mathrm{2}}}}} {\textsf{M}}{}{}{}\Elleprodnewline
\Elleprodline{|}{\Ellesym{(}  \Ellent{t}  \Ellesym{)}} {\textsf{S}}{}{}{}\Elleprodnewline
\Elleprodline{|}{\Ellesym{h(}  \Ellent{t}  \Ellesym{)}} {\textsf{M}}{}{}{}\Elleprodnewline
\Elleprodline{|}{ \mathsf{G} \Ellent{s} }{}{}{}{}}

\newcommand{\ElleI}{
\Ellerulehead{\Phi  ,\ \Psi}{::=}{}\Elleprodnewline
\Ellefirstprodline{|}{ \cdot }{}{}{}{}\Elleprodnewline
\Elleprodline{|}{\Phi_{{\mathrm{1}}}  \Ellesym{,}  \Phi_{{\mathrm{2}}}}{}{}{}{}\Elleprodnewline
\Elleprodline{|}{\Ellemv{x}  \Ellesym{:}  \Ellent{X}}{}{}{}{}\Elleprodnewline
\Elleprodline{|}{\Ellesym{(}  \Phi  \Ellesym{)}} {\textsf{S}}{}{}{}}

\newcommand{\ElleG}{
\Ellerulehead{\Gamma  ,\ \Delta}{::=}{}\Elleprodnewline
\Ellefirstprodline{|}{ \cdot }{}{}{}{}\Elleprodnewline
\Elleprodline{|}{\Ellemv{x}  \Ellesym{:}  \Ellent{A}}{}{}{}{}\Elleprodnewline
\Elleprodline{|}{\Phi}{}{}{}{}\Elleprodnewline
\Elleprodline{|}{\Gamma_{{\mathrm{1}}}  \Ellesym{;}  \Gamma_{{\mathrm{2}}}}{}{}{}{}\Elleprodnewline
\Elleprodline{|}{\Ellesym{(}  \Gamma  \Ellesym{)}} {\textsf{S}}{}{}{}}

\newcommand{\Elleformula}{
\Ellerulehead{\Ellent{formula}}{::=}{}\Elleprodnewline
\Ellefirstprodline{|}{\Ellent{judgement}}{}{}{}{}\Elleprodnewline
\Elleprodline{|}{ \Ellent{formula_{{\mathrm{1}}}}  \quad  \Ellent{formula_{{\mathrm{2}}}} } {\textsf{M}}{}{}{}\Elleprodnewline
\Elleprodline{|}{\Ellent{formula_{{\mathrm{1}}}} \, ... \, \Ellent{formula_{\Ellemv{i}}}} {\textsf{M}}{}{}{}\Elleprodnewline
\Elleprodline{|}{ \Ellent{formula} } {\textsf{S}}{}{}{}\Elleprodnewline
\Elleprodline{|}{ \Ellemv{x}  \not\in \mathsf{FV}( \Ellent{s} ) }{}{}{}{}\Elleprodnewline
\Elleprodline{|}{ \Ellemv{x}  \not\in |  \Gamma ,  \Delta ,  \Psi  | }{}{}{}{}\Elleprodnewline
\Elleprodline{|}{ \Ellemv{x}  \not\in |  \Gamma ,  \Delta  | }{}{}{}{}}

\newcommand{\Elleterminals}{
\Ellerulehead{\Ellent{terminals}}{::=}{}\Elleprodnewline
\Ellefirstprodline{|}{ \otimes }{}{}{}{}\Elleprodnewline
\Elleprodline{|}{ \triangleright }{}{}{}{}\Elleprodnewline
\Elleprodline{|}{ \circop{e} }{}{}{}{}\Elleprodnewline
\Elleprodline{|}{ \circop{w} }{}{}{}{}\Elleprodnewline
\Elleprodline{|}{ \circop{c} }{}{}{}{}\Elleprodnewline
\Elleprodline{|}{ \rightharpoonup }{}{}{}{}\Elleprodnewline
\Elleprodline{|}{ \leftharpoonup }{}{}{}{}\Elleprodnewline
\Elleprodline{|}{ \multimap }{}{}{}{}\Elleprodnewline
\Elleprodline{|}{ \vdash_\mathcal{C} }{}{}{}{}\Elleprodnewline
\Elleprodline{|}{ \vdash_\mathcal{L} }{}{}{}{}\Elleprodnewline
\Elleprodline{|}{ \leadsto }{}{}{}{}\Elleprodnewline
\Elleprodline{|}{ \leadsto_\mathsf{c} }{}{}{}{}}

\newcommand{\ElleJtype}{
\Ellerulehead{\Ellent{Jtype}}{::=}{}\Elleprodnewline
\Ellefirstprodline{|}{\Phi  \vdash_\mathcal{C}  \Ellent{t}  \Ellesym{:}  \Ellent{X}}{}{}{}{}\Elleprodnewline
\Elleprodline{|}{\Gamma  \vdash_\mathcal{L}  \Ellent{s}  \Ellesym{:}  \Ellent{A}}{}{}{}{}}

\newcommand{\Ellejudgement}{
\Ellerulehead{\Ellent{judgement}}{::=}{}\Elleprodnewline
\Ellefirstprodline{|}{\Ellent{Jtype}}{}{}{}{}}

\newcommand{\ElleuserXXsyntax}{
\Ellerulehead{\Ellent{user\_syntax}}{::=}{}\Elleprodnewline
\Ellefirstprodline{|}{\Ellemv{vars}}{}{}{}{}\Elleprodnewline
\Elleprodline{|}{\Ellemv{ivar}}{}{}{}{}\Elleprodnewline
\Elleprodline{|}{\Ellemv{const}}{}{}{}{}\Elleprodnewline
\Elleprodline{|}{\Ellent{A}}{}{}{}{}\Elleprodnewline
\Elleprodline{|}{\Ellent{W}}{}{}{}{}\Elleprodnewline
\Elleprodline{|}{\Ellent{T}}{}{}{}{}\Elleprodnewline
\Elleprodline{|}{\Ellent{p}}{}{}{}{}\Elleprodnewline
\Elleprodline{|}{\Ellent{s}}{}{}{}{}\Elleprodnewline
\Elleprodline{|}{\Ellent{t}}{}{}{}{}\Elleprodnewline
\Elleprodline{|}{\Phi}{}{}{}{}\Elleprodnewline
\Elleprodline{|}{\Gamma}{}{}{}{}\Elleprodnewline
\Elleprodline{|}{\Ellent{formula}}{}{}{}{}\Elleprodnewline
\Elleprodline{|}{\Ellent{terminals}}{}{}{}{}}

\newcommand{\Ellegrammar}{\Ellegrammartabular{
\ElleA\Elleinterrule
\ElleW\Elleinterrule
\ElleT\Elleinterrule
\Ellep\Elleinterrule
\Elles\Elleinterrule
\Ellet\Elleinterrule
\ElleI\Elleinterrule
\ElleG\Elleinterrule
\Elleformula\Elleinterrule
\Elleterminals\Elleinterrule
\ElleJtype\Elleinterrule
\Ellejudgement\Elleinterrule
\ElleuserXXsyntax\Elleafterlastrule
}}

% defnss
% defns Jtype
%% defn tty
\newcommand{\ElledruleTXXaxName}[0]{\Elledrulename{T\_ax}}
\newcommand{\ElledruleTXXax}[1]{\Elledrule[#1]{%
}{
\Ellemv{x}  \Ellesym{:}  \Ellent{X}  \vdash_\mathcal{C}  \Ellemv{x}  \Ellesym{:}  \Ellent{X}}{%
{\ElledruleTXXaxName}{}%
}}


\newcommand{\ElledruleTXXunitLName}[0]{\Elledrulename{T\_unitL}}
\newcommand{\ElledruleTXXunitL}[1]{\Elledrule[#1]{%
\Ellepremise{\Phi  \Ellesym{,}  \Psi  \vdash_\mathcal{C}  \Ellent{t}  \Ellesym{:}  \Ellent{X}}%
}{
\Phi  \Ellesym{,}  \Ellemv{x}  \Ellesym{:}   \mathsf{Unit}   \Ellesym{,}  \Psi  \vdash_\mathcal{C}   \mathsf{let}\, \Ellemv{x}  :   \mathsf{Unit}  \,\mathsf{be}\,  \mathsf{triv}  \,\mathsf{in}\, \Ellent{t}   \Ellesym{:}  \Ellent{X}}{%
{\ElledruleTXXunitLName}{}%
}}


\newcommand{\ElledruleTXXunitRName}[0]{\Elledrulename{T\_unitR}}
\newcommand{\ElledruleTXXunitR}[1]{\Elledrule[#1]{%
}{
 \cdot   \vdash_\mathcal{C}   \mathsf{triv}   \Ellesym{:}   \mathsf{Unit} }{%
{\ElledruleTXXunitRName}{}%
}}


\newcommand{\ElledruleTXXtenLName}[0]{\Elledrulename{T\_tenL}}
\newcommand{\ElledruleTXXtenL}[1]{\Elledrule[#1]{%
\Ellepremise{\Phi  \Ellesym{,}  \Ellemv{x}  \Ellesym{:}  \Ellent{X}  \Ellesym{,}  \Ellemv{y}  \Ellesym{:}  \Ellent{Y}  \Ellesym{,}  \Psi  \vdash_\mathcal{C}  \Ellent{t}  \Ellesym{:}  \Ellent{Z}}%
}{
\Phi  \Ellesym{,}  \Ellemv{z}  \Ellesym{:}  \Ellent{X}  \otimes  \Ellent{Y}  \Ellesym{,}  \Psi  \vdash_\mathcal{C}   \mathsf{let}\, \Ellemv{z}  :  \Ellent{X}  \otimes  \Ellent{Y} \,\mathsf{be}\, \Ellemv{x}  \otimes  \Ellemv{y} \,\mathsf{in}\, \Ellent{t}   \Ellesym{:}  \Ellent{Z}}{%
{\ElledruleTXXtenLName}{}%
}}


\newcommand{\ElledruleTXXtenRName}[0]{\Elledrulename{T\_tenR}}
\newcommand{\ElledruleTXXtenR}[1]{\Elledrule[#1]{%
\Ellepremise{ \Phi  \vdash_\mathcal{C}  \Ellent{t_{{\mathrm{1}}}}  \Ellesym{:}  \Ellent{X}  \quad  \Psi  \vdash_\mathcal{C}  \Ellent{t_{{\mathrm{2}}}}  \Ellesym{:}  \Ellent{Y} }%
}{
\Phi  \Ellesym{,}  \Psi  \vdash_\mathcal{C}  \Ellent{t_{{\mathrm{1}}}}  \otimes  \Ellent{t_{{\mathrm{2}}}}  \Ellesym{:}  \Ellent{X}  \otimes  \Ellent{Y}}{%
{\ElledruleTXXtenRName}{}%
}}


\newcommand{\ElledruleTXXimpLName}[0]{\Elledrulename{T\_impL}}
\newcommand{\ElledruleTXXimpL}[1]{\Elledrule[#1]{%
\Ellepremise{ \Phi  \vdash_\mathcal{C}  \Ellent{t_{{\mathrm{1}}}}  \Ellesym{:}  \Ellent{X}  \quad  \Psi_{{\mathrm{1}}}  \Ellesym{,}  \Ellemv{x}  \Ellesym{:}  \Ellent{Y}  \Ellesym{,}  \Psi_{{\mathrm{2}}}  \vdash_\mathcal{C}  \Ellent{t_{{\mathrm{2}}}}  \Ellesym{:}  \Ellent{Z} }%
}{
\Psi_{{\mathrm{1}}}  \Ellesym{,}  \Ellemv{y}  \Ellesym{:}  \Ellent{X}  \multimap  \Ellent{Y}  \Ellesym{,}  \Phi  \Ellesym{,}  \Psi_{{\mathrm{2}}}  \vdash_\mathcal{C}  \Ellesym{[}   \Ellemv{y}   \Ellent{t_{{\mathrm{1}}}}   \Ellesym{/}  \Ellemv{x}  \Ellesym{]}  \Ellent{t_{{\mathrm{2}}}}  \Ellesym{:}  \Ellent{Z}}{%
{\ElledruleTXXimpLName}{}%
}}


\newcommand{\ElledruleTXXimpRName}[0]{\Elledrulename{T\_impR}}
\newcommand{\ElledruleTXXimpR}[1]{\Elledrule[#1]{%
\Ellepremise{\Phi  \Ellesym{,}  \Ellemv{x}  \Ellesym{:}  \Ellent{X}  \Ellesym{,}  \Psi  \vdash_\mathcal{C}  \Ellent{t}  \Ellesym{:}  \Ellent{Y}}%
}{
\Phi  \Ellesym{,}  \Psi  \vdash_\mathcal{C}   \lambda  \Ellemv{x}  :  \Ellent{X} . \Ellent{t}   \Ellesym{:}  \Ellent{X}  \multimap  \Ellent{Y}}{%
{\ElledruleTXXimpRName}{}%
}}


\newcommand{\ElledruleTXXGrName}[0]{\Elledrulename{T\_Gr}}
\newcommand{\ElledruleTXXGr}[1]{\Elledrule[#1]{%
\Ellepremise{\Phi  \vdash_\mathcal{L}  \Ellent{s}  \Ellesym{:}  \Ellent{A}}%
}{
\Phi  \vdash_\mathcal{C}   \mathsf{G} \Ellent{s}   \Ellesym{:}   \mathsf{G} \Ellent{A} }{%
{\ElledruleTXXGrName}{}%
}}


\newcommand{\ElledruleTXXcutName}[0]{\Elledrulename{T\_cut}}
\newcommand{\ElledruleTXXcut}[1]{\Elledrule[#1]{%
\Ellepremise{ \Phi  \vdash_\mathcal{C}  \Ellent{t_{{\mathrm{1}}}}  \Ellesym{:}  \Ellent{X}  \quad  \Psi_{{\mathrm{1}}}  \Ellesym{,}  \Ellemv{x}  \Ellesym{:}  \Ellent{X}  \Ellesym{,}  \Psi_{{\mathrm{2}}}  \vdash_\mathcal{C}  \Ellent{t_{{\mathrm{2}}}}  \Ellesym{:}  \Ellent{Y} }%
}{
\Psi_{{\mathrm{1}}}  \Ellesym{,}  \Phi  \Ellesym{,}  \Psi_{{\mathrm{2}}}  \vdash_\mathcal{C}  \Ellesym{[}  \Ellent{t_{{\mathrm{1}}}}  \Ellesym{/}  \Ellemv{x}  \Ellesym{]}  \Ellent{t_{{\mathrm{2}}}}  \Ellesym{:}  \Ellent{Y}}{%
{\ElledruleTXXcutName}{}%
}}


\newcommand{\ElledruleTXXexName}[0]{\Elledrulename{T\_ex}}
\newcommand{\ElledruleTXXex}[1]{\Elledrule[#1]{%
\Ellepremise{\Phi  \Ellesym{,}  \Ellemv{x}  \Ellesym{:}  \Ellent{X}  \Ellesym{,}  \Ellemv{y}  \Ellesym{:}  \Ellent{Y}  \Ellesym{,}  \Psi  \vdash_\mathcal{C}  \Ellent{t}  \Ellesym{:}  \Ellent{Z}}%
}{
\Phi  \Ellesym{,}  \Ellemv{z}  \Ellesym{:}  \Ellent{Y}  \Ellesym{,}  \Ellemv{w}  \Ellesym{:}  \Ellent{X}  \Ellesym{,}  \Psi  \vdash_\mathcal{C}   \mathsf{ex}\, \Ellemv{w} , \Ellemv{z} \,\mathsf{with}\, \Ellemv{x} , \Ellemv{y} \,\mathsf{in}\, \Ellent{t}   \Ellesym{:}  \Ellent{Z}}{%
{\ElledruleTXXexName}{}%
}}

\newcommand{\Elledefntty}[1]{\begin{Elledefnblock}[#1]{$\Phi  \vdash_\mathcal{C}  \Ellent{t}  \Ellesym{:}  \Ellent{X}$}{}
\Elleusedrule{\ElledruleTXXax{}}
\Elleusedrule{\ElledruleTXXunitL{}}
\Elleusedrule{\ElledruleTXXunitR{}}
\Elleusedrule{\ElledruleTXXtenL{}}
\Elleusedrule{\ElledruleTXXtenR{}}
\Elleusedrule{\ElledruleTXXimpL{}}
\Elleusedrule{\ElledruleTXXimpR{}}
\Elleusedrule{\ElledruleTXXGr{}}
\Elleusedrule{\ElledruleTXXcut{}}
\Elleusedrule{\ElledruleTXXex{}}
\end{Elledefnblock}}

%% defn sty
\newcommand{\ElledruleSXXaxName}[0]{\Elledrulename{S\_ax}}
\newcommand{\ElledruleSXXax}[1]{\Elledrule[#1]{%
}{
\Ellemv{x}  \Ellesym{:}  \Ellent{A}  \vdash_\mathcal{L}  \Ellemv{x}  \Ellesym{:}  \Ellent{A}}{%
{\ElledruleSXXaxName}{}%
}}


\newcommand{\ElledruleSXXunitLOneName}[0]{\Elledrulename{S\_unitL1}}
\newcommand{\ElledruleSXXunitLOne}[1]{\Elledrule[#1]{%
\Ellepremise{\Gamma  \Ellesym{;}  \Delta  \vdash_\mathcal{L}  \Ellent{s}  \Ellesym{:}  \Ellent{A}}%
}{
\Gamma  \Ellesym{;}  \Ellemv{x}  \Ellesym{:}   \mathsf{Unit}   \Ellesym{;}  \Delta  \vdash_\mathcal{L}   \mathsf{let}\, \Ellemv{x}  :   \mathsf{Unit}  \,\mathsf{be}\,  \mathsf{triv}  \,\mathsf{in}\, \Ellent{s}   \Ellesym{:}  \Ellent{A}}{%
{\ElledruleSXXunitLOneName}{}%
}}


\newcommand{\ElledruleSXXunitLTwoName}[0]{\Elledrulename{S\_unitL2}}
\newcommand{\ElledruleSXXunitLTwo}[1]{\Elledrule[#1]{%
\Ellepremise{\Gamma  \Ellesym{;}  \Delta  \vdash_\mathcal{L}  \Ellent{s}  \Ellesym{:}  \Ellent{A}}%
}{
\Gamma  \Ellesym{;}  \Ellemv{x}  \Ellesym{:}   \mathsf{Unit}   \Ellesym{;}  \Delta  \vdash_\mathcal{L}   \mathsf{let}\, \Ellemv{x}  :   \mathsf{Unit}  \,\mathsf{be}\,  \mathsf{triv}  \,\mathsf{in}\, \Ellent{s}   \Ellesym{:}  \Ellent{A}}{%
{\ElledruleSXXunitLTwoName}{}%
}}


\newcommand{\ElledruleSXXunitRName}[0]{\Elledrulename{S\_unitR}}
\newcommand{\ElledruleSXXunitR}[1]{\Elledrule[#1]{%
}{
 \cdot   \vdash_\mathcal{L}   \mathsf{triv}   \Ellesym{:}   \mathsf{Unit} }{%
{\ElledruleSXXunitRName}{}%
}}


\newcommand{\ElledruleSXXexName}[0]{\Elledrulename{S\_ex}}
\newcommand{\ElledruleSXXex}[1]{\Elledrule[#1]{%
\Ellepremise{\Gamma  \Ellesym{;}  \Ellemv{x}  \Ellesym{:}  \Ellent{X}  \Ellesym{;}  \Ellemv{y}  \Ellesym{:}  \Ellent{Y}  \Ellesym{;}  \Delta  \vdash_\mathcal{L}  \Ellent{s}  \Ellesym{:}  \Ellent{A}}%
}{
\Gamma  \Ellesym{;}  \Ellemv{z}  \Ellesym{:}  \Ellent{Y}  \Ellesym{;}  \Ellemv{w}  \Ellesym{:}  \Ellent{X}  \Ellesym{;}  \Delta  \vdash_\mathcal{L}   \mathsf{ex}\, \Ellemv{w} , \Ellemv{z} \,\mathsf{with}\, \Ellemv{x} , \Ellemv{y} \,\mathsf{in}\, \Ellent{s}   \Ellesym{:}  \Ellent{A}}{%
{\ElledruleSXXexName}{}%
}}


\newcommand{\ElledruleSXXtenLOneName}[0]{\Elledrulename{S\_tenL1}}
\newcommand{\ElledruleSXXtenLOne}[1]{\Elledrule[#1]{%
\Ellepremise{\Gamma  \Ellesym{;}  \Ellemv{x}  \Ellesym{:}  \Ellent{X}  \Ellesym{;}  \Ellemv{y}  \Ellesym{:}  \Ellent{Y}  \Ellesym{;}  \Delta  \vdash_\mathcal{L}  \Ellent{s}  \Ellesym{:}  \Ellent{A}}%
}{
\Gamma  \Ellesym{;}  \Ellemv{z}  \Ellesym{:}  \Ellent{X}  \otimes  \Ellent{Y}  \Ellesym{;}  \Delta  \vdash_\mathcal{L}   \mathsf{let}\, \Ellemv{z}  :  \Ellent{X}  \otimes  \Ellent{Y} \,\mathsf{be}\, \Ellemv{x}  \otimes  \Ellemv{y} \,\mathsf{in}\, \Ellent{s}   \Ellesym{:}  \Ellent{A}}{%
{\ElledruleSXXtenLOneName}{}%
}}


\newcommand{\ElledruleSXXtenLTwoName}[0]{\Elledrulename{S\_tenL2}}
\newcommand{\ElledruleSXXtenLTwo}[1]{\Elledrule[#1]{%
\Ellepremise{\Gamma  \Ellesym{;}  \Ellemv{x}  \Ellesym{:}  \Ellent{A}  \Ellesym{;}  \Ellemv{y}  \Ellesym{:}  \Ellent{B}  \Ellesym{;}  \Delta  \vdash_\mathcal{L}  \Ellent{s}  \Ellesym{:}  \Ellent{C}}%
}{
\Gamma  \Ellesym{;}  \Ellemv{z}  \Ellesym{:}  \Ellent{A}  \triangleright  \Ellent{B}  \Ellesym{;}  \Delta  \vdash_\mathcal{L}   \mathsf{let}\, \Ellemv{z}  :  \Ellent{A}  \triangleright  \Ellent{B} \,\mathsf{be}\, \Ellemv{x}  \triangleright  \Ellemv{y} \,\mathsf{in}\, \Ellent{s}   \Ellesym{:}  \Ellent{C}}{%
{\ElledruleSXXtenLTwoName}{}%
}}


\newcommand{\ElledruleSXXtenRName}[0]{\Elledrulename{S\_tenR}}
\newcommand{\ElledruleSXXtenR}[1]{\Elledrule[#1]{%
\Ellepremise{ \Gamma  \vdash_\mathcal{L}  \Ellent{s_{{\mathrm{1}}}}  \Ellesym{:}  \Ellent{A}  \quad  \Delta  \vdash_\mathcal{L}  \Ellent{s_{{\mathrm{2}}}}  \Ellesym{:}  \Ellent{B} }%
}{
\Gamma  \Ellesym{;}  \Delta  \vdash_\mathcal{L}  \Ellent{s_{{\mathrm{1}}}}  \triangleright  \Ellent{s_{{\mathrm{2}}}}  \Ellesym{:}  \Ellent{A}  \triangleright  \Ellent{B}}{%
{\ElledruleSXXtenRName}{}%
}}


\newcommand{\ElledruleSXXimpLName}[0]{\Elledrulename{S\_impL}}
\newcommand{\ElledruleSXXimpL}[1]{\Elledrule[#1]{%
\Ellepremise{ \Phi  \vdash_\mathcal{C}  \Ellent{t}  \Ellesym{:}  \Ellent{X}  \quad  \Gamma  \Ellesym{;}  \Ellemv{x}  \Ellesym{:}  \Ellent{Y}  \Ellesym{;}  \Delta  \vdash_\mathcal{L}  \Ellent{s}  \Ellesym{:}  \Ellent{A} }%
}{
\Gamma  \Ellesym{;}  \Ellemv{y}  \Ellesym{:}  \Ellent{X}  \multimap  \Ellent{Y}  \Ellesym{;}  \Phi  \Ellesym{;}  \Delta  \vdash_\mathcal{L}  \Ellesym{[}   \Ellemv{y}   \Ellent{t}   \Ellesym{/}  \Ellemv{x}  \Ellesym{]}  \Ellent{s}  \Ellesym{:}  \Ellent{A}}{%
{\ElledruleSXXimpLName}{}%
}}


\newcommand{\ElledruleSXXimprLName}[0]{\Elledrulename{S\_imprL}}
\newcommand{\ElledruleSXXimprL}[1]{\Elledrule[#1]{%
\Ellepremise{ \Gamma  \vdash_\mathcal{L}  \Ellent{s_{{\mathrm{1}}}}  \Ellesym{:}  \Ellent{A}  \quad  \Delta_{{\mathrm{1}}}  \Ellesym{;}  \Ellemv{x}  \Ellesym{:}  \Ellent{B}  \Ellesym{;}  \Delta_{{\mathrm{2}}}  \vdash_\mathcal{L}  \Ellent{s_{{\mathrm{2}}}}  \Ellesym{:}  \Ellent{C} }%
}{
\Delta_{{\mathrm{1}}}  \Ellesym{;}  \Ellemv{y}  \Ellesym{:}  \Ellent{A}  \rightharpoonup  \Ellent{B}  \Ellesym{;}  \Gamma  \Ellesym{;}  \Delta_{{\mathrm{2}}}  \vdash_\mathcal{L}  \Ellesym{[}   \mathsf{app}_r\, \Ellemv{y} \, \Ellent{s_{{\mathrm{1}}}}   \Ellesym{/}  \Ellemv{x}  \Ellesym{]}  \Ellent{s_{{\mathrm{2}}}}  \Ellesym{:}  \Ellent{C}}{%
{\ElledruleSXXimprLName}{}%
}}


\newcommand{\ElledruleSXXimprRName}[0]{\Elledrulename{S\_imprR}}
\newcommand{\ElledruleSXXimprR}[1]{\Elledrule[#1]{%
\Ellepremise{\Gamma  \Ellesym{;}  \Ellemv{x}  \Ellesym{:}  \Ellent{A}  \vdash_\mathcal{L}  \Ellent{s}  \Ellesym{:}  \Ellent{B}}%
}{
\Gamma  \vdash_\mathcal{L}   \lambda_r  \Ellemv{x}  :  \Ellent{A} . \Ellent{s}   \Ellesym{:}  \Ellent{A}  \rightharpoonup  \Ellent{B}}{%
{\ElledruleSXXimprRName}{}%
}}


\newcommand{\ElledruleSXXimplLName}[0]{\Elledrulename{S\_implL}}
\newcommand{\ElledruleSXXimplL}[1]{\Elledrule[#1]{%
\Ellepremise{ \Gamma  \vdash_\mathcal{L}  \Ellent{s_{{\mathrm{1}}}}  \Ellesym{:}  \Ellent{A}  \quad  \Delta_{{\mathrm{1}}}  \Ellesym{;}  \Ellemv{x}  \Ellesym{:}  \Ellent{B}  \Ellesym{;}  \Delta_{{\mathrm{2}}}  \vdash_\mathcal{L}  \Ellent{s_{{\mathrm{2}}}}  \Ellesym{:}  \Ellent{C} }%
}{
\Delta_{{\mathrm{1}}}  \Ellesym{;}  \Gamma  \Ellesym{;}  \Ellemv{y}  \Ellesym{:}  \Ellent{B}  \leftharpoonup  \Ellent{A}  \Ellesym{;}  \Delta_{{\mathrm{2}}}  \vdash_\mathcal{L}  \Ellesym{[}   \mathsf{app}_l\, \Ellemv{y} \, \Ellent{s_{{\mathrm{1}}}}   \Ellesym{/}  \Ellemv{x}  \Ellesym{]}  \Ellent{s_{{\mathrm{2}}}}  \Ellesym{:}  \Ellent{C}}{%
{\ElledruleSXXimplLName}{}%
}}


\newcommand{\ElledruleSXXimplRName}[0]{\Elledrulename{S\_implR}}
\newcommand{\ElledruleSXXimplR}[1]{\Elledrule[#1]{%
\Ellepremise{\Ellemv{x}  \Ellesym{:}  \Ellent{A}  \Ellesym{;}  \Gamma  \vdash_\mathcal{L}  \Ellent{s}  \Ellesym{:}  \Ellent{B}}%
}{
\Gamma  \vdash_\mathcal{L}   \lambda_l  \Ellemv{x}  :  \Ellent{A} . \Ellent{s}   \Ellesym{:}  \Ellent{B}  \leftharpoonup  \Ellent{A}}{%
{\ElledruleSXXimplRName}{}%
}}


\newcommand{\ElledruleSXXFlName}[0]{\Elledrulename{S\_Fl}}
\newcommand{\ElledruleSXXFl}[1]{\Elledrule[#1]{%
\Ellepremise{\Gamma  \Ellesym{;}  \Ellemv{x}  \Ellesym{:}  \Ellent{X}  \Ellesym{;}  \Delta  \vdash_\mathcal{L}  \Ellent{s}  \Ellesym{:}  \Ellent{A}}%
}{
\Gamma  \Ellesym{;}  \Ellemv{y}  \Ellesym{:}   \mathsf{F} \Ellent{X}   \Ellesym{;}  \Delta  \vdash_\mathcal{L}   \mathsf{let}\, \Ellemv{y}  :   \mathsf{F} \Ellent{X}  \,\mathsf{be}\,  \mathsf{F}\, \Ellemv{x}  \,\mathsf{in}\, \Ellent{s}   \Ellesym{:}  \Ellent{A}}{%
{\ElledruleSXXFlName}{}%
}}


\newcommand{\ElledruleSXXFrName}[0]{\Elledrulename{S\_Fr}}
\newcommand{\ElledruleSXXFr}[1]{\Elledrule[#1]{%
\Ellepremise{\Phi  \vdash_\mathcal{C}  \Ellent{t}  \Ellesym{:}  \Ellent{X}}%
}{
\Phi  \vdash_\mathcal{L}   \mathsf{F} \Ellent{t}   \Ellesym{:}   \mathsf{F} \Ellent{X} }{%
{\ElledruleSXXFrName}{}%
}}


\newcommand{\ElledruleSXXGlName}[0]{\Elledrulename{S\_Gl}}
\newcommand{\ElledruleSXXGl}[1]{\Elledrule[#1]{%
\Ellepremise{\Gamma  \Ellesym{;}  \Ellemv{x}  \Ellesym{:}  \Ellent{A}  \Ellesym{;}  \Delta  \vdash_\mathcal{L}  \Ellent{s}  \Ellesym{:}  \Ellent{B}}%
}{
\Gamma  \Ellesym{;}  \Ellemv{y}  \Ellesym{:}   \mathsf{G} \Ellent{A}   \Ellesym{;}  \Delta  \vdash_\mathcal{L}   \mathsf{let}\, \Ellemv{y}  :   \mathsf{G} \Ellent{A}  \,\mathsf{be}\,  \mathsf{G}\, \Ellemv{x}  \,\mathsf{in}\, \Ellent{s}   \Ellesym{:}  \Ellent{B}}{%
{\ElledruleSXXGlName}{}%
}}


\newcommand{\ElledruleSXXcutOneName}[0]{\Elledrulename{S\_cut1}}
\newcommand{\ElledruleSXXcutOne}[1]{\Elledrule[#1]{%
\Ellepremise{ \Phi  \vdash_\mathcal{C}  \Ellent{t}  \Ellesym{:}  \Ellent{X}  \quad  \Gamma_{{\mathrm{1}}}  \Ellesym{;}  \Ellemv{x}  \Ellesym{:}  \Ellent{X}  \Ellesym{;}  \Gamma_{{\mathrm{2}}}  \vdash_\mathcal{L}  \Ellent{s}  \Ellesym{:}  \Ellent{A} }%
}{
\Gamma_{{\mathrm{1}}}  \Ellesym{;}  \Phi  \Ellesym{;}  \Gamma_{{\mathrm{1}}}  \vdash_\mathcal{L}  \Ellesym{[}  \Ellent{t}  \Ellesym{/}  \Ellemv{x}  \Ellesym{]}  \Ellent{s}  \Ellesym{:}  \Ellent{A}}{%
{\ElledruleSXXcutOneName}{}%
}}


\newcommand{\ElledruleSXXcutTwoName}[0]{\Elledrulename{S\_cut2}}
\newcommand{\ElledruleSXXcutTwo}[1]{\Elledrule[#1]{%
\Ellepremise{ \Gamma  \vdash_\mathcal{L}  \Ellent{s_{{\mathrm{1}}}}  \Ellesym{:}  \Ellent{A}  \quad  \Delta_{{\mathrm{1}}}  \Ellesym{;}  \Ellemv{x}  \Ellesym{:}  \Ellent{A}  \Ellesym{;}  \Delta_{{\mathrm{2}}}  \vdash_\mathcal{L}  \Ellent{s_{{\mathrm{2}}}}  \Ellesym{:}  \Ellent{B} }%
}{
\Delta_{{\mathrm{1}}}  \Ellesym{;}  \Gamma  \Ellesym{;}  \Delta_{{\mathrm{2}}}  \vdash_\mathcal{L}  \Ellesym{[}  \Ellent{s_{{\mathrm{1}}}}  \Ellesym{/}  \Ellemv{x}  \Ellesym{]}  \Ellent{s_{{\mathrm{2}}}}  \Ellesym{:}  \Ellent{B}}{%
{\ElledruleSXXcutTwoName}{}%
}}

\newcommand{\Elledefnsty}[1]{\begin{Elledefnblock}[#1]{$\Gamma  \vdash_\mathcal{L}  \Ellent{s}  \Ellesym{:}  \Ellent{A}$}{}
\Elleusedrule{\ElledruleSXXax{}}
\Elleusedrule{\ElledruleSXXunitLOne{}}
\Elleusedrule{\ElledruleSXXunitLTwo{}}
\Elleusedrule{\ElledruleSXXunitR{}}
\Elleusedrule{\ElledruleSXXex{}}
\Elleusedrule{\ElledruleSXXtenLOne{}}
\Elleusedrule{\ElledruleSXXtenLTwo{}}
\Elleusedrule{\ElledruleSXXtenR{}}
\Elleusedrule{\ElledruleSXXimpL{}}
\Elleusedrule{\ElledruleSXXimprL{}}
\Elleusedrule{\ElledruleSXXimprR{}}
\Elleusedrule{\ElledruleSXXimplL{}}
\Elleusedrule{\ElledruleSXXimplR{}}
\Elleusedrule{\ElledruleSXXFl{}}
\Elleusedrule{\ElledruleSXXFr{}}
\Elleusedrule{\ElledruleSXXGl{}}
\Elleusedrule{\ElledruleSXXcutOne{}}
\Elleusedrule{\ElledruleSXXcutTwo{}}
\end{Elledefnblock}}


\newcommand{\ElledefnsJtype}{
\Elledefntty{}\Elledefnsty{}}

\newcommand{\Elledefnss}{
\ElledefnsJtype
}

\newcommand{\Elleall}{\Ellemetavars\\[0pt]
\Ellegrammar\\[5.0mm]
\Elledefnss}



\title{Non-Commutative Linear Logic in an Adjoint Model}
\author[1]{Jiaming Jiang}
\author[2]{Harley Eades III}
\affil[1]{Computer Science, Augusta University, Augusta, Georgia, USA\\
  \texttt{heades@augusta.edu}}
\affil[2]{Computer Science, North Carolina State University, Raleigh, North Carolina, USA\\
  \texttt{jjiang13@ncsu.edu}}

\Copyright{Harley E. Open and Jiaming J. Access}

\subjclass{TODO}
\keywords{TODO}

\begin{document}

\maketitle 

\begin{abstract}
  TODO
\end{abstract}

\section{Introduction}
\label{sec:introduction}
Joachim Lambek first introduced the Syntactic Calculus, now known as
the Lambek Calculus, in 1958 \cite{Lambek1958}.  Since then the Lambek
Calculus has largely been motivated by providing an explanation of the
mathematics of sentence structure, and can be found at the core of
Categorial Grammar; a term first used in the title of Bar-Hillel,
Gaifman and Shamir (1960), but categorical grammar began with
Ajdukiewicz (1935) quite a few years earlier. For many years the
Lambek Calculus went without use, but around 1980 it was taken up by
logicians interested in Computational Linguistics, especially the ones
interested in Categorial Grammars.

It was computational linguists, ???, who posed the question of whether
it is possible to isolate exchange using a modality in the same way
that the of-course modality of linear logic, $!A$, isolates weakening
and contraction.  de Paiva and Eades III \cite{dePaiva2018} propose
one solution to this problem by extending the Lambek calculus with the
modality characterized by the following sequent calculus inference
rules:
\[
\small
\begin{array}{ccccccccccccccccccccc}  
  \LdruleEr{} & & \LdruleEl{} & & \LdruleEOne{} & & \LdruleETwo{} 
\end{array}
\]
The thing to note is that the modality $\kappa A$ appears on only one
of the operands being exchanged.  That is, these rules along with
those for the tensor product allow one to prove that $\kappa A \otimes
B \multimap B \otimes \kappa A$ holds.  This is somewhat at odds with
algebraic intuition, and it is unclear how this modality could be
decomposed into adjoint functors in a linear/non-linear (LNL)
formalization of the Lambek calculus.

In this paper we show how to add an exchange modality, $eA$, where the
modality now occurs on both operands being exchanged. That is, one can
show that $eA \otimes eB \multimap eB \otimes eA$ holds.  We give a
LNL natural deduction formalization for the Lambek calculus with this
new modality, and two categorical models: a LNL model and a concrete
model in dialectica spaces.  Thus giving a second solution to the
problem proposed above.

The Lambek Calculus also has the potential for many applications in
other areas of computer science, such as, modeling processes.  Linear
Logic has been at the forefront of the study of process calculi for
many years \cite{HONDA20102223,Pratt:1997,ABRAMSKY19945}. We can think
of the commutative tensor product of linear logic as a parallel
operator.  For example, given a process $A$ and a process $B$, then we
can form the process $A \otimes B$ which runs both processes in
parallel.  If we remove commutativity from the tensor product we
obtain a sequential composition instead of parallel composition.  That
is, the process $A \rhd B$ first runs process $A$ and then process $B$
in that order.  Vaughan Pratt has stated that , ``sequential
composition has no evident counterpart in type theory'' see page 11 of
\cite{Pratt:1997}.  We believe that the Lambek Calculus will lead to
filling this hole.  

%% We introduce the idea above of having a modality for exchange, but
%% what about a individual modalities for weakening and contraction?
%% Indeed it is possible to give modalities for these structural rules as
%% well using adjoint models.  Now that we have each structural rule
%% isolated into their own modality is it possible to put them together
%% to form new modalities that combine structural rules?  The answer to
%% this question has already been show to be positive, at least for
%% weakening and contraction, but we extend this line of work to include
%% exchange, and in the future associativity as well.
%% Jacobs~\cite{JACOBS199473} used monads and distributive laws to relate
%% modalities, but Melli{\'e}s~\cite{Mellies:2004} shows how to use
%% adjunctions to accomplish the same, but in a more intuitive and
%% natural way.  Thus, this part of our work is a natural extension of
%% Melli{\'e}s'.

%%% Local Variables: 
%%% mode: latex
%%% TeX-master: main.tex
%%% End:

% section introduction (end)

\section{Related Work}
\label{sec:related_work}
Polakow and Pfenning discuseed Ordered Linear Logic (OLL) \cite{}, which combines
intuitionistic, commutative linear and non-commutative linear logic, OLL contains sequents of
the form $\Gamma,\Delta,\Omega\vdash A$, where $\Gamma$ is a multiset of intuitionistic
assumptions, $\Delta$ is a multiset of commutative linear assumptions, and $\Omega$ is a list of
non-commutative linear assumptions. OLL contains logical connectives from all three the logics.
Therefore, our non-commutative adjoint model is a part of OLL and after combining with Benton's
commutative adjoint model, we would get a simplification of OLL.

Greco and Palmigiano \cite{} also presents a variant of the multiplicative fragment of
non-commutative ILL. But they focus on proper display calculi while we use sequent calculi.

% section introduction (end)
% section main_ideas (end)

\section{Category Theory Basics}
\label{sec:category_theory_basics}
This section contains the basic definitions in category theory that we will be using in our
adjoint model. Our model is based on special kinds of monoidal categories: Lambek categories and
symmetric monoidal closed categories and Lambek categories, as defined in
Definitions~\ref{def:lambek-cat} and~\ref{def:smcc}.

\begin{definition}
\label{def:mc}
  A \textbf{monoidal category} $(\cat{M},\tri,I,\alpha,\lambda,\rho)$ is a category $\cat{M}$
  consists of
  \begin{itemize}
  \item a bifunctor $\tri:\cat{M}\times\cat{M}\rightarrow\cat{M}$, called the tensor product;
  \item an object $I$, called the unit object;
  \item three natural isomorphisms $\alpha$, $\lambda$, and $\rho$ with components
        $$\alpha_{A,B,C}:(A\tri B)\tri C\rightarrow A\tri(B\tri C)$$
        $$\lambda_A:I\tri A\rightarrow A$$
        $$\rho_A:A\tri I\rightarrow A$$
        where $\alpha$ is called associator, $\lambda$ is left unitor, and $\rho$ is right
        unitor,
  \end{itemize}
  such that the following diagrams commute for any objects $A$, $B$, $C$ in $\cat{M}$:
  \begin{mathpar}
  \bfig
    \square/`->`->`->/<2100,400>[
      ((A\tri B)\tri C)\tri D`
      A\tri((B\tri C)\tri D)`
      (A\tri B)\tri(C\tri D)`
      A\tri(B\tri(C\tri D));
      `
      \alpha_{A\tri B,C,D}`
      id_A\tri\alpha_{B,C,D}`
      \alpha_{A,B,C\tri D}]
    \morphism(0,400)<1100,0>[
      ((A\tri B)\tri C)\tri D`
      (A\tri(B\tri C))\tri D;
      \alpha_{A,B,C}\tri id_D]
    \morphism(1100,400)<1000,0>[
      (A\tri(B\tri C))\tri D`
      A\tri((B\tri C)\tri D);
      \alpha_{A,B\tri C,D}]
  \efig
  \and
  \bfig
    \Vtriangle<400,400>[
      (A\tri I)\tri B`
      A\tri(I\tri B)`
      A\tri B;
      \alpha_{A,I,B}`
      \rho_A\tri id_B`
      id_A\tri\lambda_B]
  \efig
  \end{mathpar}
\end{definition}

\begin{definition}
\label{def:lambek-cat}
  A \textbf{Lambek category} (or a \textbf{biclosed monoidal category}) is a monoidal category
  $(\cat{M},\tri,I,\alpha,\lambda,\rho)$ equipped with two bifunctors
  $\rightharpoonup:\cat{M}^{op}\times\cat{M}\rightarrow\cat{M}$ and
  $\leftharpoonup:\cat{M}\times\cat{M}^{op}\rightarrow\cat{M}$ that are both right adjoint to
  the tensor product. That is, the following natural bijections hold:
  \begin{center}
  \begin{math}
  \begin{array}{lll}
    \Hom{L}{X\tri A}{B}\cong\Hom{L}{X}{A\lto B} & \quad\quad\quad\quad & 
    \Hom{L}{A\tri X}{B}\cong\Hom{L}{X}{B\rto A}
  \end{array}
  \end{math}
  \end{center}
\end{definition}

\begin{definition}
  \label{def:smcc}
  A \textbf{symmetric monoidal category} (SMCC) is a monoidal category
  $(\cat{M},\otimes,I,\alpha,\lambda,\rho)$ together with a natural transformation with
  components $\e{A,B}:A\otimes B\rightarrow B\otimes A$, called \textbf{exchange}, such that the
  following diagrams commute:
  \begin{mathpar}
  \bfig
    \Vtriangle<300,400>[A\otimes I`I\otimes A`A;\e{A,I}`\rho_A`\lambda_A]
  \efig
  \and
  \bfig
    \Vtriangle/=`->`<-/<300,400>[
      A\otimes B`A\otimes B`B\otimes A;
      id_{A\otimes B}`\e{A,B}`\e{B,A}]
  \efig
  \and
  \bfig
    \hSquares/->`->`->``->`->`->/<400>[
      (A\otimes B)\otimes C`A\otimes(B\otimes C)`(B\otimes C)\otimes A`
      (B\otimes A)\otimes C`B\otimes(A\otimes C)`B\otimes(C\otimes A);
      \alpha_{A,B,C}`\e{A,B\otimes C}`\e{A,B}\otimes id_C``
      \alpha_{B,A,C}`\alpha_{B,A,C}`id_B\otimes\e{A,C}]
  \efig
  \end{mathpar}
\end{definition}

We use $\tri$ for non-symmetric monoidal categories while $\otimes$ for symmetric ones.

\begin{definition}
  A \textbf{symmetric monoidal closed category} $(\cat{M},\otimes,I,\alpha,\lambda,\rho)$ is a
  symmetric monoidal category equipped with a bifunctor
  $\limp:\cat{M}^{op}\times\cat{M}\rightarrow\cat{M}$ that is right adjoint to the tensor
  product. That is, the following natural bijection
  $\Hom{\cat{M}}{X\otimes A}{B}\cong\Hom{\cat{M}}{X}{A\limp B}$ holds.
\end{definition}

The relation between SMMCs and Lambek categories are demonstrated in
Lemma~\ref{lemma:internal-homs-collapse} and Corollary~\ref{corollary:LC-with-ex-mc}.

\begin{lemma}
  \label{lemma:internal-homs-collapse}
  Let $A$ and $B$ be two objects in a Lambek category with the exchange natural transformation.
  Then $(A \lto B) \cong (B \rto A)$.
\end{lemma}
\begin{proof}
  First, notice that for any object $C$ we have
  \begin{center}
  \begin{math}
  \small
  \begin{array}{lllll}
    Hom[C,A\lto B]
    & \cong & Hom[C\otimes A,B] & \cat{L}\text{ is a Lambek category}\\
    & \cong & Hom[A\otimes C,B] & \text{By the exchange }\e{C,A}\\
    & \cong & Hom[C,B\rto A]    & \cat{L}\text{ is a Lambek category}
  \end{array}
  \end{math}
  \end{center}  
  Thus, $A\lto B\cong B\rto A$ by the Yoneda lemma.
\end{proof}

\begin{corollary}
  \label{corollary:LC-with-ex-mc}
  A Lambek category with exchange is symmetric monoidal closed.
\end{corollary}

The essential component in our non-commutative adjoint model is a monoidal adjunction, defined
in Definitions~\ref{def:monoidal-functor}-\ref{def:monoidal-adjunction}.

\begin{definition}
\label{def:monoidal-functor}
  Let $(\cat{M},\tri,I,\alpha,\lambda,\rho)$ and
  $(\cat{M'},\tri',I',\alpha',\lambda',\rho')$ be monoidal categories. A \textbf{monoidal
  functor} $(F,\m{})$ from $\cat{M}$ to $\cat{M'}$ is a functor $F:\cat{M}\rightarrow\cat{M'}$
  together with a morphism $\m{I}:I'\rightarrow F(I)$ and a natural transformation
  $\m{A,B}:FA'\tri FB'\rightarrow F(A\tri B)$, such that the following diagrams commute
  for any objects $A$, $B$, and $C$ in $\cat{M}$:
  \begin{mathpar}
  \bfig
    \hSquares/->`->`->``->`->`->/<400>[
      (FA\tri'FB)\tri'FC`FA\tri'(FB\tri'FC)`FA\tri'F(B\tri C)`
      F(A\tri B)\tri'FC`F((A\tri B)\tri C)`F(A\tri(B\tri C));
      \alpha'_{FA,FB,FC}`id_{FA}\tri'\m{A,B}`\m{A,B}\tri'id_{FC}``
      \m{A,B\tri C}`\m{A\tri B,C}`F\alpha_{A,B,C}]
  \efig
  \and
  \bfig
    \square/->`->`<-`->/<600,400>[
      I'\tri'FA`FA`FI\tri'FA`F(I\tri A);
      \lambda'_{FA}`\m{I}\tri id_{FA}`F\lambda_A`\m{I,A}]
  \efig
  \and
  \bfig
    \square/->`->`<-`->/<600,400>[
      FA\tri'I'`FA`FA\tri'FI`F(A\tri I);
      \rho'_{FA}`\id_{FA}\tri\m{I}`F\rho_A`\m{A,I}]
  \efig
  \end{mathpar}
\end{definition}

\begin{definition}
  Let $(\cat{M},\otimes,I,\alpha,\lambda,\rho)$ and
  $(\cat{M'},\otimes',I',\alpha',\lambda',\rho')$ be symmetric monoidal categories. A
  \textbf{symmetric monoidal functor} $F:\cat{M}\rightarrow\cat{M'}$ is a monoidal functor
  $(F,\m{})$ that satisfies the following coherence diagram:
  \begin{mathpar}
  \bfig
    \square<700,400>[
      FA\otimes'FB`FB\otimes'FA`F(A\otimes B)`F(B\otimes A);
      \e{FA,FB}`\m{A,B}`\m{B,A}`F\e{A,B}]
  \efig
  \end{mathpar}
\end{definition}

\begin{definition}
  An \textbf{adjunction} between categories $\cat{C}$ and $\cat{D}$ consists of two functors
  $F:\cat{D}\rightarrow\cat{C}$, called the \textbf{left adjoint}, and
  $G:\cat{C}\rightarrow\cat{D}$, called the \textbf{right adjoint}, and two natural
  transformations $\eta:id_\cat{D}\rightarrow GF$, called the \textbf{unit}, and
  $\varepsilon:FG\rightarrow id_\cat{C}$, called the \textbf{counit}, such that the following
  diagrams commute for any object $A$ in $\cat{C}$ and $B$ in $\cat{D}$:
  \begin{mathpar}
  \bfig
    \Vtriangle/->`=`->/<400,400>[FB`FGFB`FB;F\eta_B``\varepsilon_{FB}]
  \efig
  \and
  \bfig
    \Vtriangle/->`=`->/<400,400>[GA`GFGA`GA;\eta_{GA}``G\varepsilon_A]
  \efig
  \end{mathpar}
\end{definition}

\begin{definition}
  Let $(F,\m{})$ and $(G,\n{})$ be monoidal functors from a monoidal category
  $(\cat{M},\otimes,I,\alpha,\lambda,\rho)$ to a monoidal category
  $(\cat{M'},\otimes',I',\alpha',\lambda',\rho')$. A \textbf{monoidal natural transformation}
  from $(F,\m{})$ to $(G,\n{})$ is a natural transformation $\theta:(F,\m{})\rightarrow(G,\n{})$
  such that the following diagrams commute for any objects $A$ and $B$ in $\cat{M}$:
  \begin{mathpar}
  \bfig
    \square<700,400>[
      FA\tri'FB`F(A\tri B)`GA\tri'GB`G(A\tri B);
      \m{A,B}`\theta_A\tri'\theta_B`\theta_{A\tri B}`\n{A,B}]
  \efig
  \and
  \bfig
    \Vtriangle/->`<-`<-/<400,400>[FI`GI`I';\theta_I`\m{I}`\n{I}]
  \efig
  \end{mathpar}
\end{definition}

\begin{definition}
\label{def:monoidal-adjunction}
  Let $(\cat{M},\tri,I,\alpha,\lambda,\rho)$ and
  $(\cat{M'},\tri',I',\alpha',\lambda',\rho')$ be monoidal categories,
  $F:\cat{M}\rightarrow\cat{M'}$ and $G:\cat{M}'\rightarrow\cat{M}$ be functors. The adjunction
  $F:\cat{M}\dashv\cat{M'}:G$ is a \textbf{monoidal adjunction} if $F$ and $G$ are monoidal
  functors, and the unit $\eta$ and the counit $\varepsilon$ are monoidal natural
  transformations.
\end{definition}

In Moggi's monad model \cite{}, the monad is required to be strong, as defined in
Definitions~\ref{def:monad} and~\ref{def:strong-monad}.

\begin{definition}
\label{def:monad}
  Let $\cat{C}$ be a category. A \textbf{monad} on $\cat{C}$ consists of an endofunctor
  $T:\cat{C}\rightarrow\cat{C}$ together with two natural transformations
  $\eta:id_\cat{C}\rightarrow T$ and $\mu:T^2\rightarrow T$, where $id_\cat{C}$ is the identity
  functor on $\cat{C}$, such that the following diagrams commute:
  \begin{mathpar}
  \bfig
    \square<400,400>[T^3`T^2`T^2`T;T\mu`\mu_T`\mu`\mu]
  \efig
  \and
  \bfig
    \square<400,400>[T`T^2`T^2`T;\eta_T`T\eta`\mu`\mu]
    \morphism(0,400)/=/<400,-400>[T`T;]
  \efig
  \end{mathpar}
\end{definition}

\begin{definition}
  \label{def:strong-monad}
  Let $(\cat{M},\tri,I,\alpha,\lambda,\rho)$ be a monoidal category and $(T,\eta,\mu)$ be a
  monad on $\cat{M}$. $T$ is a \textbf{strong monad} if there is natural transformation $\tau$, 
  called the \textbf{tensorial strength}, with components
  $\tau_{A,B}:A\tri TB\rightarrow T(A\tri B)$ such that the following diagrams commute:
  \begin{mathpar}
  \bfig
    \Vtriangle<400,400>[I\tri TA`T(I\tri A)`TA;\tau_{I,A}`\lambda_{TA}`T\lambda_A]
  \efig
  \and
  \bfig
    \Vtriangle<400,400>[
      A\tri B`A\tri TB`T(A\tri B);id_A\tri\eta_B`\eta_{A\tri B}`\tau_{A,B}]
  \efig
  \and
  \bfig
    \square/->`->`->`/<1800,400>[
      (A\tri B)\tri TC`T((A\tri B)\tri C)`
      A\tri(B\tri TC)`T(A\tri(B\tri C));
      \tau_{A\tri B,C}`\alpha_{A,B,TC}`T\alpha_{A,B,C}`]
    \morphism<900,0>[A\tri(B\tri TC)`A\tri T(B\tri C);id_A\tri\tau_{B,C}]
    \morphism(900,0)<900,0>[A\tri T(B\tri C)`T(A\tri(B\tri C));\tau_{A,B\tri C}]
  \efig
  \and
  \bfig
    \square/`->`->`->/<1400,400>[
      A\tri T^2B`T^2(A\tri B)`A\tri TB`T(A\tri B);
      `id_A\tri\mu_B`\mu_{A\tri B}`\tau_{A,B}]
    \morphism(0,400)<700,0>[A\tri T^2B`T(A\tri TB);\tau_{A,TB}]
    \morphism(700,400)<700,0>[T(A\tri TB)`T^2(A\tri B);T\tau_{A,B}]
  \efig
  \end{mathpar}
\end{definition}

\begin{definition}
  Let $(\cat{M},\otimes,I,\alpha,\lambda,\rho)$ be a symmetric monoidal category with exchange
  $\e{}$, and $(T,\eta,\mu)$ be a strong monad on $\cat{M}$. Then there is a \textbf{``twisted''
  tensorial strength} $\tau'_{A,B}:TA\otimes B\rightarrow T(A\otimes B)$ defined as
  $\tau'_{A,B}=T\e{}\circ\tau_{B,A}\circ\e{}$. We can construct a pair of natural
  transformations $\Phi$, $\Phi'$ with components
  $\Phi_{A,B},\Phi'_{A,B}:TA\otimes TB\rightarrow T(A\otimes B)$ defined as
  $\Phi_{A,B}=\mu_{A\otimes B}\circ T\tau'_{A,B}\circ\tau_{TA,B}$ and
  $\Phi'_{A,B}=\mu_{A\otimes B}\circ T\tau_{A,B}\circ\tau'_{A,TB}$. If $\Phi=\Phi'$, then the
  monad $T$ is \textbf{commutative}.
\end{definition}

\begin{definition}
  Let $\cat{L}$ be a category. A \textbf{comonad} on $\cat{L}$ consists of an endofunctor
  $S:\cat{L}\rightarrow\cat{L}$ together with two natural transformations
  $\varepsilon:S\rightarrow id_\cat{L}$ and $\delta:S^2\rightarrow S$ such that the following
  diagrams commute:
  \begin{mathpar}
  \bfig
    \square<400,400>[S`S^2`S^2`S^3;\delta`\delta`S\delta`\delta_S]
  \efig
  \and
  \bfig
    \square<400,400>[S^2`S`S`S^2;S\varepsilon`\varepsilon_S`\delta`\delta]
  \efig
  \end{mathpar}
\end{definition}



\section{Lambek Adjoint Model}
\label{sec:adjoint_model}
Our adjoint model, Lambek Adjoint Model (LAM), has a similar structure as Benton's LNL model
\cite{}. Benton's LNL model consists of a symmetric monoidal adjunction
$F:\cat{C}\dashv\cat{L}:G$ between a cartesian closed category $\cat{C}$ and a symmetric
monoidal closed category $\cat{L}$. LAM consists of a monoidal adjunction between a symmetric
monoidal closed category and a Lambek category.

\begin{definition}
  A \textbf{Lambek Adjoint Model (LAM)} $(\cat{C},\cat{L},F,G,\eta,\varepsilon)$ consists of
  \begin{itemize}
  \item a symmetric monoidal closed category $(\cat{C},\otimes,I,\alpha,\lambda,\rho)$;
  \item a Lambek category $(\cat{L},\tri,I',\alpha',\lambda',\rho')$;
  \item a monoidal adjunction $F:\cat{C}\dashv\cat{L}:G$ with unit $\eta_X:X\rightarrow GFX$ and
        counit $\varepsilon:FG\rightarrow id_\cat{L}$, where $(F:\cat{C}\rightarrow\cat{L}, m)$
        and $(G:\cat{L}\rightarrow\cat{C}, n)$ are monoidal functors.
  \end{itemize}
\end{definition}

Thus, in LAM, the following four diagrams commute because $\eta$ and $\varepsilon$ are monoidal
natural transformations:
\begin{mathpar}
\bfig
  \square/=`->`->`/<1600,400>[
    X\otimes Y`X\otimes Y`GFX\otimes GFY`GF(X\otimes Y);
    id_{X\otimes Y}`\eta_X\otimes\eta_Y`\eta_{X\otimes Y}`]
  \morphism<800,0>[GFX\otimes GFY`G(FX\otimes FY);\n{FX,FY}]
  \morphism(800,0)<800,0>[G(FX\otimes FY)`GF(X\otimes Y);G\m{X,Y}]
\efig
\and
\bfig
  \square/->`=`<-`->/<400,400>[I`GFI`I`GI';\eta_I``G\m{I}`\n{I'}]
\efig
\end{mathpar}
\begin{mathpar}
\bfig
  \square/`->`->`=/<1600,400>[
    FGA\otimes FGB`FG(A\otimes B)`A\otimes B`A\otimes B;
    `
    \varepsilon_A\otimes\varepsilon_B`\varepsilon_{A\otimes B}`]
  \morphism(0,400)<800,0>[FGA\otimes FGB`F(GA\otimes GB);\m{GA,GB}]
  \morphism(800,400)<800,0>[F(GA\otimes GB)`FG(A\otimes B);F\n{A,B}]
\efig
\and
\bfig
  \square/->`<-`=`<-/<400,400>[FGI'`I'`FI`I';\varepsilon_{I'}`F\n{I'}``\m{I}]
\efig
\end{mathpar}
And the following two triangles commute because of the adjunction:
\begin{mathpar}
\bfig
  \Vtriangle/->`=`->/<400,400>[FX`FGFX`FX;F\eta_X``\varepsilon_{FX}]
\efig
\and
\bfig
  \Vtriangle/->`=`->/<400,400>[GA`GFGA`GA;\eta_{GA}``G\varepsilon_A]
\efig
\end{mathpar}

Following the tradition, we use letters $X$, $Y$, $Z$ for objects in $\cat{C}$ and $A$, $B$, $C$
for objects in $\cat{L}$. The rest of this section proves essential properties of a LAM.



\subsection{An Isomorphism}

Let $(\cat{C},\cat{L},F,G,\eta,\varepsilon)$ be a LAM, where $(F,\m{})$ and $(G,\n{})$ are
monoidal functors. Similarly as in Benton's LNL model, $\m{X,Y}$ are components of a natural
isomorphism and $\m{I}$ is an isomorphism. This is essential for deriving certain rules of
our non-commutative linear logic, such as tensor elimination in natural deduction.

We define the inverses of $\m{X,Y}:FX\tri FY\rightarrow F(X\otimes Y)$ and
$\m{I}:I'\rightarrow FI$ as:

\begin{mathpar}
\bfig
  \morphism<1000,0>[\p{X,Y}:F(X\otimes Y)`F(GFX\otimes GFY);F(\eta_X\otimes\eta_Y)]
  \morphism(1000,0)<900,0>[F(GFX\otimes GFY)`FG(FX\tri FY);F\n{FX,FY}]
  \morphism(1900,0)<750,0>[FG(FX\tri FY)`FX\tri FY;\varepsilon_{FX\tri FX}]
\efig
\and
\bfig
  \morphism<500,0>[\p{I}:FI`FGI';F\n{I'}]
  \morphism(500,0)<400,0>[FGI'`I';\varepsilon_{I'}]
\efig
\end{mathpar}

\begin{theorem}
  $\m{X,Y}$ are components of a natural isomorphism and their inverses are $\p{X,Y}$.
\end{theorem}
\begin{proof}
  We need to show that $\m{X,Y}\circ\p{X,Y}=id_{F(X\otimes Y)}$ and
  $\p{X,Y}\circ\m{X,Y}=id_{FX\tri FX}$. The two equations hold because the following diagrams
  commute: (1)-adjunction; (2)-$\eta$ is a monoidal natural transformation; (3)-naturality of
  $\varepsilon$; (4)-adjunction; (5)-naturality of $\m{}$; (6)-$\varepsilon$ is a monoidal
  natural transformation.
  \begin{mathpar}
  \bfig
    \square/->`=`->`/<1600,800>[
      F(X\otimes Y)`F(GFX\otimes GFY)`F(X\otimes Y)`FG(FX\tri FY);
      F(\eta_X\otimes\eta_Y)``F\n{FX,FY}`]
    \Atrianglepair|mmmbb|/->``->`<-`<-/<800,400>[
      FGF(X\otimes Y)`F(X\otimes Y)`FX\tri FY`FG(FX\tri FY);
      \varepsilon_{F(X\otimes Y)}``FG\m{X,Y}`\m{X,Y}`\varepsilon_{FX\tri FY}]
    \morphism(0,800)|m|<800,-400>[F(X\otimes Y)`FGF(X\otimes Y);F\eta_{X\otimes Y}]
    \morphism(300,450)//<0,0>[`;(1)]
    \morphism(1200,600)//<0,0>[`;(2)]
    \morphism(800,200)//<0,0>[`;(3)]
  \efig
  \and
  \bfig
    \square/->`=`->`/<1800,800>[
      FX\tri FY`F(X\otimes Y)`FX\tri FY`F(GFX\otimes GFY);
      \m{X,Y}``F(\eta_X\otimes\eta_Y)`]
    \Atrianglepair|mmmbb|/->``->`<-`<-/<900,400>[
      FGFX\tri FGFY`FX\tri FY`FG(FX\tri FY)`F(GFX\otimes GFY);
      \varepsilon_{FX}\tri\varepsilon_{FY}``\m{GFX,GFY}`\varepsilon_{FX\tri FY}`F\n{FX,FY}]
    \morphism(0,800)|m|<900,-400>[FX\tri FY`FGFX\tri FGFY;F\eta_X\tri F\eta_Y]
    \morphism(300,450)//<0,0>[`;(4)]
    \morphism(1300,600)//<0,0>[`;(5)]
    \morphism(900,200)//<0,0>[`;(6)]
  \efig
  \end{mathpar}
\end{proof}

\begin{theorem}
  $\m{I}$ is an isomorphism and its inverse is $\p{I}$.
\end{theorem}
\begin{proof}
  This is equivalent to equations $\m{I}\circ\p{I}=id_{FI}$ and $\p{I}\circ\m{I}=id_{I'}$,
  equivalent to the following diagrams, which commute because $\varepsilon$ is a monoidal
  natural transformation.
  \begin{mathpar}
  \bfig
    \square/->`=`->`<-/<400,400>[FI`FGI'`FI`I';F\n{I'}``\varepsilon_{I'}`\m{I}]
  \efig
  \and
  \bfig
    \square/->`=`->`<-/<400,400>[I'`FI`I'`FGI';\m{I}``F\n{I'}`\varepsilon_{I'}]
  \efig
  \end{mathpar}
\end{proof}

\subsection{Monad on $\cat{C}$}

We first show that the monad on $\cat{C}$ in LAM is strong but non-commutative. In Benton's
LNL model, the monad on the cartesian closed category is commutative.

\begin{lemma}
\label{lem:monoidal-monad}
  The monad on the symmetric monoidal closed category $\cat{C}$ in LAM is monoidal.
\end{lemma}
\begin{proof}
  Let $(\cat{C},\cat{L},F,G,\eta,\varepsilon)$ be a LAM. We define the monad
  $(T,\eta:id_\cat{C}\rightarrow T,\mu:T^2\rightarrow T)$ on $\cat{C}$ as
  $$T=GF \qquad\qquad\qquad \eta_X:X\rightarrow GFX \qquad\qquad\qquad \mu_X=G\varepsilon_{FX}:GFGFX\rightarrow GFX$$
  Since $(F,\m{})$ and $(G,\n{})$ are monoidal functors, we have
  $$\t{X,Y}=G\m{X,Y}\circ\n{FX,FY}:TX\otimes TY\rightarrow T(X\otimes Y)
  \qquad\qquad\qquad\t{I}=G\m{I}\circ\n{I'}:I\rightarrow TI$$
  The monad $T$ being monoidal means:
  \begin{enumerate}
  \item $T$ is a monoidal functor, i.e. the folllowing diagrams commute:
        \begin{mathpar}
        \bfig
          \hSquares/->`->`->``->`->`->/<400>[
            (TX\otimes TY)\otimes TZ`TX\otimes(TY\otimes TZ)`TX\otimes T(Y\otimes Z)`
            T(X\otimes Y)\otimes TZ`T((X\otimes Y)\otimes Z)`T(X\otimes(Y\otimes Z));
            \alpha_{TX,TY,TZ}`id_{TX}\otimes\t{Y,Z}`\t{X,Y}\otimes id_{TZ}``
            \t{X,Y\otimes Z}`\t{X\otimes Y,Z}`T\alpha_{X,Y,Z}]
          \morphism(1300,200)//<0,0>[`;(1)]
        \efig
        \and
        \bfig
          \square/->`->`<-`->/<600,400>[
            I\otimes TX`TX`TI\otimes TX`T(I\otimes X);
            \lambda_{TX}`\t{I}\otimes id_{TX}`T\lambda_X`\t{I,X}]
          \morphism(350,200)//<0,0>[`;(2)]
        \efig
        \and
        \bfig
          \square/->`->`<-`->/<600,400>[
            TX\otimes I`TX`TX\otimes TI`T(X\otimes I);
            \rho_{TX}`id_{TX}\otimes\t{I}`T\rho_X`\t{X,I}]
          \morphism(350,200)//<0,0>[`;(3)]
        \efig
        \end{mathpar}
        We write $GF$ instead of $T$ in the proof for clarity. \\
        By replacing $\t{X,Y}$ with its definition, diagram (1) above commutes by the following
        commutative diagram, in which the two hexagons commute because $G$ and $F$ are monoidal
        functors, and the two quadrilaterals commute by the naturality of $\n{}$.
        \begin{mathpar}
        \bfig
          \iiixiii/->`->`->``->```->`<-`->``/<1400,400>[
            (GFX\otimes GFY)\otimes GFZ`GFX\otimes(GFY\otimes GFZ)`GFX\otimes G(FY\tri FZ)`
            G(FX\tri FY)\otimes GFZ`G(FX\tri(FY\tri FZ))`GFX\otimes GF(Y\otimes Z)`
            GF(X\otimes Y)\otimes GFZ`G((FX\tri FY)\tri FZ)`G(FX\tri F(Y\otimes Z));
            \alpha_{GFX,GFY,GFZ}`id_{GFX}\otimes\n{FY,FZ}`\n{FX,FY}\otimes id_{GFZ}``
            id_{GFX}\otimes G\m{Y,Z}```G\m{X,Y}\otimes id_{GFZ}`G\alpha'_{FX,FY,FZ}`
            \n{FX,F(Y\otimes Z)}``]
          \morphism(2800,800)|m|<-1400,-400>[
            GFX\otimes G(FY\tri FZ)`G(FX\tri(FY\tri FZ));\n{FX,FY\tri FZ}]
          \morphism(0,400)|m|<1400,-400>[
            G(FX\tri FY)\otimes GFZ`G((FX\tri FY)\tri FZ);\n{FX\tri FY,FZ}]
          \morphism(1400,400)|m|<1400,-400>[
            G(FX\tri(FY\tri FZ))`G(FX\tri F(Y\otimes Z));G(id_{FX}\tri\m{Y,Z})]
          \ptriangle(0,-400)|mlm|/`->`->/<1400,400>[
            GF(X\otimes Y)\otimes GFZ`G((FX\tri FY)\tri FZ)`G(F(X\otimes Y)\tri FZ);
            `\n{F(X\otimes Y),FZ}`G(\m{X,Y}\otimes id_{FZ})]
          \morphism(0,-400)|b|<1400,0>[
            G(F(X\otimes Y)\tri FZ)`GF((X\otimes Y)\otimes Z);G\m{X\otimes Y,Z}]
          \dtriangle(1400,-400)|mrb|/`->`->/<1400,400>[
            G(FX\tri F(Y\otimes Z))`GF((X\otimes Y)\otimes Z)`GF(X\otimes(Y\otimes Z));
            `G\m{X,Y\otimes Z}`GF\alpha_{X,Y,Z}]
        \efig
        \end{mathpar}
        Diagram (2) commutes by the following commutative diagrams, in which the top
        quadrilateral commutes because $G$ is monoidal, the right quadrilateral commutes because
        $F$ is monoidal, and the left square commutes by the naturality of $\n{}$.
        \begin{mathpar}
        \bfig
          \ptriangle/->`->`/<1600,400>[
            I\otimes GFX`GFX`GI'\otimes GFX;\lambda_{GFX}`\n{I'}\otimes id_{GFX}`]
          \square(0,-400)|lmmb|<800,400>[
            GI'\otimes GFX`G(I'\tri FX)`GFI\otimes GFX`G(FI\tri FX);
            \n{I',FX}`G\m{I}\otimes id_{GFX}`G(\m{I}\tri id_{FX})`\n{FI,FX}]
          \morphism(800,0)|m|<800,400>[G(I'\tri FX)`GFX;G\lambda'_{FX}]
          \dtriangle(800,-400)/`<-`->/<800,800>[
            GFX`G(FI\tri FX)`GF(I\otimes X);
            `GF\lambda_X`G\m{I,X}]
        \efig
        \end{mathpar}
        Similarly, diagram (3) commutes as follows:
        \begin{mathpar}
        \bfig
          \ptriangle/->`->`/<1600,400>[
            GFX\otimes I`GFX`GFX\otimes GI';\rho_{GFX}`id_{GFX}\otimes\n{I'}`]
          \square(0,-400)|lmmb|<800,400>[
            GFX\otimes GI'`G(FX\tri I')`GFX\otimes GFI`G(FX\tri FI);
            \n{FX,I'}`id_{GFX}\otimes G\m{I}`G(id_{FX}\otimes\m{I})`\n{FX,FI}]
          \morphism(800,0)|m|<800,400>[G(FX\tri I')`GFX;G\rho'_{FX}]
          \dtriangle(800,-400)/`<-`->/<800,800>[
            GFX`G(FX\tri FI)`GF(X\otimes I);
            `GF\rho_X`G\m{X,I}]
        \efig
        \end{mathpar}
  \item $\eta$ is a monoidal natural transformation. In fact, since $\eta$ is the unit of the
        monoidal adjunction, $\eta$ is monoidal by definition and thus the following two
        diagrams commute.
        \begin{mathpar}
        \bfig
          \square/=`->`->`->/<600,400>[
            X\otimes Y`X\otimes Y`TX\otimes TY`T(X\otimes Y);
            `\eta_X\otimes\eta_Y`\eta_{X\otimes Y}`\t{X,Y}]
        \efig
        \and
        \bfig
          \Vtriangle/->`=`<-/<400,400>[I`TI`I;\eta_I``\t{I}]
        \efig
        \end{mathpar}
  \item $\mu$ is a monoidal natural transformation. It is obvious that since
        $\mu=G\varepsilon_{FA}$ and $\varepsilon$ is monoidal, so is $\mu$. Thus the following
        diagrams commute.
        \begin{mathpar}
        \bfig
          \square/`->`->`->/<1500,400>[
            T^2X\otimes T^2Y`T^2(X\otimes Y)`TX\otimes TY`T(X\otimes Y);
            `\mu_X\otimes\mu_Y`\mu_{X\otimes Y}`\t{X,Y}]
          \morphism(0,400)<800,0>[T^2X\otimes T^2Y`T(TX\otimes TY);\t{TX,TY}]
          \morphism(800,400)<700,0>[T(TX\otimes TY)`T^2(X\otimes Y);T\t{X,Y}]
        \efig
        \and
        \bfig
          \square/->`<-`<-`<-/<400,400>[T^2I`TI`TI`I;\mu_I`T\t{I}`\t{I}`\t{I}]
        \efig
        \end{mathpar}
  \end{enumerate}
\end{proof}

However, the monad is not symmetric becauase the following diagram does not commute, for the
Lambek category $\cat{L}$ is not symmetric.
\begin{mathpar}
\bfig
  \ptriangle/->`->`/<900,400>[
    GFX\otimes GFY`GFY\otimes GFX`G(FX\tri FY);\e{GFX,GFY}`\n{FX,FY}`]
  \morphism(900,400)<900,0>[GFY\otimes GFX`G(FY\tri FX);\n{FY,FX}]
  \dtriangle(900,0)/`->`->/<900,400>[
    G(FY\tri FX)`GF(X\otimes Y)`GF(Y\otimes X);`G\m{Y,X}`GF\e{X,Y}]
  \morphism|b|<900,0>[G(FX\tri FY)`GF(X\otimes Y);G\m{X,Y}]
\efig
\end{mathpar}

\begin{lemma}
  \label{lem:strong-monad}
  The monad on the symmetric monoidal closed category in LAM is strong.
\end{lemma}
\begin{proof}
  Let $(\cat{C},\cat{L},F,G,\eta,\varepsilon)$ be a LAM, where
  $(\cat{C},\otimes,I,\alpha,\lambda,\rho)$ is symmetric monoidal closed, \\
  $(\cat{L},\tri,I',\alpha',\lambda',\rho')$ is Lambek. In Lemma~\ref{lem:monoidal-monad}, we
  have proved that the monad $(T=GF,\eta,\mu)$ is monoidal with the natural transformation
  $\t{X,Y}:TX\otimes TY\rightarrow T(X\otimes Y)$ and the morphism $\t{I}:I\rightarrow TI$. \\
  We define the tensorial strength $\tau_{X,Y}:X\otimes TY\rightarrow T(X\otimes Y)$ as
  $\tau_{X,Y}=\t{X,Y}\circ(\eta_X\otimes id_{TY})$. \\
  Since $\eta$ is a monoidal natural transformation, we have $\eta_I=G\m{I}\circ\n{I'}$.
  Therefore $\eta_I=\t{I}$. Thus the following diagram commutes because $T$ is monoidal,
  where the composition $\t{I,X}\circ(\t{I}\otimes id_{TX})$ is the definition of $\tau_{I,X}$.
  So the first triangle in Defition~\ref{def:strong-monad} commutes.
  \begin{mathpar}
  \bfig
    \square/->`->`->`<-/<600,400>[
      I\otimes TX`TI\otimes TX`TX`T(I\otimes X);
      \t{I}\otimes id_{TX}`\lambda_{TX}`\t{I,X}`T\lambda_X]
  \efig
  \end{mathpar}
  Similarly, by using the definition of $\tau$, the the second triangle in the definition is
  equivalent to the following diagram, which commutes because $\eta$ is a monoidal natural
  transformation:
  \begin{mathpar}
  \bfig
    \square/->`->`->`<-/<600,400>[
      X\otimes Y`X\otimes TY`T(X\otimes Y)`TX\otimes TY;
      id_X\otimes\eta_Y`\eta_{X\otimes Y}`\eta_X\otimes id_{TY}`\t{X,Y}]
    \morphism(0,400)|m|<600,-400>[X\otimes Y`TX\otimes TY;\eta_X\otimes\eta_Y]
  \efig
  \end{mathpar}
  The first pentagon in the definition commutes by the following commutative diagrams, because
  $\eta$ and $\alpha$ are natural transformations and $T$ is monoidal:
  \begin{mathpar}
  \bfig
    \qtriangle|amm|/->`->`<-/<1000,400>[
      (X\otimes Y)\otimes TZ`T(X\otimes Y)\otimes TZ`(TX\otimes TY)\otimes TZ;
      \eta_{X\otimes Y}\otimes id_{TZ}`
      (\eta_X\otimes\eta_Y)\otimes id_{TZ}`
      \t{X,Y}\otimes id_{TZ}]
    \morphism(0,400)<0,-400>[(X\otimes Y)\otimes TZ`X\otimes(Y\otimes TZ);\alpha_{X,Y,TZ}]
    \morphism(1000,0)|m|<0,-400>[
      (TX\otimes TY)\otimes TZ`TX\otimes(TY\otimes TZ);\alpha_{TX,TY,TZ}]
    \Dtriangle(0,-800)|lmm|/->`->`<-/<1000,400>[
      X\otimes(Y\otimes TZ)`TX\otimes(TY\otimes TZ)`X\otimes(TY\otimes TZ);
      id_X\otimes(\eta_Y\otimes id_{TZ})`
      \eta_X\otimes(\eta_Y\otimes id_{TZ})`
      \eta_X\otimes id_{TY\otimes TZ}]
    \morphism(0,-800)|b|<1000,0>[
      X\otimes(TY\otimes TZ)`X\otimes T(Y\otimes Z);id_X\otimes\t{Y,Z}]
    \qtriangle(1000,0)|amr|/->``->/<1000,400>[
      T(X\otimes Y)\otimes TZ`T((X\otimes Y)\otimes Z)`T(X\otimes(Y\otimes Z));
      \t{X\otimes Y,Z}``T\alpha_{X,Y,Z}]
    \morphism(2000,-800)<0,800>[
      TX\otimes T(Y\otimes Z)`T(X\otimes(Y\otimes Z));\t{X,Y\otimes Z}]
    \btriangle(1000,-800)|mmb|/`->`->/<1000,400>[
      TX\otimes(TY\otimes TZ)`X\otimes T(Y\otimes Z)`TX\otimes T(Y\otimes Z);
      `id_{TX}\otimes\t{Y,Z}`\eta_X\otimes id_{T(Y\otimes Z)}]
  \efig
  \end{mathpar}
  The last diagram in the definition commtues by the following commutative diagram, because
  $T$ is a monad, $\t{}$ is a natural transformation, and $\mu$ is a monoidal natural
  transformation:
  \begin{mathpar}
  \bfig
    \ptriangle/->`->`/<700,400>[
      X\otimes T^2Y`TX\otimes T^2Y`X\otimes TY;\eta_X\otimes id_{T^2Y}`id_X\otimes\mu_Y`]
    \btriangle(0,-400)/->``->/<700,400>[
      X\otimes TY`TX\otimes TY`T(X\otimes Y);\eta_X\otimes id_{TY}``\t{X,Y}]
    \morphism(700,400)|m|<-700,-800>[TX\otimes T^2Y`TX\otimes TY;id_{TX}\otimes\mu_Y]
    \morphism(700,0)|m|<-700,-400>[TX\otimes T^2Y`TX\otimes TY;id_{TX}\otimes\mu_Y]
    \qtriangle(700,0)/->``->/<1800,400>[
      TX\otimes T^2Y`T(X\otimes TY)`T(TX\otimes TY);\t{X,TY}``T(\eta_X\otimes id_{TY})]
    \btriangle(700,0)|mmm|/=`->`<-/<900,400>[
      TX\otimes T^2Y`TX\otimes T^2Y`T^2X\otimes T^2Y;
      `T\eta_X\otimes id_{T^2Y}`\mu_X\otimes id_{T^2Y}]
    \morphism(1600,0)|m|<900,0>[T^2X\otimes T^2Y`T(TX\otimes TY);\t{TX,TY}]
    \morphism(1600,0)|m|<-1600,-400>[T^2X\otimes T^2Y`TX\otimes TY;\mu_X\otimes\mu_Y]
    \dtriangle(700,-400)/`->`<-/<1800,400>[
      T(TX\otimes TY)`T(X\otimes Y)`T^2(X\otimes Y);`T\t{X,Y}`\mu_{X\otimes Y}]
  \efig
  \end{mathpar}
\end{proof}

The following lemma is adopted from \cite{}.
\begin{lemma}
\label{lem:monad-com-iff-sym}
  Let $\cat{M}$ be a symmetric monoidal category and $T$ be a strong monad on $\cat{M}$. Then
  $T$ is commutative iff it is symmetric.
\end{lemma}

\begin{theorem}
  The monad on the SMCC in LAM is strong but non-commutative.
\end{theorem}
\begin{proof}
  The proof is obvious. Based on Lemma~\ref{lem:strong-monad} and
  Lemma~\ref{lem:monad-com-iff-sym}, the monad is non-commutative. 
\end{proof}

\subsection{Comonad on $\cat{L}$}

\begin{lemma}
  The comonad on the Lambek category in a LAM is monoidal.
\end{lemma}
\begin{proof}
  We define the comonad $(S,\varepsilon:S\rightarrow id_\cat{L},\delta:S\rightarrow S^2)$ on
  the Lambek category $\cat{L}$ as:
  $$S=FG \qquad\qquad\qquad \varepsilon_A:SA\rightarrow A \qquad\qquad\qquad \delta_A=F\eta_{GA}:SA\rightarrow S^2A$$
  Thus, we have natural transformation $\s{}$ and morphism $\s{I}$ defined as:
  $$\s{A,B}=F\n{A,B}\circ\m{GA,GB}:SA\tri SB\rightarrow SA\tri SB \qquad\qquad\qquad
  \s{I}=F\n{I'}\circ\m{I}:I'\rightarrow SI'$$
  The comonad $S$ being monoidal means
  \begin{enumerate}
  \item $S$ is a monoidal functor, i.e. the following diagrams commute:
        \begin{mathpar}
        \bfig
          \hSquares/->`->`->``->`->`->/<400>[
            (SA\tri SB)\tri SC`SA\tri(SB\tri SC)`SA\tri S(B\tri C)`
            S(A\tri B)\tri SC`S((A\tri B)\tri C)`S(A\tri(B\tri C));
            \alpha_{SA,SB,SC}'`id_{SA}\tri\s{B,C}`\s{A,B}\tri id_{SC}``
            \s{A,B\tri C}`\s{A\tri B,C}`S\alpha_{A,B,C}']
        \efig
        \and
        \bfig
          \square/->`->`<-`->/<600,400>[
            I'\tri SA`SA`SI'\tri SA`S(I'\tri A);
            \lambda_{SA}'`\s{I'}\tri id_{SA}`S\lambda_A'`\s{I',A}]
        \efig
        \and
        \bfig
          \square/->`->`<-`->/<600,400>[
            SA\tri I'`SA`SA\tri SI'`S(A\tri I');
            \rho_{SA}'`id_{SA}'\tri\s{I'}`S\rho_A'`\s{A,I'}]
        \efig
        \end{mathpar}
  \item $\varepsilon$ is a monoidal natural transformation:
        \begin{mathpar}
        \bfig
          \square/->`->`->`=/<600,400>[
            SA\tri SB`S(A\tri B)`A\tri B`A\tri B;
            \s{A,B}`\varepsilon_A\tri\varepsilon_B`\varepsilon_{A\tri B}`]
        \efig
        \and
        \bfig
          \Vtriangle/->`<-`=/<400,400>[SI'`I'`I';\varepsilon_{I'}`\s{I'}`]
        \efig
        \end{mathpar}
  \item $\delta$ is a monoidal natural transformation:
        \begin{mathpar}
        \bfig
          \square/->`->`->`/<1500,400>[
            SA\tri SA`S(A\tri B)`S^2A\tri S^2B`S^2(A\tri B);
            \s{A,B}`\delta_A\tri\delta_B`\delta_{A\tri B}`]
          \morphism<800,0>[S^2A\tri S^2B`S(SA\tri SB);\s{SA,SB}]
          \morphism(800,0)<700,0>[S(SA\tri SB)`S^2(A\tri B);S\s{A,B}]
        \efig
        \and
        \bfig
          \square/->`<-`<-`->/<400,400>[
            SI'`S^2I'`I'`SI';\delta_{I'}`\s{I'}`S\s{I'}`\s{I'}]
        \efig
        \end{mathpar}
  \end{enumerate}
  The proof for the commutativity of the diagrams are similar as the proof in
  Lemma~\ref{lem:monoidal-monad}. We do not include the proof here for simplicity.
\end{proof}

We then show that the co-Eilenberg-Moore category of the comonad $S$ is
symmetric monoidal.

\begin{definition}
  Let $(S,\varepsilon,\delta)$ be a comonad on a category $\cat{L}$. Then the
  \textbf{co-Eilenberg-Moore category $\cat{L}^S$} of the comonad has
  \begin{itemize}
  \item as objects the S-coalgebras $(A,h_A:A\rightarrow SA)$, where $A$ is an object in
        $\cat{L}$, s.t. the following diagrams commute:
        \begin{mathpar}
        \bfig
          \square<400,400>[A`SA`SA`S^2A;h_A`h_A`\delta_A`Sh_A]
        \efig
        \and
        \bfig
          \Atriangle/<-`->`=/<200,400>[SA`A`A;h_A`\varepsilon_A`]
        \efig
        \end{mathpar}
  \item as morphisms the coalgebra morphisms, i.e. morphisms $f:(A,h_A)\rightarrow(B,h_B)$
        between coalgebras s.t. the diagram commutes:
        $$\bfig
          \square<400,400>[A`B`SA`SB;f`h_A`h_B`Sf]
        \efig$$
  \end{itemize}
\end{definition}

\begin{lemma}
  \label{lem:em-exchange}
  Given a LAM $(\cat{C},\cat{L},F,G,\eta,\varepsilon)$ and the comonad $S$ on $\cat{L}$,
  the co-Eilenberg-Moore category $\cat{L}^S$ has an exchange natural transformation
  $\e{A,B}^S:A\tri B\rightarrow B\tri A$.
\end{lemma}
\begin{proof}
  We define the exchange $\e{A,B}^S:A\tri B\rightarrow B\tri A$ as
  $$\bfig
    \morphism<600,0>[A\tri B`FGA\tri FGB;h_A\tri h_B]
    \morphism(600,0)<800,0>[FGA\tri FGB`F(GA\otimes GB);\m{GA,GB}]
    \morphism(1400,0)<800,0>[F(GA\otimes GB)`F(GB\otimes GA);F\e{GA,GB}]
    \morphism(2200,0)<700,0>[F(GB\otimes GA)`FG(B\tri A);F\n{B,A}]
    \morphism(2900,0)<500,0>[FG(B\tri A)`B\tri A;\varepsilon_{B\tri A}]
  \efig$$
  in which $(F,\m{})$ and $(G,\n{})$ are monoidal functors, and $\e{}$ is the exchange for
  $\cat{C}$. Then $\e{}^S$ is a natural transformation because the following diagrams commute
  for morphisms $f:A\rightarrow A'$ and $g:B\rightarrow B'$:
  \begin{mathpar}
  \bfig
    \square|almb|<700,400>[
      A\tri B`FGA\tri FGB`A'\tri B'`FGA'\tri FGB';
      h_A\tri h_B`f\tri g`FGf\tri FGg`h_{A'}\tri h_{B'}]
    \square(700,0)|ammb|/->``->`->/<800,400>[
      FGA\tri FGB`F(GA\otimes GB)`FGA'\tri FGB'`F(GA'\otimes GB');
      \m{GA,GB}``F(Gf\otimes Gg)`\m{GA',GB'}]
    \square(1500,0)|ammb|/->``->`->/<800,400>[
      F(GA\otimes GB)`F(GB\otimes GA)`F(GA'\otimes GB')`F(GB'\otimes GA');
      F\e{A,B}``F(Gg\otimes Gf)`F\e{A',B'}]
    \square(2300,0)|ammb|/->``->`->/<800,400>[
      F(GB\otimes GA)`FG(B\tri A)`F(GB'\otimes GA')`FG(B'\tri A');
      F\n{B,A}``FG(g\tri f)`F\n{B',A'}]
    \square(3100,0)|amrb|/->``->`->/<600,400>[
      FG(B\tri A)`B\tri A`FG(B'\tri A')`B'\tri A';
      \varepsilon_{B\tri A}``g\tri f`\varepsilon_{B'\tri A'}]
  \efig
  \end{mathpar}
\end{proof}

\begin{lemma}
  \label{lem:em-ex}
  The following diagrams commute in the co-Eilenberg-Moore category $\cat{L}^S$:
  \begin{mathpar}
  \bfig
    \iiixiii/->`->`->``->```->``->`->`->/<1300,400>[
      F((GA\otimes GB)\otimes GC)`F(G(A\tri B)\otimes GC)`FG((A\tri B)\tri C)`
      F(G(B\tri A)\otimes GC)``(A\tri B)\tri C`
      F(G(B\tri A)\otimes GC)`FG((B\tri A)\tri C)`(B\tri A)\tri C;
      F(\n{A,B}\otimes id_{GC})`F(\e{A,B}\otimes id_{GC})`
      F(\e{A,B}\otimes id_{GC})``\varepsilon_{(A\tri B)\tri C}```
      F(\n{B,A}\otimes id_{GC})``\e{A,B}^S\tri id_C`
      F\n{B\tri A,C}`\varepsilon_{(B\tri A)\tri C}]
  \efig
  \end{mathpar}
  \begin{mathpar}
  \bfig
    \iiixiii/->`->`->``->```->``->`->`->/<1300,400>[
      F(GB\otimes(GC\otimes GA))`F(GB\otimes G(C\tri A))`FG(B\tri(C\tri A))`
      F(GB\otimes(GA\otimes GC))``B\tri(C\tri A)`
      F(GB\otimes G(A\tri C))`FG(B\tri(A\tri C))`B\tri(A\tri C);
      F(id_{GB}\otimes\n{C,A})`F\n{B,C\tri A}`
      F(id_{GB}\otimes\e{C,A})``\varepsilon_{B\tri(C\tri A)}```
      F(id_{GB}\otimes\n{A,C})``id_A\tri\e{C,A}^S`
      F\n{B,A\tri C}`\varepsilon_{B\tri(A\tri C)}]
  \efig
  \end{mathpar}
\end{lemma}
\begin{proof}
  We only write the proof for the first diagram. The proof for the second one is similar.
  (1), (2), (3)--naturality of $\m{}$; (4)--F is monoidal; (5), (12)--$\varepsilon$ is monoidal;
  (6), (7), (8), (9), (10)--obvious; (11)--coalgebra.
  \begin{mathpar}
  \bfig
    \Vtrianglepair|aammm|/->`->`<-``<-/<1100,400>[
      F(G(A\tri B)\otimes GC)`FG((A\tri B)\tri C)`(A\tri B)\tri C`FG(A\tri B)\tri FGC;
      F\n{A\tri B,C}`\varepsilon_{(A\tri B)\tri C}`
      \m{G(A\tri B),GC}``\varepsilon_{A\tri B}\tri\varepsilon_C]
    \qtriangle(2200,0)|amr|/=`<-`->/<1000,400>[
      (A\tri B)\tri C`(A\tri B)\tri C`(FGA\tri FGB)\tri C;
      `(\varepsilon_A\tri\varepsilon_B)\tri id_C`(h_A\tri h_B)\tri id_C]
    \morphism|l|<0,400>[
      F((GA\otimes GB)\otimes GC)`F(G(A\tri B)\otimes GC);F(\n{A,B}\otimes id_{GC})]
    \dtriangle(1100,0)|mmm|/`<-`->/<1100,400>[
      (A\tri B)\tri C`FG(A\tri B)\tri FGC`FG(A\tri B)\tri C;
      `\varepsilon_{A\tri B}\tri id_C`id\tri\varepsilon_C]
    \btriangle(0,-400)|lmm|/->`<-`/<1100,400>[
      F((GA\otimes GB)\otimes GC)`F((GB\otimes GA)\otimes GC)`F(GA\otimes GB)\tri FGC;
      F(\e{A,B}\otimes id_{GC})`\m{GA\otimes GB,GC}`]
    \btriangle(1100,-400)|mmm|/<-``->/<2100,400>[
      FG(A\tri B)\tri FGC`F(GA\otimes GB)\tri FGC`F(GA\otimes GB)\tri C;
      F\n{A,B}\tri id_{FGC}``id_{F(GA\otimes GB)}\tri\varepsilon_C]
    \qtriangle(2200,-400)|mmr|/`<-`->/<1000,400>[
      FG(A\tri B)\tri C`(FGA\tri FGB)\tri C`F(GA\otimes GB)\tri C;
      `F\n{A,B}\otimes id_C`\m{GA,GB}\tri id_C]
    \btriangle(0,-800)|lmm|/->`<-`/<2200,400>[
      F((GB\otimes GA)\otimes GC)`F(G(B\tri A)\otimes GC)`F(GB\otimes GA)\tri FGC;
      F(\n{B,A}\otimes id_{GC})`\m{GB\otimes GA,GC}`]
    \morphism(1100,-400)|m|<1100,-400>[
      F(GA\otimes GB)\tri FGC`F(GB\otimes GA)\tri FGC;F\e{A,B}\tri id_{FGC}]
    \dtriangle(2200,-800)|mra|/`->`->/<1000,400>[
      F(GA\otimes GB)\tri C`F(GB\otimes GA)\tri FGC`F(GB\otimes GA)\tri C;
      `F\e{A,B}\tri id_C`id\tri\varepsilon_C]
    \square(0,-1300)|almb|/<-`->`->`->/<1100,500>[
      F(G(B\tri A)\otimes GC)`FG(B\tri A)\tri FGC`FG((B\tri A)\tri C)`(B\tri A)\tri C;
      \m{G(B\tri A),GC}`F\n{B\tri A,C}`
      \varepsilon_{B\tri A}\tri\varepsilon_C`\varepsilon_{(B\tri A)\tri C}]
    \morphism(1100,-800)|b|/<-/<1100,0>[
      FG(B\tri A)\tri FGC`F(GB\otimes GA)\tri FGC;F\n{B,A}\tri id]
    \dtriangle(1100,-1300)/`->`<-/<2100,500>[
      F(GB\otimes GA)\tri C`(B\tri A)\tri C`FG(B\tri A)\tri C;
      `F\n{B,A}\tri id_C`\varepsilon_{B\tri A}\tri id_C]
    \morphism(1100,-800)|m|<2100,-500>[
      FG(B\tri A)\tri FGC`FG(B\tri A)\tri C;id_{FG(B\tri A)}\tri\varepsilon_C]
    \morphism(550,50)//<0,0>[`;(1)]
    \morphism(550,-300)//<0,0>[`;(2)]
    \morphism(400,-550)//<0,0>[`;(3)]
    \morphism(550,-1000)//<0,0>[`;(4)]
    \morphism(1100,300)//<0,0>[`;(5)]
    \morphism(1900,200)//<0,0>[`;(6)]
    \morphism(1800,-150)//<0,0>[`;(7)]
    \morphism(2400,-550)//<0,0>[`;(8)]
    \morphism(1550,-1050)//<0,0>[`;(9)]
    \morphism(2700,-950)//<0,0>[`;(10)]
    \morphism(3000,300)//<0,0>[`;(11)]
    \morphism(2700,50)//<0,0>[`;(12)]
  \efig
  \end{mathpar}
\end{proof}

\begin{theorem}
  The co-Eilenberg-Moore category $\cat{L}^S$ of $S$ is symmetric monoidal closed.
\end{theorem}
\begin{proof}
  Let $(\cat{L},\tri,I',\alpha',\lambda',\rho')$ be the Lambek cateogry in a SMCC-Lambek model
  and $S$ be the comonad on $\cat{L}$. Since $\cat{L}$ is a Lambek category, it is obvious that
  $\cat{L}$ is also Lambek. By Corollary~\ref{corollary:LC-with-ex-mc}, we only need to prove
  the exchange defined in Lemma~\ref{lem:em-exchange} satisfies the three commutative diagrams
  in Definition~\ref{def:smcc}.

  The first triangle in Definition~\ref{def:smcc} commutes as follows:
  (1)--coalgebra; (2)--$\varepsilon$ is monoidal; (3)--naturality of $\rho$; (4)--naturality of
  $\varepsilon$; (5)--naturality of $\m{}$; (6)--$F$ is monoidal; (7)--$\cat{C}$ is symmetric;
  (8)--naturality of $\e{}$; (9)--$G$ is monoidal.
  \begin{mathpar}
  \bfig
    \Vtriangle/->`=`/<800,800>[A\tri I'`FGA\tri FGI'`A\tri I';h_A\tri h_{I'}``]
    \Ctrianglepair(1600,0)|mmmmm|/->`<-``=`<-/<800,400>[
      FGA\tri FGI'`FGA\tri I'`FGA\tri FI`FGA\tri I';
      id_{FGA}\tri\varepsilon_{I'}`id_{FGA}\tri F\n{I'}```id_{FGA}\tri\m{I}]
    \morphism(800,400)|m|<0,-400>[FGA\tri I'`A\tri I';\varepsilon_{FGA}\tri id_{I'}]
    \morphism(1600,400)|m|<800,0>[FGA\tri FI`F(GA\otimes I);\m{GA,I}]
    \morphism(1600,0)|m|<0,-400>[FGA\tri I'`FGA;\rho_{FGA}']
    \dtriangle(1600,-400)|mmm|/->`->`<-/<800,800>[
      F(GA\otimes I)`FGA`F(I\otimes GA);
      F\rho_{GA}`F\e{GA,I}`F\lambda_{GA}]
    \square(1600,-800)/->``->`<-/<1600,1600>[
      FGA\tri FGI'`F(GA\otimes GI')`FG(I'\tri A)`F(GI'\otimes GA);
      \m{GA,GI'}``F\e{GA,GI'}`F\n{I',A}]
    \morphism(2400,400)|m|<800,400>[F(GA\otimes I)`F(GA\otimes GI');F(id_{GA}\otimes\n{I'})]
    \morphism(2400,-400)|m|<800,-400>[F(I\otimes GA)`F(GI'\otimes GA);F(\n{I'}\otimes id_{GA})]
    \morphism(0,800)|l|<0,-1200>[A\tri I'`A;\rho_A']
    \square(0,-800)|mlmb|/<-`->`<-`<-/<1600,400>[
      A`FGA`I'\tri A`FG(I'\tri A);\varepsilon_A`\lambda_A'`FG\lambda_A'`\varepsilon_{I'\tri A}]
    \morphism(800,600)//<0,0>[`;(1)]
    \morphism(1300,400)//<0,0>[`;(2)]
    \morphism(800,-200)//<0,0>[`;(3)]
    \morphism(800,-600)//<0,0>[`;(4)]
    \morphism(2150,600)//<0,0>[`;(5)]
    \morphism(1950,200)//<0,0>[`;(6)]
    \morphism(2200,-200)//<0,0>[`;(7)]
    \morphism(2800,0)//<0,0>[`;(8)]
    \morphism(2200,-600)//<0,0>[`;(9)]
  \efig
  \end{mathpar}
  The second triangle in the proof commutes as follows: (1) and (5)--coalgebra; (2) and
  (4)--$\varepsilon$ is monoidal; (3)--$\cat{C}$ is symmetric.
  \begin{mathpar}
  \bfig
    \ptriangle|amm|/->`=`->/<600,600>[
      A\tri B`FGA\tri FGB`A\tri B;h_A\tri h_B``\varepsilon_A\tri\varepsilon_B]
    \morphism(600,0)|b|<-600,0>[FG(A\tri B)`A\tri B;\varepsilon_{A\tri B}]
    \morphism(600,600)<700,0>[FGA\tri FGB`F(GA\otimes GB);\m{GA,GB}]
    \morphism(1300,0)|b|<-700,0>[F(GA\otimes GB)`FG(A\tri B);F\n{A,B}]
    \square(1300,0)/->`=`=`<-/<800,600>[
      F(GA\otimes GB)`F(GB\otimes GA)`F(GA\otimes GB)`F(GB\otimes GA);F\e{A,B}```F\e{B,A}]
    \qtriangle(2100,0)|amr|/->``->/<1400,600>[F(GB\otimes GA)`FG(B\tri A)`B\tri A;F\n{B,A}``\varepsilon_{B\tri A}]
    \morphism(2900,0)|b|<-800,0>[FGB\tri FGA`F(GB\otimes GA);\m{GB,GA}]
    \btriangle(2900,0)|mmb|/<-`=`<-/<600,400>[
      B\tri A`FGB\tri FGA`B\tri A;
      \varepsilon_B\tri\varepsilon_A``h_A\tri h_A])
    \morphism(200,500)//<0,0>[`;(1)]
    \morphism(800,300)//<0,0>[`;(2)]
    \morphism(1700,300)//<0,0>[`;(3)]
    \morphism(2500,300)//<0,0>[`;(4)]
    \morphism(3100,200)//<0,0>[`;(5)]
  \efig
  \end{mathpar}
  The third diagram commutes as follows: (1) and (7)--Lemma~\ref{lem:em-ex}; (2)--naturality of
  $\alpha'$; (3) and (8)--naturality of $\varepsilon$; (4), (6) and (12)--G is a monoidal
  functor; (5)--$\cat{C}$ is symmetrical monoidal closed; (9)--coalgebra; (10)--$\varepsilon$
  is a monoidal natural transformation; (11)--naturality of $\e{}$.
  \begin{mathpar}
  \bfig
    \morphism(0,400)<1600,0>[(A\tri B)\tri C`A\tri(B\tri C);\alpha_{A,B,C}']
    \Vtriangle(1600,0)|amm|/->`=`->/<800,400>[
      A\tri(B\tri C)`FGA\tri FG(B\tri C)`A\tri(B\tri C);
      h_A\tri h_{B\tri C}``\varepsilon_A\tri\varepsilon_{B\tri C}]
    \morphism(800,0)|m|<-800,400>[
      FG((A\tri B)\tri C)`(A\tri B)\tri C;\varepsilon_{(A\tri B)\tri C}]
    \qtriangle(800,-400)|mmm|/`->`<-/<1600,400>[
      FG((A\tri B)\tri C)`A\tri(B\tri C)`FG(A\tri(B\tri C));
      `FG\alpha_{A,B,C}'`\varepsilon_{A\tri(B\tri C)}]
    \Ctriangle(2400,-1200)|mrm|/`->`<-/<800,800>[
      FGA\tri FG(B\tri C)`FG(A\tri(B\tri C))`F(GA\otimes G(B\tri C));
      `\m{GA,G(B\tri C)}`F\n{A,B\tri C}]
    \morphism(800,-400)<0,400>[F(G(A\tri B)\otimes GC)`FG((A\tri B)\tri C);F\n{A\tri B,C}]
    \btriangle(800,-800)|mmm|/<-``->/<1400,400>[
      F(G(A\tri B)\otimes GC)`F((GA\otimes GB)\otimes GC)`F(GA\otimes(GB\otimes GC));
      F(\n{A,B}\otimes id_{GC})``F\alpha_{GA,GB,GC}]
    \morphism(800,-800)|m|<0,-400>[
      F((GA\otimes GB)\otimes GC)`F((GB\otimes GA)\otimes GC);F(\e{A,B}\otimes id_{GC})]
    \morphism(2200,-800)|m|<1000,-400>[
      F(GA\otimes(GB\otimes GC))`F(GA\otimes G(B\tri C));F(id_{GA}\otimes\n{B,C})]
    \morphism(2200,-800)|m|<0,-400>[
      F(GA\otimes(GB\otimes GC))`F((GB\otimes GC)\otimes GA);F\e{GA,GB\otimes GC}]
    \qtriangle(2200,-1600)|mmr|/`->`->/<1000,400>[
      F((GB\otimes GC)\otimes GA)`F(GA\otimes G(B\tri C))`F(G(B\tri C)\otimes GA);
      `F(\n{B,C}\otimes id_{GA})`F\e{A,B\tri C}]
    \morphism(800,-1200)|m|<0,-400>[
      F((GB\otimes GA)\otimes GC)`F(G(B\tri A)\otimes GC);F(\n{B,A}\otimes id_{GC})]
    \morphism(2200,-1200)|m|<0,-300>[
      F((GB\otimes GC)\otimes GA)`F(GB\otimes(GC\otimes GA));F\alpha_{GB,GC,GA}]
    \morphism(800,-1200)|m|<700,-600>[
      F((GB\otimes GA)\otimes GC)`F(GB\otimes(GA\otimes GC));F\alpha]
    \morphism(2200,-1500)|m|<-700,-300>[
      F(GB\otimes(GC\otimes GA))`F(GB\otimes(GA\otimes GC));F(id_{GC}\otimes\e{C,A})]
    \morphism(800,-1600)|m|<0,-400>[F(G(B\tri A)\otimes GC)`FG((B\tri A)\tri C);F\n{B\tri A,C}]
    \morphism(2200,-1500)|m|<0,-500>[
      F(GB\otimes(GC\otimes GA))`F(GB\otimes G(C\tri A));F(id_{GB}\otimes\n{C,A})]
    \morphism(2200,-2000)|m|<0,-600>[
      F(GB\otimes G(C\tri A))`FG(B\tri(C\tri A));F\n{B,C\tri A}]
    \morphism(1500,-1800)|m|<0,-400>[
      F(GB\otimes(GA\otimes GC))`F(GB\otimes G(A\tri C));F(id_{GB}\otimes\n{A,C})]
    \morphism(1500,-2200)|m|<-500,-400>[
      F(GB\otimes G(A\tri C))`FG(B\tri(A\tri C));F\n{B,A\tri C}]
    \morphism(0,400)|l|<0,-2000>[(A\tri B)\tri C`(B\tri A)\tri C;\e{A,B}^S\tri id_C]
    \morphism(800,-2000)|m|<-800,400>[
      FG((B\tri A)\tri C)`(B\tri A)\tri C;\varepsilon_{(B\tri A)\tri C}]
    \morphism(0,-1600)|l|<0,-1000>[(B\tri A)\tri C`B\tri(A\tri C);\alpha_{B,A,C}']
    \morphism(1000,-2600)|m|<-1000,0>[
      FG(B\tri(A\tri C))`B\tri(A\tri C);\varepsilon_{B\tri(A\tri C)}]
    \morphism(0,-2600)|l|<0,-400>[B\tri(A\tri C)`B\tri(C\tri A);\id_B\tri\e{A,C}^S]
    \morphism(1600,-3000)|b|<-1600,0>[(B\tri C)\tri A`B\tri(C\tri A);\alpha_{B,C,A}']
    \morphism(3200,-1600)|r|<0,-1400>[
      F(G(B\tri C)\otimes GA)`FG((B\tri C)\tri A);F\n{B\tri C,A}]
    \morphism(3200,-3000)|m|<-1000,400>[FG((B\tri C)\tri A)`FG(B\tri(C\tri A));FG\alpha_{B,C,A}]
    \morphism(3200,-3000)|b|<-1600,0>[
      FG((B\tri C)\tri A)`(B\tri C)\tri A;\varepsilon_{(B\tri C)\tri A}]
    \morphism(2200,-2600)|m|<-2200,-400>[
      FG(B\tri(C\tri A))`B\tri(C\tri A);\varepsilon_{B\tri(C\tri A)}]
    \morphism(800,-2000)|m|<200,-600>[FG((B\tri A)\tri C)`FG(B\tri(A\tri C));FG\alpha'_{B,A,C}]
    \morphism(300,-700)//<0,0>[`;(1)]
    \morphism(500,-2200)//<0,0>[`;(2)]
    \morphism(1500,100)//<0,0>[`;(3)]
    \morphism(1500,-500)//<0,0>[`;(4)]
    \morphism(1500,-1100)//<0,0>[`;(5)]
    \morphism(1100,-2200)//<0,0>[`;(6)]
    \morphism(1800,-2400)//<0,0>[`;(7)]
    \morphism(2000,-2800)//<0,0>[`;(8)]
    \morphism(2450,200)//<0,0>[`;(9)]
    \morphism(2900,-300)//<0,0>[`;(10)]
    \morphism(2800,-1200)//<0,0>[`;(11)]
    \morphism(2720,-2200)//<0,0>[`;(12)]
  \efig
  \end{mathpar}
\end{proof}


%%% Local Variables: 
%%% mode: latex
%%% TeX-master: main.tex
%%% End: 























% section categorical_models (end)

\section{Non-Commutative Linear Logic}
\label{sec:logic}
In a LAM, the SMCC $\cat{C}$ models the commutative linear logic and the Lambeck category
$\cat{L}$ models the non-commutative variant. In Section~\ref{subsec:elle}, we will present the
term assignment for sequent calculus of both sides and prove the cut elimination theorem. In
Section~\ref{subsec:elle-nd}, we present the term assignment for natural deduction of both sides
and prove the logic is strongly normalizing.

A sequent in the commutative side is of the form $\Psi  \NDsym{,}  \Phi  \vdash_\mathcal{C}  \NDnt{t}  \NDsym{:}  \NDnt{X}$. The types must be $X$,
$Y$, $Z$, etc., which are objects in the SMCC $\cat{C}$. Tye typing contexts are multisets.
Suppose $\Psi$ is the set $x_1:X_1, x_2:X_2, ..., x_m:X_m$ and $\Phi$ is the set
$y_1:Y_1, y_2:Y_2,...,y_n:Y_n$, then the categorical interpretation of the sequent is the
morphism $(X_1\otimes X_2\otimes...\otimes X_m)\otimes(Y_1\otimes Y_2\otimes...\otimes Y_n)\rightarrow X$.

A sequent in the non-commutative side is of the form $\Gamma  \NDsym{,}  \Delta  \vdash_\mathcal{L}  \NDnt{s}  \NDsym{:}  \NDnt{A}$. The types must be $A$,
$B$, $C$, etc., which are objects in the Lambek category $\cat{L}$. Tye typing contexts are
lists instead of multisets. The typing contexts are mixed in the sense that they could include
contexts from the commutative side. When a commutative context $\Psi=\{x_1:X_1,...,x_m:X_m\}$
is included, it is interpreted as the object $F(X_1\otimes...\otimes X_m)$. Therefore, the
interpretation of the sequent $\Psi  \NDsym{,}  \Gamma  \vdash_\mathcal{L}  \NDnt{s}  \NDsym{:}  \NDnt{A}$, where $\Psi$ is defined as above and $\Gamma$
is the list $y_1:A_1,...,y_n:A_n$, is the morphism
$F(X_1\otimes...\otimes X_m)\tri(A_1\tri...\tri A_n)\rightarrow A$.

For the commutative side, since the contexts are multisets, the following exchange rule
is implicit in both sequent calculus and natural deduction:

\begin{center}
  \scriptsize
  $\ElledruleTXXbeta{}$
\end{center}



%%%%%%%%%%%%%%%%%%%%%%%%%%%%%%%%%%%%%%%%%%%%%%%%%%
\subsection{Sequent Calculus}
\label{subsec:elle}

The term assignment for sequent calculus of the commutative part of the model, i.e. the SMCC of
the adjunction, is defined in Figure~\ref{fig:elle-smcc}. And the term assignme for the
non-commutative part, i.e. the Lambek category of the adjunction, is defined in
Figure~\ref{fig:elle-lambek}. We do not have the structural rules except for exchange because
the calculus is for linear logic. 

\begin{figure}[!h]
 \scriptsize
  \begin{mdframed}
    \begin{mathpar}
      \ElledruleTXXax{} \qquad\qquad \ElledruleTXXunitL{} \qquad\qquad \ElledruleTXXunitR{} \\
      \ElledruleTXXtenL{} \qquad\qquad \ElledruleTXXtenR{} \\
      \ElledruleTXXimpL{} \qquad\qquad \ElledruleTXXimpR{} \\
      \ElledruleTXXGr{} \qquad\qquad \ElledruleTXXcut{}
    \end{mathpar}
  \end{mdframed}
\caption{Sequent Calculus: Commutative Part}
\label{fig:elle-smcc}
\end{figure}

\begin{figure}[!h]
 \scriptsize
  \begin{mdframed}
    \begin{mathpar}
      \ElledruleSXXax{} \qquad\qquad \ElledruleSXXunitR{} \qquad\qquad \ElledruleSXXunitLOne{} \\
      \ElledruleSXXunitLTwo{} \qquad\qquad \ElledruleSXXbeta{} \\
      \ElledruleSXXtenLOne{} \qquad\qquad \ElledruleSXXtenLTwo{} \\
      \ElledruleSXXtenR{} \qquad\qquad \ElledruleSXXimpL{} \\
      \ElledruleSXXimprL{} \qquad\qquad \ElledruleSXXimplL{} \\
      \ElledruleSXXimprR{} \qquad\qquad \ElledruleSXXimplR{} \qquad\qquad \ElledruleSXXFr{} \\
      \ElledruleSXXFl{} \qquad\qquad \ElledruleSXXGl{} \\
      \ElledruleSXXcutOne{} \qquad\qquad \ElledruleSXXcutTwo{} \\
    \end{mathpar}
  \end{mdframed}
\caption{Sequent Calculus: Non-Commutative Part}
\label{fig:elle-lambek}
\end{figure}

Next, we prove cut elimination for the sequent calculus. We define the \textbf{degree $|X|$
(or $|A|$) of a commutative (or non-commutative) formula} to be the number of logical
connectives $|X|$ plus $1$. For instance, $|\NDnt{X}  \otimes  \NDnt{Y}| = |\NDnt{X}| + |\NDnt{Y}| + 1$. And the
\textbf{degree of a cut rule} is the degree of the cut formula. The following key cases
demonstrate how we can replace a cut with at most two cuts with lower degree. The
\textbf{degree $|\Pi|$ of a proof} $\Pi$ is the maximum of the degrees of all cut fules in the
proof and $|\Pi|=0$ if $\Pi$ is cut-free. Finally, the \textbf{height $h(\Pi)$ of a proof
$\Pi$} is the length of the longest path in the proof tree and the height of an axiom is $0$.

We consider the following $11$ key cases in proving cut elimination, each of which is a
$(R, L)$ pair for the same connective.

\begin{itemize}

\item $(\ElledruleTXXunitRName, \ElledruleTXXunitLName)$:
  \begin{center}
    \tiny
    \begin{math}
      $$\mprset{flushleft}
      \inferrule* [right={\tiny cut}] {
        $$\mprset{flushleft}
        \inferrule* [right={\tiny unitR}] {
          \,
        }{ \cdot   \vdash_\mathcal{C}   \mathsf{trivT}   \NDsym{:}   \mathsf{UnitT} }
        \\
        $$\mprset{flushleft}
        \inferrule* [right={\tiny unitL}] {
          {\Psi  \vdash_\mathcal{C}  \NDnt{t}  \NDsym{:}  \NDnt{X}}
        }{\NDmv{x}  \NDsym{:}   \mathsf{UnitT}   \NDsym{,}  \Psi  \vdash_\mathcal{C}   \mathsf{let}\, \NDmv{x}  :   \mathsf{UnitT}  \,\mathsf{be}\,  \mathsf{trivT}  \,\mathsf{in}\, \NDnt{t}   \NDsym{:}  \NDnt{X}}
      }{\Psi  \vdash_\mathcal{C}  \NDsym{[}   \mathsf{trivT}   \NDsym{/}  \NDmv{x}  \NDsym{]}  \NDsym{(}   \mathsf{let}\, \NDmv{x}  :   \mathsf{UnitT}  \,\mathsf{be}\,  \mathsf{trivT}  \,\mathsf{in}\, \NDnt{t}   \NDsym{)}  \NDsym{:}  \NDnt{X}}
    \end{math}
  \end{center}
  is transformed to 
  \begin{center}
    \tiny
    $\Psi  \vdash_\mathcal{C}  \NDnt{t}  \NDsym{:}  \NDnt{X}$
  \end{center}

\item $(\ElledruleTXXunitRName, \ElledruleSXXunitLOneName)$:
  \begin{center}
    \tiny
    \begin{math}
      $$\mprset{flushleft}
      \inferrule* [right={\tiny cut}] {
        $$\mprset{flushleft}
        \inferrule* [right={\tiny unitR}] {
          \,
        }{ \cdot   \vdash_\mathcal{C}   \mathsf{trivT}   \NDsym{:}   \mathsf{UnitT} }
        \\
        $$\mprset{flushleft}
        \inferrule* [right={\tiny unitL}] {
          {\Delta  \vdash_\mathcal{L}  \NDnt{s}  \NDsym{:}  \NDnt{A}}
        }{\NDmv{x}  \NDsym{:}   \mathsf{UnitT}   \NDsym{,}  \Delta  \vdash_\mathcal{L}   \mathsf{let}\, \NDmv{x}  :   \mathsf{UnitT}  \,\mathsf{be}\,  \mathsf{trivT}  \,\mathsf{in}\, \NDnt{s}   \NDsym{:}  \NDnt{A}}
      }{\Delta  \vdash_\mathcal{L}  \NDsym{[}   \mathsf{trivT}   \NDsym{/}  \NDmv{x}  \NDsym{]}  \NDsym{(}   \mathsf{let}\, \NDmv{x}  :   \mathsf{UnitT}  \,\mathsf{be}\,  \mathsf{trivT}  \,\mathsf{in}\, \NDnt{s}   \NDsym{)}  \NDsym{:}  \NDnt{A}}
    \end{math}
  \end{center}
  is transformed to
  \begin{center}
    \tiny
    $\Delta  \vdash_\mathcal{L}  \NDnt{s}  \NDsym{:}  \NDnt{A}$
  \end{center}

\item $(\ElledruleTXXtenRName, \ElledruleTXXtenLName)$:
  \begin{center}
    \tiny
    \begin{math}
      $$\mprset{flushleft}
      \inferrule* [right={\tiny cut}] {
        $$\mprset{flushleft}
        \inferrule* [right={\tiny tenR}] {
          {\Phi_{{\mathrm{1}}}  \vdash_\mathcal{C}  \NDnt{t_{{\mathrm{1}}}}  \NDsym{:}  \NDnt{X}} \\
          {\Phi_{{\mathrm{2}}}  \vdash_\mathcal{C}  \NDnt{t_{{\mathrm{2}}}}  \NDsym{:}  \NDnt{Y}}
        }{\Phi_{{\mathrm{1}}}  \NDsym{,}  \Phi_{{\mathrm{2}}}  \vdash_\mathcal{C}  \NDnt{t_{{\mathrm{1}}}}  \otimes  \NDnt{t_{{\mathrm{2}}}}  \NDsym{:}  \NDnt{X}  \otimes  \NDnt{Y}}
        \\
        $$\mprset{flushleft}
        \inferrule* [right={\tiny tenL}] {
          {\Psi_{{\mathrm{1}}}  \NDsym{,}  \NDmv{x}  \NDsym{:}  \NDnt{X}  \NDsym{,}  \NDmv{y}  \NDsym{:}  \NDnt{Y}  \NDsym{,}  \Psi_{{\mathrm{2}}}  \vdash_\mathcal{C}  \NDnt{t_{{\mathrm{3}}}}  \NDsym{:}  \NDnt{Z}}
        }{\Psi_{{\mathrm{1}}}  \NDsym{,}  \NDmv{z}  \NDsym{:}  \NDnt{X}  \otimes  \NDnt{Y}  \NDsym{,}  \Psi_{{\mathrm{2}}}  \vdash_\mathcal{C}   \mathsf{let}\, \NDmv{z}  :  \NDnt{X}  \otimes  \NDnt{Y} \,\mathsf{be}\, \NDmv{x}  \otimes  \NDmv{y} \,\mathsf{in}\, \NDnt{t_{{\mathrm{3}}}}   \NDsym{:}  \NDnt{Z}}
      }{\Psi_{{\mathrm{1}}}  \NDsym{,}  \Phi_{{\mathrm{1}}}  \NDsym{,}  \Phi_{{\mathrm{2}}}  \NDsym{,}  \Psi_{{\mathrm{2}}}  \vdash_\mathcal{C}  \NDsym{[}  \NDnt{t_{{\mathrm{1}}}}  \otimes  \NDnt{t_{{\mathrm{2}}}}  \NDsym{/}  \NDmv{z}  \NDsym{]}  \NDsym{(}   \mathsf{let}\, \NDmv{z}  :  \NDnt{X}  \otimes  \NDnt{Y} \,\mathsf{be}\, \NDmv{x}  \otimes  \NDmv{y} \,\mathsf{in}\, \NDnt{t_{{\mathrm{3}}}}   \NDsym{)}  \NDsym{:}  \NDnt{Z}}
    \end{math}
  \end{center}
  is transformed to
  \begin{center}
    \tiny
    \begin{math}
      $$\mprset{flushleft}
      \inferrule* [right={\tiny cut}] {
        $$\mprset{flushleft}
        \inferrule* [right={\tiny cut}] {
          {\Phi_{{\mathrm{1}}}  \vdash_\mathcal{C}  \NDnt{t_{{\mathrm{1}}}}  \NDsym{:}  \NDnt{X}} \\
          {\Psi_{{\mathrm{1}}}  \NDsym{,}  \NDmv{x}  \NDsym{:}  \NDnt{X}  \NDsym{,}  \NDmv{y}  \NDsym{:}  \NDnt{Y}  \NDsym{,}  \Psi_{{\mathrm{2}}}  \vdash_\mathcal{C}  \NDnt{t_{{\mathrm{3}}}}  \NDsym{:}  \NDnt{Z}}
        }{\Psi_{{\mathrm{1}}}  \NDsym{,}  \Phi_{{\mathrm{1}}}  \NDsym{,}  \NDmv{y}  \NDsym{:}  \NDnt{Y}  \NDsym{,}  \Psi_{{\mathrm{2}}}  \vdash_\mathcal{C}  \NDsym{[}  \NDnt{t_{{\mathrm{1}}}}  \NDsym{/}  \NDmv{x}  \NDsym{]}  \NDnt{t_{{\mathrm{3}}}}  \NDsym{:}  \NDnt{Z}} \\
        {\Phi_{{\mathrm{2}}}  \vdash_\mathcal{C}  \NDnt{t_{{\mathrm{2}}}}  \NDsym{:}  \NDnt{Y}}
      }{\Psi_{{\mathrm{1}}}  \NDsym{,}  \Phi_{{\mathrm{1}}}  \NDsym{,}  \Phi_{{\mathrm{2}}}  \NDsym{,}  \Psi_{{\mathrm{2}}}  \vdash_\mathcal{C}  \NDsym{[}  \NDnt{t_{{\mathrm{2}}}}  \NDsym{/}  \NDmv{y}  \NDsym{]}  \NDsym{[}  \NDnt{t_{{\mathrm{1}}}}  \NDsym{/}  \NDmv{x}  \NDsym{]}  \NDnt{t_{{\mathrm{3}}}}  \NDsym{:}  \NDnt{Z}}
    \end{math}
  \end{center}

\item $(\ElledruleTXXtenRName, \ElledruleSXXtenLOneName)$:
  \begin{center}
    \tiny
    \begin{math}
      $$\mprset{flushleft}
      \inferrule* [right={\tiny cut1}] {
        $$\mprset{flushleft}
        \inferrule* [right={\tiny tenR}] {
          {\Phi_{{\mathrm{1}}}  \vdash_\mathcal{C}  \NDnt{t_{{\mathrm{1}}}}  \NDsym{:}  \NDnt{X}} \\
          {\Phi_{{\mathrm{2}}}  \vdash_\mathcal{C}  \NDnt{t_{{\mathrm{2}}}}  \NDsym{:}  \NDnt{Y}}
        }{\Phi_{{\mathrm{1}}}  \NDsym{,}  \Phi_{{\mathrm{2}}}  \vdash_\mathcal{C}  \NDnt{t_{{\mathrm{1}}}}  \otimes  \NDnt{t_{{\mathrm{2}}}}  \NDsym{:}  \NDnt{X}  \otimes  \NDnt{Y}}
        \\
        $$\mprset{flushleft}
        \inferrule* [right={\tiny tenL1}] {
          {\Gamma  \NDsym{,}  \NDmv{x}  \NDsym{:}  \NDnt{X}  \NDsym{,}  \NDmv{y}  \NDsym{:}  \NDnt{Y}  \NDsym{,}  \Delta  \vdash_\mathcal{L}  \NDnt{s}  \NDsym{:}  \NDnt{A}}
        }{\Gamma  \NDsym{,}  \NDmv{z}  \NDsym{:}  \NDnt{X}  \otimes  \NDnt{Y}  \NDsym{,}  \Delta  \vdash_\mathcal{L}   \mathsf{let}\, \NDmv{z}  :  \NDnt{X}  \otimes  \NDnt{Y} \,\mathsf{be}\, \NDmv{x}  \otimes  \NDmv{y} \,\mathsf{in}\, \NDnt{s}   \NDsym{:}  \NDnt{A}}
      }{\Gamma  \NDsym{,}  \Phi_{{\mathrm{1}}}  \NDsym{,}  \Phi_{{\mathrm{2}}}  \NDsym{,}  \Delta  \vdash_\mathcal{L}  \NDsym{[}  \NDnt{t_{{\mathrm{1}}}}  \otimes  \NDnt{t_{{\mathrm{2}}}}  \NDsym{/}  \NDmv{z}  \NDsym{]}  \NDsym{(}   \mathsf{let}\, \NDmv{z}  :  \NDnt{X}  \otimes  \NDnt{Y} \,\mathsf{be}\, \NDmv{x}  \otimes  \NDmv{y} \,\mathsf{in}\, \NDnt{s}   \NDsym{)}  \NDsym{:}  \NDnt{A}}
    \end{math}
  \end{center}
  is transformed to
  \begin{center}
    \tiny
    \begin{math}
      $$\mprset{flushleft}
      \inferrule* [right={\tiny cut1}] {
        $$\mprset{flushleft}
        \inferrule* [right={\tiny cut1}] {
          {\Phi_{{\mathrm{1}}}  \vdash_\mathcal{C}  \NDnt{t_{{\mathrm{1}}}}  \NDsym{:}  \NDnt{X}} \\
          {\Gamma  \NDsym{,}  \NDmv{x}  \NDsym{:}  \NDnt{X}  \NDsym{,}  \NDmv{y}  \NDsym{:}  \NDnt{Y}  \NDsym{,}  \Delta  \vdash_\mathcal{L}  \NDnt{s}  \NDsym{:}  \NDnt{A}}
        }{\Gamma  \NDsym{,}  \Phi_{{\mathrm{1}}}  \NDsym{,}  \NDmv{y}  \NDsym{:}  \NDnt{Y}  \NDsym{,}  \Delta  \vdash_\mathcal{L}  \NDsym{[}  \NDnt{t_{{\mathrm{1}}}}  \NDsym{/}  \NDmv{x}  \NDsym{]}  \NDnt{s}  \NDsym{:}  \NDnt{A}} \\
        {\Phi_{{\mathrm{2}}}  \vdash_\mathcal{C}  \NDnt{t_{{\mathrm{2}}}}  \NDsym{:}  \NDnt{Y}}
      }{\Gamma  \NDsym{,}  \Phi_{{\mathrm{1}}}  \NDsym{,}  \Phi_{{\mathrm{2}}}  \NDsym{,}  \Delta  \vdash_\mathcal{L}  \NDsym{[}  \NDnt{t_{{\mathrm{2}}}}  \NDsym{/}  \NDmv{y}  \NDsym{]}  \NDsym{[}  \NDnt{t_{{\mathrm{1}}}}  \NDsym{/}  \NDmv{x}  \NDsym{]}  \NDnt{s}  \NDsym{:}  \NDnt{A}}
    \end{math}
  \end{center}
  
\item $(\ElledruleTXXimpRName, \ElledruleTXXimpLName)$
  \begin{center}
    \tiny
    \begin{math}
      $$\mprset{flushleft}
      \inferrule* [right={\tiny cut}] {
        $$\mprset{flushleft}
        \inferrule* [right={\tiny impR}] {
          {\Phi_{{\mathrm{1}}}  \NDsym{,}  \NDmv{x}  \NDsym{:}  \NDnt{X}  \vdash_\mathcal{C}  \NDnt{t_{{\mathrm{1}}}}  \NDsym{:}  \NDnt{Y}}
        }{\Phi_{{\mathrm{1}}}  \vdash_\mathcal{C}   \lambda  \NDmv{x}  :  \NDnt{X} . \NDnt{t_{{\mathrm{1}}}}   \NDsym{:}  \NDnt{X}  \multimap  \NDnt{Y}}
        \\
        $$\mprset{flushleft}
        \inferrule* [right={\tiny impL}] {
          {\Phi_{{\mathrm{2}}}  \vdash_\mathcal{C}  \NDnt{t_{{\mathrm{2}}}}  \NDsym{:}  \NDnt{X}} \\
          {\Psi_{{\mathrm{1}}}  \NDsym{,}  \NDmv{y}  \NDsym{:}  \NDnt{Y}  \NDsym{,}  \Psi_{{\mathrm{2}}}  \vdash_\mathcal{C}  \NDnt{t_{{\mathrm{3}}}}  \NDsym{:}  \NDnt{Z}}
        }{\Psi_{{\mathrm{1}}}  \NDsym{,}  \NDmv{z}  \NDsym{:}  \NDnt{X}  \multimap  \NDnt{Y}  \NDsym{,}  \Phi_{{\mathrm{2}}}  \NDsym{,}  \Psi_{{\mathrm{2}}}  \vdash_\mathcal{C}  \NDsym{[}   \mathsf{app}\, \NDmv{z} \, \NDnt{t_{{\mathrm{2}}}}   \NDsym{/}  \NDmv{y}  \NDsym{]}  \NDnt{t_{{\mathrm{3}}}}  \NDsym{:}  \NDnt{Z}}
      }{\Psi_{{\mathrm{1}}}  \NDsym{,}  \Phi_{{\mathrm{1}}}  \NDsym{,}  \Phi_{{\mathrm{2}}}  \NDsym{,}  \Psi_{{\mathrm{2}}}  \vdash_\mathcal{C}  \NDsym{[}  \NDsym{(}   \lambda  \NDmv{x}  :  \NDnt{X} . \NDnt{t_{{\mathrm{1}}}}   \NDsym{)}  \NDsym{/}  \NDmv{z}  \NDsym{]}  \NDsym{[}   \mathsf{app}\, \NDmv{z} \, \NDnt{t_{{\mathrm{2}}}}   \NDsym{/}  \NDmv{y}  \NDsym{]}  \NDnt{t_{{\mathrm{3}}}}  \NDsym{:}  \NDnt{Z}}
    \end{math}
  \end{center}
  is transformed to
  \begin{center}
    \tiny
    \begin{math}
      $$\mprset{flushleft}
      \inferrule* [right={\tiny cut}] {
        $$\mprset{flushleft}
        \inferrule* [right={\tiny cut}] {
          {\Phi_{{\mathrm{1}}}  \NDsym{,}  \NDmv{x}  \NDsym{:}  \NDnt{X}  \vdash_\mathcal{C}  \NDnt{t_{{\mathrm{1}}}}  \NDsym{:}  \NDnt{Y}} \\
          {\Phi_{{\mathrm{2}}}  \vdash_\mathcal{C}  \NDnt{t_{{\mathrm{2}}}}  \NDsym{:}  \NDnt{X}}
        }{\Phi_{{\mathrm{1}}}  \NDsym{,}  \Phi_{{\mathrm{2}}}  \vdash_\mathcal{C}  \NDsym{[}  \NDnt{t_{{\mathrm{2}}}}  \NDsym{/}  \NDmv{x}  \NDsym{]}  \NDnt{t_{{\mathrm{1}}}}  \NDsym{:}  \NDnt{Y}} \\
        {\Psi_{{\mathrm{1}}}  \NDsym{,}  \NDmv{y}  \NDsym{:}  \NDnt{Y}  \NDsym{,}  \Psi_{{\mathrm{2}}}  \vdash_\mathcal{C}  \NDnt{t_{{\mathrm{3}}}}  \NDsym{:}  \NDnt{Z}}
      }{\Psi_{{\mathrm{1}}}  \NDsym{,}  \Phi_{{\mathrm{1}}}  \NDsym{,}  \Phi_{{\mathrm{2}}}  \NDsym{,}  \Psi_{{\mathrm{2}}}  \vdash_\mathcal{C}  \NDsym{[}  \NDsym{(}  \NDsym{[}  \NDnt{t_{{\mathrm{2}}}}  \NDsym{/}  \NDmv{x}  \NDsym{]}  \NDnt{t_{{\mathrm{1}}}}  \NDsym{)}  \NDsym{/}  \NDmv{y}  \NDsym{]}  \NDnt{t_{{\mathrm{3}}}}  \NDsym{:}  \NDnt{Z}}
    \end{math}
  \end{center}

\item $(\ElledruleSXXunitRName, \ElledruleSXXunitLTwoName)$:
  \begin{center}
    \tiny
    \begin{math}
      $$\mprset{flushleft}
      \inferrule* [right={\tiny cut2}] {
        $$\mprset{flushleft}
        \inferrule* [right={\tiny unitR}] {
          \,
        }{ \cdot   \vdash_\mathcal{L}   \mathsf{trivS}   \NDsym{:}   \mathsf{UnitS} }
        \\
        $$\mprset{flushleft}
        \inferrule* [right={\tiny unitL2}] {
          {\Delta  \vdash_\mathcal{L}  \NDnt{s}  \NDsym{:}  \NDnt{A}}
        }{\NDmv{x}  \NDsym{:}   \mathsf{UnitS}   \NDsym{,}  \Delta  \vdash_\mathcal{L}   \mathsf{let}\, \NDmv{x}  :   \mathsf{UnitS}  \,\mathsf{be}\,  \mathsf{trivS}  \,\mathsf{in}\, \NDnt{s}   \NDsym{:}  \NDnt{A}}
      }{\Delta  \vdash_\mathcal{L}  \NDsym{[}   \mathsf{trivS}   \NDsym{/}  \NDmv{x}  \NDsym{]}  \NDsym{(}   \mathsf{let}\, \NDmv{x}  :   \mathsf{UnitS}  \,\mathsf{be}\,  \mathsf{trivS}  \,\mathsf{in}\, \NDnt{s}   \NDsym{)}  \NDsym{:}  \NDnt{A}}
    \end{math}
  \end{center}
  is transformed to 
  \begin{center}
    \tiny
    $\Delta  \vdash_\mathcal{L}  \NDnt{s}  \NDsym{:}  \NDnt{A}$
  \end{center}

\item $(\ElledruleSXXtenRName, \ElledruleSXXtenLTwoName)$:
  \begin{center}
    \tiny
    \begin{math}
      $$\mprset{flushleft}
      \inferrule* [right={\tiny cut2}] {
        $$\mprset{flushleft}
        \inferrule* [right={\tiny tenR}] {
          {\Gamma_{{\mathrm{1}}}  \vdash_\mathcal{L}  \NDnt{s_{{\mathrm{1}}}}  \NDsym{:}  \NDnt{A}} \\
          {\Gamma_{{\mathrm{2}}}  \vdash_\mathcal{L}  \NDnt{s_{{\mathrm{2}}}}  \NDsym{:}  \NDnt{B}}
        }{\Gamma_{{\mathrm{1}}}  \NDsym{,}  \Gamma_{{\mathrm{2}}}  \vdash_\mathcal{L}  \NDnt{s_{{\mathrm{1}}}}  \triangleright  \NDnt{s_{{\mathrm{2}}}}  \NDsym{:}  \NDnt{A}  \triangleright  \NDnt{B}}
        \\
        $$\mprset{flushleft}
        \inferrule* [right={\tiny tenL1}] {
          {\Delta_{{\mathrm{1}}}  \NDsym{,}  \NDmv{x}  \NDsym{:}  \NDnt{A}  \NDsym{,}  \NDmv{y}  \NDsym{:}  \NDnt{B}  \NDsym{,}  \Delta_{{\mathrm{2}}}  \vdash_\mathcal{L}  \NDnt{s_{{\mathrm{3}}}}  \NDsym{:}  \NDnt{C}}
        }{\Delta_{{\mathrm{1}}}  \NDsym{,}  \NDmv{z}  \NDsym{:}  \NDnt{A}  \triangleright  \NDnt{B}  \NDsym{,}  \Delta_{{\mathrm{2}}}  \vdash_\mathcal{L}   \mathsf{let}\, \NDmv{z}  :  \NDnt{A}  \triangleright  \NDnt{B} \,\mathsf{be}\, \NDmv{x}  \triangleright  \NDmv{y} \,\mathsf{in}\, \NDnt{s}   \NDsym{:}  \NDnt{C}}
      }{\Delta_{{\mathrm{1}}}  \NDsym{,}  \Gamma_{{\mathrm{1}}}  \NDsym{,}  \Gamma_{{\mathrm{2}}}  \NDsym{,}  \Delta_{{\mathrm{2}}}  \vdash_\mathcal{L}  \NDsym{[}  \NDnt{s_{{\mathrm{1}}}}  \triangleright  \NDnt{s_{{\mathrm{2}}}}  \NDsym{/}  \NDmv{z}  \NDsym{]}  \NDsym{(}   \mathsf{let}\, \NDmv{z}  :  \NDnt{A}  \triangleright  \NDnt{B} \,\mathsf{be}\, \NDmv{x}  \triangleright  \NDmv{y} \,\mathsf{in}\, \NDnt{s}   \NDsym{)}  \NDsym{:}  \NDnt{C}}
    \end{math}
  \end{center}
  is transformed to
  \begin{center}
    \tiny
    \begin{math}
      $$\mprset{flushleft}
      \inferrule* [right={\tiny cut2}] {
        $$\mprset{flushleft}
        \inferrule* [right={\tiny cut2}] {
          {\Gamma_{{\mathrm{1}}}  \vdash_\mathcal{L}  \NDnt{s_{{\mathrm{1}}}}  \NDsym{:}  \NDnt{A}} \\
          {\Delta_{{\mathrm{1}}}  \NDsym{,}  \NDmv{x}  \NDsym{:}  \NDnt{A}  \NDsym{,}  \NDmv{y}  \NDsym{:}  \NDnt{B}  \NDsym{,}  \Delta_{{\mathrm{2}}}  \vdash_\mathcal{L}  \NDnt{s_{{\mathrm{3}}}}  \NDsym{:}  \NDnt{C}}
        }{\Delta_{{\mathrm{1}}}  \NDsym{,}  \Gamma_{{\mathrm{1}}}  \NDsym{,}  \NDmv{y}  \NDsym{:}  \NDnt{B}  \NDsym{,}  \Delta_{{\mathrm{2}}}  \vdash_\mathcal{L}  \NDsym{[}  \NDnt{s_{{\mathrm{1}}}}  \NDsym{/}  \NDmv{x}  \NDsym{]}  \NDnt{s_{{\mathrm{3}}}}  \NDsym{:}  \NDnt{C}} \\
        {\Gamma_{{\mathrm{2}}}  \vdash_\mathcal{L}  \NDnt{s_{{\mathrm{2}}}}  \NDsym{:}  \NDnt{B}}
      }{\Delta_{{\mathrm{1}}}  \NDsym{,}  \Gamma_{{\mathrm{1}}}  \NDsym{,}  \Gamma_{{\mathrm{2}}}  \NDsym{,}  \Delta_{{\mathrm{2}}}  \vdash_\mathcal{L}  \NDsym{[}  \NDnt{s_{{\mathrm{2}}}}  \NDsym{/}  \NDmv{y}  \NDsym{]}  \NDsym{[}  \NDnt{s_{{\mathrm{1}}}}  \NDsym{/}  \NDmv{x}  \NDsym{]}  \NDnt{s_{{\mathrm{3}}}}  \NDsym{:}  \NDnt{C}}
    \end{math}
  \end{center}

\item $(\ElledruleSXXimprRName, \ElledruleSXXimprLName)$:
  \begin{center}
    \tiny
    \begin{math}
      $$\mprset{flushleft}
      \inferrule* [right={\tiny cut2}] {
        $$\mprset{flushleft}
        \inferrule* [right={\tiny imprR}] {
          {\Gamma  \NDsym{,}  \NDmv{x}  \NDsym{:}  \NDnt{A}  \vdash_\mathcal{L}  \NDnt{s_{{\mathrm{1}}}}  \NDsym{:}  \NDnt{B}}
        }{\Gamma  \vdash_\mathcal{L}   \lambda_r  \NDmv{x}  :  \NDnt{A} . \NDnt{s_{{\mathrm{1}}}}   \NDsym{:}  \NDnt{A}  \rightharpoonup  \NDnt{B}}
        \\
        $$\mprset{flushleft}
        \inferrule* [right={\tiny imprL}] {
          {\Delta_{{\mathrm{1}}}  \vdash_\mathcal{L}  \NDnt{s_{{\mathrm{2}}}}  \NDsym{:}  \NDnt{A}} \\
          {\Delta_{{\mathrm{2}}}  \NDsym{,}  \NDmv{y}  \NDsym{:}  \NDnt{B}  \vdash_\mathcal{L}  \NDnt{s_{{\mathrm{3}}}}  \NDsym{:}  \NDnt{C}}
        }{\Delta_{{\mathrm{2}}}  \NDsym{,}  \NDmv{z}  \NDsym{:}  \NDnt{A}  \rightharpoonup  \NDnt{B}  \NDsym{,}  \Delta_{{\mathrm{1}}}  \vdash_\mathcal{L}  \NDsym{[}   \mathsf{app}_r\, \NDmv{z} \, \NDnt{s_{{\mathrm{2}}}}   \NDsym{/}  \NDmv{y}  \NDsym{]}  \NDnt{s_{{\mathrm{3}}}}  \NDsym{:}  \NDnt{C}}
      }{\Delta_{{\mathrm{2}}}  \NDsym{,}  \Gamma  \NDsym{,}  \Delta_{{\mathrm{1}}}  \vdash_\mathcal{L}  \NDsym{[}  \NDsym{(}   \lambda_r  \NDmv{x}  :  \NDnt{A} . \NDnt{s_{{\mathrm{1}}}}   \NDsym{)}  \NDsym{/}  \NDmv{z}  \NDsym{]}  \NDsym{[}   \mathsf{app}_r\, \NDmv{z} \, \NDnt{s_{{\mathrm{2}}}}   \NDsym{/}  \NDmv{y}  \NDsym{]}  \NDnt{s_{{\mathrm{3}}}}  \NDsym{:}  \NDnt{C}}
    \end{math}
  \end{center}
  is transformed to
  \begin{center}
    \tiny
    \begin{math}
      $$\mprset{flushleft}
      \inferrule* [right={\tiny cut2}] {
        $$\mprset{flushleft}
        \inferrule* [right={\tiny cut2}] {
          {\Gamma  \NDsym{,}  \NDmv{x}  \NDsym{:}  \NDnt{A}  \vdash_\mathcal{L}  \NDnt{s_{{\mathrm{1}}}}  \NDsym{:}  \NDnt{B}} \\
          {\Delta_{{\mathrm{1}}}  \vdash_\mathcal{L}  \NDnt{s_{{\mathrm{2}}}}  \NDsym{:}  \NDnt{A}}
        }{\Gamma  \NDsym{,}  \Delta_{{\mathrm{1}}}  \vdash_\mathcal{L}  \NDsym{[}  \NDnt{s_{{\mathrm{2}}}}  \NDsym{/}  \NDmv{x}  \NDsym{]}  \NDnt{s_{{\mathrm{1}}}}  \NDsym{:}  \NDnt{B}} \\
        {\Delta_{{\mathrm{2}}}  \NDsym{,}  \NDmv{y}  \NDsym{:}  \NDnt{B}  \vdash_\mathcal{L}  \NDnt{s_{{\mathrm{3}}}}  \NDsym{:}  \NDnt{C}}
      }{\Delta_{{\mathrm{2}}}  \NDsym{,}  \Gamma  \NDsym{,}  \Delta_{{\mathrm{1}}}  \vdash_\mathcal{L}  \NDsym{[}  \NDsym{(}  \NDsym{[}  \NDnt{s_{{\mathrm{2}}}}  \NDsym{/}  \NDmv{x}  \NDsym{]}  \NDnt{s_{{\mathrm{1}}}}  \NDsym{)}  \NDsym{/}  \NDmv{y}  \NDsym{]}  \NDnt{s_{{\mathrm{3}}}}  \NDsym{:}  \NDnt{C}}
    \end{math}
  \end{center}

\item $(\ElledruleSXXimplRName, \ElledruleSXXimplLName)$:
  \begin{center}
    \tiny
    \begin{math}
      $$\mprset{flushleft}
      \inferrule* [right={\tiny cut2}] {
        $$\mprset{flushleft}
        \inferrule* [right={\tiny implR}] {
          {\NDmv{x}  \NDsym{:}  \NDnt{A}  \NDsym{,}  \Gamma  \vdash_\mathcal{L}  \NDnt{s_{{\mathrm{1}}}}  \NDsym{:}  \NDnt{B}}
        }{\Gamma  \vdash_\mathcal{L}   \lambda_l  \NDmv{x}  :  \NDnt{A} . \NDnt{s_{{\mathrm{1}}}}   \NDsym{:}  \NDnt{B}  \leftharpoonup  \NDnt{A}}
        \\
        $$\mprset{flushleft}
        \inferrule* [right={\tiny implL}] {
          {\Delta_{{\mathrm{1}}}  \vdash_\mathcal{L}  \NDnt{s_{{\mathrm{2}}}}  \NDsym{:}  \NDnt{A}} \\
          {\NDmv{y}  \NDsym{:}  \NDnt{B}  \NDsym{,}  \Delta_{{\mathrm{2}}}  \vdash_\mathcal{L}  \NDnt{s_{{\mathrm{3}}}}  \NDsym{:}  \NDnt{C}}
        }{\Delta_{{\mathrm{1}}}  \NDsym{,}  \NDmv{z}  \NDsym{:}  \NDnt{B}  \leftharpoonup  \NDnt{A}  \NDsym{,}  \Delta_{{\mathrm{2}}}  \vdash_\mathcal{L}  \NDsym{[}   \mathsf{app}_l\, \NDmv{z} \, \NDnt{s_{{\mathrm{2}}}}   \NDsym{/}  \NDmv{y}  \NDsym{]}  \NDnt{s_{{\mathrm{3}}}}  \NDsym{:}  \NDnt{C}}
      }{\Delta_{{\mathrm{1}}}  \NDsym{,}  \Gamma  \NDsym{,}  \Delta_{{\mathrm{2}}}  \vdash_\mathcal{L}  \NDsym{[}  \NDsym{(}   \lambda_l  \NDmv{x}  :  \NDnt{A} . \NDnt{s_{{\mathrm{1}}}}   \NDsym{)}  \NDsym{/}  \NDmv{z}  \NDsym{]}  \NDsym{[}   \mathsf{app}_l\, \NDmv{z} \, \NDnt{s_{{\mathrm{2}}}}   \NDsym{/}  \NDmv{y}  \NDsym{]}  \NDnt{s_{{\mathrm{3}}}}  \NDsym{:}  \NDnt{C}}
    \end{math}
  \end{center}
  is transformed to
  \begin{center}
    \tiny
    \begin{math}
      $$\mprset{flushleft}
      \inferrule* [right={\tiny cut2}] {
        $$\mprset{flushleft}
        \inferrule* [right={\tiny cut2}] {
          {\NDmv{x}  \NDsym{:}  \NDnt{A}  \NDsym{,}  \Gamma  \vdash_\mathcal{L}  \NDnt{s_{{\mathrm{1}}}}  \NDsym{:}  \NDnt{B}} \\
          {\Delta_{{\mathrm{1}}}  \vdash_\mathcal{L}  \NDnt{s_{{\mathrm{2}}}}  \NDsym{:}  \NDnt{A}}
        }{\Delta_{{\mathrm{1}}}  \NDsym{,}  \Gamma  \vdash_\mathcal{L}  \NDsym{[}  \NDnt{s_{{\mathrm{2}}}}  \NDsym{/}  \NDmv{x}  \NDsym{]}  \NDnt{s_{{\mathrm{1}}}}  \NDsym{:}  \NDnt{B}} \\
        {\NDmv{y}  \NDsym{:}  \NDnt{B}  \NDsym{,}  \Delta_{{\mathrm{2}}}  \vdash_\mathcal{L}  \NDnt{s_{{\mathrm{3}}}}  \NDsym{:}  \NDnt{C}}
      }{\Delta_{{\mathrm{1}}}  \NDsym{,}  \Gamma  \NDsym{,}  \Delta_{{\mathrm{2}}}  \vdash_\mathcal{L}  \NDsym{[}  \NDsym{(}  \NDsym{[}  \NDnt{s_{{\mathrm{2}}}}  \NDsym{/}  \NDmv{x}  \NDsym{]}  \NDnt{s_{{\mathrm{1}}}}  \NDsym{)}  \NDsym{/}  \NDmv{y}  \NDsym{]}  \NDnt{s_{{\mathrm{3}}}}  \NDsym{:}  \NDnt{C}}
    \end{math}
  \end{center}

\item $(\ElledruleSXXFrName, \ElledruleSXXFlName)$:
  \begin{center}
    \tiny
    \begin{math}
      $$\mprset{flushleft}
      \inferrule* [right={\tiny cut2}] {
        $$\mprset{flushleft}
        \inferrule* [right={\tiny FR}] {
          {\Phi  \vdash_\mathcal{C}  \NDnt{t}  \NDsym{:}  \NDnt{X}}
        }{\Phi  \vdash_\mathcal{L}   \mathsf{F} \NDnt{t}   \NDsym{:}   \mathsf{F} \NDnt{X} }
        \\
        $$\mprset{flushleft}
        \inferrule* [right={\tiny FL}] {
          {\Gamma  \NDsym{,}  \NDmv{x}  \NDsym{:}  \NDnt{X}  \NDsym{,}  \Delta  \vdash_\mathcal{L}  \NDnt{s}  \NDsym{:}  \NDnt{A}}
        }{\Gamma  \NDsym{,}  \NDmv{y}  \NDsym{:}   \mathsf{F} \NDnt{X}   \NDsym{,}  \Delta  \vdash_\mathcal{L}   \mathsf{let}\, \NDmv{y}  :   \mathsf{F} \NDnt{X}  \,\mathsf{be}\,  \mathsf{F}\, \NDmv{x}  \,\mathsf{in}\, \NDnt{s}   \NDsym{:}  \NDnt{A}}
      }{\Gamma  \NDsym{,}  \Phi  \NDsym{,}  \Delta  \vdash_\mathcal{L}  \NDsym{[}   \mathsf{F} \NDnt{t}   \NDsym{/}  \NDmv{y}  \NDsym{]}  \NDsym{(}   \mathsf{let}\, \NDmv{y}  :   \mathsf{F} \NDnt{X}  \,\mathsf{be}\,  \mathsf{F}\, \NDmv{x}  \,\mathsf{in}\, \NDnt{s}   \NDsym{)}  \NDsym{:}  \NDnt{A}}
    \end{math}
  \end{center}
  is transformed to
  \begin{center}
    \tiny
    \begin{math}
      $$\mprset{flushleft}
      \inferrule* [right={\tiny cut1}] {
        {\Phi  \vdash_\mathcal{C}  \NDnt{t}  \NDsym{:}  \NDnt{X}} \\
        {\Gamma  \NDsym{,}  \NDmv{x}  \NDsym{:}  \NDnt{A}  \NDsym{,}  \Delta  \vdash_\mathcal{L}  \NDnt{s}  \NDsym{:}  \NDnt{A}}
      }{\Gamma  \NDsym{,}  \Phi  \NDsym{,}  \Delta  \vdash_\mathcal{L}  \NDsym{[}  \NDnt{t}  \NDsym{/}  \NDmv{x}  \NDsym{]}  \NDnt{s}  \NDsym{:}  \NDnt{A}}
    \end{math}
  \end{center}

\item $(\ElledruleTXXGrName, \ElledruleSXXGlName)$:
  \begin{center}
    \tiny
    \begin{math}
      $$\mprset{flushleft}
      \inferrule* [right={\tiny cut1}] {
        $$\mprset{flushleft}
        \inferrule* [right={\tiny GR}] {
          {\Phi  \vdash_\mathcal{L}  \NDnt{s_{{\mathrm{1}}}}  \NDsym{:}  \NDnt{A}}
        }{\Phi  \vdash_\mathcal{C}   \mathsf{G} \NDnt{s_{{\mathrm{1}}}}   \NDsym{:}   \mathsf{G} \NDnt{A} }
        \\
        $$\mprset{flushleft}
        \inferrule* [right={\tiny GL}] {
          {\Gamma  \NDsym{,}  \NDmv{x}  \NDsym{:}  \NDnt{A}  \NDsym{,}  \Delta  \vdash_\mathcal{L}  \NDnt{s_{{\mathrm{2}}}}  \NDsym{:}  \NDnt{B}}
        }{\Gamma  \NDsym{,}  \NDmv{y}  \NDsym{:}   \mathsf{G} \NDnt{A}   \NDsym{,}  \Delta  \vdash_\mathcal{L}   \mathsf{let}\, \NDmv{y}  :   \mathsf{G} \NDnt{A}  \,\mathsf{be}\,  \mathsf{G}\, \NDmv{x}  \,\mathsf{in}\, \NDnt{s_{{\mathrm{2}}}}   \NDsym{:}  \NDnt{B}}
      }{\Gamma  \NDsym{,}  \Phi  \NDsym{,}  \Delta  \vdash_\mathcal{L}  \NDsym{[}   \mathsf{G} \NDnt{s_{{\mathrm{1}}}}   \NDsym{/}  \NDmv{y}  \NDsym{]}  \NDsym{(}   \mathsf{let}\, \NDmv{y}  :   \mathsf{G} \NDnt{A}  \,\mathsf{be}\,  \mathsf{G}\, \NDmv{x}  \,\mathsf{in}\, \NDnt{s_{{\mathrm{2}}}}   \NDsym{)}  \NDsym{:}  \NDnt{B}}
    \end{math}
  \end{center}
  is transformed to
  \begin{center}
    \tiny
    \begin{math}
      $$\mprset{flushleft}
      \inferrule* [right={\tiny cut}] {
        {\Phi  \vdash_\mathcal{L}  \NDnt{s_{{\mathrm{1}}}}  \NDsym{:}  \NDnt{A}} \\
        {\Gamma  \NDsym{,}  \NDmv{x}  \NDsym{:}  \NDnt{A}  \NDsym{,}  \Delta  \vdash_\mathcal{L}  \NDnt{s_{{\mathrm{2}}}}  \NDsym{:}  \NDnt{B}}
      }{\Gamma  \NDsym{,}  \Phi  \NDsym{,}  \Delta  \vdash_\mathcal{L}  \NDsym{[}  \NDnt{s_{{\mathrm{1}}}}  \NDsym{/}  \NDmv{x}  \NDsym{]}  \NDnt{s_{{\mathrm{2}}}}  \NDsym{:}  \NDnt{B}}
    \end{math}
  \end{center}

\end{itemize}

Based on the key cases, given a formula (either commutative or non-commutative) $L$ and proofs
$\Pi$, $\Pi'$, of sequents $M\vdash N$ and $M'\vdash N'$ respectively with degrees less than
$|L|$, there is a proof of $M,N\vdash M',N'$ with degree less than $|L|$, s.t. all currences of
formula $L$ is removed. This can be proved by induction on $h(\Pi)+h(\Pi')$. Therefore, we have
the result that given a proof of a sequent with degree $d>0$, there is a proof of the same
sequent. As a result, we have the cut elimination theorem.

\begin{theorem}[Cut Elimination]
  Let $\Pi$ be a proof of a sequent $\Phi  \vdash_\mathcal{C}  \NDnt{t}  \NDsym{:}  \NDnt{X}$ or a sequent $\Gamma  \vdash_\mathcal{L}  \NDnt{s}  \NDsym{:}  \NDnt{A}$ s.t.
  $|\Pi|>0$. Then there is a cut-free proof of the same sequent.
\end{theorem}



%%%%%%%%%%%%%%%%%%%%%%%%%%%%%%%%%%%%%%%%%%%%%%%%%%
\subsection{Natural Deduction}
\label{subsec:elle-nd}

The term assignment for natural deduction of the commutative part of the model, i.e. the SMCC
of the adjunction, is defined in Figure~\ref{fig:elle-nd-smcc}. And the term assignme for the
non-commutative part, i.e. the Lambek category of the adjunction, is defined in
Figure~\ref{fig:elle-nd-lambek}.

\begin{figure}[!h]
  \scriptsize
  \begin{mdframed}
    \begin{mathpar}
      \NDdruleTXXid{} \qquad\qquad \NDdruleTXXunitI{} \qquad\qquad \NDdruleTXXunitE{} \\
      \NDdruleTXXtenI{} \qquad\qquad \NDdruleTXXtenE{} \\
      \NDdruleTXXimpI{} \qquad\qquad \NDdruleTXXimpE{} \qquad\qquad \NDdruleTXXGI{} \\
      \NDdruleSXXbeta{}
    \end{mathpar}
  \end{mdframed}
\caption{Natural Deduction: Commutative Part}
\label{fig:elle-nd-smcc}
\end{figure}

\begin{figure}[!h]
 \scriptsize
  \begin{mdframed}
    \begin{mathpar}
      \NDdruleSXXid{} \qquad\qquad \NDdruleSXXunitI{} \qquad\qquad \NDdruleSXXunitEOne{} \\
      \NDdruleSXXunitEOne{} \qquad\qquad \NDdruleSXXunitETwo{} \\
      \NDdruleSXXtenI{} \qquad\qquad \NDdruleSXXtenEOne{} \\
      \NDdruleSXXtenETwo{} \qquad\qquad \NDdruleSXXimprI{} \\
      \NDdruleSXXimprE{} \qquad\qquad \NDdruleSXXimplI{} \\
      \NDdruleSXXimplE{} \qquad\qquad \NDdruleSXXGE{} \qquad\qquad \NDdruleSXXFI{} \\
      \NDdruleSXXFE{}
    \end{mathpar}
  \end{mdframed}
\caption{Natural Deduction: Non-Commutative Part}
\label{fig:elle-nd-lambek}
\end{figure}

We could derive exchange comonadically as follows:

\begin{center}
  \tiny
  \begin{math}
  $$\mprset{flushleft}
  \inferrule* [right={\tiny imprI}] {
    $$\mprset{flushleft}
    \inferrule* [right={\tiny tenE2}] {
      $$\mprset{flushleft}
      \inferrule* [right={\tiny id}] {
        \,
      }{\NDmv{z}  \NDsym{:}    \mathsf{F}  \mathsf{G} \NDnt{A}     \triangleright   \mathsf{F}  \mathsf{G} \NDnt{B}    \vdash_\mathcal{L}  \NDmv{z}  \NDsym{:}    \mathsf{F}  \mathsf{G} \NDnt{A}     \triangleright   \mathsf{F}  \mathsf{G} \NDnt{B}  }
        $$\mprset{flushleft}
        \inferrule* [right={\tiny FE}] {
          $$\mprset{flushleft}
          \inferrule* [right={\tiny id}] {
            \,
          }{\NDmv{x_{{\mathrm{2}}}}  \NDsym{:}   \mathsf{F}  \mathsf{G} \NDnt{A}    \vdash_\mathcal{L}  \NDmv{x_{{\mathrm{2}}}}  \NDsym{:}   \mathsf{F}  \mathsf{G} \NDnt{A}  }
            $$\mprset{flushleft}
            \inferrule* [right={\tiny FE}] {
              $$\mprset{flushleft}
              \inferrule* [right={\tiny id}] {
                \,
              }{\NDmv{y_{{\mathrm{2}}}}  \NDsym{:}   \mathsf{F}  \mathsf{G} \NDnt{B}    \vdash_\mathcal{L}  \NDmv{y_{{\mathrm{2}}}}  \NDsym{:}   \mathsf{F}  \mathsf{G} \NDnt{B}  }
              \inferrule* [right={\tiny beta}] {
                $$\mprset{flushleft}
                \inferrule* [right={\tiny FE}] {
                  $$\mprset{flushleft}
                  \inferrule* [right={\tiny FI}] {
                    $$\mprset{flushleft}
                    \inferrule* [right={\tiny id}] {
                      \,
                    }{\NDmv{y_{{\mathrm{0}}}}  \NDsym{:}   \mathsf{G} \NDnt{B}   \vdash_\mathcal{C}  \NDmv{y_{{\mathrm{0}}}}  \NDsym{:}   \mathsf{G} \NDnt{B} }
                  }{\NDmv{y_{{\mathrm{0}}}}  \NDsym{:}   \mathsf{G} \NDnt{B}   \vdash_\mathcal{L}   \mathsf{F} \NDmv{y_{{\mathrm{0}}}}   \NDsym{:}   \mathsf{F}  \mathsf{G} \NDnt{B}  }
                  $$\mprset{flushleft}
                  \inferrule* [right={\tiny FI}] {
                    $$\mprset{flushleft}
                    \inferrule* [right={\tiny id}] {
                      \,
                    }{\NDmv{x_{{\mathrm{0}}}}  \NDsym{:}   \mathsf{G} \NDnt{A}   \vdash_\mathcal{C}  \NDmv{x_{{\mathrm{0}}}}  \NDsym{:}   \mathsf{G} \NDnt{A} }
                  }{\NDmv{x_{{\mathrm{0}}}}  \NDsym{:}   \mathsf{G} \NDnt{A}   \vdash_\mathcal{L}   \mathsf{F} \NDmv{x_{{\mathrm{0}}}}   \NDsym{:}   \mathsf{F}  \mathsf{G} \NDnt{A}  }
                }{\NDmv{y_{{\mathrm{0}}}}  \NDsym{:}   \mathsf{G} \NDnt{B}   \NDsym{,}  \NDmv{x_{{\mathrm{0}}}}  \NDsym{:}   \mathsf{G} \NDnt{A}   \vdash_\mathcal{L}    \mathsf{F} \NDmv{y_{{\mathrm{0}}}}    \triangleright   \mathsf{F} \NDmv{x_{{\mathrm{0}}}}   \NDsym{:}    \mathsf{F}  \mathsf{G} \NDnt{B}     \triangleright   \mathsf{F}  \mathsf{G} \NDnt{A}  }
              }{\NDmv{x_{{\mathrm{1}}}}  \NDsym{:}   \mathsf{G} \NDnt{A}   \NDsym{,}  \NDmv{y_{{\mathrm{1}}}}  \NDsym{:}   \mathsf{G} \NDnt{B}   \vdash_\mathcal{L}   \mathsf{ex}\, \NDmv{y_{{\mathrm{1}}}} , \NDmv{x_{{\mathrm{1}}}} \,\mathsf{with}\, \NDmv{y_{{\mathrm{0}}}} , \NDmv{x_{{\mathrm{0}}}} \,\mathsf{in}\, \NDsym{(}    \mathsf{F} \NDmv{y_{{\mathrm{0}}}}    \triangleright   \mathsf{F} \NDmv{x_{{\mathrm{0}}}}   \NDsym{)}   \NDsym{:}    \mathsf{F}  \mathsf{G} \NDnt{B}     \triangleright   \mathsf{F}  \mathsf{G} \NDnt{A}  }
            }{\NDmv{x_{{\mathrm{1}}}}  \NDsym{:}   \mathsf{G} \NDnt{A}   \NDsym{,}  \NDmv{y_{{\mathrm{2}}}}  \NDsym{:}   \mathsf{F}  \mathsf{G} \NDnt{B}    \vdash_\mathcal{L}   \mathsf{let}\,  \mathsf{F} \NDmv{y_{{\mathrm{1}}}}   :   \mathsf{F}  \mathsf{G} \NDnt{B}   \,\mathsf{be}\, \NDmv{y_{{\mathrm{2}}}} \,\mathsf{in}\, \NDsym{(}   \mathsf{ex}\, \NDmv{y_{{\mathrm{1}}}} , \NDmv{x_{{\mathrm{1}}}} \,\mathsf{with}\, \NDmv{y_{{\mathrm{0}}}} , \NDmv{x_{{\mathrm{0}}}} \,\mathsf{in}\, \NDsym{(}    \mathsf{F} \NDmv{y_{{\mathrm{0}}}}    \triangleright   \mathsf{F} \NDmv{x_{{\mathrm{0}}}}   \NDsym{)}   \NDsym{)}   \NDsym{:}    \mathsf{F}  \mathsf{G} \NDnt{B}     \triangleright   \mathsf{F}  \mathsf{G} \NDnt{A}  }
          }{\NDmv{x_{{\mathrm{2}}}}  \NDsym{:}   \mathsf{F}  \mathsf{G} \NDnt{A}    \NDsym{,}  \NDmv{y_{{\mathrm{2}}}}  \NDsym{:}   \mathsf{F}  \mathsf{G} \NDnt{B}    \vdash_\mathcal{L}   \mathsf{let}\,  \mathsf{F} \NDmv{x_{{\mathrm{1}}}}   :   \mathsf{F}  \mathsf{G} \NDnt{A}   \,\mathsf{be}\, \NDmv{x_{{\mathrm{2}}}} \,\mathsf{in}\, \NDsym{(}   \mathsf{let}\,  \mathsf{F} \NDmv{y_{{\mathrm{1}}}}   :   \mathsf{F}  \mathsf{G} \NDnt{B}   \,\mathsf{be}\, \NDmv{y_{{\mathrm{2}}}} \,\mathsf{in}\, \NDsym{(}   \mathsf{ex}\, \NDmv{y_{{\mathrm{1}}}} , \NDmv{x_{{\mathrm{1}}}} \,\mathsf{with}\, \NDmv{y_{{\mathrm{0}}}} , \NDmv{x_{{\mathrm{0}}}} \,\mathsf{in}\, \NDsym{(}    \mathsf{F} \NDmv{y_{{\mathrm{0}}}}    \triangleright   \mathsf{F} \NDmv{x_{{\mathrm{0}}}}   \NDsym{)}   \NDsym{)}   \NDsym{)}   \NDsym{:}    \mathsf{F}  \mathsf{G} \NDnt{B}     \triangleright   \mathsf{F}  \mathsf{G} \NDnt{A}  }
        }{\NDmv{z}  \NDsym{:}    \mathsf{F}  \mathsf{G} \NDnt{A}     \triangleright   \mathsf{F}  \mathsf{G} \NDnt{B}    \vdash_\mathcal{L}   \mathsf{let}\, \NDmv{z}  :    \mathsf{F}  \mathsf{G} \NDnt{A}     \triangleright   \mathsf{F}  \mathsf{G} \NDnt{B}   \,\mathsf{be}\, \NDmv{x_{{\mathrm{2}}}}  \triangleright  \NDmv{y_{{\mathrm{2}}}} \,\mathsf{in}\, \NDsym{(}   \mathsf{let}\,  \mathsf{F} \NDmv{x_{{\mathrm{1}}}}   :   \mathsf{F}  \mathsf{G} \NDnt{A}   \,\mathsf{be}\, \NDmv{x_{{\mathrm{2}}}} \,\mathsf{in}\, \NDsym{(}   \mathsf{let}\,  \mathsf{F} \NDmv{y_{{\mathrm{1}}}}   :   \mathsf{F}  \mathsf{G} \NDnt{B}   \,\mathsf{be}\, \NDmv{y_{{\mathrm{2}}}} \,\mathsf{in}\, \NDsym{(}   \mathsf{ex}\, \NDmv{y_{{\mathrm{1}}}} , \NDmv{x_{{\mathrm{1}}}} \,\mathsf{with}\, \NDmv{y_{{\mathrm{0}}}} , \NDmv{x_{{\mathrm{0}}}} \,\mathsf{in}\, \NDsym{(}    \mathsf{F} \NDmv{y_{{\mathrm{0}}}}    \triangleright   \mathsf{F} \NDmv{x_{{\mathrm{0}}}}   \NDsym{)}   \NDsym{)}   \NDsym{)}   \NDsym{)}   \NDsym{:}    \mathsf{F}  \mathsf{G} \NDnt{B}     \triangleright   \mathsf{F}  \mathsf{G} \NDnt{A}  }
      }{ \cdot   \vdash_\mathcal{L}   \lambda_r  \NDmv{z}  :    \mathsf{F}  \mathsf{G} \NDnt{A}     \triangleright   \mathsf{F}  \mathsf{G} \NDnt{B}   .  \mathsf{let}\, \NDmv{z}  :    \mathsf{F}  \mathsf{G} \NDnt{A}     \triangleright   \mathsf{F}  \mathsf{G} \NDnt{B}   \,\mathsf{be}\, \NDmv{x_{{\mathrm{2}}}}  \triangleright  \NDmv{y_{{\mathrm{2}}}} \,\mathsf{in}\, \NDsym{(}   \mathsf{let}\,  \mathsf{F} \NDmv{x_{{\mathrm{1}}}}   :   \mathsf{F}  \mathsf{G} \NDnt{A}   \,\mathsf{be}\, \NDmv{x_{{\mathrm{2}}}} \,\mathsf{in}\, \NDsym{(}   \mathsf{let}\,  \mathsf{F} \NDmv{y_{{\mathrm{1}}}}   :   \mathsf{F}  \mathsf{G} \NDnt{B}   \,\mathsf{be}\, \NDmv{y_{{\mathrm{2}}}} \,\mathsf{in}\, \NDsym{(}   \mathsf{ex}\, \NDmv{y_{{\mathrm{1}}}} , \NDmv{x_{{\mathrm{1}}}} \,\mathsf{with}\, \NDmv{y_{{\mathrm{0}}}} , \NDmv{x_{{\mathrm{0}}}} \,\mathsf{in}\, \NDsym{(}    \mathsf{F} \NDmv{y_{{\mathrm{0}}}}    \triangleright   \mathsf{F} \NDmv{x_{{\mathrm{0}}}}   \NDsym{)}   \NDsym{)}   \NDsym{)}   \NDsym{)}    \NDsym{:}  \NDsym{(}    \mathsf{F}  \mathsf{G} \NDnt{A}     \triangleright   \mathsf{F}  \mathsf{G} \NDnt{B}    \NDsym{)}  \rightharpoonup  \NDsym{(}    \mathsf{F}  \mathsf{G} \NDnt{B}     \triangleright   \mathsf{F}  \mathsf{G} \NDnt{A}    \NDsym{)}}
  \end{math}
\end{center}

We also have the three cut rules derivable in the natural deduction:
(NOTE: Don't know how to prove the third one S\_cut2.)

\begin{figure}[!h]
  \scriptsize
  \begin{mathpar}
    \NDdruleTXXcut{} \qquad\qquad \NDdruleSXXcutOne{} \qquad\qquad \NDdruleSXXcutTwo{}
  \end{mathpar}
\end{figure}



%%%%%%%%%%%%%%%%%%%%%%%%%%%%%%%%%%%%%%%%%%%%%%%%%%
\subsubsection{One Step $\beta$-Reduction}

We define the normalization procedure by considering the following pairs of introduction and
elimination rules:

\begin{itemize}

\item (\NDdruleTXXunitIName, \NDdruleTXXunitEName):
  \begin{center}
    \tiny
    \begin{math}
      $$\mprset{flushleft}
      \inferrule* [right={\tiny unitE}] {
        $$\mprset{flushleft}
        \inferrule* [right={\tiny unitI}] {
          \,
        }{ \cdot   \vdash_\mathcal{C}   \mathsf{trivT}   \NDsym{:}   \mathsf{UnitT} } \\
         {\Phi  \vdash_\mathcal{C}  \NDnt{t}  \NDsym{:}  \NDnt{X}}
      }{\Phi  \vdash_\mathcal{C}   \mathsf{let}\,  \mathsf{trivT}   :   \mathsf{UnitT}  \,\mathsf{be}\,  \mathsf{trivT}  \,\mathsf{in}\, \NDnt{t}   \NDsym{:}  \NDnt{X}}
    \end{math}
  \end{center}
  normalizes to 
  \begin{center}
    \tiny
    $\Phi  \vdash_\mathcal{C}  \NDnt{t}  \NDsym{:}  \NDnt{X}$
  \end{center}

\item (\NDdruleTXXunitIName, \NDdruleSXXunitEOneName):
  \begin{center}
    \tiny
    \begin{math}
     $$\mprset{flushleft}
     \inferrule* [right={\tiny unitE2}] {
       $$\mprset{flushleft}
       \inferrule* [right={\tiny unitI}] {
         \,
        }{ \cdot   \vdash_\mathcal{C}   \mathsf{trivT}   \NDsym{:}   \mathsf{UnitT} } \\
         {\Delta  \vdash_\mathcal{L}  \NDnt{s}  \NDsym{:}  \NDnt{A}}
      }{\Delta  \vdash_\mathcal{L}   \mathsf{let}\,  \mathsf{trivT}   :   \mathsf{UnitT}  \,\mathsf{be}\,  \mathsf{trivT}  \,\mathsf{in}\, \NDnt{s}   \NDsym{:}  \NDnt{A}}
    \end{math}
  \end{center}
  normalizes to
  \begin{center}
    \tiny
    $\Delta  \vdash_\mathcal{L}  \NDnt{s}  \NDsym{:}  \NDnt{A}$
  \end{center}

\item (\NDdruleTXXtenIName, \NDdruleTXXtenEName):
  \begin{center}
    \tiny
    \begin{math}
      $$\mprset{flushleft}
      \inferrule* [right={\tiny tenE}] {
        $$\mprset{flushleft}
        \inferrule* [right={\tiny tenI}] {
          {\Phi_{{\mathrm{1}}}  \vdash_\mathcal{C}  \NDnt{t_{{\mathrm{1}}}}  \NDsym{:}  \NDnt{X}} \\
          {\Phi_{{\mathrm{2}}}  \vdash_\mathcal{C}  \NDnt{t_{{\mathrm{2}}}}  \NDsym{:}  \NDnt{Y}}
        }{\Phi_{{\mathrm{1}}}  \NDsym{,}  \Phi_{{\mathrm{2}}}  \vdash_\mathcal{C}  \NDnt{t_{{\mathrm{1}}}}  \otimes  \NDnt{t_{{\mathrm{2}}}}  \NDsym{:}  \NDnt{X}  \otimes  \NDnt{Y}} \\
         {\Psi_{{\mathrm{1}}}  \NDsym{,}  \NDmv{x}  \NDsym{:}  \NDnt{X}  \NDsym{,}  \NDmv{y}  \NDsym{:}  \NDnt{Y}  \NDsym{,}  \Psi_{{\mathrm{2}}}  \vdash_\mathcal{C}  \NDnt{t_{{\mathrm{3}}}}  \NDsym{:}  \NDnt{Z}}
      }{\Psi_{{\mathrm{1}}}  \NDsym{,}  \Phi_{{\mathrm{1}}}  \NDsym{,}  \Phi_{{\mathrm{2}}}  \NDsym{,}  \Psi_{{\mathrm{2}}}  \vdash_\mathcal{C}   \mathsf{let}\, \NDnt{t_{{\mathrm{1}}}}  \otimes  \NDnt{t_{{\mathrm{2}}}}  :  \NDnt{X}  \otimes  \NDnt{Y} \,\mathsf{be}\, \NDmv{x}  \otimes  \NDmv{y} \,\mathsf{in}\, \NDnt{t_{{\mathrm{3}}}}   \NDsym{:}  \NDnt{Z}}
    \end{math}
  \end{center}
  normalizes to
  \begin{center}
    \tiny
    \begin{math}
      $$\mprset{flushleft}
      \inferrule* [right={\tiny cut}] {
        {\Phi_{{\mathrm{1}}}  \vdash_\mathcal{C}  \NDnt{t_{{\mathrm{1}}}}  \NDsym{:}  \NDnt{X}} \\
        $$\mprset{flushleft}
        \inferrule* [right={\tiny cut}] {
          {\Phi_{{\mathrm{2}}}  \vdash_\mathcal{C}  \NDnt{t_{{\mathrm{2}}}}  \NDsym{:}  \NDnt{Y}} \\
          {\Psi_{{\mathrm{1}}}  \NDsym{,}  \NDmv{x}  \NDsym{:}  \NDnt{X}  \NDsym{,}  \NDmv{y}  \NDsym{:}  \NDnt{Y}  \NDsym{,}  \Psi_{{\mathrm{2}}}  \vdash_\mathcal{C}  \NDnt{t_{{\mathrm{3}}}}  \NDsym{:}  \NDnt{Z}}
        }{\Psi_{{\mathrm{1}}}  \NDsym{,}  \NDmv{x}  \NDsym{:}  \NDnt{X}  \NDsym{,}  \Phi_{{\mathrm{2}}}  \NDsym{,}  \Psi_{{\mathrm{2}}}  \vdash_\mathcal{C}  \NDsym{[}  \NDnt{t_{{\mathrm{2}}}}  \NDsym{/}  \NDmv{y}  \NDsym{]}  \NDnt{t_{{\mathrm{3}}}}  \NDsym{:}  \NDnt{Z}}
      }{\Psi_{{\mathrm{1}}}  \NDsym{,}  \Phi_{{\mathrm{1}}}  \NDsym{,}  \Phi_{{\mathrm{2}}}  \NDsym{,}  \Psi_{{\mathrm{2}}}  \vdash_\mathcal{C}  \NDsym{[}  \NDnt{t_{{\mathrm{1}}}}  \NDsym{/}  \NDmv{x}  \NDsym{]}  \NDsym{[}  \NDnt{t_{{\mathrm{2}}}}  \NDsym{/}  \NDmv{y}  \NDsym{]}  \NDnt{t_{{\mathrm{3}}}}  \NDsym{:}  \NDnt{Z}}
    \end{math}
  \end{center}
  
\item (\NDdruleTXXtenIName, \NDdruleSXXtenEOneName):
  \begin{center}
    \tiny
    \begin{math}
      $$\mprset{flushleft}
      \inferrule* [right={\tiny tenE1}] {
        $$\mprset{flushleft}
        \inferrule* [right={\tiny tenI}] {
          {\Phi  \vdash_\mathcal{C}  \NDnt{t_{{\mathrm{1}}}}  \NDsym{:}  \NDnt{X}} \\
          {\Psi  \vdash_\mathcal{C}  \NDnt{t_{{\mathrm{2}}}}  \NDsym{:}  \NDnt{Y}}
        }{\Phi  \NDsym{,}  \Psi  \vdash_\mathcal{C}  \NDnt{t_{{\mathrm{1}}}}  \otimes  \NDnt{t_{{\mathrm{2}}}}  \NDsym{:}  \NDnt{X}  \otimes  \NDnt{Y}} \\
         {\Gamma  \NDsym{,}  \NDmv{x}  \NDsym{:}  \NDnt{X}  \NDsym{,}  \NDmv{y}  \NDsym{:}  \NDnt{Y}  \NDsym{,}  \Delta  \vdash_\mathcal{L}  \NDnt{s}  \NDsym{:}  \NDnt{A}}
      }{\Gamma  \NDsym{,}  \Phi  \NDsym{,}  \Psi  \NDsym{,}  \Delta  \vdash_\mathcal{L}   \mathsf{let}\, \NDnt{t_{{\mathrm{1}}}}  \otimes  \NDnt{t_{{\mathrm{2}}}}  :  \NDnt{X}  \otimes  \NDnt{Y} \,\mathsf{be}\, \NDmv{x}  \otimes  \NDmv{y} \,\mathsf{in}\, \NDnt{s}   \NDsym{:}  \NDnt{A}}
    \end{math}
  \end{center}
  normalizes to
  \begin{center}
    \tiny
    \begin{math}
      $$\mprset{flushleft}
      \inferrule* [right={\tiny cut2}] {
        {\Phi  \vdash_\mathcal{C}  \NDnt{t_{{\mathrm{1}}}}  \NDsym{:}  \NDnt{X}} \\
        $$\mprset{flushleft}
        \inferrule* [right={\tiny cut2}] {
          {\Psi  \vdash_\mathcal{C}  \NDnt{t_{{\mathrm{2}}}}  \NDsym{:}  \NDnt{Y}} \\
          {\Gamma  \NDsym{,}  \NDmv{x}  \NDsym{:}  \NDnt{X}  \NDsym{,}  \NDmv{y}  \NDsym{:}  \NDnt{Y}  \NDsym{,}  \Delta  \vdash_\mathcal{L}  \NDnt{s}  \NDsym{:}  \NDnt{A}}
        }{\Gamma  \NDsym{,}  \NDmv{x}  \NDsym{:}  \NDnt{X}  \NDsym{,}  \Psi  \NDsym{,}  \Delta  \vdash_\mathcal{L}  \NDsym{[}  \NDnt{t_{{\mathrm{2}}}}  \NDsym{/}  \NDmv{y}  \NDsym{]}  \NDnt{s}  \NDsym{:}  \NDnt{A}}
      }{\Gamma  \NDsym{,}  \Phi  \NDsym{,}  \Psi  \NDsym{,}  \Delta  \vdash_\mathcal{L}  \NDsym{[}  \NDnt{t_{{\mathrm{1}}}}  \NDsym{/}  \NDmv{x}  \NDsym{]}  \NDsym{[}  \NDnt{t_{{\mathrm{2}}}}  \NDsym{/}  \NDmv{y}  \NDsym{]}  \NDnt{s}  \NDsym{:}  \NDnt{A}}
    \end{math}
  \end{center}
  
\item (\NDdruleTXXimpIName, \NDdruleTXXimpEName):
  \begin{center}
    \tiny
    \begin{math}
      $$\mprset{flushleft}
      \inferrule* [right={\tiny impE}] {
        $$\mprset{flushleft}
        \inferrule* [right={\tiny impI}] {
          {\Phi  \NDsym{,}  \NDmv{x}  \NDsym{:}  \NDnt{X}  \vdash_\mathcal{C}  \NDnt{t_{{\mathrm{1}}}}  \NDsym{:}  \NDnt{Y}}
        }{\Phi  \vdash_\mathcal{C}   \lambda  \NDmv{x}  :  \NDnt{X} . \NDnt{t_{{\mathrm{1}}}}   \NDsym{:}  \NDnt{X}  \multimap  \NDnt{Y}} \\
         {\Psi  \vdash_\mathcal{C}  \NDnt{t_{{\mathrm{2}}}}  \NDsym{:}  \NDnt{X}}
      }{\Phi  \NDsym{,}  \Psi  \vdash_\mathcal{C}   \mathsf{app}\, \NDsym{(}   \lambda  \NDmv{x}  :  \NDnt{X} . \NDnt{t_{{\mathrm{1}}}}   \NDsym{)} \, \NDnt{t_{{\mathrm{2}}}}   \NDsym{:}  \NDnt{Y}}
    \end{math}
  \end{center}
  normalizes to
  \begin{center}
    \tiny
    \begin{math}
      $$\mprset{flushleft}
      \inferrule* [right={\tiny cut}] {
        {\Phi  \NDsym{,}  \NDmv{x}  \NDsym{:}  \NDnt{X}  \vdash_\mathcal{C}  \NDnt{t_{{\mathrm{1}}}}  \NDsym{:}  \NDnt{Y}} \\
        {\Psi  \vdash_\mathcal{C}  \NDnt{t_{{\mathrm{2}}}}  \NDsym{:}  \NDnt{X}}
      }{\Phi  \NDsym{,}  \Psi  \vdash_\mathcal{C}  \NDsym{[}  \NDnt{t_{{\mathrm{2}}}}  \NDsym{/}  \NDmv{x}  \NDsym{]}  \NDnt{t_{{\mathrm{1}}}}  \NDsym{:}  \NDnt{Y}}
    \end{math}
  \end{center}

\item (\NDdruleSXXunitIName, \NDdruleSXXunitETwoName):
  \begin{center}
    \tiny
    \begin{math}
     $$\mprset{flushleft}
     \inferrule* [right={\tiny unitE2}] {
       $$\mprset{flushleft}
       \inferrule* [right={\tiny unitI}] {
         \,
        }{ \cdot   \vdash_\mathcal{L}   \mathsf{trivS}   \NDsym{:}   \mathsf{UnitS} } \\
         {\Delta  \vdash_\mathcal{L}  \NDnt{s}  \NDsym{:}  \NDnt{A}}
      }{\Delta  \vdash_\mathcal{L}   \mathsf{let}\,  \mathsf{trivS}   :   \mathsf{UnitS}  \,\mathsf{be}\,  \mathsf{trivS}  \,\mathsf{in}\, \NDnt{s}   \NDsym{:}  \NDnt{A}}
    \end{math}
  \end{center}
  normalizes to
  \begin{center}
    \tiny
    $\Delta  \vdash_\mathcal{L}  \NDnt{s}  \NDsym{:}  \NDnt{A}$
  \end{center}

\item (\NDdruleSXXtenIName, \NDdruleSXXtenETwoName):
  \begin{center}
    \tiny
    \begin{math}
     $$\mprset{flushleft}
     \inferrule* [right={\tiny tenE2}] {
       $$\mprset{flushleft}
       \inferrule* [right={\tiny tenI}] {
         {\Gamma_{{\mathrm{1}}}  \vdash_\mathcal{L}  \NDnt{s_{{\mathrm{1}}}}  \NDsym{:}  \NDnt{A}} \\
         {\Gamma_{{\mathrm{2}}}  \vdash_\mathcal{L}  \NDnt{s_{{\mathrm{2}}}}  \NDsym{:}  \NDnt{B}}
        }{\Gamma_{{\mathrm{1}}}  \NDsym{,}  \Gamma_{{\mathrm{2}}}  \vdash_\mathcal{L}  \NDnt{s_{{\mathrm{1}}}}  \triangleright  \NDnt{s_{{\mathrm{2}}}}  \NDsym{:}  \NDnt{A}  \triangleright  \NDnt{B}} \\
         {\Delta_{{\mathrm{1}}}  \NDsym{,}  \NDmv{x}  \NDsym{:}  \NDnt{A}  \NDsym{,}  \NDmv{y}  \NDsym{:}  \NDnt{B}  \NDsym{,}  \Delta_{{\mathrm{2}}}  \vdash_\mathcal{L}  \NDnt{s_{{\mathrm{3}}}}  \NDsym{:}  \NDnt{C}}
      }{\Delta_{{\mathrm{1}}}  \NDsym{,}  \Gamma_{{\mathrm{1}}}  \NDsym{,}  \Gamma_{{\mathrm{2}}}  \NDsym{,}  \Delta_{{\mathrm{2}}}  \vdash_\mathcal{L}   \mathsf{let}\, \NDnt{s_{{\mathrm{1}}}}  \triangleright  \NDnt{s_{{\mathrm{2}}}}  :  \NDnt{A}  \triangleright  \NDnt{B} \,\mathsf{be}\, \NDmv{x}  \triangleright  \NDmv{y} \,\mathsf{in}\, \NDnt{s_{{\mathrm{3}}}}   \NDsym{:}  \NDnt{C}}
    \end{math}
  \end{center}
  normalizes to
  \begin{center}
    \tiny
    \begin{math}
      $$\mprset{flushleft}
      \inferrule* [right={\tiny cut2}] {
        {\Gamma_{{\mathrm{1}}}  \vdash_\mathcal{L}  \NDnt{s_{{\mathrm{1}}}}  \NDsym{:}  \NDnt{A}} \\
        $$\mprset{flushleft}
        \inferrule* [right={\tiny cut2}] {
          {\Gamma_{{\mathrm{2}}}  \vdash_\mathcal{L}  \NDnt{s_{{\mathrm{2}}}}  \NDsym{:}  \NDnt{B}} \\
          {\Delta_{{\mathrm{1}}}  \NDsym{,}  \NDmv{x}  \NDsym{:}  \NDnt{X}  \NDsym{,}  \NDmv{y}  \NDsym{:}  \NDnt{Y}  \NDsym{,}  \Delta_{{\mathrm{2}}}  \vdash_\mathcal{L}  \NDnt{s_{{\mathrm{3}}}}  \NDsym{:}  \NDnt{C}}
        }{\Delta_{{\mathrm{1}}}  \NDsym{,}  \NDmv{x}  \NDsym{:}  \NDnt{X}  \NDsym{,}  \Gamma_{{\mathrm{2}}}  \NDsym{,}  \Delta_{{\mathrm{2}}}  \vdash_\mathcal{L}  \NDsym{[}  \NDnt{s_{{\mathrm{2}}}}  \NDsym{/}  \NDmv{y}  \NDsym{]}  \NDnt{s_{{\mathrm{3}}}}  \NDsym{:}  \NDnt{C}}
      }{\Delta_{{\mathrm{1}}}  \NDsym{,}  \Gamma_{{\mathrm{1}}}  \NDsym{,}  \Gamma_{{\mathrm{2}}}  \NDsym{,}  \Delta_{{\mathrm{2}}}  \vdash_\mathcal{L}  \NDsym{[}  \NDnt{s_{{\mathrm{1}}}}  \NDsym{/}  \NDmv{x}  \NDsym{]}  \NDsym{[}  \NDnt{s_{{\mathrm{2}}}}  \NDsym{/}  \NDmv{y}  \NDsym{]}  \NDnt{s_{{\mathrm{3}}}}  \NDsym{:}  \NDnt{C}}
    \end{math}
  \end{center}
        
\item (\NDdruleSXXimprIName, \NDdruleSXXimprEName):
  \begin{center}
    \tiny
    \begin{math}
     $$\mprset{flushleft}
     \inferrule* [right={\tiny unitE2}] {
       $$\mprset{flushleft}
       \inferrule* [right={\tiny imprI}] {
         {\Gamma  \NDsym{,}  \NDmv{x}  \NDsym{:}  \NDnt{A}  \vdash_\mathcal{L}  \NDnt{s_{{\mathrm{1}}}}  \NDsym{:}  \NDnt{B}}
        }{\Gamma  \vdash_\mathcal{L}   \lambda_r  \NDmv{x}  :  \NDnt{A} . \NDnt{s_{{\mathrm{1}}}}   \NDsym{:}  \NDnt{A}  \rightharpoonup  \NDnt{B}} \\
         {\Delta  \vdash_\mathcal{L}  \NDnt{s_{{\mathrm{2}}}}  \NDsym{:}  \NDnt{A}}
      }{\Gamma  \NDsym{,}  \Delta  \vdash_\mathcal{L}   \mathsf{app}_r\, \NDsym{(}   \lambda_r  \NDmv{x}  :  \NDnt{A} . \NDnt{s_{{\mathrm{1}}}}   \NDsym{)} \, \NDnt{s_{{\mathrm{2}}}}   \NDsym{:}  \NDnt{B}}
    \end{math}
  \end{center}
  normalizes to
  \begin{center}
    \tiny
    \begin{math}
      $$\mprset{flushleft}
      \inferrule* [right={\tiny cut2}] {
        {\Gamma  \NDsym{,}  \NDmv{x}  \NDsym{:}  \NDnt{A}  \vdash_\mathcal{L}  \NDnt{s_{{\mathrm{1}}}}  \NDsym{:}  \NDnt{B}} \\
        {\Delta  \vdash_\mathcal{L}  \NDnt{s_{{\mathrm{2}}}}  \NDsym{:}  \NDnt{A}}
      }{\Gamma  \NDsym{,}  \Delta  \vdash_\mathcal{L}  \NDsym{[}  \NDnt{s_{{\mathrm{2}}}}  \NDsym{/}  \NDmv{x}  \NDsym{]}  \NDnt{s_{{\mathrm{1}}}}  \NDsym{:}  \NDnt{B}}
    \end{math}
  \end{center}
        
\item (\NDdruleSXXimplIName, \NDdruleSXXimplEName):
  \begin{center}
    \tiny
    \begin{math}
     $$\mprset{flushleft}
     \inferrule* [right={\tiny unitE2}] {
       $$\mprset{flushleft}
       \inferrule* [right={\tiny implI}] {
         {\NDmv{x}  \NDsym{:}  \NDnt{A}  \NDsym{,}  \Gamma  \vdash_\mathcal{L}  \NDnt{s_{{\mathrm{1}}}}  \NDsym{:}  \NDnt{B}}
        }{\Gamma  \vdash_\mathcal{L}   \lambda_l  \NDmv{x}  :  \NDnt{A} . \NDnt{s_{{\mathrm{1}}}}   \NDsym{:}  \NDnt{B}  \leftharpoonup  \NDnt{A}} \\
         {\Delta  \vdash_\mathcal{L}  \NDnt{s_{{\mathrm{2}}}}  \NDsym{:}  \NDnt{A}}
      }{\Delta  \NDsym{,}  \Gamma  \vdash_\mathcal{L}   \mathsf{app}_l\, \NDsym{(}   \lambda_l  \NDmv{x}  :  \NDnt{A} . \NDnt{s_{{\mathrm{1}}}}   \NDsym{)} \, \NDnt{s_{{\mathrm{2}}}}   \NDsym{:}  \NDnt{B}}
    \end{math}
  \end{center}
  normalizes to
  \begin{center}
    \tiny
    \begin{math}
      $$\mprset{flushleft}
      \inferrule* [right={\tiny cut2}] {
        {\NDmv{x}  \NDsym{:}  \NDnt{A}  \NDsym{,}  \Gamma  \vdash_\mathcal{L}  \NDnt{s_{{\mathrm{1}}}}  \NDsym{:}  \NDnt{B}} \\
        {\Delta  \vdash_\mathcal{L}  \NDnt{s_{{\mathrm{2}}}}  \NDsym{:}  \NDnt{A}}
      }{\Delta  \NDsym{,}  \Gamma  \vdash_\mathcal{L}  \NDsym{[}  \NDnt{s_{{\mathrm{2}}}}  \NDsym{/}  \NDmv{x}  \NDsym{]}  \NDnt{s_{{\mathrm{1}}}}  \NDsym{:}  \NDnt{B}}
    \end{math}
  \end{center}
        
\item (\NDdruleSXXFIName, \NDdruleSXXFEName):
  \begin{center}
    \tiny
    \begin{math}
      $$\mprset{flushleft}
      \inferrule* [right={\tiny FE}] {
        $$\mprset{flushleft}
        \inferrule* [right={\tiny FI}] {
          {\Phi  \vdash_\mathcal{C}  \NDmv{y}  \NDsym{:}  \NDnt{X}}
        }{\Phi  \vdash_\mathcal{L}   \mathsf{F} \NDmv{y}   \NDsym{:}   \mathsf{F} \NDnt{X} } \\
         {\Delta_{{\mathrm{1}}}  \NDsym{,}  \NDmv{x}  \NDsym{:}  \NDnt{X}  \NDsym{,}  \Delta_{{\mathrm{2}}}  \vdash_\mathcal{L}  \NDnt{s}  \NDsym{:}  \NDnt{A}}
      }{\Delta_{{\mathrm{1}}}  \NDsym{,}  \Phi  \NDsym{,}  \Delta_{{\mathrm{2}}}  \vdash_\mathcal{L}   \mathsf{let}\,  \mathsf{F} \NDmv{x}   :   \mathsf{F} \NDnt{X}  \,\mathsf{be}\,  \mathsf{F}\, \NDmv{y}  \,\mathsf{in}\, \NDnt{s}   \NDsym{:}  \NDnt{A}}
    \end{math}
  \end{center}
  normalizes to
  \begin{center}
    \tiny
    \begin{math}
      $$\mprset{flushleft}
      \inferrule* [right={\tiny cut1}] {
        {\Phi  \vdash_\mathcal{C}  \NDmv{y}  \NDsym{:}  \NDnt{X}} \\
        {\Delta_{{\mathrm{1}}}  \NDsym{,}  \NDmv{x}  \NDsym{:}  \NDnt{X}  \NDsym{,}  \Delta_{{\mathrm{2}}}  \vdash_\mathcal{L}  \NDnt{s}  \NDsym{:}  \NDnt{A}}
      }{\Delta_{{\mathrm{1}}}  \NDsym{,}  \Phi  \NDsym{,}  \Delta_{{\mathrm{2}}}  \vdash_\mathcal{L}  \NDsym{[}  \NDmv{y}  \NDsym{/}  \NDmv{x}  \NDsym{]}  \NDnt{s}  \NDsym{:}  \NDnt{A}}
    \end{math}
  \end{center}

\item (\NDdruleTXXGIName, \NDdruleSXXGEName):
  \begin{center}
    \tiny
    \begin{math}
      $$\mprset{flushleft}
      \inferrule* [right={\tiny GE}] {
        $$\mprset{flushleft}
        \inferrule* [right={\tiny GI}] {
          {\Phi  \vdash_\mathcal{L}  \NDnt{s}  \NDsym{:}  \NDnt{A}}
        }{\Phi  \vdash_\mathcal{C}   \mathsf{G} \NDnt{s}   \NDsym{:}   \mathsf{G} \NDnt{A} }
      }{\Phi  \vdash_\mathcal{L}   \mathsf{derelict}\, \NDsym{(}   \mathsf{G} \NDnt{s}   \NDsym{)}   \NDsym{:}  \NDnt{A}}
    \end{math}
  \end{center}
  normalizes to
  \begin{center}
    \tiny
    $\Phi  \vdash_\mathcal{L}  \NDnt{s}  \NDsym{:}  \NDnt{A}$
  \end{center}

\end{itemize}

\begin{theorem}[Normalization]
  For a cut-free deduction $\Pi$, there is a deduction which is in normal form.
\end{theorem}
\begin{proof}
  By induction on the structure of $\Pi$.
\end{proof}



  \begin{center}
    \tiny
    $\texttt{\textcolor{red}{\texttt{\textcolor{red}{<<no parses (char 2): no*** parses (char 2): Th***eta \mbox{$\backslash{}$}mbox\{\$\mbox{$\backslash{}$}mid\$\}-c X  >>}}}}$
  \end{center}



%%%%%%%%%%%%%%%%%%%%%%%%%%%%%%%%%%%%%%%%%%%%%%%%%%
\subsubsection{Commuting Conversions}

\begin{itemize}

\item Commutation of $\mathrm{UnitT}_E$:
  \begin{itemize}

  \item (\NDdruleTXXunitEName, \NDdruleTXXunitEName):
    \begin{center}
      \tiny
      \begin{math}
        $$\mprset{flushleft}
        \inferrule* [right={\tiny unitE}] {
          $$\mprset{flushleft}
          \inferrule* [right={\tiny unitE}] {
            {\Phi_{{\mathrm{1}}}  \vdash_\mathcal{C}  \NDnt{t_{{\mathrm{1}}}}  \NDsym{:}   \mathsf{UnitT} } \\
            {\Phi_{{\mathrm{2}}}  \vdash_\mathcal{C}  \NDnt{t_{{\mathrm{2}}}}  \NDsym{:}   \mathsf{UnitT} }
          }{\Phi_{{\mathrm{2}}}  \NDsym{,}  \Phi_{{\mathrm{1}}}  \vdash_\mathcal{C}   \mathsf{let}\, \NDnt{t_{{\mathrm{2}}}}  :   \mathsf{UnitT}  \,\mathsf{be}\,  \mathsf{trivT}  \,\mathsf{in}\, \NDnt{t_{{\mathrm{1}}}}   \NDsym{:}   \mathsf{UnitT} } \\
          {\Phi_{{\mathrm{3}}}  \vdash_\mathcal{C}  \NDnt{t_{{\mathrm{3}}}}  \NDsym{:}  \NDnt{X}}
        }{\Phi_{{\mathrm{2}}}  \NDsym{,}  \Phi_{{\mathrm{1}}}  \NDsym{,}  \Phi_{{\mathrm{3}}}  \vdash_\mathcal{C}   \mathsf{let}\, \NDsym{(}   \mathsf{let}\, \NDnt{t_{{\mathrm{2}}}}  :   \mathsf{UnitT}  \,\mathsf{be}\,  \mathsf{trivT}  \,\mathsf{in}\, \NDnt{t_{{\mathrm{1}}}}   \NDsym{)}  :   \mathsf{UnitT}  \,\mathsf{be}\,  \mathsf{trivT}  \,\mathsf{in}\, \NDnt{t_{{\mathrm{3}}}}   \NDsym{:}  \NDnt{X}}
      \end{math}
    \end{center}
    commutes to
    \begin{center}
      \tiny
      \begin{math}
        $$\mprset{flushleft}
        \inferrule* [right={\tiny unitE}] {
          $$\mprset{flushleft}
          \inferrule* [right={\tiny unitE}] {
            {\Phi_{{\mathrm{1}}}  \vdash_\mathcal{C}  \NDnt{t_{{\mathrm{1}}}}  \NDsym{:}   \mathsf{UnitT} } \\
            {\Phi_{{\mathrm{3}}}  \vdash_\mathcal{C}  \NDnt{t_{{\mathrm{3}}}}  \NDsym{:}  \NDnt{X}}
          }{\Phi_{{\mathrm{1}}}  \NDsym{,}  \Phi_{{\mathrm{3}}}  \vdash_\mathcal{C}   \mathsf{let}\, \NDnt{t_{{\mathrm{1}}}}  :   \mathsf{UnitT}  \,\mathsf{be}\,  \mathsf{trivT}  \,\mathsf{in}\, \NDnt{t_{{\mathrm{3}}}}   \NDsym{:}  \NDnt{X}} \\
           {\Phi_{{\mathrm{2}}}  \vdash_\mathcal{C}  \NDnt{t_{{\mathrm{2}}}}  \NDsym{:}   \mathsf{UnitT} }
        }{\Phi_{{\mathrm{2}}}  \NDsym{,}  \Phi_{{\mathrm{1}}}  \NDsym{,}  \Phi_{{\mathrm{3}}}  \vdash_\mathcal{C}   \mathsf{let}\, \NDnt{t_{{\mathrm{2}}}}  :   \mathsf{UnitT}  \,\mathsf{be}\,  \mathsf{trivT}  \,\mathsf{in}\, \NDsym{(}   \mathsf{let}\, \NDnt{t_{{\mathrm{1}}}}  :   \mathsf{UnitT}  \,\mathsf{be}\,  \mathsf{trivT}  \,\mathsf{in}\, \NDnt{t_{{\mathrm{3}}}}   \NDsym{)}   \NDsym{:}  \NDnt{X}}
      \end{math}
    \end{center}

  \item (\NDdruleTXXunitEName, \NDdruleTXXtenEName) need multiple exchanges at the end:
    \begin{center}
      \tiny
      \begin{math}
        $$\mprset{flushleft}
        \inferrule* [right={\tiny tenE}] {
          $$\mprset{flushleft}
          \inferrule* [right={\tiny unitE}] {
            {\Phi_{{\mathrm{1}}}  \vdash_\mathcal{C}  \NDnt{t_{{\mathrm{1}}}}  \NDsym{:}  \NDnt{X}  \otimes  \NDnt{Y}} \\
            {\Phi_{{\mathrm{2}}}  \vdash_\mathcal{C}  \NDnt{t_{{\mathrm{2}}}}  \NDsym{:}   \mathsf{UnitT} }
          }{\Phi_{{\mathrm{2}}}  \NDsym{,}  \Phi_{{\mathrm{1}}}  \vdash_\mathcal{C}   \mathsf{let}\, \NDnt{t_{{\mathrm{2}}}}  :   \mathsf{UnitT}  \,\mathsf{be}\,  \mathsf{trivT}  \,\mathsf{in}\, \NDnt{t_{{\mathrm{1}}}}   \NDsym{:}  \NDnt{X}  \otimes  \NDnt{Y}} \\
           {\Psi_{{\mathrm{1}}}  \NDsym{,}  \NDmv{x}  \NDsym{:}  \NDnt{X}  \NDsym{,}  \NDmv{y}  \NDsym{:}  \NDnt{Y}  \NDsym{,}  \Psi_{{\mathrm{2}}}  \vdash_\mathcal{C}  \NDnt{t_{{\mathrm{3}}}}  \NDsym{:}  \NDnt{Z}}
        }{\Psi_{{\mathrm{1}}}  \NDsym{,}  \Phi_{{\mathrm{2}}}  \NDsym{,}  \Phi_{{\mathrm{1}}}  \NDsym{,}  \Psi_{{\mathrm{2}}}  \vdash_\mathcal{C}   \mathsf{let}\, \NDsym{(}   \mathsf{let}\, \NDnt{t_{{\mathrm{2}}}}  :   \mathsf{UnitT}  \,\mathsf{be}\,  \mathsf{trivT}  \,\mathsf{in}\, \NDnt{t_{{\mathrm{1}}}}   \NDsym{)}  :  \NDnt{X}  \otimes  \NDnt{Y} \,\mathsf{be}\, \NDmv{x}  \otimes  \NDmv{y} \,\mathsf{in}\, \NDnt{t_{{\mathrm{3}}}}   \NDsym{:}  \NDnt{Z}}
      \end{math}
    \end{center}
    commutes to
    \begin{center}
      \tiny
      \begin{math}
        $$\mprset{flushleft}
        \inferrule* [right={\tiny unitE}] {
          $$\mprset{flushleft}
          \inferrule* [right={\tiny tenE}] {
            {\Phi_{{\mathrm{1}}}  \vdash_\mathcal{C}  \NDnt{t_{{\mathrm{1}}}}  \NDsym{:}  \NDnt{X}  \otimes  \NDnt{Y}} \\
            {\Psi_{{\mathrm{1}}}  \NDsym{,}  \NDmv{x}  \NDsym{:}  \NDnt{X}  \NDsym{,}  \NDmv{y}  \NDsym{:}  \NDnt{Y}  \NDsym{,}  \Psi_{{\mathrm{2}}}  \vdash_\mathcal{C}  \NDnt{t_{{\mathrm{3}}}}  \NDsym{:}  \NDnt{Z}}
          }{\Psi_{{\mathrm{1}}}  \NDsym{,}  \Phi_{{\mathrm{1}}}  \NDsym{,}  \Psi_{{\mathrm{2}}}  \vdash_\mathcal{C}   \mathsf{let}\, \NDnt{t_{{\mathrm{1}}}}  :  \NDnt{X}  \otimes  \NDnt{Y} \,\mathsf{be}\, \NDmv{x}  \otimes  \NDmv{y} \,\mathsf{in}\, \NDnt{t_{{\mathrm{3}}}}   \NDsym{:}  \NDnt{Z}} \\
           {\Phi_{{\mathrm{2}}}  \vdash_\mathcal{C}  \NDnt{t_{{\mathrm{2}}}}  \NDsym{:}   \mathsf{UnitT} }
        }{\Phi_{{\mathrm{2}}}  \NDsym{,}  \Psi_{{\mathrm{1}}}  \NDsym{,}  \Phi_{{\mathrm{1}}}  \NDsym{,}  \Psi_{{\mathrm{2}}}  \vdash_\mathcal{C}   \mathsf{let}\, \NDnt{t_{{\mathrm{2}}}}  :   \mathsf{UnitT}  \,\mathsf{be}\,  \mathsf{trivT}  \,\mathsf{in}\, \NDsym{(}   \mathsf{let}\, \NDnt{t_{{\mathrm{1}}}}  :  \NDnt{X}  \otimes  \NDnt{Y} \,\mathsf{be}\, \NDmv{x}  \otimes  \NDmv{y} \,\mathsf{in}\, \NDnt{t_{{\mathrm{3}}}}   \NDsym{)}   \NDsym{:}  \NDnt{Z}}
      \end{math}
    \end{center}

  \item (\NDdruleTXXunitEName, \NDdruleTXXimpEName):
    \begin{center}
      \tiny
      \begin{math}
        $$\mprset{flushleft}
        \inferrule* [right={\tiny unitE}] {
          $$\mprset{flushleft}
          \inferrule* [right={\tiny tenE}] {
            {\Phi_{{\mathrm{1}}}  \vdash_\mathcal{C}  \NDnt{t_{{\mathrm{1}}}}  \NDsym{:}  \NDnt{X}  \multimap  \NDnt{Y}} \\
            {\Phi_{{\mathrm{2}}}  \vdash_\mathcal{C}  \NDnt{t_{{\mathrm{2}}}}  \NDsym{:}   \mathsf{UnitT} }
          }{\Phi_{{\mathrm{2}}}  \NDsym{,}  \Phi_{{\mathrm{1}}}  \vdash_\mathcal{C}   \mathsf{let}\, \NDnt{t_{{\mathrm{2}}}}  :   \mathsf{UnitT}  \,\mathsf{be}\,  \mathsf{trivT}  \,\mathsf{in}\, \NDnt{t_{{\mathrm{1}}}}   \NDsym{:}  \NDnt{X}  \multimap  \NDnt{Y}} \\
           {\Phi_{{\mathrm{3}}}  \vdash_\mathcal{C}  \NDnt{t_{{\mathrm{3}}}}  \NDsym{:}  \NDnt{X}}
        }{\Phi_{{\mathrm{2}}}  \NDsym{,}  \Phi_{{\mathrm{1}}}  \NDsym{,}  \Phi_{{\mathrm{3}}}  \vdash_\mathcal{C}   \mathsf{app}\, \NDsym{(}   \mathsf{let}\, \NDnt{t_{{\mathrm{2}}}}  :   \mathsf{UnitT}  \,\mathsf{be}\,  \mathsf{trivT}  \,\mathsf{in}\, \NDnt{t_{{\mathrm{1}}}}   \NDsym{)} \, \NDnt{t_{{\mathrm{3}}}}   \NDsym{:}  \NDnt{Y}}
      \end{math}
    \end{center}
    commutes to
    \begin{center}
      \tiny
      \begin{math}
        $$\mprset{flushleft}
        \inferrule* [right={\tiny tenE}] {
          $$\mprset{flushleft}
          \inferrule* [right={\tiny unitE}] {
            {\Phi_{{\mathrm{1}}}  \vdash_\mathcal{C}  \NDnt{t_{{\mathrm{1}}}}  \NDsym{:}  \NDnt{X}  \multimap  \NDnt{Y}} \\
            {\Phi_{{\mathrm{3}}}  \vdash_\mathcal{C}  \NDnt{t_{{\mathrm{3}}}}  \NDsym{:}  \NDnt{X}}
          }{\Phi_{{\mathrm{1}}}  \NDsym{,}  \Phi_{{\mathrm{3}}}  \vdash_\mathcal{C}   \mathsf{app}\, \NDnt{t_{{\mathrm{1}}}} \, \NDnt{t_{{\mathrm{3}}}}   \NDsym{:}  \NDnt{Y}} \\
           {\Phi_{{\mathrm{2}}}  \vdash_\mathcal{C}  \NDnt{t_{{\mathrm{2}}}}  \NDsym{:}   \mathsf{UnitT} }
        }{\Phi_{{\mathrm{2}}}  \NDsym{,}  \Phi_{{\mathrm{1}}}  \NDsym{,}  \Phi_{{\mathrm{3}}}  \vdash_\mathcal{C}   \mathsf{let}\, \NDnt{t_{{\mathrm{2}}}}  :   \mathsf{UnitT}  \,\mathsf{be}\,  \mathsf{trivT}  \,\mathsf{in}\, \NDsym{(}   \mathsf{app}\, \NDnt{t_{{\mathrm{1}}}} \, \NDnt{t_{{\mathrm{3}}}}   \NDsym{)}   \NDsym{:}  \NDnt{Y}}
      \end{math}
    \end{center}
  \end{itemize}


\item Commutation of $\otimes_E$:

  \begin{itemize}
  \item (\NDdruleTXXtenEName, \NDdruleTXXunitEName):
    \begin{center}
      \tiny
      \begin{math}
        $$\mprset{flushleft}
        \inferrule* [right={\tiny unitE}] {
          $$\mprset{flushleft}
          \inferrule* [right={\tiny tenE}] {
            {\Phi_{{\mathrm{1}}}  \NDsym{,}  \NDmv{x}  \NDsym{:}  \NDnt{X}  \NDsym{,}  \NDmv{y}  \NDsym{:}  \NDnt{Y}  \NDsym{,}  \Phi_{{\mathrm{2}}}  \vdash_\mathcal{C}  \NDnt{t_{{\mathrm{1}}}}  \NDsym{:}   \mathsf{UnitT} } \\
            {\Psi_{{\mathrm{1}}}  \vdash_\mathcal{C}  \NDnt{t_{{\mathrm{2}}}}  \NDsym{:}  \NDnt{X}  \otimes  \NDnt{Y}}
          }{\Phi_{{\mathrm{1}}}  \NDsym{,}  \Psi_{{\mathrm{1}}}  \NDsym{,}  \Phi_{{\mathrm{2}}}  \vdash_\mathcal{C}   \mathsf{let}\, \NDnt{t_{{\mathrm{2}}}}  :  \NDnt{X}  \otimes  \NDnt{Y} \,\mathsf{be}\, \NDmv{x}  \otimes  \NDmv{y} \,\mathsf{in}\, \NDnt{t_{{\mathrm{1}}}}   \NDsym{:}   \mathsf{UnitT} } \\
           {\Psi_{{\mathrm{2}}}  \vdash_\mathcal{C}  \NDnt{t_{{\mathrm{3}}}}  \NDsym{:}  \NDnt{Z}}
        }{\Phi_{{\mathrm{1}}}  \NDsym{,}  \Psi_{{\mathrm{1}}}  \NDsym{,}  \Phi_{{\mathrm{2}}}  \NDsym{,}  \Psi_{{\mathrm{2}}}  \vdash_\mathcal{C}   \mathsf{let}\, \NDsym{(}   \mathsf{let}\, \NDnt{t_{{\mathrm{2}}}}  :  \NDnt{X}  \otimes  \NDnt{Y} \,\mathsf{be}\, \NDmv{x}  \otimes  \NDmv{y} \,\mathsf{in}\, \NDnt{t_{{\mathrm{1}}}}   \NDsym{)}  :   \mathsf{UnitT}  \,\mathsf{be}\,  \mathsf{trivT}  \,\mathsf{in}\, \NDnt{t_{{\mathrm{3}}}}   \NDsym{:}  \NDnt{Z}}
      \end{math}
    \end{center}
    commutes to
    \begin{center}
      \tiny
      \begin{math}
        $$\mprset{flushleft}
        \inferrule* [right={\tiny tenE}] {
          $$\mprset{flushleft}
          \inferrule* [right={\tiny unitE}] {
            {\Phi_{{\mathrm{1}}}  \NDsym{,}  \NDmv{x}  \NDsym{:}  \NDnt{X}  \NDsym{,}  \NDmv{y}  \NDsym{:}  \NDnt{Y}  \NDsym{,}  \Phi_{{\mathrm{2}}}  \vdash_\mathcal{C}  \NDnt{t_{{\mathrm{1}}}}  \NDsym{:}   \mathsf{UnitT} } \\
            {\Psi_{{\mathrm{2}}}  \vdash_\mathcal{C}  \NDnt{t_{{\mathrm{3}}}}  \NDsym{:}  \NDnt{Z}}
          }{\Phi_{{\mathrm{1}}}  \NDsym{,}  \NDmv{x}  \NDsym{:}  \NDnt{X}  \NDsym{,}  \NDmv{y}  \NDsym{:}  \NDnt{Y}  \NDsym{,}  \Phi_{{\mathrm{2}}}  \NDsym{,}  \Psi_{{\mathrm{2}}}  \vdash_\mathcal{C}   \mathsf{let}\, \NDnt{t_{{\mathrm{1}}}}  :   \mathsf{UnitT}  \,\mathsf{be}\,  \mathsf{trivT}  \,\mathsf{in}\, \NDnt{t_{{\mathrm{3}}}}   \NDsym{:}  \NDnt{Z}} \\
           {\Psi_{{\mathrm{1}}}  \vdash_\mathcal{C}  \NDnt{t_{{\mathrm{2}}}}  \NDsym{:}  \NDnt{X}  \otimes  \NDnt{Y}}
        }{\Phi_{{\mathrm{1}}}  \NDsym{,}  \Psi_{{\mathrm{1}}}  \NDsym{,}  \Phi_{{\mathrm{2}}}  \NDsym{,}  \Psi_{{\mathrm{2}}}  \vdash_\mathcal{C}   \mathsf{let}\, \NDnt{t_{{\mathrm{2}}}}  :  \NDnt{X}  \otimes  \NDnt{Y} \,\mathsf{be}\, \NDmv{x}  \otimes  \NDmv{y} \,\mathsf{in}\, \NDsym{(}   \mathsf{let}\, \NDnt{t_{{\mathrm{1}}}}  :   \mathsf{UnitT}  \,\mathsf{be}\,  \mathsf{trivT}  \,\mathsf{in}\, \NDnt{t_{{\mathrm{3}}}}   \NDsym{)}   \NDsym{:}  \NDnt{Z}}
      \end{math}
    \end{center}

  \item (\NDdruleTXXtenEName, \NDdruleTXXtenEName):
    \begin{center}
      \tiny
      \begin{math}
        $$\mprset{flushleft}
        \inferrule* [right={\tiny tenE}] {
          $$\mprset{flushleft}
          \inferrule* [right={\tiny tenE}] {
            {\Phi_{{\mathrm{1}}}  \NDsym{,}  \NDmv{x}  \NDsym{:}  \NDnt{X_{{\mathrm{2}}}}  \NDsym{,}  \NDmv{y}  \NDsym{:}  \NDnt{Y_{{\mathrm{2}}}}  \NDsym{,}  \Phi_{{\mathrm{2}}}  \vdash_\mathcal{C}  \NDnt{t_{{\mathrm{1}}}}  \NDsym{:}  \NDnt{X_{{\mathrm{1}}}}  \otimes  \NDnt{Y_{{\mathrm{1}}}}} \\
            {\Psi  \vdash_\mathcal{C}  \NDnt{t_{{\mathrm{2}}}}  \NDsym{:}  \NDnt{X_{{\mathrm{2}}}}  \otimes  \NDnt{Y_{{\mathrm{2}}}}}
          }{\Phi_{{\mathrm{1}}}  \NDsym{,}  \Psi  \NDsym{,}  \Phi_{{\mathrm{2}}}  \vdash_\mathcal{C}   \mathsf{let}\, \NDnt{t_{{\mathrm{2}}}}  :  \NDnt{X_{{\mathrm{2}}}}  \otimes  \NDnt{Y_{{\mathrm{2}}}} \,\mathsf{be}\, \NDmv{x}  \otimes  \NDmv{y} \,\mathsf{in}\, \NDnt{t_{{\mathrm{1}}}}   \NDsym{:}  \NDnt{X_{{\mathrm{1}}}}  \otimes  \NDnt{Y_{{\mathrm{1}}}}} \\
           {\Psi_{{\mathrm{1}}}  \NDsym{,}  \NDmv{w}  \NDsym{:}  \NDnt{X_{{\mathrm{1}}}}  \NDsym{,}  \NDmv{z}  \NDsym{:}  \NDnt{Y_{{\mathrm{1}}}}  \NDsym{,}  \Psi_{{\mathrm{2}}}  \vdash_\mathcal{C}  \NDnt{t_{{\mathrm{3}}}}  \NDsym{:}  \NDnt{Z}}
        }{\Psi_{{\mathrm{1}}}  \NDsym{,}  \Phi_{{\mathrm{1}}}  \NDsym{,}  \Psi  \NDsym{,}  \Phi_{{\mathrm{2}}}  \NDsym{,}  \Psi_{{\mathrm{2}}}  \vdash_\mathcal{C}   \mathsf{let}\, \NDsym{(}   \mathsf{let}\, \NDnt{t_{{\mathrm{2}}}}  :  \NDnt{X_{{\mathrm{2}}}}  \otimes  \NDnt{Y_{{\mathrm{2}}}} \,\mathsf{be}\, \NDmv{x}  \otimes  \NDmv{y} \,\mathsf{in}\, \NDnt{t_{{\mathrm{1}}}}   \NDsym{)}  :  \NDnt{X_{{\mathrm{1}}}}  \otimes  \NDnt{Y_{{\mathrm{1}}}} \,\mathsf{be}\, \NDmv{w}  \otimes  \NDmv{z} \,\mathsf{in}\, \NDnt{t_{{\mathrm{3}}}}   \NDsym{:}  \NDnt{Z}}
      \end{math}
    \end{center}
    commutes to
    \begin{center}
      \tiny
      \begin{math}
        $$\mprset{flushleft}
        \inferrule* [right={\tiny tenE}] {
          $$\mprset{flushleft}
          \inferrule* [right={\tiny tenE}] {
            {\Phi_{{\mathrm{1}}}  \NDsym{,}  \NDmv{x}  \NDsym{:}  \NDnt{X_{{\mathrm{2}}}}  \NDsym{,}  \NDmv{y}  \NDsym{:}  \NDnt{Y_{{\mathrm{2}}}}  \NDsym{,}  \Phi_{{\mathrm{2}}}  \vdash_\mathcal{C}  \NDnt{t_{{\mathrm{1}}}}  \NDsym{:}  \NDnt{X_{{\mathrm{1}}}}  \otimes  \NDnt{Y_{{\mathrm{1}}}}} \\
            {\Psi_{{\mathrm{1}}}  \NDsym{,}  \NDmv{w}  \NDsym{:}  \NDnt{X_{{\mathrm{1}}}}  \NDsym{,}  \NDmv{z}  \NDsym{:}  \NDnt{Y_{{\mathrm{1}}}}  \NDsym{,}  \Psi_{{\mathrm{2}}}  \vdash_\mathcal{C}  \NDnt{t_{{\mathrm{3}}}}  \NDsym{:}  \NDnt{Z}}
          }{\Psi_{{\mathrm{1}}}  \NDsym{,}  \Phi_{{\mathrm{1}}}  \NDsym{,}  \NDmv{x}  \NDsym{:}  \NDnt{X_{{\mathrm{2}}}}  \NDsym{,}  \NDmv{y}  \NDsym{:}  \NDnt{Y_{{\mathrm{2}}}}  \NDsym{,}  \Phi_{{\mathrm{2}}}  \NDsym{,}  \Psi_{{\mathrm{2}}}  \vdash_\mathcal{C}   \mathsf{let}\, \NDnt{t_{{\mathrm{1}}}}  :  \NDnt{X_{{\mathrm{1}}}}  \otimes  \NDnt{Y_{{\mathrm{1}}}} \,\mathsf{be}\, \NDmv{w}  \otimes  \NDmv{z} \,\mathsf{in}\, \NDnt{t_{{\mathrm{3}}}}   \NDsym{:}  \NDnt{Z}} \\
           {\Psi  \vdash_\mathcal{C}  \NDnt{t_{{\mathrm{2}}}}  \NDsym{:}  \NDnt{X_{{\mathrm{2}}}}  \otimes  \NDnt{Y_{{\mathrm{2}}}}}
        }{\Psi_{{\mathrm{1}}}  \NDsym{,}  \Phi_{{\mathrm{1}}}  \NDsym{,}  \Psi  \NDsym{,}  \Phi_{{\mathrm{2}}}  \NDsym{,}  \Psi_{{\mathrm{2}}}  \vdash_\mathcal{C}   \mathsf{let}\, \NDnt{t_{{\mathrm{2}}}}  :  \NDnt{X_{{\mathrm{2}}}}  \otimes  \NDnt{Y_{{\mathrm{2}}}} \,\mathsf{be}\, \NDmv{x}  \otimes  \NDmv{y} \,\mathsf{in}\, \NDsym{(}   \mathsf{let}\, \NDnt{t_{{\mathrm{1}}}}  :  \NDnt{X_{{\mathrm{1}}}}  \otimes  \NDnt{Y_{{\mathrm{1}}}} \,\mathsf{be}\, \NDmv{w}  \otimes  \NDmv{z} \,\mathsf{in}\, \NDnt{t_{{\mathrm{3}}}}   \NDsym{)}   \NDsym{:}  \NDnt{Z}}
      \end{math}
    \end{center}

  \item (\NDdruleTXXtenEName, \NDdruleTXXimpEName):
    \begin{center}
      \tiny
      \begin{math}
        $$\mprset{flushleft}
        \inferrule* [right={\tiny tenE}] {
          $$\mprset{flushleft}
          \inferrule* [right={\tiny impE}] {
            {\Phi_{{\mathrm{1}}}  \NDsym{,}  \NDmv{x}  \NDsym{:}  \NDnt{X_{{\mathrm{2}}}}  \NDsym{,}  \NDmv{y}  \NDsym{:}  \NDnt{Y_{{\mathrm{2}}}}  \NDsym{,}  \Phi_{{\mathrm{2}}}  \vdash_\mathcal{C}  \NDnt{t_{{\mathrm{1}}}}  \NDsym{:}  \NDnt{X_{{\mathrm{1}}}}  \multimap  \NDnt{Y_{{\mathrm{1}}}}} \\
            {\Psi_{{\mathrm{1}}}  \vdash_\mathcal{C}  \NDnt{t_{{\mathrm{2}}}}  \NDsym{:}  \NDnt{X_{{\mathrm{2}}}}  \otimes  \NDnt{Y_{{\mathrm{2}}}}}
          }{\Phi_{{\mathrm{1}}}  \NDsym{,}  \Psi_{{\mathrm{1}}}  \NDsym{,}  \Phi_{{\mathrm{2}}}  \vdash_\mathcal{C}   \mathsf{let}\, \NDnt{t_{{\mathrm{2}}}}  :  \NDnt{X_{{\mathrm{2}}}}  \otimes  \NDnt{Y_{{\mathrm{2}}}} \,\mathsf{be}\, \NDmv{x}  \otimes  \NDmv{y} \,\mathsf{in}\, \NDnt{t_{{\mathrm{1}}}}   \NDsym{:}  \NDnt{X_{{\mathrm{1}}}}  \multimap  \NDnt{Y_{{\mathrm{1}}}}} \\
           {\Psi_{{\mathrm{2}}}  \vdash_\mathcal{C}  \NDnt{t_{{\mathrm{3}}}}  \NDsym{:}  \NDnt{X_{{\mathrm{1}}}}}
        }{\Phi_{{\mathrm{1}}}  \NDsym{,}  \Psi_{{\mathrm{1}}}  \NDsym{,}  \Phi_{{\mathrm{2}}}  \NDsym{,}  \Psi_{{\mathrm{2}}}  \vdash_\mathcal{C}   \mathsf{app}\, \NDsym{(}   \mathsf{let}\, \NDnt{t_{{\mathrm{2}}}}  :  \NDnt{X_{{\mathrm{2}}}}  \otimes  \NDnt{Y_{{\mathrm{2}}}} \,\mathsf{be}\, \NDmv{x}  \otimes  \NDmv{y} \,\mathsf{in}\, \NDnt{t_{{\mathrm{1}}}}   \NDsym{)} \, \NDnt{t_{{\mathrm{3}}}}   \NDsym{:}  \NDnt{Y_{{\mathrm{1}}}}}
      \end{math}
    \end{center}
    commutes to
    \begin{center}
      \tiny
      \begin{math}
        $$\mprset{flushleft}
        \inferrule* [right={\tiny impE}] {
          $$\mprset{flushleft}
          \inferrule* [right={\tiny tenE}] {
            {\Phi_{{\mathrm{1}}}  \NDsym{,}  \NDmv{x}  \NDsym{:}  \NDnt{X_{{\mathrm{2}}}}  \NDsym{,}  \NDmv{y}  \NDsym{:}  \NDnt{Y_{{\mathrm{2}}}}  \NDsym{,}  \Phi_{{\mathrm{2}}}  \vdash_\mathcal{C}  \NDnt{t_{{\mathrm{1}}}}  \NDsym{:}  \NDnt{X_{{\mathrm{1}}}}  \multimap  \NDnt{Y_{{\mathrm{1}}}}} \\
            {\Psi_{{\mathrm{2}}}  \vdash_\mathcal{C}  \NDnt{t_{{\mathrm{3}}}}  \NDsym{:}  \NDnt{X_{{\mathrm{1}}}}}
          }{\Phi_{{\mathrm{1}}}  \NDsym{,}  \NDmv{x}  \NDsym{:}  \NDnt{X_{{\mathrm{2}}}}  \NDsym{,}  \NDmv{y}  \NDsym{:}  \NDnt{Y_{{\mathrm{2}}}}  \NDsym{,}  \Phi_{{\mathrm{2}}}  \NDsym{,}  \Psi_{{\mathrm{2}}}  \vdash_\mathcal{C}   \mathsf{app}\, \NDnt{t_{{\mathrm{1}}}} \, \NDnt{t_{{\mathrm{3}}}}   \NDsym{:}  \NDnt{Y_{{\mathrm{1}}}}} \\
           {\Psi_{{\mathrm{1}}}  \vdash_\mathcal{C}  \NDnt{t_{{\mathrm{2}}}}  \NDsym{:}  \NDnt{X_{{\mathrm{2}}}}  \otimes  \NDnt{Y_{{\mathrm{2}}}}}
        }{\Phi_{{\mathrm{1}}}  \NDsym{,}  \Psi_{{\mathrm{1}}}  \NDsym{,}  \Phi_{{\mathrm{2}}}  \NDsym{,}  \Psi_{{\mathrm{2}}}  \vdash_\mathcal{C}   \mathsf{let}\, \NDnt{t_{{\mathrm{2}}}}  :  \NDnt{X_{{\mathrm{2}}}}  \otimes  \NDnt{Y_{{\mathrm{2}}}} \,\mathsf{be}\, \NDmv{x}  \otimes  \NDmv{y} \,\mathsf{in}\, \NDsym{(}   \mathsf{app}\, \NDnt{t_{{\mathrm{1}}}} \, \NDnt{t_{{\mathrm{3}}}}   \NDsym{)}   \NDsym{:}  \NDnt{Y_{{\mathrm{1}}}}}
      \end{math}
    \end{center}

  \end{itemize}

\item Commutation of $\multimap_E$:

  \begin{itemize}

  \item (\NDdruleTXXimpEName, \NDdruleTXXunitEName):
    \begin{center}
      \tiny
      \begin{math}
        $$\mprset{flushleft}
        \inferrule* [right={\tiny tenE}] {
          $$\mprset{flushleft}
          \inferrule* [right={\tiny impE}] {
            {\Phi_{{\mathrm{1}}}  \vdash_\mathcal{C}  \NDnt{t_{{\mathrm{1}}}}  \NDsym{:}   \mathsf{UnitT} } \\
            {\Phi_{{\mathrm{2}}}  \vdash_\mathcal{C}  \NDnt{t_{{\mathrm{2}}}}  \NDsym{:}   \mathsf{UnitT}   \multimap   \mathsf{UnitT} }
          }{\Phi_{{\mathrm{2}}}  \NDsym{,}  \Phi_{{\mathrm{1}}}  \vdash_\mathcal{C}   \mathsf{app}\, \NDnt{t_{{\mathrm{2}}}} \, \NDnt{t_{{\mathrm{1}}}}   \NDsym{:}   \mathsf{UnitT} } \\
           {\Phi_{{\mathrm{3}}}  \vdash_\mathcal{C}  \NDnt{t_{{\mathrm{3}}}}  \NDsym{:}   \mathsf{UnitT} }
        }{\Phi_{{\mathrm{2}}}  \NDsym{,}  \Phi_{{\mathrm{1}}}  \NDsym{,}  \Phi_{{\mathrm{3}}}  \vdash_\mathcal{C}   \mathsf{let}\, \NDsym{(}   \mathsf{app}\, \NDnt{t_{{\mathrm{2}}}} \, \NDnt{t_{{\mathrm{1}}}}   \NDsym{)}  :   \mathsf{UnitT}  \,\mathsf{be}\,  \mathsf{trivT}  \,\mathsf{in}\, \NDnt{t_{{\mathrm{3}}}}   \NDsym{:}   \mathsf{UnitT} }
      \end{math}
    \end{center}
    commutes to
    \begin{center}
      \tiny
      \begin{math}
        $$\mprset{flushleft}
        \inferrule* [right={\tiny impE}] {
          $$\mprset{flushleft}
          \inferrule* [right={\tiny tenE}] {
            {\Phi_{{\mathrm{1}}}  \vdash_\mathcal{C}  \NDnt{t_{{\mathrm{1}}}}  \NDsym{:}   \mathsf{UnitT} } \\
            {\Phi_{{\mathrm{3}}}  \vdash_\mathcal{C}  \NDnt{t_{{\mathrm{3}}}}  \NDsym{:}   \mathsf{UnitT} }
          }{\Phi_{{\mathrm{1}}}  \NDsym{,}  \Phi_{{\mathrm{3}}}  \vdash_\mathcal{C}   \mathsf{let}\, \NDnt{t_{{\mathrm{1}}}}  :   \mathsf{UnitT}  \,\mathsf{be}\,  \mathsf{trivT}  \,\mathsf{in}\, \NDnt{t_{{\mathrm{3}}}}   \NDsym{:}   \mathsf{UnitT} }
           {\Phi_{{\mathrm{2}}}  \vdash_\mathcal{C}  \NDnt{t_{{\mathrm{2}}}}  \NDsym{:}   \mathsf{UnitT}   \multimap   \mathsf{UnitT} }
        }{\Phi_{{\mathrm{2}}}  \NDsym{,}  \Phi_{{\mathrm{1}}}  \NDsym{,}  \Phi_{{\mathrm{3}}}  \vdash_\mathcal{C}   \mathsf{app}\, \NDnt{t_{{\mathrm{2}}}} \, \NDsym{(}   \mathsf{let}\, \NDnt{t_{{\mathrm{1}}}}  :   \mathsf{UnitT}  \,\mathsf{be}\,  \mathsf{trivT}  \,\mathsf{in}\, \NDnt{t_{{\mathrm{3}}}}   \NDsym{)}   \NDsym{:}   \mathsf{UnitT} }
      \end{math}
    \end{center}
  \item (\NDdruleTXXimpEName, \NDdruleTXXtenEName): ?
  \item (\NDdruleTXXimpEName, \NDdruleTXXimpEName): ?
  \end{itemize}

\item Commutation of $\tri_E$:

  \begin{itemize}

  \item (\NDdruleSXXunitETwoName, \NDdruleSXXunitETwoName):
    \begin{center}
      \tiny
      \begin{math}
        $$\mprset{flushleft}
        \inferrule* [right={\tiny unitE}] {
          $$\mprset{flushleft}
          \inferrule* [right={\tiny unitE}] {
            {\Gamma_{{\mathrm{1}}}  \vdash_\mathcal{L}  \NDnt{s_{{\mathrm{1}}}}  \NDsym{:}   \mathsf{UnitS} } \\
            {\Gamma_{{\mathrm{2}}}  \vdash_\mathcal{L}  \NDnt{s_{{\mathrm{2}}}}  \NDsym{:}   \mathsf{UnitS} }
          }{\Gamma_{{\mathrm{2}}}  \NDsym{,}  \Gamma_{{\mathrm{1}}}  \vdash_\mathcal{L}   \mathsf{let}\, \NDnt{s_{{\mathrm{2}}}}  :   \mathsf{UnitS}  \,\mathsf{be}\,  \mathsf{trivS}  \,\mathsf{in}\, \NDnt{s_{{\mathrm{1}}}}   \NDsym{:}   \mathsf{UnitS} } \\
           {\Gamma_{{\mathrm{3}}}  \vdash_\mathcal{L}  \NDnt{s_{{\mathrm{3}}}}  \NDsym{:}  \NDnt{A}}
        }{\Gamma_{{\mathrm{2}}}  \NDsym{,}  \Gamma_{{\mathrm{1}}}  \NDsym{,}  \Gamma_{{\mathrm{3}}}  \vdash_\mathcal{L}   \mathsf{let}\, \NDsym{(}   \mathsf{let}\, \NDnt{s_{{\mathrm{2}}}}  :   \mathsf{UnitS}  \,\mathsf{be}\,  \mathsf{trivS}  \,\mathsf{in}\, \NDnt{s_{{\mathrm{1}}}}   \NDsym{)}  :   \mathsf{UnitS}  \,\mathsf{be}\,  \mathsf{trivS}  \,\mathsf{in}\, \NDnt{s_{{\mathrm{3}}}}   \NDsym{:}  \NDnt{A}}
      \end{math}
    \end{center}
    commutes to
    \begin{center}
      \tiny
      \begin{math}
        $$\mprset{flushleft}
        \inferrule* [right={\tiny unitE}] {
          $$\mprset{flushleft}
          \inferrule* [right={\tiny unitE}] {
            {\Gamma_{{\mathrm{1}}}  \vdash_\mathcal{L}  \NDnt{s_{{\mathrm{1}}}}  \NDsym{:}   \mathsf{UnitS} } \\
            {\Gamma_{{\mathrm{3}}}  \vdash_\mathcal{L}  \NDnt{s_{{\mathrm{3}}}}  \NDsym{:}  \NDnt{A}}
          }{\Gamma_{{\mathrm{1}}}  \NDsym{,}  \Gamma_{{\mathrm{3}}}  \vdash_\mathcal{L}   \mathsf{let}\, \NDnt{s_{{\mathrm{1}}}}  :   \mathsf{UnitS}  \,\mathsf{be}\,  \mathsf{trivS}  \,\mathsf{in}\, \NDnt{s_{{\mathrm{3}}}}   \NDsym{:}  \NDnt{A}} \\
           {\Gamma_{{\mathrm{2}}}  \vdash_\mathcal{L}  \NDnt{s_{{\mathrm{2}}}}  \NDsym{:}   \mathsf{UnitS} }
        }{\Gamma_{{\mathrm{2}}}  \NDsym{,}  \Gamma_{{\mathrm{1}}}  \NDsym{,}  \Gamma_{{\mathrm{3}}}  \vdash_\mathcal{L}   \mathsf{let}\, \NDnt{s_{{\mathrm{2}}}}  :   \mathsf{UnitS}  \,\mathsf{be}\,  \mathsf{trivS}  \,\mathsf{in}\, \NDsym{(}   \mathsf{let}\, \NDnt{s_{{\mathrm{1}}}}  :   \mathsf{UnitS}  \,\mathsf{be}\,  \mathsf{trivS}  \,\mathsf{in}\, \NDnt{s_{{\mathrm{3}}}}   \NDsym{)}   \NDsym{:}  \NDnt{A}}
      \end{math}
    \end{center}

  \item (\NDdruleSXXunitETwoName, \NDdruleSXXtenETwoName): Does NOT commute
    \begin{center}
      \tiny
      \begin{math}
        $$\mprset{flushleft}
        \inferrule* [right={\tiny tenE2}] {
          $$\mprset{flushleft}
          \inferrule* [right={\tiny unitE}] {
            {\Gamma_{{\mathrm{1}}}  \vdash_\mathcal{L}  \NDnt{s_{{\mathrm{1}}}}  \NDsym{:}  \NDnt{A}  \triangleright  \NDnt{B}} \\
            {\Gamma_{{\mathrm{2}}}  \vdash_\mathcal{L}  \NDnt{s_{{\mathrm{2}}}}  \NDsym{:}   \mathsf{UnitS} }
          }{\Gamma_{{\mathrm{2}}}  \NDsym{,}  \Gamma_{{\mathrm{1}}}  \vdash_\mathcal{L}   \mathsf{let}\, \NDnt{s_{{\mathrm{2}}}}  :   \mathsf{UnitS}  \,\mathsf{be}\,  \mathsf{trivS}  \,\mathsf{in}\, \NDnt{s_{{\mathrm{1}}}}   \NDsym{:}  \NDnt{A}  \triangleright  \NDnt{B}} \\
           {\Delta_{{\mathrm{1}}}  \NDsym{,}  \NDmv{x}  \NDsym{:}  \NDnt{A}  \NDsym{,}  \NDmv{y}  \NDsym{:}  \NDnt{B}  \NDsym{,}  \Delta_{{\mathrm{2}}}  \vdash_\mathcal{L}  \NDnt{s_{{\mathrm{3}}}}  \NDsym{:}  \NDnt{C}}
        }{\Delta_{{\mathrm{1}}}  \NDsym{,}  \Gamma_{{\mathrm{2}}}  \NDsym{,}  \Gamma_{{\mathrm{1}}}  \NDsym{,}  \Delta_{{\mathrm{2}}}  \vdash_\mathcal{L}   \mathsf{let}\, \NDsym{(}   \mathsf{let}\, \NDnt{s_{{\mathrm{2}}}}  :   \mathsf{UnitS}  \,\mathsf{be}\,  \mathsf{trivS}  \,\mathsf{in}\, \NDnt{s_{{\mathrm{1}}}}   \NDsym{)}  :  \NDnt{A}  \triangleright  \NDnt{B} \,\mathsf{be}\, \NDmv{x}  \triangleright  \NDmv{y} \,\mathsf{in}\, \NDnt{s_{{\mathrm{3}}}}   \NDsym{:}  \NDnt{C}}
      \end{math}
    \end{center}
    commutes to
    \begin{center}
      \tiny
      \begin{math}
        $$\mprset{flushleft}
        \inferrule* [right={\tiny unitE}] {
          $$\mprset{flushleft}
          \inferrule* [right={\tiny tenE2}] {
            {\Gamma_{{\mathrm{1}}}  \vdash_\mathcal{L}  \NDnt{s_{{\mathrm{1}}}}  \NDsym{:}  \NDnt{A}  \triangleright  \NDnt{B}} \\
            {\Delta_{{\mathrm{1}}}  \NDsym{,}  \NDmv{x}  \NDsym{:}  \NDnt{A}  \NDsym{,}  \NDmv{y}  \NDsym{:}  \NDnt{B}  \NDsym{,}  \Delta_{{\mathrm{2}}}  \vdash_\mathcal{L}  \NDnt{s_{{\mathrm{3}}}}  \NDsym{:}  \NDnt{C}}
          }{\Delta_{{\mathrm{1}}}  \NDsym{,}  \Gamma_{{\mathrm{1}}}  \NDsym{,}  \Delta_{{\mathrm{2}}}  \vdash_\mathcal{L}   \mathsf{let}\, \NDnt{s_{{\mathrm{1}}}}  :  \NDnt{A}  \triangleright  \NDnt{B} \,\mathsf{be}\, \NDmv{x}  \triangleright  \NDmv{y} \,\mathsf{in}\, \NDnt{s_{{\mathrm{3}}}}   \NDsym{:}  \NDnt{C}} \\
           {\Gamma_{{\mathrm{2}}}  \vdash_\mathcal{L}  \NDnt{s_{{\mathrm{2}}}}  \NDsym{:}   \mathsf{UnitS} }
        }{\Gamma_{{\mathrm{2}}}  \NDsym{,}  \Delta_{{\mathrm{1}}}  \NDsym{,}  \Gamma_{{\mathrm{1}}}  \NDsym{,}  \Delta_{{\mathrm{2}}}  \vdash_\mathcal{L}   \mathsf{let}\, \NDnt{s_{{\mathrm{2}}}}  :   \mathsf{UnitS}  \,\mathsf{be}\,  \mathsf{trivS}  \,\mathsf{in}\, \NDsym{(}   \mathsf{let}\, \NDnt{s_{{\mathrm{1}}}}  :  \NDnt{A}  \triangleright  \NDnt{B} \,\mathsf{be}\, \NDmv{x}  \triangleright  \NDmv{y} \,\mathsf{in}\, \NDnt{s_{{\mathrm{3}}}}   \NDsym{)}   \NDsym{:}  \NDnt{C}}
      \end{math}
    \end{center}

  \item (\NDdruleSXXunitETwoName, \NDdruleSXXimprEName):
    \begin{center}
      \tiny
      \begin{math}
        $$\mprset{flushleft}
        \inferrule* [right={\tiny imprE}] {
          $$\mprset{flushleft}
          \inferrule* [right={\tiny unitE}] {
            {\Gamma_{{\mathrm{1}}}  \vdash_\mathcal{L}  \NDnt{s_{{\mathrm{1}}}}  \NDsym{:}  \NDnt{A}  \rightharpoonup  \NDnt{B}} \\
            {\Gamma_{{\mathrm{2}}}  \vdash_\mathcal{L}  \NDnt{s_{{\mathrm{2}}}}  \NDsym{:}   \mathsf{UnitS} }
          }{\Gamma_{{\mathrm{2}}}  \NDsym{,}  \Gamma_{{\mathrm{1}}}  \vdash_\mathcal{L}   \mathsf{let}\, \NDnt{s_{{\mathrm{2}}}}  :   \mathsf{UnitS}  \,\mathsf{be}\,  \mathsf{trivS}  \,\mathsf{in}\, \NDnt{s_{{\mathrm{1}}}}   \NDsym{:}  \NDnt{A}  \rightharpoonup  \NDnt{B}} \\
           {\Gamma_{{\mathrm{3}}}  \vdash_\mathcal{L}  \NDnt{s_{{\mathrm{3}}}}  \NDsym{:}  \NDnt{A}}
        }{\Gamma_{{\mathrm{2}}}  \NDsym{,}  \Gamma_{{\mathrm{1}}}  \NDsym{,}  \Gamma_{{\mathrm{3}}}  \vdash_\mathcal{L}   \mathsf{app}_r\, \NDsym{(}   \mathsf{let}\, \NDnt{s_{{\mathrm{2}}}}  :   \mathsf{UnitS}  \,\mathsf{be}\,  \mathsf{trivS}  \,\mathsf{in}\, \NDnt{s_{{\mathrm{1}}}}   \NDsym{)} \, \NDnt{s_{{\mathrm{3}}}}   \NDsym{:}  \NDnt{B}}
      \end{math}
    \end{center}
    commutes to
    \begin{center}
      \tiny
      \begin{math}
        $$\mprset{flushleft}
        \inferrule* [right={\tiny unitE}] {
          $$\mprset{flushleft}
          \inferrule* [right={\tiny imprE}] {
            {\Gamma_{{\mathrm{1}}}  \vdash_\mathcal{L}  \NDnt{s_{{\mathrm{1}}}}  \NDsym{:}  \NDnt{A}  \rightharpoonup  \NDnt{B}} \\
            {\Gamma_{{\mathrm{3}}}  \vdash_\mathcal{L}  \NDnt{s_{{\mathrm{3}}}}  \NDsym{:}  \NDnt{A}}
          }{\Gamma_{{\mathrm{1}}}  \NDsym{,}  \Gamma_{{\mathrm{3}}}  \vdash_\mathcal{L}   \mathsf{app}_r\, \NDnt{s_{{\mathrm{1}}}} \, \NDnt{s_{{\mathrm{3}}}}   \NDsym{:}  \NDnt{B}} \\
           {\Gamma_{{\mathrm{2}}}  \vdash_\mathcal{L}  \NDnt{s_{{\mathrm{2}}}}  \NDsym{:}   \mathsf{UnitS} }
        }{\Gamma_{{\mathrm{2}}}  \NDsym{,}  \Gamma_{{\mathrm{1}}}  \NDsym{,}  \Gamma_{{\mathrm{3}}}  \vdash_\mathcal{L}   \mathsf{let}\, \NDnt{s_{{\mathrm{2}}}}  :   \mathsf{UnitS}  \,\mathsf{be}\,  \mathsf{trivS}  \,\mathsf{in}\, \NDsym{(}   \mathsf{app}_r\, \NDnt{s_{{\mathrm{1}}}} \, \NDnt{s_{{\mathrm{3}}}}   \NDsym{)}   \NDsym{:}  \NDnt{B}}
      \end{math}
    \end{center}

  \item (\NDdruleSXXunitETwoName, \NDdruleSXXimplEName): Does NOT commute.
    \begin{center}
      \tiny
      \begin{math}
        $$\mprset{flushleft}
        \inferrule* [right={\tiny imprE}] {
          $$\mprset{flushleft}
          \inferrule* [right={\tiny unitE}] {
            {\Gamma_{{\mathrm{1}}}  \vdash_\mathcal{L}  \NDnt{s_{{\mathrm{1}}}}  \NDsym{:}  \NDnt{B}  \leftharpoonup  \NDnt{A}} \\
            {\Gamma_{{\mathrm{2}}}  \vdash_\mathcal{L}  \NDnt{s_{{\mathrm{2}}}}  \NDsym{:}   \mathsf{UnitS} }
          }{\Gamma_{{\mathrm{2}}}  \NDsym{,}  \Gamma_{{\mathrm{1}}}  \vdash_\mathcal{L}   \mathsf{let}\, \NDnt{s_{{\mathrm{2}}}}  :   \mathsf{UnitS}  \,\mathsf{be}\,  \mathsf{trivS}  \,\mathsf{in}\, \NDnt{s_{{\mathrm{1}}}}   \NDsym{:}  \NDnt{B}  \leftharpoonup  \NDnt{A}} \\
           {\Gamma_{{\mathrm{3}}}  \vdash_\mathcal{L}  \NDnt{s_{{\mathrm{3}}}}  \NDsym{:}  \NDnt{A}}
        }{\Gamma_{{\mathrm{3}}}  \NDsym{,}  \Gamma_{{\mathrm{2}}}  \NDsym{,}  \Gamma_{{\mathrm{1}}}  \vdash_\mathcal{L}   \mathsf{app}_l\, \NDsym{(}   \mathsf{let}\, \NDnt{s_{{\mathrm{2}}}}  :   \mathsf{UnitS}  \,\mathsf{be}\,  \mathsf{trivS}  \,\mathsf{in}\, \NDnt{s_{{\mathrm{1}}}}   \NDsym{)} \, \NDnt{s_{{\mathrm{3}}}}   \NDsym{:}  \NDnt{B}}
      \end{math}
    \end{center}
    commutes to
    \begin{center}
      \tiny
      \begin{math}
        $$\mprset{flushleft}
        \inferrule* [right={\tiny unitE}] {
          $$\mprset{flushleft}
          \inferrule* [right={\tiny imprE}] {
            {\Gamma_{{\mathrm{1}}}  \vdash_\mathcal{L}  \NDnt{s_{{\mathrm{1}}}}  \NDsym{:}  \NDnt{B}  \leftharpoonup  \NDnt{A}} \\
            {\Gamma_{{\mathrm{3}}}  \vdash_\mathcal{L}  \NDnt{s_{{\mathrm{3}}}}  \NDsym{:}  \NDnt{A}}
          }{\Gamma_{{\mathrm{3}}}  \NDsym{,}  \Gamma_{{\mathrm{1}}}  \vdash_\mathcal{L}   \mathsf{app}_l\, \NDnt{s_{{\mathrm{1}}}} \, \NDnt{s_{{\mathrm{3}}}}   \NDsym{:}  \NDnt{B}} \\
           {\Gamma_{{\mathrm{2}}}  \vdash_\mathcal{L}  \NDnt{s_{{\mathrm{2}}}}  \NDsym{:}   \mathsf{UnitS} }
        }{\Gamma_{{\mathrm{2}}}  \NDsym{,}  \Gamma_{{\mathrm{3}}}  \NDsym{,}  \Gamma_{{\mathrm{1}}}  \vdash_\mathcal{L}   \mathsf{let}\, \NDnt{s_{{\mathrm{2}}}}  :   \mathsf{UnitS}  \,\mathsf{be}\,  \mathsf{trivS}  \,\mathsf{in}\, \NDsym{(}   \mathsf{app}_l\, \NDnt{s_{{\mathrm{1}}}} \, \NDnt{s_{{\mathrm{3}}}}   \NDsym{)}   \NDsym{:}  \NDnt{B}}
      \end{math}
    \end{center}

  \item (\NDdruleSXXtenETwoName, \NDdruleSXXunitETwoName):
    \begin{center}
      \tiny
      \begin{math}
        $$\mprset{flushleft}
        \inferrule* [right={\tiny unitE2}] {
          $$\mprset{flushleft}
          \inferrule* [right={\tiny tenE2}] {
            {\Gamma_{{\mathrm{1}}}  \NDsym{,}  \NDmv{x}  \NDsym{:}  \NDnt{A}  \NDsym{,}  \NDmv{y}  \NDsym{:}  \NDnt{B}  \NDsym{,}  \Gamma_{{\mathrm{2}}}  \vdash_\mathcal{L}  \NDnt{s_{{\mathrm{1}}}}  \NDsym{:}   \mathsf{UnitS} } \\
            {\Delta_{{\mathrm{1}}}  \vdash_\mathcal{L}  \NDnt{s_{{\mathrm{2}}}}  \NDsym{:}  \NDnt{A}  \triangleright  \NDnt{B}}
          }{\Gamma_{{\mathrm{1}}}  \NDsym{,}  \Delta_{{\mathrm{1}}}  \NDsym{,}  \Gamma_{{\mathrm{2}}}  \vdash_\mathcal{L}   \mathsf{let}\, \NDnt{s_{{\mathrm{2}}}}  :  \NDnt{A}  \triangleright  \NDnt{B} \,\mathsf{be}\, \NDmv{x}  \triangleright  \NDmv{y} \,\mathsf{in}\, \NDnt{s_{{\mathrm{1}}}}   \NDsym{:}   \mathsf{UnitS} } \\
           {\Delta_{{\mathrm{2}}}  \vdash_\mathcal{L}  \NDnt{s_{{\mathrm{3}}}}  \NDsym{:}  \NDnt{C}}
        }{\Gamma_{{\mathrm{1}}}  \NDsym{,}  \Delta_{{\mathrm{1}}}  \NDsym{,}  \Gamma_{{\mathrm{2}}}  \NDsym{,}  \Delta_{{\mathrm{2}}}  \vdash_\mathcal{L}   \mathsf{let}\, \NDsym{(}   \mathsf{let}\, \NDnt{s_{{\mathrm{2}}}}  :  \NDnt{A}  \triangleright  \NDnt{B} \,\mathsf{be}\, \NDmv{x}  \triangleright  \NDmv{y} \,\mathsf{in}\, \NDnt{s_{{\mathrm{1}}}}   \NDsym{)}  :   \mathsf{UnitS}  \,\mathsf{be}\,  \mathsf{trivS}  \,\mathsf{in}\, \NDnt{s_{{\mathrm{3}}}}   \NDsym{:}  \NDnt{C}}
      \end{math}
    \end{center}
    commutes to
    \begin{center}
      \tiny
      \begin{math}
        $$\mprset{flushleft}
        \inferrule* [right={\tiny tenE2}] {
          $$\mprset{flushleft}
          \inferrule* [right={\tiny unitE2}] {
            {\Gamma_{{\mathrm{1}}}  \NDsym{,}  \NDmv{x}  \NDsym{:}  \NDnt{A}  \NDsym{,}  \NDmv{y}  \NDsym{:}  \NDnt{B}  \NDsym{,}  \Gamma_{{\mathrm{2}}}  \vdash_\mathcal{L}  \NDnt{s_{{\mathrm{1}}}}  \NDsym{:}   \mathsf{UnitS} } \\
            {\Delta_{{\mathrm{2}}}  \vdash_\mathcal{L}  \NDnt{s_{{\mathrm{3}}}}  \NDsym{:}  \NDnt{C}}
          }{\Gamma_{{\mathrm{1}}}  \NDsym{,}  \NDmv{x}  \NDsym{:}  \NDnt{A}  \NDsym{,}  \NDmv{y}  \NDsym{:}  \NDnt{B}  \NDsym{,}  \Gamma_{{\mathrm{2}}}  \NDsym{,}  \Delta_{{\mathrm{2}}}  \vdash_\mathcal{L}   \mathsf{let}\, \NDnt{s_{{\mathrm{1}}}}  :   \mathsf{UnitS}  \,\mathsf{be}\,  \mathsf{trivS}  \,\mathsf{in}\, \NDnt{s_{{\mathrm{3}}}}   \NDsym{:}  \NDnt{C}} \\
           {\Delta_{{\mathrm{1}}}  \vdash_\mathcal{L}  \NDnt{s_{{\mathrm{2}}}}  \NDsym{:}  \NDnt{A}  \triangleright  \NDnt{B}}
        }{\Gamma_{{\mathrm{1}}}  \NDsym{,}  \Delta_{{\mathrm{1}}}  \NDsym{,}  \Gamma_{{\mathrm{2}}}  \NDsym{,}  \Delta_{{\mathrm{2}}}  \vdash_\mathcal{L}   \mathsf{let}\, \NDnt{s_{{\mathrm{2}}}}  :  \NDnt{A}  \triangleright  \NDnt{B} \,\mathsf{be}\, \NDmv{x}  \triangleright  \NDmv{y} \,\mathsf{in}\, \NDsym{(}   \mathsf{let}\, \NDnt{s_{{\mathrm{1}}}}  :   \mathsf{UnitS}  \,\mathsf{be}\,  \mathsf{trivS}  \,\mathsf{in}\, \NDnt{s_{{\mathrm{3}}}}   \NDsym{)}   \NDsym{:}  \NDnt{C}}
      \end{math}
    \end{center}

  \item (\NDdruleSXXtenETwoName, \NDdruleSXXtenETwoName):
    \begin{center}
      \tiny
      \begin{math}
        $$\mprset{flushleft}
        \inferrule* [right={\tiny tenE2}] {
          $$\mprset{flushleft}
          \inferrule* [right={\tiny tenE2}] {
            {\Gamma_{{\mathrm{1}}}  \NDsym{,}  \NDmv{x}  \NDsym{:}  \NDnt{A_{{\mathrm{2}}}}  \NDsym{,}  \NDmv{y}  \NDsym{:}  \NDnt{B_{{\mathrm{2}}}}  \NDsym{,}  \Gamma_{{\mathrm{2}}}  \vdash_\mathcal{L}  \NDnt{s_{{\mathrm{1}}}}  \NDsym{:}  \NDnt{A_{{\mathrm{1}}}}  \triangleright  \NDnt{B_{{\mathrm{1}}}}} \\
            {\Gamma  \vdash_\mathcal{L}  \NDnt{s_{{\mathrm{2}}}}  \NDsym{:}  \NDnt{A_{{\mathrm{2}}}}  \triangleright  \NDnt{B_{{\mathrm{2}}}}}
          }{\Gamma_{{\mathrm{1}}}  \NDsym{,}  \Gamma  \NDsym{,}  \Gamma_{{\mathrm{2}}}  \vdash_\mathcal{L}   \mathsf{let}\, \NDnt{s_{{\mathrm{2}}}}  :  \NDnt{A_{{\mathrm{2}}}}  \triangleright  \NDnt{B_{{\mathrm{2}}}} \,\mathsf{be}\, \NDmv{x}  \triangleright  \NDmv{y} \,\mathsf{in}\, \NDnt{s_{{\mathrm{1}}}}   \NDsym{:}  \NDnt{A_{{\mathrm{1}}}}  \triangleright  \NDnt{B_{{\mathrm{1}}}}} \\
           {\Delta_{{\mathrm{1}}}  \NDsym{,}  \NDmv{w}  \NDsym{:}  \NDnt{A_{{\mathrm{1}}}}  \NDsym{,}  \NDmv{z}  \NDsym{:}  \NDnt{B_{{\mathrm{1}}}}  \NDsym{,}  \Delta_{{\mathrm{2}}}  \vdash_\mathcal{L}  \NDnt{s_{{\mathrm{3}}}}  \NDsym{:}  \NDnt{C}}
        }{\Delta_{{\mathrm{1}}}  \NDsym{,}  \Gamma_{{\mathrm{1}}}  \NDsym{,}  \Gamma  \NDsym{,}  \Gamma_{{\mathrm{2}}}  \NDsym{,}  \Delta_{{\mathrm{2}}}  \vdash_\mathcal{L}   \mathsf{let}\, \NDsym{(}   \mathsf{let}\, \NDnt{s_{{\mathrm{2}}}}  :  \NDnt{A_{{\mathrm{2}}}}  \triangleright  \NDnt{B_{{\mathrm{2}}}} \,\mathsf{be}\, \NDmv{x}  \triangleright  \NDmv{y} \,\mathsf{in}\, \NDnt{s_{{\mathrm{1}}}}   \NDsym{)}  :  \NDnt{A_{{\mathrm{1}}}}  \triangleright  \NDnt{B_{{\mathrm{1}}}} \,\mathsf{be}\, \NDmv{w}  \triangleright  \NDmv{z} \,\mathsf{in}\, \NDnt{s_{{\mathrm{3}}}}   \NDsym{:}  \NDnt{C}}
      \end{math}
    \end{center}
    commutes to
    \begin{center}
      \tiny
      \begin{math}
        $$\mprset{flushleft}
        \inferrule* [right={\tiny tenE2}] {
          $$\mprset{flushleft}
          \inferrule* [right={\tiny tenE2}] {
            {\Gamma_{{\mathrm{1}}}  \NDsym{,}  \NDmv{x}  \NDsym{:}  \NDnt{A_{{\mathrm{2}}}}  \NDsym{,}  \NDmv{y}  \NDsym{:}  \NDnt{B_{{\mathrm{2}}}}  \NDsym{,}  \Gamma_{{\mathrm{2}}}  \vdash_\mathcal{L}  \NDnt{s_{{\mathrm{1}}}}  \NDsym{:}  \NDnt{A_{{\mathrm{1}}}}  \triangleright  \NDnt{B_{{\mathrm{1}}}}} \\
            {\Delta_{{\mathrm{1}}}  \NDsym{,}  \NDmv{w}  \NDsym{:}  \NDnt{A_{{\mathrm{1}}}}  \NDsym{,}  \NDmv{z}  \NDsym{:}  \NDnt{B_{{\mathrm{1}}}}  \NDsym{,}  \Delta_{{\mathrm{2}}}  \vdash_\mathcal{L}  \NDnt{s_{{\mathrm{3}}}}  \NDsym{:}  \NDnt{C}}
          }{\Delta_{{\mathrm{1}}}  \NDsym{,}  \Gamma_{{\mathrm{1}}}  \NDsym{,}  \NDmv{x}  \NDsym{:}  \NDnt{A_{{\mathrm{2}}}}  \NDsym{,}  \NDmv{y}  \NDsym{:}  \NDnt{B_{{\mathrm{2}}}}  \NDsym{,}  \Gamma_{{\mathrm{2}}}  \NDsym{,}  \Delta_{{\mathrm{2}}}  \vdash_\mathcal{L}   \mathsf{let}\, \NDnt{s_{{\mathrm{1}}}}  :  \NDnt{A_{{\mathrm{1}}}}  \triangleright  \NDnt{B_{{\mathrm{1}}}} \,\mathsf{be}\, \NDmv{w}  \triangleright  \NDmv{z} \,\mathsf{in}\, \NDnt{s_{{\mathrm{3}}}}   \NDsym{:}  \NDnt{C}}
            {\Gamma  \vdash_\mathcal{L}  \NDnt{s_{{\mathrm{2}}}}  \NDsym{:}  \NDnt{A_{{\mathrm{2}}}}  \triangleright  \NDnt{B_{{\mathrm{2}}}}}
        }{\Delta_{{\mathrm{1}}}  \NDsym{,}  \Gamma_{{\mathrm{1}}}  \NDsym{,}  \Gamma  \NDsym{,}  \Gamma_{{\mathrm{2}}}  \NDsym{,}  \Delta_{{\mathrm{2}}}  \vdash_\mathcal{L}   \mathsf{let}\, \NDnt{s_{{\mathrm{2}}}}  :  \NDnt{A_{{\mathrm{2}}}}  \triangleright  \NDnt{B_{{\mathrm{2}}}} \,\mathsf{be}\, \NDmv{x}  \triangleright  \NDmv{y} \,\mathsf{in}\, \NDsym{(}   \mathsf{let}\, \NDnt{s_{{\mathrm{1}}}}  :  \NDnt{A_{{\mathrm{1}}}}  \triangleright  \NDnt{B_{{\mathrm{1}}}} \,\mathsf{be}\, \NDmv{x}  \triangleright  \NDmv{z} \,\mathsf{in}\, \NDnt{s_{{\mathrm{3}}}}   \NDsym{)}   \NDsym{:}  \NDnt{C}}
      \end{math}
    \end{center}

  \item (\NDdruleSXXtenETwoName, \NDdruleSXXimprEName):
    \begin{center}
      \tiny
      \begin{math}
        $$\mprset{flushleft}
        \inferrule* [right={\tiny imprE}] {
          $$\mprset{flushleft}
          \inferrule* [right={\tiny tenE2}] {
            {\Gamma_{{\mathrm{1}}}  \NDsym{,}  \NDmv{x}  \NDsym{:}  \NDnt{A_{{\mathrm{2}}}}  \NDsym{,}  \NDmv{y}  \NDsym{:}  \NDnt{B_{{\mathrm{2}}}}  \NDsym{,}  \Gamma_{{\mathrm{2}}}  \vdash_\mathcal{L}  \NDnt{s_{{\mathrm{1}}}}  \NDsym{:}  \NDnt{A_{{\mathrm{1}}}}  \rightharpoonup  \NDnt{B_{{\mathrm{1}}}}} \\
            {\Delta_{{\mathrm{1}}}  \vdash_\mathcal{L}  \NDnt{s_{{\mathrm{2}}}}  \NDsym{:}  \NDnt{A_{{\mathrm{2}}}}  \triangleright  \NDnt{B_{{\mathrm{2}}}}}
          }{\Gamma_{{\mathrm{1}}}  \NDsym{,}  \Delta_{{\mathrm{1}}}  \NDsym{,}  \Gamma_{{\mathrm{2}}}  \vdash_\mathcal{L}   \mathsf{let}\, \NDnt{s_{{\mathrm{2}}}}  :  \NDnt{A_{{\mathrm{2}}}}  \triangleright  \NDnt{B_{{\mathrm{2}}}} \,\mathsf{be}\, \NDmv{x}  \triangleright  \NDmv{y} \,\mathsf{in}\, \NDnt{s_{{\mathrm{1}}}}   \NDsym{:}  \NDnt{A_{{\mathrm{1}}}}  \rightharpoonup  \NDnt{B_{{\mathrm{1}}}}} \\
           {\Delta_{{\mathrm{2}}}  \vdash_\mathcal{L}  \NDnt{s_{{\mathrm{3}}}}  \NDsym{:}  \NDnt{A_{{\mathrm{1}}}}}
        }{\Gamma_{{\mathrm{1}}}  \NDsym{,}  \Delta_{{\mathrm{1}}}  \NDsym{,}  \Gamma_{{\mathrm{2}}}  \NDsym{,}  \Delta_{{\mathrm{2}}}  \vdash_\mathcal{L}   \mathsf{app}_r\, \NDsym{(}   \mathsf{let}\, \NDnt{s_{{\mathrm{2}}}}  :  \NDnt{A_{{\mathrm{2}}}}  \triangleright  \NDnt{B_{{\mathrm{2}}}} \,\mathsf{be}\, \NDmv{x}  \triangleright  \NDmv{y} \,\mathsf{in}\, \NDnt{s_{{\mathrm{1}}}}   \NDsym{)} \, \NDnt{s_{{\mathrm{3}}}}   \NDsym{:}  \NDnt{B_{{\mathrm{1}}}}}
      \end{math}
    \end{center}
    commutes to
    \begin{center}
      \tiny
      \begin{math}
        $$\mprset{flushleft}
        \inferrule* [right={\tiny tenE2}] {
          $$\mprset{flushleft}
          \inferrule* [right={\tiny imprE}] {
            {\Gamma_{{\mathrm{1}}}  \NDsym{,}  \NDmv{x}  \NDsym{:}  \NDnt{A_{{\mathrm{2}}}}  \NDsym{,}  \NDmv{y}  \NDsym{:}  \NDnt{B_{{\mathrm{2}}}}  \NDsym{,}  \Gamma_{{\mathrm{2}}}  \vdash_\mathcal{L}  \NDnt{s_{{\mathrm{1}}}}  \NDsym{:}  \NDnt{A_{{\mathrm{1}}}}  \rightharpoonup  \NDnt{B_{{\mathrm{1}}}}} \\
            {\Delta_{{\mathrm{2}}}  \vdash_\mathcal{L}  \NDnt{s_{{\mathrm{3}}}}  \NDsym{:}  \NDnt{A_{{\mathrm{1}}}}}
          }{\Gamma_{{\mathrm{1}}}  \NDsym{,}  \NDmv{x}  \NDsym{:}  \NDnt{A_{{\mathrm{2}}}}  \NDsym{,}  \NDmv{y}  \NDsym{:}  \NDnt{B_{{\mathrm{2}}}}  \NDsym{,}  \Gamma_{{\mathrm{2}}}  \NDsym{,}  \Delta_{{\mathrm{2}}}  \vdash_\mathcal{L}   \mathsf{app}_r\, \NDnt{s_{{\mathrm{1}}}} \, \NDnt{s_{{\mathrm{3}}}}   \NDsym{:}  \NDnt{B_{{\mathrm{1}}}}} \\
            {\Delta_{{\mathrm{1}}}  \vdash_\mathcal{L}  \NDnt{s_{{\mathrm{2}}}}  \NDsym{:}  \NDnt{A_{{\mathrm{2}}}}  \triangleright  \NDnt{B_{{\mathrm{2}}}}}
        }{\Gamma_{{\mathrm{1}}}  \NDsym{,}  \Delta_{{\mathrm{1}}}  \NDsym{,}  \Gamma_{{\mathrm{2}}}  \NDsym{,}  \Delta_{{\mathrm{2}}}  \vdash_\mathcal{L}   \mathsf{let}\, \NDnt{s_{{\mathrm{2}}}}  :  \NDnt{A_{{\mathrm{2}}}}  \triangleright  \NDnt{B_{{\mathrm{2}}}} \,\mathsf{be}\, \NDmv{x}  \triangleright  \NDmv{y} \,\mathsf{in}\, \NDsym{(}   \mathsf{app}_r\, \NDnt{s_{{\mathrm{1}}}} \, \NDnt{s_{{\mathrm{3}}}}   \NDsym{)}   \NDsym{:}  \NDnt{B_{{\mathrm{1}}}}}
      \end{math}
    \end{center}

  \item (\NDdruleSXXtenETwoName, \NDdruleSXXimplEName):
    \begin{center}
      \tiny
      \begin{math}
        $$\mprset{flushleft}
        \inferrule* [right={\tiny imprE}] {
          $$\mprset{flushleft}
          \inferrule* [right={\tiny tenE2}] {
            {\Gamma_{{\mathrm{1}}}  \NDsym{,}  \NDmv{x}  \NDsym{:}  \NDnt{A_{{\mathrm{2}}}}  \NDsym{,}  \NDmv{y}  \NDsym{:}  \NDnt{B_{{\mathrm{2}}}}  \NDsym{,}  \Gamma_{{\mathrm{2}}}  \vdash_\mathcal{L}  \NDnt{s_{{\mathrm{1}}}}  \NDsym{:}  \NDnt{B_{{\mathrm{1}}}}  \leftharpoonup  \NDnt{A_{{\mathrm{1}}}}} \\
            {\Delta_{{\mathrm{1}}}  \vdash_\mathcal{L}  \NDnt{s_{{\mathrm{2}}}}  \NDsym{:}  \NDnt{A_{{\mathrm{2}}}}  \triangleright  \NDnt{B_{{\mathrm{2}}}}}
          }{\Gamma_{{\mathrm{1}}}  \NDsym{,}  \Delta_{{\mathrm{1}}}  \NDsym{,}  \Gamma_{{\mathrm{2}}}  \vdash_\mathcal{L}   \mathsf{let}\, \NDnt{s_{{\mathrm{2}}}}  :  \NDnt{A_{{\mathrm{2}}}}  \triangleright  \NDnt{B_{{\mathrm{2}}}} \,\mathsf{be}\, \NDmv{x}  \triangleright  \NDmv{y} \,\mathsf{in}\, \NDnt{s_{{\mathrm{1}}}}   \NDsym{:}  \NDnt{B_{{\mathrm{1}}}}  \leftharpoonup  \NDnt{A_{{\mathrm{1}}}}} \\
           {\Delta_{{\mathrm{2}}}  \vdash_\mathcal{L}  \NDnt{s_{{\mathrm{3}}}}  \NDsym{:}  \NDnt{A_{{\mathrm{1}}}}}
        }{\Delta_{{\mathrm{2}}}  \NDsym{,}  \Gamma_{{\mathrm{1}}}  \NDsym{,}  \Delta_{{\mathrm{1}}}  \NDsym{,}  \Gamma_{{\mathrm{2}}}  \vdash_\mathcal{L}   \mathsf{app}_l\, \NDsym{(}   \mathsf{let}\, \NDnt{s_{{\mathrm{2}}}}  :  \NDnt{A_{{\mathrm{2}}}}  \triangleright  \NDnt{B_{{\mathrm{2}}}} \,\mathsf{be}\, \NDmv{x}  \triangleright  \NDmv{y} \,\mathsf{in}\, \NDnt{s_{{\mathrm{1}}}}   \NDsym{)} \, \NDnt{s_{{\mathrm{3}}}}   \NDsym{:}  \NDnt{B_{{\mathrm{1}}}}}
      \end{math}
    \end{center}
    commutes to
    \begin{center}
      \tiny
      \begin{math}
        $$\mprset{flushleft}
        \inferrule* [right={\tiny tenE2}] {
          $$\mprset{flushleft}
          \inferrule* [right={\tiny imprE}] {
            {\Gamma_{{\mathrm{1}}}  \NDsym{,}  \NDmv{x}  \NDsym{:}  \NDnt{A_{{\mathrm{2}}}}  \NDsym{,}  \NDmv{y}  \NDsym{:}  \NDnt{B_{{\mathrm{2}}}}  \NDsym{,}  \Gamma_{{\mathrm{2}}}  \vdash_\mathcal{L}  \NDnt{s_{{\mathrm{1}}}}  \NDsym{:}  \NDnt{B_{{\mathrm{1}}}}  \leftharpoonup  \NDnt{A_{{\mathrm{1}}}}} \\
            {\Delta_{{\mathrm{2}}}  \vdash_\mathcal{L}  \NDnt{s_{{\mathrm{3}}}}  \NDsym{:}  \NDnt{A_{{\mathrm{1}}}}}
          }{\Delta_{{\mathrm{2}}}  \NDsym{,}  \Gamma_{{\mathrm{1}}}  \NDsym{,}  \NDmv{x}  \NDsym{:}  \NDnt{A_{{\mathrm{2}}}}  \NDsym{,}  \NDmv{y}  \NDsym{:}  \NDnt{B_{{\mathrm{2}}}}  \NDsym{,}  \Gamma_{{\mathrm{2}}}  \vdash_\mathcal{L}   \mathsf{app}_l\, \NDnt{s_{{\mathrm{1}}}} \, \NDnt{s_{{\mathrm{3}}}}   \NDsym{:}  \NDnt{B_{{\mathrm{1}}}}} \\
            {\Delta_{{\mathrm{1}}}  \vdash_\mathcal{L}  \NDnt{s_{{\mathrm{2}}}}  \NDsym{:}  \NDnt{A_{{\mathrm{2}}}}  \triangleright  \NDnt{B_{{\mathrm{2}}}}}
        }{\Delta_{{\mathrm{2}}}  \NDsym{,}  \Gamma_{{\mathrm{1}}}  \NDsym{,}  \Delta_{{\mathrm{1}}}  \NDsym{,}  \Gamma_{{\mathrm{2}}}  \vdash_\mathcal{L}   \mathsf{let}\, \NDnt{s_{{\mathrm{2}}}}  :  \NDnt{A_{{\mathrm{2}}}}  \triangleright  \NDnt{B_{{\mathrm{2}}}} \,\mathsf{be}\, \NDmv{x}  \triangleright  \NDmv{y} \,\mathsf{in}\, \NDsym{(}   \mathsf{app}_l\, \NDnt{s_{{\mathrm{1}}}} \, \NDnt{s_{{\mathrm{3}}}}   \NDsym{)}   \NDsym{:}  \NDnt{B_{{\mathrm{1}}}}}
      \end{math}
    \end{center}
  
  \end{itemize}

\item Commutation of $F_E$:
  \begin{itemize}
  \item (\NDdruleSXXFEName, \NDdruleSXXunitETwoName):
    \begin{center}
      \tiny
      \begin{math}
        $$\mprset{flushleft}
        \inferrule* [right={\tiny unitE2}] {
          $$\mprset{flushleft}
          \inferrule* [right={\tiny FE}] {
            {\Gamma_{{\mathrm{1}}}  \NDsym{,}  \NDmv{x}  \NDsym{:}  \NDnt{X}  \NDsym{,}  \Gamma_{{\mathrm{2}}}  \vdash_\mathcal{L}  \NDnt{s_{{\mathrm{1}}}}  \NDsym{:}   \mathsf{UnitS} } \\
            {\Delta_{{\mathrm{1}}}  \vdash_\mathcal{L}  \NDmv{y}  \NDsym{:}   \mathsf{F} \NDnt{X} }
          }{\Gamma_{{\mathrm{1}}}  \NDsym{,}  \Delta_{{\mathrm{1}}}  \NDsym{,}  \Gamma_{{\mathrm{2}}}  \vdash_\mathcal{L}   \mathsf{let}\,  \mathsf{F} \NDmv{x}   :   \mathsf{F} \NDnt{X}  \,\mathsf{be}\, \NDmv{y} \,\mathsf{in}\, \NDnt{s_{{\mathrm{1}}}}   \NDsym{:}   \mathsf{UnitS} } \\
           {\Delta_{{\mathrm{2}}}  \vdash_\mathcal{L}  \NDnt{s_{{\mathrm{2}}}}  \NDsym{:}  \NDnt{A}}
        }{\Gamma_{{\mathrm{1}}}  \NDsym{,}  \Delta_{{\mathrm{1}}}  \NDsym{,}  \Gamma_{{\mathrm{2}}}  \NDsym{,}  \Delta_{{\mathrm{2}}}  \vdash_\mathcal{L}   \mathsf{let}\, \NDsym{(}   \mathsf{let}\,  \mathsf{F} \NDmv{x}   :   \mathsf{F} \NDnt{X}  \,\mathsf{be}\, \NDmv{y} \,\mathsf{in}\, \NDnt{s_{{\mathrm{1}}}}   \NDsym{)}  :   \mathsf{UnitS}  \,\mathsf{be}\,  \mathsf{trivS}  \,\mathsf{in}\, \NDnt{s_{{\mathrm{2}}}}   \NDsym{:}  \NDnt{A}}
      \end{math}
    \end{center}
    commutes to
    \begin{center}
      \tiny
      \begin{math}
        $$\mprset{flushleft}
        \inferrule* [right={\tiny FE}] {
          $$\mprset{flushleft}
          \inferrule* [right={\tiny unitE2}] {
            {\Gamma_{{\mathrm{1}}}  \NDsym{,}  \NDmv{x}  \NDsym{:}  \NDnt{X}  \NDsym{,}  \Gamma_{{\mathrm{2}}}  \vdash_\mathcal{L}  \NDnt{s_{{\mathrm{1}}}}  \NDsym{:}   \mathsf{UnitS} } \\
            {\Delta_{{\mathrm{2}}}  \vdash_\mathcal{L}  \NDnt{s_{{\mathrm{2}}}}  \NDsym{:}  \NDnt{A}}
          }{\Gamma_{{\mathrm{1}}}  \NDsym{,}  \NDmv{x}  \NDsym{:}  \NDnt{X}  \NDsym{,}  \Gamma_{{\mathrm{2}}}  \NDsym{,}  \Delta_{{\mathrm{2}}}  \vdash_\mathcal{L}   \mathsf{let}\, \NDnt{s_{{\mathrm{1}}}}  :   \mathsf{UnitS}  \,\mathsf{be}\,  \mathsf{trivS}  \,\mathsf{in}\, \NDnt{s_{{\mathrm{2}}}}   \NDsym{:}  \NDnt{A}} \\
           {\Delta_{{\mathrm{1}}}  \vdash_\mathcal{L}  \NDmv{y}  \NDsym{:}   \mathsf{F} \NDnt{X} }
        }{\Gamma_{{\mathrm{1}}}  \NDsym{,}  \Delta_{{\mathrm{1}}}  \NDsym{,}  \Gamma_{{\mathrm{2}}}  \NDsym{,}  \Delta_{{\mathrm{2}}}  \vdash_\mathcal{L}   \mathsf{let}\,  \mathsf{F} \NDmv{x}   :   \mathsf{F} \NDnt{X}  \,\mathsf{be}\, \NDmv{y} \,\mathsf{in}\, \NDsym{(}   \mathsf{let}\, \NDnt{s_{{\mathrm{1}}}}  :   \mathsf{UnitS}  \,\mathsf{be}\,  \mathsf{trivS}  \,\mathsf{in}\, \NDnt{s_{{\mathrm{2}}}}   \NDsym{)}   \NDsym{:}  \NDnt{A}}
      \end{math}
    \end{center}
  \item (\NDdruleSXXFEName, \NDdruleSXXtenETwoName):
    \begin{center}
      \tiny
      \begin{math}
        $$\mprset{flushleft}
        \inferrule* [right={\tiny tenE2}] {
          $$\mprset{flushleft}
          \inferrule* [right={\tiny FE}] {
            {\Gamma_{{\mathrm{1}}}  \NDsym{,}  \NDmv{x}  \NDsym{:}  \NDnt{X}  \NDsym{,}  \Gamma_{{\mathrm{2}}}  \vdash_\mathcal{L}  \NDnt{s_{{\mathrm{1}}}}  \NDsym{:}  \NDnt{A}  \triangleright  \NDnt{B}} \\
            {\Delta  \vdash_\mathcal{L}  \NDmv{y}  \NDsym{:}   \mathsf{F} \NDnt{X} }
          }{\Gamma_{{\mathrm{1}}}  \NDsym{,}  \Delta  \NDsym{,}  \Gamma_{{\mathrm{2}}}  \vdash_\mathcal{L}   \mathsf{let}\,  \mathsf{F} \NDmv{x}   :   \mathsf{F} \NDnt{X}  \,\mathsf{be}\, \NDmv{y} \,\mathsf{in}\, \NDnt{s_{{\mathrm{1}}}}   \NDsym{:}  \NDnt{A}  \triangleright  \NDnt{B}} \\
           {\Delta_{{\mathrm{1}}}  \NDsym{,}  \NDmv{x}  \NDsym{:}  \NDnt{A}  \NDsym{,}  \NDmv{y}  \NDsym{:}  \NDnt{B}  \NDsym{,}  \Delta_{{\mathrm{2}}}  \vdash_\mathcal{L}  \NDnt{s_{{\mathrm{2}}}}  \NDsym{:}  \NDnt{C}}
        }{\Delta_{{\mathrm{1}}}  \NDsym{,}  \Gamma_{{\mathrm{1}}}  \NDsym{,}  \Delta  \NDsym{,}  \Gamma_{{\mathrm{2}}}  \NDsym{,}  \Delta_{{\mathrm{2}}}  \vdash_\mathcal{L}   \mathsf{let}\, \NDsym{(}   \mathsf{let}\,  \mathsf{F} \NDmv{x}   :   \mathsf{F} \NDnt{X}  \,\mathsf{be}\, \NDmv{y} \,\mathsf{in}\, \NDnt{s_{{\mathrm{1}}}}   \NDsym{)}  :  \NDnt{A}  \triangleright  \NDnt{B} \,\mathsf{be}\, \NDmv{x}  \triangleright  \NDmv{y} \,\mathsf{in}\, \NDnt{s_{{\mathrm{2}}}}   \NDsym{:}  \NDnt{C}}
      \end{math}
    \end{center}
    commutes to
    \begin{center}
      \tiny
      \begin{math}
        $$\mprset{flushleft}
        \inferrule* [right={\tiny FE}] {
          $$\mprset{flushleft}
          \inferrule* [right={\tiny tenE2}] {
            {\Gamma_{{\mathrm{1}}}  \NDsym{,}  \NDmv{x}  \NDsym{:}  \NDnt{X}  \NDsym{,}  \Gamma_{{\mathrm{2}}}  \vdash_\mathcal{L}  \NDnt{s_{{\mathrm{1}}}}  \NDsym{:}  \NDnt{A}  \triangleright  \NDnt{B}} \\
            {\Delta_{{\mathrm{1}}}  \NDsym{,}  \NDmv{x}  \NDsym{:}  \NDnt{A}  \NDsym{,}  \NDmv{y}  \NDsym{:}  \NDnt{B}  \NDsym{,}  \Delta_{{\mathrm{2}}}  \vdash_\mathcal{L}  \NDnt{s_{{\mathrm{2}}}}  \NDsym{:}  \NDnt{C}}
          }{\Delta_{{\mathrm{1}}}  \NDsym{,}  \Gamma_{{\mathrm{1}}}  \NDsym{,}  \NDmv{x}  \NDsym{:}  \NDnt{X}  \NDsym{,}  \Gamma_{{\mathrm{2}}}  \NDsym{,}  \Delta_{{\mathrm{2}}}  \vdash_\mathcal{L}   \mathsf{let}\, \NDnt{s_{{\mathrm{1}}}}  :  \NDnt{A}  \triangleright  \NDnt{B} \,\mathsf{be}\, \NDmv{x}  \triangleright  \NDmv{y} \,\mathsf{in}\, \NDnt{s_{{\mathrm{2}}}}   \NDsym{:}  \NDnt{C}} \\
           {\Delta  \vdash_\mathcal{L}  \NDmv{y}  \NDsym{:}   \mathsf{F} \NDnt{X} }
        }{\Delta_{{\mathrm{1}}}  \NDsym{,}  \Gamma_{{\mathrm{1}}}  \NDsym{,}  \Delta  \NDsym{,}  \Gamma_{{\mathrm{2}}}  \NDsym{,}  \Delta_{{\mathrm{2}}}  \vdash_\mathcal{L}   \mathsf{let}\,  \mathsf{F} \NDmv{x}   :   \mathsf{F} \NDnt{X}  \,\mathsf{be}\, \NDmv{y} \,\mathsf{in}\, \NDsym{(}   \mathsf{let}\, \NDnt{s_{{\mathrm{1}}}}  :  \NDnt{A}  \triangleright  \NDnt{B} \,\mathsf{be}\, \NDmv{x}  \triangleright  \NDmv{y} \,\mathsf{in}\, \NDnt{s_{{\mathrm{2}}}}   \NDsym{)}   \NDsym{:}  \NDnt{C}}
      \end{math}
    \end{center}
  \item (\NDdruleSXXFEName, \NDdruleSXXimprEName):
    \begin{center}
      \tiny
      \begin{math}
        $$\mprset{flushleft}
        \inferrule* [right={\tiny imprE}] {
          $$\mprset{flushleft}
          \inferrule* [right={\tiny FE}] {
            {\Gamma_{{\mathrm{1}}}  \NDsym{,}  \NDmv{x}  \NDsym{:}  \NDnt{X}  \NDsym{,}  \Gamma_{{\mathrm{2}}}  \vdash_\mathcal{L}  \NDnt{s_{{\mathrm{1}}}}  \NDsym{:}  \NDnt{A}  \rightharpoonup  \NDnt{B}} \\
            {\Delta_{{\mathrm{1}}}  \vdash_\mathcal{L}  \NDmv{y}  \NDsym{:}   \mathsf{F} \NDnt{X} }
          }{\Gamma_{{\mathrm{1}}}  \NDsym{,}  \Delta_{{\mathrm{1}}}  \NDsym{,}  \Gamma_{{\mathrm{2}}}  \vdash_\mathcal{L}   \mathsf{let}\,  \mathsf{F} \NDmv{x}   :   \mathsf{F} \NDnt{X}  \,\mathsf{be}\, \NDmv{y} \,\mathsf{in}\, \NDnt{s_{{\mathrm{1}}}}   \NDsym{:}  \NDnt{A}  \rightharpoonup  \NDnt{B}} \\
           {\Delta_{{\mathrm{2}}}  \vdash_\mathcal{L}  \NDnt{s_{{\mathrm{2}}}}  \NDsym{:}  \NDnt{A}}
        }{\Gamma_{{\mathrm{1}}}  \NDsym{,}  \Delta_{{\mathrm{1}}}  \NDsym{,}  \Gamma_{{\mathrm{2}}}  \NDsym{,}  \Delta_{{\mathrm{2}}}  \vdash_\mathcal{L}   \mathsf{app}_r\, \NDsym{(}   \mathsf{let}\,  \mathsf{F} \NDmv{x}   :   \mathsf{F} \NDnt{X}  \,\mathsf{be}\, \NDmv{y} \,\mathsf{in}\, \NDnt{s_{{\mathrm{1}}}}   \NDsym{)} \, \NDnt{s_{{\mathrm{2}}}}   \NDsym{:}  \NDnt{B}}
      \end{math}
    \end{center}
    commutes to
    \begin{center}
      \tiny
      \begin{math}
        $$\mprset{flushleft}
        \inferrule* [right={\tiny FE}] {
          $$\mprset{flushleft}
          \inferrule* [right={\tiny imprE}] {
            {\Gamma_{{\mathrm{1}}}  \NDsym{,}  \NDmv{x}  \NDsym{:}  \NDnt{X}  \NDsym{,}  \Gamma_{{\mathrm{2}}}  \vdash_\mathcal{L}  \NDnt{s_{{\mathrm{1}}}}  \NDsym{:}  \NDnt{A}  \rightharpoonup  \NDnt{B}} \\
            {\Delta_{{\mathrm{2}}}  \vdash_\mathcal{L}  \NDnt{s_{{\mathrm{2}}}}  \NDsym{:}  \NDnt{A}}
          }{\Gamma_{{\mathrm{1}}}  \NDsym{,}  \NDmv{x}  \NDsym{:}  \NDnt{X}  \NDsym{,}  \Gamma_{{\mathrm{2}}}  \NDsym{,}  \Delta_{{\mathrm{2}}}  \vdash_\mathcal{L}   \mathsf{app}_r\, \NDnt{s_{{\mathrm{1}}}} \, \NDnt{s_{{\mathrm{2}}}}   \NDsym{:}  \NDnt{B}} \\
           {\Delta_{{\mathrm{1}}}  \vdash_\mathcal{L}  \NDmv{y}  \NDsym{:}   \mathsf{F} \NDnt{X} }
        }{\Gamma_{{\mathrm{1}}}  \NDsym{,}  \Delta_{{\mathrm{1}}}  \NDsym{,}  \Gamma_{{\mathrm{2}}}  \NDsym{,}  \Delta_{{\mathrm{2}}}  \vdash_\mathcal{L}   \mathsf{let}\,  \mathsf{F} \NDmv{x}   :   \mathsf{F} \NDnt{X}  \,\mathsf{be}\, \NDmv{y} \,\mathsf{in}\, \NDsym{(}   \mathsf{app}_r\, \NDnt{s_{{\mathrm{1}}}} \, \NDnt{s_{{\mathrm{2}}}}   \NDsym{)}   \NDsym{:}  \NDnt{B}}
      \end{math}
    \end{center}
  \item (\NDdruleSXXFEName, \NDdruleSXXimplEName):
    \begin{center}
      \tiny
      \begin{math}
        $$\mprset{flushleft}
        \inferrule* [right={\tiny implE}] {
          $$\mprset{flushleft}
          \inferrule* [right={\tiny FE}] {
            {\Gamma_{{\mathrm{1}}}  \NDsym{,}  \NDmv{x}  \NDsym{:}  \NDnt{X}  \NDsym{,}  \Gamma_{{\mathrm{2}}}  \vdash_\mathcal{L}  \NDnt{s_{{\mathrm{1}}}}  \NDsym{:}  \NDnt{A}  \leftharpoonup  \NDnt{B}} \\
            {\Delta_{{\mathrm{1}}}  \vdash_\mathcal{L}  \NDmv{y}  \NDsym{:}   \mathsf{F} \NDnt{X} }
          }{\Gamma_{{\mathrm{1}}}  \NDsym{,}  \Delta_{{\mathrm{1}}}  \NDsym{,}  \Gamma_{{\mathrm{2}}}  \vdash_\mathcal{L}   \mathsf{let}\,  \mathsf{F} \NDmv{x}   :   \mathsf{F} \NDnt{X}  \,\mathsf{be}\, \NDmv{y} \,\mathsf{in}\, \NDnt{s_{{\mathrm{1}}}}   \NDsym{:}  \NDnt{A}  \leftharpoonup  \NDnt{B}} \\
           {\Delta_{{\mathrm{2}}}  \vdash_\mathcal{L}  \NDnt{s_{{\mathrm{2}}}}  \NDsym{:}  \NDnt{A}}
        }{\Delta_{{\mathrm{2}}}  \NDsym{,}  \Gamma_{{\mathrm{1}}}  \NDsym{,}  \Delta_{{\mathrm{1}}}  \NDsym{,}  \Gamma_{{\mathrm{2}}}  \vdash_\mathcal{L}   \mathsf{app}_l\, \NDsym{(}   \mathsf{let}\,  \mathsf{F} \NDmv{x}   :   \mathsf{F} \NDnt{X}  \,\mathsf{be}\, \NDmv{y} \,\mathsf{in}\, \NDnt{s_{{\mathrm{1}}}}   \NDsym{)} \, \NDnt{s_{{\mathrm{2}}}}   \NDsym{:}  \NDnt{B}}
      \end{math}
    \end{center}
    commutes to
    \begin{center}
      \tiny
      \begin{math}
        $$\mprset{flushleft}
        \inferrule* [right={\tiny FE}] {
          $$\mprset{flushleft}
          \inferrule* [right={\tiny imprE}] {
            {\Gamma_{{\mathrm{1}}}  \NDsym{,}  \NDmv{x}  \NDsym{:}  \NDnt{X}  \NDsym{,}  \Gamma_{{\mathrm{2}}}  \vdash_\mathcal{L}  \NDnt{s_{{\mathrm{1}}}}  \NDsym{:}  \NDnt{A}  \leftharpoonup  \NDnt{B}} \\
            {\Delta_{{\mathrm{2}}}  \vdash_\mathcal{L}  \NDnt{s_{{\mathrm{2}}}}  \NDsym{:}  \NDnt{A}}
          }{\Delta_{{\mathrm{2}}}  \NDsym{,}  \Gamma_{{\mathrm{1}}}  \NDsym{,}  \NDmv{x}  \NDsym{:}  \NDnt{X}  \NDsym{,}  \Gamma_{{\mathrm{2}}}  \vdash_\mathcal{L}   \mathsf{app}_l\, \NDnt{s_{{\mathrm{1}}}} \, \NDnt{s_{{\mathrm{2}}}}   \NDsym{:}  \NDnt{B}} \\
           {\Delta_{{\mathrm{1}}}  \vdash_\mathcal{L}  \NDmv{y}  \NDsym{:}   \mathsf{F} \NDnt{X} }
        }{\Delta_{{\mathrm{2}}}  \NDsym{,}  \Gamma_{{\mathrm{1}}}  \NDsym{,}  \Delta_{{\mathrm{1}}}  \NDsym{,}  \Gamma_{{\mathrm{2}}}  \vdash_\mathcal{L}   \mathsf{let}\,  \mathsf{F} \NDmv{x}   :   \mathsf{F} \NDnt{X}  \,\mathsf{be}\, \NDmv{y} \,\mathsf{in}\, \NDsym{(}   \mathsf{app}_l\, \NDnt{s_{{\mathrm{1}}}} \, \NDnt{s_{{\mathrm{2}}}}   \NDsym{)}   \NDsym{:}  \NDnt{B}}
      \end{math}
    \end{center}

  \end{itemize}

\end{itemize}



%%%%%%%%%%%%%%%%%%%%%%%%%%%%%%%%%%%%%%%%%%%%%%%%%%
\subsection{Mappings Between Sequent Calculus and Natural Deduction}

Function $S:ND\rightarrow SE$ maps a proof in the natural deduction to a proof of the same
sequent in the sequent calculus. The function is defined as follows:

\begin{itemize}
\item The axioms map to axioms.
\item Introduction rules map to right rules.
\item Elimination rules map to combinations of left rules with cuts:
  \begin{itemize}
  \item \NDdruleTXXunitEName:
    \begin{center}
      \tiny
      $\NDdruleTXXunitE{}$
    \end{center}
    maps to
    \begin{center}
      \tiny
      \begin{math}
        $$\mprset{flushleft}
        \inferrule* [right={\tiny cut}] {
          {\Phi  \vdash_\mathcal{C}  \NDnt{t_{{\mathrm{1}}}}  \NDsym{:}   \mathsf{UnitT} } \\
          $$\mprset{flushleft}
          \inferrule* [right={\tiny unitL}] {
            {\Psi  \vdash_\mathcal{C}  \NDnt{t_{{\mathrm{2}}}}  \NDsym{:}  \NDnt{Y}}
          }{\NDmv{x}  \NDsym{:}   \mathsf{UnitT}   \NDsym{,}  \Psi  \vdash_\mathcal{C}   \mathsf{let}\, \NDmv{x}  :   \mathsf{UnitT}  \,\mathsf{be}\,  \mathsf{trivT}  \,\mathsf{in}\, \NDnt{t_{{\mathrm{2}}}}   \NDsym{:}  \NDnt{Y}}
        }{\Phi  \NDsym{,}  \Psi  \vdash_\mathcal{C}  \NDsym{[}  \NDnt{t_{{\mathrm{1}}}}  \NDsym{/}  \NDmv{x}  \NDsym{]}  \NDsym{(}   \mathsf{let}\, \NDmv{x}  :   \mathsf{UnitT}  \,\mathsf{be}\,  \mathsf{trivT}  \,\mathsf{in}\, \NDnt{t_{{\mathrm{2}}}}   \NDsym{)}  \NDsym{:}  \NDnt{Y}}
      \end{math}
    \end{center}
  \item \NDdruleTXXtenEName:
    \begin{center}
      \tiny
      $\NDdruleTXXtenE{}$
    \end{center}
    maps to
    \begin{center}
      \tiny
      \begin{math}
        $$\mprset{flushleft}
        \inferrule* [right={\tiny cut}] {
          {\Phi  \vdash_\mathcal{C}  \NDnt{t_{{\mathrm{1}}}}  \NDsym{:}  \NDnt{X}  \otimes  \NDnt{Y}} \\
          $$\mprset{flushleft}
          \inferrule* [right={\tiny unitL}] {
            {\Psi_{{\mathrm{1}}}  \NDsym{,}  \NDmv{x}  \NDsym{:}  \NDnt{X}  \NDsym{,}  \NDmv{y}  \NDsym{:}  \NDnt{Y}  \NDsym{,}  \Psi_{{\mathrm{2}}}  \vdash_\mathcal{C}  \NDnt{t_{{\mathrm{2}}}}  \NDsym{:}  \NDnt{Z}}
          }{\Psi_{{\mathrm{1}}}  \NDsym{,}  \NDmv{z}  \NDsym{:}  \NDnt{X}  \otimes  \NDnt{Y}  \NDsym{,}  \Psi_{{\mathrm{2}}}  \vdash_\mathcal{C}   \mathsf{let}\, \NDmv{z}  :  \NDnt{X}  \otimes  \NDnt{Y} \,\mathsf{be}\, \NDmv{x}  \otimes  \NDmv{y} \,\mathsf{in}\, \NDnt{t_{{\mathrm{2}}}}   \NDsym{:}  \NDnt{Z}}
        }{\Psi_{{\mathrm{1}}}  \NDsym{,}  \Phi  \NDsym{,}  \Psi_{{\mathrm{2}}}  \vdash_\mathcal{C}  \NDsym{[}  \NDnt{t_{{\mathrm{1}}}}  \NDsym{/}  \NDmv{z}  \NDsym{]}  \NDsym{(}   \mathsf{let}\, \NDmv{z}  :  \NDnt{X}  \otimes  \NDnt{Y} \,\mathsf{be}\, \NDmv{x}  \otimes  \NDmv{y} \,\mathsf{in}\, \NDnt{t_{{\mathrm{2}}}}   \NDsym{)}  \NDsym{:}  \NDnt{Z}}
      \end{math}
    \end{center}
  \item \NDdruleTXXimpEName:
    \begin{center}
      \tiny
      $\NDdruleTXXimpE{}$
    \end{center}
    maps to
    \begin{center}
      \tiny
      \begin{math}
        $$\mprset{flushleft}
        \inferrule* [right={\tiny cut}] {
          {\Phi  \vdash_\mathcal{C}  \NDnt{t_{{\mathrm{1}}}}  \NDsym{:}  \NDnt{X}  \multimap  \NDnt{Y}} \\
          $$\mprset{flushleft}
          \inferrule* [right={\tiny unitL}] {
            {\Psi  \vdash_\mathcal{C}  \NDnt{t_{{\mathrm{2}}}}  \NDsym{:}  \NDnt{X}} \\
            {\NDmv{x}  \NDsym{:}  \NDnt{Y}  \vdash_\mathcal{C}  \NDmv{x}  \NDsym{:}  \NDnt{Y}}
          }{\NDmv{y}  \NDsym{:}  \NDnt{X}  \multimap  \NDnt{Y}  \NDsym{,}  \Psi  \vdash_\mathcal{C}  \NDsym{[}   \mathsf{app}\, \NDmv{y} \, \NDnt{t_{{\mathrm{2}}}}   \NDsym{/}  \NDmv{x}  \NDsym{]}  \NDmv{x}  \NDsym{:}  \NDnt{Y}}
        }{\Phi  \NDsym{,}  \Psi  \vdash_\mathcal{C}  \NDsym{[}  \NDnt{t_{{\mathrm{1}}}}  \NDsym{/}  \NDmv{y}  \NDsym{]}  \NDsym{[}   \mathsf{app}\, \NDmv{y} \, \NDnt{t_{{\mathrm{2}}}}   \NDsym{/}  \NDmv{x}  \NDsym{]}  \NDmv{x}  \NDsym{:}  \NDnt{Y}}
      \end{math}
    \end{center}
  \item \NDdruleSXXunitEOneName:
    \begin{center}
      \tiny
      $\NDdruleSXXunitEOne{}$
    \end{center}
    maps to
    \begin{center}
      \tiny
      \begin{math}
        $$\mprset{flushleft}
        \inferrule* [right={\tiny cut1}] {
          {\Phi  \vdash_\mathcal{C}  \NDnt{t}  \NDsym{:}   \mathsf{UnitT} } \\
          $$\mprset{flushleft}
          \inferrule* [right={\tiny unitL1}] {
            {\Gamma  \vdash_\mathcal{L}  \NDnt{s}  \NDsym{:}  \NDnt{A}}
          }{\NDmv{x}  \NDsym{:}   \mathsf{UnitT}   \NDsym{,}  \Gamma  \vdash_\mathcal{L}   \mathsf{let}\, \NDmv{x}  :   \mathsf{UnitT}  \,\mathsf{be}\,  \mathsf{trivT}  \,\mathsf{in}\, \NDnt{s}   \NDsym{:}  \NDnt{A}}
        }{\Phi  \NDsym{,}  \Psi  \vdash_\mathcal{L}  \NDsym{[}  \NDnt{t}  \NDsym{/}  \NDmv{x}  \NDsym{]}  \NDsym{(}   \mathsf{let}\, \NDmv{x}  :   \mathsf{UnitT}  \,\mathsf{be}\,  \mathsf{trivT}  \,\mathsf{in}\, \NDnt{s}   \NDsym{)}  \NDsym{:}  \NDnt{A}}
      \end{math}
    \end{center}
  \item \NDdruleSXXunitETwoName:
    \begin{center}
      \tiny
      $\NDdruleSXXunitETwo{}$
    \end{center}
    maps to
    \begin{center}
      \tiny
      \begin{math}
        $$\mprset{flushleft}
        \inferrule* [right={\tiny cut2}] {
          {\Gamma  \vdash_\mathcal{L}  \NDnt{s_{{\mathrm{1}}}}  \NDsym{:}   \mathsf{UnitS} } \\
          $$\mprset{flushleft}
          \inferrule* [right={\tiny unitL2}] {
            {\Delta  \vdash_\mathcal{L}  \NDnt{s_{{\mathrm{2}}}}  \NDsym{:}  \NDnt{A}}
          }{\NDmv{x}  \NDsym{:}   \mathsf{UnitS}   \NDsym{,}  \Delta  \vdash_\mathcal{L}   \mathsf{let}\, \NDmv{x}  :   \mathsf{UnitS}  \,\mathsf{be}\,  \mathsf{trivS}  \,\mathsf{in}\, \NDnt{s_{{\mathrm{2}}}}   \NDsym{:}  \NDnt{A}}
        }{\Gamma  \NDsym{,}  \Delta  \vdash_\mathcal{L}  \NDsym{[}  \NDnt{s_{{\mathrm{1}}}}  \NDsym{/}  \NDmv{x}  \NDsym{]}  \NDsym{(}   \mathsf{let}\, \NDmv{x}  :   \mathsf{UnitS}  \,\mathsf{be}\,  \mathsf{trivS}  \,\mathsf{in}\, \NDnt{s_{{\mathrm{2}}}}   \NDsym{)}  \NDsym{:}  \NDnt{A}}
      \end{math}
    \end{center}
  \item \NDdruleSXXtenEOneName:
    \begin{center}
      \tiny
      $\NDdruleSXXtenEOne{}$
    \end{center}
    maps to
    \begin{center}
      \tiny
      \begin{math}
        $$\mprset{flushleft}
        \inferrule* [right={\tiny cut1}] {
          {\Phi  \vdash_\mathcal{C}  \NDnt{t}  \NDsym{:}  \NDnt{X}  \otimes  \NDnt{Y}} \\
          $$\mprset{flushleft}
          \inferrule* [right={\tiny tenL1}] {
            {\Gamma_{{\mathrm{1}}}  \NDsym{,}  \NDmv{x}  \NDsym{:}  \NDnt{X}  \NDsym{,}  \NDmv{y}  \NDsym{:}  \NDnt{Y}  \NDsym{,}  \Gamma_{{\mathrm{2}}}  \vdash_\mathcal{L}  \NDnt{s}  \NDsym{:}  \NDnt{A}}
          }{\Gamma_{{\mathrm{1}}}  \NDsym{,}  \NDmv{z}  \NDsym{:}  \NDnt{X}  \otimes  \NDnt{Y}  \NDsym{,}  \Gamma_{{\mathrm{2}}}  \vdash_\mathcal{L}   \mathsf{let}\, \NDmv{z}  :  \NDnt{X}  \otimes  \NDnt{Y} \,\mathsf{be}\, \NDmv{x}  \otimes  \NDmv{y} \,\mathsf{in}\, \NDnt{s}   \NDsym{:}  \NDnt{A}}
        }{\Gamma_{{\mathrm{1}}}  \NDsym{,}  \Phi  \NDsym{,}  \Gamma_{{\mathrm{2}}}  \vdash_\mathcal{L}  \NDsym{[}  \NDnt{t}  \NDsym{/}  \NDmv{z}  \NDsym{]}  \NDsym{(}   \mathsf{let}\, \NDmv{z}  :  \NDnt{X}  \otimes  \NDnt{Y} \,\mathsf{be}\, \NDmv{x}  \otimes  \NDmv{y} \,\mathsf{in}\, \NDnt{s}   \NDsym{)}  \NDsym{:}  \NDnt{A}}
      \end{math}
    \end{center}
  \item \NDdruleSXXtenETwoName:
    \begin{center}
      \tiny
      $\NDdruleSXXtenETwo{}$
    \end{center}
    maps to
    \begin{center}
      \tiny
      \begin{math}
        $$\mprset{flushleft}
        \inferrule* [right={\tiny cut2}] {
          {\Gamma  \vdash_\mathcal{L}  \NDnt{s_{{\mathrm{1}}}}  \NDsym{:}  \NDnt{A}  \triangleright  \NDnt{B}} \\
          $$\mprset{flushleft}
          \inferrule* [right={\tiny tenL2}] {
            {\Delta_{{\mathrm{1}}}  \NDsym{,}  \NDmv{x}  \NDsym{:}  \NDnt{A}  \NDsym{,}  \NDmv{y}  \NDsym{:}  \NDnt{B}  \NDsym{,}  \Delta_{{\mathrm{2}}}  \vdash_\mathcal{L}  \NDnt{s_{{\mathrm{2}}}}  \NDsym{:}  \NDnt{C}}
          }{\Delta_{{\mathrm{1}}}  \NDsym{,}  \NDmv{z}  \NDsym{:}  \NDnt{A}  \triangleright  \NDnt{B}  \NDsym{,}  \Delta_{{\mathrm{2}}}  \vdash_\mathcal{L}   \mathsf{let}\, \NDmv{z}  :  \NDnt{A}  \triangleright  \NDnt{B} \,\mathsf{be}\, \NDmv{x}  \triangleright  \NDmv{y} \,\mathsf{in}\, \NDnt{s_{{\mathrm{2}}}}   \NDsym{:}  \NDnt{C}}
        }{\Delta_{{\mathrm{1}}}  \NDsym{,}  \Gamma  \NDsym{,}  \Delta_{{\mathrm{2}}}  \vdash_\mathcal{L}  \NDsym{[}  \NDnt{s_{{\mathrm{1}}}}  \NDsym{/}  \NDmv{z}  \NDsym{]}  \NDsym{(}   \mathsf{let}\, \NDmv{z}  :  \NDnt{A}  \triangleright  \NDnt{B} \,\mathsf{be}\, \NDmv{x}  \triangleright  \NDmv{y} \,\mathsf{in}\, \NDnt{s_{{\mathrm{2}}}}   \NDsym{)}  \NDsym{:}  \NDnt{C}}
      \end{math}
    \end{center}
  \item \NDdruleSXXimprEName: (NOT SURE)
    \begin{center}
      \tiny
      $\NDdruleSXXimprE{}$
    \end{center}
    maps to
    \begin{center}
      \tiny
      \begin{math}
        $$\mprset{flushleft}
        \inferrule* [right={\tiny cut2}] {
          {\Gamma  \vdash_\mathcal{L}  \NDnt{s_{{\mathrm{1}}}}  \NDsym{:}  \NDnt{A}  \rightharpoonup  \NDnt{B}} \\
          $$\mprset{flushleft}
          \inferrule* [right={\tiny imprL}] {
            {\Delta  \vdash_\mathcal{L}  \NDnt{s_{{\mathrm{2}}}}  \NDsym{:}  \NDnt{A}} \\
            {\NDmv{x}  \NDsym{:}  \NDnt{B}  \vdash_\mathcal{L}  \NDmv{x}  \NDsym{:}  \NDnt{B}}
          }{\NDmv{y}  \NDsym{:}  \NDnt{A}  \rightharpoonup  \NDnt{B}  \NDsym{,}  \Delta  \vdash_\mathcal{L}  \NDsym{[}   \mathsf{app}_r\, \NDmv{y} \, \NDnt{s_{{\mathrm{2}}}}   \NDsym{/}  \NDmv{x}  \NDsym{]}  \NDmv{x}  \NDsym{:}  \NDnt{B}}
        }{\Gamma  \NDsym{,}  \Delta  \vdash_\mathcal{L}  \NDsym{[}  \NDnt{s_{{\mathrm{1}}}}  \NDsym{/}  \NDmv{y}  \NDsym{]}  \NDsym{[}   \mathsf{app}_r\, \NDmv{y} \, \NDnt{s_{{\mathrm{2}}}}   \NDsym{/}  \NDmv{x}  \NDsym{]}  \NDmv{x}  \NDsym{:}  \NDnt{B}}
      \end{math}
    \end{center}
  \item \NDdruleSXXimplEName: (NOT SURE)
    \begin{center}
      \tiny
      $\NDdruleSXXimplE{}$
    \end{center}
    maps to
    \begin{center}
      \tiny
      \begin{math}
        $$\mprset{flushleft}
        \inferrule* [right={\tiny cut2}] {
          {\Gamma  \vdash_\mathcal{L}  \NDnt{s_{{\mathrm{1}}}}  \NDsym{:}  \NDnt{B}  \leftharpoonup  \NDnt{A}} \\
          $$\mprset{flushleft}
          \inferrule* [right={\tiny implL}] {
            {\Delta  \vdash_\mathcal{L}  \NDnt{s_{{\mathrm{2}}}}  \NDsym{:}  \NDnt{A}} \\
            {\NDmv{x}  \NDsym{:}  \NDnt{B}  \vdash_\mathcal{L}  \NDmv{x}  \NDsym{:}  \NDnt{B}}
          }{\Delta  \NDsym{,}  \NDmv{y}  \NDsym{:}  \NDnt{B}  \leftharpoonup  \NDnt{A}  \vdash_\mathcal{L}  \NDsym{[}   \mathsf{app}_l\, \NDmv{y} \, \NDnt{s_{{\mathrm{2}}}}   \NDsym{/}  \NDmv{x}  \NDsym{]}  \NDmv{x}  \NDsym{:}  \NDnt{B}}
        }{\Delta  \NDsym{,}  \Gamma  \vdash_\mathcal{L}  \NDsym{[}  \NDnt{s_{{\mathrm{1}}}}  \NDsym{/}  \NDmv{y}  \NDsym{]}  \NDsym{[}   \mathsf{app}_l\, \NDmv{y} \, \NDnt{s_{{\mathrm{2}}}}   \NDsym{/}  \NDmv{x}  \NDsym{]}  \NDmv{x}  \NDsym{:}  \NDnt{B}}
      \end{math}
    \end{center}
  \item \NDdruleSXXFEName:
    \begin{center}
      \tiny
      $\NDdruleSXXFE{}$
    \end{center}
    maps to
    \begin{center}
      \tiny
      \begin{math}
        $$\mprset{flushleft}
        \inferrule* [right={\tiny cut2}] {
          {\Gamma  \vdash_\mathcal{L}  \NDmv{y}  \NDsym{:}   \mathsf{F} \NDnt{X} } \\
          $$\mprset{flushleft}
          \inferrule* [right={\tiny FL}] {
            {\Delta_{{\mathrm{1}}}  \NDsym{,}  \NDmv{x}  \NDsym{:}  \NDnt{X}  \NDsym{,}  \Delta_{{\mathrm{2}}}  \vdash_\mathcal{L}  \NDnt{s}  \NDsym{:}  \NDnt{A}}
          }{\Delta_{{\mathrm{1}}}  \NDsym{,}  \NDmv{z}  \NDsym{:}   \mathsf{F} \NDnt{X}   \NDsym{,}  \Delta_{{\mathrm{2}}}  \vdash_\mathcal{L}   \mathsf{let}\, \NDmv{z}  :   \mathsf{F} \NDnt{X}  \,\mathsf{be}\,  \mathsf{F}\, \NDmv{x}  \,\mathsf{in}\, \NDnt{s}   \NDsym{:}  \NDnt{A}}
        }{\Delta_{{\mathrm{1}}}  \NDsym{,}  \Gamma  \NDsym{,}  \Delta_{{\mathrm{2}}}  \vdash_\mathcal{L}  \NDsym{[}  \NDmv{y}  \NDsym{/}  \NDmv{z}  \NDsym{]}  \NDsym{(}   \mathsf{let}\, \NDmv{y}  :   \mathsf{F} \NDnt{X}  \,\mathsf{be}\,  \mathsf{F}\, \NDmv{x}  \,\mathsf{in}\, \NDnt{s}   \NDsym{)}  \NDsym{:}  \NDnt{A}}
      \end{math}
    \end{center}
  \item \NDdruleSXXGEName:
    \begin{center}
      \tiny
      $\NDdruleSXXGE{}$
    \end{center}
    maps to
    \begin{center}
      \tiny
      \begin{math}
        $$\mprset{flushleft}
        \inferrule* [right={\tiny cut1}] {
          $$\mprset{flushleft}
          \inferrule* [right={\tiny GL}] {
            {\NDmv{x}  \NDsym{:}  \NDnt{A}  \vdash_\mathcal{L}  \NDmv{x}  \NDsym{:}  \NDnt{A}}
          }{\NDmv{y}  \NDsym{:}   \mathsf{G} \NDnt{A}   \vdash_\mathcal{L}   \mathsf{let}\, \NDmv{y}  :   \mathsf{G} \NDnt{A}  \,\mathsf{be}\,  \mathsf{G}\, \NDmv{x}  \,\mathsf{in}\, \NDmv{x}   \NDsym{:}  \NDnt{A}} \\
           {\Phi  \vdash_\mathcal{C}  \NDnt{t}  \NDsym{:}   \mathsf{G} \NDnt{A} }
        }{\Phi  \vdash_\mathcal{L}  \NDsym{[}  \NDnt{t}  \NDsym{/}  \NDmv{y}  \NDsym{]}  \NDsym{(}   \mathsf{let}\, \NDmv{y}  :   \mathsf{G} \NDnt{A}  \,\mathsf{be}\,  \mathsf{G}\, \NDmv{x}  \,\mathsf{in}\, \NDmv{x}   \NDsym{)}  \NDsym{:}  \NDnt{A}}
      \end{math}
    \end{center}
  \end{itemize}
\end{itemize}

Function $N:SE\rightarrow ND$ maps a proof in the sequent calculus to a proof of the same
sequent in the natural deduction. The function is defined as follows:

\begin{itemize}
\item Axioms map to axioms.
\item Instances of cut rules map to the admissible substitution rules.
\item Right rules map to introductions.
\item Left rules map to eliminations modulo some structural fiddling.
  \begin{itemize}
  \item \ElledruleTXXunitLName:
    \begin{center}
      \tiny
      $\ElledruleTXXunitL{}$
    \end{center}
    maps to
    \begin{center}
      \tiny
      \begin{math}
        $$\mprset{flushleft}
        \inferrule* [right={\tiny unitE}] {
          {\NDmv{x}  \NDsym{:}   \mathsf{UnitT}   \vdash_\mathcal{C}  \NDmv{x}  \NDsym{:}   \mathsf{UnitT} } \\
          {\Psi  \vdash_\mathcal{C}  \NDnt{t}  \NDsym{:}  \NDnt{X}}
        }{\NDmv{x}  \NDsym{:}   \mathsf{UnitT}   \NDsym{,}  \Psi  \vdash_\mathcal{C}   \mathsf{let}\, \NDmv{x}  :   \mathsf{UnitT}  \,\mathsf{be}\,  \mathsf{trivT}  \,\mathsf{in}\, \NDnt{t}   \NDsym{:}  \NDnt{X}}
      \end{math}
    \end{center}
  \item \ElledruleTXXtenLName:
    \begin{center}
      \tiny
      $\ElledruleTXXtenL{}$
    \end{center}
    maps to
    \begin{center}
      \tiny
      \begin{math}
        $$\mprset{flushleft}
        \inferrule* [right={\tiny tenE}] {
          {\NDmv{z}  \NDsym{:}  \NDnt{X}  \otimes  \NDnt{Y}  \vdash_\mathcal{C}  \NDmv{z}  \NDsym{:}  \NDnt{X}  \otimes  \NDnt{Y}} \\
          {\Phi  \NDsym{,}  \NDmv{x}  \NDsym{:}  \NDnt{X}  \NDsym{,}  \NDmv{y}  \NDsym{:}  \NDnt{Y}  \NDsym{,}  \Psi  \vdash_\mathcal{C}  \NDnt{t}  \NDsym{:}  \NDnt{Z}}
        }{\Phi  \NDsym{,}  \NDmv{z}  \NDsym{:}  \NDnt{X}  \otimes  \NDnt{Y}  \NDsym{,}  \Psi  \vdash_\mathcal{C}   \mathsf{let}\, \NDmv{z}  :  \NDnt{X}  \otimes  \NDnt{Y} \,\mathsf{be}\, \NDmv{x}  \otimes  \NDmv{y} \,\mathsf{in}\, \NDnt{t}   \NDsym{:}  \NDnt{Z}}
      \end{math}
    \end{center}
  \item \ElledruleTXXimpLName:
    \begin{center}
      \tiny
      $\ElledruleTXXimpL{}$
    \end{center}
    maps to
    \begin{center}
      \tiny
      \begin{math}
        $$\mprset{flushleft}
        \inferrule* [right={\tiny cut1}] {
          $$\mprset{flushleft}
          \inferrule* [right={\tiny impE}] {
            {\NDmv{z}  \NDsym{:}  \NDnt{X}  \multimap  \NDnt{Y}  \vdash_\mathcal{C}  \NDmv{z}  \NDsym{:}  \NDnt{X}  \multimap  \NDnt{Y}} \\
            {\Phi  \vdash_\mathcal{C}  \NDnt{t_{{\mathrm{1}}}}  \NDsym{:}  \NDnt{X}}
          }{\NDmv{z}  \NDsym{:}  \NDnt{X}  \multimap  \NDnt{Y}  \NDsym{,}  \Phi  \vdash_\mathcal{C}   \mathsf{app}\, \NDmv{z} \, \NDnt{t_{{\mathrm{1}}}}   \NDsym{:}  \NDnt{Y}} \\
           {\Psi_{{\mathrm{1}}}  \NDsym{,}  \NDmv{x}  \NDsym{:}  \NDnt{Y}  \NDsym{,}  \Psi_{{\mathrm{2}}}  \vdash_\mathcal{C}  \NDnt{t_{{\mathrm{2}}}}  \NDsym{:}  \NDnt{Z}}
        }{\Psi_{{\mathrm{1}}}  \NDsym{,}  \NDmv{z}  \NDsym{:}  \NDnt{X}  \multimap  \NDnt{Y}  \NDsym{,}  \Phi  \NDsym{,}  \Psi_{{\mathrm{2}}}  \vdash_\mathcal{C}  \NDsym{[}   \mathsf{app}\, \NDmv{z} \, \NDnt{t_{{\mathrm{2}}}}   \NDsym{/}  \NDmv{x}  \NDsym{]}  \NDnt{t_{{\mathrm{2}}}}  \NDsym{:}  \NDnt{Z}}
      \end{math}
    \end{center}
  \item \ElledruleSXXunitLOneName:
    \begin{center}
      \tiny
      $\ElledruleSXXunitLOne{}$
    \end{center}
    maps to
    \begin{center}
      \tiny
      \begin{math}
        $$\mprset{flushleft}
        \inferrule* [right={\tiny unitE1}] {
          {\NDmv{x}  \NDsym{:}   \mathsf{UnitT}   \vdash_\mathcal{C}  \NDmv{x}  \NDsym{:}   \mathsf{UnitT} } \\
          {\Delta  \vdash_\mathcal{L}  \NDnt{s}  \NDsym{:}  \NDnt{A}}
        }{\NDmv{x}  \NDsym{:}   \mathsf{UnitT}   \NDsym{,}  \Delta  \vdash_\mathcal{L}   \mathsf{let}\, \NDmv{x}  :   \mathsf{UnitT}  \,\mathsf{be}\,  \mathsf{trivT}  \,\mathsf{in}\, \NDnt{s}   \NDsym{:}  \NDnt{A}}
      \end{math}
    \end{center}
  \item \ElledruleSXXunitLTwoName:
    \begin{center}
      \tiny
      $\ElledruleSXXunitLTwo{}$
    \end{center}
    maps to
    \begin{center}
      \tiny
      \begin{math}
        $$\mprset{flushleft}
        \inferrule* [right={\tiny unitE2}] {
          {\NDmv{x}  \NDsym{:}   \mathsf{UnitS}   \vdash_\mathcal{L}  \NDmv{x}  \NDsym{:}   \mathsf{UnitS} } \\
          {\Delta  \vdash_\mathcal{L}  \NDnt{s}  \NDsym{:}  \NDnt{A}}
        }{\NDmv{x}  \NDsym{:}   \mathsf{UnitS}   \NDsym{,}  \Delta  \vdash_\mathcal{L}   \mathsf{let}\, \NDmv{x}  :   \mathsf{UnitS}  \,\mathsf{be}\,  \mathsf{trivS}  \,\mathsf{in}\, \NDnt{s}   \NDsym{:}  \NDnt{A}}
      \end{math}
    \end{center}
  \item \ElledruleSXXtenLOneName:
    \begin{center}
      \tiny
      $\ElledruleSXXtenLOne{}$
    \end{center}
    maps to
    \begin{center}
      \tiny
      \begin{math}
        $$\mprset{flushleft}
        \inferrule* [right={\tiny tenE1}] {
          {\NDmv{z}  \NDsym{:}  \NDnt{X}  \otimes  \NDnt{Y}  \vdash_\mathcal{C}  \NDmv{z}  \NDsym{:}  \NDnt{X}  \otimes  \NDnt{Y}} \\
          {\Gamma  \NDsym{,}  \NDmv{x}  \NDsym{:}  \NDnt{X}  \NDsym{,}  \NDmv{y}  \NDsym{:}  \NDnt{Y}  \NDsym{,}  \Delta  \vdash_\mathcal{L}  \NDnt{s}  \NDsym{:}  \NDnt{A}}
        }{\Gamma  \NDsym{,}  \NDmv{z}  \NDsym{:}  \NDnt{X}  \otimes  \NDnt{Y}  \NDsym{,}  \Delta  \vdash_\mathcal{L}   \mathsf{let}\, \NDmv{z}  :  \NDnt{X}  \otimes  \NDnt{Y} \,\mathsf{be}\, \NDmv{x}  \otimes  \NDmv{y} \,\mathsf{in}\, \NDnt{s}   \NDsym{:}  \NDnt{A}}
      \end{math}
    \end{center}
  \item \ElledruleSXXtenLTwoName:
    \begin{center}
      \tiny
      $\ElledruleSXXtenLTwo{}$
    \end{center}
    maps to
    \begin{center}
      \tiny
      \begin{math}
        $$\mprset{flushleft}
        \inferrule* [right={\tiny tenE2}] {
          {\NDmv{z}  \NDsym{:}  \NDnt{A}  \triangleright  \NDnt{B}  \vdash_\mathcal{L}  \NDmv{z}  \NDsym{:}  \NDnt{A}  \triangleright  \NDnt{B}} \\
          {\Gamma  \NDsym{,}  \NDmv{x}  \NDsym{:}  \NDnt{A}  \NDsym{,}  \NDmv{y}  \NDsym{:}  \NDnt{B}  \NDsym{,}  \Delta  \vdash_\mathcal{L}  \NDnt{s}  \NDsym{:}  \NDnt{C}}
        }{\Gamma  \NDsym{,}  \NDmv{z}  \NDsym{:}  \NDnt{A}  \triangleright  \NDnt{B}  \NDsym{,}  \Delta  \vdash_\mathcal{L}   \mathsf{let}\, \NDmv{z}  :  \NDnt{A}  \triangleright  \NDnt{B} \,\mathsf{be}\, \NDmv{x}  \triangleright  \NDmv{y} \,\mathsf{in}\, \NDnt{s}   \NDsym{:}  \NDnt{C}}
      \end{math}
    \end{center}
  \item \ElledruleSXXimpLName:
    \begin{center}
      \tiny
      $\ElledruleSXXimpL{}$
    \end{center}
    maps to
    \begin{center}
      \tiny
      \begin{math}
        $$\mprset{flushleft}
        \inferrule* [right={\tiny cut1}] {
          $$\mprset{flushleft}
          \inferrule* [right={\tiny impE}] {
            {\NDmv{z}  \NDsym{:}  \NDnt{X}  \multimap  \NDnt{Y}  \vdash_\mathcal{C}  \NDmv{z}  \NDsym{:}  \NDnt{X}  \multimap  \NDnt{Y}} \\
            {\Phi  \vdash_\mathcal{C}  \NDnt{t}  \NDsym{:}  \NDnt{X}}
          }{\NDmv{z}  \NDsym{:}  \NDnt{X}  \multimap  \NDnt{Y}  \NDsym{,}  \Phi  \vdash_\mathcal{C}   \mathsf{app}\, \NDmv{z} \, \NDnt{t}   \NDsym{:}  \NDnt{Y}} \\
           {\Gamma  \NDsym{,}  \NDmv{x}  \NDsym{:}  \NDnt{Y}  \NDsym{,}  \Delta  \vdash_\mathcal{L}  \NDnt{s}  \NDsym{:}  \NDnt{A}}
        }{\Gamma  \NDsym{,}  \NDmv{z}  \NDsym{:}  \NDnt{X}  \multimap  \NDnt{Y}  \NDsym{,}  \Phi  \NDsym{,}  \Delta  \vdash_\mathcal{L}  \NDsym{[}   \mathsf{app}\, \NDmv{z} \, \NDnt{t}   \NDsym{/}  \NDmv{x}  \NDsym{]}  \NDnt{s}  \NDsym{:}  \NDnt{A}}
      \end{math}
    \end{center}
  \item \ElledruleSXXimprLName:
    \begin{center}
      \tiny
      $\ElledruleSXXimprL{}$
    \end{center}
    maps to
    \begin{center}
      \tiny
      \begin{math}
        $$\mprset{flushleft}
        \inferrule* [right={\tiny cut2}] {
          $$\mprset{flushleft}
          \inferrule* [right={\tiny imprE}] {
            {\NDmv{z}  \NDsym{:}  \NDnt{A}  \rightharpoonup  \NDnt{B}  \vdash_\mathcal{L}  \NDmv{z}  \NDsym{:}  \NDnt{A}  \rightharpoonup  \NDnt{B}} \\
            {\Gamma  \vdash_\mathcal{L}  \NDnt{s_{{\mathrm{1}}}}  \NDsym{:}  \NDnt{A}}
          }{\NDmv{z}  \NDsym{:}  \NDnt{A}  \rightharpoonup  \NDnt{B}  \NDsym{,}  \Gamma  \vdash_\mathcal{L}   \mathsf{app}_r\, \NDmv{z} \, \NDnt{s_{{\mathrm{1}}}}   \NDsym{:}  \NDnt{B}} \\
           {\Delta  \NDsym{,}  \NDmv{x}  \NDsym{:}  \NDnt{B}  \vdash_\mathcal{L}  \NDnt{s_{{\mathrm{2}}}}  \NDsym{:}  \NDnt{C}}
        }{\Delta  \NDsym{,}  \NDmv{z}  \NDsym{:}  \NDnt{A}  \rightharpoonup  \NDnt{B}  \NDsym{,}  \Gamma  \vdash_\mathcal{L}  \NDsym{[}   \mathsf{app}_r\, \NDmv{z} \, \NDnt{s_{{\mathrm{1}}}}   \NDsym{/}  \NDmv{x}  \NDsym{]}  \NDnt{s_{{\mathrm{2}}}}  \NDsym{:}  \NDnt{C}}
      \end{math}
    \end{center}
  \item \ElledruleSXXimplLName:
    \begin{center}
      \tiny
      $\ElledruleSXXimplL{}$
    \end{center}
    maps to
    \begin{center}
      \tiny
      \begin{math}
        $$\mprset{flushleft}
        \inferrule* [right={\tiny cut2}] {
          $$\mprset{flushleft}
          \inferrule* [right={\tiny implE}] {
            {\NDmv{z}  \NDsym{:}  \NDnt{B}  \leftharpoonup  \NDnt{A}  \vdash_\mathcal{L}  \NDmv{z}  \NDsym{:}  \NDnt{B}  \leftharpoonup  \NDnt{A}} \\
            {\Gamma  \vdash_\mathcal{L}  \NDnt{s_{{\mathrm{1}}}}  \NDsym{:}  \NDnt{A}}
          }{\Gamma  \NDsym{,}  \NDmv{z}  \NDsym{:}  \NDnt{B}  \leftharpoonup  \NDnt{A}  \vdash_\mathcal{L}   \mathsf{app}_l\, \NDmv{z} \, \NDnt{s_{{\mathrm{1}}}}   \NDsym{:}  \NDnt{B}} \\
           {\NDmv{x}  \NDsym{:}  \NDnt{B}  \NDsym{,}  \Delta  \vdash_\mathcal{L}  \NDnt{s_{{\mathrm{2}}}}  \NDsym{:}  \NDnt{C}}
        }{\Gamma  \NDsym{,}  \NDmv{z}  \NDsym{:}  \NDnt{B}  \leftharpoonup  \NDnt{A}  \NDsym{,}  \Delta  \vdash_\mathcal{L}  \NDsym{[}   \mathsf{app}_l\, \NDmv{z} \, \NDnt{s_{{\mathrm{1}}}}   \NDsym{/}  \NDmv{x}  \NDsym{]}  \NDnt{s_{{\mathrm{2}}}}  \NDsym{:}  \NDnt{C}}
      \end{math}
    \end{center}
  \item \ElledruleSXXFlName:
    \begin{center}
      \tiny
      $\ElledruleSXXFl{}$
    \end{center}
    maps to
    \begin{center}
      \tiny
      \begin{math}
        $$\mprset{flushleft}
        \inferrule* [right={\tiny FE}] {
          {\NDmv{z}  \NDsym{:}   \mathsf{F} \NDnt{X}   \vdash_\mathcal{L}  \NDmv{z}  \NDsym{:}   \mathsf{F} \NDnt{X} } \\
          {\Gamma  \NDsym{,}  \NDmv{x}  \NDsym{:}  \NDnt{X}  \NDsym{,}  \Delta  \vdash_\mathcal{L}  \NDnt{s}  \NDsym{:}  \NDnt{A}}
        }{\Gamma  \NDsym{,}  \NDmv{z}  \NDsym{:}   \mathsf{F} \NDnt{X}   \NDsym{,}  \Delta  \vdash_\mathcal{L}   \mathsf{let}\,  \mathsf{F} \NDmv{x}   :   \mathsf{F} \NDnt{X}  \,\mathsf{be}\, \NDmv{z} \,\mathsf{in}\, \NDnt{s}   \NDsym{:}  \NDnt{A}}
      \end{math}
    \end{center}
  \item \ElledruleSXXGlName:
    \begin{center}
      \tiny
      $\ElledruleSXXGl{}$
    \end{center}
    maps to
    \begin{center}
      \tiny
      \begin{math}
        $$\mprset{flushleft}
        \inferrule* [right={\tiny cut2}] {
          $$\mprset{flushleft}
          \inferrule* [right={\tiny GE}] {
            {\NDmv{y}  \NDsym{:}   \mathsf{G} \NDnt{A}   \vdash_\mathcal{C}  \NDmv{y}  \NDsym{:}   \mathsf{G} \NDnt{A} }
          }{\NDmv{y}  \NDsym{:}   \mathsf{G} \NDnt{A}   \vdash_\mathcal{L}   \mathsf{derelict}\, \NDmv{y}   \NDsym{:}  \NDnt{A}} \\
           {\Gamma  \NDsym{,}  \NDmv{x}  \NDsym{:}  \NDnt{A}  \NDsym{,}  \Delta  \vdash_\mathcal{L}  \NDnt{s}  \NDsym{:}  \NDnt{B}}
        }{\Gamma  \NDsym{,}  \NDmv{y}  \NDsym{:}   \mathsf{G} \NDnt{A}   \NDsym{,}  \Delta  \vdash_\mathcal{L}  \NDsym{[}   \mathsf{derelict}\, \NDmv{y}   \NDsym{/}  \NDmv{x}  \NDsym{]}  \NDnt{s}  \NDsym{:}  \NDnt{B}}
      \end{math}
    \end{center}
    
  \end{itemize}
\end{itemize}

\subsection{Strong Normalization of LAM Logic}
\label{subsec:strong_normalization_of_lam_logic}
\input{Elle-to-LNL-ott}
% subsection strong_normalization_of_lam_logic (end)

% section logic (end)

\section{Combining with Benton's Adjoint Model}
\label{sec:combining}
% section combining (end)

\section{Applications}
\label{sec:applications}
% section applications (end)

\section{Conclusion}
\label{sec:conclusion}
TODO
% section conclusion (end)

\bibliographystyle{plainurl}
\bibliography{ref}

\appendix
\section{Appendix}
\label{sec:appendix}
% \section{Proof For Lemma~\ref{lem:cut-reduction}}
\label{app:cut-reduction}


\subsection{Commuting Conversion Cut vs. Cut}

\subsubsection{$\SCdruleTXXcutName$ vs. $\SCdruleTXXcutName$}
\begin{itemize}
% C-Cut vs. C-Cut Case 1
\item Case 1:
      \begin{center}
        \scriptsize
        \begin{math}
          \begin{array}{c}
            \Pi_1 \\
            {[[I |-c X]]}
          \end{array}
        \end{math}
        \qquad\qquad
        $\Pi_2:$
        \begin{math}
          $$\mprset{flushleft}
          \inferrule* [right={\tiny cut}] {
            {
              \begin{array}{cc}
                \pi_1 & \pi_2 \\
                {[[P2, X, P3 |-c Y]]} & {[[P1, Y, P4 |-c Z]]}
              \end{array}
            }
          }{[[P1, P2, X, P3, P4 |-c Z]]}
        \end{math}
      \end{center}
      By assumption, $c(\Pi_1),c(\Pi_2)\leq |X|$. Therefore, $c(\pi_1)$,
      $c(\pi_2)\leq |X|$. Since $Y$ is the cut formula on $\pi_1$ and
      $\pi_2$, we have $|Y|+1\leq|X|$. By induction on $\Pi_1$ and $\pi_1$
      there exists a proof $\Pi'$ for sequent $[[P2, I, P3 |-c Y]]$ s.t.
      $c(\Pi')\leq|X|$. So $\Pi$ can be constructed as follows, with
      $c(\Pi)\leq max\{c(\Pi'),c(\pi_2),|Y|+1\}\leq |X|$.
      \begin{center}
        \scriptsize
        \begin{math}
          $$\mprset{flushleft}
          \inferrule* [right={\tiny cut}] {
            {
              \begin{array}{cc}
                \Pi' & \pi_2 \\
                {[[P2, I, P3 |-c Y]]} & {[[P1, Y, P4 |-c Z]]}
              \end{array}
            }
          }{[[P1, P2, I, P3, P4 |-c Z]]}
        \end{math}
      \end{center}

% C-Cut vs. C-Cut Case 2
\item Case 2:
      \begin{center}
        \scriptsize
        $\Pi_1$:
        \begin{math}
          $$\mprset{flushleft}
          \inferrule* [right={\tiny cut}] {
            {
              \begin{array}{cc}
                \pi_1 & \pi_2 \\
                {[[I |-c X]]} & {[[P2, X, P3 |-c Y]]}
              \end{array}
            }
          }{[[P2, I, P3 |-c Y]]}
        \end{math}
        \qquad\qquad
        \begin{math}
          \begin{array}{c}
            \Pi_2 \\
            {[[P1, Y, P4 |-c Z]]}
          \end{array}
        \end{math}
      \end{center}
      By assumption, $c(\Pi_1),c(\Pi_2)\leq |Y|$. Since the cut rank of the last cut in
      $\Pi_1$ is $|X|+1$, then $|X|+1\leq |Y|$. By induction on $\Pi_1$ and $\Pi_2$, there is
      a proof $\Pi'$ for sequent $[[P1, P2, X, P3, P4 |-c Z]]$ s.t. $c(\Pi')\leq|Y|$.
      Therefore, the proof $\Pi$ can be constructed as follows, and
      $c(\Pi)\leq max\{c(\pi_1),c(\Pi'),|X|+1\}\leq |Y|$.
      \begin{center}
        \scriptsize
        \begin{math}
          $$\mprset{flushleft}
          \inferrule* [right={\tiny cut}] {
            {
              \begin{array}{cc}
                \pi_1 & \Pi' \\
                {[[I |-c X]]} & {[[P1, P2, X, P3, P4 |-c Z]]}
              \end{array}
            }
          }{[[P1, P2, I, P3, P4 |-c Z]]}
        \end{math}
      \end{center}
\end{itemize}



% C-Cut vs. LC-Cut Case 1
\subsubsection{$\SCdruleTXXcutName$ vs. $\SCdruleSXXcutOneName$}
\begin{itemize}
\item Case 1:
      \begin{center}
        \scriptsize
        \begin{math}
          \begin{array}{c}
            \Pi_1 \\
            {[[I |-c X]]}
          \end{array}
        \end{math}
        \qquad\qquad
        $\Pi_2:$
        \begin{math}
          $$\mprset{flushleft}
          \inferrule* [right={\tiny cut1}] {
            {
              \begin{array}{cc}
                \pi_2 & \pi_3 \\
                {[[P1, X, P2 |-c Y]]} & {[[G1; Y; G2 |-l A]]}
              \end{array}
            }
          }{[[G1; P1; X; P2; G2 |-l A]]}
        \end{math}
      \end{center}
      By assumption, $c(\Pi_1),c(\Pi_2)\leq |X|$. Therefore, $c(\pi_1)$,
      $c(\pi_2)\leq |X|$. Since $Y$ is the cut formula on $\pi_1$ and
      $\pi_2$, we have $|Y|+1\leq|X|$. By induction on $\Pi_1$ and $\pi_1$,
      there exists a proof $\Pi'$ for sequent $[[P1, I, P2 |-c Y]]$ s.t.
      $c(\Pi')\leq|X|$. So $\Pi$ can be constructed as follows, with
      $c(\Pi)\leq max\{c(\Pi'),c(\pi_2),|Y|+1\}\leq |X|$.
      \begin{center}
        \scriptsize
        \begin{math}
          $$\mprset{flushleft}
          \inferrule* [right={\tiny cut1}] {
            {
              \begin{array}{cc}
                \Pi' & \pi_2 \\
                {[[P1, I, P2 |-c Y]]} & {[[G1; Y; G2 |-l A]]}
              \end{array}
            }
          }{[[G1; P1; I; P2; G2 |-l A]]}
        \end{math}
      \end{center}

% C-Cut vs. LC-Cut Case 2
\item Case 2:
      \begin{center}
        \scriptsize
        $\Pi_1$:
        \begin{math}
          $$\mprset{flushleft}
          \inferrule* [right={\tiny cut}] {
            {
              \begin{array}{cc}
                \pi_1 & \pi_2 \\
                {[[I |-c X]]} & {[[P1, X, P2 |-c Y]]}
              \end{array}
            }
          }{[[P1, I, P2 |-c Y]]}
        \end{math}
        \qquad\qquad
        \begin{math}
          \begin{array}{c}
            \Pi_2 \\
            {[[G1; Y; G2 |-l A]]}
          \end{array}
        \end{math}
      \end{center}
      By assumption, $c(\Pi_1),c(\Pi_2)\leq |Y|$. Similar as above,
      $|X|+1\leq |Y|$ and there is a proof $\Pi'$ constructed from $\pi_2$
      and $\Pi_2$ for sequent $[[G1; P1; X; P2; G2 |-l A]]$ s.t.
      $c(\Pi')\leq|Y|$. Therefore, the proof $\Pi$ can be constructed as
      follows, and $c(\Pi)\leq max\{c(\pi_1),c(\Pi'),|X|+1\}\leq |Y|$.
      \begin{center}
        \scriptsize
        \begin{math}
          $$\mprset{flushleft}
          \inferrule* [right={\tiny cut}] {
            {
              \begin{array}{cc}
                \pi_1 & \Pi'\\
                {[[I |-c X]]} & {[[G1; P1; X; P2; G2 |-l A]]}
              \end{array}
            }
          }{[[G1; P1; I; P2; G2 |-l A]]}
        \end{math}
      \end{center}
\end{itemize}

% LC-Cut vs. L-Cut Case 1
\subsubsection{$\SCdruleSXXcutOneName$ vs. $\SCdruleSXXcutTwoName$}
\begin{itemize}
\item Case 1:
      \begin{center}
        \scriptsize
        \begin{math}
          \begin{array}{c}
            \Pi_1 \\
            {[[I |-c X]]}
          \end{array}
        \end{math}
        \qquad\qquad
        $\Pi_2:$
        \begin{math}
          $$\mprset{flushleft}
          \inferrule* [right={\tiny cut2}] {
            {
              \begin{array}{cc}
                \pi_1 & \pi_2 \\
                {[[G2; X; G3 |-l A]]} & {[[G1; A; G4 |-l B]]}
              \end{array}
            }
          }{[[G1; G2; X; G3; G4 |-l B]]}
        \end{math}
      \end{center}
      By assumption, $c(\Pi_1),c(\Pi_2)\leq |X|$. Therefore, $c(\pi_1)$,
      $c(\pi_2)\leq |X|$. Since $A$ is the cut formula on $\pi_1$ and
      $\pi_2$, we have $|A|+1\leq|X|$. By induction on $\Pi_1$ and $\pi_1$,
      there exists a proof $\Pi'$ for sequent $[[G2; I; G3 |-l A]]$ s.t.
      $c(\Pi')\leq|X|$. So $\Pi$ can be constructed as follows, with
      $c(\Pi)\leq max\{c(\Pi'),c(\pi_2),|A|+1\}\leq |X|$.
      \begin{center}
        \scriptsize
        \begin{math}
          $$\mprset{flushleft}
          \inferrule* [right={\tiny cut2}] {
            {
              \begin{array}{cc}
                \Pi' & \pi_2 \\
                {[[G2; I; G3 |-l A]]} & {[[G1; A; G4 |-l B]]}
              \end{array}
            }
          }{[[G1; G2; I; G3; G4 |-l B]]}
        \end{math}
      \end{center}

% LC-Cut vs. L-Cut Case 2
\item Case 2:
      \begin{center}
        \scriptsize
        $\Pi_1$:
        \begin{math}
          $$\mprset{flushleft}
          \inferrule* [right={\tiny cut}] {
            {
              \begin{array}{cc}
                \pi_1 & \pi_2 \\
                {[[I |-c X]]} & {[[G2; X; G3 |-l A]]}
              \end{array}
            }
          }{[[G2; I; G3 |-l A]]}
        \end{math}
        \qquad\qquad
        \begin{math}
          \begin{array}{c}
            \Pi_2 \\
            {[[G1; A; G4 |-l B]]}
          \end{array}
        \end{math}
      \end{center}
      By assumption, $c(\Pi_1),c(\Pi_2)\leq |A|$. Similar as above,
      $|X|+1\leq |A|$ and there is a proof $\Pi'$ constructed from'
      $\pi_2$ and $\Pi_2$ for sequent $[[G1; G2; X; G3; G4 |-l B]]$ s.t.
      $c(\Pi')\leq|A|$. Therefore, the proof $\Pi$ can be constructed as
      follows, and $c(\Pi)\leq max\{c(\pi_1),c(\Pi'),|X|+1\}\leq |A|$.
      \begin{center}
        \scriptsize
        \begin{math}
          $$\mprset{flushleft}
          \inferrule* [right={\tiny cut}] {
            {
              \begin{array}{cc}
                \pi_1  & \Pi' \\
                {[[I |-c X]]} & {[[G1; G2; X; G3; G4 |-l B]]}
              \end{array}
            }
          }{[[G1; G2; I; G3; G4 |-l B]]}
        \end{math}
      \end{center}
\end{itemize}

% L-Cut vs. L-Cut Case 1
\subsubsection{$\SCdruleSXXcutTwoName$ vs. $\SCdruleSXXcutTwoName$}
\begin{itemize}
\item Case 1:
      \begin{center}
        \scriptsize
        \begin{math}
          \begin{array}{c}
            \Pi_1 \\
            {[[G |-l A]]}
          \end{array}
        \end{math}
        \qquad\qquad
        $\Pi_2:$
        \begin{math}
          $$\mprset{flushleft}
          \inferrule* [right={\tiny cut2}] {
            {
              \begin{array}{cc}
                \pi_1 & \pi_2 \\
                {[[D2; A; D3 |-l B]]} & {[[D1; B; D4 |-l C]]}
              \end{array}
            }
          }{[[D1; D2; A; D3; D4 |-l C]]}
        \end{math}
      \end{center}
      By assumption, $c(\Pi_1),c(\Pi_2)\leq |A|$. Therefore, $c(\pi_1)$,
      $c(\pi_2)\leq |A|$. Since $B$ is the cut formula on $\pi_1$ and
      $\pi_3$, we have $|B|+1\leq|A|$. By induction on $\Pi_1$ and
      $\pi_1$, there exists a proof $\Pi'$ for sequent
      $[[D2; G; D3 |-l B]]$ s.t. $c(\Pi')\leq|A|$. So $\Pi$ can be
      constructed as follows,  with
      $c(\Pi)\leq max\{c(\Pi'),c(\pi_2),|B|+1\}\leq |A|$.
      \begin{center}
        \scriptsize
        \begin{math}
          $$\mprset{flushleft}
          \inferrule* [right={\tiny cut}] {
            {
              \begin{array}{cc}
                \Pi' & \pi_2 \\
                {[[D2; G; D3 |-l B]]} & {[[D1; B; D4 |-l C]]}
              \end{array}
            }
          }{[[D1; D2; G; D3; D4 |-l C]]}
        \end{math}
      \end{center}

% L-Cut vs. L-Cut Case 2
\item Case 2:
      \begin{center}
        \scriptsize
        $\Pi_1$:
        \begin{math}
          $$\mprset{flushleft}
          \inferrule* [right={\tiny cut}] {
            {
              \begin{array}{cc}
                \pi_1 & \pi_2 \\
                {[[D |-l A]]} & {[[D2; A; D3 |-l B]]}
              \end{array}
            }
          }{[[D2; D; D3 |-l A]]}
        \end{math}
        \qquad\qquad
        \begin{math}
          \begin{array}{c}
            \Pi_2 \\
            {[[D1; B; D4 |-l C]]}
          \end{array}
        \end{math}
      \end{center}
      By assumption, $c(\Pi_1),c(\Pi_2)\leq |B|$. Similar as above,
      $|A|+1\leq |B|$ and there is a proof $\Pi'$ constructed from $\pi_2$ 
      and $\Pi_2$ for sequent $[[D1; D2; A; D3; D4 |-l C]]$ s.t.
      $c(\Pi')\leq|A|$. Therefore, the proof $\Pi$ can be constructed as
      follows, and $c(\Pi)\leq max\{c(\pi_1),c(\Pi'),|A|+1\}\leq |B|$.
      \begin{center}
        \scriptsize
        \begin{math}
          $$\mprset{flushleft}
          \inferrule* [right={\tiny cut}] {
            {
              \begin{array}{cc}
                \pi_1 & \Pi' \\
                {[[G |-l A]]} & {[[D1; D2; A; D3; D4 |-l C]]}
              \end{array}
            }
          }{[[D1; D2; G; D3; D4 |-l C]]}
        \end{math}
      \end{center}

\end{itemize}
% End of subsubsection Commuting conversion cut vs. cut



\subsection{The Axiom Steps}

\subsubsection{$\SCdruleTXXaxName$}
\begin{itemize}
% C-id Case 1
\item Case 1:
      \begin{center}
        \scriptsize
        $\Pi_1$:
        \begin{math}
          $$\mprset{flushleft}
          \inferrule* [right={\tiny ax}] {
            \,
          }{[[X |-c X]]}
        \end{math}
        \qquad\qquad
        \begin{math}
          \begin{array}{c}
            \Pi_2 \\
            {[[I1, X, I2 |-c Y]]}
          \end{array}
        \end{math}
      \end{center}
      By assumption, $c(\Pi_1),c(\Pi_2)\leq |X|$. The proof $\Pi$ is the
      same as $\Pi_2$.

% C-id Case 2
\item Case 2:
      \begin{center}
        \scriptsize
        $\Pi_1$:
        \begin{math}
          \begin{array}{c}
            \Pi_1 \\
            {[[I |-c X]]}
          \end{array}
        \end{math}
        \qquad\qquad
        $\Pi_2$:
        \begin{math}
          $$\mprset{flushleft}
          \inferrule* [right={\tiny ax}] {
            \,
          }{[[X |-c X]]}
        \end{math}
      \end{center}
      By assumption, $c(\Pi_1),c(\Pi_2)\leq |X|$. The proof $\Pi$ is the
      same as $\Pi_1$.

% C-id Case 3
\item Case 3:
      \begin{center}
        \scriptsize
        $\Pi_1$:
        \begin{math}
          $$\mprset{flushleft}
          \inferrule* [right={\tiny ax}] {
            \,
          }{[[X |-c X]]}
        \end{math}
        \qquad\qquad
        \begin{math}
          \begin{array}{c}
            \Pi_2 \\
            {[[G1; X; G2 |-l A]]}
          \end{array}
        \end{math}
      \end{center}
      By assumption, $c(\Pi_1),c(\Pi_2)\leq |X|$. The proof $\Pi$ is the
      same as $\Pi_2$.
\end{itemize}

% L-id Case 1
\subsubsection{$\SCdruleTXXaxName$}
\begin{itemize}
\item Case 1:
      \begin{center}
        \scriptsize
        $\Pi_1$:
        \begin{math}
          $$\mprset{flushleft}
          \inferrule* [right={\tiny ax}] {
            \,
          }{[[A |-l A]]}
        \end{math}
        \qquad\qquad
        \begin{math}
          \begin{array}{c}
            \Pi_2 \\
            {[[G1; A; G2 |-l B]]}
          \end{array}
        \end{math}
      \end{center}
      By assumption, $c(\Pi_1),c(\Pi_2)\leq |A|$. The proof $\Pi$ is the
      same as $\Pi_2$.

% L-id Case 2
\item Case 2:
      \begin{center}
        \scriptsize
        $\Pi_1$:
        \begin{math}
          \begin{array}{c}
            \Pi_1 \\
            {[[D |-l A]]}
          \end{array}
        \end{math}
        \qquad\qquad
        $\Pi_2$:
        \begin{math}
          $$\mprset{flushleft}
          \inferrule* [right={\tiny ax}] {
            \,
          }{[[A |-l A]]}
        \end{math}
      \end{center}
      By assumption, $c(\Pi_1),c(\Pi_2)\leq |X|$. The proof $\Pi$ is the
      same as $\Pi_1$.
\end{itemize}
% End of subsubsection Axiom steps



\subsection{The Exchange Steps}

\subsubsection{$\SCdruleTXXexName$}

\begin{itemize}
% Conclusion vs. C-ex Case 1
\item Case 1:
      \begin{center}
        \scriptsize
        \begin{math}
          \begin{array}{c}
            \Pi_1 \\
            {[[P |-c X1]]}
          \end{array}
        \end{math}
        \qquad\qquad
        $\Pi_2$:
        \begin{math}
          $$\mprset{flushleft}
          \inferrule* [right={\tiny ex}] {
            {
              \begin{array}{c}
                \pi \\
                {[[I1, X1, X2, I2 |-c Y]]}
              \end{array}
            }
          }{[[I1, X2, X1, I2 |-c Y]]}
        \end{math}
      \end{center}
      By assumption, $c(\Pi_1),c(\Pi_2)\leq |X_1|$. By induction on $\pi$
      and $\Pi_1$, there is a proof $\Pi'$ for sequent
      $[[I1, P, X2, I2 |-c Y]]$ s.t. $c(\Pi')\leq|X_1|$. Therefore, the
      proof $\Pi$ can be constructed as follows, and
      $c(\Pi)=c(\Pi')\leq|X_1|$.
      \begin{center}
        \scriptsize
        \begin{math}
          $$\mprset{flushleft}
          \inferrule* [right={\tiny series of ex}] {
            {
              \begin{array}{c}
                \Pi' \\
                {[[I1, P, X2, I2 |-c Y]]}
              \end{array}
            }
          }{[[I1, X2, P, I2 |-c Y]]}
        \end{math}
      \end{center}

% Conclusion vs. C-ex Case 2
\item Case 2:
      \begin{center}
        \scriptsize
        \begin{math}
          \begin{array}{c}
            \Pi_1 \\
            {[[P |-c X2]]}
          \end{array}
        \end{math}
        \qquad\qquad
        $\Pi_2$:
        \begin{math}
          $$\mprset{flushleft}
          \inferrule* [right={\tiny ex}] {
            {
              \begin{array}{c}
                \pi \\
                {[[I1, X1, X2, I2 |-c Y]]}
              \end{array}
            }
          }{[[I1, X2, X1, I2 |-c Y]]}
        \end{math}
      \end{center}
      By assumption, $c(\Pi_1),c(\Pi_2)\leq |X_2|$. By induction on $\pi$
      and $\Pi_1$, there is a proof $\Pi'$ for sequent
      $[[I1, X1, P, I2 |-c Y]]$ s.t. $c(\Pi')\leq|X_2|$. Therefore, the
      proof $\Pi$ can be constructed as follows, and
      $c(\Pi)=c(\Pi')\leq|X_2|$.
      \begin{center}
        \scriptsize
        \begin{math}
          $$\mprset{flushleft}
          \inferrule* [right={\tiny series of ex}] {
            {
              \begin{array}{c}
                \Pi' \\
                {[[I1, X1, P, I2 |-c Y]]}
              \end{array}
            }
          }{[[I1, P, X1, I2 |-c Y]]}
        \end{math}
      \end{center}
\end{itemize}

% Conclusion vs. LC-ex Case 1
\subsubsection{$\SCdruleSXXexName$}
\begin{itemize}
\item Case 1:
      \begin{center}
        \scriptsize
        \begin{math}
          \begin{array}{c}
            \Pi_1 \\
            {[[P |-c X1]]}
          \end{array}
        \end{math}
        \qquad\qquad
        $\Pi_2$:
        \begin{math}
          $$\mprset{flushleft}
          \inferrule* [right={\tiny ex}] {
            {
              \begin{array}{c}
                \pi \\
                {[[D1; X1; X2; D2 |-l A]]}
              \end{array}
            }
          }{[[D1; X2; X1; D2 |-l A]]}
        \end{math}
      \end{center}
      By assumption, $c(\Pi_1),c(\Pi_2)\leq |X_1|$. By induction on $\pi$
      and $\Pi_1$, there is a proof $\Pi'$ for sequent
      $[[D1; P; X2; D2 |-l A]]$ s.t. $c(\Pi')\leq|X_1|$. Therefore, the
      proof $\Pi$ can be constructed as follows, and
      $c(\Pi)=c(\Pi')\leq|X_1|$.
      \begin{center}
        \scriptsize
        \begin{math}
          $$\mprset{flushleft}
          \inferrule* [right={\tiny series of ex}] {
            {
              \begin{array}{c}
                \Pi' \\
                {[[D1; P; X2; D2 |-l A]]}
              \end{array}
            }
          }{[[D1; X2; P; D2 |-l A]]}
        \end{math}
      \end{center}

% Conclusion vs. LC-ex Case 2
\item Case 2:
      \begin{center}
        \scriptsize
        \begin{math}
          \begin{array}{c}
            \Pi_1 \\
            {[[P |-c X2]]}
          \end{array}
        \end{math}
        \qquad\qquad
        $\Pi_2$:
        \begin{math}
          $$\mprset{flushleft}
          \inferrule* [right={\tiny ex}] {
            {
              \begin{array}{c}
                \pi \\
                {[[D1; X1; X2; D2 |-l A]]}
              \end{array}
            }
          }{[[D1; X2; X1; D2 |-l A]]}
        \end{math}
      \end{center}
      By assumption, $c(\Pi_1),c(\Pi_2)\leq |X_2|$. By induction on $\pi$
      and $\Pi_1$, there is a proof $\Pi'$ for sequent
      $[[D1; X1; P; D2 |-l A]]$ s.t. $c(\Pi')\leq|X_2|$. Therefore, the
      proof $\Pi$ can be constructed as follows, and
      $c(\Pi)=c(\Pi')\leq|X_2|$.
      \begin{center}
        \scriptsize
        \begin{math}
          $$\mprset{flushleft}
          \inferrule* [right={\tiny series of ex}] {
            {
              \begin{array}{c}
                \Pi' \\
                {[[D1; X1; P; I2 |-l A]]}
              \end{array}
            }
          }{[[I1; P; X1; I2 |-l A]]}
        \end{math}
      \end{center}
\end{itemize}



\subsection{Principal Formula vs. Principal Formula} 

\subsubsection{The Commutative Tensor Product $\otimes$}
\begin{center}
  \scriptsize
  $\Pi_1:$
  \begin{math}
    $$\mprset{flushleft}
    \inferrule* [right={\tiny tenR}] {
      {
        \begin{array}{cc}
          \pi_1 & \pi_2 \\
          {[[I1 |-c X]]} & {[[I2 |-c Y]]}
        \end{array}
      }
    }{[[I1, I2 |-c X (*) Y]]}
  \end{math}
  \qquad\qquad
  $\Pi_2:$
  \begin{math}
    $$\mprset{flushleft}
    \inferrule* [right={\tiny tenL}] {
      {
        \begin{array}{c}
          \pi_3 \\
          {[[P1, X, Y, P2 |-c Z]]}
        \end{array}
      }
    }{[[P1, X (*) Y, P2 |-c Z]]}
  \end{math}
\end{center}
By assumption, $c(\Pi_1),c(\Pi_2)\leq |[[X (*) Y]]| = |X|+|Y|+1$. The proof
$\Pi$ can be constructed as follows, and
$c(\Pi)\leq max\{c(\pi_1),c(\pi_2),c(\pi_3),|X|+1,|Y|+1\}\leq |X|+|Y|+1 = |[[X (*) Y]]|$.
\begin{center}
  \scriptsize
  \begin{math}
    $$\mprset{flushleft}
    \inferrule* [right={\tiny cut}] {
      {
        \begin{array}{c}
          \pi_1 \\
          {[[I1 |-c X]]}
        \end{array}
      }
      $$\mprset{flushleft}
      \inferrule* [right={\tiny cut}] {
      {
        \begin{array}{cc}
          \pi_2 & \pi_3 \\
          {[[I2 |-c Y]]} & {[[P1, X, Y, P2 |-c Z]]}
        \end{array}
      }
      }{[[P1, X, I2, P2 |-c Z]]}
    }{[[P1, I1, I2, P2 |-c Z]]}
  \end{math}
\end{center}

\subsubsection{The Non-commutative Tensor Product $\tri$}
\begin{center}
  \scriptsize
  $\Pi_1:$
  \begin{math}
    $$\mprset{flushleft}
    \inferrule* [right={\tiny tenR}] {
      {
        \begin{array}{cc}
          \pi_1 & \pi_2 \\
          {[[G1 |-l A]]} & {[[G2 |-l B]]}
        \end{array}
      }
    }{[[G1; G2 |-l A (>) B]]}
  \end{math}
  \qquad\qquad
  $\Pi_2:$
  \begin{math}
    $$\mprset{flushleft}
    \inferrule* [right={\tiny tenL1}] {
      {
        \begin{array}{c}
          \pi_3 \\
          {[[D1; A; B; D2 |-l C]]}
        \end{array}
      }
    }{[[D1; A (>) B; D2 |-l C]]}
  \end{math}
\end{center}
By assumption, $c(\Pi_1),c(\Pi_2)\leq |[[A (>) B]]| = |X|+|Y|+1$. The proof
$\Pi$ can be constructed as follows, and
$c(\Pi)\leq max\{c(\pi_1),c(\pi_2),c(\pi_3),|A|+1,|B|+1\}\leq |A|+|B|+1 = |[[A (>) B]]|$.
\begin{center}
  \scriptsize
  \begin{math}
    $$\mprset{flushleft}
    \inferrule* [right={\tiny cut2}] {
      {
        \begin{array}{c}
          \pi_1 \\
          {[[G1 |-l A]]}
        \end{array}
      }
      $$\mprset{flushleft}
      \inferrule* [right={\tiny cut2}] {
      {
        \begin{array}{cc}
          \pi_2 & \pi_3 \\
          {[[G2 |-l B]]} & {[[D1; A; B; D2 |-l C]]}
        \end{array}
      }
      }{[[D1; A; G2; D2 |-l C]]}
    }{[[D1; G1; G2; P2 |-l C]]}
  \end{math}
\end{center}

\subsubsection{The Commutative Implication $\multimap$}
\begin{center}
  \scriptsize
  $\Pi_1:$
  \begin{math}
    $$\mprset{flushleft}
    \inferrule* [right={\tiny tenR}] {
      {
        \begin{array}{c}
          \pi_1 \\
          {[[I1, X |-c Y]]}
        \end{array}
      }
    }{[[I1 |-c X -o Y]]}
  \end{math}
  \qquad\qquad
  $\Pi_2:$
  \begin{math}
    $$\mprset{flushleft}
    \inferrule* [right={\tiny tenL}] {
      {
        \begin{array}{cc}
          \pi_2 & \pi_3 \\
          {[[I2 |-c X]]} & {[[P1, Y, P2 |-c Z]]}
        \end{array}
      }
    }{[[P1, X -o Y, I, P2 |-c Z]]}
  \end{math}
\end{center}
By assumption, $c(\Pi_1),c(\Pi_2)\leq |[[X -o Y]]| = |X|+|Y|+1$. The proof 
$\Pi$ is constructed as follows
$c(\Pi)\leq max\{c(\pi_1),c(\pi_2),c(\pi_3),|X|+1,|Y|+1\}\leq |X|+|Y|+1 = |[[X -o Y]]|$.
\begin{center}
  \scriptsize
  \begin{math}
    $$\mprset{flushleft}
    \inferrule* [right={\tiny tenR}] {
      $$\mprset{flushleft}
      \inferrule* [right={\tiny tenR}] {
        {
          \begin{array}{cc}
            \pi_1 & \pi_2 \\
            {[[I1, X |-c Y]]} & {[[I2 |-c X]]}
          \end{array}
        }
      }{[[I1, I2 |-c Y]]} \\
       {
         \begin{array}{c}
           \pi_3 \\
           {[[P1, Y, P2 |-c Z]]}
         \end{array}
       }
    }{[[P1, I1, I2, P2 |-c Z]]}
  \end{math}
\end{center}

\subsubsection{The Non-commutative Right Implication $\lto$}
\begin{center}
  \scriptsize
  $\Pi_1:$
  \begin{math}
    $$\mprset{flushleft}
    \inferrule* [right={\tiny imprR}] {
      {
        \begin{array}{c}
          \pi_1 \\
          {[[G; A |-l B]]}
        \end{array}
      }
    }{[[G |-l A -> B]]}
  \end{math}
  \qquad\qquad
  $\Pi_2:$
  \begin{math}
    $$\mprset{flushleft}
    \inferrule* [right={\tiny imprL}] {
      {
        \begin{array}{cc}
          \pi_2 & \pi_3 \\
          {[[D1 |-l A]]} & {[[D2; B |-l C]]}
        \end{array}
      }
    }{[[D2; A -> B; D1 |-l C]]}
  \end{math}
\end{center}
By assumption, $c(\Pi_1),c(\Pi_2)\leq |[[A -> B]]| = |A|+|B|+1$. The proof
$\Pi$ is constructed as follows, and
$c(\Pi)\leq max\{c(\pi_1),c(\pi_2),c(\pi_3),|A|+1,|B|+1\}\leq |A|+|B|+1 = |[[A -> B]]|$.
\begin{center}
  \scriptsize
  \begin{math}
    $$\mprset{flushleft}
    \inferrule* [right={\tiny cut2}] {
      $$\mprset{flushleft}
      \inferrule* [right={\tiny cut2}] {
        {
          \begin{array}{cc}
            \pi_1 & \pi_2 \\
            {[[G; A |-l B]]} & {[[D1 |-l A]]}
          \end{array}
        }
      }{[[G; D1 |-l B]]}
       {
         \begin{array}{c}
           \pi_3 \\
           {[[D2; B |-l C]]}
         \end{array}
       }
    }{[[D2; G; D1 |-l C]]}
  \end{math}
\end{center}

\subsubsection{The Non-commutative Left Implication $\rto$}
\begin{center}
  \scriptsize
  $\Pi_1:$
  \begin{math}
    $$\mprset{flushleft}
    \inferrule* [right={\tiny implR}] {
      {
        \begin{array}{c}
          \pi_1 \\
          {[[A; G |-l B]]}
        \end{array}
      }
    }{[[G |-l B <- A]]}
  \end{math}
  \qquad\qquad
  $\Pi_2:$
  \begin{math}
    $$\mprset{flushleft}
    \inferrule* [right={\tiny implL}] {
      {
        \begin{array}{cc}
          \pi_2 & \pi_3 \\
          {[[D1 |-l A]]} & {[[B; D2 |-l C]]}
        \end{array}
      }
    }{[[D1; B <- A; D2 |-l C]]}
  \end{math}
\end{center}
By assumption, $c(\Pi_1),c(\Pi_2)\leq |[[B <- A]]| = |A|+|B|+1$. The
proof $\Pi$ is constructed as follows, and
$c(\Pi)\leq max\{c(\pi_1),c(\pi_2),c(\pi_3),|A|+1,|B|+1\}\leq |A|+|B|+1 = |[[B <- A]]|$.
\begin{center}
  \scriptsize
  \begin{math}
    $$\mprset{flushleft}
    \inferrule* [right={\tiny cut1}] {
      $$\mprset{flushleft}
      \inferrule* [right={\tiny cut2}] {
        {
          \begin{array}{cc}
            \pi_1 & \pi_2 \\
            {[[A; G |-l B]]} & {[[D1 |-l A]]}
          \end{array}
        }
      }{[[D1; G |-l B]]}
       {
         \begin{array}{c}
           \pi_3 \\
           {[[B; D2 |-l C]]}
         \end{array}
       }
    }{[[D1; G; D2 |-l C]]}
  \end{math}
\end{center}



\subsubsection{The Commutative Unit $[[UnitT]]$}
\begin{itemize}
\item Case 1:
      \begin{center}
        \scriptsize
        $\Pi_1:$
        \begin{math}
          $$\mprset{flushleft}
          \inferrule* [right={\tiny unitR}] {
            \,
          }{[[. |-c UnitT]]}
        \end{math}
        \qquad\qquad
        $\Pi_2:$
        \begin{math}
          $$\mprset{flushleft}
          \inferrule* [right={\tiny unitL}] {
            {
              \begin{array}{c}
                \pi \\
                {[[I, P |-c X]]}
              \end{array}
            }
          }{[[I, UnitT, P |-c X]]}
        \end{math}
      \end{center}
      By assumption, $c(\Pi_1),c(\Pi_2)\leq |[[UnitT]]|$. The proof $\Pi$
      is the subproof $\pi$ in $\Pi_2$ for sequent $[[I |-c X]]$. So
      $c(\Pi)=c(\Pi_2)\leq |[[UnitT]]|$.

\item Case 2:
      \begin{center}
        \scriptsize
        $\Pi_1:$
        \begin{math}
          $$\mprset{flushleft}
          \inferrule* [right={\tiny unitR}] {
            \,
          }{[[. |-c UnitT]]}
        \end{math}
        \qquad\qquad
        $\Pi_2:$
        \begin{math}
          $$\mprset{flushleft}
          \inferrule* [right={\tiny unitL1}] {
            {
              \begin{array}{c}
                \pi \\
                {[[G; D |-l A]]}
              \end{array}
            }
          }{[[G; UnitT; D |-l A]]}
        \end{math}
      \end{center}
      Similar as above, $\Pi$ is $\pi$.
\end{itemize}


\subsubsection{The Non-commutative Unit $[[UnitS]]$}
\begin{center}
  \scriptsize
  $\Pi_1:$
  \begin{math}
    $$\mprset{flushleft}
    \inferrule* [right={\tiny unitR}] {
      \,
    }{[[. |-l UnitS]]}
  \end{math}
  \qquad\qquad
  $\Pi_2:$
  \begin{math}
    $$\mprset{flushleft}
    \inferrule* [right={\tiny unitL2}] {
      {
        \begin{array}{c}
          \pi \\
          {[[G; D |-l A]]}
        \end{array}
      }
    }{[[G; UnitS; D |-l A]]}
  \end{math}
\end{center}
By assumption, $c(\Pi_1),c(\Pi_2)\leq |[[UnitS]]|$. The proof $\Pi$ is the
subproof $\pi$ in $\Pi_2$ for sequent $[[D |-l A]]$. So
$c(\Pi)=c(\Pi_2)\leq |[[UnitS]]|$.

\subsubsection{The Functor $F$}
\begin{center}
  \scriptsize
  $\Pi_1:$
  \begin{math}
    $$\mprset{flushleft}
    \inferrule* [right={\tiny FR}] {
      {
        \begin{array}{c}
          \pi_1 \\
          {[[I |-c X]]}
        \end{array}
      }
    }{[[I |-l F X]]}
  \end{math}
  \qquad\qquad
  $\Pi_2:$
  \begin{math}
    $$\mprset{flushleft}
    \inferrule* [right={\tiny FL}] {
      {
        \begin{array}{c}
          \pi_2 \\
          {[[G; X; D |-l A]]}
        \end{array}
      }
    }{[[G; F X; D |-l A]]}
  \end{math}
\end{center}
By assumption, $c(\Pi_1),c(\Pi_2)\leq |[[F X]]| = |X|+1$. The proof
$\Pi$ is constructed as follows, and \\
$c(\Pi)\leq max\{c(\pi_1),c(\pi_2),|X|+1\}\leq |[[F X]]|$.
\begin{center}
  \scriptsize
  \begin{math}
    $$\mprset{flushleft}
    \inferrule* [right={\tiny cut2}] {
      {
        \begin{array}{cc}
          \pi_1 & \pi_2 \\
          {[[I |-c X]]} & {[[G; X; D |-l A]]}
        \end{array}
      }
    }{[[G; I; D |-l A]]}
  \end{math}
\end{center}

\subsubsection{The Functor $G$}
\begin{center}
  \scriptsize
  $\Pi_1:$
  \begin{math}
    $$\mprset{flushleft}
    \inferrule* [right={\tiny GR}] {
      {
        \begin{array}{c}
          \pi_1 \\
          {[[I |-l A]]}
        \end{array}
      }
    }{[[I |-c Gf A]]}
  \end{math}
  \qquad\qquad
  $\Pi_2:$
  \begin{math}
    $$\mprset{flushleft}
    \inferrule* [right={\tiny GL}] {
      {
        \begin{array}{c}
          \pi_2 \\
          {[[G; A; D |-l B]]}
        \end{array}
      }
    }{[[G; Gf A; D |-l B]]}
  \end{math}
\end{center}
By assumption, $c(\Pi_1),c(\Pi_2)\leq |[[Gf A]]| = |A|+1$. The proof $\Pi$ 
is constructed as follows, and \\
$c(\Pi)\leq max\{c(\pi_1),c(\pi_2),|A|+1\}\leq |[[Gf A]]|$.
\begin{center}
  \scriptsize
  \begin{math}
    $$\mprset{flushleft}
    \inferrule* [right={\tiny GL}] {
      {
        \begin{array}{cc}
          \pi_1 & \pi_2 \\
          {[[I |-l A]]} & {[[G; A; D |-l B]]}
        \end{array}
      }
    }{[[G; I; D |-l B]]}
  \end{math}
\end{center}



\subsection{Secondary Conclusion}

\subsubsection{Left introduction of the commutative implication $\multimap$}
\begin{itemize}
\item Case 1:
      \begin{center}
        \scriptsize
        $\Pi_1$:
        \begin{math}
          $$\mprset{flushleft}
          \inferrule* [right={\tiny impL}] {
            {
              \begin{array}{cc}
                \pi_1 & \pi_2 \\
                {[[I1 |-c X1]]} & {[[I2, X2, I3 |-c Y]]}
              \end{array}
            }
          }{[[I2, X1 -o X2, I1, I3 |-c Y]]}
        \end{math}
        \qquad\qquad
        \begin{math}
          \begin{array}{c}
            \Pi_2 \\
            {[[P1, Y, P2 |-c Z]]}
          \end{array}
        \end{math}
      \end{center}
      By assumption, $c(\Pi_1),c(\Pi_2)\leq |Y|$. By induction, there is a
      proof $\Pi'$ from $\pi_2$ and $\Pi_2$ for sequent
      $[[P1, I2, X2, I3, P2 |-c Z]]$ s.t. $c(\Pi')\leq |Y|$. Therefore,
      the proof $\Pi$ can be constructed as follows with $c(\Pi)\leq |Y|$.
      \begin{center}
        \scriptsize
        \begin{math}
          $$\mprset{flushleft}
          \inferrule* [right={\tiny impL}] {
            {
              \begin{array}{c}
                \pi_1 \\
                {[[I1 |-c X1]]}
              \end{array}
            }
            $$\mprset{flushleft}
            \inferrule* [right={\tiny cut}] {
              {
                \begin{array}{cc}
                  \pi_2 & \Pi_2 \\
                  {[[I2, X2, I3 |-c Y]]} & {[[P1, Y, P2 |-c Z]]}
                \end{array}
              }
            }{[[P1, I2, X2, I3, P2 |-c Z]]}
          }{[[P1, I2, X1 -o X2, I1, I3, P2 |-c Z]]}
        \end{math}
      \end{center}

\item Case 2:
      \begin{center}
        \scriptsize
        $\Pi_1$:
        \begin{math}
          $$\mprset{flushleft}
          \inferrule* [right={\tiny impL}] {
            {
              \begin{array}{cc}
                \pi_1 & \pi_2 \\
                {[[I1 |-c X1]]} & {[[I2, X2, I3 |-c Y]]}
              \end{array}
            }
          }{[[I2, X1 -o X2, I1, I3 |-c Y]]}
        \end{math}
        \qquad\qquad
        \begin{math}
          \begin{array}{c}
            \Pi_2 \\
            {[[G1; Y; G2 |-l A]]}
          \end{array}
        \end{math}
      \end{center}
      By assumption, $c(\Pi_1),c(\Pi_2)\leq |Y|$. By induction, there is a
      proof $\Pi'$ from $\pi_2$ and $\Pi_2$ for sequent
      $[[G1; I2; X2; I3; G2 |-l A]]$ s.t. $c(\Pi')\leq |Y|$. Therefore, the
      proof $\Pi$ can be constructed as follows with $c(\Pi)\leq |Y|$.
      \begin{center}
        \scriptsize
        \begin{math}
          $$\mprset{flushleft}
          \inferrule* [right={\tiny impL}] {
            {
              \begin{array}{c}
                \pi_1 \\
                {[[I1 |-c X1]]}
              \end{array}
            }
            $$\mprset{flushleft}
            \inferrule* [right={\tiny cut}] {
              {
                \begin{array}{cc}
                  \pi_2 & \Pi_2 \\
                  {[[I2, X2, I3 |-c Y]]} & {[[G1; Y; G2 |-l A]]}
                \end{array}
              }
            }{[[G1; I2; X2; I3; G2 |-l A]]}
          }{[[G1; I2; X1 -o X2; I1; I3; G2 |-l A]]}
        \end{math}
      \end{center}
\end{itemize}



\subsubsection{Left introduction of the non-commutative left implication $\lto$}
\begin{center}
\scriptsize
  $\Pi_1$:
  \begin{math}
    $$\mprset{flushleft}
    \inferrule* [right={\tiny impL}] {
      {
        \begin{array}{cc}
          \pi_1 & \pi_2 \\
          {[[G1 |-l A1]]} & {[[G2; A2; G3 |-l B]]}
        \end{array}
      }
    }{[[G2; A1 -> A2; G1; G3 |-l B]]}
  \end{math}
  \qquad\qquad
  \begin{math}
    \begin{array}{c}
      \Pi_2 \\
      {[[D1; B; D2 |-l C]]}
    \end{array}
  \end{math}
\end{center}
By assumption, $c(\Pi_1),c(\Pi_2)\leq |B|$. By induction, there is a
proof $\Pi'$ from $\pi_2$ and $\Pi_2$ for sequent
$[[D1; G2; A2; G3; D2 |-l C]]$ s.t. $c(\Pi')\leq |B|$.
Therefore, the proof $\Pi$ can be constructed as follows with
$c(\Pi)\leq |B|$.
\begin{center}
  \scriptsize
  \begin{math}
    $$\mprset{flushleft}
    \inferrule* [right={\tiny impL}] {
      {
        \begin{array}{c}
          \pi_1 \\
          {[[G1 |-l A1 ]]}
        \end{array}
      }
      $$\mprset{flushleft}
      \inferrule* [right={\tiny cut}] {
        {
          \begin{array}{cc}
            \pi_2 & \Pi_2 \\
            {[[G2; A2; G3 |-l B]]} & {[[D1; B; D2 |-l C]]}
          \end{array}
        }
      }{[[D1; G2; A2; G3; D2 |-l C]]}
    }{[[D1; G2; A1 -> A2; G1; G3; D2 |-l C]]}
  \end{math}
\end{center}


\subsubsection{Left introduction of the non-commutative right implication $\rto$}
\begin{center}
  \scriptsize
  $\Pi_1$:
  \begin{math}
    $$\mprset{flushleft}
    \inferrule* [right={\tiny impL}] {
      {
        \begin{array}{cc}
          \pi_1 & \pi_2 \\
          {[[G1 |-l A1]]} & {[[G2; A2; G3 |-l B]]}
        \end{array}
      }
    }{[[G2; G1; A2 <- A1; G3 |-l B]]}
  \end{math}
  \qquad\qquad
  \begin{math}
    \begin{array}{c}
      \Pi_2 \\
      {[[D1; B; D2 |-l C]]}
    \end{array}
  \end{math}
\end{center}
By assumption, $c(\Pi_1),c(\Pi_2)\leq |B|$. By induction, there is a
proof $\Pi'$ from $\pi_2$ and $\Pi_2$ for sequent
$[[D1; G2; A2; G3; D2 |-l C]]$ s.t. $c(\Pi')\leq |B|$. Therefore, the
proof $\Pi$ can be constructed as follows with $c(\Pi)\leq |B|$.
\begin{center}
  \scriptsize
  \begin{math}
    $$\mprset{flushleft}
    \inferrule* [right={\tiny impL}] {
      {
        \begin{array}{c}
          \pi_1 \\
          {[[G1 |-l A1 ]]}
        \end{array}
      }
      $$\mprset{flushleft}
      \inferrule* [right={\tiny cut}] {
        {
          \begin{array}{cc}
            \pi_2 & \Pi_2 \\
            {[[G2; A2; G3 |-l B]]} & {[[D1; B; D2 |-l C]]}
          \end{array}
        }
      }{[[D1; G2; A2; G3; D2 |-l C]]}
    }{[[D1; G2; G1; A2 <- A1; G3; D2 |-l C]]}
  \end{math}
\end{center}

% C-ex Case 1
\subsubsection{$\SCdruleTXXexName$}
\begin{itemize}
\item Case 1:
      \begin{center}
        \scriptsize
        $\Pi_1$:
        \begin{math}
          $$\mprset{flushleft}
          \inferrule* [right={\tiny ex}] {
            {
              \begin{array}{c}
                \pi \\
                {[[I1, X1, X2, I2 |-c Y]]}
              \end{array}
            }
          }{[[I1, X2, X1, I2 |-c Y]]}
        \end{math}
        \qquad\qquad
        \begin{math}
          \begin{array}{c}
            \Pi_2 \\
            {[[P1, Y, P2 |-c Z]]}
          \end{array}
        \end{math}
      \end{center}
      By assumption, $c(\Pi_1),c(\Pi_2)\leq |Y|$. By induction on $\pi$
      and $\Pi_2$, there is a proof $\Pi'$ for sequent
      $[[P1, I1, X1, X2, I2, P2 |-c Z]]$ s.t. $c(\Pi')\leq|Y|$. Therefore,
      the proof $\Pi$ can be constructed as follows, and
      $c(\Pi)=c(\Pi')\leq|Y|$.
      \begin{center}
        \scriptsize
        \begin{math}
          $$\mprset{flushleft}
          \inferrule* [right={\tiny ex}] {
            {
              \begin{array}{c}
                \Pi' \\
                {[[P1, I1, X1, X2, I2, P2 |-c Z]]}
              \end{array}
            }
          }{[[P1, I1, X2, X1, I2, P2 |-c Z]]}
        \end{math}
      \end{center}

% C-ex Case 2
\item Case 2:
      \begin{center}
        \scriptsize
        $\Pi_1$:
        \begin{math}
          $$\mprset{flushleft}
          \inferrule* [right={\tiny beta}] {
            {
              \begin{array}{c}
                \pi \\
                {[[I1, X, Y, I2 |-c Z]]}
              \end{array}
            }
          }{[[I1, Y, X, I2 |-c Z]]}
        \end{math}
        \qquad\qquad
        \begin{math}
          \begin{array}{c}
            \Pi_2 \\
            {[[G1; Z; G2 |-l A]]}
          \end{array}
        \end{math}
      \end{center}
      By assumption, $c(\Pi_1),c(\Pi_2)\leq |Z|$. Similar as above, there
      is a proof $\Pi'$ constructed from $\pi$ and $\Pi_2$ for 
      $[[G1; I1; X; Y; I2; G2 |-l A]]$ s.t. $c(\Pi')\leq|Z|$. Therefore,
      the proof $\Pi$ can be constructed as follows, and
      $c(\Pi)=c(\Pi')\leq|Z|$.
      \begin{center}
        \scriptsize
        \begin{math}
          $$\mprset{flushleft}
          \inferrule* [right={\tiny beta}] {
            {
              \begin{array}{c}
                \Pi' \\
                {[[G1; I1; X; Y; I2; G2 |-l A]]}
              \end{array}
            }
          }{[[G1; I1; Y; X; I2; G2 |-l A]]}
        \end{math}
      \end{center}
\end{itemize}

% LC-ex
\subsubsection{$\SCdruleSXXexName$}
\begin{center}
  \scriptsize
  $\Pi_1$:
  \begin{math}
    $$\mprset{flushleft}
    \inferrule* [right={\tiny beta}] {
      {
        \begin{array}{c}
          \pi \\
          {[[G1; X; Y; G2 |-l A]]}
        \end{array}
      }
    }{[[G1; Y; X; G2 |-l A]]}
  \end{math}
  \qquad\qquad
  \begin{math}
    \begin{array}{c}
      \Pi_2 \\
      {[[D1; A; D2 |-l B]]}
    \end{array}
  \end{math}
\end{center}
By assumption, $c(\Pi_1),c(\Pi_2)\leq |A|$. Similar as above, there
is a proof $\Pi'$ constructed from $\pi$ and $\Pi_2$ for sequent
$[[D1; G1; X; Y; G2; D2 |-l B]]$ s.t. $c(\Pi')\leq|A|$. Therefore,
the proof $\Pi$ can be constructed as follows, and
$c(\Pi)=c(\Pi')\leq|A|$.
\begin{center}
  \scriptsize
  \begin{math}
    $$\mprset{flushleft}
    \inferrule* [right={\tiny beta}] {
      {
        \begin{array}{cc}
          \Pi' \\
          {[[D1; G1; X; Y; G2; D2 |-l B]]}
        \end{array}
      }
    }{[[D1; G1; Y; X; G2; D2 |-l B]]}
  \end{math}
\end{center}





\subsubsection{Left introduction of the commutative tensor product $\otimes$}
\begin{itemize}
\item Case 1:
      \begin{center}
        \scriptsize
        $\Pi_1$:
        \begin{math}
          $$\mprset{flushleft}
          \inferrule* [right={\tiny tenL}] {
            {
              \begin{array}{c}
                \pi \\
                {[[I1, X1, X2, I2 |-c Y]]}
              \end{array}
            }
          }{[[I1, X1 (*) X2, I2 |-c Y]]}
        \end{math}
        \qquad\qquad
        \begin{math}
          \begin{array}{c}
            \Pi_2 \\
            {[[P1, Y, P2 |-c Z]]}
          \end{array}
        \end{math}
      \end{center}
      By assumption, $c(\Pi_1),c(\Pi_2)\leq |Y|$. By induction, there is a
      proof $\Pi'$ from $\pi$ and $\Pi_2$ for sequent
      $[[P1, I1, X1, X2, I2, P2 |-c Z]]$ s.t. $c(\Pi')\leq |Y|$. Therefore,
      the proof $\Pi$ can be constructed as follows with $c(\Pi)\leq |Y|$.
      \begin{center}
        \scriptsize
        \begin{math}
          $$\mprset{flushleft}
          \inferrule* [right={\tiny tenL}] {
            $$\mprset{flushleft}
            \inferrule* [right={\tiny cut}] {
              {
                \begin{array}{cc}
                  \pi & \Pi_2 \\
                  {[[I1, X1, X2, I2 |-c Y]]} & {[[P1, Y, P2 |-c Z]]}
                \end{array}
              }
            }{[[P1, I1, X1, X2, I2, P2 |-c Z]]}
          }{[[P1, I1, X1 (*) X2, I2, P2 |-c Z]]}
        \end{math}
      \end{center}

\item Case 2:
      \begin{center}
        \scriptsize
        $\Pi_1$:
        \begin{math}
          $$\mprset{flushleft}
          \inferrule* [right={\tiny tenL}] {
            {
              \begin{array}{c}
                \pi \\
                {[[I1, X1, X2, I2 |-c Y]]}
              \end{array}
            }
          }{[[I1, X1 (*) X2, I2 |-c Y]]}
        \end{math}
        \qquad\qquad
        \begin{math}
          \begin{array}{c}
            \Pi_2 \\
            {[[G1; Y; G2 |-l A]]}
          \end{array}
        \end{math}
      \end{center}
      By assumption, $c(\Pi_1),c(\Pi_2)\leq |Y|$. By induction, there is a
      proof $\Pi'$ from $\pi$ and $\Pi_2$ for sequent
      $[[G1; I1; X1; X2; I2; G2 |-l A]]$ s.t. $c(\Pi')\leq |Y|$. Therefore,
      the proof $\Pi$ can be constructed as follows with $c(\Pi)\leq |Y|$.
      \begin{center}
        \scriptsize
        \begin{math}
          $$\mprset{flushleft}
          \inferrule* [right={\tiny tenL1}] {
            $$\mprset{flushleft}
            \inferrule* [right={\tiny cut1}] {
              {
                \begin{array}{cc}
                  \pi & \Pi_2 \\
                  {[[I1, X1, X2, I2 |-c Y]]} & {[[G1; Y; G2 |-l A]]}
                \end{array}
              }
            }{[[G1; I1; X1; X2; I2; G2 |-l A]]}
          }{[[G1; I1; X1 (*) X2; I2; G2 |-l A]]}
        \end{math}
      \end{center}

\item Case 3:
      \begin{center}
        \scriptsize
        $\Pi_1$:
        \begin{math}
          $$\mprset{flushleft}
          \inferrule* [right={\tiny tenL}] {
            {
              \begin{array}{c}
                \pi \\
                {[[G1; X; Y; G2 |-l A]]}
              \end{array}
            }
          }{[[G1; X (*) Y; G2 |-l A]]}
        \end{math}
        \qquad\qquad
        \begin{math}
          \begin{array}{c}
            \Pi_2 \\
            {[[D1; A; D2 |-l B]]}
          \end{array}
        \end{math}
      \end{center}
      By assumption, $c(\Pi_1),c(\Pi_2)\leq |A|$. By induction, there is a
      proof $\Pi'$ from $\pi$ and $\Pi_2$ for sequent
      $[[D1; X; Y; G2; D2 |-l B]]$ s.t. $c(\Pi')\leq |A|$. Therefore, the
      proof $\Pi$ can be constructed as follows with $c(\Pi)\leq |A|$.
      \begin{center}
        \scriptsize
        \begin{math}
          $$\mprset{flushleft}
          \inferrule* [right={\tiny tenL1}] {
            $$\mprset{flushleft}
            \inferrule* [right={\tiny cut2}] {
              {
                \begin{array}{cc}
                  \pi & \Pi_2 \\
                  {[[G1; X; Y; G2 |-l A]]} & {[[D1; A; D2 |-l B]]}
                \end{array}
              }
            }{[[D1; G1; X; Y; G2; D2 |-l B]]}
          }{[[D1; G1; X (*) Y; G2; D2 |-l B]]}
        \end{math}
      \end{center}
\end{itemize}

\subsubsection{Left introduction of the non-commutative tensor products $\tri$}
\begin{center}
  \scriptsize
  $\Pi_1$:
  \begin{math}
    $$\mprset{flushleft}
    \inferrule* [right={\tiny tenL2}] {
      {
        \begin{array}{c}
          \pi \\
          {[[G1; A1; A2; G2 |-l B]]}
        \end{array}
      }
    }{[[G1; A1 (>) A2; G2 |-l B]]}
  \end{math}
  \qquad\qquad
  \begin{math}
    \begin{array}{c}
      \Pi_2 \\
      {[[D1; B; D2 |-l C]]}
    \end{array}
  \end{math}
\end{center}
By assumption, $c(\Pi_1),c(\Pi_2)\leq |B|$. By induction, there is a
proof $\Pi'$ from $\pi$ and $\Pi_2$ for sequent \\
$[[D1; G1; A1; A2; G2; D2 |-l C]]$ s.t. $c(\Pi')\leq |B|$.
Therefore, the proof $\Pi$ can be constructed as follows with
$c(\Pi)\leq |B|$.
\begin{center}
  \scriptsize
  \begin{math}
    $$\mprset{flushleft}
    \inferrule* [right={\tiny tenL2}] {
      $$\mprset{flushleft}
      \inferrule* [right={\tiny cut2}] {
        {
          \begin{array}{cc}
            \pi & \Pi_2 \\
            {[[G1; A1; A2; G2 |-l B]]} & {[[D1; B; D2 |-l C]]}
          \end{array}
        }
      }{[[D1; G1; A1; A2; G2; D2 |-l C]]}
    }{[[D1; G1; A1 (>) A2; G2; D2 |-l C]]}
  \end{math}
\end{center}



\subsubsection{Left introduction of the commutative unit $[[UnitT]]$}
\begin{itemize}
\item Case 1:
      \begin{center}
        \scriptsize
        $\Pi_1$:
        \begin{math}
          $$\mprset{flushleft}
          \inferrule* [right={\tiny unitL}] {
            {
              \begin{array}{c}
                \pi \\
                {[[I1, I2 |-c X]]}
              \end{array}
            }
          }{[[I1, UnitT, I2 |-c X]]}
        \end{math}
        \qquad\qquad
        \begin{math}
          \begin{array}{c}
            \Pi_2 \\
            {[[P1, X, P2 |-c Y]]}
          \end{array}
        \end{math}
      \end{center}
      By assumption, $c(\Pi_1),c(\Pi_2)\leq |X|$. By induction, there is a
      proof $\Pi'$ from $\pi$ and $\Pi_2$ for sequent
      $[[P1, I1, I2, P2 |-c Y]]$
      s.t. $c(\Pi')\leq |X|$. Therefore, the proof $\Pi$ can be constructed
      as follows, and $c(\Pi)=c(\Pi')\leq |X|$.
      \begin{center}
        \scriptsize
        \begin{math}
          $$\mprset{flushleft}
          \inferrule* [right={\tiny unitL}] {
            {
              \begin{array}{c}
                \Pi' \\
                {[[P1, I1, I2, P2 |-c Y]]}
              \end{array}
            }
          }{[[P1, I1, UnitT, I2, P2 |-c Y]]}
        \end{math}
      \end{center}

\item Case 2:
      \begin{center}
        \scriptsize
        $\Pi_1$:
        \begin{math}
          $$\mprset{flushleft}
          \inferrule* [right={\tiny unitL}] {
            {
              \begin{array}{c}
                \pi \\
                {[[I1, I2 |-c X]]}
              \end{array}
            }
          }{[[I1, UnitT, I2 |-c X]]}
        \end{math}
        \qquad\qquad
        \begin{math}
          \begin{array}{c}
            \Pi_2 \\
            {[[G1; X; G2 |-l A]]}
          \end{array}
        \end{math}
      \end{center}
      By assumption, $c(\Pi_1),c(\Pi_2)\leq |X|$. By induction, there is a
      proof $\Pi'$ from $\pi$ and $\Pi_2$ for sequent
      $[[G1; I1; I2; G2 |-l A]]$
      s.t. $c(\Pi')\leq |X|$. Therefore, the proof $\Pi$ can be constructed
      as follows, and $c(\Pi)=c(\Pi')\leq |X|$.
      \begin{center}
        \scriptsize
        \begin{math}
          $$\mprset{flushleft}
          \inferrule* [right={\tiny unitL}] {
            {
              \begin{array}{c}
                \Pi' \\
                {[[G1; I1; I2; G2 |-l A]]}
              \end{array}
            }
          }{[[G1; I1; UnitT; I2; G2 |-l A]]}
        \end{math}
      \end{center}

\item Case 3:
      \begin{center}
        \scriptsize
        $\Pi_1$:
        \begin{math}
          $$\mprset{flushleft}
          \inferrule* [right={\tiny unitL}] {
            {
              \begin{array}{c}
                \pi \\
                {[[D1; D2 |-l A]]}
              \end{array}
            }
          }{[[D1; UnitT; D2 |-l A]]}
        \end{math}
        \qquad\qquad
        \begin{math}
          \begin{array}{c}
            \Pi_2 \\
            {[[G1; A; G2 |-l B]]}
          \end{array}
        \end{math}
      \end{center}
      By assumption, $c(\Pi_1),c(\Pi_2)\leq |X|$. By induction, there is a
      proof $\Pi'$ from $\pi$ and $\Pi_2$ for sequent
      $[[G1; D1; D2; G2 |-l B]]$
      s.t. $c(\Pi')\leq |X|$. Therefore, the proof $\Pi$ can be constructed
      as follows, and $c(\Pi)=c(\Pi')\leq |X|$.
      \begin{center}
        \scriptsize
        \begin{math}
          $$\mprset{flushleft}
          \inferrule* [right={\tiny unitL}] {
            {
              \begin{array}{c}
                \Pi' \\
                {[[G1; D1; D2; G2 |-l B]]}
              \end{array}
            }
          }{[[G1; D1; UnitT; D2; G2 |-l B]]}
        \end{math}
      \end{center}
\end{itemize}



\subsubsection{Left introduction of the non-commutative unit $[[UnitS]]$}
\begin{center}
  \scriptsize
  $\Pi_1$:
  \begin{math}
    $$\mprset{flushleft}
    \inferrule* [right={\tiny unitL}] {
      {
        \begin{array}{c}
          \pi \\
          {[[D1; D2 |-l A]]}
        \end{array}
      }
    }{[[D1; UnitS; D2 |-l A]]}
  \end{math}
  \qquad\qquad
  \begin{math}
    \begin{array}{c}
      \Pi_2 \\
      {[[G1; A; G2 |-l B]]}
    \end{array}
  \end{math}
\end{center}
By assumption, $c(\Pi_1),c(\Pi_2)\leq |X|$. By induction, there is a
proof $\Pi'$ from $\pi$ and $\Pi_2$ for sequent \\
$[[G1; D1; D2; G2 |-l B]]$ s.t. $c(\Pi')\leq |X|$. Therefore, the proof
$\Pi$ can be constructed as follows, and \\
$c(\Pi)=c(\Pi')\leq |X|$.
\begin{center}
  \scriptsize
  \begin{math}
    $$\mprset{flushleft}
    \inferrule* [right={\tiny unitL}] {
      {
        \begin{array}{c}
          \Pi' \\
          {[[G1; D1; D2; G2 |-l B]]}
        \end{array}
      }
    }{[[G1; D1; UnitS; D2; G2 |-l B]]}
  \end{math}
\end{center}



\subsubsection{Left introduction of the functor $F$}
\begin{center}
  \scriptsize
  $\Pi_1$:
  \begin{math}
    $$\mprset{flushleft}
    \inferrule* [right={\tiny FL}] {
      {
        \begin{array}{c}
          \pi_1 \\
          {[[G1; X; G2 |-l A]]}
        \end{array}
      }
    }{[[G1; F X; G2 |-l A]]}
  \end{math}
  \qquad\qquad
  \begin{math}
    \begin{array}{c}
      \Pi_2 \\
      {[[D1; A; D2 |-l B]]}
    \end{array}
  \end{math}
\end{center}
By assumption, $c(\Pi_1),c(\Pi_2)\leq |A|$. By induction, there is a
proof $\Pi'$ from $\pi_2$ and $\Pi_2$ for sequent
$[[D1; G1; X; G2; D2 |-l B]]$ s.t. $c(\Pi')\leq |A|$. Therefore, the
proof $\Pi$ can be constructed as follows with $c(\Pi)\leq |A|$.
\begin{center}
  \scriptsize
  \begin{math}
    $$\mprset{flushleft}
    \inferrule* [right={\tiny FL}] {
      $$\mprset{flushleft}
      \inferrule* [right={\tiny cut2}] {
        {
          \begin{array}{cc}
            \pi_2 & \Pi_2 \\
            {[[G1; X; G2 |-l A]]} & {[[D1; A; D2 |-l B]]}
          \end{array}
        }
      }{[[D1; G1; X; G2; D2 |-l B]]}
    }{[[D1; G1; F X; G2; D2 |-l B]]}
  \end{math}
\end{center}

\subsubsection{Left introduction of the functor $G$}
\begin{center}
  \scriptsize
  $\Pi_1$:
  \begin{math}
    $$\mprset{flushleft}
    \inferrule* [right={\tiny GL}] {
      {
        \begin{array}{c}
          \pi_1 \\
          {[[G1; A; G2 |-l B]]}
        \end{array}
      }
    }{[[G1; Gf A; G2 |-l B]]}
  \end{math}
  \qquad\qquad
  \begin{math}
    \begin{array}{c}
      \Pi_2 \\
      {[[D1; B; D2 |-l C]]}
    \end{array}
  \end{math}
\end{center}
By assumption, $c(\Pi_1),c(\Pi_2)\leq |B|$. By induction, there is a
proof $\Pi'$ from $\pi_2$ and $\Pi_2$ for sequent
$[[D1; G1; A; G2; D2 |-l C]]$ s.t. $c(\Pi')\leq |B|$. Therefore, the
proof $\Pi$ can be constructed as follows with $c(\Pi)\leq |B|$.
\begin{center}
  \scriptsize
  \begin{math}
    $$\mprset{flushleft}
    \inferrule* [right={\tiny GL}] {
      $$\mprset{flushleft}
      \inferrule* [right={\tiny cut2}] {
        {
          \begin{array}{cc}
            \pi_2 & \Pi_2 \\
            {[[G1; A; G2 |-l B]]} & {[[D1; B; D2 |-l C]]}
          \end{array}
        }
      }{[[D1; G1; A; G2; D2 |-l C]]}
    }{[[D1; G1; Gf A; G2; D2 |-l C]]}
  \end{math}
\end{center}



\subsection{Secondary Hypothesis}

\subsubsection{Right introduction of the commutative tensor product $\otimes$}
\begin{itemize}
\item Case 1:
      \begin{center}
        \scriptsize
        \begin{math}
          \begin{array}{c}
            \Pi_1 \\
            {[[I2 |-c X]]}
          \end{array}
        \end{math}
        \qquad\qquad
        $\Pi_2$:
        \begin{math}
          $$\mprset{flushleft}
          \inferrule* [right={\tiny tenR}] {
            {
              \begin{array}{cc}
                \pi_1 & \pi_2 \\
                {[[P1, X, P2 |-c Y1]]} & {[[I1 |-c Y2]]}
              \end{array}
            }
          }{[[P1, X, P2, I1 |-c Y1 (*) Y2]]}
        \end{math}
      \end{center}
      By assumption, $c(\Pi_1),c(\Pi_2)\leq |X|$. By induction on $\Pi_1$
      and $\pi_1$, there is a proof $\Pi'$ for sequent
      $[[P1, I2, P2 |-c Y1]]$ s.t. $c(\Pi') \leq |X|$. Therefore, the proof
      $\Pi$ can be constructed as follows with $c(\Pi) = c(\Pi') \leq |X|$.
      \begin{center}
        \scriptsize
        \begin{math}
          $$\mprset{flushleft}
          \inferrule* [right={\tiny tenR}] {
            {
              \begin{array}{cc}
                \Pi' & \pi_1 \\
                {[[P1, I2, P2 |-c Y1]]} & {[[I1 |-c Y2]]}
              \end{array}
            }
          }{[[P1, I2, P2, I1 |-c Y1 (*) Y2]]}
        \end{math}
      \end{center}

\item Case 2:
      \begin{center}
        \scriptsize
        \begin{math}
          \begin{array}{c}
            \Pi_1 \\
            {[[I2 |-c X]]}
          \end{array}
        \end{math}
        \qquad\qquad
        $\Pi_2$:
        \begin{math}
          $$\mprset{flushleft}
          \inferrule* [right={\tiny tenR}] {
            {
              \begin{array}{cc}
                \pi_1 & \pi_2 \\
                {[[I1 |-c Y1]]} & {[[P1, X, P2 |-c Y2]]}
              \end{array}
            }
          }{[[I1, P1, X, P2 |-c Y1 (*) Y2]]}
        \end{math}
      \end{center}
      By assumption, $c(\Pi_1),c(\Pi_2)\leq |X|$. By induction on $\Pi_1$
      and $\pi_2$, there is a proof $\Pi'$ for sequent
      $[[P1, I2, P2 |-c Y2]]$ s.t. $c(\Pi') \leq |X|$. Therefore, the proof
      $\Pi$ can be constructed as follows with $c(\Pi) = c(\Pi') \leq |X|$.
      \begin{center}
        \scriptsize
        \begin{math}
          $$\mprset{flushleft}
          \inferrule* [right={\tiny tenR}] {
            {
              \begin{array}{cc}
                \pi_1 & \Pi' \\
                {[[I1 |-c Y1]]} & {[[P1, I2, P2 |-c Y2]]}
              \end{array}
            }
          }{[[I1, P1, I2, P2 |-c Y1 (*) Y2]]}
        \end{math}
      \end{center}
\end{itemize}



\subsubsection{Right introduction of the non-commutative tensor product $\tri$}
\begin{itemize}
\item Case 1:
      \begin{center}
        \scriptsize
        \begin{math}
          \begin{array}{c}
            \Pi_1 \\
            {[[I |-c X]]}
          \end{array}
        \end{math}
        \qquad\qquad
        $\Pi_2$:
        \begin{math}
          $$\mprset{flushleft}
          \inferrule* [right={\tiny tenR}] {
            {
              \begin{array}{cc}
                \pi_1 & \pi_2 \\
                {[[G1; X; G2 |-l A]]} & {[[G3 |-l B]]}
              \end{array}
            }
          }{[[G1; X; G2; G3 |-l A (>) B]]}
        \end{math}
      \end{center}
      By assumption, $c(\Pi_1),c(\Pi_2)\leq |X|$. By induction on $\Pi_1$
      and $\pi_1$, there is a proof $\Pi'$ for sequent
      $[[G1; I; G2 |-l A]]$ s.t. $c(\Pi') \leq |X|$. Therefore, the proof
      $\Pi$ can be constructed as follows with $c(\Pi) = c(\Pi') \leq |X|$.
      \begin{center}
        \scriptsize
        \begin{math}
          $$\mprset{flushleft}
          \inferrule* [right={\tiny tenR}] {
            {
              \begin{array}{cc}
                \Pi' & \pi_1 \\
                {[[G1; I; G2 |-l A]]} & {[[G3 |-l B]]}
              \end{array}
            }
          }{[[G1; I; G2; G3 |-l A (>) B]]}
        \end{math}
      \end{center}

\item Case 2:
      \begin{center}
        \scriptsize
        \begin{math}
          \begin{array}{c}
            \Pi_1 \\
            {[[D |-l C]]}
          \end{array}
        \end{math}
        \qquad\qquad
        $\Pi_2$:
        \begin{math}
          $$\mprset{flushleft}
          \inferrule* [right={\tiny tenR}] {
            {
              \begin{array}{cc}
                \pi_1 & \pi_2 \\
                {[[G1; C; G2 |-l A]]} & {[[G3 |-l B]]}
              \end{array}
            }
          }{[[G1; C; G2; G3 |-l A (>) B]]}
        \end{math}
      \end{center}
      By assumption, $c(\Pi_1),c(\Pi_2)\leq |C|$. By induction on $\Pi_1$
      and $\pi_1$, there is a proof $\Pi'$ for sequent
      $[[G1; D; G2 |-l A]]$ s.t. $c(\Pi') \leq |C|$. Therefore, the proof
      $\Pi$ can be constructed as follows with $c(\Pi) = c(\Pi') \leq |C|$.
      \begin{center}
        \scriptsize
        \begin{math}
          $$\mprset{flushleft}
          \inferrule* [right={\tiny tenR}] {
            {
              \begin{array}{cc}
                \Pi' & \pi_1 \\
                {[[G1; D; G2 |-l A]]} & {[[G3 |-l B]]}
              \end{array}
            }
          }{[[G1; D; G2; G3 |-l A (>) B]]}
        \end{math}
      \end{center}

\item Case 3:
      \begin{center}
        \scriptsize
        \begin{math}
          \begin{array}{c}
            \Pi_1 \\
            {[[I |-c X]]}
          \end{array}
        \end{math}
        \qquad\qquad
        $\Pi_2$:
        \begin{math}
          $$\mprset{flushleft}
          \inferrule* [right={\tiny tenR}] {
            {
              \begin{array}{cc}
                \pi_1 & \pi_2 \\
                {[[G1 |-l A]]} & {[[G2; X; G3 |-l B]]}
              \end{array}
            }
          }{[[G1; G2; X; G3 |-l A (>) B]]}
        \end{math}
      \end{center}
      By assumption, $c(\Pi_1),c(\Pi_2)\leq |X|$. By induction on $\Pi_1$
      and $\pi_2$, there is a proof $\Pi'$ for sequent
      $[[G2; I; G3 |-l B]]$ s.t. $c(\Pi') \leq |X|$. Therefore, the proof
      $\Pi$ can be constructed as follows with $c(\Pi) = c(\Pi') \leq |X|$.
      \begin{center}
        \scriptsize
        \begin{math}
          $$\mprset{flushleft}
          \inferrule* [right={\tiny tenR}] {
            {
              \begin{array}{cc}
                \pi_1 & \Pi' \\
                {[[G1 |-l A]]} & {[[G2; I; G3 |-l B]]}
              \end{array}
            }
          }{[[G1; G2; I; G3 |-l A (>) B]]}
        \end{math}
      \end{center}

\item Case 4:
      \begin{center}
        \scriptsize
        \begin{math}
          \begin{array}{c}
            \Pi_1 \\
            {[[D |-l C]]}
          \end{array}
        \end{math}
        \qquad\qquad
        $\Pi_2$:
        \begin{math}
          $$\mprset{flushleft}
          \inferrule* [right={\tiny tenR}] {
            {
              \begin{array}{cc}
                \pi_1 & \pi_2 \\
                {[[G1 |-l A]]} & {[[G2; C; G3 |-l B]]}
              \end{array}
            }
          }{[[G1; G2; C; G3 |-l A (>) B]]}
        \end{math}
      \end{center}
      By assumption, $c(\Pi_1),c(\Pi_2)\leq |C|$. By induction on $\Pi_1$
      and $\pi_2$, there is a proof $\Pi'$ for sequent
      $[[G2; D; G3 |-l B]]$ s.t. $c(\Pi') \leq |C|$. Therefore, the proof
      $\Pi$ can be constructed as follows with $c(\Pi) = c(\Pi') \leq |C|$.
      \begin{center}
        \scriptsize
        \begin{math}
          $$\mprset{flushleft}
          \inferrule* [right={\tiny tenR}] {
            {
              \begin{array}{cc}
                \pi_1 & \Pi' \\
                {[[G1 |-l A]]} & {[[G2; D; G3 |-l B]]}
              \end{array}
            }
          }{[[G1; G2; D; G3 |-l A (>) B]]}
        \end{math}
      \end{center}
\end{itemize}



\subsubsection{Left introduction of the commutative implication $\multimap$}
\begin{itemize}
\item Case 1:
      \begin{center}
        \scriptsize
        \begin{math}
          \begin{array}{c}
            \Pi_1 \\
            {[[I |-c X]]}
          \end{array}
        \end{math}
        \qquad\qquad
        $\Pi_2$:
        \begin{math}
          $$\mprset{flushleft}
          \inferrule* [right={\tiny impL}] {
            {
              \begin{array}{cc}
                \pi_1 & \pi_2 \\
                {[[P2, X, P3 |-c Y1]]} & {[[P1, Y2, P4 |-c Z]]}
              \end{array}
            }
          }{[[P1, Y1 -o Y2, P2, X, P3, P4 |-c Z]]}
        \end{math}
      \end{center}
      By assumption, $c(\Pi_1),c(\Pi_2)\leq |X|$. By induction on $\Pi_1$ and $\pi_1$, there is
      a proof $\Pi'$ for sequent $[[P2, I, P3 |-c Y1]]$ s.t. $c(\Pi') \leq |X|$. Therefore, the
      proof $\Pi$ can be constructed as follows with $c(\Pi) = c(\Pi') \leq |X|$.
      \begin{center}
        \scriptsize
        \begin{math}
          $$\mprset{flushleft}
          \inferrule* [right={\tiny impL}] {
            {
              \begin{array}{cc}
                \Pi' & \pi_2 \\
                {[[P2, I, P3 |-c Y1]]} & {[[P1, Y2, P4 |-c Z]]}
              \end{array}
            }
          }{[[P1, Y1 -o Y2, P2, I, P3, P4 |-c Z]]}
        \end{math}
      \end{center}

\item Case 2:
      \begin{center}
        \scriptsize
        \begin{math}
          \begin{array}{c}
            \Pi_1 \\
            {[[I |-c X]]}
          \end{array}
        \end{math}
        \qquad\qquad
        $\Pi_2$:
        \begin{math}
          $$\mprset{flushleft}
          \inferrule* [right={\tiny impL}] {
            {
              \begin{array}{cc}
                \pi_1 & \pi_2 \\
                {[[P3 |-c Y1]]} & {[[P1, X, P2, Y2, P4 |-c Z]]}
              \end{array}
            }
          }{[[P1, X, P2, Y1 -o Y2, P3, P4 |-c Z]]}
        \end{math}
      \end{center}
      By assumption, $c(\Pi_1),c(\Pi_2)\leq |X|$. By induction on $\Pi_1$ and $\pi_2$, there is
      a proof $\Pi'$ for sequent $[[P1, I, P2, Y2, P4 |-c Z]]$ s.t. $c(\Pi') \leq |X|$.
      Therefore, the proof $\Pi$ can be constructed as follows with
      $c(\Pi) = c(\Pi') \leq |X|$.
      \begin{center}
        \scriptsize
        \begin{math}
          $$\mprset{flushleft}
          \inferrule* [right={\tiny impL}] {
            {
              \begin{array}{cc}
                \pi_1 & \Pi' \\
                {[[P3 |-c Y1]]} & {[[P1, I, P2, Y2, P4 |-c Z]]}
              \end{array}
            }
          }{[[P1, I1, P2, Y1 -o Y2, P3, P4 |-c Z]]}
        \end{math}
      \end{center}

\item Case 3:
      \begin{center}
        \scriptsize
        \begin{math}
          \begin{array}{c}
            \Pi_1 \\
            {[[I |-c X]]}
          \end{array}
        \end{math}
        \qquad\qquad
        $\Pi_2$:
        \begin{math}
          $$\mprset{flushleft}
          \inferrule* [right={\tiny impL}] {
            {
              \begin{array}{cc}
                \pi_1 & \pi_2 \\
                {[[P2 |-c Y1]]} & {[[P1, Y2, P3, X, P4 |-c Z]]}
              \end{array}
            }
          }{[[P1, Y1 -o Y2, P2, P3, X, P4 |-c Z]]}
        \end{math}
      \end{center}
      By assumption, $c(\Pi_1),c(\Pi_2)\leq |X|$. By induction on $\Pi_1$
      and $\pi_2$, there is a proof $\Pi'$ for sequent
      $[[P1, I, P2, Y2, P4 |-c Z]]$ s.t. $c(\Pi') \leq |X|$. Therefore,
      the proof $\Pi$ can be constructed as follows with
      $c(\Pi) = c(\Pi') \leq |X|$.
      \begin{center}
        \scriptsize
        \begin{math}
          $$\mprset{flushleft}
          \inferrule* [right={\tiny impL}] {
            {
              \begin{array}{cc}
                \pi_1 & \Pi' \\
                {[[P2 |-c Y1]]} & {[[P1, Y2, P3, I, P4 |-c Z]]}
              \end{array}
            }
          }{[[P1, Y1 -o Y2, P2, P3, I, P4 |-c Z]]}
        \end{math}
      \end{center}

\item Case 4:
      \begin{center}
        \scriptsize
        \begin{math}
          \begin{array}{c}
            \Pi_1 \\
            {[[I |-c X]]}
          \end{array}
        \end{math}
        \qquad\qquad
        $\Pi_2$:
        \begin{math}
          $$\mprset{flushleft}
          \inferrule* [right={\tiny impL}] {
            {
              \begin{array}{cc}
                \pi_1 & \pi_2 \\
                {[[P1, X, P2 |-c Y1]]} & {[[G1; Y2; G2 |-l A]]}
              \end{array}
            }
          }{[[G1; Y1 -o Y2; P1; X; P2; G2 |-l A]]}
        \end{math}
      \end{center}
      By assumption, $c(\Pi_1),c(\Pi_2)\leq |X|$. By induction on $\Pi_1$
      and $\pi_1$, there is a proof $\Pi'$ for sequent
      $[[P1, I, P2 |-c Y1]]$ s.t. $c(\Pi') \leq |X|$. Therefore, the proof
      $\Pi$ can be constructed as follows with $c(\Pi) = c(\Pi') \leq |X|$.
      \begin{center}
        \scriptsize
        \begin{math}
          $$\mprset{flushleft}
          \inferrule* [right={\tiny impL}] {
            {
              \begin{array}{cc}
                \Pi' & \pi_2 \\
                {[[P1, I, P2 |-c Y1]]} & {[[G1; Y2; G2 |-l A]]}
              \end{array}
            }
          }{[[G1; Y1 -o Y2; P1; I; P2; G2 |-l A]]}
        \end{math}
      \end{center}

\item Case 5:
      \begin{center}
        \scriptsize
        \begin{math}
          \begin{array}{c}
            \Pi_1 \\
            {[[I |-c X]]}
          \end{array}
        \end{math}
        \qquad\qquad
        $\Pi_2$:
        \begin{math}
          $$\mprset{flushleft}
          \inferrule* [right={\tiny impL}] {
            {
              \begin{array}{cc}
                \pi_1 & \pi_2 \\
                {[[P |-c Y1]]} & {[[G1; X; G2; Y2; G3 |-l A]]}
              \end{array}
            }
          }{[[G1; X; G2; Y1 -o Y2; P; G3 |-l A]]}
        \end{math}
      \end{center}
      By assumption, $c(\Pi_1),c(\Pi_2)\leq |X|$. By induction on $\Pi_1$
      and $\pi_2$, there is a proof $\Pi'$ for sequent
      $[[G1; I; G2; Y2; G3 |-l A]]$ s.t. $c(\Pi') \leq |X|$. Therefore, the
      proof $\Pi$ can be constructed as follows with
      $c(\Pi) = c(\Pi') \leq |X|$.
      \begin{center}
        \scriptsize
        \begin{math}
          $$\mprset{flushleft}
          \inferrule* [right={\tiny impL}] {
            {
              \begin{array}{cc}
                \pi_1 & \Pi' \\
                {[[P |-c Y1]]} & {[[G1; I; G2; Y2; G3 |-l A]]}
              \end{array}
            }
          }{[[G1; I; G2; Y1 -o Y2; P; G3 |-l A]]}
        \end{math}
      \end{center}

\item Case 6:
      \begin{center}
        \scriptsize
        \begin{math}
          \begin{array}{c}
            \Pi_1 \\
            {[[D |-l B]]}
          \end{array}
        \end{math}
        \qquad\qquad
        $\Pi_2$:
        \begin{math}
          $$\mprset{flushleft}
          \inferrule* [right={\tiny impL}] {
            {
              \begin{array}{cc}
                \pi_1 & \pi_2 \\
                {[[P |-c Y1]]} & {[[G1; B; G2; Y2; G3 |-l A]]}
              \end{array}
            }
          }{[[G1; B; G2; Y1 -o Y2; P; G3 |-l A]]}
        \end{math}
      \end{center}
      By assumption, $c(\Pi_1),c(\Pi_2)\leq |B|$. By induction on $\Pi_1$
      and $\pi_2$, there is a proof $\Pi'$ for sequent
      $[[G1; D; G2; Y2; G3 |-l A]]$ s.t. $c(\Pi') \leq |B|$. Therefore, the
      proof $\Pi$ can be constructed as follows with
      $c(\Pi) = c(\Pi') \leq |B|$.
      \begin{center}
        \scriptsize
        \begin{math}
          $$\mprset{flushleft}
          \inferrule* [right={\tiny impL}] {
            {
              \begin{array}{cc}
                \pi_1 & \Pi' \\
                {[[P |-c Y1]]} & {[[G1; D; G2; Y2; G3 |-l A]]}
              \end{array}
            }
          }{[[G1; D; G2; Y1 -o Y2; P; G3 |-l A]]}
        \end{math}
      \end{center}

\item Case 7:
      \begin{center}
        \scriptsize
        \begin{math}
          \begin{array}{c}
            \Pi_1 \\
            {[[I |-c X]]}
          \end{array}
        \end{math}
        \qquad\qquad
        $\Pi_2$:
        \begin{math}
          $$\mprset{flushleft}
          \inferrule* [right={\tiny impL}] {
            {
              \begin{array}{cc}
                \pi_1 & \pi_2 \\
                {[[P |-c Y1]]} & {[[G1; Y2; G2; X; G3 |-l A]]}
              \end{array}
            }
          }{[[G1; Y1 -o Y2; P; G2; X; G3 |-l A]]}
        \end{math}
      \end{center}
      By assumption, $c(\Pi_1),c(\Pi_2)\leq |X|$. By induction on $\Pi_1$
      and $\pi_2$, there is a proof $\Pi'$ for sequent
      $[[G1; Y2; G2; I; G3 |-l A]]$ s.t. $c(\Pi') \leq |X|$. Therefore, the
      proof $\Pi$ can be constructed as follows with
      $c(\Pi) = c(\Pi') \leq |X|$.
      \begin{center}
        \scriptsize
        \begin{math}
          $$\mprset{flushleft}
          \inferrule* [right={\tiny impL}] {
            {
              \begin{array}{cc}
                \pi_1 & \Pi' \\
                {[[P |-c Y1]]} & {[[G1; Y2; G2; I; G3 |-l A]]}
              \end{array}
            }
          }{[[G1; Y1 -o Y2; P; G2; I; G3 |-l A]]}
        \end{math}
      \end{center}

\item Case 8:
      \begin{center}
        \scriptsize
        \begin{math}
          \begin{array}{c}
            \Pi_1 \\
            {[[D |-l B]]}
          \end{array}
        \end{math}
        \qquad\qquad
        $\Pi_2$:
        \begin{math}
          $$\mprset{flushleft}
          \inferrule* [right={\tiny impL}] {
            {
              \begin{array}{cc}
                \pi_1 & \pi_2 \\
                {[[P |-c Y1]]} & {[[G1; Y2; G2; B; G3 |-l A]]}
              \end{array}
            }
          }{[[G1; Y1 -o Y2; P; G2; B; G3 |-l A]]}
        \end{math}
      \end{center}
      By assumption, $c(\Pi_1),c(\Pi_2)\leq |B|$. By induction on $\Pi_1$
      and $\pi_2$, there is a proof $\Pi'$ for sequent
      $[[G1; Y2; G2; D; G3 |-l A]]$ s.t. $c(\Pi') \leq |B|$. Therefore,
      the proof $\Pi$ can be constructed as follows with
      $c(\Pi) = c(\Pi') \leq |B|$.
      \begin{center}
        \scriptsize
        \begin{math}
          $$\mprset{flushleft}
          \inferrule* [right={\tiny impL}] {
            {
              \begin{array}{cc}
                \pi_1 & \Pi' \\
                {[[P |-c Y1]]} & {[[G1; Y2; G2; D; G3 |-l A]]}
              \end{array}
            }
          }{[[G1; Y1 -o Y2; P; G2; D; G3 |-l A]]}
        \end{math}
      \end{center}
\end{itemize}



\subsubsection{Left introduction of the non-commutative left implication $\lto$}
\begin{itemize}
\item Case 1:
      \begin{center}
        \scriptsize
        \begin{math}
          \begin{array}{c}
            \Pi_1 \\
            {[[I |-c X]]}
          \end{array}
        \end{math}
        \qquad\qquad
        $\Pi_2$:
        \begin{math}
          $$\mprset{flushleft}
          \inferrule* [right={\tiny imprL}] {
            {
              \begin{array}{cc}
                \pi_1 & \pi_2 \\
                {[[D1; X; D2 |-l A1]]} & {[[G1; A2; G2 |-l B]]}
              \end{array}
            }
          }{[[G1; A1 -> A2; D1; X; D2; G2 |-l B]]}
        \end{math}
      \end{center}
      By assumption, $c(\Pi_1),c(\Pi_2)\leq |X|$. By induction on $\Pi_1$
      and $\pi_1$, there is a proof $\Pi'$ for sequent
      $[[D1; I; D2 |-l A1]]$ s.t. $c(\Pi') \leq |X|$. Therefore, the proof
      $\Pi$ can be constructed as follows with $c(\Pi) = c(\Pi') \leq |X|$.
      \begin{center}
        \scriptsize
        \begin{math}
          $$\mprset{flushleft}
          \inferrule* [right={\tiny impL}] {
            {
              \begin{array}{cc}
                \Pi' & \pi_2 \\
                {[[D1; I; D2 |-l A1]]} & {[[G1; A2; G2 |-l B]]}
              \end{array}
            }
          }{[[G1; A1 -> A2; D1; I; D2; G2 |-l B]]}
        \end{math}
      \end{center}

\item Case 2:
      \begin{center}
        \scriptsize
        \begin{math}
          \begin{array}{c}
            \Pi_1 \\
            {[[G |-l C]]}
          \end{array}
        \end{math}
        \qquad\qquad
        $\Pi_2$:
        \begin{math}
          $$\mprset{flushleft}
          \inferrule* [right={\tiny imprL}] {
            {
              \begin{array}{cc}
                \pi_1 & \pi_2 \\
                {[[D1; C; D2 |-l A1]]} & {[[G1; A2; G2 |-l B]]}
              \end{array}
            }
          }{[[G1; A1 -> A2; D1; C; D2; G2 |-l B]]}
        \end{math}
      \end{center}
      By assumption, $c(\Pi_1),c(\Pi_2)\leq |C|$. By induction on $\Pi_1$
      and $\pi_1$, there is a proof $\Pi'$ for sequent
      $[[D1; G; D2 |-l A1]]$ s.t. $c(\Pi') \leq |C|$. Therefore, the proof
      $\Pi$ can be constructed as follows with $c(\Pi) = c(\Pi') \leq |C|$.
      \begin{center}
        \scriptsize
        \begin{math}
          $$\mprset{flushleft}
          \inferrule* [right={\tiny imprL}] {
            {
              \begin{array}{cc}
                \Pi' & \pi_2 \\
                {[[D1; G; D2 |-l A1]]} & {[[G1; A2; G2 |-l B]]}
              \end{array}
            }
          }{[[G1; A1 -> A2; D1; G; D2; G2 |-l B]]}
        \end{math}
      \end{center}

\item Case 3:
      \begin{center}
        \scriptsize
        \begin{math}
          \begin{array}{c}
            \Pi_1 \\
            {[[I |-c X]]}
          \end{array}
        \end{math}
        \qquad\qquad
        $\Pi_2$:
        \begin{math}
          $$\mprset{flushleft}
          \inferrule* [right={\tiny imprL}] {
            {
              \begin{array}{cc}
                \pi_1 & \pi_2 \\
                {[[D |-l A1]]} & {[[G1; X; G2; A2; G3 |-l B]]}
              \end{array}
            }
          }{[[G1; X; G2; A1 -> A2; D; G3 |-l B]]}
        \end{math}
      \end{center}
      By assumption, $c(\Pi_1),c(\Pi_2)\leq |X|$. By induction on $\Pi_1$
      and $\pi_2$, there is a proof $\Pi'$ for sequent
      $[[G1; I; G2; A2; G3 |-l B]]$ s.t. $c(\Pi') \leq |X|$. Therefore,
      the proof $\Pi$ can be constructed as follows with
      $c(\Pi) = c(\Pi') \leq |X|$.
      \begin{center}
        \scriptsize
        \begin{math}
          $$\mprset{flushleft}
          \inferrule* [right={\tiny imprL}] {
            {
              \begin{array}{cc}
                \pi_1 & \Pi' \\
                {[[D |-l A1]]} & {[[G1; I; G2; A2; G3 |-l B]]}
              \end{array}
            }
          }{[[G1; I; G2; A1 -> A2; D; G3 |-l B]]}
        \end{math}
      \end{center}

\item Case 4:
      \begin{center}
        \scriptsize
        \begin{math}
          \begin{array}{c}
            \Pi_1 \\
            {[[D1 |-l B]]}
          \end{array}
        \end{math}
        \qquad\qquad
        $\Pi_2$:
        \begin{math}
          $$\mprset{flushleft}
          \inferrule* [right={\tiny imprL}] {
            {
              \begin{array}{cc}
                \pi_1 & \pi_2 \\
                {[[D2 |-l A1]]} & {[[G1; B; G2; A2; G3 |-l C]]}
              \end{array}
            }
          }{[[G1; B; G2; A1 -> A2; D2; G3 |-l C]]}
        \end{math}
      \end{center}
      By assumption, $c(\Pi_1),c(\Pi_2)\leq |B|$. By induction on $\Pi_1$
      and $\pi_2$, there is a proof $\Pi'$ for sequent
      $[[G1; D1; G2; A2; G3 |-l C]]$ s.t. $c(\Pi') \leq |B|$. Therefore,
      the proof $\Pi$ can be constructed as follows with
      $c(\Pi) = c(\Pi') \leq |B|$.
      \begin{center}
        \scriptsize
        \begin{math}
          $$\mprset{flushleft}
          \inferrule* [right={\tiny imprL}] {
            {
              \begin{array}{cc}
                \pi_1 & \Pi' \\
                {[[D2 |-l A1]]} & {[[G1; D1; G2; A2; G3 |-l C]]}
              \end{array}
            }
          }{[[G1; D1; G2; A1 -> A2; D2; G3 |-l C]]}
        \end{math}
      \end{center}

\item Case 5:
      \begin{center}
        \scriptsize
        \begin{math}
          \begin{array}{c}
            \Pi_1 \\
            {[[I |-c X]]}
          \end{array}
        \end{math}
        \qquad\qquad
        $\Pi_2$:
        \begin{math}
          $$\mprset{flushleft}
          \inferrule* [right={\tiny imprL}] {
            {
              \begin{array}{cc}
                \pi_1 & \pi_2 \\
                {[[D |-l A1]]} & {[[G1; A2; G2; X; G3 |-l B]]}
              \end{array}
            }
          }{[[G1; A1 -> A2; D; G2; X; G3 |-l B]]}
        \end{math}
      \end{center}
      By assumption, $c(\Pi_1),c(\Pi_2)\leq |X|$. By induction on $\Pi_1$
      and $\pi_2$, there is a proof $\Pi'$ for sequent
      $[[G1; A2; G2; I; G3 |-l B]]$ s.t. $c(\Pi') \leq |X|$. Therefore, the
      proof $\Pi$ can be constructed as follows with
      $c(\Pi) = c(\Pi') \leq |X|$.
      \begin{center}
        \scriptsize
        \begin{math}
          $$\mprset{flushleft}
          \inferrule* [right={\tiny imprL}] {
            {
              \begin{array}{cc}
                \pi_1 & \Pi' \\
                {[[D |-l A1]]} & {[[G1; A2; G2; I; G3 |-l B]]}
              \end{array}
            }
          }{[[G1; A1 -> A2; D; G2; I; G3 |-l B]]}
        \end{math}
      \end{center}

\item Case 6:
      \begin{center}
        \scriptsize
        \begin{math}
          \begin{array}{c}
            \Pi_1 \\
            {[[D1 |-l B]]}
          \end{array}
        \end{math}
        \qquad\qquad
        $\Pi_2$:
        \begin{math}
          $$\mprset{flushleft}
          \inferrule* [right={\tiny imprL}] {
            {
              \begin{array}{cc}
                \pi_1 & \pi_2 \\
                {[[D2 |-l A1]]} & {[[G1; A2; G2; B; G3 |-l C]]}
              \end{array}
            }
          }{[[G1; A1 -> A2; D2; G2; B; G3 |-l C]]}
        \end{math}
      \end{center}
      By assumption, $c(\Pi_1),c(\Pi_2)\leq |B|$. By induction on $\Pi_1$
      and $\pi_2$, there is a proof $\Pi'$ for sequent
      $[[G1; A2; G2; D1; G3 |-l C]]$ s.t. $c(\Pi') \leq |B|$. Therefore,
      the proof $\Pi$ can be constructed as follows with
      $c(\Pi) = c(\Pi') \leq |B|$.
      \begin{center}
        \scriptsize
        \begin{math}
          $$\mprset{flushleft}
          \inferrule* [right={\tiny imprL}] {
            {
              \begin{array}{cc}
                \pi_1 & \Pi' \\
                {[[D2 |-l A1]]} & {[[G1; A2; G2; D1; G3 |-l C]]}
              \end{array}
            }
          }{[[G1; A1 -> A2; D2; G2; D1; G3 |-l C]]}
        \end{math}
      \end{center}
\end{itemize}


\subsubsection{Left introduction of the non-commutative right implication $\rto$}
\begin{itemize}
\item Case 1:
      \begin{center}
        \scriptsize
        \begin{math}
          \begin{array}{c}
            \Pi_1 \\
            {[[I |-c X]]}
          \end{array}
        \end{math}
        \qquad\qquad
        $\Pi_2$:
        \begin{math}
          $$\mprset{flushleft}
          \inferrule* [right={\tiny implL}] {
            {
              \begin{array}{cc}
                \pi_1 & \pi_2 \\
                {[[D1; X; D2 |-l A1]]} & {[[G1; A2; G2 |-l B]]}
              \end{array}
            }
          }{[[G1; D1; A2 <- A1; X; D2; G2 |-l B]]}
        \end{math}
      \end{center}
      By assumption, $c(\Pi_1),c(\Pi_2)\leq |X|$. By induction on $\Pi_1$
      and $\pi_1$, there is a proof $\Pi'$ for sequent
      $[[D1; I; D2 |-l A1]]$ s.t. $c(\Pi') \leq |X|$. Therefore, the proof
      $\Pi$ can be constructed as follows with $c(\Pi) = c(\Pi') \leq |X|$.
      \begin{center}
        \scriptsize
        \begin{math}
          $$\mprset{flushleft}
          \inferrule* [right={\tiny implL}] {
            {
              \begin{array}{cc}
                \Pi' & \pi_2 \\
                {[[D1; I; D2 |-l A1]]} & {[[G1; A2; G2 |-l B]]}
              \end{array}
            }
          }{[[G1; D1; A2 <- A1; I; D2; G2 |-l B]]}
        \end{math}
      \end{center}

\item Case 2:
      \begin{center}
        \scriptsize
        \begin{math}
          \begin{array}{c}
            \Pi_1 \\
            {[[G |-l C]]}
          \end{array}
        \end{math}
        \qquad\qquad
        $\Pi_2$:
        \begin{math}
          $$\mprset{flushleft}
          \inferrule* [right={\tiny implL}] {
            {
              \begin{array}{cc}
                \pi_1 & \pi_2 \\
                {[[D1; C; D2 |-l A1]]} & {[[G1; A2; G2 |-l B]]}
              \end{array}
            }
          }{[[G1; D1; C; D2; A2 <- A1; G2 |-l B]]}
        \end{math}
      \end{center}
      By assumption, $c(\Pi_1),c(\Pi_2)\leq |C|$. By induction on $\Pi_1$
      and $\pi_1$, there is a proof $\Pi'$ for sequent
      $[[D1; G; D2 |-l A1]]$ s.t. $c(\Pi') \leq |C|$. Therefore, the proof
      $\Pi$ can be constructed as follows with $c(\Pi) = c(\Pi') \leq |C|$.
      \begin{center}
        \scriptsize
        \begin{math}
          $$\mprset{flushleft}
          \inferrule* [right={\tiny implL}] {
            {
              \begin{array}{cc}
                \Pi' & \pi_2 \\
                {[[D1; G; D2 |-l A1]]} & {[[G1; A2; G2 |-l B]]}
              \end{array}
            }
          }{[[G1; D1; G; D2; A2 <- A1; G2 |-l B]]}
        \end{math}
      \end{center}

\item Case 3:
      \begin{center}
        \scriptsize
        \begin{math}
          \begin{array}{c}
            \Pi_1 \\
            {[[I |-c X]]}
          \end{array}
        \end{math}
        \qquad\qquad
        $\Pi_2$:
        \begin{math}
          $$\mprset{flushleft}
          \inferrule* [right={\tiny implL}] {
            {
              \begin{array}{cc}
                \pi_1 & \pi_2 \\
                {[[D |-l A1]]} & {[[G1; X; G2; A2; G3 |-l B]]}
              \end{array}
            }
          }{[[G1; X; G2; D; A2 <- A1; G3 |-l B]]}
        \end{math}
      \end{center}
      By assumption, $c(\Pi_1),c(\Pi_2)\leq |X|$. By induction on $\Pi_1$
      and $\pi_2$, there is a proof $\Pi'$ for sequent
      $[[G1; I; G2; A2; G3 |-l B]]$ s.t. $c(\Pi') \leq |X|$. Therefore, the
      proof $\Pi$ can be constructed as follows with
      $c(\Pi) = c(\Pi') \leq |X|$.
      \begin{center}
        \scriptsize
        \begin{math}
          $$\mprset{flushleft}
          \inferrule* [right={\tiny implL}] {
            {
              \begin{array}{cc}
                \pi_1 & \Pi' \\
                {[[D |-l A1]]} & {[[G1; I; G2; A2; G3 |-l B]]}
              \end{array}
            }
          }{[[G1; I; G2; D; A2 <- A1; G3 |-l B]]}
        \end{math}
      \end{center}

\item Case 4:
      \begin{center}
        \scriptsize
        \begin{math}
          \begin{array}{c}
            \Pi_1 \\
            {[[D1 |-l B]]}
          \end{array}
        \end{math}
        \qquad\qquad
        $\Pi_2$:
        \begin{math}
          $$\mprset{flushleft}
          \inferrule* [right={\tiny implL}] {
            {
              \begin{array}{cc}
                \pi_1 & \pi_2 \\
                {[[D2 |-l A1]]} & {[[G1; B; G2; A2; G3 |-l C]]}
              \end{array}
            }
          }{[[G1; B; G2; D2; A2 <- A1; G3 |-l C]]}
        \end{math}
      \end{center}
      By assumption, $c(\Pi_1),c(\Pi_2)\leq |B|$. By induction on $\Pi_1$
      and $\pi_2$, there is a proof $\Pi'$ for sequent
      $[[G1; D1; G2; A2; G3 |-l C]]$ s.t. $c(\Pi') \leq |B|$. Therefore,
      the proof $\Pi$ can be constructed as follows with
      $c(\Pi) = c(\Pi') \leq |B|$.
      \begin{center}
        \scriptsize
        \begin{math}
          $$\mprset{flushleft}
          \inferrule* [right={\tiny implL}] {
            {
              \begin{array}{cc}
                \pi_1 & \Pi' \\
                {[[D2 |-l A1]]} & {[[G1; D1; G2; A2; G3 |-l C]]}
              \end{array}
            }
          }{[[G1; D1; G2; D2; A2 <- A1; G3 |-l C]]}
        \end{math}
      \end{center}

\item Case 5:
      \begin{center}
        \scriptsize
        \begin{math}
          \begin{array}{c}
            \Pi_1 \\
            {[[I |-c X]]}
          \end{array}
        \end{math}
        \qquad\qquad
        $\Pi_2$:
        \begin{math}
          $$\mprset{flushleft}
          \inferrule* [right={\tiny implL}] {
            {
              \begin{array}{cc}
                \pi_1 & \pi_2 \\
                {[[D |-l A1]]} & {[[G1; A2; G2; X; G3 |-l B]]}
              \end{array}
            }
          }{[[G1; D; A2 <- A1; D; G2; X; G3 |-l B]]}
        \end{math}
      \end{center}
      By assumption, $c(\Pi_1),c(\Pi_2)\leq |X|$. By induction on $\Pi_1$
      and $\pi_2$, there is a proof $\Pi'$ for sequent
      $[[G1; A2; G2; I; G3 |-l B]]$ s.t. $c(\Pi') \leq |X|$. Therefore, the
      proof $\Pi$ can be constructed as follows with
      $c(\Pi) = c(\Pi') \leq |X|$.
      \begin{center}
        \scriptsize
        \begin{math}
          $$\mprset{flushleft}
          \inferrule* [right={\tiny implL}] {
            {
              \begin{array}{cc}
                \pi_1 & \Pi' \\
                {[[D |-l A1]]} & {[[G1; A2; G2; I; G3 |-l B]]}
              \end{array}
            }
          }{[[G1; D; A2 <- A1; G2; I; G3 |-l B]]}
        \end{math}
      \end{center}

\item Case 6:
    \begin{center}
      \scriptsize
      \begin{math}
        \begin{array}{c}
          \Pi_1 \\
          {[[D1 |-l B]]}
        \end{array}
      \end{math}
      \qquad\qquad
      $\Pi_2$:
      \begin{math}
        $$\mprset{flushleft}
        \inferrule* [right={\tiny implL}] {
          {
            \begin{array}{cc}
              \pi_1 & \pi_2 \\
              {[[D2 |-l A1]]} & {[[G1; A2; G2; B; G3 |-l C]]}
            \end{array}
          }
        }{[[G1; D2; A2 <- A1; G2; B; G3 |-l C]]}
      \end{math}
    \end{center}
    By assumption, $c(\Pi_1),c(\Pi_2)\leq |B|$. By induction on $\Pi_1$ and
    $\pi_2$, there is a proof $\Pi'$ for sequent
    $[[G1; A2; G2; D1; G3 |-l C]]$ s.t. $c(\Pi') \leq |B|$. Therefore, the
    proof $\Pi$ can be constructed as follows with
    $c(\Pi) = c(\Pi') \leq |B|$.
    \begin{center}
      \scriptsize
      \begin{math}
        $$\mprset{flushleft}
        \inferrule* [right={\tiny implL}] {
          {
            \begin{array}{cc}
              \pi_1 & \Pi' \\
              {[[D2 |-l A1]]} & {[[G1; A2; G2; D1; G3 |-l C]]}
            \end{array}
          }
        }{[[G1; D2; A2 <- A1; G2; D1; G3 |-l C]]}
      \end{math}
    \end{center}
\end{itemize}




\subsubsection{Left introduction of the commutative tensor $\otimes$ (with low priority)}
\begin{itemize}
\item Case 1:
      \begin{center}
        \scriptsize
        \begin{math}
          \begin{array}{c}
            \Pi_1 \\
            {[[I |-c X]]}
          \end{array}
        \end{math}
        \qquad\qquad
        $\Pi_2$:
        \begin{math}
          $$\mprset{flushleft}
          \inferrule* [right={\tiny tenL}] {
            {
              \begin{array}{c}
                \pi \\
                {[[P1, X, P2, Y1, Y2, P3 |-c Z]]}
              \end{array}
            }
          }{[[P1, X, P2, Y1 (*) Y2, P3 |-c Z]]}
        \end{math}
      \end{center}
      By assumption, $c(\Pi_1),c(\Pi_2)\leq |X|$. By induction on $\Pi_1$
      and $\pi$, there is a proof $\Pi'$ for sequent
      $[[P1, I, P2, Y1, Y2, P3 |-c Z]]$ s.t. $c(\Pi') \leq |X|$. Therefore,
      the proof $\Pi$ can be constructed as follows with
      $c(\Pi) = c(\Pi') \leq |X|$.
      \begin{center}
        \scriptsize
        \begin{math}
          $$\mprset{flushleft}
          \inferrule* [right={\tiny tenL}] {
            {
              \begin{array}{c}
                \Pi' \\
                {[[P1, I, P2, Y1, Y2, P3 |-c Z]]}
              \end{array}
            }
          }{[[P1, I, P2, Y1 (*) Y2, P3 |-c Z]]}
        \end{math}
      \end{center}

\item Case 2:
      \begin{center}
        \scriptsize
        \begin{math}
          \begin{array}{c}
            \Pi_1 \\
            {[[I |-c X]]}
          \end{array}
        \end{math}
        \qquad\qquad
        $\Pi_2$:
        \begin{math}
          $$\mprset{flushleft}
          \inferrule* [right={\tiny tenL}] {
            {
              \begin{array}{c}
                \pi \\
                {[[P1, Y1, Y2, P2, X, P3 |-c Z]]}
              \end{array}
            }
          }{[[P1, Y1 (*) Y2, P2, X, P3 |-c Z]]}
        \end{math}
      \end{center}
      By assumption, $c(\Pi_1),c(\Pi_2)\leq |X|$. By induction on $\Pi_1$
      and $\pi$, there is a proof $\Pi'$ for sequent
      $[[P1, Y1, Y2, P2, I, P3 |-c Z]]$ s.t. $c(\Pi') \leq |X|$. Therefore,
      the proof $\Pi$ can be constructed as follows with
      $c(\Pi) = c(\Pi') \leq |X|$.
      \begin{center}
        \scriptsize
        \begin{math}
          $$\mprset{flushleft}
          \inferrule* [right={\tiny tenL}] {
            {
              \begin{array}{c}
                \Pi' \\
                {[[P1, Y1, Y2, P2, I, P3 |-c Z]]}
              \end{array}
            }
          }{[[P1, Y1 (*) Y2, P2, I, P3 |-c Z]]}
        \end{math}
      \end{center}

\item Case 3:
      \begin{center}
        \scriptsize
        \begin{math}
          \begin{array}{c}
            \Pi_1 \\
            {[[I |-c X]]}
          \end{array}
        \end{math}
        \qquad\qquad
        $\Pi_2$:
        \begin{math}
          $$\mprset{flushleft}
          \inferrule* [right={\tiny tenL}] {
            {
              \begin{array}{c}
                \pi \\
                {[[G1; X; G2; Y1; Y2; G3 |-l A]]}
              \end{array}
            }
          }{[[G1; X; G2; Y1 (*) Y2; G3 |-l A]]}
        \end{math}
      \end{center}
      By assumption, $c(\Pi_1),c(\Pi_2)\leq |X|$. By induction on $\Pi_1$
      and $\pi$, there is a proof $\Pi'$ for sequent
      $[[G1; I; G2; Y1; Y2; G3 |-l A]]$ s.t. $c(\Pi') \leq |X|$. Therefore,
      the proof $\Pi$ can be constructed as follows with
      $c(\Pi) = c(\Pi') \leq |X|$.
      \begin{center}
        \scriptsize
        \begin{math}
          $$\mprset{flushleft}
          \inferrule* [right={\tiny tenL}] {
            {
              \begin{array}{c}
                \Pi' \\
                {[[G1; I; G2; Y1; Y2; G3 |-l A]]}
              \end{array}
            }
          }{[[G1; I; G2; Y1 (*) Y2; G3 |-l A]]}
        \end{math}
      \end{center}

\item Case 4:
      \begin{center}
        \scriptsize
        \begin{math}
          \begin{array}{c}
            \Pi_1 \\
            {[[D |-l B]]}
          \end{array}
        \end{math}
        \qquad\qquad
        $\Pi_2$:
        \begin{math}
          $$\mprset{flushleft}
          \inferrule* [right={\tiny tenL}] {
            {
              \begin{array}{c}
                \pi \\
                {[[G1; B; G2; Y1; Y2; G3 |-l A]]}
              \end{array}
            }
          }{[[G1; B; G2; Y1 (*) Y2; G3 |-l A]]}
        \end{math}
      \end{center}
      By assumption, $c(\Pi_1),c(\Pi_2)\leq |B|$. By induction on $\Pi_1$
      and $\pi$, there is a proof $\Pi'$ for sequent
      $[[G1; B; G2; Y1; Y2; G3 |-l A]]$ s.t. $c(\Pi') \leq |B|$. Therefore,
      the proof $\Pi$ can be constructed as follows with
      $c(\Pi) = c(\Pi') \leq |B|$.
      \begin{center}
        \scriptsize
        \begin{math}
          $$\mprset{flushleft}
          \inferrule* [right={\tiny tenL}] {
            {
              \begin{array}{c}
                \Pi' \\
                {[[G1; D; G2; Y1; Y2; G3 |-l A]]}
              \end{array}
            }
          }{[[G1; D; G2; Y1 (*) Y2; G3 |-l A]]}
        \end{math}
      \end{center}

\item Case 5:
      \begin{center}
        \scriptsize
        \begin{math}
          \begin{array}{c}
            \Pi_1 \\
            {[[I |-c X]]}
          \end{array}
        \end{math}
        \qquad\qquad
        $\Pi_2$:
        \begin{math}
          $$\mprset{flushleft}
          \inferrule* [right={\tiny tenL}] {
            {
              \begin{array}{c}
                \pi \\
                {[[G1; Y1; Y2; G2; X; G3 |-l A]]}
              \end{array}
            }
          }{[[G1; Y1 (*) Y2; G2; X; G3 |-l A]]}
        \end{math}
      \end{center}
      By assumption, $c(\Pi_1),c(\Pi_2)\leq |X|$. By induction on $\Pi_1$
      and $\pi$, there is a proof $\Pi'$ for sequent
      $[[G1; Y1; Y2; G2; I; G3 |-l A]]$ s.t. $c(\Pi') \leq |X|$. Therefore,
      the proof $\Pi$ can be constructed as follows with
      $c(\Pi) = c(\Pi') \leq |X|$.
      \begin{center}
        \scriptsize
        \begin{math}
          $$\mprset{flushleft}
          \inferrule* [right={\tiny tenL}] {
            {
              \begin{array}{c}
                \Pi' \\
                {[[G1; Y1; Y2; G2; I; G3 |-l A]]}
              \end{array}
            }
          }{[[G1; Y1 (*) Y2; G2; I; G3 |-l A]]}
        \end{math}
      \end{center}

\item Case 6:
      \begin{center}
        \scriptsize
        \begin{math}
          \begin{array}{c}
            \Pi_1 \\
            {[[D |-l B]]}
          \end{array}
        \end{math}
        \qquad\qquad
        $\Pi_2$:
        \begin{math}
          $$\mprset{flushleft}
          \inferrule* [right={\tiny tenL}] {
            {
              \begin{array}{c}
                \pi \\
                {[[G1; Y1; Y2; G2; B; G3 |-l A]]}
              \end{array}
            }
          }{[[G1; Y1 (*) Y2; G2; B; G3 |-l A]]}
        \end{math}
      \end{center}
      By assumption, $c(\Pi_1),c(\Pi_2)\leq |B|$. By induction on $\Pi_1$
      and $\pi$, there is a proof $\Pi'$ for sequent
      $[[G1; Y1; Y2; G2; D; G3 |-l A]]$ s.t. $c(\Pi') \leq |B|$. Therefore,
      the proof $\Pi$ can be constructed as follows with
      $c(\Pi) = c(\Pi') \leq |B|$.
      \begin{center}
        \scriptsize
        \begin{math}
          $$\mprset{flushleft}
          \inferrule* [right={\tiny tenL}] {
            {
              \begin{array}{c}
                \Pi' \\
                {[[G1; Y1; Y2; G2; D; G3 |-l A]]}
              \end{array}
            }
          }{[[G1; Y1 (*) Y2; G2; D; G3 |-l A]]}
        \end{math}
      \end{center}
\end{itemize}


\subsubsection{Left introduction of the non-commutative tensor $\tri$ (with low priority)}:
\begin{itemize}
\item Case 1:
      \begin{center}
        \scriptsize
        \begin{math}
          \begin{array}{c}
            \Pi_1 \\
            {[[I |-c X]]}
          \end{array}
        \end{math}
        \qquad\qquad
        $\Pi_2$:
        \begin{math}
          $$\mprset{flushleft}
          \inferrule* [right={\tiny tenL}] {
            {
              \begin{array}{c}
                \pi \\
                {[[G1; X; G2; A1; A2; G3 |-l B]]}
              \end{array}
            }
          }{[[G1; X; G2; A1 (>) A2; G3 |-l B]]}
        \end{math}
      \end{center}
      By assumption, $c(\Pi_1),c(\Pi_2)\leq |X|$. By induction on $\Pi_1$
      and $\pi$, there is a proof $\Pi'$ for sequent
      $[[G1; I; G2; A1; A2; G3 |-l B]]$ s.t. $c(\Pi') \leq |X|$. Therefore,
      the proof $\Pi$ can be constructed as follows with
      $c(\Pi) = c(\Pi') \leq |X|$.
      \begin{center}
        \scriptsize
        \begin{math}
          $$\mprset{flushleft}
          \inferrule* [right={\tiny tenL}] {
            {
              \begin{array}{c}
                \Pi' \\
                {[[G1; I; G2; A1; A2; G3 |-l B]]}
              \end{array}
            }
          }{[[G1; I; G2; A1 (>) A2; G3 |-l B]]}
        \end{math}
      \end{center}

\item Case 2:
      \begin{center}
        \scriptsize
        \begin{math}
          \begin{array}{c}
            \Pi_1 \\
            {[[D |-l B]]}
          \end{array}
        \end{math}
        \qquad\qquad
        $\Pi_2$:
        \begin{math}
          $$\mprset{flushleft}
          \inferrule* [right={\tiny tenL}] {
            {
              \begin{array}{c}
                \pi \\
                {[[G1; B; G2; A1; A2; G3 |-l C]]}
              \end{array}
            }
          }{[[G1; B; G2; A1 (>) A2; G3 |-l C]]}
        \end{math}
      \end{center}
      By assumption, $c(\Pi_1),c(\Pi_2)\leq |B|$. By induction on $\Pi_1$
      and $\pi$, there is a proof $\Pi'$ for sequent
      $[[G1; D; G2; A1; A2; G3 |-l C]]$ s.t. $c(\Pi') \leq |B|$. Therefore,
      the proof $\Pi$ can be constructed as follows with
      $c(\Pi) = c(\Pi') \leq |B|$.
      \begin{center}
        \scriptsize
        \begin{math}
          $$\mprset{flushleft}
          \inferrule* [right={\tiny tenL}] {
            {
              \begin{array}{c}
                \Pi' \\
                {[[G1; D; G2; A1; A2; G3 |-l C]]}
              \end{array}
            }
          }{[[G1; D; G2; A1 (>) A2; G3 |-l C]]}
        \end{math}
      \end{center}

\item Case 3:
      \begin{center}
        \scriptsize
        \begin{math}
          \begin{array}{c}
            \Pi_1 \\
            {[[I |-c X]]}
          \end{array}
        \end{math}
        \qquad\qquad
        $\Pi_2$:
        \begin{math}
          $$\mprset{flushleft}
          \inferrule* [right={\tiny tenL}] {
            {
              \begin{array}{c}
                \pi \\
                {[[G1; A1; A2; G2; X; G3 |-l B]]}
              \end{array}
            }
          }{[[G1; A1 (>) A2; G2; X; G3 |-l B]]}
        \end{math}
      \end{center}
      By assumption, $c(\Pi_1),c(\Pi_2)\leq |X|$. By induction on $\Pi_1$
      and $\pi$, there is a proof $\Pi'$ for sequent
      $[[G1; A1; A2; G2; I; G3 |-l A]]$ s.t. $c(\Pi') \leq |X|$. Therefore,
      the proof $\Pi$ can be constructed as follows with
      $c(\Pi) = c(\Pi') \leq |X|$.
      \begin{center}
        \scriptsize
        \begin{math}
          $$\mprset{flushleft}
          \inferrule* [right={\tiny tenL}] {
            {
              \begin{array}{c}
                \Pi' \\
                {[[G1; A1; A2; G2; I; G3 |-l B]]}
              \end{array}
            }
          }{[[G1; A1 (>) A2; G2; I; G3 |-l B]]}
        \end{math}
      \end{center}

\item Case 4:
      \begin{center}
        \scriptsize
        \begin{math}
          \begin{array}{c}
            \Pi_1 \\
            {[[D |-l B]]}
          \end{array}
        \end{math}
        \qquad\qquad
        $\Pi_2$:
        \begin{math}
          $$\mprset{flushleft}
          \inferrule* [right={\tiny tenL}] {
            {
              \begin{array}{c}
                \pi \\
                {[[G1; A1; A2; G2; B; G3 |-l C]]}
              \end{array}
            }
          }{[[G1; A1 (>) A2; G2; B; G3 |-l C]]}
        \end{math}
      \end{center}
      By assumption, $c(\Pi_1),c(\Pi_2)\leq |B|$. By induction on $\Pi_1$
      and $\pi$, there is a proof $\Pi'$ for sequent
      $[[G1; A1; A2; G2; D; G3 |-l C]]$ s.t. $c(\Pi') \leq |B|$. Therefore,
      the proof $\Pi$ can be constructed as follows with
      $c(\Pi) = c(\Pi') \leq |B|$.
      \begin{center}
        \scriptsize
        \begin{math}
          $$\mprset{flushleft}
          \inferrule* [right={\tiny tenL}] {
            {
              \begin{array}{c}
                \Pi' \\
                {[[G1; A1; A2; G2; D; G3 |-l C]]}
              \end{array}
            }
          }{[[G1; A1 (>) A2; G2; D; G3 |-l C]]}
        \end{math}
      \end{center}
\end{itemize}



\subsubsection{$\SCdruleTXXexName$}
\begin{itemize}
\item Case 1:
      \begin{center}
        \scriptsize
        \begin{math}
          \begin{array}{c}
            \Pi_1 \\
            {[[I |-c X]]}
          \end{array}
        \end{math}
        \qquad\qquad
        $\Pi_2$:
        \begin{math}
          $$\mprset{flushleft}
          \inferrule* [right={\tiny beta}] {
            {
              \begin{array}{c}
                \pi \\
                {[[P1, X, P2, Y1, Y2, P3 |-c Z]]}
              \end{array}
            }
          }{[[P1, X, P2, Y2, Y1, P3 |-c Z]]}
        \end{math}
      \end{center}
      By assumption, $c(\Pi_1),c(\Pi_2)\leq |X|$. By induction on $\Pi_1$
      and $\pi$, there is a proof $\Pi'$ for sequent
      $[[P1, I, P2, Y1, Y2, P3 |-c Z]]$ s.t. $c(\Pi') \leq |X|$. Therefore,
      the proof $\Pi$ can be constructed as follows with
      $c(\Pi) = c(\Pi') \leq |X|$.
      \begin{center}
        \scriptsize
        \begin{math}
          $$\mprset{flushleft}
          \inferrule* [right={\tiny cut}] {
            {
              \begin{array}{cc}
                \Pi' \\
                {[[P1, I, P2, Y1, Y2, P3 |-c Z]]}
              \end{array}
            }
          }{[[P1, I, P2, Y2, Y1, P3 |-c Z]]}
        \end{math}
      \end{center}

\item Case 2:
      \begin{center}
        \scriptsize
        \begin{math}
          \begin{array}{c}
            \Pi_1 \\
            {[[I |-c X]]}
          \end{array}
        \end{math}
        \qquad\qquad
        $\Pi_2$:
        \begin{math}
          $$\mprset{flushleft}
          \inferrule* [right={\tiny beta}] {
            {
              \begin{array}{c}
                \pi \\
                {[[P1, Y1, Y2, P2, X, P3 |-c Z]]}
              \end{array}
            }
          }{[[P1, X, P2, Y2, Y1, P3 |-c Z]]}
        \end{math}
      \end{center}
      By assumption, $c(\Pi_1),c(\Pi_2)\leq |X|$. By induction on $\Pi_1$
      and $\pi$, there is a proof $\Pi'$ for sequent
      $[[P1, Y1, Y2, P2, I, P3 |-c Z]]$ s.t. $c(\Pi') \leq |X|$. Therefore,
      the proof $\Pi$ can be constructed as follows with
      $c(\Pi) = c(\Pi') \leq |X|$.
      \begin{center}
        \scriptsize
        \begin{math}
          $$\mprset{flushleft}
          \inferrule* [right={\tiny cut}] {
            {
              \begin{array}{cc}
                \Pi' \\
                {[[P1, Y1, Y2, P2, I, P3 |-c Z]]}
              \end{array}
            }
          }{[[P1, Y2, Y1, P2, I, P3 |-c Z]]}
        \end{math}
      \end{center}
\end{itemize}


\subsubsection{$\SCdruleSXXexName$}
\begin{itemize}
\item Case 1:
      \begin{center}
        \scriptsize
        \begin{math}
          \begin{array}{c}
            \Pi_1 \\
            {[[I |-c X]]}
          \end{array}
        \end{math}
        \qquad\qquad
        $\Pi_2$:
        \begin{math}
          $$\mprset{flushleft}
          \inferrule* [right={\tiny beta}] {
            {
              \begin{array}{c}
                \pi \\
                {[[G1; X; G2; Y1; Y2; G3 |-l A]]}
              \end{array}
            }
          }{[[G1; X; G2; Y2; Y1; G3 |-l A]]}
        \end{math}
      \end{center}
      By assumption, $c(\Pi_1),c(\Pi_2)\leq |X|$. By induction on $\Pi_1$
      and $\pi$, there is a proof $\Pi'$ for sequent
      $[[G1; I; G2; Y1; Y2; G3 |-l A]]$ s.t. $c(\Pi') \leq |X|$. Therefore,
      the proof $\Pi$ can be constructed as follows with
      $c(\Pi) = c(\Pi') \leq |X|$.
      \begin{center}
        \scriptsize
        \begin{math}
          $$\mprset{flushleft}
          \inferrule* [right={\tiny cut}] {
            {
              \begin{array}{cc}
                \Pi' \\
                {[[G1; I; G2; Y1; Y2; G3 |-l A]]}
              \end{array}
            }
          }{[[G1; I; G2; Y2; Y1; G3 |-l A]]}
        \end{math}
      \end{center}

\item Case 2:
      \begin{center}
        \scriptsize
        \begin{math}
          \begin{array}{c}
            \Pi_1 \\
            {[[D |-l B]]}
          \end{array}
        \end{math}
        \qquad\qquad
        $\Pi_2$:
        \begin{math}
          $$\mprset{flushleft}
          \inferrule* [right={\tiny beta}] {
            {
              \begin{array}{c}
                \pi \\
                {[[G1; B; G2; Y1; Y2; G3 |-l A]]}
              \end{array}
            }
          }{[[G1; B; G2; Y2; Y1; G3 |-l A]]}
        \end{math}
      \end{center}
      By assumption, $c(\Pi_1),c(\Pi_2)\leq |X|$. By induction on $\Pi_1$
      and $\pi$, there is a proof $\Pi'$ for sequent
      $[[G1; D; G2; Y1; Y2; G3 |-l A]]$ s.t. $c(\Pi') \leq |X|$. Therefore,
      the proof $\Pi$ can be constructed as follows with
      $c(\Pi) = c(\Pi') \leq |X|$.
      \begin{center}
        \scriptsize
        \begin{math}
          $$\mprset{flushleft}
          \inferrule* [right={\tiny cut}] {
            {
              \begin{array}{cc}
                \Pi' \\
                {[[G1; D; G2; Y1; Y2; G3 |-l A]]}
              \end{array}
            }
          }{[[G1; D; G2; Y2; Y1; G3 |-l A]]}
        \end{math}
      \end{center}

\item Case 3:
      \begin{center}
        \scriptsize
        \begin{math}
          \begin{array}{c}
            \Pi_1 \\
            {[[I |-c X]]}
          \end{array}
        \end{math}
        \qquad\qquad
        $\Pi_2$:
        \begin{math}
          $$\mprset{flushleft}
          \inferrule* [right={\tiny beta}] {
            {
              \begin{array}{c}
                \pi \\
                {[[G1; Y1; Y2; G2; X; G3 |-l A]]}
              \end{array}
            }
          }{[[G1; X; G2; Y2; Y1; G3 |-l A]]}
        \end{math}
      \end{center}
      By assumption, $c(\Pi_1),c(\Pi_2)\leq |X|$. By induction on $\Pi_1$
      and $\pi$, there is a proof $\Pi'$ for sequent
      $[[G1; Y1; Y2; G2; I; G3 |-l A]]$ s.t. $c(\Pi') \leq |X|$. Therefore,
      the proof $\Pi$ can be constructed as follows with
      $c(\Pi) = c(\Pi') \leq |X|$.
      \begin{center}
        \scriptsize
        \begin{math}
          $$\mprset{flushleft}
          \inferrule* [right={\tiny cut}] {
            {
              \begin{array}{cc}
                \Pi' \\
                {[[G1; Y1; Y2; G2; I; G3 |-l A]]}
              \end{array}
            }
          }{[[G1; Y2; Y1; G2; I; G3 |-l A]]}
        \end{math}
      \end{center}

\item Case 4:
      \begin{center}
        \scriptsize
        \begin{math}
          \begin{array}{c}
            \Pi_1 \\
            {[[D |-l B]]}
          \end{array}
        \end{math}
        \qquad\qquad
        $\Pi_2$:
        \begin{math}
          $$\mprset{flushleft}
          \inferrule* [right={\tiny beta}] {
            {
              \begin{array}{c}
                \pi \\
                {[[G1; Y1; Y2; G2; B; G3 |-l A]]}
              \end{array}
            }
          }{[[G1; Y2; Y1; G2; B; G3 |-l A]]}
        \end{math}
      \end{center}
      By assumption, $c(\Pi_1),c(\Pi_2)\leq |X|$. By induction on $\Pi_1$
      and $\pi$, there is a proof $\Pi'$ for sequent
      $[[G1; Y1; Y2; G2; D; G3 |-l A]]$ s.t. $c(\Pi') \leq |X|$. Therefore,
      the proof $\Pi$ can be constructed as follows with
      $c(\Pi) = c(\Pi') \leq |X|$.
      \begin{center}
        \scriptsize
        \begin{math}
          $$\mprset{flushleft}
          \inferrule* [right={\tiny cut}] {
            {
              \begin{array}{cc}
                \Pi' \\
                {[[G1; Y1; Y2; G2; D; G3 |-l A]]}
              \end{array}
            }
          }{[[G1; Y2; Y1; G2; D; G3 |-l A]]}
        \end{math}
      \end{center}
\end{itemize}



\subsubsection{Left introduction of the commutative unit $[[UnitT]]$ (with low priority)}
\begin{itemize}
\item Case 1:
      \begin{center}
        \scriptsize
        \begin{math}
          \begin{array}{c}
            \Pi_1 \\
            {[[P |-c X]]}
          \end{array}
        \end{math}
        \qquad\qquad
        $\Pi_2$:
        \begin{math}
          $$\mprset{flushleft}
          \inferrule* [right={\tiny unitL}] {
            {
              \begin{array}{c}
                \pi \\
                {[[I1, I2, X, I3 |-c Y]]}
              \end{array}
            }
          }{[[I1, UnitT, I2, X, I3 |-c Y]]}
        \end{math}
      \end{center}
      By assumption, $c(\Pi_1),c(\Pi_2)\leq |X|$. By induction on $\Pi_1$
      and $\pi$, there is a proof $\Pi'$ for sequent
      $[[I1, I2, P, I3 |-c Y]]$
      s.t. $c(\Pi') \leq |X|$. Therefore, the proof $\Pi$ can be
      constructed as follows with $c(\Pi) = c(\Pi') \leq |X|$.
      \begin{center}
        \scriptsize
        \begin{math}
          $$\mprset{flushleft}
          \inferrule* [right={\tiny unitL}] {
            {
              \begin{array}{c}
                \Pi' \\
                {[[I1, I2, P, I3 |-c Y]]}
              \end{array}
            }
          }{[[I1, UnitT, I2, P, I3 |-c Y]]}
        \end{math}
      \end{center}

\item Case 2:
      \begin{center}
        \scriptsize
        \begin{math}
          \begin{array}{c}
            \Pi_1 \\
            {[[I |-c X]]}
          \end{array}
        \end{math}
        \qquad\qquad
        $\Pi_2$:
        \begin{math}
          $$\mprset{flushleft}
          \inferrule* [right={\tiny unitL1}] {
            {
              \begin{array}{c}
                \pi \\
                {[[G1; G2; X; G3 |-l A]]}
              \end{array}
            }
          }{[[G1; UnitT; G2; X; G3 |-l A]]}
        \end{math}
      \end{center}
      By assumption, $c(\Pi_1),c(\Pi_2)\leq |X|$. By induction on $\Pi_1$
      and $\pi$, there is a proof $\Pi'$ for sequent
      $[[G1; G2; I; G2 |-l A]]$
      s.t. $c(\Pi') \leq |X|$. Therefore, the proof $\Pi$ can be
      constructed as follows with $c(\Pi) = c(\Pi') \leq |X|$.
      \begin{center}
        \scriptsize
        \begin{math}
          $$\mprset{flushleft}
          \inferrule* [right={\tiny unitL1}] {
            {
              \begin{array}{c}
                \Pi' \\
                {[[G1; G2; I; G3 |-l A]]}
              \end{array}
            }
          }{[[G1; UnitT; G2; I; G3 |-l A]]}
        \end{math}
      \end{center}

\item Case 3:
      \begin{center}
        \scriptsize
        \begin{math}
          \begin{array}{c}
            \Pi_1 \\
            {[[D |-l B]]}
          \end{array}
        \end{math}
        \qquad\qquad
        $\Pi_2$:
        \begin{math}
          $$\mprset{flushleft}
          \inferrule* [right={\tiny unitL1}] {
            {
              \begin{array}{c}
                \pi \\
                {[[G1; G2; B; G3 |-l A]]}
              \end{array}
            }
          }{[[G1; UnitT; G2; B; G3 |-l A]]}
        \end{math}
      \end{center}
      By assumption, $c(\Pi_1),c(\Pi_2)\leq |B|$. By induction on $\Pi_1$
      and $\pi$, there is a proof $\Pi'$ for sequent
      $[[G1; G2; D; G3 |-l A]]$
      s.t. $c(\Pi') \leq |B|$. Therefore, the proof $\Pi$ can be
      constructed as follows with $c(\Pi) = c(\Pi') \leq |B|$.
      \begin{center}
        \scriptsize
        \begin{math}
          $$\mprset{flushleft}
          \inferrule* [right={\tiny unitL1}] {
            {
              \begin{array}{c}
                \Pi' \\
                {[[G1; G2; D; G3 |-l A]]}
              \end{array}
            }
          }{[[G1; UnitT; G2; D; G3 |-l A]]}
        \end{math}
      \end{center}
\end{itemize}

\subsubsection{Left introduction of the non-commutative unit $[[UnitS]]$ (with low priority)}
\begin{itemize}
\item Case 1:
      \begin{center}
        \scriptsize
        \begin{math}
          \begin{array}{c}
            \Pi_1 \\
            {[[I |-c X]]}
          \end{array}
        \end{math}
        \qquad\qquad
        $\Pi_2$:
        \begin{math}
          $$\mprset{flushleft}
          \inferrule* [right={\tiny unitL2}] {
            {
              \begin{array}{c}
                \pi \\
                {[[G1; G2; X; G3 |-l A]]}
              \end{array}
            }
          }{[[G1; UnitS; G2; X; G3 |-l A]]}
        \end{math}
      \end{center}
      By assumption, $c(\Pi_1),c(\Pi_2)\leq |X|$. By induction on $\Pi_1$
      and $\pi$, there is a proof $\Pi'$ for sequent
      $[[G1; G2; I; G3 |-l A]]$
      s.t. $c(\Pi') \leq |X|$. Therefore, the proof $\Pi$ can be
      constructed as follows with $c(\Pi) = c(\Pi') \leq |X|$.
      \begin{center}
        \scriptsize
        \begin{math}
          $$\mprset{flushleft}
          \inferrule* [right={\tiny unitL2}] {
            {
              \begin{array}{c}
                \Pi' \\
                {[[G1; G2; I; G3 |-l A]]}
              \end{array}
            }
          }{[[G1; UnitS; G2; I; G3 |-l A]]}
        \end{math}
      \end{center}

\item Case 2:
      \begin{center}
        \scriptsize
        \begin{math}
          \begin{array}{c}
            \Pi_1 \\
            {[[D |-l B]]}
          \end{array}
        \end{math}
        \qquad\qquad
        $\Pi_2$:
        \begin{math}
          $$\mprset{flushleft}
          \inferrule* [right={\tiny unitL2}] {
            {
              \begin{array}{c}
                \pi \\
                {[[G1; G2; B; G3 |-l A]]}
              \end{array}
            }
          }{[[G1; UnitS; G2; B; G3 |-l A]]}
        \end{math}
      \end{center}
      By assumption, $c(\Pi_1),c(\Pi_2)\leq |B|$. By induction on $\Pi_1$
      and $\pi$, there is a proof $\Pi'$ for sequent
      $[[G1; G2; D; G3 |-l A]]$
      s.t. $c(\Pi') \leq |B|$. Therefore, the proof $\Pi$ can be
      constructed as follows with $c(\Pi) = c(\Pi') \leq |B|$.
      \begin{center}
        \scriptsize
        \begin{math}
          $$\mprset{flushleft}
          \inferrule* [right={\tiny unitL2}] {
            {
              \begin{array}{c}
                \Pi' \\
                {[[G1; G2; D; G3 |-l A]]}
              \end{array}
            }
          }{[[G1; UnitS; G2; D; G3 |-l A]]}
        \end{math}
      \end{center}
\end{itemize}



\subsubsection{Right introduction of the commutative implication $\multimap$ (with low priority)}
\begin{center}
  \scriptsize
  \begin{math}
    \begin{array}{c}
      \Pi_1 \\
      {[[I |-c X]]}
    \end{array}
  \end{math}
  \qquad\qquad
  $\Pi_2$:
  \begin{math}
    $$\mprset{flushleft}
    \inferrule* [right={\tiny impR}] {
      {
        \begin{array}{c}
          \pi \\
          {[[P1, X, P2, Y1 |-c Y2]]}
        \end{array}
      }
    }{[[P1, X, P2 |-c Y1 -o Y2]]}
  \end{math}
\end{center}
By assumption, $c(\Pi_1),c(\Pi_2)\leq |X|$. By induction on $\Pi_1$
and $\pi$, there is a proof $\Pi'$ for sequent \\
$[[P1, I, P2, Y1 |-c Y2]]$ s.t. $c(\Pi') \leq |X|$. Therefore, the
proof $\Pi$ can be constructed as follows with \\
$c(\Pi) = c(\Pi') \leq |X|$.
\begin{center}
  \scriptsize
  \begin{math}
    $$\mprset{flushleft}
    \inferrule* [right={\tiny impR}] {
      {
        \begin{array}{c}
          \Pi' \\
          {[[P1, I, P2, Y1 |-c Y2]]}
        \end{array}
      }
    }{[[P1, I, P2 |-c Y1 -o Y2]]}
  \end{math}
\end{center}



\subsubsection{Right introduction of the non-commutative left implication $\lto$ (with low priority)}
\begin{itemize}
\item Case 1:
      \begin{center}
        \scriptsize
        \begin{math}
          \begin{array}{c}
            \Pi_1 \\
            {[[I |-c X]]}
          \end{array}
        \end{math}
        \qquad\qquad
        $\Pi_2$:
        \begin{math}
          $$\mprset{flushleft}
          \inferrule* [right={\tiny impR}] {
            {
              \begin{array}{c}
                \pi \\
                {[[G1; X; G2; A |-l B]]}
              \end{array}
            }
          }{[[G1; X; G2 |-l A -> B]]}
        \end{math}
      \end{center}
      By assumption, $c(\Pi_1),c(\Pi_2)\leq |X|$. By induction on $\Pi_1$
      and $\pi$, there is a proof $\Pi'$ for sequent \\
      $[[G1; I; G2; A |-l B]]$ s.t. $c(\Pi') \leq |X|$. Therefore, the
      proof $\Pi$ can be constructed as follows with \\
      $c(\Pi) = c(\Pi') \leq |X|$.
      \begin{center}
        \scriptsize
        \begin{math}
          $$\mprset{flushleft}
          \inferrule* [right={\tiny implR}] {
            {
              \begin{array}{c}
                \Pi' \\
                {[[G1; I; G2; A |-l B]]}
              \end{array}
            }
          }{[[G1; I; G2 |-l A -> B]]}
        \end{math}
      \end{center}

\item Case 2:
      \begin{center}
        \scriptsize
        \begin{math}
          \begin{array}{c}
            \Pi_1 \\
            {[[D |-l C]]}
          \end{array}
        \end{math}
        \qquad\qquad
        $\Pi_2$:
        \begin{math}
          $$\mprset{flushleft}
          \inferrule* [right={\tiny impR}] {
            {
              \begin{array}{c}
                \pi \\
                {[[G1; C; G2; A |-l B]]}
              \end{array}
            }
          }{[[G1; C; G2 |-l A -> B]]}
        \end{math}
      \end{center}
      By assumption, $c(\Pi_1),c(\Pi_2)\leq |C|$. By induction on $\Pi_1$
      and $\pi$, there is a proof $\Pi'$ for sequent
      $[[G1; D; G2; A |-l B]]$ s.t. $c(\Pi') \leq |C|$. Therefore, the
      proof $\Pi$ can be constructed as follows with \\
      $c(\Pi) = c(\Pi') \leq |C|$.
      \begin{center}
        \scriptsize
        \begin{math}
          $$\mprset{flushleft}
          \inferrule* [right={\tiny implR}] {
            {
              \begin{array}{c}
                \Pi' \\
                {[[G1; D; G2; A |-l B]]}
              \end{array}
            }
          }{[[G1; D; G2 |-l A -> B]]}
        \end{math}
      \end{center}
\end{itemize}




\subsubsection{Right introduction of the non-commutative right implication $\rto$ (with low priority)}
\begin{itemize}
\item Case 1:
      \begin{center}
        \scriptsize
        \begin{math}
          \begin{array}{c}
            \Pi_1 \\
            {[[I |-c X]]}
          \end{array}
        \end{math}
        \qquad\qquad
        $\Pi_2$:
        \begin{math}
          $$\mprset{flushleft}
          \inferrule* [right={\tiny impL}] {
            {
              \begin{array}{c}
                \pi \\
                {[[A; G1; X; G2|-l B]]}
              \end{array}
            }
          }{[[G1; X; G2 |-l B <- A]]}
        \end{math}
      \end{center}
      By assumption, $c(\Pi_1),c(\Pi_2)\leq |X|$. By induction on $\Pi_1$
      and $\pi$, there is a proof $\Pi'$ for sequent
      $[[A; G1; I; G2 |-l B]]$ s.t. $c(\Pi') \leq |X|$. Therefore, the
      proof $\Pi$ can be constructed as follows with \\
      $c(\Pi) = c(\Pi') \leq |X|$.
      \begin{center}
        \scriptsize
        \begin{math}
          $$\mprset{flushleft}
          \inferrule* [right={\tiny impR}] {
            {
              \begin{array}{c}
                \Pi' \\
                {[[A; G1; I; G2 |-l B]]}
              \end{array}
            }
          }{[[G1; I; G2 |-l B <- A]]}
        \end{math}
      \end{center}

\item Case 2:
      \begin{center}
        \scriptsize
        \begin{math}
          \begin{array}{c}
            \Pi_1 \\
            {[[D |-l C]]}
          \end{array}
        \end{math}
        \qquad\qquad
        $\Pi_2$:
        \begin{math}
          $$\mprset{flushleft}
          \inferrule* [right={\tiny impR}] {
            {
              \begin{array}{c}
                \pi \\
                {[[A; G1; C; G2 |-l B]]}
              \end{array}
            }
          }{[[G1; C; G2 |-l B <- A]]}
        \end{math}
      \end{center}
      By assumption, $c(\Pi_1),c(\Pi_2)\leq |C|$. By induction on $\Pi_1$
      and $\pi$, there is a proof $\Pi'$ for sequent
      $[[G1; D; G2; A |-l B]]$ s.t. $c(\Pi') \leq |C|$. Therefore, the
      proof $\Pi$ can be constructed as follows with \\
      $c(\Pi) = c(\Pi') \leq |C|$.
      \begin{center}
        \scriptsize
        \begin{math}
          $$\mprset{flushleft}
          \inferrule* [right={\tiny impR}] {
            {
              \begin{array}{c}
                \Pi' \\
                {[[A; G1; D; G2 |-l B]]}
              \end{array}
            }
          }{[[G1; D; G2 |-l B <- A]]}
        \end{math}
      \end{center}
\end{itemize}



\subsubsection{Right introduction of the functor $F$}
\begin{center}
  \scriptsize
  \begin{math}
    \begin{array}{c}
      \Pi_1 \\
      {[[I |-c X]]}
    \end{array}
  \end{math}
  \qquad\qquad
  $\Pi_2$:
  \begin{math}
    $$\mprset{flushleft}
    \inferrule* [right={\tiny Fr}] {
      {
        \begin{array}{c}
          \pi \\
          {[[P1, X, P2 |-c Y]]}
        \end{array}
      }
    }{[[P1, X, P2 |-l F Y]]}
  \end{math}
\end{center}
By assumption, $c(\Pi_1),c(\Pi_2)\leq |X|$. By induction on $\Pi_1$
and $\pi$, there is a proof $\Pi'$ for sequent \\
$[[P1, I, P2 |-c Y]]$ s.t. $c(\Pi') \leq |X|$. Therefore, the proof $\Pi$
can be constructed as follows with \\
$c(\Pi) = c(\Pi') \leq |X|$.
\begin{center}
  \scriptsize
  \begin{math}
    $$\mprset{flushleft}
    \inferrule* [right={\tiny Fr}] {
      {
        \begin{array}{c}
          \Pi' \\
          {[[P1, I, P2 |-c Y]]}
        \end{array}
      }
    }{[[P1, I, P2 |-l F Y]]}
  \end{math}
\end{center}



\subsubsection{Left introduction of the functor $F$ (with low priority)}
\begin{itemize}
\item Case 1:
      \begin{center}
        \scriptsize
        \begin{math}
          \begin{array}{c}
            \Pi_1 \\
            {[[I |-c X]]}
          \end{array}
        \end{math}
        \qquad\qquad
        $\Pi_2$:
        \begin{math}
          $$\mprset{flushleft}
          \inferrule* [right={\tiny Fl}] {
            {
              \begin{array}{c}
                \pi \\
                {[[G1; X; G2; Y; G3 |-l A]]}
              \end{array}
            }
          }{[[G1; X; G2; F Y; G3 |-l A]]}
        \end{math}
      \end{center}
      By assumption, $c(\Pi_1),c(\Pi_2)\leq |X|$. By induction on $\Pi_1$
      and $\pi$, there is a proof $\Pi'$ for sequent
      $[[G1; I; G2; Y; G3 |-l A]]$ s.t. $c(\Pi') \leq |X|$. Therefore, the
      proof $\Pi$ can be constructed as follows with
      $c(\Pi) = c(\Pi') \leq |X|$.
      \begin{center}
        \scriptsize
        \begin{math}
          $$\mprset{flushleft}
          \inferrule* [right={\tiny Fl}] {
            {
              \begin{array}{c}
                \Pi' \\
                {[[G1; I; G2; Y; G3 |-l A]]}
              \end{array}
            }
          }{[[G1; I; G2; F Y; G3 |-l A]]}
        \end{math}
      \end{center}

\item Case 2:
      \begin{center}
        \scriptsize
        \begin{math}
          \begin{array}{c}
            \Pi_1 \\
            {[[D |-l B]]}
          \end{array}
        \end{math}
        \qquad\qquad
        $\Pi_2$:
        \begin{math}
          $$\mprset{flushleft}
          \inferrule* [right={\tiny Fl}] {
            {
              \begin{array}{c}
                \pi \\
                {[[G1; B; G2; Y; G3 |-l A]]}
              \end{array}
            }
          }{[[G1; B; G2; F Y; G3 |-l A]]}
        \end{math}
      \end{center}
      By assumption, $c(\Pi_1),c(\Pi_2)\leq |B|$. By induction on $\Pi_1$
      and $\pi$, there is a proof $\Pi'$ for sequent
      $[[G1; D; G2; Y; G3 |-l A]]$ s.t. $c(\Pi') \leq |B|$. Therefore, the
      proof $\Pi$ can be constructed as follows with
      $c(\Pi) = c(\Pi') \leq |B|$.
      \begin{center}
        \scriptsize
        \begin{math}
          $$\mprset{flushleft}
          \inferrule* [right={\tiny Fl}] {
            {
              \begin{array}{c}
                \Pi' \\
                {[[G1; D; G2; Y; G3 |-l A]]}
              \end{array}
            }
          }{[[G1; D; G2; F Y; G3 |-l A]]}
        \end{math}
      \end{center}

\item Case 3:
      \begin{center}
        \scriptsize
        \begin{math}
          \begin{array}{c}
            \Pi_1 \\
            {[[I |-c X]]}
          \end{array}
        \end{math}
        \qquad\qquad
        $\Pi_2$:
        \begin{math}
          $$\mprset{flushleft}
          \inferrule* [right={\tiny Fl}] {
            {
              \begin{array}{c}
                \pi \\
                {[[G1; Y; G2; X; G3 |-l A]]}
              \end{array}
            }
          }{[[G1; F Y; G2; X; G3 |-l A]]}
        \end{math}
      \end{center}
      By assumption, $c(\Pi_1),c(\Pi_2)\leq |X|$. By induction on $\Pi_1$
      and $\pi$, there is a proof $\Pi'$ for sequent
      $[[G1; Y; G2; I; G3 |-l A]]$ s.t. $c(\Pi') \leq |X|$. Therefore, the
      proof $\Pi$ can be constructed as follows with
      $c(\Pi) = c(\Pi') \leq |X|$.
      \begin{center}
        \scriptsize
        \begin{math}
          $$\mprset{flushleft}
          \inferrule* [right={\tiny Fl}] {
            {
              \begin{array}{c}
                \Pi' \\
                {[[G1; Y; G2; I; G3 |-l A]]}
              \end{array}
            }
          }{[[G1; F Y; G2; I; G3 |-l A]]}
        \end{math}
      \end{center}

\item Case 4:
      \begin{center}
        \scriptsize
        \begin{math}
          \begin{array}{c}
            \Pi_1 \\
            {[[D |-l B]]}
          \end{array}
        \end{math}
        \qquad\qquad
        $\Pi_2$:
        \begin{math}
          $$\mprset{flushleft}
          \inferrule* [right={\tiny Fl}] {
            {
              \begin{array}{c}
                \pi \\
                {[[G1; Y; G2; B; G3 |-l A]]}
              \end{array}
            }
          }{[[G1; F Y; G2; D; G3 |-l A]]}
        \end{math}
      \end{center}
      By assumption, $c(\Pi_1),c(\Pi_2)\leq |B|$. By induction on $\Pi_1$
      and $\pi$, there is a proof $\Pi'$ for sequent
      $[[G1; Y; G2; D; G3 |-l A]]$ s.t. $c(\Pi') \leq |B|$. Therefore, the
      proof $\Pi$ can be constructed as follows with
      $c(\Pi) = c(\Pi') \leq |B|$.
      \begin{center}
        \scriptsize
        \begin{math}
          $$\mprset{flushleft}
          \inferrule* [right={\tiny Fl}] {
            {
              \begin{array}{c}
                \Pi' \\
                {[[G1; Y; G2; D; G3 |-l A]]}
              \end{array}
            }
          }{[[G1; F Y; G2; D; G3 |-l A]]}
        \end{math}
      \end{center}
\end{itemize}




\subsubsection{Right introduction of the functor $G$ (with low priority)}
\begin{center}
  \scriptsize
  \begin{math}
    \begin{array}{c}
      \Pi_1 \\
      {[[I |-c X]]}
    \end{array}
  \end{math}
  \qquad\qquad
  $\Pi_2$:
  \begin{math}
    $$\mprset{flushleft}
    \inferrule* [right={\tiny Gr}] {
      {
        \begin{array}{c}
          \pi \\
          {[[P1; X; P2 |-l A]]}
        \end{array}
      }
    }{[[P1, X, P2 |-c Gf A]]}
  \end{math}
\end{center}
By assumption, $c(\Pi_1),c(\Pi_2)\leq |X|$. By induction on $\Pi_1$
and $\pi$, there is a proof $\Pi'$ for sequent \\
$[[P1, I, P2 |-l A]]$ s.t. $c(\Pi') \leq |X|$. Therefore, the proof $\Pi$
can be constructed as follows with \\
$c(\Pi) = c(\Pi') \leq |X|$.
\begin{center}
  \scriptsize
  \begin{math}
    $$\mprset{flushleft}
    \inferrule* [right={\tiny Gr}] {
      {
        \begin{array}{c}
          \Pi' \\
          {[[P1; I; P2 |-l A]]}
        \end{array}
      }
    }{[[P1, I, P2 |-c Gf A]]}
  \end{math}
\end{center}




\subsubsection{Left introduction of the functor $G$ (with low priority)}
\begin{itemize}
\item Case 1:
      \begin{center}
        \scriptsize
        \begin{math}
          \begin{array}{c}
            \Pi_1 \\
            {[[I |-c X]]}
          \end{array}
        \end{math}
        \qquad\qquad
        $\Pi_2$:
        \begin{math}
          $$\mprset{flushleft}
          \inferrule* [right={\tiny Gl}] {
            {
              \begin{array}{c}
                \pi \\
                {[[G1; X; G2; B; G3 |-l A]]}
              \end{array}
            }
          }{[[G1; X; G2; Gf B; G3 |-l A]]}
        \end{math}
      \end{center}
      By assumption, $c(\Pi_1),c(\Pi_2)\leq |X|$. By induction on $\Pi_1$
      and $\pi$, there is a proof $\Pi'$ for sequent
      $[[G1; I; G2; B; G3 |-l A]]$ s.t. $c(\Pi') \leq |X|$. Therefore, the
      proof $\Pi$ can be constructed as follows with
      $c(\Pi) = c(\Pi') \leq |X|$.
      \begin{center}
        \scriptsize
        \begin{math}
          $$\mprset{flushleft}
          \inferrule* [right={\tiny Gl}] {
            {
              \begin{array}{c}
                \Pi' \\
                {[[G1; I; G2; B; G3 |-l A]]}
              \end{array}
            }
          }{[[G1; I; G2; Gf B; G3 |-l A]]}
        \end{math}
      \end{center}

\item Case 2:
      \begin{center}
        \scriptsize
        \begin{math}
          \begin{array}{c}
            \Pi_1 \\
            {[[D |-l B]]}
          \end{array}
        \end{math}
        \qquad\qquad
        $\Pi_2$:
        \begin{math}
          $$\mprset{flushleft}
          \inferrule* [right={\tiny Gl}] {
            {
              \begin{array}{c}
                \pi \\
                {[[G1; B; G2; C; G3 |-l A]]}
              \end{array}
            }
          }{[[G1; B; G2; Gf C; G3 |-l A]]}
        \end{math}
      \end{center}
      By assumption, $c(\Pi_1),c(\Pi_2)\leq |B|$. By induction on $\Pi_1$
      and $\pi$, there is a proof $\Pi'$ for sequent
      $[[G1; D; G2; C; G3 |-l A]]$ s.t. $c(\Pi') \leq |B|$. Therefore, the
      proof $\Pi$ can be constructed as follows with
      $c(\Pi) = c(\Pi') \leq |B|$.
      \begin{center}
        \scriptsize
        \begin{math}
          $$\mprset{flushleft}
          \inferrule* [right={\tiny Gl}] {
            {
              \begin{array}{c}
                \Pi' \\
                {[[G1; D; G2; C; G3 |-l A]]}
              \end{array}
            }
          }{[[G1; D; G2; Gf C; G3 |-l A]]}
        \end{math}
      \end{center}

\item Case 3:
      \begin{center}
        \scriptsize
        \begin{math}
          \begin{array}{c}
            \Pi_1 \\
            {[[I |-c X]]}
          \end{array}
        \end{math}
        \qquad\qquad
        $\Pi_2$:
        \begin{math}
          $$\mprset{flushleft}
          \inferrule* [right={\tiny Gl}] {
            {
              \begin{array}{c}
                \pi \\
                {[[G1; B; G2; X; G3 |-l A]]}
              \end{array}
            }
          }{[[G1; Gf B; G2; X; G3 |-l A]]}
        \end{math}
      \end{center}
      By assumption, $c(\Pi_1),c(\Pi_2)\leq |X|$. By induction on $\Pi_1$
      and $\pi$, there is a proof $\Pi'$ for sequent
      $[[G1; B; G2; I; G3 |-l A]]$ s.t. $c(\Pi') \leq |X|$. Therefore, the
      proof $\Pi$ can be constructed as follows with
      $c(\Pi) = c(\Pi') \leq |X|$.
      \begin{center}
        \scriptsize
        \begin{math}
          $$\mprset{flushleft}
          \inferrule* [right={\tiny Gl}] {
            {
              \begin{array}{c}
                \Pi' \\
                {[[G1; B; G2; I; G3 |-l A]]}
              \end{array}
            }
          }{[[G1; Gf B; G2; I; G3 |-l A]]}
        \end{math}
      \end{center}

\item Case 4:
      \begin{center}
        \scriptsize
        \begin{math}
          \begin{array}{c}
            \Pi_1 \\
            {[[D |-l B]]}
          \end{array}
        \end{math}
        \qquad\qquad
        $\Pi_2$:
        \begin{math}
          $$\mprset{flushleft}
          \inferrule* [right={\tiny Gl}] {
            {
              \begin{array}{c}
                \pi \\
                {[[G1; C; G2; B; G3 |-l A]]}
              \end{array}
            }
          }{[[G1; Gf C; G2; B; G3 |-l A]]}
        \end{math}
      \end{center}
      By assumption, $c(\Pi_1),c(\Pi_2)\leq |B|$. By induction on $\Pi_1$
      and $\pi$, there is a proof $\Pi'$ for sequent
      $[[G1; C; G2; D; G3 |-l A]]$ s.t. $c(\Pi') \leq |B|$. Therefore, the
      proof $\Pi$ can be constructed as follows with
      $c(\Pi) = c(\Pi') \leq |B|$.
      \begin{center}
        \scriptsize
        \begin{math}
          $$\mprset{flushleft}
          \inferrule* [right={\tiny Gl}] {
            {
              \begin{array}{c}
                \Pi' \\
                {[[G1; C; G2; D; G3 |-l A]]}
              \end{array}
            }
          }{[[G1; Gf C; G2; D; G3 |-l A]]}
        \end{math}
      \end{center}

\end{itemize}



%--------------------------------------------------
%--------------------------------------------------
\section{Proof For Lemma~\ref{lem:monoidal-monad}}
\label{app:monoidal-monad}

Let $(\cat{C},\cat{L},F,G,\eta,\varepsilon)$ be a LAM. We define the monad
$(T,\eta:id_\cat{C}\rightarrow T,\mu:T^2\rightarrow T)$ on $\cat{C}$ as
$T=GF$, $\eta_X:X\rightarrow GFX$, and
$\mu_X=G\varepsilon_{FX}:GFGFX\rightarrow GFX$. Since $(F,\m{})$ and
$(G,\n{})$ are monoidal functors, we have
$$\t{X,Y}=G\m{X,Y}\circ\n{FX,FY}:TX\otimes TY\rightarrow T(X\otimes Y) \qquad\mbox{and}\qquad\t{I}=G\m{I}\circ\n{I'}:I\rightarrow TI.$$
The monad $T$ being monoidal means:
\begin{enumerate}
\item $T$ is a monoidal functor, i.e. the following diagrams commute:
      \begin{mathpar}
      \bfig
        \hSquares/->`->`->``->`->`->/<400>[
          (TX\otimes TY)\otimes TZ`TX\otimes(TY\otimes TZ)`TX\otimes T(Y\otimes Z)`
          T(X\otimes Y)\otimes TZ`T((X\otimes Y)\otimes Z)`T(X\otimes(Y\otimes Z));
          \alpha_{TX,TY,TZ}`id_{TX}\otimes\t{Y,Z}`\t{X,Y}\otimes id_{TZ}``
          \t{X,Y\otimes Z}`\t{X\otimes Y,Z}`T\alpha_{X,Y,Z}]
        \morphism(1300,200)//<0,0>[`;(1)]
      \efig
      \and
      \bfig
        \square/->`->`<-`->/<600,400>[
          I\otimes TX`TX`TI\otimes TX`T(I\otimes X);
          \lambda_{TX}`\t{I}\otimes id_{TX}`T\lambda_X`\t{I,X}]
        \morphism(350,200)//<0,0>[`;(2)]
      \efig
      \and
      \bfig
        \square/->`->`<-`->/<600,400>[
          TX\otimes I`TX`TX\otimes TI`T(X\otimes I);
          \rho_{TX}`id_{TX}\otimes\t{I}`T\rho_X`\t{X,I}]
        \morphism(350,200)//<0,0>[`;(3)]
      \efig
      \end{mathpar}
      We write $GF$ instead of $T$ in the proof for clarity. \\
      By replacing $\t{X,Y}$ with its definition, diagram (1) above
      commutes by the following commutative diagram, in which the two
      hexagons commute because $G$ and $F$ are monoidal functors, and the
      two quadrilaterals commute by the naturality of $\n{}$.
      \begin{mathpar}
      \bfig
        \iiixiii/->`->`->``->```->`<-`->``/<1400,400>[
          (GFX\otimes GFY)\otimes GFZ`GFX\otimes(GFY\otimes GFZ)`GFX\otimes G(FY\tri FZ)`
          G(FX\tri FY)\otimes GFZ`G(FX\tri(FY\tri FZ))`GFX\otimes GF(Y\otimes Z)`
          GF(X\otimes Y)\otimes GFZ`G((FX\tri FY)\tri FZ)`G(FX\tri F(Y\otimes Z));
          \alpha_{GFX,GFY,GFZ}`id_{GFX}\otimes\n{FY,FZ}`\n{FX,FY}\otimes id_{GFZ}``
          id_{GFX}\otimes G\m{Y,Z}```G\m{X,Y}\otimes id_{GFZ}`G\alpha'_{FX,FY,FZ}`
          \n{FX,F(Y\otimes Z)}``]
        \morphism(2800,800)|m|<-1400,-400>[
          GFX\otimes G(FY\tri FZ)`G(FX\tri(FY\tri FZ));\n{FX,FY\tri FZ}]
        \morphism(0,400)|m|<1400,-400>[
          G(FX\tri FY)\otimes GFZ`G((FX\tri FY)\tri FZ);\n{FX\tri FY,FZ}]
        \morphism(1400,400)|m|<1400,-400>[
          G(FX\tri(FY\tri FZ))`G(FX\tri F(Y\otimes Z));G(id_{FX}\tri\m{Y,Z})]
        \ptriangle(0,-400)|mlm|/`->`->/<1400,400>[
          GF(X\otimes Y)\otimes GFZ`G((FX\tri FY)\tri FZ)`G(F(X\otimes Y)\tri FZ);
          `\n{F(X\otimes Y),FZ}`G(\m{X,Y}\otimes id_{FZ})]
        \morphism(0,-400)|b|<1400,0>[
          G(F(X\otimes Y)\tri FZ)`GF((X\otimes Y)\otimes Z);G\m{X\otimes Y,Z}]
        \dtriangle(1400,-400)|mrb|/`->`->/<1400,400>[
          G(FX\tri F(Y\otimes Z))`GF((X\otimes Y)\otimes Z)`GF(X\otimes(Y\otimes Z));
          `G\m{X,Y\otimes Z}`GF\alpha_{X,Y,Z}]
      \efig
      \end{mathpar}
      Diagram (2) commutes by the following commutative diagrams, in which
      the top quadrilateral commutes because $G$ is monoidal, the right
      quadrilateral commutes because $F$ is monoidal, and the left square
      commutes by the naturality of $\n{}$.
      \begin{mathpar}
      \bfig
        \ptriangle/->`->`/<1600,400>[
          I\otimes GFX`GFX`GI'\otimes GFX;\lambda_{GFX}`\n{I'}\otimes id_{GFX}`]
        \square(0,-400)|lmmb|<800,400>[
          GI'\otimes GFX`G(I'\tri FX)`GFI\otimes GFX`G(FI\tri FX);
          \n{I',FX}`G\m{I}\otimes id_{GFX}`G(\m{I}\tri id_{FX})`\n{FI,FX}]
        \morphism(800,0)|m|<800,400>[G(I'\tri FX)`GFX;G\lambda'_{FX}]
        \dtriangle(800,-400)/`<-`->/<800,800>[
          GFX`G(FI\tri FX)`GF(I\otimes X);
          `GF\lambda_X`G\m{I,X}]
      \efig
      \end{mathpar}
      Similarly, diagram (3) commutes as follows:
      \begin{mathpar}
      \bfig
        \ptriangle/->`->`/<1600,400>[
          GFX\otimes I`GFX`GFX\otimes GI';\rho_{GFX}`id_{GFX}\otimes\n{I'}`]
        \square(0,-400)|lmmb|<800,400>[
          GFX\otimes GI'`G(FX\tri I')`GFX\otimes GFI`G(FX\tri FI);
          \n{FX,I'}`id_{GFX}\otimes G\m{I}`G(id_{FX}\otimes\m{I})`\n{FX,FI}]
        \morphism(800,0)|m|<800,400>[G(FX\tri I')`GFX;G\rho'_{FX}]
        \dtriangle(800,-400)/`<-`->/<800,800>[
          GFX`G(FX\tri FI)`GF(X\otimes I);
          `GF\rho_X`G\m{X,I}]
      \efig
      \end{mathpar}
\item $\eta$ is a monoidal natural transformation. In fact, since $\eta$
      is the unit of the monoidal adjunction, $\eta$ is monoidal by
      definition and thus the following two diagrams commute.
      \begin{mathpar}
      \bfig
        \square/=`->`->`->/<600,400>[
          X\otimes Y`X\otimes Y`TX\otimes TY`T(X\otimes Y);
          `\eta_X\otimes\eta_Y`\eta_{X\otimes Y}`\t{X,Y}]
      \efig
      \and
      \bfig
        \Vtriangle/->`=`<-/<400,400>[I`TI`I;\eta_I``\t{I}]
      \efig
      \end{mathpar}
\item $\mu$ is a monoidal natural transformation. It is obvious that since
      $\mu=G\varepsilon_{FA}$ and $\varepsilon$ is monoidal, so is $\mu$.
      Thus the following diagrams commute.
      \begin{mathpar}
      \bfig
        \square/`->`->`->/<1500,400>[
          T^2X\otimes T^2Y`T^2(X\otimes Y)`TX\otimes TY`T(X\otimes Y);
          `\mu_X\otimes\mu_Y`\mu_{X\otimes Y}`\t{X,Y}]
        \morphism(0,400)<800,0>[T^2X\otimes T^2Y`T(TX\otimes TY);\t{TX,TY}]
        \morphism(800,400)<700,0>[T(TX\otimes TY)`T^2(X\otimes Y);T\t{X,Y}]
      \efig
      \and
      \bfig
        \square/->`<-`<-`<-/<400,400>[T^2I`TI`TI`I;\mu_I`T\t{I}`\t{I}`\t{I}]
      \efig
      \end{mathpar}
\end{enumerate}



%--------------------------------------------------
%--------------------------------------------------
\section{Proof For Lemma~\ref{lem:strong-monad}}
\label{app:strong-monad}

\begin{definition}
\label{def:strong-monad}
Let $(\cat{M},\tri,I,\alpha,\lambda,\rho)$ be a monoidal category and
$(T,\eta,\mu)$ be a monad on $\cat{M}$. $T$ is a \textbf{strong monad} if
there is natural transformation $\tau$, called the \textbf{tensorial
strength}, with components \\
$\tau_{A,B}:A\tri TB\rightarrow T(A\tri B)$ such that the following
diagrams commute:
\begin{mathpar}
\bfig
  \Vtriangle<400,400>[I\tri TA`T(I\tri A)`TA;\tau_{I,A}`\lambda_{TA}`T\lambda_A]
\efig
\and
\bfig
  \Vtriangle<400,400>[
    A\tri B`A\tri TB`T(A\tri B);id_A\tri\eta_B`\eta_{A\tri B}`\tau_{A,B}]
\efig
\and
\bfig
  \square/->`->`->`/<1800,400>[
    (A\tri B)\tri TC`T((A\tri B)\tri C)`
    A\tri(B\tri TC)`T(A\tri(B\tri C));
    \tau_{A\tri B,C}`\alpha_{A,B,TC}`T\alpha_{A,B,C}`]
  \morphism<900,0>[A\tri(B\tri TC)`A\tri T(B\tri C);id_A\tri\tau_{B,C}]
  \morphism(900,0)<900,0>[A\tri T(B\tri C)`T(A\tri(B\tri C));\tau_{A,B\tri C}]  \efig
\and
\bfig
  \square/`->`->`->/<1400,400>[
    A\tri T^2B`T^2(A\tri B)`A\tri TB`T(A\tri B);
    `id_A\tri\mu_B`\mu_{A\tri B}`\tau_{A,B}]
  \morphism(0,400)<700,0>[A\tri T^2B`T(A\tri TB);\tau_{A,TB}]
  \morphism(700,400)<700,0>[T(A\tri TB)`T^2(A\tri B);T\tau_{A,B}]
\efig
\end{mathpar}
\end{definition}
\noindent
The proof for Lemma~\ref{lem:strong-monad} goes as follows.
\noindent
Let $(\cat{C},\cat{L},F,G,\eta,\varepsilon)$ be a LAM, where
$(\cat{C},\otimes,I,\alpha,\lambda,\rho)$ is symmetric monoidal closed,
and \\ $(\cat{L},\tri,I',\alpha',\lambda',\rho')$ is Lambek. In
Lemma~\ref{lem:monoidal-monad}, we have proved that the monad
$(T=GF,\eta,\mu)$ is monoidal with the natural transformation
$\t{X,Y}:TX\otimes TY\rightarrow T(X\otimes Y)$ and the morphism
$\t{I}:I\rightarrow TI$.
\noindent
We define the tensorial strength
$\tau_{X,Y}:X\otimes TY\rightarrow T(X\otimes Y)$ as
$$\tau_{X,Y}=\t{X,Y}\circ(\eta_X\otimes id_{TY}).$$
Since $\eta$ is a monoidal natural transformation, we have
$\eta_I=G\m{I}\circ\n{I'}$, and thus $\eta_I=\t{I}$. The following diagram
commutes because $T$ is monoidal, where the composition
$\t{I,X}\circ(\t{I}\otimes id_{TX})$ is the definition of $\tau_{I,X}$. So
the first triangle in Definition~\ref{def:strong-monad} commutes.
\begin{mathpar}
\bfig
  \square/->`->`->`<-/<600,400>[
    I\otimes TX`TI\otimes TX`TX`T(I\otimes X);
    \t{I}\otimes id_{TX}`\lambda_{TX}`\t{I,X}`T\lambda_X]
\efig
\end{mathpar}
Similarly, by using the definition of $\tau$, the the second triangle in the definition is
equivalent to the following diagram, which commutes because $\eta$ is a monoidal natural
transformation:
\begin{mathpar}
\bfig
  \square/->`->`->`<-/<600,400>[
    X\otimes Y`X\otimes TY`T(X\otimes Y)`TX\otimes TY;
    id_X\otimes\eta_Y`\eta_{X\otimes Y}`\eta_X\otimes id_{TY}`\t{X,Y}]
  \morphism(0,400)|m|<600,-400>[X\otimes Y`TX\otimes TY;\eta_X\otimes\eta_Y]
\efig
\end{mathpar}
The first pentagon in the definition commutes by the following commutative diagrams, because
$\eta$ and $\alpha$ are natural transformations and $T$ is monoidal:
\begin{mathpar}
\bfig
  \qtriangle|amm|/->`->`<-/<1000,400>[
    (X\otimes Y)\otimes TZ`T(X\otimes Y)\otimes TZ`(TX\otimes TY)\otimes TZ;
    \eta_{X\otimes Y}\otimes id_{TZ}`
    (\eta_X\otimes\eta_Y)\otimes id_{TZ}`
    \t{X,Y}\otimes id_{TZ}]
  \morphism(0,400)<0,-400>[(X\otimes Y)\otimes TZ`X\otimes(Y\otimes TZ);\alpha_{X,Y,TZ}]
  \morphism(1000,0)|m|<0,-400>[
    (TX\otimes TY)\otimes TZ`TX\otimes(TY\otimes TZ);\alpha_{TX,TY,TZ}]
  \Dtriangle(0,-800)|lmm|/->`->`<-/<1000,400>[
    X\otimes(Y\otimes TZ)`TX\otimes(TY\otimes TZ)`X\otimes(TY\otimes TZ);
    id_X\otimes(\eta_Y\otimes id_{TZ})`
    \eta_X\otimes(\eta_Y\otimes id_{TZ})`
    \eta_X\otimes id_{TY\otimes TZ}]
  \morphism(0,-800)|b|<1000,0>[
    X\otimes(TY\otimes TZ)`X\otimes T(Y\otimes Z);id_X\otimes\t{Y,Z}]
  \qtriangle(1000,0)|amr|/->``->/<1000,400>[
    T(X\otimes Y)\otimes TZ`T((X\otimes Y)\otimes Z)`T(X\otimes(Y\otimes Z));
    \t{X\otimes Y,Z}``T\alpha_{X,Y,Z}]
  \morphism(2000,-800)<0,800>[
    TX\otimes T(Y\otimes Z)`T(X\otimes(Y\otimes Z));\t{X,Y\otimes Z}]
  \btriangle(1000,-800)|mmb|/`->`->/<1000,400>[
    TX\otimes(TY\otimes TZ)`X\otimes T(Y\otimes Z)`TX\otimes T(Y\otimes Z);
    `id_{TX}\otimes\t{Y,Z}`\eta_X\otimes id_{T(Y\otimes Z)}]
\efig
\end{mathpar}
The last diagram in the definition commutes by the following commutative diagram, because
$T$ is a monad, $\t{}$ is a natural transformation, and $\mu$ is a monoidal natural
transformation:
\begin{mathpar}
\bfig
  \ptriangle/->`->`/<700,400>[
    X\otimes T^2Y`TX\otimes T^2Y`X\otimes TY;\eta_X\otimes id_{T^2Y}`id_X\otimes\mu_Y`]
  \btriangle(0,-400)/->``->/<700,400>[
    X\otimes TY`TX\otimes TY`T(X\otimes Y);\eta_X\otimes id_{TY}``\t{X,Y}]
  \morphism(700,400)|m|<-700,-800>[TX\otimes T^2Y`TX\otimes TY;id_{TX}\otimes\mu_Y]
  \morphism(700,0)|m|<-700,-400>[TX\otimes T^2Y`TX\otimes TY;id_{TX}\otimes\mu_Y]
  \qtriangle(700,0)/->``->/<1800,400>[
    TX\otimes T^2Y`T(X\otimes TY)`T(TX\otimes TY);\t{X,TY}``T(\eta_X\otimes id_{TY})]
  \btriangle(700,0)|mmm|/=`->`<-/<900,400>[
    TX\otimes T^2Y`TX\otimes T^2Y`T^2X\otimes T^2Y;
    `T\eta_X\otimes id_{T^2Y}`\mu_X\otimes id_{T^2Y}]
  \morphism(1600,0)|m|<900,0>[T^2X\otimes T^2Y`T(TX\otimes TY);\t{TX,TY}]
  \morphism(1600,0)|m|<-1600,-400>[T^2X\otimes T^2Y`TX\otimes TY;\mu_X\otimes\mu_Y]
  \dtriangle(700,-400)/`->`<-/<1800,400>[
    T(TX\otimes TY)`T(X\otimes Y)`T^2(X\otimes Y);`T\t{X,Y}`\mu_{X\otimes Y}]
\efig
\end{mathpar}




% section appendix (end)

\end{document}

%%% Local Variables: 
%%% mode: latex
%%% TeX-master: t
%%% End:

