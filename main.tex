\documentclass[a4paper,UKenglish]{lipics-v2016}

\usepackage{amssymb,amsmath}
\usepackage{amsthm}
\usepackage{cmll}
\usepackage{txfonts}
\usepackage{graphicx}
\usepackage{stmaryrd}
\usepackage{todonotes}
\usepackage{mathpartir}
\usepackage{hyperref}
\usepackage{mdframed}
\usepackage[barr]{xy}
\usepackage{comment}
\usepackage{graphicx}
\usepackage[inline]{enumitem}

\usepackage{caption}
\captionsetup[figure]{name=Diagram}

%% This renames Barr's \to to \mto.  This allows us to use \to for imp
%% and \mto for a inline morphism.
\let\mto\to
\let\to\relax
\newcommand{\to}{\rightarrow}
\newcommand{\ndto}[1]{\to_{#1}}
\newcommand{\ndwedge}[1]{\wedge_{#1}}
\newcommand{\rto}{\leftharpoonup}
\newcommand{\lto}{\rightharpoonup}

% Commands that are useful for writing about type theory and programming language design.
%% \newcommand{\case}[4]{\text{case}\ #1\ \text{of}\ #2\text{.}#3\text{,}#2\text{.}#4}
\newcommand{\interp}[1]{\llbracket #1 \rrbracket}
\newcommand{\normto}[0]{\rightsquigarrow^{!}}
\newcommand{\join}[0]{\downarrow}
\newcommand{\redto}[0]{\rightsquigarrow}
\newcommand{\nat}[0]{\mathbb{N}}
\newcommand{\fun}[2]{\lambda #1.#2}
\newcommand{\CRI}[0]{\text{CR-Norm}}
\newcommand{\CRII}[0]{\text{CR-Pres}}
\newcommand{\CRIII}[0]{\text{CR-Prog}}
\newcommand{\subexp}[0]{\sqsubseteq}
%% Must include \usepackage{mathrsfs} for this to work.

\date{}

\let\b\relax
\let\d\relax
\let\t\relax
\let\r\relax
\let\c\relax
\let\j\relax
\let\wn\relax
\let\H\relax

% Cat commands.
\newcommand{\powerset}[1]{\mathcal{P}(#1)}
\newcommand{\cat}[1]{\mathcal{#1}}
\newcommand{\func}[1]{\mathsf{#1}}
\newcommand{\iso}[0]{\mathsf{iso}}
\newcommand{\H}[0]{\func{H}}
\newcommand{\J}[0]{\func{J}}
\newcommand{\catop}[1]{\cat{#1}^{\mathsf{op}}}
\newcommand{\Hom}[3]{\mathsf{Hom}_{\cat{#1}}(#2,#3)}
\newcommand{\limp}[0]{\multimap}
\newcommand{\colimp}[0]{\multimapdotinv}
\newcommand{\dial}[1]{\mathsf{Dial_{#1}}(\mathsf{Sets^{op}})}
\newcommand{\dialSets}[1]{\mathsf{Dial_{#1}}(\mathsf{Sets})}
\newcommand{\dcSets}[1]{\mathsf{DC_{#1}}(\mathsf{Sets})}
\newcommand{\sets}[0]{\mathsf{Sets}}
\newcommand{\obj}[1]{\mathsf{Obj}(#1)}
\newcommand{\mor}[1]{\mathsf{Mor(#1)}}
\newcommand{\id}[0]{\mathsf{id}}
\newcommand{\lett}[0]{\mathsf{let}\,}
\newcommand{\inn}[0]{\,\mathsf{in}\,}
\newcommand{\cur}[1]{\mathsf{cur}(#1)}
\newcommand{\curi}[1]{\mathsf{cur}^{-1}(#1)}

\newcommand{\w}[1]{\mathsf{weak}_{#1}}
\newcommand{\c}[1]{\mathsf{contra}_{#1}}
\newcommand{\cL}[1]{\mathsf{contraL}_{#1}}
\newcommand{\cR}[1]{\mathsf{contraR}_{#1}}
\newcommand{\e}[1]{\mathsf{ex}_{#1}}

\newcommand{\m}[1]{\mathsf{m}_{#1}}
\newcommand{\n}[1]{\mathsf{n}_{#1}}
\newcommand{\b}[1]{\mathsf{b}_{#1}}
\newcommand{\d}[1]{\mathsf{d}_{#1}}
\newcommand{\h}[1]{\mathsf{h}_{#1}}
\newcommand{\p}[1]{\mathsf{p}_{#1}}
\newcommand{\q}[1]{\mathsf{q}_{#1}}
\newcommand{\t}[0]{\mathsf{t}}
\newcommand{\r}[1]{\mathsf{r}_{#1}}
\newcommand{\s}[1]{\mathsf{s}_{#1}}
\newcommand{\j}[1]{\mathsf{j}_{#1}}
\newcommand{\jinv}[1]{\mathsf{j}^{-1}_{#1}}
\newcommand{\wn}[0]{\mathop{?}}
\newcommand{\codiag}[1]{\bigtriangledown_{#1}}

\newcommand{\seq}{\rhd}

\newenvironment{changemargin}[2]{%
  \begin{list}{}{%
    \setlength{\topsep}{0pt}%
    \setlength{\leftmargin}{#1}%
    \setlength{\rightmargin}{#2}%
    \setlength{\listparindent}{\parindent}%
    \setlength{\itemindent}{\parindent}%
    \setlength{\parsep}{\parskip}%
  }%
  \item[]}{\end{list}}

\newenvironment{diagram}{
  \begin{center}
    \begin{math}
      \bfig
}{
      \efig
    \end{math}
  \end{center}
}

%% Ott
% % generated by Ott 0.25 from: Elle/Elle.ott
\newcommand{\Elledrule}[4][]{{\displaystyle\frac{\begin{array}{l}#2\end{array}}{#3}\quad\Elledrulename{#4}}}
\newcommand{\Elleusedrule}[1]{\[#1\]}
\newcommand{\Ellepremise}[1]{ #1 \\}
\newenvironment{Elledefnblock}[3][]{ \framebox{\mbox{#2}} \quad #3 \\[0pt]}{}
\newenvironment{Ellefundefnblock}[3][]{ \framebox{\mbox{#2}} \quad #3 \\[0pt]\begin{displaymath}\begin{array}{l}}{\end{array}\end{displaymath}}
\newcommand{\Ellefunclause}[2]{ #1 \equiv #2 \\}
\newcommand{\Ellent}[1]{\mathit{#1}}
\newcommand{\Ellemv}[1]{\mathit{#1}}
\newcommand{\Ellekw}[1]{\mathbf{#1}}
\newcommand{\Ellesym}[1]{#1}
\newcommand{\Ellecom}[1]{\text{#1}}
\newcommand{\Elledrulename}[1]{\textsc{#1}}
\newcommand{\Ellecomplu}[5]{\overline{#1}^{\,#2\in #3 #4 #5}}
\newcommand{\Ellecompu}[3]{\overline{#1}^{\,#2<#3}}
\newcommand{\Ellecomp}[2]{\overline{#1}^{\,#2}}
\newcommand{\Ellegrammartabular}[1]{\begin{supertabular}{llcllllll}#1\end{supertabular}}
\newcommand{\Ellemetavartabular}[1]{\begin{supertabular}{ll}#1\end{supertabular}}
\newcommand{\Ellerulehead}[3]{$#1$ & & $#2$ & & & \multicolumn{2}{l}{#3}}
\newcommand{\Elleprodline}[6]{& & $#1$ & $#2$ & $#3 #4$ & $#5$ & $#6$}
\newcommand{\Ellefirstprodline}[6]{\Elleprodline{#1}{#2}{#3}{#4}{#5}{#6}}
\newcommand{\Ellelongprodline}[2]{& & $#1$ & \multicolumn{4}{l}{$#2$}}
\newcommand{\Ellefirstlongprodline}[2]{\Ellelongprodline{#1}{#2}}
\newcommand{\Ellebindspecprodline}[6]{\Elleprodline{#1}{#2}{#3}{#4}{#5}{#6}}
\newcommand{\Elleprodnewline}{\\}
\newcommand{\Elleinterrule}{\\[5.0mm]}
\newcommand{\Elleafterlastrule}{\\}
\newcommand{\Ellemetavars}{
\Ellemetavartabular{
 $ \Ellemv{vars} ,\, \Ellemv{n} ,\, \Ellemv{a} ,\, \Ellemv{x} ,\, \Ellemv{y} ,\, \Ellemv{z} ,\, \Ellemv{w} ,\, \Ellemv{m} ,\, \Ellemv{o} $ &  \\
 $ \Ellemv{ivar} ,\, \Ellemv{i} ,\, \Ellemv{k} ,\, \Ellemv{j} ,\, \Ellemv{l} $ &  \\
 $ \Ellemv{const} ,\, \Ellemv{b} $ &  \\
}}

\newcommand{\ElleA}{
\Ellerulehead{\Ellent{A}  ,\ \Ellent{B}  ,\ \Ellent{C}  ,\ D}{::=}{}\Elleprodnewline
\Ellefirstprodline{|}{ \mathsf{B} }{}{}{}{}\Elleprodnewline
\Elleprodline{|}{ \mathsf{Unit} }{}{}{}{}\Elleprodnewline
\Elleprodline{|}{\Ellent{A}  \triangleright  \Ellent{B}}{}{}{}{}\Elleprodnewline
\Elleprodline{|}{\Ellent{A}  \rightharpoonup  \Ellent{B}}{}{}{}{}\Elleprodnewline
\Elleprodline{|}{\Ellent{A}  \leftharpoonup  \Ellent{B}}{}{}{}{}\Elleprodnewline
\Elleprodline{|}{\Ellesym{(}  \Ellent{A}  \Ellesym{)}} {\textsf{M}}{}{}{}\Elleprodnewline
\Elleprodline{|}{ \Ellent{A} } {\textsf{M}}{}{}{}\Elleprodnewline
\Elleprodline{|}{ \mathsf{F} \Ellent{X} }{}{}{}{}}

\newcommand{\ElleW}{
\Ellerulehead{\Ellent{W}  ,\ \Ellent{X}  ,\ \Ellent{Y}  ,\ \Ellent{Z}}{::=}{}\Elleprodnewline
\Ellefirstprodline{|}{ \mathsf{B} }{}{}{}{}\Elleprodnewline
\Elleprodline{|}{ \mathsf{Unit} }{}{}{}{}\Elleprodnewline
\Elleprodline{|}{\Ellent{X}  \otimes  \Ellent{Y}}{}{}{}{}\Elleprodnewline
\Elleprodline{|}{\Ellent{X}  \multimap  \Ellent{Y}}{}{}{}{}\Elleprodnewline
\Elleprodline{|}{\Ellesym{(}  \Ellent{X}  \Ellesym{)}} {\textsf{M}}{}{}{}\Elleprodnewline
\Elleprodline{|}{ \Ellent{X} } {\textsf{M}}{}{}{}\Elleprodnewline
\Elleprodline{|}{ \mathsf{G} \Ellent{A} }{}{}{}{}}

\newcommand{\ElleT}{
\Ellerulehead{\Ellent{T}}{::=}{}\Elleprodnewline
\Ellefirstprodline{|}{\Ellent{A}}{}{}{}{}\Elleprodnewline
\Elleprodline{|}{\Ellent{X}}{}{}{}{}}

\newcommand{\Ellep}{
\Ellerulehead{\Ellent{p}  ,\ \Ellent{q}}{::=}{}\Elleprodnewline
\Ellefirstprodline{|}{ \star }{}{}{}{}\Elleprodnewline
\Elleprodline{|}{\Ellemv{x}}{}{}{}{}\Elleprodnewline
\Elleprodline{|}{ \mathsf{triv} }{}{}{}{}\Elleprodnewline
\Elleprodline{|}{ \mathsf{triv} }{}{}{}{}\Elleprodnewline
\Elleprodline{|}{\Ellent{p}  \otimes  \Ellent{p'}}{}{}{}{}\Elleprodnewline
\Elleprodline{|}{\Ellent{p}  \triangleright  \Ellent{p'}}{}{}{}{}\Elleprodnewline
\Elleprodline{|}{ \mathsf{F}\, \Ellent{p} }{}{}{}{}\Elleprodnewline
\Elleprodline{|}{ \mathsf{G}\, \Ellent{p} }{}{}{}{}}

\newcommand{\Elles}{
\Ellerulehead{\Ellent{s}}{::=}{}\Elleprodnewline
\Ellefirstprodline{|}{\Ellemv{x}}{}{}{}{}\Elleprodnewline
\Elleprodline{|}{\Ellemv{b}}{}{}{}{}\Elleprodnewline
\Elleprodline{|}{ \mathsf{triv} }{}{}{}{}\Elleprodnewline
\Elleprodline{|}{ \mathsf{let}\, \Ellent{s_{{\mathrm{1}}}}  :  \Ellent{A} \,\mathsf{be}\, \Ellent{p} \,\mathsf{in}\, \Ellent{s_{{\mathrm{2}}}} }{}{}{}{}\Elleprodnewline
\Elleprodline{|}{ \mathsf{let}\, \Ellent{t}  :  \Ellent{X} \,\mathsf{be}\, \Ellent{p} \,\mathsf{in}\, \Ellent{s} }{}{}{}{}\Elleprodnewline
\Elleprodline{|}{\Ellent{s_{{\mathrm{1}}}}  \triangleright  \Ellent{s_{{\mathrm{2}}}}}{}{}{}{}\Elleprodnewline
\Elleprodline{|}{ \lambda_l  \Ellemv{x}  :  \Ellent{A} . \Ellent{s} }{}{}{}{}\Elleprodnewline
\Elleprodline{|}{ \lambda_r  \Ellemv{x}  :  \Ellent{A} . \Ellent{s} }{}{}{}{}\Elleprodnewline
\Elleprodline{|}{ \mathsf{app}_l\, \Ellent{s_{{\mathrm{1}}}} \, \Ellent{s_{{\mathrm{2}}}} }{}{}{}{}\Elleprodnewline
\Elleprodline{|}{ \mathsf{app}_r\, \Ellent{s_{{\mathrm{1}}}} \, \Ellent{s_{{\mathrm{2}}}} }{}{}{}{}\Elleprodnewline
\Elleprodline{|}{ \mathsf{derelict}\, \Ellent{t} }{}{}{}{}\Elleprodnewline
\Elleprodline{|}{ \mathsf{ex}\, \Ellent{s_{{\mathrm{1}}}} , \Ellent{s_{{\mathrm{2}}}} \,\mathsf{with}\, \Ellemv{x_{{\mathrm{1}}}} , \Ellemv{x_{{\mathrm{2}}}} \,\mathsf{in}\, \Ellent{s_{{\mathrm{3}}}} }{}{}{}{}\Elleprodnewline
\Elleprodline{|}{\Ellesym{[}  \Ellent{s_{{\mathrm{1}}}}  \Ellesym{/}  \Ellemv{x}  \Ellesym{]}  \Ellent{s_{{\mathrm{2}}}}} {\textsf{M}}{}{}{}\Elleprodnewline
\Elleprodline{|}{\Ellesym{[}  \Ellent{t}  \Ellesym{/}  \Ellemv{x}  \Ellesym{]}  \Ellent{s}} {\textsf{M}}{}{}{}\Elleprodnewline
\Elleprodline{|}{\Ellesym{(}  \Ellent{s}  \Ellesym{)}} {\textsf{S}}{}{}{}\Elleprodnewline
\Elleprodline{|}{ \Ellent{s} } {\textsf{M}}{}{}{}\Elleprodnewline
\Elleprodline{|}{ \mathsf{F} \Ellent{t} }{}{}{}{}}

\newcommand{\Ellet}{
\Ellerulehead{\Ellent{t}}{::=}{}\Elleprodnewline
\Ellefirstprodline{|}{\Ellemv{x}}{}{}{}{}\Elleprodnewline
\Elleprodline{|}{\Ellemv{b}}{}{}{}{}\Elleprodnewline
\Elleprodline{|}{ \mathsf{triv} }{}{}{}{}\Elleprodnewline
\Elleprodline{|}{ \mathsf{let}\, \Ellent{t_{{\mathrm{1}}}}  :  \Ellent{X} \,\mathsf{be}\, \Ellent{p} \,\mathsf{in}\, \Ellent{t_{{\mathrm{2}}}} }{}{}{}{}\Elleprodnewline
\Elleprodline{|}{\Ellent{t_{{\mathrm{1}}}}  \otimes  \Ellent{t_{{\mathrm{2}}}}}{}{}{}{}\Elleprodnewline
\Elleprodline{|}{ \lambda  \Ellemv{x}  :  \Ellent{X} . \Ellent{t} }{}{}{}{}\Elleprodnewline
\Elleprodline{|}{ \Ellent{t_{{\mathrm{1}}}}   \Ellent{t_{{\mathrm{2}}}} }{}{}{}{}\Elleprodnewline
\Elleprodline{|}{ \mathsf{ex}\, \Ellent{t_{{\mathrm{1}}}} , \Ellent{t_{{\mathrm{2}}}} \,\mathsf{with}\, \Ellemv{x_{{\mathrm{1}}}} , \Ellemv{x_{{\mathrm{2}}}} \,\mathsf{in}\, \Ellent{t_{{\mathrm{3}}}} }{}{}{}{}\Elleprodnewline
\Elleprodline{|}{\Ellesym{[}  \Ellent{t_{{\mathrm{1}}}}  \Ellesym{/}  \Ellemv{x}  \Ellesym{]}  \Ellent{t_{{\mathrm{2}}}}} {\textsf{M}}{}{}{}\Elleprodnewline
\Elleprodline{|}{\Ellesym{(}  \Ellent{t}  \Ellesym{)}} {\textsf{S}}{}{}{}\Elleprodnewline
\Elleprodline{|}{\Ellesym{h(}  \Ellent{t}  \Ellesym{)}} {\textsf{M}}{}{}{}\Elleprodnewline
\Elleprodline{|}{ \mathsf{G} \Ellent{s} }{}{}{}{}}

\newcommand{\ElleI}{
\Ellerulehead{\Phi  ,\ \Psi}{::=}{}\Elleprodnewline
\Ellefirstprodline{|}{ \cdot }{}{}{}{}\Elleprodnewline
\Elleprodline{|}{\Phi_{{\mathrm{1}}}  \Ellesym{,}  \Phi_{{\mathrm{2}}}}{}{}{}{}\Elleprodnewline
\Elleprodline{|}{\Ellemv{x}  \Ellesym{:}  \Ellent{X}}{}{}{}{}\Elleprodnewline
\Elleprodline{|}{\Ellesym{(}  \Phi  \Ellesym{)}} {\textsf{S}}{}{}{}}

\newcommand{\ElleG}{
\Ellerulehead{\Gamma  ,\ \Delta}{::=}{}\Elleprodnewline
\Ellefirstprodline{|}{ \cdot }{}{}{}{}\Elleprodnewline
\Elleprodline{|}{\Ellemv{x}  \Ellesym{:}  \Ellent{A}}{}{}{}{}\Elleprodnewline
\Elleprodline{|}{\Phi}{}{}{}{}\Elleprodnewline
\Elleprodline{|}{\Gamma_{{\mathrm{1}}}  \Ellesym{;}  \Gamma_{{\mathrm{2}}}}{}{}{}{}\Elleprodnewline
\Elleprodline{|}{\Ellesym{(}  \Gamma  \Ellesym{)}} {\textsf{S}}{}{}{}}

\newcommand{\Elleformula}{
\Ellerulehead{\Ellent{formula}}{::=}{}\Elleprodnewline
\Ellefirstprodline{|}{\Ellent{judgement}}{}{}{}{}\Elleprodnewline
\Elleprodline{|}{ \Ellent{formula_{{\mathrm{1}}}}  \quad  \Ellent{formula_{{\mathrm{2}}}} } {\textsf{M}}{}{}{}\Elleprodnewline
\Elleprodline{|}{\Ellent{formula_{{\mathrm{1}}}} \, ... \, \Ellent{formula_{\Ellemv{i}}}} {\textsf{M}}{}{}{}\Elleprodnewline
\Elleprodline{|}{ \Ellent{formula} } {\textsf{S}}{}{}{}\Elleprodnewline
\Elleprodline{|}{ \Ellemv{x}  \not\in \mathsf{FV}( \Ellent{s} ) }{}{}{}{}\Elleprodnewline
\Elleprodline{|}{ \Ellemv{x}  \not\in |  \Gamma ,  \Delta ,  \Psi  | }{}{}{}{}\Elleprodnewline
\Elleprodline{|}{ \Ellemv{x}  \not\in |  \Gamma ,  \Delta  | }{}{}{}{}}

\newcommand{\Elleterminals}{
\Ellerulehead{\Ellent{terminals}}{::=}{}\Elleprodnewline
\Ellefirstprodline{|}{ \otimes }{}{}{}{}\Elleprodnewline
\Elleprodline{|}{ \triangleright }{}{}{}{}\Elleprodnewline
\Elleprodline{|}{ \circop{e} }{}{}{}{}\Elleprodnewline
\Elleprodline{|}{ \circop{w} }{}{}{}{}\Elleprodnewline
\Elleprodline{|}{ \circop{c} }{}{}{}{}\Elleprodnewline
\Elleprodline{|}{ \rightharpoonup }{}{}{}{}\Elleprodnewline
\Elleprodline{|}{ \leftharpoonup }{}{}{}{}\Elleprodnewline
\Elleprodline{|}{ \multimap }{}{}{}{}\Elleprodnewline
\Elleprodline{|}{ \vdash_\mathcal{C} }{}{}{}{}\Elleprodnewline
\Elleprodline{|}{ \vdash_\mathcal{L} }{}{}{}{}\Elleprodnewline
\Elleprodline{|}{ \leadsto }{}{}{}{}\Elleprodnewline
\Elleprodline{|}{ \leadsto_\mathsf{c} }{}{}{}{}}

\newcommand{\ElleJtype}{
\Ellerulehead{\Ellent{Jtype}}{::=}{}\Elleprodnewline
\Ellefirstprodline{|}{\Phi  \vdash_\mathcal{C}  \Ellent{t}  \Ellesym{:}  \Ellent{X}}{}{}{}{}\Elleprodnewline
\Elleprodline{|}{\Gamma  \vdash_\mathcal{L}  \Ellent{s}  \Ellesym{:}  \Ellent{A}}{}{}{}{}}

\newcommand{\Ellejudgement}{
\Ellerulehead{\Ellent{judgement}}{::=}{}\Elleprodnewline
\Ellefirstprodline{|}{\Ellent{Jtype}}{}{}{}{}}

\newcommand{\ElleuserXXsyntax}{
\Ellerulehead{\Ellent{user\_syntax}}{::=}{}\Elleprodnewline
\Ellefirstprodline{|}{\Ellemv{vars}}{}{}{}{}\Elleprodnewline
\Elleprodline{|}{\Ellemv{ivar}}{}{}{}{}\Elleprodnewline
\Elleprodline{|}{\Ellemv{const}}{}{}{}{}\Elleprodnewline
\Elleprodline{|}{\Ellent{A}}{}{}{}{}\Elleprodnewline
\Elleprodline{|}{\Ellent{W}}{}{}{}{}\Elleprodnewline
\Elleprodline{|}{\Ellent{T}}{}{}{}{}\Elleprodnewline
\Elleprodline{|}{\Ellent{p}}{}{}{}{}\Elleprodnewline
\Elleprodline{|}{\Ellent{s}}{}{}{}{}\Elleprodnewline
\Elleprodline{|}{\Ellent{t}}{}{}{}{}\Elleprodnewline
\Elleprodline{|}{\Phi}{}{}{}{}\Elleprodnewline
\Elleprodline{|}{\Gamma}{}{}{}{}\Elleprodnewline
\Elleprodline{|}{\Ellent{formula}}{}{}{}{}\Elleprodnewline
\Elleprodline{|}{\Ellent{terminals}}{}{}{}{}}

\newcommand{\Ellegrammar}{\Ellegrammartabular{
\ElleA\Elleinterrule
\ElleW\Elleinterrule
\ElleT\Elleinterrule
\Ellep\Elleinterrule
\Elles\Elleinterrule
\Ellet\Elleinterrule
\ElleI\Elleinterrule
\ElleG\Elleinterrule
\Elleformula\Elleinterrule
\Elleterminals\Elleinterrule
\ElleJtype\Elleinterrule
\Ellejudgement\Elleinterrule
\ElleuserXXsyntax\Elleafterlastrule
}}

% defnss
% defns Jtype
%% defn tty
\newcommand{\ElledruleTXXaxName}[0]{\Elledrulename{T\_ax}}
\newcommand{\ElledruleTXXax}[1]{\Elledrule[#1]{%
}{
\Ellemv{x}  \Ellesym{:}  \Ellent{X}  \vdash_\mathcal{C}  \Ellemv{x}  \Ellesym{:}  \Ellent{X}}{%
{\ElledruleTXXaxName}{}%
}}


\newcommand{\ElledruleTXXunitLName}[0]{\Elledrulename{T\_unitL}}
\newcommand{\ElledruleTXXunitL}[1]{\Elledrule[#1]{%
\Ellepremise{\Phi  \Ellesym{,}  \Psi  \vdash_\mathcal{C}  \Ellent{t}  \Ellesym{:}  \Ellent{X}}%
}{
\Phi  \Ellesym{,}  \Ellemv{x}  \Ellesym{:}   \mathsf{Unit}   \Ellesym{,}  \Psi  \vdash_\mathcal{C}   \mathsf{let}\, \Ellemv{x}  :   \mathsf{Unit}  \,\mathsf{be}\,  \mathsf{triv}  \,\mathsf{in}\, \Ellent{t}   \Ellesym{:}  \Ellent{X}}{%
{\ElledruleTXXunitLName}{}%
}}


\newcommand{\ElledruleTXXunitRName}[0]{\Elledrulename{T\_unitR}}
\newcommand{\ElledruleTXXunitR}[1]{\Elledrule[#1]{%
}{
 \cdot   \vdash_\mathcal{C}   \mathsf{triv}   \Ellesym{:}   \mathsf{Unit} }{%
{\ElledruleTXXunitRName}{}%
}}


\newcommand{\ElledruleTXXtenLName}[0]{\Elledrulename{T\_tenL}}
\newcommand{\ElledruleTXXtenL}[1]{\Elledrule[#1]{%
\Ellepremise{\Phi  \Ellesym{,}  \Ellemv{x}  \Ellesym{:}  \Ellent{X}  \Ellesym{,}  \Ellemv{y}  \Ellesym{:}  \Ellent{Y}  \Ellesym{,}  \Psi  \vdash_\mathcal{C}  \Ellent{t}  \Ellesym{:}  \Ellent{Z}}%
}{
\Phi  \Ellesym{,}  \Ellemv{z}  \Ellesym{:}  \Ellent{X}  \otimes  \Ellent{Y}  \Ellesym{,}  \Psi  \vdash_\mathcal{C}   \mathsf{let}\, \Ellemv{z}  :  \Ellent{X}  \otimes  \Ellent{Y} \,\mathsf{be}\, \Ellemv{x}  \otimes  \Ellemv{y} \,\mathsf{in}\, \Ellent{t}   \Ellesym{:}  \Ellent{Z}}{%
{\ElledruleTXXtenLName}{}%
}}


\newcommand{\ElledruleTXXtenRName}[0]{\Elledrulename{T\_tenR}}
\newcommand{\ElledruleTXXtenR}[1]{\Elledrule[#1]{%
\Ellepremise{ \Phi  \vdash_\mathcal{C}  \Ellent{t_{{\mathrm{1}}}}  \Ellesym{:}  \Ellent{X}  \quad  \Psi  \vdash_\mathcal{C}  \Ellent{t_{{\mathrm{2}}}}  \Ellesym{:}  \Ellent{Y} }%
}{
\Phi  \Ellesym{,}  \Psi  \vdash_\mathcal{C}  \Ellent{t_{{\mathrm{1}}}}  \otimes  \Ellent{t_{{\mathrm{2}}}}  \Ellesym{:}  \Ellent{X}  \otimes  \Ellent{Y}}{%
{\ElledruleTXXtenRName}{}%
}}


\newcommand{\ElledruleTXXimpLName}[0]{\Elledrulename{T\_impL}}
\newcommand{\ElledruleTXXimpL}[1]{\Elledrule[#1]{%
\Ellepremise{ \Phi  \vdash_\mathcal{C}  \Ellent{t_{{\mathrm{1}}}}  \Ellesym{:}  \Ellent{X}  \quad  \Psi_{{\mathrm{1}}}  \Ellesym{,}  \Ellemv{x}  \Ellesym{:}  \Ellent{Y}  \Ellesym{,}  \Psi_{{\mathrm{2}}}  \vdash_\mathcal{C}  \Ellent{t_{{\mathrm{2}}}}  \Ellesym{:}  \Ellent{Z} }%
}{
\Psi_{{\mathrm{1}}}  \Ellesym{,}  \Ellemv{y}  \Ellesym{:}  \Ellent{X}  \multimap  \Ellent{Y}  \Ellesym{,}  \Phi  \Ellesym{,}  \Psi_{{\mathrm{2}}}  \vdash_\mathcal{C}  \Ellesym{[}   \Ellemv{y}   \Ellent{t_{{\mathrm{1}}}}   \Ellesym{/}  \Ellemv{x}  \Ellesym{]}  \Ellent{t_{{\mathrm{2}}}}  \Ellesym{:}  \Ellent{Z}}{%
{\ElledruleTXXimpLName}{}%
}}


\newcommand{\ElledruleTXXimpRName}[0]{\Elledrulename{T\_impR}}
\newcommand{\ElledruleTXXimpR}[1]{\Elledrule[#1]{%
\Ellepremise{\Phi  \Ellesym{,}  \Ellemv{x}  \Ellesym{:}  \Ellent{X}  \Ellesym{,}  \Psi  \vdash_\mathcal{C}  \Ellent{t}  \Ellesym{:}  \Ellent{Y}}%
}{
\Phi  \Ellesym{,}  \Psi  \vdash_\mathcal{C}   \lambda  \Ellemv{x}  :  \Ellent{X} . \Ellent{t}   \Ellesym{:}  \Ellent{X}  \multimap  \Ellent{Y}}{%
{\ElledruleTXXimpRName}{}%
}}


\newcommand{\ElledruleTXXGrName}[0]{\Elledrulename{T\_Gr}}
\newcommand{\ElledruleTXXGr}[1]{\Elledrule[#1]{%
\Ellepremise{\Phi  \vdash_\mathcal{L}  \Ellent{s}  \Ellesym{:}  \Ellent{A}}%
}{
\Phi  \vdash_\mathcal{C}   \mathsf{G} \Ellent{s}   \Ellesym{:}   \mathsf{G} \Ellent{A} }{%
{\ElledruleTXXGrName}{}%
}}


\newcommand{\ElledruleTXXcutName}[0]{\Elledrulename{T\_cut}}
\newcommand{\ElledruleTXXcut}[1]{\Elledrule[#1]{%
\Ellepremise{ \Phi  \vdash_\mathcal{C}  \Ellent{t_{{\mathrm{1}}}}  \Ellesym{:}  \Ellent{X}  \quad  \Psi_{{\mathrm{1}}}  \Ellesym{,}  \Ellemv{x}  \Ellesym{:}  \Ellent{X}  \Ellesym{,}  \Psi_{{\mathrm{2}}}  \vdash_\mathcal{C}  \Ellent{t_{{\mathrm{2}}}}  \Ellesym{:}  \Ellent{Y} }%
}{
\Psi_{{\mathrm{1}}}  \Ellesym{,}  \Phi  \Ellesym{,}  \Psi_{{\mathrm{2}}}  \vdash_\mathcal{C}  \Ellesym{[}  \Ellent{t_{{\mathrm{1}}}}  \Ellesym{/}  \Ellemv{x}  \Ellesym{]}  \Ellent{t_{{\mathrm{2}}}}  \Ellesym{:}  \Ellent{Y}}{%
{\ElledruleTXXcutName}{}%
}}


\newcommand{\ElledruleTXXexName}[0]{\Elledrulename{T\_ex}}
\newcommand{\ElledruleTXXex}[1]{\Elledrule[#1]{%
\Ellepremise{\Phi  \Ellesym{,}  \Ellemv{x}  \Ellesym{:}  \Ellent{X}  \Ellesym{,}  \Ellemv{y}  \Ellesym{:}  \Ellent{Y}  \Ellesym{,}  \Psi  \vdash_\mathcal{C}  \Ellent{t}  \Ellesym{:}  \Ellent{Z}}%
}{
\Phi  \Ellesym{,}  \Ellemv{z}  \Ellesym{:}  \Ellent{Y}  \Ellesym{,}  \Ellemv{w}  \Ellesym{:}  \Ellent{X}  \Ellesym{,}  \Psi  \vdash_\mathcal{C}   \mathsf{ex}\, \Ellemv{w} , \Ellemv{z} \,\mathsf{with}\, \Ellemv{x} , \Ellemv{y} \,\mathsf{in}\, \Ellent{t}   \Ellesym{:}  \Ellent{Z}}{%
{\ElledruleTXXexName}{}%
}}

\newcommand{\Elledefntty}[1]{\begin{Elledefnblock}[#1]{$\Phi  \vdash_\mathcal{C}  \Ellent{t}  \Ellesym{:}  \Ellent{X}$}{}
\Elleusedrule{\ElledruleTXXax{}}
\Elleusedrule{\ElledruleTXXunitL{}}
\Elleusedrule{\ElledruleTXXunitR{}}
\Elleusedrule{\ElledruleTXXtenL{}}
\Elleusedrule{\ElledruleTXXtenR{}}
\Elleusedrule{\ElledruleTXXimpL{}}
\Elleusedrule{\ElledruleTXXimpR{}}
\Elleusedrule{\ElledruleTXXGr{}}
\Elleusedrule{\ElledruleTXXcut{}}
\Elleusedrule{\ElledruleTXXex{}}
\end{Elledefnblock}}

%% defn sty
\newcommand{\ElledruleSXXaxName}[0]{\Elledrulename{S\_ax}}
\newcommand{\ElledruleSXXax}[1]{\Elledrule[#1]{%
}{
\Ellemv{x}  \Ellesym{:}  \Ellent{A}  \vdash_\mathcal{L}  \Ellemv{x}  \Ellesym{:}  \Ellent{A}}{%
{\ElledruleSXXaxName}{}%
}}


\newcommand{\ElledruleSXXunitLOneName}[0]{\Elledrulename{S\_unitL1}}
\newcommand{\ElledruleSXXunitLOne}[1]{\Elledrule[#1]{%
\Ellepremise{\Gamma  \Ellesym{;}  \Delta  \vdash_\mathcal{L}  \Ellent{s}  \Ellesym{:}  \Ellent{A}}%
}{
\Gamma  \Ellesym{;}  \Ellemv{x}  \Ellesym{:}   \mathsf{Unit}   \Ellesym{;}  \Delta  \vdash_\mathcal{L}   \mathsf{let}\, \Ellemv{x}  :   \mathsf{Unit}  \,\mathsf{be}\,  \mathsf{triv}  \,\mathsf{in}\, \Ellent{s}   \Ellesym{:}  \Ellent{A}}{%
{\ElledruleSXXunitLOneName}{}%
}}


\newcommand{\ElledruleSXXunitLTwoName}[0]{\Elledrulename{S\_unitL2}}
\newcommand{\ElledruleSXXunitLTwo}[1]{\Elledrule[#1]{%
\Ellepremise{\Gamma  \Ellesym{;}  \Delta  \vdash_\mathcal{L}  \Ellent{s}  \Ellesym{:}  \Ellent{A}}%
}{
\Gamma  \Ellesym{;}  \Ellemv{x}  \Ellesym{:}   \mathsf{Unit}   \Ellesym{;}  \Delta  \vdash_\mathcal{L}   \mathsf{let}\, \Ellemv{x}  :   \mathsf{Unit}  \,\mathsf{be}\,  \mathsf{triv}  \,\mathsf{in}\, \Ellent{s}   \Ellesym{:}  \Ellent{A}}{%
{\ElledruleSXXunitLTwoName}{}%
}}


\newcommand{\ElledruleSXXunitRName}[0]{\Elledrulename{S\_unitR}}
\newcommand{\ElledruleSXXunitR}[1]{\Elledrule[#1]{%
}{
 \cdot   \vdash_\mathcal{L}   \mathsf{triv}   \Ellesym{:}   \mathsf{Unit} }{%
{\ElledruleSXXunitRName}{}%
}}


\newcommand{\ElledruleSXXexName}[0]{\Elledrulename{S\_ex}}
\newcommand{\ElledruleSXXex}[1]{\Elledrule[#1]{%
\Ellepremise{\Gamma  \Ellesym{;}  \Ellemv{x}  \Ellesym{:}  \Ellent{X}  \Ellesym{;}  \Ellemv{y}  \Ellesym{:}  \Ellent{Y}  \Ellesym{;}  \Delta  \vdash_\mathcal{L}  \Ellent{s}  \Ellesym{:}  \Ellent{A}}%
}{
\Gamma  \Ellesym{;}  \Ellemv{z}  \Ellesym{:}  \Ellent{Y}  \Ellesym{;}  \Ellemv{w}  \Ellesym{:}  \Ellent{X}  \Ellesym{;}  \Delta  \vdash_\mathcal{L}   \mathsf{ex}\, \Ellemv{w} , \Ellemv{z} \,\mathsf{with}\, \Ellemv{x} , \Ellemv{y} \,\mathsf{in}\, \Ellent{s}   \Ellesym{:}  \Ellent{A}}{%
{\ElledruleSXXexName}{}%
}}


\newcommand{\ElledruleSXXtenLOneName}[0]{\Elledrulename{S\_tenL1}}
\newcommand{\ElledruleSXXtenLOne}[1]{\Elledrule[#1]{%
\Ellepremise{\Gamma  \Ellesym{;}  \Ellemv{x}  \Ellesym{:}  \Ellent{X}  \Ellesym{;}  \Ellemv{y}  \Ellesym{:}  \Ellent{Y}  \Ellesym{;}  \Delta  \vdash_\mathcal{L}  \Ellent{s}  \Ellesym{:}  \Ellent{A}}%
}{
\Gamma  \Ellesym{;}  \Ellemv{z}  \Ellesym{:}  \Ellent{X}  \otimes  \Ellent{Y}  \Ellesym{;}  \Delta  \vdash_\mathcal{L}   \mathsf{let}\, \Ellemv{z}  :  \Ellent{X}  \otimes  \Ellent{Y} \,\mathsf{be}\, \Ellemv{x}  \otimes  \Ellemv{y} \,\mathsf{in}\, \Ellent{s}   \Ellesym{:}  \Ellent{A}}{%
{\ElledruleSXXtenLOneName}{}%
}}


\newcommand{\ElledruleSXXtenLTwoName}[0]{\Elledrulename{S\_tenL2}}
\newcommand{\ElledruleSXXtenLTwo}[1]{\Elledrule[#1]{%
\Ellepremise{\Gamma  \Ellesym{;}  \Ellemv{x}  \Ellesym{:}  \Ellent{A}  \Ellesym{;}  \Ellemv{y}  \Ellesym{:}  \Ellent{B}  \Ellesym{;}  \Delta  \vdash_\mathcal{L}  \Ellent{s}  \Ellesym{:}  \Ellent{C}}%
}{
\Gamma  \Ellesym{;}  \Ellemv{z}  \Ellesym{:}  \Ellent{A}  \triangleright  \Ellent{B}  \Ellesym{;}  \Delta  \vdash_\mathcal{L}   \mathsf{let}\, \Ellemv{z}  :  \Ellent{A}  \triangleright  \Ellent{B} \,\mathsf{be}\, \Ellemv{x}  \triangleright  \Ellemv{y} \,\mathsf{in}\, \Ellent{s}   \Ellesym{:}  \Ellent{C}}{%
{\ElledruleSXXtenLTwoName}{}%
}}


\newcommand{\ElledruleSXXtenRName}[0]{\Elledrulename{S\_tenR}}
\newcommand{\ElledruleSXXtenR}[1]{\Elledrule[#1]{%
\Ellepremise{ \Gamma  \vdash_\mathcal{L}  \Ellent{s_{{\mathrm{1}}}}  \Ellesym{:}  \Ellent{A}  \quad  \Delta  \vdash_\mathcal{L}  \Ellent{s_{{\mathrm{2}}}}  \Ellesym{:}  \Ellent{B} }%
}{
\Gamma  \Ellesym{;}  \Delta  \vdash_\mathcal{L}  \Ellent{s_{{\mathrm{1}}}}  \triangleright  \Ellent{s_{{\mathrm{2}}}}  \Ellesym{:}  \Ellent{A}  \triangleright  \Ellent{B}}{%
{\ElledruleSXXtenRName}{}%
}}


\newcommand{\ElledruleSXXimpLName}[0]{\Elledrulename{S\_impL}}
\newcommand{\ElledruleSXXimpL}[1]{\Elledrule[#1]{%
\Ellepremise{ \Phi  \vdash_\mathcal{C}  \Ellent{t}  \Ellesym{:}  \Ellent{X}  \quad  \Gamma  \Ellesym{;}  \Ellemv{x}  \Ellesym{:}  \Ellent{Y}  \Ellesym{;}  \Delta  \vdash_\mathcal{L}  \Ellent{s}  \Ellesym{:}  \Ellent{A} }%
}{
\Gamma  \Ellesym{;}  \Ellemv{y}  \Ellesym{:}  \Ellent{X}  \multimap  \Ellent{Y}  \Ellesym{;}  \Phi  \Ellesym{;}  \Delta  \vdash_\mathcal{L}  \Ellesym{[}   \Ellemv{y}   \Ellent{t}   \Ellesym{/}  \Ellemv{x}  \Ellesym{]}  \Ellent{s}  \Ellesym{:}  \Ellent{A}}{%
{\ElledruleSXXimpLName}{}%
}}


\newcommand{\ElledruleSXXimprLName}[0]{\Elledrulename{S\_imprL}}
\newcommand{\ElledruleSXXimprL}[1]{\Elledrule[#1]{%
\Ellepremise{ \Gamma  \vdash_\mathcal{L}  \Ellent{s_{{\mathrm{1}}}}  \Ellesym{:}  \Ellent{A}  \quad  \Delta_{{\mathrm{1}}}  \Ellesym{;}  \Ellemv{x}  \Ellesym{:}  \Ellent{B}  \Ellesym{;}  \Delta_{{\mathrm{2}}}  \vdash_\mathcal{L}  \Ellent{s_{{\mathrm{2}}}}  \Ellesym{:}  \Ellent{C} }%
}{
\Delta_{{\mathrm{1}}}  \Ellesym{;}  \Ellemv{y}  \Ellesym{:}  \Ellent{A}  \rightharpoonup  \Ellent{B}  \Ellesym{;}  \Gamma  \Ellesym{;}  \Delta_{{\mathrm{2}}}  \vdash_\mathcal{L}  \Ellesym{[}   \mathsf{app}_r\, \Ellemv{y} \, \Ellent{s_{{\mathrm{1}}}}   \Ellesym{/}  \Ellemv{x}  \Ellesym{]}  \Ellent{s_{{\mathrm{2}}}}  \Ellesym{:}  \Ellent{C}}{%
{\ElledruleSXXimprLName}{}%
}}


\newcommand{\ElledruleSXXimprRName}[0]{\Elledrulename{S\_imprR}}
\newcommand{\ElledruleSXXimprR}[1]{\Elledrule[#1]{%
\Ellepremise{\Gamma  \Ellesym{;}  \Ellemv{x}  \Ellesym{:}  \Ellent{A}  \vdash_\mathcal{L}  \Ellent{s}  \Ellesym{:}  \Ellent{B}}%
}{
\Gamma  \vdash_\mathcal{L}   \lambda_r  \Ellemv{x}  :  \Ellent{A} . \Ellent{s}   \Ellesym{:}  \Ellent{A}  \rightharpoonup  \Ellent{B}}{%
{\ElledruleSXXimprRName}{}%
}}


\newcommand{\ElledruleSXXimplLName}[0]{\Elledrulename{S\_implL}}
\newcommand{\ElledruleSXXimplL}[1]{\Elledrule[#1]{%
\Ellepremise{ \Gamma  \vdash_\mathcal{L}  \Ellent{s_{{\mathrm{1}}}}  \Ellesym{:}  \Ellent{A}  \quad  \Delta_{{\mathrm{1}}}  \Ellesym{;}  \Ellemv{x}  \Ellesym{:}  \Ellent{B}  \Ellesym{;}  \Delta_{{\mathrm{2}}}  \vdash_\mathcal{L}  \Ellent{s_{{\mathrm{2}}}}  \Ellesym{:}  \Ellent{C} }%
}{
\Delta_{{\mathrm{1}}}  \Ellesym{;}  \Gamma  \Ellesym{;}  \Ellemv{y}  \Ellesym{:}  \Ellent{B}  \leftharpoonup  \Ellent{A}  \Ellesym{;}  \Delta_{{\mathrm{2}}}  \vdash_\mathcal{L}  \Ellesym{[}   \mathsf{app}_l\, \Ellemv{y} \, \Ellent{s_{{\mathrm{1}}}}   \Ellesym{/}  \Ellemv{x}  \Ellesym{]}  \Ellent{s_{{\mathrm{2}}}}  \Ellesym{:}  \Ellent{C}}{%
{\ElledruleSXXimplLName}{}%
}}


\newcommand{\ElledruleSXXimplRName}[0]{\Elledrulename{S\_implR}}
\newcommand{\ElledruleSXXimplR}[1]{\Elledrule[#1]{%
\Ellepremise{\Ellemv{x}  \Ellesym{:}  \Ellent{A}  \Ellesym{;}  \Gamma  \vdash_\mathcal{L}  \Ellent{s}  \Ellesym{:}  \Ellent{B}}%
}{
\Gamma  \vdash_\mathcal{L}   \lambda_l  \Ellemv{x}  :  \Ellent{A} . \Ellent{s}   \Ellesym{:}  \Ellent{B}  \leftharpoonup  \Ellent{A}}{%
{\ElledruleSXXimplRName}{}%
}}


\newcommand{\ElledruleSXXFlName}[0]{\Elledrulename{S\_Fl}}
\newcommand{\ElledruleSXXFl}[1]{\Elledrule[#1]{%
\Ellepremise{\Gamma  \Ellesym{;}  \Ellemv{x}  \Ellesym{:}  \Ellent{X}  \Ellesym{;}  \Delta  \vdash_\mathcal{L}  \Ellent{s}  \Ellesym{:}  \Ellent{A}}%
}{
\Gamma  \Ellesym{;}  \Ellemv{y}  \Ellesym{:}   \mathsf{F} \Ellent{X}   \Ellesym{;}  \Delta  \vdash_\mathcal{L}   \mathsf{let}\, \Ellemv{y}  :   \mathsf{F} \Ellent{X}  \,\mathsf{be}\,  \mathsf{F}\, \Ellemv{x}  \,\mathsf{in}\, \Ellent{s}   \Ellesym{:}  \Ellent{A}}{%
{\ElledruleSXXFlName}{}%
}}


\newcommand{\ElledruleSXXFrName}[0]{\Elledrulename{S\_Fr}}
\newcommand{\ElledruleSXXFr}[1]{\Elledrule[#1]{%
\Ellepremise{\Phi  \vdash_\mathcal{C}  \Ellent{t}  \Ellesym{:}  \Ellent{X}}%
}{
\Phi  \vdash_\mathcal{L}   \mathsf{F} \Ellent{t}   \Ellesym{:}   \mathsf{F} \Ellent{X} }{%
{\ElledruleSXXFrName}{}%
}}


\newcommand{\ElledruleSXXGlName}[0]{\Elledrulename{S\_Gl}}
\newcommand{\ElledruleSXXGl}[1]{\Elledrule[#1]{%
\Ellepremise{\Gamma  \Ellesym{;}  \Ellemv{x}  \Ellesym{:}  \Ellent{A}  \Ellesym{;}  \Delta  \vdash_\mathcal{L}  \Ellent{s}  \Ellesym{:}  \Ellent{B}}%
}{
\Gamma  \Ellesym{;}  \Ellemv{y}  \Ellesym{:}   \mathsf{G} \Ellent{A}   \Ellesym{;}  \Delta  \vdash_\mathcal{L}   \mathsf{let}\, \Ellemv{y}  :   \mathsf{G} \Ellent{A}  \,\mathsf{be}\,  \mathsf{G}\, \Ellemv{x}  \,\mathsf{in}\, \Ellent{s}   \Ellesym{:}  \Ellent{B}}{%
{\ElledruleSXXGlName}{}%
}}


\newcommand{\ElledruleSXXcutOneName}[0]{\Elledrulename{S\_cut1}}
\newcommand{\ElledruleSXXcutOne}[1]{\Elledrule[#1]{%
\Ellepremise{ \Phi  \vdash_\mathcal{C}  \Ellent{t}  \Ellesym{:}  \Ellent{X}  \quad  \Gamma_{{\mathrm{1}}}  \Ellesym{;}  \Ellemv{x}  \Ellesym{:}  \Ellent{X}  \Ellesym{;}  \Gamma_{{\mathrm{2}}}  \vdash_\mathcal{L}  \Ellent{s}  \Ellesym{:}  \Ellent{A} }%
}{
\Gamma_{{\mathrm{1}}}  \Ellesym{;}  \Phi  \Ellesym{;}  \Gamma_{{\mathrm{1}}}  \vdash_\mathcal{L}  \Ellesym{[}  \Ellent{t}  \Ellesym{/}  \Ellemv{x}  \Ellesym{]}  \Ellent{s}  \Ellesym{:}  \Ellent{A}}{%
{\ElledruleSXXcutOneName}{}%
}}


\newcommand{\ElledruleSXXcutTwoName}[0]{\Elledrulename{S\_cut2}}
\newcommand{\ElledruleSXXcutTwo}[1]{\Elledrule[#1]{%
\Ellepremise{ \Gamma  \vdash_\mathcal{L}  \Ellent{s_{{\mathrm{1}}}}  \Ellesym{:}  \Ellent{A}  \quad  \Delta_{{\mathrm{1}}}  \Ellesym{;}  \Ellemv{x}  \Ellesym{:}  \Ellent{A}  \Ellesym{;}  \Delta_{{\mathrm{2}}}  \vdash_\mathcal{L}  \Ellent{s_{{\mathrm{2}}}}  \Ellesym{:}  \Ellent{B} }%
}{
\Delta_{{\mathrm{1}}}  \Ellesym{;}  \Gamma  \Ellesym{;}  \Delta_{{\mathrm{2}}}  \vdash_\mathcal{L}  \Ellesym{[}  \Ellent{s_{{\mathrm{1}}}}  \Ellesym{/}  \Ellemv{x}  \Ellesym{]}  \Ellent{s_{{\mathrm{2}}}}  \Ellesym{:}  \Ellent{B}}{%
{\ElledruleSXXcutTwoName}{}%
}}

\newcommand{\Elledefnsty}[1]{\begin{Elledefnblock}[#1]{$\Gamma  \vdash_\mathcal{L}  \Ellent{s}  \Ellesym{:}  \Ellent{A}$}{}
\Elleusedrule{\ElledruleSXXax{}}
\Elleusedrule{\ElledruleSXXunitLOne{}}
\Elleusedrule{\ElledruleSXXunitLTwo{}}
\Elleusedrule{\ElledruleSXXunitR{}}
\Elleusedrule{\ElledruleSXXex{}}
\Elleusedrule{\ElledruleSXXtenLOne{}}
\Elleusedrule{\ElledruleSXXtenLTwo{}}
\Elleusedrule{\ElledruleSXXtenR{}}
\Elleusedrule{\ElledruleSXXimpL{}}
\Elleusedrule{\ElledruleSXXimprL{}}
\Elleusedrule{\ElledruleSXXimprR{}}
\Elleusedrule{\ElledruleSXXimplL{}}
\Elleusedrule{\ElledruleSXXimplR{}}
\Elleusedrule{\ElledruleSXXFl{}}
\Elleusedrule{\ElledruleSXXFr{}}
\Elleusedrule{\ElledruleSXXGl{}}
\Elleusedrule{\ElledruleSXXcutOne{}}
\Elleusedrule{\ElledruleSXXcutTwo{}}
\end{Elledefnblock}}


\newcommand{\ElledefnsJtype}{
\Elledefntty{}\Elledefnsty{}}

\newcommand{\Elledefnss}{
\ElledefnsJtype
}

\newcommand{\Elleall}{\Ellemetavars\\[0pt]
\Ellegrammar\\[5.0mm]
\Elledefnss}



\title{On Linear Based Intuitionistic Substructural Logics}
\author[1]{Harley Eades III}
\author[2]{Jiaming Jiang}
\affil[1]{Computer Science, Augusta University, Augusta, Georgia, USA\\
  \texttt{heades@augusta.edu}}
\affil[2]{Computer Science, North Carolina State University, Raleigh, North Carolina, USA\\
  \texttt{jjiang13@ncsu.edu}}

\Copyright{Harley E. Open and Jiaming J. Access}

\subjclass{TODO}
\keywords{TODO}

\begin{document}

\maketitle 

\begin{abstract}
  TODO
\end{abstract}

\section{Introduction}
\label{sec:introduction}
% \input{introduction-ott}
% section introduction (end)

\section{Main Ideas}
\label{sec:main_ideas}

% section main_ideas (end)

\section{Categorical Models}
\label{sec:categorical_models}
\begin{definition}
\label{def:mc}
  A \textbf{monoidal category} $(\cat{M},\otimes,I,\alpha,\lambda,\rho)$ is a category $\cat{M}$
  consists of
  \begin{itemize}
  \item a bifunctor $\otimes:\cat{M}\times\cat{M}\rightarrow\cat{M}$, called the tensor product;
  \item an object $I$, called the unit object;
  \item three natural isomorphisms $\alpha$, $\lambda$, and $\rho$ with components
        $$\alpha_{A,B,C}:(A\otimes B)\otimes C\rightarrow A\otimes(B\otimes C)$$
        $$\lambda_A:I\otimes A\rightarrow A$$
        $$\rho_A:A\otimes I\rightarrow A$$
        where $\alpha$ is called associator, $\lambda$ is left unitor, and $\rho$ is right
        unitor,
  \end{itemize}
  such that the following diagrams commute for any objects $A$, $B$, $C$ in $\cat{M}$:
  \begin{mathpar}
  \bfig
    \square/`->`->`->/<2100,400>[
      ((A\otimes B)\otimes C)\otimes D`
      A\otimes((B\otimes C)\otimes D)`
      (A\otimes B)\otimes(C\otimes D)`
      A\otimes(B\otimes(C\otimes D));
      `
      \alpha_{A\otimes B,C,D}`
      id_A\otimes\alpha_{B,C,D}`
      \alpha_{A,B,C\otimes D}]
    \morphism(0,400)<1100,0>[
      ((A\otimes B)\otimes C)\otimes D`
      (A\otimes(B\otimes C))\otimes D;
      \alpha_{A,B,C}\otimes id_D]
    \morphism(1100,400)<1000,0>[
      (A\otimes(B\otimes C))\otimes D`
      A\otimes((B\otimes C)\otimes D);
      \alpha_{A,B\otimes C,D}]
  \efig
  \and
  \bfig
    \Vtriangle<400,400>[
      (A\otimes I)\otimes B`
      A\otimes(I\otimes B)`
      A\otimes B;
      \alpha_{A,I,B}`
      \rho_A\otimes id_B`
      id_A\otimes\lambda_B]
  \efig
  \end{mathpar}
\end{definition}

\begin{definition}
  A \textbf{Lambek category} (or a \textbf{biclosed monoidal category}) is a monoidal category
  $(\cat{M},\otimes,I,\alpha,\lambda,\rho)$ equipped with two bifunctors
  $\rightharpoonup:\cat{M}^{op}\times\cat{M}\rightarrow\cat{M}$ and
  $\leftharpoonup:\cat{M}\times\cat{M}^{op}\rightarrow\cat{M}$ that are both right adjoint to
  the tensor product. That is, the following natural bijections hold:
  \begin{center}
  \begin{math}
  \begin{array}{lll}
    \Hom{L}{X\otimes A}{B}\cong\Hom{L}{X}{A\lto B} & \quad\quad\quad\quad & 
    \Hom{L}{A\otimes X}{B}\cong\Hom{L}{X}{B\rto A}
  \end{array}
  \end{math}
  \end{center}
\end{definition}

\begin{definition}
  A \textbf{symmetric monoidal category} (SMCC) is a monoidal category
  $(\cat{M},\otimes,I,\alpha,\lambda,\rho)$ together with a natural transformation with
  components $\e{A,B}:A\otimes B\rightarrow B\otimes A$, called \textbf{exchange}, such that the
  following diagrams commute:
  \begin{mathpar}
  \bfig
    \Vtriangle<300,400>[A\otimes I`I\otimes A`A;\e{A,I}`\rho_A`\lambda_A]
  \efig
  \and
  \bfig
    \Vtriangle/=`->`<-/<300,400>[
      A\otimes B`A\otimes B`B\otimes A;
      id_{A\otimes B}`\e{A,B}`\e{B,A}]
  \efig
  \and
  \bfig
    \hSquares/->`->`->``->`->`->/<400>[
      (A\otimes B)\otimes C`A\otimes(B\otimes C)`(B\otimes C)\otimes A`
      (B\otimes A)\otimes C`B\otimes(A\otimes C)`B\otimes(C\otimes A);
      \alpha_{A,B,C}`\e{A,B\otimes C}`\e{A,B}\otimes id_C``
      \alpha_{B,A,C}`\alpha_{B,A,C}`id_B\otimes\e{A,C}]
  \efig
  \end{mathpar}
\end{definition}

\begin{definition}
  A \textbf{symmetric monoidal closed category} $(\cat{M},\otimes,I,\alpha,\lambda,\rho)$ is a
  symmetric monoidal category equipped with a bifunctor
  $\limp:\cat{M}^{op}\times\cat{M}\rightarrow\cat{M}$ that is right adjoint to the tensor
  product. That is, the following natural bijection
  $\Hom{\cat{M}}{X\otimes A}{B}\cong\Hom{\cat{M}}{X}{A\limp B}$ holds.
\end{definition}

\begin{lemma}
  \label{lemma:internal-homs-collapse}
  Let $A$ and $B$ be two objects in a Lambek category with the exchange natural transformation.
  Then $(A \lto B) \cong (B \rto A)$.
\end{lemma}
\begin{proof}
  First, notice that for any object $C$ we have
  \begin{center}
  \begin{math}
  \small
  \begin{array}{lllll}
    Hom[C,A\lto B]
    & \cong & Hom[C\otimes A,B] & \cat{L}\text{ is a Lambek category}\\
    & \cong & Hom[A\otimes C,B] & \text{By the exchange }\e{C,A}\\
    & \cong & Hom[C,B\rto A]    & \cat{L}\text{ is a Lambek category}
  \end{array}
  \end{math}
  \end{center}  
  Thus, $A\lto B\cong B\rto A$ by the Yoneda lemma.
\end{proof}
\begin{corollary}
  \label{corollary:LC-with-ex-mc}
  A Lambek category with exchange is symmetric monoidal closed.
\end{corollary}

\begin{definition}
  Let $(\cat{M},\otimes,I,\alpha,\lambda,\rho)$ and
  $(\cat{M'},\otimes',I',\alpha',\lambda',\rho')$ be monoidal categories. A \textbf{monoidal
  functor} $(F,\m{})$ from $\cat{M}$ to $\cat{M'}$ is a functor $F:\cat{M}\rightarrow\cat{M'}$
  together with a morphism $\m{I}:I'\rightarrow F(I)$ and a natural transformation
  $\m{A,B}:FA'\otimes FB'\rightarrow F(A\otimes B)$, such that the following diagrams commute
  for any objects $A$, $B$, and $C$ in $\cat{M}$:
  \begin{mathpar}
  \bfig
    \hSquares/->`->`->``->`->`->/<400>[
      (FA\otimes'FB)\otimes'FC`FA\otimes'(FB\otimes'FC)`FA\otimes'F(B\otimes C)`
      F(A\otimes B)\otimes'FC`F((A\otimes B)\otimes C)`F(A\otimes(B\otimes C));
      \alpha'_{FA,FB,FC}`id_{FA}\otimes'\m{A,B}`\m{A,B}\otimes'id_{FC}``
      \m{A,B\otimes C}`\m{A\otimes B,C}`F\alpha_{A,B,C}]
  \efig
  \and
  \bfig
    \square/->`->`<-`->/<600,400>[
      I'\otimes'FA`FA`FI\otimes'FA`F(I\otimes A);
      \lambda'_{FA}`\m{I}\otimes id_{FA}`F\lambda_A`\m{I,A}]
  \efig
  \and
  \bfig
    \square/->`->`<-`->/<600,400>[
      FA\otimes'I'`FA`FA\otimes'FI`F(A\otimes I);
      \rho'_{FA}`\id_{FA}\otimes\m{I}`F\rho_A`\m{A,I}]
  \efig
  \end{mathpar}
\end{definition}

\begin{definition}
  Let $(\cat{M},\otimes,I,\alpha,\lambda,\rho)$ and
  $(\cat{M'},\otimes',I',\alpha',\lambda',\rho')$ be monoidal categories. A \textbf{symmetric
  monoidal functor} $F:\cat{M}\rightarrow\cat{M'}$ is a monoidal functor $(F,\m{})$ that
  satisfies the following coherence diagram:
  \begin{mathpar}
  \bfig
    \square<700,400>[
      FA\otimes'FB`FB\otimes'FA`F(A\otimes B)`F(B\otimes A);
      \e{FA,FB}`\m{A,B}`\m{B,A}`F\e{A,B}]
  \efig
  \end{mathpar}
\end{definition}

\begin{definition}
  An \textbf{adjunction} between categories $\cat{C}$ and $\cat{D}$ consists of two functors
  $F:\cat{D}\rightarrow\cat{C}$, called the \textbf{left adjoint}, and
  $G:\cat{C}\rightarrow\cat{D}$, called the \textbf{right adjoint}, and two natural
  transformations $\eta:id_\cat{D}\rightarrow GF$, called the \textbf{unit}, and
  $\varepsilon:FG\rightarrow id_\cat{C}$, called the \textbf{counit}, such that the following
  diagrams commute for any object $A$ in $\cat{C}$ and $B$ in $\cat{D}$:
  \begin{mathpar}
  \bfig
    \Vtriangle/->`=`->/<400,400>[FB`FGFB`FB;F\eta_B``\varepsilon_{FB}]
  \efig
  \and
  \bfig
    \Vtriangle/->`=`->/<400,400>[GA`GFGA`GA;\eta_{GA}``G\varepsilon_A]
  \efig
  \end{mathpar}
\end{definition}

\begin{definition}
  Let $(F,\m{})$ and $(G,\n{})$ be monoidal functors from a monoidal category $\cat{M}$ to a
  monoidal category $\cat{M'}$. A \textbf{monoidal natural transformation} from $(F,\m{})$ to 
  $(G,\n{})$ is a natural transformation $\theta:(F,\m{})\rightarrow(G,\n{})$ such that the
  following diagrams commute for any objects $A$ and $B$ in $\cat{M}$:
  \begin{mathpar}
  \bfig
    \square<700,400>[
      FA\otimes'FB`F(A\otimes B)`GA\otimes'GB`G(A\otimes B);
      \m{A,B}`\theta_A\otimes'\theta_B`\theta_{A\otimes B}`\n{A,B}]
  \efig
  \and
  \bfig
    \Vtriangle/->`<-`<-/<400,400>[FI`GI`I';\theta_I`\m{I}`\n{I}]
  \efig
  \end{mathpar}
\end{definition}

\begin{definition}
  Let $(\cat{M},\otimes,I,\alpha,\lambda,\rho)$ and
  $(\cat{M'},\otimes',I',\alpha',\lambda',\rho')$ be monoidal categories,
  $F:\cat{M}\rightarrow\cat{M'}$ and $G:\cat{M}'\rightarrow\cat{M}$ be functors. The adjunction
  $F:\cat{M}\dashv\cat{M'}:G$ is a \textbf{monoidal adjunction} if $F$ and $G$ are monoidal
  functors, and the unit $\eta$ and the counit $\varepsilon$ are monoidal natural
  transformations.
\end{definition}

\begin{definition}
  A \textbf{SMCC-Lambek model} consists of
  \begin{itemize}
  \item a symmetric monoidal closed category $(\cat{C},\otimes,I,\alpha,\lambda,\rho)$;
  \item a Lambek category $(\cat{L},\otimes',I',\alpha',\lambda',\rho')$;
  \item a monoidal adjunction $F:\cat{C}\dashv\cat{L}:G$, where $F:\cat{C}\rightarrow\cat{L}$
        and $G:\cat{L}\rightarrow\cat{C}$ are monoidal functors.
  \end{itemize}
\end{definition}

Thus, in a SMCC-Lambek model, the following four diagrams commute because $\eta$ and
$\varepsilon$ are monoidal natural transformations:
\begin{mathpar}
\bfig
  \square/=`->`->`/<1600,400>[
    A\otimes B`A\otimes B`GFA\otimes GFB`GF(A\otimes B);
    id_{A\otimes B}`\eta_A\otimes\eta_B`\eta_{A\otimes B}`]
  \morphism<800,0>[GFA\otimes GFB`G(FA\otimes FB);\n{FA,FB}]
  \morphism(800,0)<800,0>[G(FA\otimes FB)`GF(A\otimes B);G\m{A,B}]
\efig
\and
\bfig
  \square/->`=`<-`->/<400,400>[I`GFI`I`GI';\eta_I``G\m{I}`\n{I'}]
\efig
\end{mathpar}
\begin{mathpar}
\bfig
  \square/`->`->`=/<1600,400>[
    FGA\otimes FGB`FG(A\otimes B)`A\otimes B`A\otimes B;
    `
    \varepsilon_A\otimes\varepsilon_B`\varepsilon_{A\otimes B}`]
  \morphism(0,400)<800,0>[FGA\otimes FGB`F(GA\otimes GB);\m{GA,GB}]
  \morphism(800,400)<800,0>[F(GA\otimes GB)`FG(A\otimes B);F\n{A,B}]
\efig
\and
\bfig
  \square/->`<-`=`<-/<400,400>[FGI'`I'`FI`I';\varepsilon_{I'}`F\n{I'}``\m{I}]
\efig
\end{mathpar}
And the following two diagrams commute because of the adjunction:
\begin{mathpar}
\bfig
  \Vtriangle/->`=`->/<400,400>[FA`FGFA`FA;F\eta_A``\varepsilon_{FA}]
\efig
\and
\bfig
  \Vtriangle/->`=`->/<400,400>[GB`GFGB`GB;\eta_{GX}``G\varepsilon_B]
\efig
\end{mathpar}

\begin{definition}
  Let $\cat{C}$ be a category. A \textbf{monad} on $\cat{C}$ consists of an endofunctor
  $T:\cat{C}\rightarrow\cat{C}$ together with two natural transformations
  $\eta:id_\cat{C}\rightarrow T$ and $\mu:T^2\rightarrow id_\cat{C}$, where $id_\cat{C}$
  is the identity functor on $\cat{C}$, such that the following diagrams commute:
  \begin{mathpar}
  \bfig
    \square<400,400>[T^3`T^2`T^2`T;T\mu`\mu_T`\mu`\mu]
  \efig
  \and
  \bfig
    \square<400,400>[T`T^2`T^2`T;\eta_T`T\eta`\mu`\mu]
    \morphism(0,400)/=/<400,-400>[T`T;]
  \efig
  \end{mathpar}
\end{definition}

\begin{lemma}
  \label{lem:monoidal-monad}
  The monad on the SMCC $\cat{C}$ in a SMCC-Lambek model is monoidal.
\end{lemma}
\begin{proof}
  We define the monad $T$ on the $\cat{C}$ in the adjunction of a SMCC-Lambek model as $T=GF$,
  and the two corresponding natural transformations $\eta:id_\cat{C}\rightarrow T$ and
  $\mu:T^2\rightarrow T$ are defined as
  $$\eta:id_\cat{C}\rightarrow GF$$
  $$\mu=GF\varepsilon_A=\varepsilon_{GFA}:GFGF\rightarrow GF$$
  where $\eta$ is the unit and $\mu$ is the counit in the adjunction $F:\cat{C}\dashv\cat{L}:G$,
  and $(F,\m{})$ and $(G,\n{})$ are monoidal functors. \\
  Thus, we have
  $$\q{A,B}=G\m{A,B}\circ\n{FA,FB}:TA\otimes TB\rightarrow T(A\otimes B)$$
  $$\q{I}=G\m{I}\circ\n{I'}:I\rightarrow TI$$
  The monad $T$ being monoidal means
  \begin{enumerate}
  \item $T$ is a monoidal functor i.e. the folllowing diagrams commute:
        \begin{mathpar}
        \bfig
          \hSquares/->`->`->``->`->`->/<400>[
            (TA\otimes TB)\otimes TC`TA\otimes(TB\otimes TC)`TA\otimes T(B\otimes C)`
            T(A\otimes B)\otimes TC`T((A\otimes B)\otimes C)`T(A\otimes(B\otimes C));
            \alpha_{TA,TB,TC}`id_{TA}\otimes\q{B,C}`\q{A,B}\otimes id_{TC}``
            \q{A,B\otimes C}`\q{A\otimes B,C}`T\alpha_{A,B,C}]
        \efig
        \and
        \bfig
          \square/->`->`<-`->/<600,400>[
            I\otimes TA`TA`TI\otimes TA`T(I\otimes A);
            \lambda_{TA}`\q{I}\otimes id_{TA}`T\lambda_A`\q{I,A}]
        \efig
        \and
        \bfig
          \square/->`->`<-`->/<600,400>[
            TA\otimes I`TA`TA\otimes TI`T(A\otimes I);
            \rho_{TA}`id_{TA}\otimes\q{I}`T\rho_A`\q{A,I}]
        \efig
        \end{mathpar}
        We write $GF$ instead of $T$ in the diagram chasings for clarity. \\
        By replacing $\q{}$ with its definition, the first diagram above commutes by the
        following diagram chasing, where the two hexagons commute because $G$ and $F$ are
        monoidal functors, and the two quadrilaterals commute by the naturality of $\n{}$.
        \begin{mathpar}
        \bfig
          \iiixiii/->`->`->``->```->`<-`->``/<1400,400>[
            (GFA\otimes GFB)\otimes GFC`GFA\otimes(GFB\otimes GFC)`GFA\otimes G(FB\otimes'FC)`
            G(FA\otimes'FB)\otimes GFC`G(FA\otimes'(FB\otimes'FC))`GFA\otimes GF(B\otimes C)`
            GF(A\otimes B)\otimes GFC`G((FA\otimes'FB)\otimes'FC)`G(FA\otimes'F(B\otimes C));
            \alpha_{GFA,GFB,GFC}`id_{GFA}\otimes\n{FB,FC}`\n{FA,FB}\otimes id_{GFC}``
            id_{GFA}\otimes G\m{B,C}```G\m{A,B}\otimes id_{GFC}`G\alpha'_{FA,FB,FC}`
            \n{FA,F(B\otimes C)}``]
          \morphism(2800,800)|m|<-1400,-400>[
            GFA\otimes G(FB\otimes'FC)`G(FA\otimes'(FB\otimes'FC));\n{FA,FB\otimes'FC}]
          \morphism(0,400)|m|<1400,-400>[
            G(FA\otimes'FB)\otimes GFC`G((FA\otimes'FB)\otimes'FC);\n{FA\otimes'FB,FC}]
          \morphism(1400,400)|m|<1400,-400>[
            G(FA\otimes'(FB\otimes'FC))`G(FA\otimes'F(B\otimes C));G(id_{FA}\otimes'\m{B,C})]
          \ptriangle(0,-400)|mlm|/`->`->/<1400,400>[
            GF(A\otimes B)\otimes GFC`G((FA\otimes'FB)\otimes'FC)`G(F(A\otimes B)\otimes'FC);
            `\n{F(A\otimes B),FC}`G(\m{A,B}\otimes id_{FC})]
          \morphism(0,-400)|b|<1400,0>[
            G(F(A\otimes B)\otimes'FC)`GF((A\otimes B)\otimes C);G\m{A\otimes B,C}]
          \dtriangle(1400,-400)|mrb|/`->`->/<1400,400>[
            G(FA\otimes'F(B\otimes C))`GF((A\otimes B)\otimes C)`GF(A\otimes(B\otimes C));
            `G\m{A,B\otimes C}`GF\alpha_{A,B,C}]
        \efig
        \end{mathpar}
        The first square above commutes by the following diagram chasing, in which the top
        quadrilateral commutes because $G$ is monoidal, the right quadrilateral commutes because
        $F$ is monoidal, and the left square commutes by the naturality of $\n{}$.
        \begin{mathpar}
        \bfig
          \ptriangle/->`->`/<1600,400>[
            I\otimes GFA`GFA`GI'\otimes GFA;\lambda_{GFA}`\n{I'}\otimes id_{GFA}`]
          \square(0,-400)|lmmb|<800,400>[
            GI'\otimes GFA`G(I'\otimes'FA)`GFI\otimes GFA`G(FI\otimes'FA);
            \n{I',FA}`G\m{I}\otimes id_{GFA}`G(\m{I}\otimes'id_{FA})`\n{FI,FA}]
          \morphism(800,0)|m|<800,400>[G(I'\otimes'FA)`GFA;G\lambda'_{FA}]
          \dtriangle(800,-400)/`<-`->/<800,800>[
            GFA`G(FI\otimes'FA)`GF(I\otimes A);
            `GF\lambda_A`G\m{I,A}]
        \efig
        \end{mathpar}
        Similarly, the second square above commutes by the following diagram chasing:
        \begin{mathpar}
        \bfig
          \ptriangle/->`->`/<1600,400>[
            GFA\otimes I`GFA`GFA\otimes GI';\rho_{GFA}`id_{GFA}\otimes\n{I'}`]
          \square(0,-400)|lmmb|<800,400>[
            GFA\otimes GI'`G(FA\otimes'I')`GFA\otimes GFI`G(FA\otimes'FI);
            \n{FA,I'}`id_{GFA}\otimes G\m{I}`G(id_{FA}\otimes\m{I})`\n{FA,FI}]
          \morphism(800,0)|m|<800,400>[G(FA\otimes'I')`GFA;G\rho'_{FA}]
          \dtriangle(800,-400)/`<-`->/<800,800>[
            GFA`G(FA\otimes'FI)`GF(A\otimes I);
            `GF\rho_A`G\m{A,I}]
        \efig
        \end{mathpar}
  \item $\eta$ is a monoidal natural transformation, i.e. the following diagrams commute. In
        fact, since $\eta$ is the unit of the monoidal adjunction, $\eta$ is monoidal and thus
        the following two diagrams commute.
        \begin{mathpar}
        \bfig
          \square/=`->`->`->/<600,400>[
            A\otimes B`A\otimes B`TA\otimes TB`T(A\otimes B);
            `\eta_A\otimes\eta_B`\eta_{A\otimes B}`\q{A,B}]
        \efig
        \and
        \bfig
          \Vtriangle/->`=`<-/<400,400>[I`TI`I;\eta_I``\q{I}]
        \efig
        \end{mathpar}
  \item $\mu$ is a monoidal natural transformation, i.e. the following diagrams commute. Since
        $\mu=\varepsilon_{GFA}$ and $\varepsilon$ is monoidal, so is $\mu$. Thus the following
        diagrams commute.
        \begin{mathpar}
        \bfig
          \square/`->`->`->/<1500,400>[
            T^2A\otimes T^2B`T^2(A\otimes B)`TA\otimes TB`T(A\otimes B);
            `\mu_A\otimes\mu_B`\mu_{A\otimes B}`\q{A,B}]
          \morphism(0,400)<800,0>[T^2A\otimes T^2B`T(TA\otimes TB);\q{TA,TB}]
          \morphism(800,400)<700,0>[T(TA\otimes TB)`T^2(A\otimes B);T\q{A,B}]
        \efig
        \and
        \bfig
          \square/->`<-`<-`<-/<400,400>[T^2I`TI`TI`I;\mu_I`T\q{I}`\q{I}`\q{I}]
        \efig
        \end{mathpar}
  \end{enumerate}
\end{proof}

However, the monad $T$ we get from the SMCC-Lambek model is not symmetric because the following
diagram does not commute:
\begin{mathpar}
\bfig
  \hSquares/->`->`->``->`->`->/<400>[
    GFA\otimes GFB`GFB\otimes GFA`G(FB\otimes'FA)`G(FA\otimes'FB)`GF(A\otimes B)`GF(B\otimes A);
    \e{GFA,GFB}`\n{FB,FA}`\n{FA,FB}``G\m{B,A}`G\m{A,B}`GF\e{A,B}]
\efig
\end{mathpar}

\begin{definition}
  \label{def:strong-monad}
  Let $(\cat{M},\otimes,I,\alpha,\lambda,\rho)$ be a monoidal category and $(T,\eta,\mu)$ be a
  monad on $\cat{M}$. $T$ is a \textbf{strong monad} if there is natural transformation $\tau$, 
  called the \textbf{tensorial strength}, with components
  $\tau_{A,B}:A\otimes TB\rightarrow T(A\otimes B)$ such that the following diagrams commute:
  \begin{mathpar}
  \bfig
    \Vtriangle<400,400>[I\otimes TA`T(I\otimes A)`TA;\tau_{I,A}`\lambda_{TA}`T\lambda_A]
  \efig
  \and
  \bfig
    \Vtriangle<400,400>[
      A\otimes B`A\otimes TB`T(A\otimes B);id_A\otimes\eta_B`\eta_{A\otimes B}`\tau_{A,B}]
  \efig
  \and
  \bfig
    \square/->`->`->`/<1800,400>[
      (A\otimes B)\otimes TC`T((A\otimes B)\otimes C)`
      A\otimes(B\otimes TC)`T(A\otimes(B\otimes C));
      \tau_{A\otimes B,C}`\alpha_{A,B,TC}`T\alpha_{A,B,C}`]
    \morphism<900,0>[A\otimes(B\otimes TC)`A\otimes T(B\otimes C);id_A\otimes\tau_{B,C}]
    \morphism(900,0)<900,0>[A\otimes T(B\otimes C)`T(A\otimes(B\otimes C));\tau_{A,B\otimes C}]
  \efig
  \and
  \bfig
    \square/`->`->`->/<1400,400>[
      A\otimes T^2B`T^2(A\otimes B)`A\otimes TB`T(A\otimes B);
      `id_A\otimes\mu_B`\mu_{A\otimes B}`\tau_{A,B}]
    \morphism(0,400)<700,0>[A\otimes T^2B`T(A\otimes TB);\tau_{A,TB}]
    \morphism(700,400)<700,0>[T(A\otimes TB)`T^2(A\otimes B);T\tau_{A,B}]
  \efig
  \end{mathpar}
\end{definition}

\begin{lemma}
  \label{lem:strong-monad}
  The monad on the SMCC in a SMCC-Lambek model is strong.
\end{lemma}
\begin{proof}
  Let $F:\cat{C}\vdash\cat{L}:G$ be a SMCC-Lambek model, where
  $(\cat{C},\otimes,I,\alpha,\lambda,\rho)$ is an SMCC,
  $(\cat{M},\otimes',I',\alpha',\lambda',\rho')$ is a Lambek category, and $(F,\m{})$ and
  $(G,\n{})$ are monoidal functors. Let $(T,\eta,\mu)$ be the monad on $\cat{C}$ where
  $T=GF$. We have proved that $T$ is monoidal with the natural transformation
  $\q{A,B}:TA\otimes TB\rightarrow T(A\otimes B)$ and the morphism $\q{I}:I\rightarrow TI$
  defined as in Lemma~\ref{lem:monoidal-monad}. \\
  We define the tensorial strength $\tau_{A,B}:A\otimes TB\rightarrow T(A\otimes B)$ as
  $\tau_{A,B}=\q{A,B}\circ\eta_A\otimes id_{TB}$. \\
  Since $\eta$ is a monoidal natural transformation, we have $\eta_I=G\m{I}\circ\n{I'}$.
  Therefore $\eta_I=\q{I}$. Thus the following diagram commutes because $T$ is monoidal,
  where the composition $\q{I,A}\circ\q{I}\otimes id_{TA}$ is the definition of $\tau_{I,A}$. So
  the first triangle in Defition~\ref{def:strong-monad} commutes.
  \begin{mathpar}
  \bfig
    \square/->`->`->`<-/<600,400>[
      I\otimes TA`TI\otimes TA`TA`T(I\otimes A);
      \q{I}\otimes id_{TA}`\lambda_{TA}`\q{I,A}`T\lambda_A]
  \efig
  \end{mathpar}
  Similarly, by using the definition of $\tau$, the the second triangle in the definition is
  equivalent to the following diagram, which commutes because $\eta$ is a monoidal natural
  transformation:
  \begin{mathpar}
  \bfig
    \square/->`->`->`<-/<600,400>[
      A\otimes B`A\otimes TB`T(A\otimes B)`TA\otimes TB;
      id_A\otimes\eta_B`\eta_{A\otimes B}`\eta_A\otimes id_{TB}`\q{A,B}]
    \morphism(0,400)|m|<600,-400>[A\otimes B`TA\otimes TB;\eta_A\otimes\eta_B]
  \efig
  \end{mathpar}
  The first pentagon in the definition commutes by the following diagram chasing, because
  $\eta$ are $\alpha$ natural transformations and $T$ is monoidal:
  \begin{mathpar}
  \bfig
    \qtriangle|amm|/->`->`<-/<1000,400>[
      (A\otimes B)\otimes TC`T(A\otimes B)\otimes TC`(TA\otimes TB)\otimes TC;
      \eta_{A\otimes B}\otimes id_{TC}`
      (\eta_A\otimes\eta_B)\otimes id_{TC}`
      \q{A,B}\otimes id_{TC}]
    \morphism(0,400)<0,-400>[(A\otimes B)\otimes TC`A\otimes(B\otimes TC);\alpha_{A,B,TC}]
    \morphism(1000,0)|m|<0,-400>[
      (TA\otimes TB)\otimes TC`TA\otimes(TB\otimes TC);\alpha_{TA,TB,TC}]
    \Dtriangle(0,-800)|lmm|/->`->`<-/<1000,400>[
      A\otimes(B\otimes TC)`TA\otimes(TB\otimes TC)`A\otimes(TB\otimes TC);
      id_A\otimes(\eta_B\otimes id_{TC})`
      \eta_A\otimes(\eta_B\otimes id_{TC})`
      \eta_A\otimes id_{TB\otimes TC}]
    \morphism(0,-800)|b|<1000,0>[
      A\otimes(TB\otimes TC)`A\otimes T(B\otimes C);id_A\otimes\q{B,C}]
    \qtriangle(1000,0)|amr|/->``->/<1000,400>[
      T(A\otimes B)\otimes TC`T((A\otimes B)\otimes C)`T(A\otimes(B\otimes C));
      \q{A\otimes B,C}``T\alpha_{A,B,C}]
    \morphism(2000,-800)<0,800>[
      TA\otimes T(B\otimes C)`T(A\otimes(B\otimes C));\q{A,B\otimes C}]
    \btriangle(1000,-800)|mmb|/`->`->/<1000,400>[
      TA\otimes(TB\otimes TC)`A\otimes T(B\otimes C)`TA\otimes T(B\otimes C);
      `id_{TA}\otimes\q{B,C}`\eta_A\otimes id_{T(B\otimes C)}]
  \efig
  \end{mathpar}
  The last diagram in the definition commtues by the following diagram chasing, because $T$ is a
  monad, $\q{}$ is a natural transformation, and $\mu$ is a monoidal natural transformation:
  \begin{mathpar}
  \bfig
    \ptriangle/->`->`/<700,400>[
      A\otimes T^2B`TA\otimes T^2B`A\otimes TB;\eta_A\otimes id_{T^2B}`id_A\otimes\mu_B`]
    \btriangle(0,-400)/->``->/<700,400>[
      A\otimes TB`TA\otimes TB`T(A\otimes B);\eta_A\otimes id_{TB}``\q{A,B}]
    \morphism(700,400)|m|<-700,-800>[TA\otimes T^2B`TA\otimes TB;id_{TA}\otimes\mu_B]
    \morphism(700,0)|m|<-700,-400>[TA\otimes T^2B`TA\otimes TB;id_{TA}\otimes\mu_B]
    \qtriangle(700,0)/->``->/<1800,400>[
      TA\otimes T^2B`T(A\otimes TB)`T(TA\otimes TB);\q{A,TB}``T(\eta_A\otimes id_{TB})]
    \btriangle(700,0)|mmm|/=`->`<-/<900,400>[
      TA\otimes T^2B`TA\otimes T^2B`T^2A\otimes T^2B;
      `T\eta_A\otimes id_{T^2B}`\mu_A\otimes id_{T^2B}]
    \morphism(1600,0)|m|<900,0>[T^2A\otimes T^2B`T(TA\otimes TB);\q{TA,TB}]
    \morphism(1600,0)|m|<-1600,-400>[T^2A\otimes T^2B`TA\otimes TB;\mu_A\otimes\mu_B]
    \dtriangle(700,-400)/`->`<-/<1800,400>[
      T(TA\otimes TB)`T(A\otimes B)`T^2(A\otimes B);`T\q{A,B}`\mu_{A\otimes B}]
  \efig
  \end{mathpar}
\end{proof}

\begin{definition}
  Let $(\cat{M},\otimes,I,\alpha,\lambda,\rho)$ be a symmetric monoidal category with exchange
  $\e{}$, and $(T,\eta,\mu)$ be a strong monad on $\cat{M}$. Then there is a \textbf{``twisted''
  tensorial strength} $\tau'_{A,B}:TA\otimes B\rightarrow T(A\otimes B)$ defined as
  $\tau'_{A,B}=T\e{}\circ\tau_{B,A}\circ\e{}$. We can construct a pair of natural
  transformations $\Phi$, $\Phi'$ with components
  $\Phi_{A,B},\Phi'_{A,B}:TA\otimes TB\rightarrow T(A\otimes B)$ defined as
  $\Phi_{A,B}=\mu_{A\otimes B}\circ T\tau'_{A,B}\circ\tau_{TA,B}$ and
  $\Phi'_{A,B}=\mu_{A\otimes B}\circ T\tau_{A,B}\circ\tau'_{A,TB}$. If $\Phi=\Phi'$, then the
  monad $T$ is \textbf{commutative}.
\end{definition}

\begin{lemma}
  Let $\cat{M}$ be a symmetric monoidal category and $T$ be a strong monad on $\cat{M}$. Then
  $T$ is a symmetric monoidal functor iff it is commutative.
\end{lemma}

\begin{theorem}
  The monad on the SMCC in a SMCC-Lambek model is not commutative.
\end{theorem}



%%% Local Variables: 
%%% mode: latex
%%% TeX-master: main.tex
%%% End: 























% section categorical_models (end)

\section{Logic}
\label{sec:logic}
% section logic (end)

\section{Applications}
\label{sec:applications}
% section applications (end)

\section{Related Work}
\label{sec:related_work}
TODO
% section related_work (end)

\section{Conclusion}
\label{sec:conclusion}
TODO
% section conclusion (end)

%\bibliographystyle{plainurl}
%\bibliography{ref}

\appendix
\section{Appendix}
\label{sec:appendix}
% \section{Proof For Lemma~\ref{lem:cut-reduction}}
\label{app:cut-reduction}


\subsection{Commuting Conversion Cut vs. Cut}

\subsubsection{$\SCdruleTXXcutName$ vs. $\SCdruleTXXcutName$}
\begin{itemize}
% C-Cut vs. C-Cut Case 1
\item Case 1:
      \begin{center}
        \scriptsize
        \begin{math}
          \begin{array}{c}
            \Pi_1 \\
            {[[I |-c X]]}
          \end{array}
        \end{math}
        \qquad\qquad
        $\Pi_2:$
        \begin{math}
          $$\mprset{flushleft}
          \inferrule* [right={\tiny cut}] {
            {
              \begin{array}{cc}
                \pi_1 & \pi_2 \\
                {[[P2, X, P3 |-c Y]]} & {[[P1, Y, P4 |-c Z]]}
              \end{array}
            }
          }{[[P1, P2, X, P3, P4 |-c Z]]}
        \end{math}
      \end{center}
      By assumption, $c(\Pi_1),c(\Pi_2)\leq |X|$. Therefore, $c(\pi_1)$,
      $c(\pi_2)\leq |X|$. Since $Y$ is the cut formula on $\pi_1$ and
      $\pi_2$, we have $|Y|+1\leq|X|$. By induction on $\Pi_1$ and $\pi_1$
      there exists a proof $\Pi'$ for sequent $[[P2, I, P3 |-c Y]]$ s.t.
      $c(\Pi')\leq|X|$. So $\Pi$ can be constructed as follows, with
      $c(\Pi)\leq max\{c(\Pi'),c(\pi_2),|Y|+1\}\leq |X|$.
      \begin{center}
        \scriptsize
        \begin{math}
          $$\mprset{flushleft}
          \inferrule* [right={\tiny cut}] {
            {
              \begin{array}{cc}
                \Pi' & \pi_2 \\
                {[[P2, I, P3 |-c Y]]} & {[[P1, Y, P4 |-c Z]]}
              \end{array}
            }
          }{[[P1, P2, I, P3, P4 |-c Z]]}
        \end{math}
      \end{center}

% C-Cut vs. C-Cut Case 2
\item Case 2:
      \begin{center}
        \scriptsize
        $\Pi_1$:
        \begin{math}
          $$\mprset{flushleft}
          \inferrule* [right={\tiny cut}] {
            {
              \begin{array}{cc}
                \pi_1 & \pi_2 \\
                {[[I |-c X]]} & {[[P2, X, P3 |-c Y]]}
              \end{array}
            }
          }{[[P2, I, P3 |-c Y]]}
        \end{math}
        \qquad\qquad
        \begin{math}
          \begin{array}{c}
            \Pi_2 \\
            {[[P1, Y, P4 |-c Z]]}
          \end{array}
        \end{math}
      \end{center}
      By assumption, $c(\Pi_1),c(\Pi_2)\leq |Y|$. Since the cut rank of the last cut in
      $\Pi_1$ is $|X|+1$, then $|X|+1\leq |Y|$. By induction on $\Pi_1$ and $\Pi_2$, there is
      a proof $\Pi'$ for sequent $[[P1, P2, X, P3, P4 |-c Z]]$ s.t. $c(\Pi')\leq|Y|$.
      Therefore, the proof $\Pi$ can be constructed as follows, and
      $c(\Pi)\leq max\{c(\pi_1),c(\Pi'),|X|+1\}\leq |Y|$.
      \begin{center}
        \scriptsize
        \begin{math}
          $$\mprset{flushleft}
          \inferrule* [right={\tiny cut}] {
            {
              \begin{array}{cc}
                \pi_1 & \Pi' \\
                {[[I |-c X]]} & {[[P1, P2, X, P3, P4 |-c Z]]}
              \end{array}
            }
          }{[[P1, P2, I, P3, P4 |-c Z]]}
        \end{math}
      \end{center}
\end{itemize}



% C-Cut vs. LC-Cut Case 1
\subsubsection{$\SCdruleTXXcutName$ vs. $\SCdruleSXXcutOneName$}
\begin{itemize}
\item Case 1:
      \begin{center}
        \scriptsize
        \begin{math}
          \begin{array}{c}
            \Pi_1 \\
            {[[I |-c X]]}
          \end{array}
        \end{math}
        \qquad\qquad
        $\Pi_2:$
        \begin{math}
          $$\mprset{flushleft}
          \inferrule* [right={\tiny cut1}] {
            {
              \begin{array}{cc}
                \pi_2 & \pi_3 \\
                {[[P1, X, P2 |-c Y]]} & {[[G1; Y; G2 |-l A]]}
              \end{array}
            }
          }{[[G1; P1; X; P2; G2 |-l A]]}
        \end{math}
      \end{center}
      By assumption, $c(\Pi_1),c(\Pi_2)\leq |X|$. Therefore, $c(\pi_1)$,
      $c(\pi_2)\leq |X|$. Since $Y$ is the cut formula on $\pi_1$ and
      $\pi_2$, we have $|Y|+1\leq|X|$. By induction on $\Pi_1$ and $\pi_1$,
      there exists a proof $\Pi'$ for sequent $[[P1, I, P2 |-c Y]]$ s.t.
      $c(\Pi')\leq|X|$. So $\Pi$ can be constructed as follows, with
      $c(\Pi)\leq max\{c(\Pi'),c(\pi_2),|Y|+1\}\leq |X|$.
      \begin{center}
        \scriptsize
        \begin{math}
          $$\mprset{flushleft}
          \inferrule* [right={\tiny cut1}] {
            {
              \begin{array}{cc}
                \Pi' & \pi_2 \\
                {[[P1, I, P2 |-c Y]]} & {[[G1; Y; G2 |-l A]]}
              \end{array}
            }
          }{[[G1; P1; I; P2; G2 |-l A]]}
        \end{math}
      \end{center}

% C-Cut vs. LC-Cut Case 2
\item Case 2:
      \begin{center}
        \scriptsize
        $\Pi_1$:
        \begin{math}
          $$\mprset{flushleft}
          \inferrule* [right={\tiny cut}] {
            {
              \begin{array}{cc}
                \pi_1 & \pi_2 \\
                {[[I |-c X]]} & {[[P1, X, P2 |-c Y]]}
              \end{array}
            }
          }{[[P1, I, P2 |-c Y]]}
        \end{math}
        \qquad\qquad
        \begin{math}
          \begin{array}{c}
            \Pi_2 \\
            {[[G1; Y; G2 |-l A]]}
          \end{array}
        \end{math}
      \end{center}
      By assumption, $c(\Pi_1),c(\Pi_2)\leq |Y|$. Similar as above,
      $|X|+1\leq |Y|$ and there is a proof $\Pi'$ constructed from $\pi_2$
      and $\Pi_2$ for sequent $[[G1; P1; X; P2; G2 |-l A]]$ s.t.
      $c(\Pi')\leq|Y|$. Therefore, the proof $\Pi$ can be constructed as
      follows, and $c(\Pi)\leq max\{c(\pi_1),c(\Pi'),|X|+1\}\leq |Y|$.
      \begin{center}
        \scriptsize
        \begin{math}
          $$\mprset{flushleft}
          \inferrule* [right={\tiny cut}] {
            {
              \begin{array}{cc}
                \pi_1 & \Pi'\\
                {[[I |-c X]]} & {[[G1; P1; X; P2; G2 |-l A]]}
              \end{array}
            }
          }{[[G1; P1; I; P2; G2 |-l A]]}
        \end{math}
      \end{center}
\end{itemize}

% LC-Cut vs. L-Cut Case 1
\subsubsection{$\SCdruleSXXcutOneName$ vs. $\SCdruleSXXcutTwoName$}
\begin{itemize}
\item Case 1:
      \begin{center}
        \scriptsize
        \begin{math}
          \begin{array}{c}
            \Pi_1 \\
            {[[I |-c X]]}
          \end{array}
        \end{math}
        \qquad\qquad
        $\Pi_2:$
        \begin{math}
          $$\mprset{flushleft}
          \inferrule* [right={\tiny cut2}] {
            {
              \begin{array}{cc}
                \pi_1 & \pi_2 \\
                {[[G2; X; G3 |-l A]]} & {[[G1; A; G4 |-l B]]}
              \end{array}
            }
          }{[[G1; G2; X; G3; G4 |-l B]]}
        \end{math}
      \end{center}
      By assumption, $c(\Pi_1),c(\Pi_2)\leq |X|$. Therefore, $c(\pi_1)$,
      $c(\pi_2)\leq |X|$. Since $A$ is the cut formula on $\pi_1$ and
      $\pi_2$, we have $|A|+1\leq|X|$. By induction on $\Pi_1$ and $\pi_1$,
      there exists a proof $\Pi'$ for sequent $[[G2; I; G3 |-l A]]$ s.t.
      $c(\Pi')\leq|X|$. So $\Pi$ can be constructed as follows, with
      $c(\Pi)\leq max\{c(\Pi'),c(\pi_2),|A|+1\}\leq |X|$.
      \begin{center}
        \scriptsize
        \begin{math}
          $$\mprset{flushleft}
          \inferrule* [right={\tiny cut2}] {
            {
              \begin{array}{cc}
                \Pi' & \pi_2 \\
                {[[G2; I; G3 |-l A]]} & {[[G1; A; G4 |-l B]]}
              \end{array}
            }
          }{[[G1; G2; I; G3; G4 |-l B]]}
        \end{math}
      \end{center}

% LC-Cut vs. L-Cut Case 2
\item Case 2:
      \begin{center}
        \scriptsize
        $\Pi_1$:
        \begin{math}
          $$\mprset{flushleft}
          \inferrule* [right={\tiny cut}] {
            {
              \begin{array}{cc}
                \pi_1 & \pi_2 \\
                {[[I |-c X]]} & {[[G2; X; G3 |-l A]]}
              \end{array}
            }
          }{[[G2; I; G3 |-l A]]}
        \end{math}
        \qquad\qquad
        \begin{math}
          \begin{array}{c}
            \Pi_2 \\
            {[[G1; A; G4 |-l B]]}
          \end{array}
        \end{math}
      \end{center}
      By assumption, $c(\Pi_1),c(\Pi_2)\leq |A|$. Similar as above,
      $|X|+1\leq |A|$ and there is a proof $\Pi'$ constructed from'
      $\pi_2$ and $\Pi_2$ for sequent $[[G1; G2; X; G3; G4 |-l B]]$ s.t.
      $c(\Pi')\leq|A|$. Therefore, the proof $\Pi$ can be constructed as
      follows, and $c(\Pi)\leq max\{c(\pi_1),c(\Pi'),|X|+1\}\leq |A|$.
      \begin{center}
        \scriptsize
        \begin{math}
          $$\mprset{flushleft}
          \inferrule* [right={\tiny cut}] {
            {
              \begin{array}{cc}
                \pi_1  & \Pi' \\
                {[[I |-c X]]} & {[[G1; G2; X; G3; G4 |-l B]]}
              \end{array}
            }
          }{[[G1; G2; I; G3; G4 |-l B]]}
        \end{math}
      \end{center}
\end{itemize}

% L-Cut vs. L-Cut Case 1
\subsubsection{$\SCdruleSXXcutTwoName$ vs. $\SCdruleSXXcutTwoName$}
\begin{itemize}
\item Case 1:
      \begin{center}
        \scriptsize
        \begin{math}
          \begin{array}{c}
            \Pi_1 \\
            {[[G |-l A]]}
          \end{array}
        \end{math}
        \qquad\qquad
        $\Pi_2:$
        \begin{math}
          $$\mprset{flushleft}
          \inferrule* [right={\tiny cut2}] {
            {
              \begin{array}{cc}
                \pi_1 & \pi_2 \\
                {[[D2; A; D3 |-l B]]} & {[[D1; B; D4 |-l C]]}
              \end{array}
            }
          }{[[D1; D2; A; D3; D4 |-l C]]}
        \end{math}
      \end{center}
      By assumption, $c(\Pi_1),c(\Pi_2)\leq |A|$. Therefore, $c(\pi_1)$,
      $c(\pi_2)\leq |A|$. Since $B$ is the cut formula on $\pi_1$ and
      $\pi_3$, we have $|B|+1\leq|A|$. By induction on $\Pi_1$ and
      $\pi_1$, there exists a proof $\Pi'$ for sequent
      $[[D2; G; D3 |-l B]]$ s.t. $c(\Pi')\leq|A|$. So $\Pi$ can be
      constructed as follows,  with
      $c(\Pi)\leq max\{c(\Pi'),c(\pi_2),|B|+1\}\leq |A|$.
      \begin{center}
        \scriptsize
        \begin{math}
          $$\mprset{flushleft}
          \inferrule* [right={\tiny cut}] {
            {
              \begin{array}{cc}
                \Pi' & \pi_2 \\
                {[[D2; G; D3 |-l B]]} & {[[D1; B; D4 |-l C]]}
              \end{array}
            }
          }{[[D1; D2; G; D3; D4 |-l C]]}
        \end{math}
      \end{center}

% L-Cut vs. L-Cut Case 2
\item Case 2:
      \begin{center}
        \scriptsize
        $\Pi_1$:
        \begin{math}
          $$\mprset{flushleft}
          \inferrule* [right={\tiny cut}] {
            {
              \begin{array}{cc}
                \pi_1 & \pi_2 \\
                {[[D |-l A]]} & {[[D2; A; D3 |-l B]]}
              \end{array}
            }
          }{[[D2; D; D3 |-l A]]}
        \end{math}
        \qquad\qquad
        \begin{math}
          \begin{array}{c}
            \Pi_2 \\
            {[[D1; B; D4 |-l C]]}
          \end{array}
        \end{math}
      \end{center}
      By assumption, $c(\Pi_1),c(\Pi_2)\leq |B|$. Similar as above,
      $|A|+1\leq |B|$ and there is a proof $\Pi'$ constructed from $\pi_2$ 
      and $\Pi_2$ for sequent $[[D1; D2; A; D3; D4 |-l C]]$ s.t.
      $c(\Pi')\leq|A|$. Therefore, the proof $\Pi$ can be constructed as
      follows, and $c(\Pi)\leq max\{c(\pi_1),c(\Pi'),|A|+1\}\leq |B|$.
      \begin{center}
        \scriptsize
        \begin{math}
          $$\mprset{flushleft}
          \inferrule* [right={\tiny cut}] {
            {
              \begin{array}{cc}
                \pi_1 & \Pi' \\
                {[[G |-l A]]} & {[[D1; D2; A; D3; D4 |-l C]]}
              \end{array}
            }
          }{[[D1; D2; G; D3; D4 |-l C]]}
        \end{math}
      \end{center}

\end{itemize}
% End of subsubsection Commuting conversion cut vs. cut



\subsection{The Axiom Steps}

\subsubsection{$\SCdruleTXXaxName$}
\begin{itemize}
% C-id Case 1
\item Case 1:
      \begin{center}
        \scriptsize
        $\Pi_1$:
        \begin{math}
          $$\mprset{flushleft}
          \inferrule* [right={\tiny ax}] {
            \,
          }{[[X |-c X]]}
        \end{math}
        \qquad\qquad
        \begin{math}
          \begin{array}{c}
            \Pi_2 \\
            {[[I1, X, I2 |-c Y]]}
          \end{array}
        \end{math}
      \end{center}
      By assumption, $c(\Pi_1),c(\Pi_2)\leq |X|$. The proof $\Pi$ is the
      same as $\Pi_2$.

% C-id Case 2
\item Case 2:
      \begin{center}
        \scriptsize
        $\Pi_1$:
        \begin{math}
          \begin{array}{c}
            \Pi_1 \\
            {[[I |-c X]]}
          \end{array}
        \end{math}
        \qquad\qquad
        $\Pi_2$:
        \begin{math}
          $$\mprset{flushleft}
          \inferrule* [right={\tiny ax}] {
            \,
          }{[[X |-c X]]}
        \end{math}
      \end{center}
      By assumption, $c(\Pi_1),c(\Pi_2)\leq |X|$. The proof $\Pi$ is the
      same as $\Pi_1$.

% C-id Case 3
\item Case 3:
      \begin{center}
        \scriptsize
        $\Pi_1$:
        \begin{math}
          $$\mprset{flushleft}
          \inferrule* [right={\tiny ax}] {
            \,
          }{[[X |-c X]]}
        \end{math}
        \qquad\qquad
        \begin{math}
          \begin{array}{c}
            \Pi_2 \\
            {[[G1; X; G2 |-l A]]}
          \end{array}
        \end{math}
      \end{center}
      By assumption, $c(\Pi_1),c(\Pi_2)\leq |X|$. The proof $\Pi$ is the
      same as $\Pi_2$.
\end{itemize}

% L-id Case 1
\subsubsection{$\SCdruleTXXaxName$}
\begin{itemize}
\item Case 1:
      \begin{center}
        \scriptsize
        $\Pi_1$:
        \begin{math}
          $$\mprset{flushleft}
          \inferrule* [right={\tiny ax}] {
            \,
          }{[[A |-l A]]}
        \end{math}
        \qquad\qquad
        \begin{math}
          \begin{array}{c}
            \Pi_2 \\
            {[[G1; A; G2 |-l B]]}
          \end{array}
        \end{math}
      \end{center}
      By assumption, $c(\Pi_1),c(\Pi_2)\leq |A|$. The proof $\Pi$ is the
      same as $\Pi_2$.

% L-id Case 2
\item Case 2:
      \begin{center}
        \scriptsize
        $\Pi_1$:
        \begin{math}
          \begin{array}{c}
            \Pi_1 \\
            {[[D |-l A]]}
          \end{array}
        \end{math}
        \qquad\qquad
        $\Pi_2$:
        \begin{math}
          $$\mprset{flushleft}
          \inferrule* [right={\tiny ax}] {
            \,
          }{[[A |-l A]]}
        \end{math}
      \end{center}
      By assumption, $c(\Pi_1),c(\Pi_2)\leq |X|$. The proof $\Pi$ is the
      same as $\Pi_1$.
\end{itemize}
% End of subsubsection Axiom steps



\subsection{The Exchange Steps}

\subsubsection{$\SCdruleTXXexName$}

\begin{itemize}
% Conclusion vs. C-ex Case 1
\item Case 1:
      \begin{center}
        \scriptsize
        \begin{math}
          \begin{array}{c}
            \Pi_1 \\
            {[[P |-c X1]]}
          \end{array}
        \end{math}
        \qquad\qquad
        $\Pi_2$:
        \begin{math}
          $$\mprset{flushleft}
          \inferrule* [right={\tiny ex}] {
            {
              \begin{array}{c}
                \pi \\
                {[[I1, X1, X2, I2 |-c Y]]}
              \end{array}
            }
          }{[[I1, X2, X1, I2 |-c Y]]}
        \end{math}
      \end{center}
      By assumption, $c(\Pi_1),c(\Pi_2)\leq |X_1|$. By induction on $\pi$
      and $\Pi_1$, there is a proof $\Pi'$ for sequent
      $[[I1, P, X2, I2 |-c Y]]$ s.t. $c(\Pi')\leq|X_1|$. Therefore, the
      proof $\Pi$ can be constructed as follows, and
      $c(\Pi)=c(\Pi')\leq|X_1|$.
      \begin{center}
        \scriptsize
        \begin{math}
          $$\mprset{flushleft}
          \inferrule* [right={\tiny series of ex}] {
            {
              \begin{array}{c}
                \Pi' \\
                {[[I1, P, X2, I2 |-c Y]]}
              \end{array}
            }
          }{[[I1, X2, P, I2 |-c Y]]}
        \end{math}
      \end{center}

% Conclusion vs. C-ex Case 2
\item Case 2:
      \begin{center}
        \scriptsize
        \begin{math}
          \begin{array}{c}
            \Pi_1 \\
            {[[P |-c X2]]}
          \end{array}
        \end{math}
        \qquad\qquad
        $\Pi_2$:
        \begin{math}
          $$\mprset{flushleft}
          \inferrule* [right={\tiny ex}] {
            {
              \begin{array}{c}
                \pi \\
                {[[I1, X1, X2, I2 |-c Y]]}
              \end{array}
            }
          }{[[I1, X2, X1, I2 |-c Y]]}
        \end{math}
      \end{center}
      By assumption, $c(\Pi_1),c(\Pi_2)\leq |X_2|$. By induction on $\pi$
      and $\Pi_1$, there is a proof $\Pi'$ for sequent
      $[[I1, X1, P, I2 |-c Y]]$ s.t. $c(\Pi')\leq|X_2|$. Therefore, the
      proof $\Pi$ can be constructed as follows, and
      $c(\Pi)=c(\Pi')\leq|X_2|$.
      \begin{center}
        \scriptsize
        \begin{math}
          $$\mprset{flushleft}
          \inferrule* [right={\tiny series of ex}] {
            {
              \begin{array}{c}
                \Pi' \\
                {[[I1, X1, P, I2 |-c Y]]}
              \end{array}
            }
          }{[[I1, P, X1, I2 |-c Y]]}
        \end{math}
      \end{center}
\end{itemize}

% Conclusion vs. LC-ex Case 1
\subsubsection{$\SCdruleSXXexName$}
\begin{itemize}
\item Case 1:
      \begin{center}
        \scriptsize
        \begin{math}
          \begin{array}{c}
            \Pi_1 \\
            {[[P |-c X1]]}
          \end{array}
        \end{math}
        \qquad\qquad
        $\Pi_2$:
        \begin{math}
          $$\mprset{flushleft}
          \inferrule* [right={\tiny ex}] {
            {
              \begin{array}{c}
                \pi \\
                {[[D1; X1; X2; D2 |-l A]]}
              \end{array}
            }
          }{[[D1; X2; X1; D2 |-l A]]}
        \end{math}
      \end{center}
      By assumption, $c(\Pi_1),c(\Pi_2)\leq |X_1|$. By induction on $\pi$
      and $\Pi_1$, there is a proof $\Pi'$ for sequent
      $[[D1; P; X2; D2 |-l A]]$ s.t. $c(\Pi')\leq|X_1|$. Therefore, the
      proof $\Pi$ can be constructed as follows, and
      $c(\Pi)=c(\Pi')\leq|X_1|$.
      \begin{center}
        \scriptsize
        \begin{math}
          $$\mprset{flushleft}
          \inferrule* [right={\tiny series of ex}] {
            {
              \begin{array}{c}
                \Pi' \\
                {[[D1; P; X2; D2 |-l A]]}
              \end{array}
            }
          }{[[D1; X2; P; D2 |-l A]]}
        \end{math}
      \end{center}

% Conclusion vs. LC-ex Case 2
\item Case 2:
      \begin{center}
        \scriptsize
        \begin{math}
          \begin{array}{c}
            \Pi_1 \\
            {[[P |-c X2]]}
          \end{array}
        \end{math}
        \qquad\qquad
        $\Pi_2$:
        \begin{math}
          $$\mprset{flushleft}
          \inferrule* [right={\tiny ex}] {
            {
              \begin{array}{c}
                \pi \\
                {[[D1; X1; X2; D2 |-l A]]}
              \end{array}
            }
          }{[[D1; X2; X1; D2 |-l A]]}
        \end{math}
      \end{center}
      By assumption, $c(\Pi_1),c(\Pi_2)\leq |X_2|$. By induction on $\pi$
      and $\Pi_1$, there is a proof $\Pi'$ for sequent
      $[[D1; X1; P; D2 |-l A]]$ s.t. $c(\Pi')\leq|X_2|$. Therefore, the
      proof $\Pi$ can be constructed as follows, and
      $c(\Pi)=c(\Pi')\leq|X_2|$.
      \begin{center}
        \scriptsize
        \begin{math}
          $$\mprset{flushleft}
          \inferrule* [right={\tiny series of ex}] {
            {
              \begin{array}{c}
                \Pi' \\
                {[[D1; X1; P; I2 |-l A]]}
              \end{array}
            }
          }{[[I1; P; X1; I2 |-l A]]}
        \end{math}
      \end{center}
\end{itemize}



\subsection{Principal Formula vs. Principal Formula} 

\subsubsection{The Commutative Tensor Product $\otimes$}
\begin{center}
  \scriptsize
  $\Pi_1:$
  \begin{math}
    $$\mprset{flushleft}
    \inferrule* [right={\tiny tenR}] {
      {
        \begin{array}{cc}
          \pi_1 & \pi_2 \\
          {[[I1 |-c X]]} & {[[I2 |-c Y]]}
        \end{array}
      }
    }{[[I1, I2 |-c X (*) Y]]}
  \end{math}
  \qquad\qquad
  $\Pi_2:$
  \begin{math}
    $$\mprset{flushleft}
    \inferrule* [right={\tiny tenL}] {
      {
        \begin{array}{c}
          \pi_3 \\
          {[[P1, X, Y, P2 |-c Z]]}
        \end{array}
      }
    }{[[P1, X (*) Y, P2 |-c Z]]}
  \end{math}
\end{center}
By assumption, $c(\Pi_1),c(\Pi_2)\leq |[[X (*) Y]]| = |X|+|Y|+1$. The proof
$\Pi$ can be constructed as follows, and
$c(\Pi)\leq max\{c(\pi_1),c(\pi_2),c(\pi_3),|X|+1,|Y|+1\}\leq |X|+|Y|+1 = |[[X (*) Y]]|$.
\begin{center}
  \scriptsize
  \begin{math}
    $$\mprset{flushleft}
    \inferrule* [right={\tiny cut}] {
      {
        \begin{array}{c}
          \pi_1 \\
          {[[I1 |-c X]]}
        \end{array}
      }
      $$\mprset{flushleft}
      \inferrule* [right={\tiny cut}] {
      {
        \begin{array}{cc}
          \pi_2 & \pi_3 \\
          {[[I2 |-c Y]]} & {[[P1, X, Y, P2 |-c Z]]}
        \end{array}
      }
      }{[[P1, X, I2, P2 |-c Z]]}
    }{[[P1, I1, I2, P2 |-c Z]]}
  \end{math}
\end{center}

\subsubsection{The Non-commutative Tensor Product $\tri$}
\begin{center}
  \scriptsize
  $\Pi_1:$
  \begin{math}
    $$\mprset{flushleft}
    \inferrule* [right={\tiny tenR}] {
      {
        \begin{array}{cc}
          \pi_1 & \pi_2 \\
          {[[G1 |-l A]]} & {[[G2 |-l B]]}
        \end{array}
      }
    }{[[G1; G2 |-l A (>) B]]}
  \end{math}
  \qquad\qquad
  $\Pi_2:$
  \begin{math}
    $$\mprset{flushleft}
    \inferrule* [right={\tiny tenL1}] {
      {
        \begin{array}{c}
          \pi_3 \\
          {[[D1; A; B; D2 |-l C]]}
        \end{array}
      }
    }{[[D1; A (>) B; D2 |-l C]]}
  \end{math}
\end{center}
By assumption, $c(\Pi_1),c(\Pi_2)\leq |[[A (>) B]]| = |X|+|Y|+1$. The proof
$\Pi$ can be constructed as follows, and
$c(\Pi)\leq max\{c(\pi_1),c(\pi_2),c(\pi_3),|A|+1,|B|+1\}\leq |A|+|B|+1 = |[[A (>) B]]|$.
\begin{center}
  \scriptsize
  \begin{math}
    $$\mprset{flushleft}
    \inferrule* [right={\tiny cut2}] {
      {
        \begin{array}{c}
          \pi_1 \\
          {[[G1 |-l A]]}
        \end{array}
      }
      $$\mprset{flushleft}
      \inferrule* [right={\tiny cut2}] {
      {
        \begin{array}{cc}
          \pi_2 & \pi_3 \\
          {[[G2 |-l B]]} & {[[D1; A; B; D2 |-l C]]}
        \end{array}
      }
      }{[[D1; A; G2; D2 |-l C]]}
    }{[[D1; G1; G2; P2 |-l C]]}
  \end{math}
\end{center}

\subsubsection{The Commutative Implication $\multimap$}
\begin{center}
  \scriptsize
  $\Pi_1:$
  \begin{math}
    $$\mprset{flushleft}
    \inferrule* [right={\tiny tenR}] {
      {
        \begin{array}{c}
          \pi_1 \\
          {[[I1, X |-c Y]]}
        \end{array}
      }
    }{[[I1 |-c X -o Y]]}
  \end{math}
  \qquad\qquad
  $\Pi_2:$
  \begin{math}
    $$\mprset{flushleft}
    \inferrule* [right={\tiny tenL}] {
      {
        \begin{array}{cc}
          \pi_2 & \pi_3 \\
          {[[I2 |-c X]]} & {[[P1, Y, P2 |-c Z]]}
        \end{array}
      }
    }{[[P1, X -o Y, I, P2 |-c Z]]}
  \end{math}
\end{center}
By assumption, $c(\Pi_1),c(\Pi_2)\leq |[[X -o Y]]| = |X|+|Y|+1$. The proof 
$\Pi$ is constructed as follows
$c(\Pi)\leq max\{c(\pi_1),c(\pi_2),c(\pi_3),|X|+1,|Y|+1\}\leq |X|+|Y|+1 = |[[X -o Y]]|$.
\begin{center}
  \scriptsize
  \begin{math}
    $$\mprset{flushleft}
    \inferrule* [right={\tiny tenR}] {
      $$\mprset{flushleft}
      \inferrule* [right={\tiny tenR}] {
        {
          \begin{array}{cc}
            \pi_1 & \pi_2 \\
            {[[I1, X |-c Y]]} & {[[I2 |-c X]]}
          \end{array}
        }
      }{[[I1, I2 |-c Y]]} \\
       {
         \begin{array}{c}
           \pi_3 \\
           {[[P1, Y, P2 |-c Z]]}
         \end{array}
       }
    }{[[P1, I1, I2, P2 |-c Z]]}
  \end{math}
\end{center}

\subsubsection{The Non-commutative Right Implication $\lto$}
\begin{center}
  \scriptsize
  $\Pi_1:$
  \begin{math}
    $$\mprset{flushleft}
    \inferrule* [right={\tiny imprR}] {
      {
        \begin{array}{c}
          \pi_1 \\
          {[[G; A |-l B]]}
        \end{array}
      }
    }{[[G |-l A -> B]]}
  \end{math}
  \qquad\qquad
  $\Pi_2:$
  \begin{math}
    $$\mprset{flushleft}
    \inferrule* [right={\tiny imprL}] {
      {
        \begin{array}{cc}
          \pi_2 & \pi_3 \\
          {[[D1 |-l A]]} & {[[D2; B |-l C]]}
        \end{array}
      }
    }{[[D2; A -> B; D1 |-l C]]}
  \end{math}
\end{center}
By assumption, $c(\Pi_1),c(\Pi_2)\leq |[[A -> B]]| = |A|+|B|+1$. The proof
$\Pi$ is constructed as follows, and
$c(\Pi)\leq max\{c(\pi_1),c(\pi_2),c(\pi_3),|A|+1,|B|+1\}\leq |A|+|B|+1 = |[[A -> B]]|$.
\begin{center}
  \scriptsize
  \begin{math}
    $$\mprset{flushleft}
    \inferrule* [right={\tiny cut2}] {
      $$\mprset{flushleft}
      \inferrule* [right={\tiny cut2}] {
        {
          \begin{array}{cc}
            \pi_1 & \pi_2 \\
            {[[G; A |-l B]]} & {[[D1 |-l A]]}
          \end{array}
        }
      }{[[G; D1 |-l B]]}
       {
         \begin{array}{c}
           \pi_3 \\
           {[[D2; B |-l C]]}
         \end{array}
       }
    }{[[D2; G; D1 |-l C]]}
  \end{math}
\end{center}

\subsubsection{The Non-commutative Left Implication $\rto$}
\begin{center}
  \scriptsize
  $\Pi_1:$
  \begin{math}
    $$\mprset{flushleft}
    \inferrule* [right={\tiny implR}] {
      {
        \begin{array}{c}
          \pi_1 \\
          {[[A; G |-l B]]}
        \end{array}
      }
    }{[[G |-l B <- A]]}
  \end{math}
  \qquad\qquad
  $\Pi_2:$
  \begin{math}
    $$\mprset{flushleft}
    \inferrule* [right={\tiny implL}] {
      {
        \begin{array}{cc}
          \pi_2 & \pi_3 \\
          {[[D1 |-l A]]} & {[[B; D2 |-l C]]}
        \end{array}
      }
    }{[[D1; B <- A; D2 |-l C]]}
  \end{math}
\end{center}
By assumption, $c(\Pi_1),c(\Pi_2)\leq |[[B <- A]]| = |A|+|B|+1$. The
proof $\Pi$ is constructed as follows, and
$c(\Pi)\leq max\{c(\pi_1),c(\pi_2),c(\pi_3),|A|+1,|B|+1\}\leq |A|+|B|+1 = |[[B <- A]]|$.
\begin{center}
  \scriptsize
  \begin{math}
    $$\mprset{flushleft}
    \inferrule* [right={\tiny cut1}] {
      $$\mprset{flushleft}
      \inferrule* [right={\tiny cut2}] {
        {
          \begin{array}{cc}
            \pi_1 & \pi_2 \\
            {[[A; G |-l B]]} & {[[D1 |-l A]]}
          \end{array}
        }
      }{[[D1; G |-l B]]}
       {
         \begin{array}{c}
           \pi_3 \\
           {[[B; D2 |-l C]]}
         \end{array}
       }
    }{[[D1; G; D2 |-l C]]}
  \end{math}
\end{center}



\subsubsection{The Commutative Unit $[[UnitT]]$}
\begin{itemize}
\item Case 1:
      \begin{center}
        \scriptsize
        $\Pi_1:$
        \begin{math}
          $$\mprset{flushleft}
          \inferrule* [right={\tiny unitR}] {
            \,
          }{[[. |-c UnitT]]}
        \end{math}
        \qquad\qquad
        $\Pi_2:$
        \begin{math}
          $$\mprset{flushleft}
          \inferrule* [right={\tiny unitL}] {
            {
              \begin{array}{c}
                \pi \\
                {[[I, P |-c X]]}
              \end{array}
            }
          }{[[I, UnitT, P |-c X]]}
        \end{math}
      \end{center}
      By assumption, $c(\Pi_1),c(\Pi_2)\leq |[[UnitT]]|$. The proof $\Pi$
      is the subproof $\pi$ in $\Pi_2$ for sequent $[[I |-c X]]$. So
      $c(\Pi)=c(\Pi_2)\leq |[[UnitT]]|$.

\item Case 2:
      \begin{center}
        \scriptsize
        $\Pi_1:$
        \begin{math}
          $$\mprset{flushleft}
          \inferrule* [right={\tiny unitR}] {
            \,
          }{[[. |-c UnitT]]}
        \end{math}
        \qquad\qquad
        $\Pi_2:$
        \begin{math}
          $$\mprset{flushleft}
          \inferrule* [right={\tiny unitL1}] {
            {
              \begin{array}{c}
                \pi \\
                {[[G; D |-l A]]}
              \end{array}
            }
          }{[[G; UnitT; D |-l A]]}
        \end{math}
      \end{center}
      Similar as above, $\Pi$ is $\pi$.
\end{itemize}


\subsubsection{The Non-commutative Unit $[[UnitS]]$}
\begin{center}
  \scriptsize
  $\Pi_1:$
  \begin{math}
    $$\mprset{flushleft}
    \inferrule* [right={\tiny unitR}] {
      \,
    }{[[. |-l UnitS]]}
  \end{math}
  \qquad\qquad
  $\Pi_2:$
  \begin{math}
    $$\mprset{flushleft}
    \inferrule* [right={\tiny unitL2}] {
      {
        \begin{array}{c}
          \pi \\
          {[[G; D |-l A]]}
        \end{array}
      }
    }{[[G; UnitS; D |-l A]]}
  \end{math}
\end{center}
By assumption, $c(\Pi_1),c(\Pi_2)\leq |[[UnitS]]|$. The proof $\Pi$ is the
subproof $\pi$ in $\Pi_2$ for sequent $[[D |-l A]]$. So
$c(\Pi)=c(\Pi_2)\leq |[[UnitS]]|$.

\subsubsection{The Functor $F$}
\begin{center}
  \scriptsize
  $\Pi_1:$
  \begin{math}
    $$\mprset{flushleft}
    \inferrule* [right={\tiny FR}] {
      {
        \begin{array}{c}
          \pi_1 \\
          {[[I |-c X]]}
        \end{array}
      }
    }{[[I |-l F X]]}
  \end{math}
  \qquad\qquad
  $\Pi_2:$
  \begin{math}
    $$\mprset{flushleft}
    \inferrule* [right={\tiny FL}] {
      {
        \begin{array}{c}
          \pi_2 \\
          {[[G; X; D |-l A]]}
        \end{array}
      }
    }{[[G; F X; D |-l A]]}
  \end{math}
\end{center}
By assumption, $c(\Pi_1),c(\Pi_2)\leq |[[F X]]| = |X|+1$. The proof
$\Pi$ is constructed as follows, and \\
$c(\Pi)\leq max\{c(\pi_1),c(\pi_2),|X|+1\}\leq |[[F X]]|$.
\begin{center}
  \scriptsize
  \begin{math}
    $$\mprset{flushleft}
    \inferrule* [right={\tiny cut2}] {
      {
        \begin{array}{cc}
          \pi_1 & \pi_2 \\
          {[[I |-c X]]} & {[[G; X; D |-l A]]}
        \end{array}
      }
    }{[[G; I; D |-l A]]}
  \end{math}
\end{center}

\subsubsection{The Functor $G$}
\begin{center}
  \scriptsize
  $\Pi_1:$
  \begin{math}
    $$\mprset{flushleft}
    \inferrule* [right={\tiny GR}] {
      {
        \begin{array}{c}
          \pi_1 \\
          {[[I |-l A]]}
        \end{array}
      }
    }{[[I |-c Gf A]]}
  \end{math}
  \qquad\qquad
  $\Pi_2:$
  \begin{math}
    $$\mprset{flushleft}
    \inferrule* [right={\tiny GL}] {
      {
        \begin{array}{c}
          \pi_2 \\
          {[[G; A; D |-l B]]}
        \end{array}
      }
    }{[[G; Gf A; D |-l B]]}
  \end{math}
\end{center}
By assumption, $c(\Pi_1),c(\Pi_2)\leq |[[Gf A]]| = |A|+1$. The proof $\Pi$ 
is constructed as follows, and \\
$c(\Pi)\leq max\{c(\pi_1),c(\pi_2),|A|+1\}\leq |[[Gf A]]|$.
\begin{center}
  \scriptsize
  \begin{math}
    $$\mprset{flushleft}
    \inferrule* [right={\tiny GL}] {
      {
        \begin{array}{cc}
          \pi_1 & \pi_2 \\
          {[[I |-l A]]} & {[[G; A; D |-l B]]}
        \end{array}
      }
    }{[[G; I; D |-l B]]}
  \end{math}
\end{center}



\subsection{Secondary Conclusion}

\subsubsection{Left introduction of the commutative implication $\multimap$}
\begin{itemize}
\item Case 1:
      \begin{center}
        \scriptsize
        $\Pi_1$:
        \begin{math}
          $$\mprset{flushleft}
          \inferrule* [right={\tiny impL}] {
            {
              \begin{array}{cc}
                \pi_1 & \pi_2 \\
                {[[I1 |-c X1]]} & {[[I2, X2, I3 |-c Y]]}
              \end{array}
            }
          }{[[I2, X1 -o X2, I1, I3 |-c Y]]}
        \end{math}
        \qquad\qquad
        \begin{math}
          \begin{array}{c}
            \Pi_2 \\
            {[[P1, Y, P2 |-c Z]]}
          \end{array}
        \end{math}
      \end{center}
      By assumption, $c(\Pi_1),c(\Pi_2)\leq |Y|$. By induction, there is a
      proof $\Pi'$ from $\pi_2$ and $\Pi_2$ for sequent
      $[[P1, I2, X2, I3, P2 |-c Z]]$ s.t. $c(\Pi')\leq |Y|$. Therefore,
      the proof $\Pi$ can be constructed as follows with $c(\Pi)\leq |Y|$.
      \begin{center}
        \scriptsize
        \begin{math}
          $$\mprset{flushleft}
          \inferrule* [right={\tiny impL}] {
            {
              \begin{array}{c}
                \pi_1 \\
                {[[I1 |-c X1]]}
              \end{array}
            }
            $$\mprset{flushleft}
            \inferrule* [right={\tiny cut}] {
              {
                \begin{array}{cc}
                  \pi_2 & \Pi_2 \\
                  {[[I2, X2, I3 |-c Y]]} & {[[P1, Y, P2 |-c Z]]}
                \end{array}
              }
            }{[[P1, I2, X2, I3, P2 |-c Z]]}
          }{[[P1, I2, X1 -o X2, I1, I3, P2 |-c Z]]}
        \end{math}
      \end{center}

\item Case 2:
      \begin{center}
        \scriptsize
        $\Pi_1$:
        \begin{math}
          $$\mprset{flushleft}
          \inferrule* [right={\tiny impL}] {
            {
              \begin{array}{cc}
                \pi_1 & \pi_2 \\
                {[[I1 |-c X1]]} & {[[I2, X2, I3 |-c Y]]}
              \end{array}
            }
          }{[[I2, X1 -o X2, I1, I3 |-c Y]]}
        \end{math}
        \qquad\qquad
        \begin{math}
          \begin{array}{c}
            \Pi_2 \\
            {[[G1; Y; G2 |-l A]]}
          \end{array}
        \end{math}
      \end{center}
      By assumption, $c(\Pi_1),c(\Pi_2)\leq |Y|$. By induction, there is a
      proof $\Pi'$ from $\pi_2$ and $\Pi_2$ for sequent
      $[[G1; I2; X2; I3; G2 |-l A]]$ s.t. $c(\Pi')\leq |Y|$. Therefore, the
      proof $\Pi$ can be constructed as follows with $c(\Pi)\leq |Y|$.
      \begin{center}
        \scriptsize
        \begin{math}
          $$\mprset{flushleft}
          \inferrule* [right={\tiny impL}] {
            {
              \begin{array}{c}
                \pi_1 \\
                {[[I1 |-c X1]]}
              \end{array}
            }
            $$\mprset{flushleft}
            \inferrule* [right={\tiny cut}] {
              {
                \begin{array}{cc}
                  \pi_2 & \Pi_2 \\
                  {[[I2, X2, I3 |-c Y]]} & {[[G1; Y; G2 |-l A]]}
                \end{array}
              }
            }{[[G1; I2; X2; I3; G2 |-l A]]}
          }{[[G1; I2; X1 -o X2; I1; I3; G2 |-l A]]}
        \end{math}
      \end{center}
\end{itemize}



\subsubsection{Left introduction of the non-commutative left implication $\lto$}
\begin{center}
\scriptsize
  $\Pi_1$:
  \begin{math}
    $$\mprset{flushleft}
    \inferrule* [right={\tiny impL}] {
      {
        \begin{array}{cc}
          \pi_1 & \pi_2 \\
          {[[G1 |-l A1]]} & {[[G2; A2; G3 |-l B]]}
        \end{array}
      }
    }{[[G2; A1 -> A2; G1; G3 |-l B]]}
  \end{math}
  \qquad\qquad
  \begin{math}
    \begin{array}{c}
      \Pi_2 \\
      {[[D1; B; D2 |-l C]]}
    \end{array}
  \end{math}
\end{center}
By assumption, $c(\Pi_1),c(\Pi_2)\leq |B|$. By induction, there is a
proof $\Pi'$ from $\pi_2$ and $\Pi_2$ for sequent
$[[D1; G2; A2; G3; D2 |-l C]]$ s.t. $c(\Pi')\leq |B|$.
Therefore, the proof $\Pi$ can be constructed as follows with
$c(\Pi)\leq |B|$.
\begin{center}
  \scriptsize
  \begin{math}
    $$\mprset{flushleft}
    \inferrule* [right={\tiny impL}] {
      {
        \begin{array}{c}
          \pi_1 \\
          {[[G1 |-l A1 ]]}
        \end{array}
      }
      $$\mprset{flushleft}
      \inferrule* [right={\tiny cut}] {
        {
          \begin{array}{cc}
            \pi_2 & \Pi_2 \\
            {[[G2; A2; G3 |-l B]]} & {[[D1; B; D2 |-l C]]}
          \end{array}
        }
      }{[[D1; G2; A2; G3; D2 |-l C]]}
    }{[[D1; G2; A1 -> A2; G1; G3; D2 |-l C]]}
  \end{math}
\end{center}


\subsubsection{Left introduction of the non-commutative right implication $\rto$}
\begin{center}
  \scriptsize
  $\Pi_1$:
  \begin{math}
    $$\mprset{flushleft}
    \inferrule* [right={\tiny impL}] {
      {
        \begin{array}{cc}
          \pi_1 & \pi_2 \\
          {[[G1 |-l A1]]} & {[[G2; A2; G3 |-l B]]}
        \end{array}
      }
    }{[[G2; G1; A2 <- A1; G3 |-l B]]}
  \end{math}
  \qquad\qquad
  \begin{math}
    \begin{array}{c}
      \Pi_2 \\
      {[[D1; B; D2 |-l C]]}
    \end{array}
  \end{math}
\end{center}
By assumption, $c(\Pi_1),c(\Pi_2)\leq |B|$. By induction, there is a
proof $\Pi'$ from $\pi_2$ and $\Pi_2$ for sequent
$[[D1; G2; A2; G3; D2 |-l C]]$ s.t. $c(\Pi')\leq |B|$. Therefore, the
proof $\Pi$ can be constructed as follows with $c(\Pi)\leq |B|$.
\begin{center}
  \scriptsize
  \begin{math}
    $$\mprset{flushleft}
    \inferrule* [right={\tiny impL}] {
      {
        \begin{array}{c}
          \pi_1 \\
          {[[G1 |-l A1 ]]}
        \end{array}
      }
      $$\mprset{flushleft}
      \inferrule* [right={\tiny cut}] {
        {
          \begin{array}{cc}
            \pi_2 & \Pi_2 \\
            {[[G2; A2; G3 |-l B]]} & {[[D1; B; D2 |-l C]]}
          \end{array}
        }
      }{[[D1; G2; A2; G3; D2 |-l C]]}
    }{[[D1; G2; G1; A2 <- A1; G3; D2 |-l C]]}
  \end{math}
\end{center}

% C-ex Case 1
\subsubsection{$\SCdruleTXXexName$}
\begin{itemize}
\item Case 1:
      \begin{center}
        \scriptsize
        $\Pi_1$:
        \begin{math}
          $$\mprset{flushleft}
          \inferrule* [right={\tiny ex}] {
            {
              \begin{array}{c}
                \pi \\
                {[[I1, X1, X2, I2 |-c Y]]}
              \end{array}
            }
          }{[[I1, X2, X1, I2 |-c Y]]}
        \end{math}
        \qquad\qquad
        \begin{math}
          \begin{array}{c}
            \Pi_2 \\
            {[[P1, Y, P2 |-c Z]]}
          \end{array}
        \end{math}
      \end{center}
      By assumption, $c(\Pi_1),c(\Pi_2)\leq |Y|$. By induction on $\pi$
      and $\Pi_2$, there is a proof $\Pi'$ for sequent
      $[[P1, I1, X1, X2, I2, P2 |-c Z]]$ s.t. $c(\Pi')\leq|Y|$. Therefore,
      the proof $\Pi$ can be constructed as follows, and
      $c(\Pi)=c(\Pi')\leq|Y|$.
      \begin{center}
        \scriptsize
        \begin{math}
          $$\mprset{flushleft}
          \inferrule* [right={\tiny ex}] {
            {
              \begin{array}{c}
                \Pi' \\
                {[[P1, I1, X1, X2, I2, P2 |-c Z]]}
              \end{array}
            }
          }{[[P1, I1, X2, X1, I2, P2 |-c Z]]}
        \end{math}
      \end{center}

% C-ex Case 2
\item Case 2:
      \begin{center}
        \scriptsize
        $\Pi_1$:
        \begin{math}
          $$\mprset{flushleft}
          \inferrule* [right={\tiny beta}] {
            {
              \begin{array}{c}
                \pi \\
                {[[I1, X, Y, I2 |-c Z]]}
              \end{array}
            }
          }{[[I1, Y, X, I2 |-c Z]]}
        \end{math}
        \qquad\qquad
        \begin{math}
          \begin{array}{c}
            \Pi_2 \\
            {[[G1; Z; G2 |-l A]]}
          \end{array}
        \end{math}
      \end{center}
      By assumption, $c(\Pi_1),c(\Pi_2)\leq |Z|$. Similar as above, there
      is a proof $\Pi'$ constructed from $\pi$ and $\Pi_2$ for 
      $[[G1; I1; X; Y; I2; G2 |-l A]]$ s.t. $c(\Pi')\leq|Z|$. Therefore,
      the proof $\Pi$ can be constructed as follows, and
      $c(\Pi)=c(\Pi')\leq|Z|$.
      \begin{center}
        \scriptsize
        \begin{math}
          $$\mprset{flushleft}
          \inferrule* [right={\tiny beta}] {
            {
              \begin{array}{c}
                \Pi' \\
                {[[G1; I1; X; Y; I2; G2 |-l A]]}
              \end{array}
            }
          }{[[G1; I1; Y; X; I2; G2 |-l A]]}
        \end{math}
      \end{center}
\end{itemize}

% LC-ex
\subsubsection{$\SCdruleSXXexName$}
\begin{center}
  \scriptsize
  $\Pi_1$:
  \begin{math}
    $$\mprset{flushleft}
    \inferrule* [right={\tiny beta}] {
      {
        \begin{array}{c}
          \pi \\
          {[[G1; X; Y; G2 |-l A]]}
        \end{array}
      }
    }{[[G1; Y; X; G2 |-l A]]}
  \end{math}
  \qquad\qquad
  \begin{math}
    \begin{array}{c}
      \Pi_2 \\
      {[[D1; A; D2 |-l B]]}
    \end{array}
  \end{math}
\end{center}
By assumption, $c(\Pi_1),c(\Pi_2)\leq |A|$. Similar as above, there
is a proof $\Pi'$ constructed from $\pi$ and $\Pi_2$ for sequent
$[[D1; G1; X; Y; G2; D2 |-l B]]$ s.t. $c(\Pi')\leq|A|$. Therefore,
the proof $\Pi$ can be constructed as follows, and
$c(\Pi)=c(\Pi')\leq|A|$.
\begin{center}
  \scriptsize
  \begin{math}
    $$\mprset{flushleft}
    \inferrule* [right={\tiny beta}] {
      {
        \begin{array}{cc}
          \Pi' \\
          {[[D1; G1; X; Y; G2; D2 |-l B]]}
        \end{array}
      }
    }{[[D1; G1; Y; X; G2; D2 |-l B]]}
  \end{math}
\end{center}





\subsubsection{Left introduction of the commutative tensor product $\otimes$}
\begin{itemize}
\item Case 1:
      \begin{center}
        \scriptsize
        $\Pi_1$:
        \begin{math}
          $$\mprset{flushleft}
          \inferrule* [right={\tiny tenL}] {
            {
              \begin{array}{c}
                \pi \\
                {[[I1, X1, X2, I2 |-c Y]]}
              \end{array}
            }
          }{[[I1, X1 (*) X2, I2 |-c Y]]}
        \end{math}
        \qquad\qquad
        \begin{math}
          \begin{array}{c}
            \Pi_2 \\
            {[[P1, Y, P2 |-c Z]]}
          \end{array}
        \end{math}
      \end{center}
      By assumption, $c(\Pi_1),c(\Pi_2)\leq |Y|$. By induction, there is a
      proof $\Pi'$ from $\pi$ and $\Pi_2$ for sequent
      $[[P1, I1, X1, X2, I2, P2 |-c Z]]$ s.t. $c(\Pi')\leq |Y|$. Therefore,
      the proof $\Pi$ can be constructed as follows with $c(\Pi)\leq |Y|$.
      \begin{center}
        \scriptsize
        \begin{math}
          $$\mprset{flushleft}
          \inferrule* [right={\tiny tenL}] {
            $$\mprset{flushleft}
            \inferrule* [right={\tiny cut}] {
              {
                \begin{array}{cc}
                  \pi & \Pi_2 \\
                  {[[I1, X1, X2, I2 |-c Y]]} & {[[P1, Y, P2 |-c Z]]}
                \end{array}
              }
            }{[[P1, I1, X1, X2, I2, P2 |-c Z]]}
          }{[[P1, I1, X1 (*) X2, I2, P2 |-c Z]]}
        \end{math}
      \end{center}

\item Case 2:
      \begin{center}
        \scriptsize
        $\Pi_1$:
        \begin{math}
          $$\mprset{flushleft}
          \inferrule* [right={\tiny tenL}] {
            {
              \begin{array}{c}
                \pi \\
                {[[I1, X1, X2, I2 |-c Y]]}
              \end{array}
            }
          }{[[I1, X1 (*) X2, I2 |-c Y]]}
        \end{math}
        \qquad\qquad
        \begin{math}
          \begin{array}{c}
            \Pi_2 \\
            {[[G1; Y; G2 |-l A]]}
          \end{array}
        \end{math}
      \end{center}
      By assumption, $c(\Pi_1),c(\Pi_2)\leq |Y|$. By induction, there is a
      proof $\Pi'$ from $\pi$ and $\Pi_2$ for sequent
      $[[G1; I1; X1; X2; I2; G2 |-l A]]$ s.t. $c(\Pi')\leq |Y|$. Therefore,
      the proof $\Pi$ can be constructed as follows with $c(\Pi)\leq |Y|$.
      \begin{center}
        \scriptsize
        \begin{math}
          $$\mprset{flushleft}
          \inferrule* [right={\tiny tenL1}] {
            $$\mprset{flushleft}
            \inferrule* [right={\tiny cut1}] {
              {
                \begin{array}{cc}
                  \pi & \Pi_2 \\
                  {[[I1, X1, X2, I2 |-c Y]]} & {[[G1; Y; G2 |-l A]]}
                \end{array}
              }
            }{[[G1; I1; X1; X2; I2; G2 |-l A]]}
          }{[[G1; I1; X1 (*) X2; I2; G2 |-l A]]}
        \end{math}
      \end{center}

\item Case 3:
      \begin{center}
        \scriptsize
        $\Pi_1$:
        \begin{math}
          $$\mprset{flushleft}
          \inferrule* [right={\tiny tenL}] {
            {
              \begin{array}{c}
                \pi \\
                {[[G1; X; Y; G2 |-l A]]}
              \end{array}
            }
          }{[[G1; X (*) Y; G2 |-l A]]}
        \end{math}
        \qquad\qquad
        \begin{math}
          \begin{array}{c}
            \Pi_2 \\
            {[[D1; A; D2 |-l B]]}
          \end{array}
        \end{math}
      \end{center}
      By assumption, $c(\Pi_1),c(\Pi_2)\leq |A|$. By induction, there is a
      proof $\Pi'$ from $\pi$ and $\Pi_2$ for sequent
      $[[D1; X; Y; G2; D2 |-l B]]$ s.t. $c(\Pi')\leq |A|$. Therefore, the
      proof $\Pi$ can be constructed as follows with $c(\Pi)\leq |A|$.
      \begin{center}
        \scriptsize
        \begin{math}
          $$\mprset{flushleft}
          \inferrule* [right={\tiny tenL1}] {
            $$\mprset{flushleft}
            \inferrule* [right={\tiny cut2}] {
              {
                \begin{array}{cc}
                  \pi & \Pi_2 \\
                  {[[G1; X; Y; G2 |-l A]]} & {[[D1; A; D2 |-l B]]}
                \end{array}
              }
            }{[[D1; G1; X; Y; G2; D2 |-l B]]}
          }{[[D1; G1; X (*) Y; G2; D2 |-l B]]}
        \end{math}
      \end{center}
\end{itemize}

\subsubsection{Left introduction of the non-commutative tensor products $\tri$}
\begin{center}
  \scriptsize
  $\Pi_1$:
  \begin{math}
    $$\mprset{flushleft}
    \inferrule* [right={\tiny tenL2}] {
      {
        \begin{array}{c}
          \pi \\
          {[[G1; A1; A2; G2 |-l B]]}
        \end{array}
      }
    }{[[G1; A1 (>) A2; G2 |-l B]]}
  \end{math}
  \qquad\qquad
  \begin{math}
    \begin{array}{c}
      \Pi_2 \\
      {[[D1; B; D2 |-l C]]}
    \end{array}
  \end{math}
\end{center}
By assumption, $c(\Pi_1),c(\Pi_2)\leq |B|$. By induction, there is a
proof $\Pi'$ from $\pi$ and $\Pi_2$ for sequent \\
$[[D1; G1; A1; A2; G2; D2 |-l C]]$ s.t. $c(\Pi')\leq |B|$.
Therefore, the proof $\Pi$ can be constructed as follows with
$c(\Pi)\leq |B|$.
\begin{center}
  \scriptsize
  \begin{math}
    $$\mprset{flushleft}
    \inferrule* [right={\tiny tenL2}] {
      $$\mprset{flushleft}
      \inferrule* [right={\tiny cut2}] {
        {
          \begin{array}{cc}
            \pi & \Pi_2 \\
            {[[G1; A1; A2; G2 |-l B]]} & {[[D1; B; D2 |-l C]]}
          \end{array}
        }
      }{[[D1; G1; A1; A2; G2; D2 |-l C]]}
    }{[[D1; G1; A1 (>) A2; G2; D2 |-l C]]}
  \end{math}
\end{center}



\subsubsection{Left introduction of the commutative unit $[[UnitT]]$}
\begin{itemize}
\item Case 1:
      \begin{center}
        \scriptsize
        $\Pi_1$:
        \begin{math}
          $$\mprset{flushleft}
          \inferrule* [right={\tiny unitL}] {
            {
              \begin{array}{c}
                \pi \\
                {[[I1, I2 |-c X]]}
              \end{array}
            }
          }{[[I1, UnitT, I2 |-c X]]}
        \end{math}
        \qquad\qquad
        \begin{math}
          \begin{array}{c}
            \Pi_2 \\
            {[[P1, X, P2 |-c Y]]}
          \end{array}
        \end{math}
      \end{center}
      By assumption, $c(\Pi_1),c(\Pi_2)\leq |X|$. By induction, there is a
      proof $\Pi'$ from $\pi$ and $\Pi_2$ for sequent
      $[[P1, I1, I2, P2 |-c Y]]$
      s.t. $c(\Pi')\leq |X|$. Therefore, the proof $\Pi$ can be constructed
      as follows, and $c(\Pi)=c(\Pi')\leq |X|$.
      \begin{center}
        \scriptsize
        \begin{math}
          $$\mprset{flushleft}
          \inferrule* [right={\tiny unitL}] {
            {
              \begin{array}{c}
                \Pi' \\
                {[[P1, I1, I2, P2 |-c Y]]}
              \end{array}
            }
          }{[[P1, I1, UnitT, I2, P2 |-c Y]]}
        \end{math}
      \end{center}

\item Case 2:
      \begin{center}
        \scriptsize
        $\Pi_1$:
        \begin{math}
          $$\mprset{flushleft}
          \inferrule* [right={\tiny unitL}] {
            {
              \begin{array}{c}
                \pi \\
                {[[I1, I2 |-c X]]}
              \end{array}
            }
          }{[[I1, UnitT, I2 |-c X]]}
        \end{math}
        \qquad\qquad
        \begin{math}
          \begin{array}{c}
            \Pi_2 \\
            {[[G1; X; G2 |-l A]]}
          \end{array}
        \end{math}
      \end{center}
      By assumption, $c(\Pi_1),c(\Pi_2)\leq |X|$. By induction, there is a
      proof $\Pi'$ from $\pi$ and $\Pi_2$ for sequent
      $[[G1; I1; I2; G2 |-l A]]$
      s.t. $c(\Pi')\leq |X|$. Therefore, the proof $\Pi$ can be constructed
      as follows, and $c(\Pi)=c(\Pi')\leq |X|$.
      \begin{center}
        \scriptsize
        \begin{math}
          $$\mprset{flushleft}
          \inferrule* [right={\tiny unitL}] {
            {
              \begin{array}{c}
                \Pi' \\
                {[[G1; I1; I2; G2 |-l A]]}
              \end{array}
            }
          }{[[G1; I1; UnitT; I2; G2 |-l A]]}
        \end{math}
      \end{center}

\item Case 3:
      \begin{center}
        \scriptsize
        $\Pi_1$:
        \begin{math}
          $$\mprset{flushleft}
          \inferrule* [right={\tiny unitL}] {
            {
              \begin{array}{c}
                \pi \\
                {[[D1; D2 |-l A]]}
              \end{array}
            }
          }{[[D1; UnitT; D2 |-l A]]}
        \end{math}
        \qquad\qquad
        \begin{math}
          \begin{array}{c}
            \Pi_2 \\
            {[[G1; A; G2 |-l B]]}
          \end{array}
        \end{math}
      \end{center}
      By assumption, $c(\Pi_1),c(\Pi_2)\leq |X|$. By induction, there is a
      proof $\Pi'$ from $\pi$ and $\Pi_2$ for sequent
      $[[G1; D1; D2; G2 |-l B]]$
      s.t. $c(\Pi')\leq |X|$. Therefore, the proof $\Pi$ can be constructed
      as follows, and $c(\Pi)=c(\Pi')\leq |X|$.
      \begin{center}
        \scriptsize
        \begin{math}
          $$\mprset{flushleft}
          \inferrule* [right={\tiny unitL}] {
            {
              \begin{array}{c}
                \Pi' \\
                {[[G1; D1; D2; G2 |-l B]]}
              \end{array}
            }
          }{[[G1; D1; UnitT; D2; G2 |-l B]]}
        \end{math}
      \end{center}
\end{itemize}



\subsubsection{Left introduction of the non-commutative unit $[[UnitS]]$}
\begin{center}
  \scriptsize
  $\Pi_1$:
  \begin{math}
    $$\mprset{flushleft}
    \inferrule* [right={\tiny unitL}] {
      {
        \begin{array}{c}
          \pi \\
          {[[D1; D2 |-l A]]}
        \end{array}
      }
    }{[[D1; UnitS; D2 |-l A]]}
  \end{math}
  \qquad\qquad
  \begin{math}
    \begin{array}{c}
      \Pi_2 \\
      {[[G1; A; G2 |-l B]]}
    \end{array}
  \end{math}
\end{center}
By assumption, $c(\Pi_1),c(\Pi_2)\leq |X|$. By induction, there is a
proof $\Pi'$ from $\pi$ and $\Pi_2$ for sequent \\
$[[G1; D1; D2; G2 |-l B]]$ s.t. $c(\Pi')\leq |X|$. Therefore, the proof
$\Pi$ can be constructed as follows, and \\
$c(\Pi)=c(\Pi')\leq |X|$.
\begin{center}
  \scriptsize
  \begin{math}
    $$\mprset{flushleft}
    \inferrule* [right={\tiny unitL}] {
      {
        \begin{array}{c}
          \Pi' \\
          {[[G1; D1; D2; G2 |-l B]]}
        \end{array}
      }
    }{[[G1; D1; UnitS; D2; G2 |-l B]]}
  \end{math}
\end{center}



\subsubsection{Left introduction of the functor $F$}
\begin{center}
  \scriptsize
  $\Pi_1$:
  \begin{math}
    $$\mprset{flushleft}
    \inferrule* [right={\tiny FL}] {
      {
        \begin{array}{c}
          \pi_1 \\
          {[[G1; X; G2 |-l A]]}
        \end{array}
      }
    }{[[G1; F X; G2 |-l A]]}
  \end{math}
  \qquad\qquad
  \begin{math}
    \begin{array}{c}
      \Pi_2 \\
      {[[D1; A; D2 |-l B]]}
    \end{array}
  \end{math}
\end{center}
By assumption, $c(\Pi_1),c(\Pi_2)\leq |A|$. By induction, there is a
proof $\Pi'$ from $\pi_2$ and $\Pi_2$ for sequent
$[[D1; G1; X; G2; D2 |-l B]]$ s.t. $c(\Pi')\leq |A|$. Therefore, the
proof $\Pi$ can be constructed as follows with $c(\Pi)\leq |A|$.
\begin{center}
  \scriptsize
  \begin{math}
    $$\mprset{flushleft}
    \inferrule* [right={\tiny FL}] {
      $$\mprset{flushleft}
      \inferrule* [right={\tiny cut2}] {
        {
          \begin{array}{cc}
            \pi_2 & \Pi_2 \\
            {[[G1; X; G2 |-l A]]} & {[[D1; A; D2 |-l B]]}
          \end{array}
        }
      }{[[D1; G1; X; G2; D2 |-l B]]}
    }{[[D1; G1; F X; G2; D2 |-l B]]}
  \end{math}
\end{center}

\subsubsection{Left introduction of the functor $G$}
\begin{center}
  \scriptsize
  $\Pi_1$:
  \begin{math}
    $$\mprset{flushleft}
    \inferrule* [right={\tiny GL}] {
      {
        \begin{array}{c}
          \pi_1 \\
          {[[G1; A; G2 |-l B]]}
        \end{array}
      }
    }{[[G1; Gf A; G2 |-l B]]}
  \end{math}
  \qquad\qquad
  \begin{math}
    \begin{array}{c}
      \Pi_2 \\
      {[[D1; B; D2 |-l C]]}
    \end{array}
  \end{math}
\end{center}
By assumption, $c(\Pi_1),c(\Pi_2)\leq |B|$. By induction, there is a
proof $\Pi'$ from $\pi_2$ and $\Pi_2$ for sequent
$[[D1; G1; A; G2; D2 |-l C]]$ s.t. $c(\Pi')\leq |B|$. Therefore, the
proof $\Pi$ can be constructed as follows with $c(\Pi)\leq |B|$.
\begin{center}
  \scriptsize
  \begin{math}
    $$\mprset{flushleft}
    \inferrule* [right={\tiny GL}] {
      $$\mprset{flushleft}
      \inferrule* [right={\tiny cut2}] {
        {
          \begin{array}{cc}
            \pi_2 & \Pi_2 \\
            {[[G1; A; G2 |-l B]]} & {[[D1; B; D2 |-l C]]}
          \end{array}
        }
      }{[[D1; G1; A; G2; D2 |-l C]]}
    }{[[D1; G1; Gf A; G2; D2 |-l C]]}
  \end{math}
\end{center}



\subsection{Secondary Hypothesis}

\subsubsection{Right introduction of the commutative tensor product $\otimes$}
\begin{itemize}
\item Case 1:
      \begin{center}
        \scriptsize
        \begin{math}
          \begin{array}{c}
            \Pi_1 \\
            {[[I2 |-c X]]}
          \end{array}
        \end{math}
        \qquad\qquad
        $\Pi_2$:
        \begin{math}
          $$\mprset{flushleft}
          \inferrule* [right={\tiny tenR}] {
            {
              \begin{array}{cc}
                \pi_1 & \pi_2 \\
                {[[P1, X, P2 |-c Y1]]} & {[[I1 |-c Y2]]}
              \end{array}
            }
          }{[[P1, X, P2, I1 |-c Y1 (*) Y2]]}
        \end{math}
      \end{center}
      By assumption, $c(\Pi_1),c(\Pi_2)\leq |X|$. By induction on $\Pi_1$
      and $\pi_1$, there is a proof $\Pi'$ for sequent
      $[[P1, I2, P2 |-c Y1]]$ s.t. $c(\Pi') \leq |X|$. Therefore, the proof
      $\Pi$ can be constructed as follows with $c(\Pi) = c(\Pi') \leq |X|$.
      \begin{center}
        \scriptsize
        \begin{math}
          $$\mprset{flushleft}
          \inferrule* [right={\tiny tenR}] {
            {
              \begin{array}{cc}
                \Pi' & \pi_1 \\
                {[[P1, I2, P2 |-c Y1]]} & {[[I1 |-c Y2]]}
              \end{array}
            }
          }{[[P1, I2, P2, I1 |-c Y1 (*) Y2]]}
        \end{math}
      \end{center}

\item Case 2:
      \begin{center}
        \scriptsize
        \begin{math}
          \begin{array}{c}
            \Pi_1 \\
            {[[I2 |-c X]]}
          \end{array}
        \end{math}
        \qquad\qquad
        $\Pi_2$:
        \begin{math}
          $$\mprset{flushleft}
          \inferrule* [right={\tiny tenR}] {
            {
              \begin{array}{cc}
                \pi_1 & \pi_2 \\
                {[[I1 |-c Y1]]} & {[[P1, X, P2 |-c Y2]]}
              \end{array}
            }
          }{[[I1, P1, X, P2 |-c Y1 (*) Y2]]}
        \end{math}
      \end{center}
      By assumption, $c(\Pi_1),c(\Pi_2)\leq |X|$. By induction on $\Pi_1$
      and $\pi_2$, there is a proof $\Pi'$ for sequent
      $[[P1, I2, P2 |-c Y2]]$ s.t. $c(\Pi') \leq |X|$. Therefore, the proof
      $\Pi$ can be constructed as follows with $c(\Pi) = c(\Pi') \leq |X|$.
      \begin{center}
        \scriptsize
        \begin{math}
          $$\mprset{flushleft}
          \inferrule* [right={\tiny tenR}] {
            {
              \begin{array}{cc}
                \pi_1 & \Pi' \\
                {[[I1 |-c Y1]]} & {[[P1, I2, P2 |-c Y2]]}
              \end{array}
            }
          }{[[I1, P1, I2, P2 |-c Y1 (*) Y2]]}
        \end{math}
      \end{center}
\end{itemize}



\subsubsection{Right introduction of the non-commutative tensor product $\tri$}
\begin{itemize}
\item Case 1:
      \begin{center}
        \scriptsize
        \begin{math}
          \begin{array}{c}
            \Pi_1 \\
            {[[I |-c X]]}
          \end{array}
        \end{math}
        \qquad\qquad
        $\Pi_2$:
        \begin{math}
          $$\mprset{flushleft}
          \inferrule* [right={\tiny tenR}] {
            {
              \begin{array}{cc}
                \pi_1 & \pi_2 \\
                {[[G1; X; G2 |-l A]]} & {[[G3 |-l B]]}
              \end{array}
            }
          }{[[G1; X; G2; G3 |-l A (>) B]]}
        \end{math}
      \end{center}
      By assumption, $c(\Pi_1),c(\Pi_2)\leq |X|$. By induction on $\Pi_1$
      and $\pi_1$, there is a proof $\Pi'$ for sequent
      $[[G1; I; G2 |-l A]]$ s.t. $c(\Pi') \leq |X|$. Therefore, the proof
      $\Pi$ can be constructed as follows with $c(\Pi) = c(\Pi') \leq |X|$.
      \begin{center}
        \scriptsize
        \begin{math}
          $$\mprset{flushleft}
          \inferrule* [right={\tiny tenR}] {
            {
              \begin{array}{cc}
                \Pi' & \pi_1 \\
                {[[G1; I; G2 |-l A]]} & {[[G3 |-l B]]}
              \end{array}
            }
          }{[[G1; I; G2; G3 |-l A (>) B]]}
        \end{math}
      \end{center}

\item Case 2:
      \begin{center}
        \scriptsize
        \begin{math}
          \begin{array}{c}
            \Pi_1 \\
            {[[D |-l C]]}
          \end{array}
        \end{math}
        \qquad\qquad
        $\Pi_2$:
        \begin{math}
          $$\mprset{flushleft}
          \inferrule* [right={\tiny tenR}] {
            {
              \begin{array}{cc}
                \pi_1 & \pi_2 \\
                {[[G1; C; G2 |-l A]]} & {[[G3 |-l B]]}
              \end{array}
            }
          }{[[G1; C; G2; G3 |-l A (>) B]]}
        \end{math}
      \end{center}
      By assumption, $c(\Pi_1),c(\Pi_2)\leq |C|$. By induction on $\Pi_1$
      and $\pi_1$, there is a proof $\Pi'$ for sequent
      $[[G1; D; G2 |-l A]]$ s.t. $c(\Pi') \leq |C|$. Therefore, the proof
      $\Pi$ can be constructed as follows with $c(\Pi) = c(\Pi') \leq |C|$.
      \begin{center}
        \scriptsize
        \begin{math}
          $$\mprset{flushleft}
          \inferrule* [right={\tiny tenR}] {
            {
              \begin{array}{cc}
                \Pi' & \pi_1 \\
                {[[G1; D; G2 |-l A]]} & {[[G3 |-l B]]}
              \end{array}
            }
          }{[[G1; D; G2; G3 |-l A (>) B]]}
        \end{math}
      \end{center}

\item Case 3:
      \begin{center}
        \scriptsize
        \begin{math}
          \begin{array}{c}
            \Pi_1 \\
            {[[I |-c X]]}
          \end{array}
        \end{math}
        \qquad\qquad
        $\Pi_2$:
        \begin{math}
          $$\mprset{flushleft}
          \inferrule* [right={\tiny tenR}] {
            {
              \begin{array}{cc}
                \pi_1 & \pi_2 \\
                {[[G1 |-l A]]} & {[[G2; X; G3 |-l B]]}
              \end{array}
            }
          }{[[G1; G2; X; G3 |-l A (>) B]]}
        \end{math}
      \end{center}
      By assumption, $c(\Pi_1),c(\Pi_2)\leq |X|$. By induction on $\Pi_1$
      and $\pi_2$, there is a proof $\Pi'$ for sequent
      $[[G2; I; G3 |-l B]]$ s.t. $c(\Pi') \leq |X|$. Therefore, the proof
      $\Pi$ can be constructed as follows with $c(\Pi) = c(\Pi') \leq |X|$.
      \begin{center}
        \scriptsize
        \begin{math}
          $$\mprset{flushleft}
          \inferrule* [right={\tiny tenR}] {
            {
              \begin{array}{cc}
                \pi_1 & \Pi' \\
                {[[G1 |-l A]]} & {[[G2; I; G3 |-l B]]}
              \end{array}
            }
          }{[[G1; G2; I; G3 |-l A (>) B]]}
        \end{math}
      \end{center}

\item Case 4:
      \begin{center}
        \scriptsize
        \begin{math}
          \begin{array}{c}
            \Pi_1 \\
            {[[D |-l C]]}
          \end{array}
        \end{math}
        \qquad\qquad
        $\Pi_2$:
        \begin{math}
          $$\mprset{flushleft}
          \inferrule* [right={\tiny tenR}] {
            {
              \begin{array}{cc}
                \pi_1 & \pi_2 \\
                {[[G1 |-l A]]} & {[[G2; C; G3 |-l B]]}
              \end{array}
            }
          }{[[G1; G2; C; G3 |-l A (>) B]]}
        \end{math}
      \end{center}
      By assumption, $c(\Pi_1),c(\Pi_2)\leq |C|$. By induction on $\Pi_1$
      and $\pi_2$, there is a proof $\Pi'$ for sequent
      $[[G2; D; G3 |-l B]]$ s.t. $c(\Pi') \leq |C|$. Therefore, the proof
      $\Pi$ can be constructed as follows with $c(\Pi) = c(\Pi') \leq |C|$.
      \begin{center}
        \scriptsize
        \begin{math}
          $$\mprset{flushleft}
          \inferrule* [right={\tiny tenR}] {
            {
              \begin{array}{cc}
                \pi_1 & \Pi' \\
                {[[G1 |-l A]]} & {[[G2; D; G3 |-l B]]}
              \end{array}
            }
          }{[[G1; G2; D; G3 |-l A (>) B]]}
        \end{math}
      \end{center}
\end{itemize}



\subsubsection{Left introduction of the commutative implication $\multimap$}
\begin{itemize}
\item Case 1:
      \begin{center}
        \scriptsize
        \begin{math}
          \begin{array}{c}
            \Pi_1 \\
            {[[I |-c X]]}
          \end{array}
        \end{math}
        \qquad\qquad
        $\Pi_2$:
        \begin{math}
          $$\mprset{flushleft}
          \inferrule* [right={\tiny impL}] {
            {
              \begin{array}{cc}
                \pi_1 & \pi_2 \\
                {[[P2, X, P3 |-c Y1]]} & {[[P1, Y2, P4 |-c Z]]}
              \end{array}
            }
          }{[[P1, Y1 -o Y2, P2, X, P3, P4 |-c Z]]}
        \end{math}
      \end{center}
      By assumption, $c(\Pi_1),c(\Pi_2)\leq |X|$. By induction on $\Pi_1$ and $\pi_1$, there is
      a proof $\Pi'$ for sequent $[[P2, I, P3 |-c Y1]]$ s.t. $c(\Pi') \leq |X|$. Therefore, the
      proof $\Pi$ can be constructed as follows with $c(\Pi) = c(\Pi') \leq |X|$.
      \begin{center}
        \scriptsize
        \begin{math}
          $$\mprset{flushleft}
          \inferrule* [right={\tiny impL}] {
            {
              \begin{array}{cc}
                \Pi' & \pi_2 \\
                {[[P2, I, P3 |-c Y1]]} & {[[P1, Y2, P4 |-c Z]]}
              \end{array}
            }
          }{[[P1, Y1 -o Y2, P2, I, P3, P4 |-c Z]]}
        \end{math}
      \end{center}

\item Case 2:
      \begin{center}
        \scriptsize
        \begin{math}
          \begin{array}{c}
            \Pi_1 \\
            {[[I |-c X]]}
          \end{array}
        \end{math}
        \qquad\qquad
        $\Pi_2$:
        \begin{math}
          $$\mprset{flushleft}
          \inferrule* [right={\tiny impL}] {
            {
              \begin{array}{cc}
                \pi_1 & \pi_2 \\
                {[[P3 |-c Y1]]} & {[[P1, X, P2, Y2, P4 |-c Z]]}
              \end{array}
            }
          }{[[P1, X, P2, Y1 -o Y2, P3, P4 |-c Z]]}
        \end{math}
      \end{center}
      By assumption, $c(\Pi_1),c(\Pi_2)\leq |X|$. By induction on $\Pi_1$ and $\pi_2$, there is
      a proof $\Pi'$ for sequent $[[P1, I, P2, Y2, P4 |-c Z]]$ s.t. $c(\Pi') \leq |X|$.
      Therefore, the proof $\Pi$ can be constructed as follows with
      $c(\Pi) = c(\Pi') \leq |X|$.
      \begin{center}
        \scriptsize
        \begin{math}
          $$\mprset{flushleft}
          \inferrule* [right={\tiny impL}] {
            {
              \begin{array}{cc}
                \pi_1 & \Pi' \\
                {[[P3 |-c Y1]]} & {[[P1, I, P2, Y2, P4 |-c Z]]}
              \end{array}
            }
          }{[[P1, I1, P2, Y1 -o Y2, P3, P4 |-c Z]]}
        \end{math}
      \end{center}

\item Case 3:
      \begin{center}
        \scriptsize
        \begin{math}
          \begin{array}{c}
            \Pi_1 \\
            {[[I |-c X]]}
          \end{array}
        \end{math}
        \qquad\qquad
        $\Pi_2$:
        \begin{math}
          $$\mprset{flushleft}
          \inferrule* [right={\tiny impL}] {
            {
              \begin{array}{cc}
                \pi_1 & \pi_2 \\
                {[[P2 |-c Y1]]} & {[[P1, Y2, P3, X, P4 |-c Z]]}
              \end{array}
            }
          }{[[P1, Y1 -o Y2, P2, P3, X, P4 |-c Z]]}
        \end{math}
      \end{center}
      By assumption, $c(\Pi_1),c(\Pi_2)\leq |X|$. By induction on $\Pi_1$
      and $\pi_2$, there is a proof $\Pi'$ for sequent
      $[[P1, I, P2, Y2, P4 |-c Z]]$ s.t. $c(\Pi') \leq |X|$. Therefore,
      the proof $\Pi$ can be constructed as follows with
      $c(\Pi) = c(\Pi') \leq |X|$.
      \begin{center}
        \scriptsize
        \begin{math}
          $$\mprset{flushleft}
          \inferrule* [right={\tiny impL}] {
            {
              \begin{array}{cc}
                \pi_1 & \Pi' \\
                {[[P2 |-c Y1]]} & {[[P1, Y2, P3, I, P4 |-c Z]]}
              \end{array}
            }
          }{[[P1, Y1 -o Y2, P2, P3, I, P4 |-c Z]]}
        \end{math}
      \end{center}

\item Case 4:
      \begin{center}
        \scriptsize
        \begin{math}
          \begin{array}{c}
            \Pi_1 \\
            {[[I |-c X]]}
          \end{array}
        \end{math}
        \qquad\qquad
        $\Pi_2$:
        \begin{math}
          $$\mprset{flushleft}
          \inferrule* [right={\tiny impL}] {
            {
              \begin{array}{cc}
                \pi_1 & \pi_2 \\
                {[[P1, X, P2 |-c Y1]]} & {[[G1; Y2; G2 |-l A]]}
              \end{array}
            }
          }{[[G1; Y1 -o Y2; P1; X; P2; G2 |-l A]]}
        \end{math}
      \end{center}
      By assumption, $c(\Pi_1),c(\Pi_2)\leq |X|$. By induction on $\Pi_1$
      and $\pi_1$, there is a proof $\Pi'$ for sequent
      $[[P1, I, P2 |-c Y1]]$ s.t. $c(\Pi') \leq |X|$. Therefore, the proof
      $\Pi$ can be constructed as follows with $c(\Pi) = c(\Pi') \leq |X|$.
      \begin{center}
        \scriptsize
        \begin{math}
          $$\mprset{flushleft}
          \inferrule* [right={\tiny impL}] {
            {
              \begin{array}{cc}
                \Pi' & \pi_2 \\
                {[[P1, I, P2 |-c Y1]]} & {[[G1; Y2; G2 |-l A]]}
              \end{array}
            }
          }{[[G1; Y1 -o Y2; P1; I; P2; G2 |-l A]]}
        \end{math}
      \end{center}

\item Case 5:
      \begin{center}
        \scriptsize
        \begin{math}
          \begin{array}{c}
            \Pi_1 \\
            {[[I |-c X]]}
          \end{array}
        \end{math}
        \qquad\qquad
        $\Pi_2$:
        \begin{math}
          $$\mprset{flushleft}
          \inferrule* [right={\tiny impL}] {
            {
              \begin{array}{cc}
                \pi_1 & \pi_2 \\
                {[[P |-c Y1]]} & {[[G1; X; G2; Y2; G3 |-l A]]}
              \end{array}
            }
          }{[[G1; X; G2; Y1 -o Y2; P; G3 |-l A]]}
        \end{math}
      \end{center}
      By assumption, $c(\Pi_1),c(\Pi_2)\leq |X|$. By induction on $\Pi_1$
      and $\pi_2$, there is a proof $\Pi'$ for sequent
      $[[G1; I; G2; Y2; G3 |-l A]]$ s.t. $c(\Pi') \leq |X|$. Therefore, the
      proof $\Pi$ can be constructed as follows with
      $c(\Pi) = c(\Pi') \leq |X|$.
      \begin{center}
        \scriptsize
        \begin{math}
          $$\mprset{flushleft}
          \inferrule* [right={\tiny impL}] {
            {
              \begin{array}{cc}
                \pi_1 & \Pi' \\
                {[[P |-c Y1]]} & {[[G1; I; G2; Y2; G3 |-l A]]}
              \end{array}
            }
          }{[[G1; I; G2; Y1 -o Y2; P; G3 |-l A]]}
        \end{math}
      \end{center}

\item Case 6:
      \begin{center}
        \scriptsize
        \begin{math}
          \begin{array}{c}
            \Pi_1 \\
            {[[D |-l B]]}
          \end{array}
        \end{math}
        \qquad\qquad
        $\Pi_2$:
        \begin{math}
          $$\mprset{flushleft}
          \inferrule* [right={\tiny impL}] {
            {
              \begin{array}{cc}
                \pi_1 & \pi_2 \\
                {[[P |-c Y1]]} & {[[G1; B; G2; Y2; G3 |-l A]]}
              \end{array}
            }
          }{[[G1; B; G2; Y1 -o Y2; P; G3 |-l A]]}
        \end{math}
      \end{center}
      By assumption, $c(\Pi_1),c(\Pi_2)\leq |B|$. By induction on $\Pi_1$
      and $\pi_2$, there is a proof $\Pi'$ for sequent
      $[[G1; D; G2; Y2; G3 |-l A]]$ s.t. $c(\Pi') \leq |B|$. Therefore, the
      proof $\Pi$ can be constructed as follows with
      $c(\Pi) = c(\Pi') \leq |B|$.
      \begin{center}
        \scriptsize
        \begin{math}
          $$\mprset{flushleft}
          \inferrule* [right={\tiny impL}] {
            {
              \begin{array}{cc}
                \pi_1 & \Pi' \\
                {[[P |-c Y1]]} & {[[G1; D; G2; Y2; G3 |-l A]]}
              \end{array}
            }
          }{[[G1; D; G2; Y1 -o Y2; P; G3 |-l A]]}
        \end{math}
      \end{center}

\item Case 7:
      \begin{center}
        \scriptsize
        \begin{math}
          \begin{array}{c}
            \Pi_1 \\
            {[[I |-c X]]}
          \end{array}
        \end{math}
        \qquad\qquad
        $\Pi_2$:
        \begin{math}
          $$\mprset{flushleft}
          \inferrule* [right={\tiny impL}] {
            {
              \begin{array}{cc}
                \pi_1 & \pi_2 \\
                {[[P |-c Y1]]} & {[[G1; Y2; G2; X; G3 |-l A]]}
              \end{array}
            }
          }{[[G1; Y1 -o Y2; P; G2; X; G3 |-l A]]}
        \end{math}
      \end{center}
      By assumption, $c(\Pi_1),c(\Pi_2)\leq |X|$. By induction on $\Pi_1$
      and $\pi_2$, there is a proof $\Pi'$ for sequent
      $[[G1; Y2; G2; I; G3 |-l A]]$ s.t. $c(\Pi') \leq |X|$. Therefore, the
      proof $\Pi$ can be constructed as follows with
      $c(\Pi) = c(\Pi') \leq |X|$.
      \begin{center}
        \scriptsize
        \begin{math}
          $$\mprset{flushleft}
          \inferrule* [right={\tiny impL}] {
            {
              \begin{array}{cc}
                \pi_1 & \Pi' \\
                {[[P |-c Y1]]} & {[[G1; Y2; G2; I; G3 |-l A]]}
              \end{array}
            }
          }{[[G1; Y1 -o Y2; P; G2; I; G3 |-l A]]}
        \end{math}
      \end{center}

\item Case 8:
      \begin{center}
        \scriptsize
        \begin{math}
          \begin{array}{c}
            \Pi_1 \\
            {[[D |-l B]]}
          \end{array}
        \end{math}
        \qquad\qquad
        $\Pi_2$:
        \begin{math}
          $$\mprset{flushleft}
          \inferrule* [right={\tiny impL}] {
            {
              \begin{array}{cc}
                \pi_1 & \pi_2 \\
                {[[P |-c Y1]]} & {[[G1; Y2; G2; B; G3 |-l A]]}
              \end{array}
            }
          }{[[G1; Y1 -o Y2; P; G2; B; G3 |-l A]]}
        \end{math}
      \end{center}
      By assumption, $c(\Pi_1),c(\Pi_2)\leq |B|$. By induction on $\Pi_1$
      and $\pi_2$, there is a proof $\Pi'$ for sequent
      $[[G1; Y2; G2; D; G3 |-l A]]$ s.t. $c(\Pi') \leq |B|$. Therefore,
      the proof $\Pi$ can be constructed as follows with
      $c(\Pi) = c(\Pi') \leq |B|$.
      \begin{center}
        \scriptsize
        \begin{math}
          $$\mprset{flushleft}
          \inferrule* [right={\tiny impL}] {
            {
              \begin{array}{cc}
                \pi_1 & \Pi' \\
                {[[P |-c Y1]]} & {[[G1; Y2; G2; D; G3 |-l A]]}
              \end{array}
            }
          }{[[G1; Y1 -o Y2; P; G2; D; G3 |-l A]]}
        \end{math}
      \end{center}
\end{itemize}



\subsubsection{Left introduction of the non-commutative left implication $\lto$}
\begin{itemize}
\item Case 1:
      \begin{center}
        \scriptsize
        \begin{math}
          \begin{array}{c}
            \Pi_1 \\
            {[[I |-c X]]}
          \end{array}
        \end{math}
        \qquad\qquad
        $\Pi_2$:
        \begin{math}
          $$\mprset{flushleft}
          \inferrule* [right={\tiny imprL}] {
            {
              \begin{array}{cc}
                \pi_1 & \pi_2 \\
                {[[D1; X; D2 |-l A1]]} & {[[G1; A2; G2 |-l B]]}
              \end{array}
            }
          }{[[G1; A1 -> A2; D1; X; D2; G2 |-l B]]}
        \end{math}
      \end{center}
      By assumption, $c(\Pi_1),c(\Pi_2)\leq |X|$. By induction on $\Pi_1$
      and $\pi_1$, there is a proof $\Pi'$ for sequent
      $[[D1; I; D2 |-l A1]]$ s.t. $c(\Pi') \leq |X|$. Therefore, the proof
      $\Pi$ can be constructed as follows with $c(\Pi) = c(\Pi') \leq |X|$.
      \begin{center}
        \scriptsize
        \begin{math}
          $$\mprset{flushleft}
          \inferrule* [right={\tiny impL}] {
            {
              \begin{array}{cc}
                \Pi' & \pi_2 \\
                {[[D1; I; D2 |-l A1]]} & {[[G1; A2; G2 |-l B]]}
              \end{array}
            }
          }{[[G1; A1 -> A2; D1; I; D2; G2 |-l B]]}
        \end{math}
      \end{center}

\item Case 2:
      \begin{center}
        \scriptsize
        \begin{math}
          \begin{array}{c}
            \Pi_1 \\
            {[[G |-l C]]}
          \end{array}
        \end{math}
        \qquad\qquad
        $\Pi_2$:
        \begin{math}
          $$\mprset{flushleft}
          \inferrule* [right={\tiny imprL}] {
            {
              \begin{array}{cc}
                \pi_1 & \pi_2 \\
                {[[D1; C; D2 |-l A1]]} & {[[G1; A2; G2 |-l B]]}
              \end{array}
            }
          }{[[G1; A1 -> A2; D1; C; D2; G2 |-l B]]}
        \end{math}
      \end{center}
      By assumption, $c(\Pi_1),c(\Pi_2)\leq |C|$. By induction on $\Pi_1$
      and $\pi_1$, there is a proof $\Pi'$ for sequent
      $[[D1; G; D2 |-l A1]]$ s.t. $c(\Pi') \leq |C|$. Therefore, the proof
      $\Pi$ can be constructed as follows with $c(\Pi) = c(\Pi') \leq |C|$.
      \begin{center}
        \scriptsize
        \begin{math}
          $$\mprset{flushleft}
          \inferrule* [right={\tiny imprL}] {
            {
              \begin{array}{cc}
                \Pi' & \pi_2 \\
                {[[D1; G; D2 |-l A1]]} & {[[G1; A2; G2 |-l B]]}
              \end{array}
            }
          }{[[G1; A1 -> A2; D1; G; D2; G2 |-l B]]}
        \end{math}
      \end{center}

\item Case 3:
      \begin{center}
        \scriptsize
        \begin{math}
          \begin{array}{c}
            \Pi_1 \\
            {[[I |-c X]]}
          \end{array}
        \end{math}
        \qquad\qquad
        $\Pi_2$:
        \begin{math}
          $$\mprset{flushleft}
          \inferrule* [right={\tiny imprL}] {
            {
              \begin{array}{cc}
                \pi_1 & \pi_2 \\
                {[[D |-l A1]]} & {[[G1; X; G2; A2; G3 |-l B]]}
              \end{array}
            }
          }{[[G1; X; G2; A1 -> A2; D; G3 |-l B]]}
        \end{math}
      \end{center}
      By assumption, $c(\Pi_1),c(\Pi_2)\leq |X|$. By induction on $\Pi_1$
      and $\pi_2$, there is a proof $\Pi'$ for sequent
      $[[G1; I; G2; A2; G3 |-l B]]$ s.t. $c(\Pi') \leq |X|$. Therefore,
      the proof $\Pi$ can be constructed as follows with
      $c(\Pi) = c(\Pi') \leq |X|$.
      \begin{center}
        \scriptsize
        \begin{math}
          $$\mprset{flushleft}
          \inferrule* [right={\tiny imprL}] {
            {
              \begin{array}{cc}
                \pi_1 & \Pi' \\
                {[[D |-l A1]]} & {[[G1; I; G2; A2; G3 |-l B]]}
              \end{array}
            }
          }{[[G1; I; G2; A1 -> A2; D; G3 |-l B]]}
        \end{math}
      \end{center}

\item Case 4:
      \begin{center}
        \scriptsize
        \begin{math}
          \begin{array}{c}
            \Pi_1 \\
            {[[D1 |-l B]]}
          \end{array}
        \end{math}
        \qquad\qquad
        $\Pi_2$:
        \begin{math}
          $$\mprset{flushleft}
          \inferrule* [right={\tiny imprL}] {
            {
              \begin{array}{cc}
                \pi_1 & \pi_2 \\
                {[[D2 |-l A1]]} & {[[G1; B; G2; A2; G3 |-l C]]}
              \end{array}
            }
          }{[[G1; B; G2; A1 -> A2; D2; G3 |-l C]]}
        \end{math}
      \end{center}
      By assumption, $c(\Pi_1),c(\Pi_2)\leq |B|$. By induction on $\Pi_1$
      and $\pi_2$, there is a proof $\Pi'$ for sequent
      $[[G1; D1; G2; A2; G3 |-l C]]$ s.t. $c(\Pi') \leq |B|$. Therefore,
      the proof $\Pi$ can be constructed as follows with
      $c(\Pi) = c(\Pi') \leq |B|$.
      \begin{center}
        \scriptsize
        \begin{math}
          $$\mprset{flushleft}
          \inferrule* [right={\tiny imprL}] {
            {
              \begin{array}{cc}
                \pi_1 & \Pi' \\
                {[[D2 |-l A1]]} & {[[G1; D1; G2; A2; G3 |-l C]]}
              \end{array}
            }
          }{[[G1; D1; G2; A1 -> A2; D2; G3 |-l C]]}
        \end{math}
      \end{center}

\item Case 5:
      \begin{center}
        \scriptsize
        \begin{math}
          \begin{array}{c}
            \Pi_1 \\
            {[[I |-c X]]}
          \end{array}
        \end{math}
        \qquad\qquad
        $\Pi_2$:
        \begin{math}
          $$\mprset{flushleft}
          \inferrule* [right={\tiny imprL}] {
            {
              \begin{array}{cc}
                \pi_1 & \pi_2 \\
                {[[D |-l A1]]} & {[[G1; A2; G2; X; G3 |-l B]]}
              \end{array}
            }
          }{[[G1; A1 -> A2; D; G2; X; G3 |-l B]]}
        \end{math}
      \end{center}
      By assumption, $c(\Pi_1),c(\Pi_2)\leq |X|$. By induction on $\Pi_1$
      and $\pi_2$, there is a proof $\Pi'$ for sequent
      $[[G1; A2; G2; I; G3 |-l B]]$ s.t. $c(\Pi') \leq |X|$. Therefore, the
      proof $\Pi$ can be constructed as follows with
      $c(\Pi) = c(\Pi') \leq |X|$.
      \begin{center}
        \scriptsize
        \begin{math}
          $$\mprset{flushleft}
          \inferrule* [right={\tiny imprL}] {
            {
              \begin{array}{cc}
                \pi_1 & \Pi' \\
                {[[D |-l A1]]} & {[[G1; A2; G2; I; G3 |-l B]]}
              \end{array}
            }
          }{[[G1; A1 -> A2; D; G2; I; G3 |-l B]]}
        \end{math}
      \end{center}

\item Case 6:
      \begin{center}
        \scriptsize
        \begin{math}
          \begin{array}{c}
            \Pi_1 \\
            {[[D1 |-l B]]}
          \end{array}
        \end{math}
        \qquad\qquad
        $\Pi_2$:
        \begin{math}
          $$\mprset{flushleft}
          \inferrule* [right={\tiny imprL}] {
            {
              \begin{array}{cc}
                \pi_1 & \pi_2 \\
                {[[D2 |-l A1]]} & {[[G1; A2; G2; B; G3 |-l C]]}
              \end{array}
            }
          }{[[G1; A1 -> A2; D2; G2; B; G3 |-l C]]}
        \end{math}
      \end{center}
      By assumption, $c(\Pi_1),c(\Pi_2)\leq |B|$. By induction on $\Pi_1$
      and $\pi_2$, there is a proof $\Pi'$ for sequent
      $[[G1; A2; G2; D1; G3 |-l C]]$ s.t. $c(\Pi') \leq |B|$. Therefore,
      the proof $\Pi$ can be constructed as follows with
      $c(\Pi) = c(\Pi') \leq |B|$.
      \begin{center}
        \scriptsize
        \begin{math}
          $$\mprset{flushleft}
          \inferrule* [right={\tiny imprL}] {
            {
              \begin{array}{cc}
                \pi_1 & \Pi' \\
                {[[D2 |-l A1]]} & {[[G1; A2; G2; D1; G3 |-l C]]}
              \end{array}
            }
          }{[[G1; A1 -> A2; D2; G2; D1; G3 |-l C]]}
        \end{math}
      \end{center}
\end{itemize}


\subsubsection{Left introduction of the non-commutative right implication $\rto$}
\begin{itemize}
\item Case 1:
      \begin{center}
        \scriptsize
        \begin{math}
          \begin{array}{c}
            \Pi_1 \\
            {[[I |-c X]]}
          \end{array}
        \end{math}
        \qquad\qquad
        $\Pi_2$:
        \begin{math}
          $$\mprset{flushleft}
          \inferrule* [right={\tiny implL}] {
            {
              \begin{array}{cc}
                \pi_1 & \pi_2 \\
                {[[D1; X; D2 |-l A1]]} & {[[G1; A2; G2 |-l B]]}
              \end{array}
            }
          }{[[G1; D1; A2 <- A1; X; D2; G2 |-l B]]}
        \end{math}
      \end{center}
      By assumption, $c(\Pi_1),c(\Pi_2)\leq |X|$. By induction on $\Pi_1$
      and $\pi_1$, there is a proof $\Pi'$ for sequent
      $[[D1; I; D2 |-l A1]]$ s.t. $c(\Pi') \leq |X|$. Therefore, the proof
      $\Pi$ can be constructed as follows with $c(\Pi) = c(\Pi') \leq |X|$.
      \begin{center}
        \scriptsize
        \begin{math}
          $$\mprset{flushleft}
          \inferrule* [right={\tiny implL}] {
            {
              \begin{array}{cc}
                \Pi' & \pi_2 \\
                {[[D1; I; D2 |-l A1]]} & {[[G1; A2; G2 |-l B]]}
              \end{array}
            }
          }{[[G1; D1; A2 <- A1; I; D2; G2 |-l B]]}
        \end{math}
      \end{center}

\item Case 2:
      \begin{center}
        \scriptsize
        \begin{math}
          \begin{array}{c}
            \Pi_1 \\
            {[[G |-l C]]}
          \end{array}
        \end{math}
        \qquad\qquad
        $\Pi_2$:
        \begin{math}
          $$\mprset{flushleft}
          \inferrule* [right={\tiny implL}] {
            {
              \begin{array}{cc}
                \pi_1 & \pi_2 \\
                {[[D1; C; D2 |-l A1]]} & {[[G1; A2; G2 |-l B]]}
              \end{array}
            }
          }{[[G1; D1; C; D2; A2 <- A1; G2 |-l B]]}
        \end{math}
      \end{center}
      By assumption, $c(\Pi_1),c(\Pi_2)\leq |C|$. By induction on $\Pi_1$
      and $\pi_1$, there is a proof $\Pi'$ for sequent
      $[[D1; G; D2 |-l A1]]$ s.t. $c(\Pi') \leq |C|$. Therefore, the proof
      $\Pi$ can be constructed as follows with $c(\Pi) = c(\Pi') \leq |C|$.
      \begin{center}
        \scriptsize
        \begin{math}
          $$\mprset{flushleft}
          \inferrule* [right={\tiny implL}] {
            {
              \begin{array}{cc}
                \Pi' & \pi_2 \\
                {[[D1; G; D2 |-l A1]]} & {[[G1; A2; G2 |-l B]]}
              \end{array}
            }
          }{[[G1; D1; G; D2; A2 <- A1; G2 |-l B]]}
        \end{math}
      \end{center}

\item Case 3:
      \begin{center}
        \scriptsize
        \begin{math}
          \begin{array}{c}
            \Pi_1 \\
            {[[I |-c X]]}
          \end{array}
        \end{math}
        \qquad\qquad
        $\Pi_2$:
        \begin{math}
          $$\mprset{flushleft}
          \inferrule* [right={\tiny implL}] {
            {
              \begin{array}{cc}
                \pi_1 & \pi_2 \\
                {[[D |-l A1]]} & {[[G1; X; G2; A2; G3 |-l B]]}
              \end{array}
            }
          }{[[G1; X; G2; D; A2 <- A1; G3 |-l B]]}
        \end{math}
      \end{center}
      By assumption, $c(\Pi_1),c(\Pi_2)\leq |X|$. By induction on $\Pi_1$
      and $\pi_2$, there is a proof $\Pi'$ for sequent
      $[[G1; I; G2; A2; G3 |-l B]]$ s.t. $c(\Pi') \leq |X|$. Therefore, the
      proof $\Pi$ can be constructed as follows with
      $c(\Pi) = c(\Pi') \leq |X|$.
      \begin{center}
        \scriptsize
        \begin{math}
          $$\mprset{flushleft}
          \inferrule* [right={\tiny implL}] {
            {
              \begin{array}{cc}
                \pi_1 & \Pi' \\
                {[[D |-l A1]]} & {[[G1; I; G2; A2; G3 |-l B]]}
              \end{array}
            }
          }{[[G1; I; G2; D; A2 <- A1; G3 |-l B]]}
        \end{math}
      \end{center}

\item Case 4:
      \begin{center}
        \scriptsize
        \begin{math}
          \begin{array}{c}
            \Pi_1 \\
            {[[D1 |-l B]]}
          \end{array}
        \end{math}
        \qquad\qquad
        $\Pi_2$:
        \begin{math}
          $$\mprset{flushleft}
          \inferrule* [right={\tiny implL}] {
            {
              \begin{array}{cc}
                \pi_1 & \pi_2 \\
                {[[D2 |-l A1]]} & {[[G1; B; G2; A2; G3 |-l C]]}
              \end{array}
            }
          }{[[G1; B; G2; D2; A2 <- A1; G3 |-l C]]}
        \end{math}
      \end{center}
      By assumption, $c(\Pi_1),c(\Pi_2)\leq |B|$. By induction on $\Pi_1$
      and $\pi_2$, there is a proof $\Pi'$ for sequent
      $[[G1; D1; G2; A2; G3 |-l C]]$ s.t. $c(\Pi') \leq |B|$. Therefore,
      the proof $\Pi$ can be constructed as follows with
      $c(\Pi) = c(\Pi') \leq |B|$.
      \begin{center}
        \scriptsize
        \begin{math}
          $$\mprset{flushleft}
          \inferrule* [right={\tiny implL}] {
            {
              \begin{array}{cc}
                \pi_1 & \Pi' \\
                {[[D2 |-l A1]]} & {[[G1; D1; G2; A2; G3 |-l C]]}
              \end{array}
            }
          }{[[G1; D1; G2; D2; A2 <- A1; G3 |-l C]]}
        \end{math}
      \end{center}

\item Case 5:
      \begin{center}
        \scriptsize
        \begin{math}
          \begin{array}{c}
            \Pi_1 \\
            {[[I |-c X]]}
          \end{array}
        \end{math}
        \qquad\qquad
        $\Pi_2$:
        \begin{math}
          $$\mprset{flushleft}
          \inferrule* [right={\tiny implL}] {
            {
              \begin{array}{cc}
                \pi_1 & \pi_2 \\
                {[[D |-l A1]]} & {[[G1; A2; G2; X; G3 |-l B]]}
              \end{array}
            }
          }{[[G1; D; A2 <- A1; D; G2; X; G3 |-l B]]}
        \end{math}
      \end{center}
      By assumption, $c(\Pi_1),c(\Pi_2)\leq |X|$. By induction on $\Pi_1$
      and $\pi_2$, there is a proof $\Pi'$ for sequent
      $[[G1; A2; G2; I; G3 |-l B]]$ s.t. $c(\Pi') \leq |X|$. Therefore, the
      proof $\Pi$ can be constructed as follows with
      $c(\Pi) = c(\Pi') \leq |X|$.
      \begin{center}
        \scriptsize
        \begin{math}
          $$\mprset{flushleft}
          \inferrule* [right={\tiny implL}] {
            {
              \begin{array}{cc}
                \pi_1 & \Pi' \\
                {[[D |-l A1]]} & {[[G1; A2; G2; I; G3 |-l B]]}
              \end{array}
            }
          }{[[G1; D; A2 <- A1; G2; I; G3 |-l B]]}
        \end{math}
      \end{center}

\item Case 6:
    \begin{center}
      \scriptsize
      \begin{math}
        \begin{array}{c}
          \Pi_1 \\
          {[[D1 |-l B]]}
        \end{array}
      \end{math}
      \qquad\qquad
      $\Pi_2$:
      \begin{math}
        $$\mprset{flushleft}
        \inferrule* [right={\tiny implL}] {
          {
            \begin{array}{cc}
              \pi_1 & \pi_2 \\
              {[[D2 |-l A1]]} & {[[G1; A2; G2; B; G3 |-l C]]}
            \end{array}
          }
        }{[[G1; D2; A2 <- A1; G2; B; G3 |-l C]]}
      \end{math}
    \end{center}
    By assumption, $c(\Pi_1),c(\Pi_2)\leq |B|$. By induction on $\Pi_1$ and
    $\pi_2$, there is a proof $\Pi'$ for sequent
    $[[G1; A2; G2; D1; G3 |-l C]]$ s.t. $c(\Pi') \leq |B|$. Therefore, the
    proof $\Pi$ can be constructed as follows with
    $c(\Pi) = c(\Pi') \leq |B|$.
    \begin{center}
      \scriptsize
      \begin{math}
        $$\mprset{flushleft}
        \inferrule* [right={\tiny implL}] {
          {
            \begin{array}{cc}
              \pi_1 & \Pi' \\
              {[[D2 |-l A1]]} & {[[G1; A2; G2; D1; G3 |-l C]]}
            \end{array}
          }
        }{[[G1; D2; A2 <- A1; G2; D1; G3 |-l C]]}
      \end{math}
    \end{center}
\end{itemize}




\subsubsection{Left introduction of the commutative tensor $\otimes$ (with low priority)}
\begin{itemize}
\item Case 1:
      \begin{center}
        \scriptsize
        \begin{math}
          \begin{array}{c}
            \Pi_1 \\
            {[[I |-c X]]}
          \end{array}
        \end{math}
        \qquad\qquad
        $\Pi_2$:
        \begin{math}
          $$\mprset{flushleft}
          \inferrule* [right={\tiny tenL}] {
            {
              \begin{array}{c}
                \pi \\
                {[[P1, X, P2, Y1, Y2, P3 |-c Z]]}
              \end{array}
            }
          }{[[P1, X, P2, Y1 (*) Y2, P3 |-c Z]]}
        \end{math}
      \end{center}
      By assumption, $c(\Pi_1),c(\Pi_2)\leq |X|$. By induction on $\Pi_1$
      and $\pi$, there is a proof $\Pi'$ for sequent
      $[[P1, I, P2, Y1, Y2, P3 |-c Z]]$ s.t. $c(\Pi') \leq |X|$. Therefore,
      the proof $\Pi$ can be constructed as follows with
      $c(\Pi) = c(\Pi') \leq |X|$.
      \begin{center}
        \scriptsize
        \begin{math}
          $$\mprset{flushleft}
          \inferrule* [right={\tiny tenL}] {
            {
              \begin{array}{c}
                \Pi' \\
                {[[P1, I, P2, Y1, Y2, P3 |-c Z]]}
              \end{array}
            }
          }{[[P1, I, P2, Y1 (*) Y2, P3 |-c Z]]}
        \end{math}
      \end{center}

\item Case 2:
      \begin{center}
        \scriptsize
        \begin{math}
          \begin{array}{c}
            \Pi_1 \\
            {[[I |-c X]]}
          \end{array}
        \end{math}
        \qquad\qquad
        $\Pi_2$:
        \begin{math}
          $$\mprset{flushleft}
          \inferrule* [right={\tiny tenL}] {
            {
              \begin{array}{c}
                \pi \\
                {[[P1, Y1, Y2, P2, X, P3 |-c Z]]}
              \end{array}
            }
          }{[[P1, Y1 (*) Y2, P2, X, P3 |-c Z]]}
        \end{math}
      \end{center}
      By assumption, $c(\Pi_1),c(\Pi_2)\leq |X|$. By induction on $\Pi_1$
      and $\pi$, there is a proof $\Pi'$ for sequent
      $[[P1, Y1, Y2, P2, I, P3 |-c Z]]$ s.t. $c(\Pi') \leq |X|$. Therefore,
      the proof $\Pi$ can be constructed as follows with
      $c(\Pi) = c(\Pi') \leq |X|$.
      \begin{center}
        \scriptsize
        \begin{math}
          $$\mprset{flushleft}
          \inferrule* [right={\tiny tenL}] {
            {
              \begin{array}{c}
                \Pi' \\
                {[[P1, Y1, Y2, P2, I, P3 |-c Z]]}
              \end{array}
            }
          }{[[P1, Y1 (*) Y2, P2, I, P3 |-c Z]]}
        \end{math}
      \end{center}

\item Case 3:
      \begin{center}
        \scriptsize
        \begin{math}
          \begin{array}{c}
            \Pi_1 \\
            {[[I |-c X]]}
          \end{array}
        \end{math}
        \qquad\qquad
        $\Pi_2$:
        \begin{math}
          $$\mprset{flushleft}
          \inferrule* [right={\tiny tenL}] {
            {
              \begin{array}{c}
                \pi \\
                {[[G1; X; G2; Y1; Y2; G3 |-l A]]}
              \end{array}
            }
          }{[[G1; X; G2; Y1 (*) Y2; G3 |-l A]]}
        \end{math}
      \end{center}
      By assumption, $c(\Pi_1),c(\Pi_2)\leq |X|$. By induction on $\Pi_1$
      and $\pi$, there is a proof $\Pi'$ for sequent
      $[[G1; I; G2; Y1; Y2; G3 |-l A]]$ s.t. $c(\Pi') \leq |X|$. Therefore,
      the proof $\Pi$ can be constructed as follows with
      $c(\Pi) = c(\Pi') \leq |X|$.
      \begin{center}
        \scriptsize
        \begin{math}
          $$\mprset{flushleft}
          \inferrule* [right={\tiny tenL}] {
            {
              \begin{array}{c}
                \Pi' \\
                {[[G1; I; G2; Y1; Y2; G3 |-l A]]}
              \end{array}
            }
          }{[[G1; I; G2; Y1 (*) Y2; G3 |-l A]]}
        \end{math}
      \end{center}

\item Case 4:
      \begin{center}
        \scriptsize
        \begin{math}
          \begin{array}{c}
            \Pi_1 \\
            {[[D |-l B]]}
          \end{array}
        \end{math}
        \qquad\qquad
        $\Pi_2$:
        \begin{math}
          $$\mprset{flushleft}
          \inferrule* [right={\tiny tenL}] {
            {
              \begin{array}{c}
                \pi \\
                {[[G1; B; G2; Y1; Y2; G3 |-l A]]}
              \end{array}
            }
          }{[[G1; B; G2; Y1 (*) Y2; G3 |-l A]]}
        \end{math}
      \end{center}
      By assumption, $c(\Pi_1),c(\Pi_2)\leq |B|$. By induction on $\Pi_1$
      and $\pi$, there is a proof $\Pi'$ for sequent
      $[[G1; B; G2; Y1; Y2; G3 |-l A]]$ s.t. $c(\Pi') \leq |B|$. Therefore,
      the proof $\Pi$ can be constructed as follows with
      $c(\Pi) = c(\Pi') \leq |B|$.
      \begin{center}
        \scriptsize
        \begin{math}
          $$\mprset{flushleft}
          \inferrule* [right={\tiny tenL}] {
            {
              \begin{array}{c}
                \Pi' \\
                {[[G1; D; G2; Y1; Y2; G3 |-l A]]}
              \end{array}
            }
          }{[[G1; D; G2; Y1 (*) Y2; G3 |-l A]]}
        \end{math}
      \end{center}

\item Case 5:
      \begin{center}
        \scriptsize
        \begin{math}
          \begin{array}{c}
            \Pi_1 \\
            {[[I |-c X]]}
          \end{array}
        \end{math}
        \qquad\qquad
        $\Pi_2$:
        \begin{math}
          $$\mprset{flushleft}
          \inferrule* [right={\tiny tenL}] {
            {
              \begin{array}{c}
                \pi \\
                {[[G1; Y1; Y2; G2; X; G3 |-l A]]}
              \end{array}
            }
          }{[[G1; Y1 (*) Y2; G2; X; G3 |-l A]]}
        \end{math}
      \end{center}
      By assumption, $c(\Pi_1),c(\Pi_2)\leq |X|$. By induction on $\Pi_1$
      and $\pi$, there is a proof $\Pi'$ for sequent
      $[[G1; Y1; Y2; G2; I; G3 |-l A]]$ s.t. $c(\Pi') \leq |X|$. Therefore,
      the proof $\Pi$ can be constructed as follows with
      $c(\Pi) = c(\Pi') \leq |X|$.
      \begin{center}
        \scriptsize
        \begin{math}
          $$\mprset{flushleft}
          \inferrule* [right={\tiny tenL}] {
            {
              \begin{array}{c}
                \Pi' \\
                {[[G1; Y1; Y2; G2; I; G3 |-l A]]}
              \end{array}
            }
          }{[[G1; Y1 (*) Y2; G2; I; G3 |-l A]]}
        \end{math}
      \end{center}

\item Case 6:
      \begin{center}
        \scriptsize
        \begin{math}
          \begin{array}{c}
            \Pi_1 \\
            {[[D |-l B]]}
          \end{array}
        \end{math}
        \qquad\qquad
        $\Pi_2$:
        \begin{math}
          $$\mprset{flushleft}
          \inferrule* [right={\tiny tenL}] {
            {
              \begin{array}{c}
                \pi \\
                {[[G1; Y1; Y2; G2; B; G3 |-l A]]}
              \end{array}
            }
          }{[[G1; Y1 (*) Y2; G2; B; G3 |-l A]]}
        \end{math}
      \end{center}
      By assumption, $c(\Pi_1),c(\Pi_2)\leq |B|$. By induction on $\Pi_1$
      and $\pi$, there is a proof $\Pi'$ for sequent
      $[[G1; Y1; Y2; G2; D; G3 |-l A]]$ s.t. $c(\Pi') \leq |B|$. Therefore,
      the proof $\Pi$ can be constructed as follows with
      $c(\Pi) = c(\Pi') \leq |B|$.
      \begin{center}
        \scriptsize
        \begin{math}
          $$\mprset{flushleft}
          \inferrule* [right={\tiny tenL}] {
            {
              \begin{array}{c}
                \Pi' \\
                {[[G1; Y1; Y2; G2; D; G3 |-l A]]}
              \end{array}
            }
          }{[[G1; Y1 (*) Y2; G2; D; G3 |-l A]]}
        \end{math}
      \end{center}
\end{itemize}


\subsubsection{Left introduction of the non-commutative tensor $\tri$ (with low priority)}:
\begin{itemize}
\item Case 1:
      \begin{center}
        \scriptsize
        \begin{math}
          \begin{array}{c}
            \Pi_1 \\
            {[[I |-c X]]}
          \end{array}
        \end{math}
        \qquad\qquad
        $\Pi_2$:
        \begin{math}
          $$\mprset{flushleft}
          \inferrule* [right={\tiny tenL}] {
            {
              \begin{array}{c}
                \pi \\
                {[[G1; X; G2; A1; A2; G3 |-l B]]}
              \end{array}
            }
          }{[[G1; X; G2; A1 (>) A2; G3 |-l B]]}
        \end{math}
      \end{center}
      By assumption, $c(\Pi_1),c(\Pi_2)\leq |X|$. By induction on $\Pi_1$
      and $\pi$, there is a proof $\Pi'$ for sequent
      $[[G1; I; G2; A1; A2; G3 |-l B]]$ s.t. $c(\Pi') \leq |X|$. Therefore,
      the proof $\Pi$ can be constructed as follows with
      $c(\Pi) = c(\Pi') \leq |X|$.
      \begin{center}
        \scriptsize
        \begin{math}
          $$\mprset{flushleft}
          \inferrule* [right={\tiny tenL}] {
            {
              \begin{array}{c}
                \Pi' \\
                {[[G1; I; G2; A1; A2; G3 |-l B]]}
              \end{array}
            }
          }{[[G1; I; G2; A1 (>) A2; G3 |-l B]]}
        \end{math}
      \end{center}

\item Case 2:
      \begin{center}
        \scriptsize
        \begin{math}
          \begin{array}{c}
            \Pi_1 \\
            {[[D |-l B]]}
          \end{array}
        \end{math}
        \qquad\qquad
        $\Pi_2$:
        \begin{math}
          $$\mprset{flushleft}
          \inferrule* [right={\tiny tenL}] {
            {
              \begin{array}{c}
                \pi \\
                {[[G1; B; G2; A1; A2; G3 |-l C]]}
              \end{array}
            }
          }{[[G1; B; G2; A1 (>) A2; G3 |-l C]]}
        \end{math}
      \end{center}
      By assumption, $c(\Pi_1),c(\Pi_2)\leq |B|$. By induction on $\Pi_1$
      and $\pi$, there is a proof $\Pi'$ for sequent
      $[[G1; D; G2; A1; A2; G3 |-l C]]$ s.t. $c(\Pi') \leq |B|$. Therefore,
      the proof $\Pi$ can be constructed as follows with
      $c(\Pi) = c(\Pi') \leq |B|$.
      \begin{center}
        \scriptsize
        \begin{math}
          $$\mprset{flushleft}
          \inferrule* [right={\tiny tenL}] {
            {
              \begin{array}{c}
                \Pi' \\
                {[[G1; D; G2; A1; A2; G3 |-l C]]}
              \end{array}
            }
          }{[[G1; D; G2; A1 (>) A2; G3 |-l C]]}
        \end{math}
      \end{center}

\item Case 3:
      \begin{center}
        \scriptsize
        \begin{math}
          \begin{array}{c}
            \Pi_1 \\
            {[[I |-c X]]}
          \end{array}
        \end{math}
        \qquad\qquad
        $\Pi_2$:
        \begin{math}
          $$\mprset{flushleft}
          \inferrule* [right={\tiny tenL}] {
            {
              \begin{array}{c}
                \pi \\
                {[[G1; A1; A2; G2; X; G3 |-l B]]}
              \end{array}
            }
          }{[[G1; A1 (>) A2; G2; X; G3 |-l B]]}
        \end{math}
      \end{center}
      By assumption, $c(\Pi_1),c(\Pi_2)\leq |X|$. By induction on $\Pi_1$
      and $\pi$, there is a proof $\Pi'$ for sequent
      $[[G1; A1; A2; G2; I; G3 |-l A]]$ s.t. $c(\Pi') \leq |X|$. Therefore,
      the proof $\Pi$ can be constructed as follows with
      $c(\Pi) = c(\Pi') \leq |X|$.
      \begin{center}
        \scriptsize
        \begin{math}
          $$\mprset{flushleft}
          \inferrule* [right={\tiny tenL}] {
            {
              \begin{array}{c}
                \Pi' \\
                {[[G1; A1; A2; G2; I; G3 |-l B]]}
              \end{array}
            }
          }{[[G1; A1 (>) A2; G2; I; G3 |-l B]]}
        \end{math}
      \end{center}

\item Case 4:
      \begin{center}
        \scriptsize
        \begin{math}
          \begin{array}{c}
            \Pi_1 \\
            {[[D |-l B]]}
          \end{array}
        \end{math}
        \qquad\qquad
        $\Pi_2$:
        \begin{math}
          $$\mprset{flushleft}
          \inferrule* [right={\tiny tenL}] {
            {
              \begin{array}{c}
                \pi \\
                {[[G1; A1; A2; G2; B; G3 |-l C]]}
              \end{array}
            }
          }{[[G1; A1 (>) A2; G2; B; G3 |-l C]]}
        \end{math}
      \end{center}
      By assumption, $c(\Pi_1),c(\Pi_2)\leq |B|$. By induction on $\Pi_1$
      and $\pi$, there is a proof $\Pi'$ for sequent
      $[[G1; A1; A2; G2; D; G3 |-l C]]$ s.t. $c(\Pi') \leq |B|$. Therefore,
      the proof $\Pi$ can be constructed as follows with
      $c(\Pi) = c(\Pi') \leq |B|$.
      \begin{center}
        \scriptsize
        \begin{math}
          $$\mprset{flushleft}
          \inferrule* [right={\tiny tenL}] {
            {
              \begin{array}{c}
                \Pi' \\
                {[[G1; A1; A2; G2; D; G3 |-l C]]}
              \end{array}
            }
          }{[[G1; A1 (>) A2; G2; D; G3 |-l C]]}
        \end{math}
      \end{center}
\end{itemize}



\subsubsection{$\SCdruleTXXexName$}
\begin{itemize}
\item Case 1:
      \begin{center}
        \scriptsize
        \begin{math}
          \begin{array}{c}
            \Pi_1 \\
            {[[I |-c X]]}
          \end{array}
        \end{math}
        \qquad\qquad
        $\Pi_2$:
        \begin{math}
          $$\mprset{flushleft}
          \inferrule* [right={\tiny beta}] {
            {
              \begin{array}{c}
                \pi \\
                {[[P1, X, P2, Y1, Y2, P3 |-c Z]]}
              \end{array}
            }
          }{[[P1, X, P2, Y2, Y1, P3 |-c Z]]}
        \end{math}
      \end{center}
      By assumption, $c(\Pi_1),c(\Pi_2)\leq |X|$. By induction on $\Pi_1$
      and $\pi$, there is a proof $\Pi'$ for sequent
      $[[P1, I, P2, Y1, Y2, P3 |-c Z]]$ s.t. $c(\Pi') \leq |X|$. Therefore,
      the proof $\Pi$ can be constructed as follows with
      $c(\Pi) = c(\Pi') \leq |X|$.
      \begin{center}
        \scriptsize
        \begin{math}
          $$\mprset{flushleft}
          \inferrule* [right={\tiny cut}] {
            {
              \begin{array}{cc}
                \Pi' \\
                {[[P1, I, P2, Y1, Y2, P3 |-c Z]]}
              \end{array}
            }
          }{[[P1, I, P2, Y2, Y1, P3 |-c Z]]}
        \end{math}
      \end{center}

\item Case 2:
      \begin{center}
        \scriptsize
        \begin{math}
          \begin{array}{c}
            \Pi_1 \\
            {[[I |-c X]]}
          \end{array}
        \end{math}
        \qquad\qquad
        $\Pi_2$:
        \begin{math}
          $$\mprset{flushleft}
          \inferrule* [right={\tiny beta}] {
            {
              \begin{array}{c}
                \pi \\
                {[[P1, Y1, Y2, P2, X, P3 |-c Z]]}
              \end{array}
            }
          }{[[P1, X, P2, Y2, Y1, P3 |-c Z]]}
        \end{math}
      \end{center}
      By assumption, $c(\Pi_1),c(\Pi_2)\leq |X|$. By induction on $\Pi_1$
      and $\pi$, there is a proof $\Pi'$ for sequent
      $[[P1, Y1, Y2, P2, I, P3 |-c Z]]$ s.t. $c(\Pi') \leq |X|$. Therefore,
      the proof $\Pi$ can be constructed as follows with
      $c(\Pi) = c(\Pi') \leq |X|$.
      \begin{center}
        \scriptsize
        \begin{math}
          $$\mprset{flushleft}
          \inferrule* [right={\tiny cut}] {
            {
              \begin{array}{cc}
                \Pi' \\
                {[[P1, Y1, Y2, P2, I, P3 |-c Z]]}
              \end{array}
            }
          }{[[P1, Y2, Y1, P2, I, P3 |-c Z]]}
        \end{math}
      \end{center}
\end{itemize}


\subsubsection{$\SCdruleSXXexName$}
\begin{itemize}
\item Case 1:
      \begin{center}
        \scriptsize
        \begin{math}
          \begin{array}{c}
            \Pi_1 \\
            {[[I |-c X]]}
          \end{array}
        \end{math}
        \qquad\qquad
        $\Pi_2$:
        \begin{math}
          $$\mprset{flushleft}
          \inferrule* [right={\tiny beta}] {
            {
              \begin{array}{c}
                \pi \\
                {[[G1; X; G2; Y1; Y2; G3 |-l A]]}
              \end{array}
            }
          }{[[G1; X; G2; Y2; Y1; G3 |-l A]]}
        \end{math}
      \end{center}
      By assumption, $c(\Pi_1),c(\Pi_2)\leq |X|$. By induction on $\Pi_1$
      and $\pi$, there is a proof $\Pi'$ for sequent
      $[[G1; I; G2; Y1; Y2; G3 |-l A]]$ s.t. $c(\Pi') \leq |X|$. Therefore,
      the proof $\Pi$ can be constructed as follows with
      $c(\Pi) = c(\Pi') \leq |X|$.
      \begin{center}
        \scriptsize
        \begin{math}
          $$\mprset{flushleft}
          \inferrule* [right={\tiny cut}] {
            {
              \begin{array}{cc}
                \Pi' \\
                {[[G1; I; G2; Y1; Y2; G3 |-l A]]}
              \end{array}
            }
          }{[[G1; I; G2; Y2; Y1; G3 |-l A]]}
        \end{math}
      \end{center}

\item Case 2:
      \begin{center}
        \scriptsize
        \begin{math}
          \begin{array}{c}
            \Pi_1 \\
            {[[D |-l B]]}
          \end{array}
        \end{math}
        \qquad\qquad
        $\Pi_2$:
        \begin{math}
          $$\mprset{flushleft}
          \inferrule* [right={\tiny beta}] {
            {
              \begin{array}{c}
                \pi \\
                {[[G1; B; G2; Y1; Y2; G3 |-l A]]}
              \end{array}
            }
          }{[[G1; B; G2; Y2; Y1; G3 |-l A]]}
        \end{math}
      \end{center}
      By assumption, $c(\Pi_1),c(\Pi_2)\leq |X|$. By induction on $\Pi_1$
      and $\pi$, there is a proof $\Pi'$ for sequent
      $[[G1; D; G2; Y1; Y2; G3 |-l A]]$ s.t. $c(\Pi') \leq |X|$. Therefore,
      the proof $\Pi$ can be constructed as follows with
      $c(\Pi) = c(\Pi') \leq |X|$.
      \begin{center}
        \scriptsize
        \begin{math}
          $$\mprset{flushleft}
          \inferrule* [right={\tiny cut}] {
            {
              \begin{array}{cc}
                \Pi' \\
                {[[G1; D; G2; Y1; Y2; G3 |-l A]]}
              \end{array}
            }
          }{[[G1; D; G2; Y2; Y1; G3 |-l A]]}
        \end{math}
      \end{center}

\item Case 3:
      \begin{center}
        \scriptsize
        \begin{math}
          \begin{array}{c}
            \Pi_1 \\
            {[[I |-c X]]}
          \end{array}
        \end{math}
        \qquad\qquad
        $\Pi_2$:
        \begin{math}
          $$\mprset{flushleft}
          \inferrule* [right={\tiny beta}] {
            {
              \begin{array}{c}
                \pi \\
                {[[G1; Y1; Y2; G2; X; G3 |-l A]]}
              \end{array}
            }
          }{[[G1; X; G2; Y2; Y1; G3 |-l A]]}
        \end{math}
      \end{center}
      By assumption, $c(\Pi_1),c(\Pi_2)\leq |X|$. By induction on $\Pi_1$
      and $\pi$, there is a proof $\Pi'$ for sequent
      $[[G1; Y1; Y2; G2; I; G3 |-l A]]$ s.t. $c(\Pi') \leq |X|$. Therefore,
      the proof $\Pi$ can be constructed as follows with
      $c(\Pi) = c(\Pi') \leq |X|$.
      \begin{center}
        \scriptsize
        \begin{math}
          $$\mprset{flushleft}
          \inferrule* [right={\tiny cut}] {
            {
              \begin{array}{cc}
                \Pi' \\
                {[[G1; Y1; Y2; G2; I; G3 |-l A]]}
              \end{array}
            }
          }{[[G1; Y2; Y1; G2; I; G3 |-l A]]}
        \end{math}
      \end{center}

\item Case 4:
      \begin{center}
        \scriptsize
        \begin{math}
          \begin{array}{c}
            \Pi_1 \\
            {[[D |-l B]]}
          \end{array}
        \end{math}
        \qquad\qquad
        $\Pi_2$:
        \begin{math}
          $$\mprset{flushleft}
          \inferrule* [right={\tiny beta}] {
            {
              \begin{array}{c}
                \pi \\
                {[[G1; Y1; Y2; G2; B; G3 |-l A]]}
              \end{array}
            }
          }{[[G1; Y2; Y1; G2; B; G3 |-l A]]}
        \end{math}
      \end{center}
      By assumption, $c(\Pi_1),c(\Pi_2)\leq |X|$. By induction on $\Pi_1$
      and $\pi$, there is a proof $\Pi'$ for sequent
      $[[G1; Y1; Y2; G2; D; G3 |-l A]]$ s.t. $c(\Pi') \leq |X|$. Therefore,
      the proof $\Pi$ can be constructed as follows with
      $c(\Pi) = c(\Pi') \leq |X|$.
      \begin{center}
        \scriptsize
        \begin{math}
          $$\mprset{flushleft}
          \inferrule* [right={\tiny cut}] {
            {
              \begin{array}{cc}
                \Pi' \\
                {[[G1; Y1; Y2; G2; D; G3 |-l A]]}
              \end{array}
            }
          }{[[G1; Y2; Y1; G2; D; G3 |-l A]]}
        \end{math}
      \end{center}
\end{itemize}



\subsubsection{Left introduction of the commutative unit $[[UnitT]]$ (with low priority)}
\begin{itemize}
\item Case 1:
      \begin{center}
        \scriptsize
        \begin{math}
          \begin{array}{c}
            \Pi_1 \\
            {[[P |-c X]]}
          \end{array}
        \end{math}
        \qquad\qquad
        $\Pi_2$:
        \begin{math}
          $$\mprset{flushleft}
          \inferrule* [right={\tiny unitL}] {
            {
              \begin{array}{c}
                \pi \\
                {[[I1, I2, X, I3 |-c Y]]}
              \end{array}
            }
          }{[[I1, UnitT, I2, X, I3 |-c Y]]}
        \end{math}
      \end{center}
      By assumption, $c(\Pi_1),c(\Pi_2)\leq |X|$. By induction on $\Pi_1$
      and $\pi$, there is a proof $\Pi'$ for sequent
      $[[I1, I2, P, I3 |-c Y]]$
      s.t. $c(\Pi') \leq |X|$. Therefore, the proof $\Pi$ can be
      constructed as follows with $c(\Pi) = c(\Pi') \leq |X|$.
      \begin{center}
        \scriptsize
        \begin{math}
          $$\mprset{flushleft}
          \inferrule* [right={\tiny unitL}] {
            {
              \begin{array}{c}
                \Pi' \\
                {[[I1, I2, P, I3 |-c Y]]}
              \end{array}
            }
          }{[[I1, UnitT, I2, P, I3 |-c Y]]}
        \end{math}
      \end{center}

\item Case 2:
      \begin{center}
        \scriptsize
        \begin{math}
          \begin{array}{c}
            \Pi_1 \\
            {[[I |-c X]]}
          \end{array}
        \end{math}
        \qquad\qquad
        $\Pi_2$:
        \begin{math}
          $$\mprset{flushleft}
          \inferrule* [right={\tiny unitL1}] {
            {
              \begin{array}{c}
                \pi \\
                {[[G1; G2; X; G3 |-l A]]}
              \end{array}
            }
          }{[[G1; UnitT; G2; X; G3 |-l A]]}
        \end{math}
      \end{center}
      By assumption, $c(\Pi_1),c(\Pi_2)\leq |X|$. By induction on $\Pi_1$
      and $\pi$, there is a proof $\Pi'$ for sequent
      $[[G1; G2; I; G2 |-l A]]$
      s.t. $c(\Pi') \leq |X|$. Therefore, the proof $\Pi$ can be
      constructed as follows with $c(\Pi) = c(\Pi') \leq |X|$.
      \begin{center}
        \scriptsize
        \begin{math}
          $$\mprset{flushleft}
          \inferrule* [right={\tiny unitL1}] {
            {
              \begin{array}{c}
                \Pi' \\
                {[[G1; G2; I; G3 |-l A]]}
              \end{array}
            }
          }{[[G1; UnitT; G2; I; G3 |-l A]]}
        \end{math}
      \end{center}

\item Case 3:
      \begin{center}
        \scriptsize
        \begin{math}
          \begin{array}{c}
            \Pi_1 \\
            {[[D |-l B]]}
          \end{array}
        \end{math}
        \qquad\qquad
        $\Pi_2$:
        \begin{math}
          $$\mprset{flushleft}
          \inferrule* [right={\tiny unitL1}] {
            {
              \begin{array}{c}
                \pi \\
                {[[G1; G2; B; G3 |-l A]]}
              \end{array}
            }
          }{[[G1; UnitT; G2; B; G3 |-l A]]}
        \end{math}
      \end{center}
      By assumption, $c(\Pi_1),c(\Pi_2)\leq |B|$. By induction on $\Pi_1$
      and $\pi$, there is a proof $\Pi'$ for sequent
      $[[G1; G2; D; G3 |-l A]]$
      s.t. $c(\Pi') \leq |B|$. Therefore, the proof $\Pi$ can be
      constructed as follows with $c(\Pi) = c(\Pi') \leq |B|$.
      \begin{center}
        \scriptsize
        \begin{math}
          $$\mprset{flushleft}
          \inferrule* [right={\tiny unitL1}] {
            {
              \begin{array}{c}
                \Pi' \\
                {[[G1; G2; D; G3 |-l A]]}
              \end{array}
            }
          }{[[G1; UnitT; G2; D; G3 |-l A]]}
        \end{math}
      \end{center}
\end{itemize}

\subsubsection{Left introduction of the non-commutative unit $[[UnitS]]$ (with low priority)}
\begin{itemize}
\item Case 1:
      \begin{center}
        \scriptsize
        \begin{math}
          \begin{array}{c}
            \Pi_1 \\
            {[[I |-c X]]}
          \end{array}
        \end{math}
        \qquad\qquad
        $\Pi_2$:
        \begin{math}
          $$\mprset{flushleft}
          \inferrule* [right={\tiny unitL2}] {
            {
              \begin{array}{c}
                \pi \\
                {[[G1; G2; X; G3 |-l A]]}
              \end{array}
            }
          }{[[G1; UnitS; G2; X; G3 |-l A]]}
        \end{math}
      \end{center}
      By assumption, $c(\Pi_1),c(\Pi_2)\leq |X|$. By induction on $\Pi_1$
      and $\pi$, there is a proof $\Pi'$ for sequent
      $[[G1; G2; I; G3 |-l A]]$
      s.t. $c(\Pi') \leq |X|$. Therefore, the proof $\Pi$ can be
      constructed as follows with $c(\Pi) = c(\Pi') \leq |X|$.
      \begin{center}
        \scriptsize
        \begin{math}
          $$\mprset{flushleft}
          \inferrule* [right={\tiny unitL2}] {
            {
              \begin{array}{c}
                \Pi' \\
                {[[G1; G2; I; G3 |-l A]]}
              \end{array}
            }
          }{[[G1; UnitS; G2; I; G3 |-l A]]}
        \end{math}
      \end{center}

\item Case 2:
      \begin{center}
        \scriptsize
        \begin{math}
          \begin{array}{c}
            \Pi_1 \\
            {[[D |-l B]]}
          \end{array}
        \end{math}
        \qquad\qquad
        $\Pi_2$:
        \begin{math}
          $$\mprset{flushleft}
          \inferrule* [right={\tiny unitL2}] {
            {
              \begin{array}{c}
                \pi \\
                {[[G1; G2; B; G3 |-l A]]}
              \end{array}
            }
          }{[[G1; UnitS; G2; B; G3 |-l A]]}
        \end{math}
      \end{center}
      By assumption, $c(\Pi_1),c(\Pi_2)\leq |B|$. By induction on $\Pi_1$
      and $\pi$, there is a proof $\Pi'$ for sequent
      $[[G1; G2; D; G3 |-l A]]$
      s.t. $c(\Pi') \leq |B|$. Therefore, the proof $\Pi$ can be
      constructed as follows with $c(\Pi) = c(\Pi') \leq |B|$.
      \begin{center}
        \scriptsize
        \begin{math}
          $$\mprset{flushleft}
          \inferrule* [right={\tiny unitL2}] {
            {
              \begin{array}{c}
                \Pi' \\
                {[[G1; G2; D; G3 |-l A]]}
              \end{array}
            }
          }{[[G1; UnitS; G2; D; G3 |-l A]]}
        \end{math}
      \end{center}
\end{itemize}



\subsubsection{Right introduction of the commutative implication $\multimap$ (with low priority)}
\begin{center}
  \scriptsize
  \begin{math}
    \begin{array}{c}
      \Pi_1 \\
      {[[I |-c X]]}
    \end{array}
  \end{math}
  \qquad\qquad
  $\Pi_2$:
  \begin{math}
    $$\mprset{flushleft}
    \inferrule* [right={\tiny impR}] {
      {
        \begin{array}{c}
          \pi \\
          {[[P1, X, P2, Y1 |-c Y2]]}
        \end{array}
      }
    }{[[P1, X, P2 |-c Y1 -o Y2]]}
  \end{math}
\end{center}
By assumption, $c(\Pi_1),c(\Pi_2)\leq |X|$. By induction on $\Pi_1$
and $\pi$, there is a proof $\Pi'$ for sequent \\
$[[P1, I, P2, Y1 |-c Y2]]$ s.t. $c(\Pi') \leq |X|$. Therefore, the
proof $\Pi$ can be constructed as follows with \\
$c(\Pi) = c(\Pi') \leq |X|$.
\begin{center}
  \scriptsize
  \begin{math}
    $$\mprset{flushleft}
    \inferrule* [right={\tiny impR}] {
      {
        \begin{array}{c}
          \Pi' \\
          {[[P1, I, P2, Y1 |-c Y2]]}
        \end{array}
      }
    }{[[P1, I, P2 |-c Y1 -o Y2]]}
  \end{math}
\end{center}



\subsubsection{Right introduction of the non-commutative left implication $\lto$ (with low priority)}
\begin{itemize}
\item Case 1:
      \begin{center}
        \scriptsize
        \begin{math}
          \begin{array}{c}
            \Pi_1 \\
            {[[I |-c X]]}
          \end{array}
        \end{math}
        \qquad\qquad
        $\Pi_2$:
        \begin{math}
          $$\mprset{flushleft}
          \inferrule* [right={\tiny impR}] {
            {
              \begin{array}{c}
                \pi \\
                {[[G1; X; G2; A |-l B]]}
              \end{array}
            }
          }{[[G1; X; G2 |-l A -> B]]}
        \end{math}
      \end{center}
      By assumption, $c(\Pi_1),c(\Pi_2)\leq |X|$. By induction on $\Pi_1$
      and $\pi$, there is a proof $\Pi'$ for sequent \\
      $[[G1; I; G2; A |-l B]]$ s.t. $c(\Pi') \leq |X|$. Therefore, the
      proof $\Pi$ can be constructed as follows with \\
      $c(\Pi) = c(\Pi') \leq |X|$.
      \begin{center}
        \scriptsize
        \begin{math}
          $$\mprset{flushleft}
          \inferrule* [right={\tiny implR}] {
            {
              \begin{array}{c}
                \Pi' \\
                {[[G1; I; G2; A |-l B]]}
              \end{array}
            }
          }{[[G1; I; G2 |-l A -> B]]}
        \end{math}
      \end{center}

\item Case 2:
      \begin{center}
        \scriptsize
        \begin{math}
          \begin{array}{c}
            \Pi_1 \\
            {[[D |-l C]]}
          \end{array}
        \end{math}
        \qquad\qquad
        $\Pi_2$:
        \begin{math}
          $$\mprset{flushleft}
          \inferrule* [right={\tiny impR}] {
            {
              \begin{array}{c}
                \pi \\
                {[[G1; C; G2; A |-l B]]}
              \end{array}
            }
          }{[[G1; C; G2 |-l A -> B]]}
        \end{math}
      \end{center}
      By assumption, $c(\Pi_1),c(\Pi_2)\leq |C|$. By induction on $\Pi_1$
      and $\pi$, there is a proof $\Pi'$ for sequent
      $[[G1; D; G2; A |-l B]]$ s.t. $c(\Pi') \leq |C|$. Therefore, the
      proof $\Pi$ can be constructed as follows with \\
      $c(\Pi) = c(\Pi') \leq |C|$.
      \begin{center}
        \scriptsize
        \begin{math}
          $$\mprset{flushleft}
          \inferrule* [right={\tiny implR}] {
            {
              \begin{array}{c}
                \Pi' \\
                {[[G1; D; G2; A |-l B]]}
              \end{array}
            }
          }{[[G1; D; G2 |-l A -> B]]}
        \end{math}
      \end{center}
\end{itemize}




\subsubsection{Right introduction of the non-commutative right implication $\rto$ (with low priority)}
\begin{itemize}
\item Case 1:
      \begin{center}
        \scriptsize
        \begin{math}
          \begin{array}{c}
            \Pi_1 \\
            {[[I |-c X]]}
          \end{array}
        \end{math}
        \qquad\qquad
        $\Pi_2$:
        \begin{math}
          $$\mprset{flushleft}
          \inferrule* [right={\tiny impL}] {
            {
              \begin{array}{c}
                \pi \\
                {[[A; G1; X; G2|-l B]]}
              \end{array}
            }
          }{[[G1; X; G2 |-l B <- A]]}
        \end{math}
      \end{center}
      By assumption, $c(\Pi_1),c(\Pi_2)\leq |X|$. By induction on $\Pi_1$
      and $\pi$, there is a proof $\Pi'$ for sequent
      $[[A; G1; I; G2 |-l B]]$ s.t. $c(\Pi') \leq |X|$. Therefore, the
      proof $\Pi$ can be constructed as follows with \\
      $c(\Pi) = c(\Pi') \leq |X|$.
      \begin{center}
        \scriptsize
        \begin{math}
          $$\mprset{flushleft}
          \inferrule* [right={\tiny impR}] {
            {
              \begin{array}{c}
                \Pi' \\
                {[[A; G1; I; G2 |-l B]]}
              \end{array}
            }
          }{[[G1; I; G2 |-l B <- A]]}
        \end{math}
      \end{center}

\item Case 2:
      \begin{center}
        \scriptsize
        \begin{math}
          \begin{array}{c}
            \Pi_1 \\
            {[[D |-l C]]}
          \end{array}
        \end{math}
        \qquad\qquad
        $\Pi_2$:
        \begin{math}
          $$\mprset{flushleft}
          \inferrule* [right={\tiny impR}] {
            {
              \begin{array}{c}
                \pi \\
                {[[A; G1; C; G2 |-l B]]}
              \end{array}
            }
          }{[[G1; C; G2 |-l B <- A]]}
        \end{math}
      \end{center}
      By assumption, $c(\Pi_1),c(\Pi_2)\leq |C|$. By induction on $\Pi_1$
      and $\pi$, there is a proof $\Pi'$ for sequent
      $[[G1; D; G2; A |-l B]]$ s.t. $c(\Pi') \leq |C|$. Therefore, the
      proof $\Pi$ can be constructed as follows with \\
      $c(\Pi) = c(\Pi') \leq |C|$.
      \begin{center}
        \scriptsize
        \begin{math}
          $$\mprset{flushleft}
          \inferrule* [right={\tiny impR}] {
            {
              \begin{array}{c}
                \Pi' \\
                {[[A; G1; D; G2 |-l B]]}
              \end{array}
            }
          }{[[G1; D; G2 |-l B <- A]]}
        \end{math}
      \end{center}
\end{itemize}



\subsubsection{Right introduction of the functor $F$}
\begin{center}
  \scriptsize
  \begin{math}
    \begin{array}{c}
      \Pi_1 \\
      {[[I |-c X]]}
    \end{array}
  \end{math}
  \qquad\qquad
  $\Pi_2$:
  \begin{math}
    $$\mprset{flushleft}
    \inferrule* [right={\tiny Fr}] {
      {
        \begin{array}{c}
          \pi \\
          {[[P1, X, P2 |-c Y]]}
        \end{array}
      }
    }{[[P1, X, P2 |-l F Y]]}
  \end{math}
\end{center}
By assumption, $c(\Pi_1),c(\Pi_2)\leq |X|$. By induction on $\Pi_1$
and $\pi$, there is a proof $\Pi'$ for sequent \\
$[[P1, I, P2 |-c Y]]$ s.t. $c(\Pi') \leq |X|$. Therefore, the proof $\Pi$
can be constructed as follows with \\
$c(\Pi) = c(\Pi') \leq |X|$.
\begin{center}
  \scriptsize
  \begin{math}
    $$\mprset{flushleft}
    \inferrule* [right={\tiny Fr}] {
      {
        \begin{array}{c}
          \Pi' \\
          {[[P1, I, P2 |-c Y]]}
        \end{array}
      }
    }{[[P1, I, P2 |-l F Y]]}
  \end{math}
\end{center}



\subsubsection{Left introduction of the functor $F$ (with low priority)}
\begin{itemize}
\item Case 1:
      \begin{center}
        \scriptsize
        \begin{math}
          \begin{array}{c}
            \Pi_1 \\
            {[[I |-c X]]}
          \end{array}
        \end{math}
        \qquad\qquad
        $\Pi_2$:
        \begin{math}
          $$\mprset{flushleft}
          \inferrule* [right={\tiny Fl}] {
            {
              \begin{array}{c}
                \pi \\
                {[[G1; X; G2; Y; G3 |-l A]]}
              \end{array}
            }
          }{[[G1; X; G2; F Y; G3 |-l A]]}
        \end{math}
      \end{center}
      By assumption, $c(\Pi_1),c(\Pi_2)\leq |X|$. By induction on $\Pi_1$
      and $\pi$, there is a proof $\Pi'$ for sequent
      $[[G1; I; G2; Y; G3 |-l A]]$ s.t. $c(\Pi') \leq |X|$. Therefore, the
      proof $\Pi$ can be constructed as follows with
      $c(\Pi) = c(\Pi') \leq |X|$.
      \begin{center}
        \scriptsize
        \begin{math}
          $$\mprset{flushleft}
          \inferrule* [right={\tiny Fl}] {
            {
              \begin{array}{c}
                \Pi' \\
                {[[G1; I; G2; Y; G3 |-l A]]}
              \end{array}
            }
          }{[[G1; I; G2; F Y; G3 |-l A]]}
        \end{math}
      \end{center}

\item Case 2:
      \begin{center}
        \scriptsize
        \begin{math}
          \begin{array}{c}
            \Pi_1 \\
            {[[D |-l B]]}
          \end{array}
        \end{math}
        \qquad\qquad
        $\Pi_2$:
        \begin{math}
          $$\mprset{flushleft}
          \inferrule* [right={\tiny Fl}] {
            {
              \begin{array}{c}
                \pi \\
                {[[G1; B; G2; Y; G3 |-l A]]}
              \end{array}
            }
          }{[[G1; B; G2; F Y; G3 |-l A]]}
        \end{math}
      \end{center}
      By assumption, $c(\Pi_1),c(\Pi_2)\leq |B|$. By induction on $\Pi_1$
      and $\pi$, there is a proof $\Pi'$ for sequent
      $[[G1; D; G2; Y; G3 |-l A]]$ s.t. $c(\Pi') \leq |B|$. Therefore, the
      proof $\Pi$ can be constructed as follows with
      $c(\Pi) = c(\Pi') \leq |B|$.
      \begin{center}
        \scriptsize
        \begin{math}
          $$\mprset{flushleft}
          \inferrule* [right={\tiny Fl}] {
            {
              \begin{array}{c}
                \Pi' \\
                {[[G1; D; G2; Y; G3 |-l A]]}
              \end{array}
            }
          }{[[G1; D; G2; F Y; G3 |-l A]]}
        \end{math}
      \end{center}

\item Case 3:
      \begin{center}
        \scriptsize
        \begin{math}
          \begin{array}{c}
            \Pi_1 \\
            {[[I |-c X]]}
          \end{array}
        \end{math}
        \qquad\qquad
        $\Pi_2$:
        \begin{math}
          $$\mprset{flushleft}
          \inferrule* [right={\tiny Fl}] {
            {
              \begin{array}{c}
                \pi \\
                {[[G1; Y; G2; X; G3 |-l A]]}
              \end{array}
            }
          }{[[G1; F Y; G2; X; G3 |-l A]]}
        \end{math}
      \end{center}
      By assumption, $c(\Pi_1),c(\Pi_2)\leq |X|$. By induction on $\Pi_1$
      and $\pi$, there is a proof $\Pi'$ for sequent
      $[[G1; Y; G2; I; G3 |-l A]]$ s.t. $c(\Pi') \leq |X|$. Therefore, the
      proof $\Pi$ can be constructed as follows with
      $c(\Pi) = c(\Pi') \leq |X|$.
      \begin{center}
        \scriptsize
        \begin{math}
          $$\mprset{flushleft}
          \inferrule* [right={\tiny Fl}] {
            {
              \begin{array}{c}
                \Pi' \\
                {[[G1; Y; G2; I; G3 |-l A]]}
              \end{array}
            }
          }{[[G1; F Y; G2; I; G3 |-l A]]}
        \end{math}
      \end{center}

\item Case 4:
      \begin{center}
        \scriptsize
        \begin{math}
          \begin{array}{c}
            \Pi_1 \\
            {[[D |-l B]]}
          \end{array}
        \end{math}
        \qquad\qquad
        $\Pi_2$:
        \begin{math}
          $$\mprset{flushleft}
          \inferrule* [right={\tiny Fl}] {
            {
              \begin{array}{c}
                \pi \\
                {[[G1; Y; G2; B; G3 |-l A]]}
              \end{array}
            }
          }{[[G1; F Y; G2; D; G3 |-l A]]}
        \end{math}
      \end{center}
      By assumption, $c(\Pi_1),c(\Pi_2)\leq |B|$. By induction on $\Pi_1$
      and $\pi$, there is a proof $\Pi'$ for sequent
      $[[G1; Y; G2; D; G3 |-l A]]$ s.t. $c(\Pi') \leq |B|$. Therefore, the
      proof $\Pi$ can be constructed as follows with
      $c(\Pi) = c(\Pi') \leq |B|$.
      \begin{center}
        \scriptsize
        \begin{math}
          $$\mprset{flushleft}
          \inferrule* [right={\tiny Fl}] {
            {
              \begin{array}{c}
                \Pi' \\
                {[[G1; Y; G2; D; G3 |-l A]]}
              \end{array}
            }
          }{[[G1; F Y; G2; D; G3 |-l A]]}
        \end{math}
      \end{center}
\end{itemize}




\subsubsection{Right introduction of the functor $G$ (with low priority)}
\begin{center}
  \scriptsize
  \begin{math}
    \begin{array}{c}
      \Pi_1 \\
      {[[I |-c X]]}
    \end{array}
  \end{math}
  \qquad\qquad
  $\Pi_2$:
  \begin{math}
    $$\mprset{flushleft}
    \inferrule* [right={\tiny Gr}] {
      {
        \begin{array}{c}
          \pi \\
          {[[P1; X; P2 |-l A]]}
        \end{array}
      }
    }{[[P1, X, P2 |-c Gf A]]}
  \end{math}
\end{center}
By assumption, $c(\Pi_1),c(\Pi_2)\leq |X|$. By induction on $\Pi_1$
and $\pi$, there is a proof $\Pi'$ for sequent \\
$[[P1, I, P2 |-l A]]$ s.t. $c(\Pi') \leq |X|$. Therefore, the proof $\Pi$
can be constructed as follows with \\
$c(\Pi) = c(\Pi') \leq |X|$.
\begin{center}
  \scriptsize
  \begin{math}
    $$\mprset{flushleft}
    \inferrule* [right={\tiny Gr}] {
      {
        \begin{array}{c}
          \Pi' \\
          {[[P1; I; P2 |-l A]]}
        \end{array}
      }
    }{[[P1, I, P2 |-c Gf A]]}
  \end{math}
\end{center}




\subsubsection{Left introduction of the functor $G$ (with low priority)}
\begin{itemize}
\item Case 1:
      \begin{center}
        \scriptsize
        \begin{math}
          \begin{array}{c}
            \Pi_1 \\
            {[[I |-c X]]}
          \end{array}
        \end{math}
        \qquad\qquad
        $\Pi_2$:
        \begin{math}
          $$\mprset{flushleft}
          \inferrule* [right={\tiny Gl}] {
            {
              \begin{array}{c}
                \pi \\
                {[[G1; X; G2; B; G3 |-l A]]}
              \end{array}
            }
          }{[[G1; X; G2; Gf B; G3 |-l A]]}
        \end{math}
      \end{center}
      By assumption, $c(\Pi_1),c(\Pi_2)\leq |X|$. By induction on $\Pi_1$
      and $\pi$, there is a proof $\Pi'$ for sequent
      $[[G1; I; G2; B; G3 |-l A]]$ s.t. $c(\Pi') \leq |X|$. Therefore, the
      proof $\Pi$ can be constructed as follows with
      $c(\Pi) = c(\Pi') \leq |X|$.
      \begin{center}
        \scriptsize
        \begin{math}
          $$\mprset{flushleft}
          \inferrule* [right={\tiny Gl}] {
            {
              \begin{array}{c}
                \Pi' \\
                {[[G1; I; G2; B; G3 |-l A]]}
              \end{array}
            }
          }{[[G1; I; G2; Gf B; G3 |-l A]]}
        \end{math}
      \end{center}

\item Case 2:
      \begin{center}
        \scriptsize
        \begin{math}
          \begin{array}{c}
            \Pi_1 \\
            {[[D |-l B]]}
          \end{array}
        \end{math}
        \qquad\qquad
        $\Pi_2$:
        \begin{math}
          $$\mprset{flushleft}
          \inferrule* [right={\tiny Gl}] {
            {
              \begin{array}{c}
                \pi \\
                {[[G1; B; G2; C; G3 |-l A]]}
              \end{array}
            }
          }{[[G1; B; G2; Gf C; G3 |-l A]]}
        \end{math}
      \end{center}
      By assumption, $c(\Pi_1),c(\Pi_2)\leq |B|$. By induction on $\Pi_1$
      and $\pi$, there is a proof $\Pi'$ for sequent
      $[[G1; D; G2; C; G3 |-l A]]$ s.t. $c(\Pi') \leq |B|$. Therefore, the
      proof $\Pi$ can be constructed as follows with
      $c(\Pi) = c(\Pi') \leq |B|$.
      \begin{center}
        \scriptsize
        \begin{math}
          $$\mprset{flushleft}
          \inferrule* [right={\tiny Gl}] {
            {
              \begin{array}{c}
                \Pi' \\
                {[[G1; D; G2; C; G3 |-l A]]}
              \end{array}
            }
          }{[[G1; D; G2; Gf C; G3 |-l A]]}
        \end{math}
      \end{center}

\item Case 3:
      \begin{center}
        \scriptsize
        \begin{math}
          \begin{array}{c}
            \Pi_1 \\
            {[[I |-c X]]}
          \end{array}
        \end{math}
        \qquad\qquad
        $\Pi_2$:
        \begin{math}
          $$\mprset{flushleft}
          \inferrule* [right={\tiny Gl}] {
            {
              \begin{array}{c}
                \pi \\
                {[[G1; B; G2; X; G3 |-l A]]}
              \end{array}
            }
          }{[[G1; Gf B; G2; X; G3 |-l A]]}
        \end{math}
      \end{center}
      By assumption, $c(\Pi_1),c(\Pi_2)\leq |X|$. By induction on $\Pi_1$
      and $\pi$, there is a proof $\Pi'$ for sequent
      $[[G1; B; G2; I; G3 |-l A]]$ s.t. $c(\Pi') \leq |X|$. Therefore, the
      proof $\Pi$ can be constructed as follows with
      $c(\Pi) = c(\Pi') \leq |X|$.
      \begin{center}
        \scriptsize
        \begin{math}
          $$\mprset{flushleft}
          \inferrule* [right={\tiny Gl}] {
            {
              \begin{array}{c}
                \Pi' \\
                {[[G1; B; G2; I; G3 |-l A]]}
              \end{array}
            }
          }{[[G1; Gf B; G2; I; G3 |-l A]]}
        \end{math}
      \end{center}

\item Case 4:
      \begin{center}
        \scriptsize
        \begin{math}
          \begin{array}{c}
            \Pi_1 \\
            {[[D |-l B]]}
          \end{array}
        \end{math}
        \qquad\qquad
        $\Pi_2$:
        \begin{math}
          $$\mprset{flushleft}
          \inferrule* [right={\tiny Gl}] {
            {
              \begin{array}{c}
                \pi \\
                {[[G1; C; G2; B; G3 |-l A]]}
              \end{array}
            }
          }{[[G1; Gf C; G2; B; G3 |-l A]]}
        \end{math}
      \end{center}
      By assumption, $c(\Pi_1),c(\Pi_2)\leq |B|$. By induction on $\Pi_1$
      and $\pi$, there is a proof $\Pi'$ for sequent
      $[[G1; C; G2; D; G3 |-l A]]$ s.t. $c(\Pi') \leq |B|$. Therefore, the
      proof $\Pi$ can be constructed as follows with
      $c(\Pi) = c(\Pi') \leq |B|$.
      \begin{center}
        \scriptsize
        \begin{math}
          $$\mprset{flushleft}
          \inferrule* [right={\tiny Gl}] {
            {
              \begin{array}{c}
                \Pi' \\
                {[[G1; C; G2; D; G3 |-l A]]}
              \end{array}
            }
          }{[[G1; Gf C; G2; D; G3 |-l A]]}
        \end{math}
      \end{center}

\end{itemize}



%--------------------------------------------------
%--------------------------------------------------
\section{Proof For Lemma~\ref{lem:monoidal-monad}}
\label{app:monoidal-monad}

Let $(\cat{C},\cat{L},F,G,\eta,\varepsilon)$ be a LAM. We define the monad
$(T,\eta:id_\cat{C}\rightarrow T,\mu:T^2\rightarrow T)$ on $\cat{C}$ as
$T=GF$, $\eta_X:X\rightarrow GFX$, and
$\mu_X=G\varepsilon_{FX}:GFGFX\rightarrow GFX$. Since $(F,\m{})$ and
$(G,\n{})$ are monoidal functors, we have
$$\t{X,Y}=G\m{X,Y}\circ\n{FX,FY}:TX\otimes TY\rightarrow T(X\otimes Y) \qquad\mbox{and}\qquad\t{I}=G\m{I}\circ\n{I'}:I\rightarrow TI.$$
The monad $T$ being monoidal means:
\begin{enumerate}
\item $T$ is a monoidal functor, i.e. the following diagrams commute:
      \begin{mathpar}
      \bfig
        \hSquares/->`->`->``->`->`->/<400>[
          (TX\otimes TY)\otimes TZ`TX\otimes(TY\otimes TZ)`TX\otimes T(Y\otimes Z)`
          T(X\otimes Y)\otimes TZ`T((X\otimes Y)\otimes Z)`T(X\otimes(Y\otimes Z));
          \alpha_{TX,TY,TZ}`id_{TX}\otimes\t{Y,Z}`\t{X,Y}\otimes id_{TZ}``
          \t{X,Y\otimes Z}`\t{X\otimes Y,Z}`T\alpha_{X,Y,Z}]
        \morphism(1300,200)//<0,0>[`;(1)]
      \efig
      \and
      \bfig
        \square/->`->`<-`->/<600,400>[
          I\otimes TX`TX`TI\otimes TX`T(I\otimes X);
          \lambda_{TX}`\t{I}\otimes id_{TX}`T\lambda_X`\t{I,X}]
        \morphism(350,200)//<0,0>[`;(2)]
      \efig
      \and
      \bfig
        \square/->`->`<-`->/<600,400>[
          TX\otimes I`TX`TX\otimes TI`T(X\otimes I);
          \rho_{TX}`id_{TX}\otimes\t{I}`T\rho_X`\t{X,I}]
        \morphism(350,200)//<0,0>[`;(3)]
      \efig
      \end{mathpar}
      We write $GF$ instead of $T$ in the proof for clarity. \\
      By replacing $\t{X,Y}$ with its definition, diagram (1) above
      commutes by the following commutative diagram, in which the two
      hexagons commute because $G$ and $F$ are monoidal functors, and the
      two quadrilaterals commute by the naturality of $\n{}$.
      \begin{mathpar}
      \bfig
        \iiixiii/->`->`->``->```->`<-`->``/<1400,400>[
          (GFX\otimes GFY)\otimes GFZ`GFX\otimes(GFY\otimes GFZ)`GFX\otimes G(FY\tri FZ)`
          G(FX\tri FY)\otimes GFZ`G(FX\tri(FY\tri FZ))`GFX\otimes GF(Y\otimes Z)`
          GF(X\otimes Y)\otimes GFZ`G((FX\tri FY)\tri FZ)`G(FX\tri F(Y\otimes Z));
          \alpha_{GFX,GFY,GFZ}`id_{GFX}\otimes\n{FY,FZ}`\n{FX,FY}\otimes id_{GFZ}``
          id_{GFX}\otimes G\m{Y,Z}```G\m{X,Y}\otimes id_{GFZ}`G\alpha'_{FX,FY,FZ}`
          \n{FX,F(Y\otimes Z)}``]
        \morphism(2800,800)|m|<-1400,-400>[
          GFX\otimes G(FY\tri FZ)`G(FX\tri(FY\tri FZ));\n{FX,FY\tri FZ}]
        \morphism(0,400)|m|<1400,-400>[
          G(FX\tri FY)\otimes GFZ`G((FX\tri FY)\tri FZ);\n{FX\tri FY,FZ}]
        \morphism(1400,400)|m|<1400,-400>[
          G(FX\tri(FY\tri FZ))`G(FX\tri F(Y\otimes Z));G(id_{FX}\tri\m{Y,Z})]
        \ptriangle(0,-400)|mlm|/`->`->/<1400,400>[
          GF(X\otimes Y)\otimes GFZ`G((FX\tri FY)\tri FZ)`G(F(X\otimes Y)\tri FZ);
          `\n{F(X\otimes Y),FZ}`G(\m{X,Y}\otimes id_{FZ})]
        \morphism(0,-400)|b|<1400,0>[
          G(F(X\otimes Y)\tri FZ)`GF((X\otimes Y)\otimes Z);G\m{X\otimes Y,Z}]
        \dtriangle(1400,-400)|mrb|/`->`->/<1400,400>[
          G(FX\tri F(Y\otimes Z))`GF((X\otimes Y)\otimes Z)`GF(X\otimes(Y\otimes Z));
          `G\m{X,Y\otimes Z}`GF\alpha_{X,Y,Z}]
      \efig
      \end{mathpar}
      Diagram (2) commutes by the following commutative diagrams, in which
      the top quadrilateral commutes because $G$ is monoidal, the right
      quadrilateral commutes because $F$ is monoidal, and the left square
      commutes by the naturality of $\n{}$.
      \begin{mathpar}
      \bfig
        \ptriangle/->`->`/<1600,400>[
          I\otimes GFX`GFX`GI'\otimes GFX;\lambda_{GFX}`\n{I'}\otimes id_{GFX}`]
        \square(0,-400)|lmmb|<800,400>[
          GI'\otimes GFX`G(I'\tri FX)`GFI\otimes GFX`G(FI\tri FX);
          \n{I',FX}`G\m{I}\otimes id_{GFX}`G(\m{I}\tri id_{FX})`\n{FI,FX}]
        \morphism(800,0)|m|<800,400>[G(I'\tri FX)`GFX;G\lambda'_{FX}]
        \dtriangle(800,-400)/`<-`->/<800,800>[
          GFX`G(FI\tri FX)`GF(I\otimes X);
          `GF\lambda_X`G\m{I,X}]
      \efig
      \end{mathpar}
      Similarly, diagram (3) commutes as follows:
      \begin{mathpar}
      \bfig
        \ptriangle/->`->`/<1600,400>[
          GFX\otimes I`GFX`GFX\otimes GI';\rho_{GFX}`id_{GFX}\otimes\n{I'}`]
        \square(0,-400)|lmmb|<800,400>[
          GFX\otimes GI'`G(FX\tri I')`GFX\otimes GFI`G(FX\tri FI);
          \n{FX,I'}`id_{GFX}\otimes G\m{I}`G(id_{FX}\otimes\m{I})`\n{FX,FI}]
        \morphism(800,0)|m|<800,400>[G(FX\tri I')`GFX;G\rho'_{FX}]
        \dtriangle(800,-400)/`<-`->/<800,800>[
          GFX`G(FX\tri FI)`GF(X\otimes I);
          `GF\rho_X`G\m{X,I}]
      \efig
      \end{mathpar}
\item $\eta$ is a monoidal natural transformation. In fact, since $\eta$
      is the unit of the monoidal adjunction, $\eta$ is monoidal by
      definition and thus the following two diagrams commute.
      \begin{mathpar}
      \bfig
        \square/=`->`->`->/<600,400>[
          X\otimes Y`X\otimes Y`TX\otimes TY`T(X\otimes Y);
          `\eta_X\otimes\eta_Y`\eta_{X\otimes Y}`\t{X,Y}]
      \efig
      \and
      \bfig
        \Vtriangle/->`=`<-/<400,400>[I`TI`I;\eta_I``\t{I}]
      \efig
      \end{mathpar}
\item $\mu$ is a monoidal natural transformation. It is obvious that since
      $\mu=G\varepsilon_{FA}$ and $\varepsilon$ is monoidal, so is $\mu$.
      Thus the following diagrams commute.
      \begin{mathpar}
      \bfig
        \square/`->`->`->/<1500,400>[
          T^2X\otimes T^2Y`T^2(X\otimes Y)`TX\otimes TY`T(X\otimes Y);
          `\mu_X\otimes\mu_Y`\mu_{X\otimes Y}`\t{X,Y}]
        \morphism(0,400)<800,0>[T^2X\otimes T^2Y`T(TX\otimes TY);\t{TX,TY}]
        \morphism(800,400)<700,0>[T(TX\otimes TY)`T^2(X\otimes Y);T\t{X,Y}]
      \efig
      \and
      \bfig
        \square/->`<-`<-`<-/<400,400>[T^2I`TI`TI`I;\mu_I`T\t{I}`\t{I}`\t{I}]
      \efig
      \end{mathpar}
\end{enumerate}



%--------------------------------------------------
%--------------------------------------------------
\section{Proof For Lemma~\ref{lem:strong-monad}}
\label{app:strong-monad}

\begin{definition}
\label{def:strong-monad}
Let $(\cat{M},\tri,I,\alpha,\lambda,\rho)$ be a monoidal category and
$(T,\eta,\mu)$ be a monad on $\cat{M}$. $T$ is a \textbf{strong monad} if
there is natural transformation $\tau$, called the \textbf{tensorial
strength}, with components \\
$\tau_{A,B}:A\tri TB\rightarrow T(A\tri B)$ such that the following
diagrams commute:
\begin{mathpar}
\bfig
  \Vtriangle<400,400>[I\tri TA`T(I\tri A)`TA;\tau_{I,A}`\lambda_{TA}`T\lambda_A]
\efig
\and
\bfig
  \Vtriangle<400,400>[
    A\tri B`A\tri TB`T(A\tri B);id_A\tri\eta_B`\eta_{A\tri B}`\tau_{A,B}]
\efig
\and
\bfig
  \square/->`->`->`/<1800,400>[
    (A\tri B)\tri TC`T((A\tri B)\tri C)`
    A\tri(B\tri TC)`T(A\tri(B\tri C));
    \tau_{A\tri B,C}`\alpha_{A,B,TC}`T\alpha_{A,B,C}`]
  \morphism<900,0>[A\tri(B\tri TC)`A\tri T(B\tri C);id_A\tri\tau_{B,C}]
  \morphism(900,0)<900,0>[A\tri T(B\tri C)`T(A\tri(B\tri C));\tau_{A,B\tri C}]  \efig
\and
\bfig
  \square/`->`->`->/<1400,400>[
    A\tri T^2B`T^2(A\tri B)`A\tri TB`T(A\tri B);
    `id_A\tri\mu_B`\mu_{A\tri B}`\tau_{A,B}]
  \morphism(0,400)<700,0>[A\tri T^2B`T(A\tri TB);\tau_{A,TB}]
  \morphism(700,400)<700,0>[T(A\tri TB)`T^2(A\tri B);T\tau_{A,B}]
\efig
\end{mathpar}
\end{definition}
\noindent
The proof for Lemma~\ref{lem:strong-monad} goes as follows.
\noindent
Let $(\cat{C},\cat{L},F,G,\eta,\varepsilon)$ be a LAM, where
$(\cat{C},\otimes,I,\alpha,\lambda,\rho)$ is symmetric monoidal closed,
and \\ $(\cat{L},\tri,I',\alpha',\lambda',\rho')$ is Lambek. In
Lemma~\ref{lem:monoidal-monad}, we have proved that the monad
$(T=GF,\eta,\mu)$ is monoidal with the natural transformation
$\t{X,Y}:TX\otimes TY\rightarrow T(X\otimes Y)$ and the morphism
$\t{I}:I\rightarrow TI$.
\noindent
We define the tensorial strength
$\tau_{X,Y}:X\otimes TY\rightarrow T(X\otimes Y)$ as
$$\tau_{X,Y}=\t{X,Y}\circ(\eta_X\otimes id_{TY}).$$
Since $\eta$ is a monoidal natural transformation, we have
$\eta_I=G\m{I}\circ\n{I'}$, and thus $\eta_I=\t{I}$. The following diagram
commutes because $T$ is monoidal, where the composition
$\t{I,X}\circ(\t{I}\otimes id_{TX})$ is the definition of $\tau_{I,X}$. So
the first triangle in Definition~\ref{def:strong-monad} commutes.
\begin{mathpar}
\bfig
  \square/->`->`->`<-/<600,400>[
    I\otimes TX`TI\otimes TX`TX`T(I\otimes X);
    \t{I}\otimes id_{TX}`\lambda_{TX}`\t{I,X}`T\lambda_X]
\efig
\end{mathpar}
Similarly, by using the definition of $\tau$, the the second triangle in the definition is
equivalent to the following diagram, which commutes because $\eta$ is a monoidal natural
transformation:
\begin{mathpar}
\bfig
  \square/->`->`->`<-/<600,400>[
    X\otimes Y`X\otimes TY`T(X\otimes Y)`TX\otimes TY;
    id_X\otimes\eta_Y`\eta_{X\otimes Y}`\eta_X\otimes id_{TY}`\t{X,Y}]
  \morphism(0,400)|m|<600,-400>[X\otimes Y`TX\otimes TY;\eta_X\otimes\eta_Y]
\efig
\end{mathpar}
The first pentagon in the definition commutes by the following commutative diagrams, because
$\eta$ and $\alpha$ are natural transformations and $T$ is monoidal:
\begin{mathpar}
\bfig
  \qtriangle|amm|/->`->`<-/<1000,400>[
    (X\otimes Y)\otimes TZ`T(X\otimes Y)\otimes TZ`(TX\otimes TY)\otimes TZ;
    \eta_{X\otimes Y}\otimes id_{TZ}`
    (\eta_X\otimes\eta_Y)\otimes id_{TZ}`
    \t{X,Y}\otimes id_{TZ}]
  \morphism(0,400)<0,-400>[(X\otimes Y)\otimes TZ`X\otimes(Y\otimes TZ);\alpha_{X,Y,TZ}]
  \morphism(1000,0)|m|<0,-400>[
    (TX\otimes TY)\otimes TZ`TX\otimes(TY\otimes TZ);\alpha_{TX,TY,TZ}]
  \Dtriangle(0,-800)|lmm|/->`->`<-/<1000,400>[
    X\otimes(Y\otimes TZ)`TX\otimes(TY\otimes TZ)`X\otimes(TY\otimes TZ);
    id_X\otimes(\eta_Y\otimes id_{TZ})`
    \eta_X\otimes(\eta_Y\otimes id_{TZ})`
    \eta_X\otimes id_{TY\otimes TZ}]
  \morphism(0,-800)|b|<1000,0>[
    X\otimes(TY\otimes TZ)`X\otimes T(Y\otimes Z);id_X\otimes\t{Y,Z}]
  \qtriangle(1000,0)|amr|/->``->/<1000,400>[
    T(X\otimes Y)\otimes TZ`T((X\otimes Y)\otimes Z)`T(X\otimes(Y\otimes Z));
    \t{X\otimes Y,Z}``T\alpha_{X,Y,Z}]
  \morphism(2000,-800)<0,800>[
    TX\otimes T(Y\otimes Z)`T(X\otimes(Y\otimes Z));\t{X,Y\otimes Z}]
  \btriangle(1000,-800)|mmb|/`->`->/<1000,400>[
    TX\otimes(TY\otimes TZ)`X\otimes T(Y\otimes Z)`TX\otimes T(Y\otimes Z);
    `id_{TX}\otimes\t{Y,Z}`\eta_X\otimes id_{T(Y\otimes Z)}]
\efig
\end{mathpar}
The last diagram in the definition commutes by the following commutative diagram, because
$T$ is a monad, $\t{}$ is a natural transformation, and $\mu$ is a monoidal natural
transformation:
\begin{mathpar}
\bfig
  \ptriangle/->`->`/<700,400>[
    X\otimes T^2Y`TX\otimes T^2Y`X\otimes TY;\eta_X\otimes id_{T^2Y}`id_X\otimes\mu_Y`]
  \btriangle(0,-400)/->``->/<700,400>[
    X\otimes TY`TX\otimes TY`T(X\otimes Y);\eta_X\otimes id_{TY}``\t{X,Y}]
  \morphism(700,400)|m|<-700,-800>[TX\otimes T^2Y`TX\otimes TY;id_{TX}\otimes\mu_Y]
  \morphism(700,0)|m|<-700,-400>[TX\otimes T^2Y`TX\otimes TY;id_{TX}\otimes\mu_Y]
  \qtriangle(700,0)/->``->/<1800,400>[
    TX\otimes T^2Y`T(X\otimes TY)`T(TX\otimes TY);\t{X,TY}``T(\eta_X\otimes id_{TY})]
  \btriangle(700,0)|mmm|/=`->`<-/<900,400>[
    TX\otimes T^2Y`TX\otimes T^2Y`T^2X\otimes T^2Y;
    `T\eta_X\otimes id_{T^2Y}`\mu_X\otimes id_{T^2Y}]
  \morphism(1600,0)|m|<900,0>[T^2X\otimes T^2Y`T(TX\otimes TY);\t{TX,TY}]
  \morphism(1600,0)|m|<-1600,-400>[T^2X\otimes T^2Y`TX\otimes TY;\mu_X\otimes\mu_Y]
  \dtriangle(700,-400)/`->`<-/<1800,400>[
    T(TX\otimes TY)`T(X\otimes Y)`T^2(X\otimes Y);`T\t{X,Y}`\mu_{X\otimes Y}]
\efig
\end{mathpar}




% section appendix (end)

\end{document}

%%% Local Variables: 
%%% mode: latex
%%% TeX-master: t
%%% End: 

